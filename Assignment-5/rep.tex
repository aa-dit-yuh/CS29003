\title{Assignment 5 -CS29003}
\author{
        Aditya Narayan \\
        13EC30001\\
}
\date{\today}

\documentclass[12pt]{article}
\usepackage{pgfplots}
\usepackage[utf8]{inputenc}
\usepackage{times}

\pgfplotsset{width=10cm}

\begin{document}
\maketitle

\begin{abstract}
An analytical approach to Event-driven Simulation using heaps.
\end{abstract}

\section{Introduction}
This experiment intends to simulate the collision of  balls on a 2D planar region bounded by straight walls. The simulation is to be done efficiently using the application of the priority queue data structure. \\

\section{Event Driven Simulation}
The objective of the experiment is to devise the most efficient use of the Heap data structure to simulate the collision of particles using Event driven simulation. \\
Event driven simulation allows us to simulate events in a chronological order.\\

\section{Implementation}\label{Implementation}
The implementation of the priority queue is described as below:\\
\paragraph{Local Minima}
The priority queue, Local Minima (LM), used in this experiment is a compromise between having all events organized in a single priority queue (where invalidating events is not efficient) or having all the events in a set of linked lists, one per object (where finding the next event is not efficient). To achieve this balance, we define the local minimum associated to a ball i , as the
event of smallest time between all events E i ( x ) scheduled for object i .\\
The events E i ( x ) are stored in a simple linked list L i , which is used to obtain the local minimum for the object i . There are N linked lists L i ; one for every ball i . Note, that the events E j ( i ) are stored in the list L j and not in the list L i. To compare the N local minima, we use a heap (CBT) which performs a binary tournament between all local minima. That is, each leaf has a ball number and each internal node (recursively up to the root) has the ball number with smaller local minimum of its two children. So, the root of the tree has the smallest local minima. Every time that a new local minimum is computed for ball i , the tournament
is updated for all nodes in the path from the leaf labelled with i to the root. The tournament uses a complete binary tree because N is fixed. \\
An event E ( i, j ) can be invalidated if in a previous simulation step, Cancel( j ) was called. Not all events associated to an object are invalidated at the same time. Cancel( j ) invalidates
all events E ( j, x ), but events E (k, j ) will stay in the PQ until are invalidated by Cancel( k ) or they are detected and deleted when a local minimum for k is computed. \\
Every time that we extract a smallest event E ( i, j ) that has been invalidated, a new local minimum for i is computed and the HEAP updated, extracting again the next chronological event. We call this operation a Reschedule .\\

\paragraph{Insertion}
Pseudocode for insertion of events: \\ \\
\indent \indent  \textbf{Insert( E ( i, j ) )}\\
\indent \indent \indent \textbf{if} ( j is a moving ball)\\
\indent \indent \indent \indent E ( i, j ).c $\leftarrow$ O[ j ].c;\\
\indent \indent \indent O [ i ]. L $\leftarrow$ \textbf{InsertList} ( i , E ( i, j ) );\\
\indent \indent  \textbf{EndInsert}\\
\\
\paragraph{Next Event}
Pseudocode for extracting the next event from the Priority Queue:\\ \\
\indent \indent  \textbf{NextEvent ()}\\
\indent \indent \indent \textbf{while} (1)\\
\indent \indent \indent \indent i $\leftarrow$ HEAP [ root ]\\
\indent \indent \indent \indent E i ( j ) $\leftarrow$ O [ i ]. L ;\\
\indent \indent \indent \indent \textbf{if} ( j is a moving ball ) and ( E ( i, j ).c != O [ j ].c ) )\\
\indent \indent \indent \indent \indent /* Reschedule */\\
\indent \indent \indent \indent \indent O [ i ]. L $\leftarrow$ \textbf{LocalMin} ( i );\\
\indent \indent \indent \indent \indent \textbf{Heapify} ( i );\\
\indent \indent \indent \indent \textbf{else return} ( E i ( j ) );\\
\indent \indent  \textbf{EndNextEvent}\\
\paragraph{Deletion}
Pseudocode for deletion of events: \\ \\
\indent \indent  \textbf{Delete (i)}\\
\indent \indent \indent O [ i ]. L $\leftarrow$ NULL;\\
\indent \indent \indent O [ i ].c $\leftarrow$ O[ i ].c + 1;\\
\indent \indent  \textbf{EndDelete}\\
\\

\subsection{Complexity Analysis}\label{Complexity Analysis}
\paragraph{Insertion}
Insertion requires adding an event to its specific linked list followed by updating the heap upwards from the linked list towards the root. This operation consumes $\theta(log(n))$ number of operations, where $\mathbf{n}$ is the number of particles in the system.

\paragraph{Next Event}
The operation to extract the next event in the queue is a constant time process if the dequeued event is valid. If the dequeued is invalid, the \textbf{LocalMinima} for the ball element is evaluated and the \textbf{Heap} is then heapified to maintain heap properties. This consumes $\theta(c)$ (constant) operations for the valid case, and $\theta(n * log(n))$ operations for the invalid case. In practise, the valid case occurs far greater times than the invalid case.\\
 
\paragraph{Deletion or Invalidation}
Deletion is a contant time operation in the LM Priority Queue, as it only involves clearing the linked list pertaining to a ball.

\section{Results}\label{results}
Thus, we observe that the Priority Queue being used, the \textbf{LocalMinima} is one of the most efficient Priority Queues for the given application.
\end{document}
