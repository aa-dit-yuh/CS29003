\title{Assignment 5 -CS29003}
\author{
        Aditya Narayan \\
        13EC30001\\
}
\date{\today}

\documentclass[12pt]{article}
\usepackage{pgfplots}
\usepackage[utf8]{inputenc}
\usepackage{times}

\pgfplotsset{width=10cm}

\begin{document}
\maketitle

\begin{abstract}
An analytical approach to Event-driven Simulation using heaps.
\end{abstract}

\section{Introduction}
This experiment intends to simulate the collision of  balls on a 2D planar region bounded by straight walls. The simulation is to be done efficiently using the application of the priority queue data structure. \\

\section{Event Driven Simulation}
The objective of the experiment is to devise the most efficient use of the Heap data structure to simulate the collision of particles using Event driven simulation. \\
Event driven simulation allows us to simulate events in a chronological order.\\

\section{Implementation}\label{Implementation}
The implementation of the priority queue is described as below:\\
\paragraph{Local Minima}
The priority queue, Local Minima (LM), used in this experiment is a compromise between having all events organized in a single priority queue (where invalidating events is not efficient) or having all the events in a set of linked lists, one per object (where finding the next event is not efficient). To achieve this balance, we define the local minimum associated to a ball i , as the
event of smallest time between all events E i ( x ) scheduled for object i .\\
The events E i ( x ) are stored in a simple linked list L i , which is used to obtain the local minimum for the object i . There are N linked lists L i ; one for every ball i . Note, that the events E j ( i ) are stored in the list L j and not in the list L i. To compare the N local minima, we use a heap (CBT) which performs a binary tournament between all local minima. That is, each leaf has a ball number and each internal node (recursively up to the root) has the ball number with smaller local minimum of its two children. So, the root of the tree has the smallest local minima. Every time that a new local minimum is computed for ball i , the tournament
is updated for all nodes in the path from the leaf labelled with i to the root. The tournament uses a complete binary tree because N is fixed. \\
An event E ( i, j ) can be invalidated if in a previous simulation step, Cancel( j ) was called. Not all events associated to an object are invalidated at the same time. Cancel( j ) invalidates
all events E ( j, x ), but events E (k, j ) will stay in the PQ until are invalidated by Cancel( k ) or they are detected and deleted when a local minimum for k is computed. \\
Every time that we extract a smallest event E ( i, j ) that has been invalidated, a new local minimum for i is computed and the HEAP updated, extracting again the next chronological event. We call this operation a Reschedule .\\

\paragraph{Insertion}
Pseudocode for insertion of events: \\ \\
\indent \indent  \textbf{Insert( E ( i, j ) )}\\
\indent \indent \indent \textbf{if} ( j is a moving ball)\\
\indent \indent \indent \indent E ( i, j ).c $\leftarrow$ O[ j ].c;\\
\indent \indent \indent O [ i ]. L $\leftarrow$ \textbf{InsertList} ( i , E ( i, j ) );\\
\indent \indent  \textbf{EndInsert}\\
\\
\paragraph{Next Event}
Pseudocode for extracting the next event from the Priority Queue:\\ \\
\indent \indent  \textbf{NextEvent ()}\\
\indent \indent \indent \textbf{while} (1)\\
\indent \indent \indent \indent i $\leftarrow$ HEAP [ root ]\\
\indent \indent \indent \indent E i ( j ) $\leftarrow$ O [ i ]. L ;\\
\indent \indent \indent \indent \textbf{if} ( j is a moving ball ) and ( E ( i, j ).c != O [ j ].c ) )\\
\indent \indent \indent \indent \indent /* Reschedule */\\
\indent \indent \indent \indent \indent O [ i ]. L $\leftarrow$ \textbf{LocalMin} ( i );\\
\indent \indent \indent \indent \indent \textbf{Heapify} ( i );\\
\indent \indent \indent \indent \textbf{else return} ( E i ( j ) );\\
\indent \indent  \textbf{EndNextEvent}\\
\paragraph{Deletion}
Pseudocode for deletion of events: \\ \\
\indent \indent  \textbf{Delete (i)}\\
\indent \indent \indent O [ i ]. L $\leftarrow$ NULL;\\
\indent \indent \indent O [ i ].c $\leftarrow$ O[ i ].c + 1;\\
\indent \indent  \textbf{EndDelete}\\
\\

\subsection{Complexity Analysis}\label{Complexity Analysis}
\paragraph{Insertion}
Insertion requires adding an event to its specific linked list followed by updating the heap upwards from the linked list towards the root. This operation consumes $\theta(log(n))$ number of operations, where $\mathbf{n}$ is the number of particles in the system.

\paragraph{Next Event}
The operation to extract the next event in the queue is a constant time process if the dequeued event is valid. If the dequeued is invalid, the \textbf{LocalMinima} for the ball element is evaluated and the \textbf{Heap} is then heapified to maintain heap properties. This consumes $\theta(c)$ (constant) operations for the valid case, and $\theta(n * log(n))$ operations for the invalid case. In practise, the valid case occurs far greater times than the invalid case.\\
 
\paragraph{Deletion or Invalidation}
Deletion is a contant time operation in the LM Priority Queue, as it only involves clearing the linked list pertaining to a ball.

\section{Results}\label{results}
Thus, we observe that the Priority Queue being used, the \textbf{LocalMinima} is one of the most efficient Priority Queues for the given application.

\section{Plots}\label{plots}
Plot for Ball 1:\\
% GNUPLOT: LaTeX picture
\setlength{\unitlength}{0.240900pt}
\ifx\plotpoint\undefined\newsavebox{\plotpoint}\fi
\sbox{\plotpoint}{\rule[-0.200pt]{0.400pt}{0.400pt}}%
\begin{picture}(1500,900)(0,0)
\sbox{\plotpoint}{\rule[-0.200pt]{0.400pt}{0.400pt}}%
\put(130.0,90.0){\rule[-0.200pt]{4.818pt}{0.400pt}}
\put(110,90){\makebox(0,0)[r]{ 0}}
\put(1419.0,90.0){\rule[-0.200pt]{4.818pt}{0.400pt}}
\put(130.0,242.0){\rule[-0.200pt]{4.818pt}{0.400pt}}
\put(110,242){\makebox(0,0)[r]{ 0.2}}
\put(1419.0,242.0){\rule[-0.200pt]{4.818pt}{0.400pt}}
\put(130.0,394.0){\rule[-0.200pt]{4.818pt}{0.400pt}}
\put(110,394){\makebox(0,0)[r]{ 0.4}}
\put(1419.0,394.0){\rule[-0.200pt]{4.818pt}{0.400pt}}
\put(130.0,547.0){\rule[-0.200pt]{4.818pt}{0.400pt}}
\put(110,547){\makebox(0,0)[r]{ 0.6}}
\put(1419.0,547.0){\rule[-0.200pt]{4.818pt}{0.400pt}}
\put(130.0,699.0){\rule[-0.200pt]{4.818pt}{0.400pt}}
\put(110,699){\makebox(0,0)[r]{ 0.8}}
\put(1419.0,699.0){\rule[-0.200pt]{4.818pt}{0.400pt}}
\put(130.0,851.0){\rule[-0.200pt]{4.818pt}{0.400pt}}
\put(110,851){\makebox(0,0)[r]{ 1}}
\put(1419.0,851.0){\rule[-0.200pt]{4.818pt}{0.400pt}}
\put(130.0,82.0){\rule[-0.200pt]{0.400pt}{4.818pt}}
\put(130,41){\makebox(0,0){ 0}}
\put(130.0,839.0){\rule[-0.200pt]{0.400pt}{4.818pt}}
\put(392.0,82.0){\rule[-0.200pt]{0.400pt}{4.818pt}}
\put(392,41){\makebox(0,0){ 0.2}}
\put(392.0,839.0){\rule[-0.200pt]{0.400pt}{4.818pt}}
\put(654.0,82.0){\rule[-0.200pt]{0.400pt}{4.818pt}}
\put(654,41){\makebox(0,0){ 0.4}}
\put(654.0,839.0){\rule[-0.200pt]{0.400pt}{4.818pt}}
\put(915.0,82.0){\rule[-0.200pt]{0.400pt}{4.818pt}}
\put(915,41){\makebox(0,0){ 0.6}}
\put(915.0,839.0){\rule[-0.200pt]{0.400pt}{4.818pt}}
\put(1177.0,82.0){\rule[-0.200pt]{0.400pt}{4.818pt}}
\put(1177,41){\makebox(0,0){ 0.8}}
\put(1177.0,839.0){\rule[-0.200pt]{0.400pt}{4.818pt}}
\put(1439.0,82.0){\rule[-0.200pt]{0.400pt}{4.818pt}}
\put(1439,41){\makebox(0,0){ 1}}
\put(1439.0,839.0){\rule[-0.200pt]{0.400pt}{4.818pt}}
\put(130.0,82.0){\rule[-0.200pt]{0.400pt}{187.179pt}}
\put(130.0,82.0){\rule[-0.200pt]{315.338pt}{0.400pt}}
\put(1439.0,82.0){\rule[-0.200pt]{0.400pt}{187.179pt}}
\put(130.0,859.0){\rule[-0.200pt]{315.338pt}{0.400pt}}
\put(1279,819){\makebox(0,0)[r]{'-'}}
\put(1299.0,819.0){\rule[-0.200pt]{24.090pt}{0.400pt}}
\put(856,602){\usebox{\plotpoint}}
\put(856,600.67){\rule{0.723pt}{0.400pt}}
\multiput(856.00,601.17)(1.500,-1.000){2}{\rule{0.361pt}{0.400pt}}
\put(859,599.67){\rule{0.723pt}{0.400pt}}
\multiput(859.00,600.17)(1.500,-1.000){2}{\rule{0.361pt}{0.400pt}}
\put(862,598.17){\rule{0.700pt}{0.400pt}}
\multiput(862.00,599.17)(1.547,-2.000){2}{\rule{0.350pt}{0.400pt}}
\put(865,596.67){\rule{0.723pt}{0.400pt}}
\multiput(865.00,597.17)(1.500,-1.000){2}{\rule{0.361pt}{0.400pt}}
\put(868,595.67){\rule{0.723pt}{0.400pt}}
\multiput(868.00,596.17)(1.500,-1.000){2}{\rule{0.361pt}{0.400pt}}
\put(871,594.67){\rule{0.723pt}{0.400pt}}
\multiput(871.00,595.17)(1.500,-1.000){2}{\rule{0.361pt}{0.400pt}}
\put(874,593.67){\rule{0.723pt}{0.400pt}}
\multiput(874.00,594.17)(1.500,-1.000){2}{\rule{0.361pt}{0.400pt}}
\put(877,592.67){\rule{0.723pt}{0.400pt}}
\multiput(877.00,593.17)(1.500,-1.000){2}{\rule{0.361pt}{0.400pt}}
\put(880,591.17){\rule{0.700pt}{0.400pt}}
\multiput(880.00,592.17)(1.547,-2.000){2}{\rule{0.350pt}{0.400pt}}
\put(883,589.67){\rule{0.964pt}{0.400pt}}
\multiput(883.00,590.17)(2.000,-1.000){2}{\rule{0.482pt}{0.400pt}}
\put(887,588.67){\rule{0.723pt}{0.400pt}}
\multiput(887.00,589.17)(1.500,-1.000){2}{\rule{0.361pt}{0.400pt}}
\put(890,587.67){\rule{0.723pt}{0.400pt}}
\multiput(890.00,588.17)(1.500,-1.000){2}{\rule{0.361pt}{0.400pt}}
\put(893,586.67){\rule{0.723pt}{0.400pt}}
\multiput(893.00,587.17)(1.500,-1.000){2}{\rule{0.361pt}{0.400pt}}
\put(896,585.67){\rule{0.723pt}{0.400pt}}
\multiput(896.00,586.17)(1.500,-1.000){2}{\rule{0.361pt}{0.400pt}}
\put(899,584.17){\rule{0.700pt}{0.400pt}}
\multiput(899.00,585.17)(1.547,-2.000){2}{\rule{0.350pt}{0.400pt}}
\put(902,582.67){\rule{0.723pt}{0.400pt}}
\multiput(902.00,583.17)(1.500,-1.000){2}{\rule{0.361pt}{0.400pt}}
\put(905,581.67){\rule{0.723pt}{0.400pt}}
\multiput(905.00,582.17)(1.500,-1.000){2}{\rule{0.361pt}{0.400pt}}
\put(908,580.67){\rule{0.723pt}{0.400pt}}
\multiput(908.00,581.17)(1.500,-1.000){2}{\rule{0.361pt}{0.400pt}}
\put(911,579.67){\rule{0.723pt}{0.400pt}}
\multiput(911.00,580.17)(1.500,-1.000){2}{\rule{0.361pt}{0.400pt}}
\put(914,578.67){\rule{0.723pt}{0.400pt}}
\multiput(914.00,579.17)(1.500,-1.000){2}{\rule{0.361pt}{0.400pt}}
\put(917,577.17){\rule{0.700pt}{0.400pt}}
\multiput(917.00,578.17)(1.547,-2.000){2}{\rule{0.350pt}{0.400pt}}
\put(920,575.67){\rule{0.723pt}{0.400pt}}
\multiput(920.00,576.17)(1.500,-1.000){2}{\rule{0.361pt}{0.400pt}}
\put(923,574.67){\rule{0.723pt}{0.400pt}}
\multiput(923.00,575.17)(1.500,-1.000){2}{\rule{0.361pt}{0.400pt}}
\put(926,573.67){\rule{0.723pt}{0.400pt}}
\multiput(926.00,574.17)(1.500,-1.000){2}{\rule{0.361pt}{0.400pt}}
\put(929,572.67){\rule{0.723pt}{0.400pt}}
\multiput(929.00,573.17)(1.500,-1.000){2}{\rule{0.361pt}{0.400pt}}
\put(932,571.67){\rule{0.723pt}{0.400pt}}
\multiput(932.00,572.17)(1.500,-1.000){2}{\rule{0.361pt}{0.400pt}}
\put(935,570.17){\rule{0.700pt}{0.400pt}}
\multiput(935.00,571.17)(1.547,-2.000){2}{\rule{0.350pt}{0.400pt}}
\put(938,568.67){\rule{0.723pt}{0.400pt}}
\multiput(938.00,569.17)(1.500,-1.000){2}{\rule{0.361pt}{0.400pt}}
\put(941,567.67){\rule{0.723pt}{0.400pt}}
\multiput(941.00,568.17)(1.500,-1.000){2}{\rule{0.361pt}{0.400pt}}
\put(944,566.67){\rule{0.964pt}{0.400pt}}
\multiput(944.00,567.17)(2.000,-1.000){2}{\rule{0.482pt}{0.400pt}}
\put(948,565.67){\rule{0.723pt}{0.400pt}}
\multiput(948.00,566.17)(1.500,-1.000){2}{\rule{0.361pt}{0.400pt}}
\put(951,564.67){\rule{0.723pt}{0.400pt}}
\multiput(951.00,565.17)(1.500,-1.000){2}{\rule{0.361pt}{0.400pt}}
\put(954,563.17){\rule{0.700pt}{0.400pt}}
\multiput(954.00,564.17)(1.547,-2.000){2}{\rule{0.350pt}{0.400pt}}
\put(957,561.67){\rule{0.723pt}{0.400pt}}
\multiput(957.00,562.17)(1.500,-1.000){2}{\rule{0.361pt}{0.400pt}}
\put(960,560.67){\rule{0.723pt}{0.400pt}}
\multiput(960.00,561.17)(1.500,-1.000){2}{\rule{0.361pt}{0.400pt}}
\put(963,559.67){\rule{0.723pt}{0.400pt}}
\multiput(963.00,560.17)(1.500,-1.000){2}{\rule{0.361pt}{0.400pt}}
\put(966,558.67){\rule{0.723pt}{0.400pt}}
\multiput(966.00,559.17)(1.500,-1.000){2}{\rule{0.361pt}{0.400pt}}
\put(969,557.67){\rule{0.723pt}{0.400pt}}
\multiput(969.00,558.17)(1.500,-1.000){2}{\rule{0.361pt}{0.400pt}}
\put(972,556.17){\rule{0.700pt}{0.400pt}}
\multiput(972.00,557.17)(1.547,-2.000){2}{\rule{0.350pt}{0.400pt}}
\put(975,554.67){\rule{0.723pt}{0.400pt}}
\multiput(975.00,555.17)(1.500,-1.000){2}{\rule{0.361pt}{0.400pt}}
\put(978,553.67){\rule{0.723pt}{0.400pt}}
\multiput(978.00,554.17)(1.500,-1.000){2}{\rule{0.361pt}{0.400pt}}
\put(981,552.67){\rule{0.723pt}{0.400pt}}
\multiput(981.00,553.17)(1.500,-1.000){2}{\rule{0.361pt}{0.400pt}}
\put(984,551.67){\rule{0.723pt}{0.400pt}}
\multiput(984.00,552.17)(1.500,-1.000){2}{\rule{0.361pt}{0.400pt}}
\put(987,550.67){\rule{0.723pt}{0.400pt}}
\multiput(987.00,551.17)(1.500,-1.000){2}{\rule{0.361pt}{0.400pt}}
\put(990,549.17){\rule{0.700pt}{0.400pt}}
\multiput(990.00,550.17)(1.547,-2.000){2}{\rule{0.350pt}{0.400pt}}
\put(993,547.67){\rule{0.723pt}{0.400pt}}
\multiput(993.00,548.17)(1.500,-1.000){2}{\rule{0.361pt}{0.400pt}}
\put(996,546.67){\rule{0.723pt}{0.400pt}}
\multiput(996.00,547.17)(1.500,-1.000){2}{\rule{0.361pt}{0.400pt}}
\put(999,545.67){\rule{0.723pt}{0.400pt}}
\multiput(999.00,546.17)(1.500,-1.000){2}{\rule{0.361pt}{0.400pt}}
\put(1002,544.67){\rule{0.964pt}{0.400pt}}
\multiput(1002.00,545.17)(2.000,-1.000){2}{\rule{0.482pt}{0.400pt}}
\put(1006,543.67){\rule{0.723pt}{0.400pt}}
\multiput(1006.00,544.17)(1.500,-1.000){2}{\rule{0.361pt}{0.400pt}}
\put(1009,542.17){\rule{0.700pt}{0.400pt}}
\multiput(1009.00,543.17)(1.547,-2.000){2}{\rule{0.350pt}{0.400pt}}
\put(1012,540.67){\rule{0.723pt}{0.400pt}}
\multiput(1012.00,541.17)(1.500,-1.000){2}{\rule{0.361pt}{0.400pt}}
\put(1015,539.67){\rule{0.723pt}{0.400pt}}
\multiput(1015.00,540.17)(1.500,-1.000){2}{\rule{0.361pt}{0.400pt}}
\put(1018,538.67){\rule{0.723pt}{0.400pt}}
\multiput(1018.00,539.17)(1.500,-1.000){2}{\rule{0.361pt}{0.400pt}}
\put(1021,537.67){\rule{0.723pt}{0.400pt}}
\multiput(1021.00,538.17)(1.500,-1.000){2}{\rule{0.361pt}{0.400pt}}
\put(1024,536.67){\rule{0.723pt}{0.400pt}}
\multiput(1024.00,537.17)(1.500,-1.000){2}{\rule{0.361pt}{0.400pt}}
\put(1027,535.17){\rule{0.700pt}{0.400pt}}
\multiput(1027.00,536.17)(1.547,-2.000){2}{\rule{0.350pt}{0.400pt}}
\put(1030,533.67){\rule{0.723pt}{0.400pt}}
\multiput(1030.00,534.17)(1.500,-1.000){2}{\rule{0.361pt}{0.400pt}}
\put(1033,532.67){\rule{0.723pt}{0.400pt}}
\multiput(1033.00,533.17)(1.500,-1.000){2}{\rule{0.361pt}{0.400pt}}
\put(1036,531.67){\rule{0.723pt}{0.400pt}}
\multiput(1036.00,532.17)(1.500,-1.000){2}{\rule{0.361pt}{0.400pt}}
\put(1039,530.67){\rule{0.723pt}{0.400pt}}
\multiput(1039.00,531.17)(1.500,-1.000){2}{\rule{0.361pt}{0.400pt}}
\put(1042,529.67){\rule{0.723pt}{0.400pt}}
\multiput(1042.00,530.17)(1.500,-1.000){2}{\rule{0.361pt}{0.400pt}}
\put(1045,528.17){\rule{0.700pt}{0.400pt}}
\multiput(1045.00,529.17)(1.547,-2.000){2}{\rule{0.350pt}{0.400pt}}
\put(1048,526.67){\rule{0.723pt}{0.400pt}}
\multiput(1048.00,527.17)(1.500,-1.000){2}{\rule{0.361pt}{0.400pt}}
\put(1051,525.67){\rule{0.723pt}{0.400pt}}
\multiput(1051.00,526.17)(1.500,-1.000){2}{\rule{0.361pt}{0.400pt}}
\put(1054,524.67){\rule{0.723pt}{0.400pt}}
\multiput(1054.00,525.17)(1.500,-1.000){2}{\rule{0.361pt}{0.400pt}}
\put(1057,523.67){\rule{0.723pt}{0.400pt}}
\multiput(1057.00,524.17)(1.500,-1.000){2}{\rule{0.361pt}{0.400pt}}
\put(1060,522.67){\rule{0.964pt}{0.400pt}}
\multiput(1060.00,523.17)(2.000,-1.000){2}{\rule{0.482pt}{0.400pt}}
\put(1064,521.17){\rule{0.700pt}{0.400pt}}
\multiput(1064.00,522.17)(1.547,-2.000){2}{\rule{0.350pt}{0.400pt}}
\put(1067,519.67){\rule{0.723pt}{0.400pt}}
\multiput(1067.00,520.17)(1.500,-1.000){2}{\rule{0.361pt}{0.400pt}}
\put(1070,518.67){\rule{0.723pt}{0.400pt}}
\multiput(1070.00,519.17)(1.500,-1.000){2}{\rule{0.361pt}{0.400pt}}
\put(1073,517.67){\rule{0.723pt}{0.400pt}}
\multiput(1073.00,518.17)(1.500,-1.000){2}{\rule{0.361pt}{0.400pt}}
\put(1076,516.67){\rule{0.723pt}{0.400pt}}
\multiput(1076.00,517.17)(1.500,-1.000){2}{\rule{0.361pt}{0.400pt}}
\put(1079,515.67){\rule{0.723pt}{0.400pt}}
\multiput(1079.00,516.17)(1.500,-1.000){2}{\rule{0.361pt}{0.400pt}}
\put(1082,514.17){\rule{0.700pt}{0.400pt}}
\multiput(1082.00,515.17)(1.547,-2.000){2}{\rule{0.350pt}{0.400pt}}
\put(1085,512.67){\rule{0.723pt}{0.400pt}}
\multiput(1085.00,513.17)(1.500,-1.000){2}{\rule{0.361pt}{0.400pt}}
\put(1088,511.67){\rule{0.723pt}{0.400pt}}
\multiput(1088.00,512.17)(1.500,-1.000){2}{\rule{0.361pt}{0.400pt}}
\put(1091,510.67){\rule{0.723pt}{0.400pt}}
\multiput(1091.00,511.17)(1.500,-1.000){2}{\rule{0.361pt}{0.400pt}}
\put(1094,509.67){\rule{0.723pt}{0.400pt}}
\multiput(1094.00,510.17)(1.500,-1.000){2}{\rule{0.361pt}{0.400pt}}
\put(1097,508.67){\rule{0.723pt}{0.400pt}}
\multiput(1097.00,509.17)(1.500,-1.000){2}{\rule{0.361pt}{0.400pt}}
\put(1100,507.67){\rule{0.723pt}{0.400pt}}
\multiput(1100.00,508.17)(1.500,-1.000){2}{\rule{0.361pt}{0.400pt}}
\put(1103,506.17){\rule{0.700pt}{0.400pt}}
\multiput(1103.00,507.17)(1.547,-2.000){2}{\rule{0.350pt}{0.400pt}}
\put(1106,504.67){\rule{0.723pt}{0.400pt}}
\multiput(1106.00,505.17)(1.500,-1.000){2}{\rule{0.361pt}{0.400pt}}
\put(1109,503.67){\rule{0.723pt}{0.400pt}}
\multiput(1109.00,504.17)(1.500,-1.000){2}{\rule{0.361pt}{0.400pt}}
\put(1112,502.67){\rule{0.723pt}{0.400pt}}
\multiput(1112.00,503.17)(1.500,-1.000){2}{\rule{0.361pt}{0.400pt}}
\put(1115,501.67){\rule{0.723pt}{0.400pt}}
\multiput(1115.00,502.17)(1.500,-1.000){2}{\rule{0.361pt}{0.400pt}}
\put(1118,500.67){\rule{0.964pt}{0.400pt}}
\multiput(1118.00,501.17)(2.000,-1.000){2}{\rule{0.482pt}{0.400pt}}
\put(1122,499.17){\rule{0.700pt}{0.400pt}}
\multiput(1122.00,500.17)(1.547,-2.000){2}{\rule{0.350pt}{0.400pt}}
\put(1125,497.67){\rule{0.723pt}{0.400pt}}
\multiput(1125.00,498.17)(1.500,-1.000){2}{\rule{0.361pt}{0.400pt}}
\put(1128,496.67){\rule{0.723pt}{0.400pt}}
\multiput(1128.00,497.17)(1.500,-1.000){2}{\rule{0.361pt}{0.400pt}}
\put(1131,495.67){\rule{0.723pt}{0.400pt}}
\multiput(1131.00,496.17)(1.500,-1.000){2}{\rule{0.361pt}{0.400pt}}
\put(1134,494.67){\rule{0.723pt}{0.400pt}}
\multiput(1134.00,495.17)(1.500,-1.000){2}{\rule{0.361pt}{0.400pt}}
\put(1137,493.67){\rule{0.723pt}{0.400pt}}
\multiput(1137.00,494.17)(1.500,-1.000){2}{\rule{0.361pt}{0.400pt}}
\put(1140,492.17){\rule{0.700pt}{0.400pt}}
\multiput(1140.00,493.17)(1.547,-2.000){2}{\rule{0.350pt}{0.400pt}}
\put(1143,490.67){\rule{0.723pt}{0.400pt}}
\multiput(1143.00,491.17)(1.500,-1.000){2}{\rule{0.361pt}{0.400pt}}
\put(1146,489.67){\rule{0.723pt}{0.400pt}}
\multiput(1146.00,490.17)(1.500,-1.000){2}{\rule{0.361pt}{0.400pt}}
\put(1149,488.67){\rule{0.723pt}{0.400pt}}
\multiput(1149.00,489.17)(1.500,-1.000){2}{\rule{0.361pt}{0.400pt}}
\put(1152,487.67){\rule{0.723pt}{0.400pt}}
\multiput(1152.00,488.17)(1.500,-1.000){2}{\rule{0.361pt}{0.400pt}}
\put(1155,486.67){\rule{0.723pt}{0.400pt}}
\multiput(1155.00,487.17)(1.500,-1.000){2}{\rule{0.361pt}{0.400pt}}
\put(1158,485.17){\rule{0.700pt}{0.400pt}}
\multiput(1158.00,486.17)(1.547,-2.000){2}{\rule{0.350pt}{0.400pt}}
\put(1161,483.67){\rule{0.723pt}{0.400pt}}
\multiput(1161.00,484.17)(1.500,-1.000){2}{\rule{0.361pt}{0.400pt}}
\put(1164,482.67){\rule{0.723pt}{0.400pt}}
\multiput(1164.00,483.17)(1.500,-1.000){2}{\rule{0.361pt}{0.400pt}}
\put(1167,481.67){\rule{0.723pt}{0.400pt}}
\multiput(1167.00,482.17)(1.500,-1.000){2}{\rule{0.361pt}{0.400pt}}
\put(1170,480.67){\rule{0.723pt}{0.400pt}}
\multiput(1170.00,481.17)(1.500,-1.000){2}{\rule{0.361pt}{0.400pt}}
\put(1173,479.67){\rule{0.723pt}{0.400pt}}
\multiput(1173.00,480.17)(1.500,-1.000){2}{\rule{0.361pt}{0.400pt}}
\put(1176,478.17){\rule{0.900pt}{0.400pt}}
\multiput(1176.00,479.17)(2.132,-2.000){2}{\rule{0.450pt}{0.400pt}}
\put(1180,476.67){\rule{0.723pt}{0.400pt}}
\multiput(1180.00,477.17)(1.500,-1.000){2}{\rule{0.361pt}{0.400pt}}
\put(1183,475.67){\rule{0.723pt}{0.400pt}}
\multiput(1183.00,476.17)(1.500,-1.000){2}{\rule{0.361pt}{0.400pt}}
\put(1186,474.67){\rule{0.723pt}{0.400pt}}
\multiput(1186.00,475.17)(1.500,-1.000){2}{\rule{0.361pt}{0.400pt}}
\put(1189,473.67){\rule{0.723pt}{0.400pt}}
\multiput(1189.00,474.17)(1.500,-1.000){2}{\rule{0.361pt}{0.400pt}}
\put(1192,472.67){\rule{0.723pt}{0.400pt}}
\multiput(1192.00,473.17)(1.500,-1.000){2}{\rule{0.361pt}{0.400pt}}
\put(1195,471.17){\rule{0.700pt}{0.400pt}}
\multiput(1195.00,472.17)(1.547,-2.000){2}{\rule{0.350pt}{0.400pt}}
\put(1198,469.67){\rule{0.723pt}{0.400pt}}
\multiput(1198.00,470.17)(1.500,-1.000){2}{\rule{0.361pt}{0.400pt}}
\put(1201,468.67){\rule{0.723pt}{0.400pt}}
\multiput(1201.00,469.17)(1.500,-1.000){2}{\rule{0.361pt}{0.400pt}}
\put(1204,467.67){\rule{0.723pt}{0.400pt}}
\multiput(1204.00,468.17)(1.500,-1.000){2}{\rule{0.361pt}{0.400pt}}
\put(1207,466.67){\rule{0.723pt}{0.400pt}}
\multiput(1207.00,467.17)(1.500,-1.000){2}{\rule{0.361pt}{0.400pt}}
\put(1210,465.67){\rule{0.723pt}{0.400pt}}
\multiput(1210.00,466.17)(1.500,-1.000){2}{\rule{0.361pt}{0.400pt}}
\put(1213,464.17){\rule{0.700pt}{0.400pt}}
\multiput(1213.00,465.17)(1.547,-2.000){2}{\rule{0.350pt}{0.400pt}}
\put(1216,462.67){\rule{0.723pt}{0.400pt}}
\multiput(1216.00,463.17)(1.500,-1.000){2}{\rule{0.361pt}{0.400pt}}
\put(1219,461.67){\rule{0.723pt}{0.400pt}}
\multiput(1219.00,462.17)(1.500,-1.000){2}{\rule{0.361pt}{0.400pt}}
\put(1222,460.67){\rule{0.723pt}{0.400pt}}
\multiput(1222.00,461.17)(1.500,-1.000){2}{\rule{0.361pt}{0.400pt}}
\put(1225,459.67){\rule{0.723pt}{0.400pt}}
\multiput(1225.00,460.17)(1.500,-1.000){2}{\rule{0.361pt}{0.400pt}}
\put(1228,458.67){\rule{0.723pt}{0.400pt}}
\multiput(1228.00,459.17)(1.500,-1.000){2}{\rule{0.361pt}{0.400pt}}
\put(1231,457.17){\rule{0.700pt}{0.400pt}}
\multiput(1231.00,458.17)(1.547,-2.000){2}{\rule{0.350pt}{0.400pt}}
\put(1234,455.67){\rule{0.964pt}{0.400pt}}
\multiput(1234.00,456.17)(2.000,-1.000){2}{\rule{0.482pt}{0.400pt}}
\put(1238,454.67){\rule{0.723pt}{0.400pt}}
\multiput(1238.00,455.17)(1.500,-1.000){2}{\rule{0.361pt}{0.400pt}}
\put(1241,453.67){\rule{0.723pt}{0.400pt}}
\multiput(1241.00,454.17)(1.500,-1.000){2}{\rule{0.361pt}{0.400pt}}
\put(1244,452.67){\rule{0.723pt}{0.400pt}}
\multiput(1244.00,453.17)(1.500,-1.000){2}{\rule{0.361pt}{0.400pt}}
\put(1247,451.67){\rule{0.723pt}{0.400pt}}
\multiput(1247.00,452.17)(1.500,-1.000){2}{\rule{0.361pt}{0.400pt}}
\put(1250,450.17){\rule{0.700pt}{0.400pt}}
\multiput(1250.00,451.17)(1.547,-2.000){2}{\rule{0.350pt}{0.400pt}}
\put(1253,448.67){\rule{0.723pt}{0.400pt}}
\multiput(1253.00,449.17)(1.500,-1.000){2}{\rule{0.361pt}{0.400pt}}
\put(1256,447.67){\rule{0.723pt}{0.400pt}}
\multiput(1256.00,448.17)(1.500,-1.000){2}{\rule{0.361pt}{0.400pt}}
\put(1259,446.67){\rule{0.723pt}{0.400pt}}
\multiput(1259.00,447.17)(1.500,-1.000){2}{\rule{0.361pt}{0.400pt}}
\put(1262,445.67){\rule{0.723pt}{0.400pt}}
\multiput(1262.00,446.17)(1.500,-1.000){2}{\rule{0.361pt}{0.400pt}}
\put(1265,444.67){\rule{0.723pt}{0.400pt}}
\multiput(1265.00,445.17)(1.500,-1.000){2}{\rule{0.361pt}{0.400pt}}
\put(1268,443.17){\rule{0.700pt}{0.400pt}}
\multiput(1268.00,444.17)(1.547,-2.000){2}{\rule{0.350pt}{0.400pt}}
\put(1271,441.67){\rule{0.723pt}{0.400pt}}
\multiput(1271.00,442.17)(1.500,-1.000){2}{\rule{0.361pt}{0.400pt}}
\put(1274,440.67){\rule{0.723pt}{0.400pt}}
\multiput(1274.00,441.17)(1.500,-1.000){2}{\rule{0.361pt}{0.400pt}}
\put(1277,439.67){\rule{0.723pt}{0.400pt}}
\multiput(1277.00,440.17)(1.500,-1.000){2}{\rule{0.361pt}{0.400pt}}
\put(1280,438.67){\rule{0.723pt}{0.400pt}}
\multiput(1280.00,439.17)(1.500,-1.000){2}{\rule{0.361pt}{0.400pt}}
\put(1283,437.67){\rule{0.723pt}{0.400pt}}
\multiput(1283.00,438.17)(1.500,-1.000){2}{\rule{0.361pt}{0.400pt}}
\put(1286,436.17){\rule{0.700pt}{0.400pt}}
\multiput(1286.00,437.17)(1.547,-2.000){2}{\rule{0.350pt}{0.400pt}}
\put(1289,434.67){\rule{0.723pt}{0.400pt}}
\multiput(1289.00,435.17)(1.500,-1.000){2}{\rule{0.361pt}{0.400pt}}
\put(1292,433.67){\rule{0.723pt}{0.400pt}}
\multiput(1292.00,434.17)(1.500,-1.000){2}{\rule{0.361pt}{0.400pt}}
\put(1295,432.67){\rule{0.964pt}{0.400pt}}
\multiput(1295.00,433.17)(2.000,-1.000){2}{\rule{0.482pt}{0.400pt}}
\put(1299,431.67){\rule{0.723pt}{0.400pt}}
\multiput(1299.00,432.17)(1.500,-1.000){2}{\rule{0.361pt}{0.400pt}}
\put(1302,430.67){\rule{0.723pt}{0.400pt}}
\multiput(1302.00,431.17)(1.500,-1.000){2}{\rule{0.361pt}{0.400pt}}
\put(1305,429.17){\rule{0.700pt}{0.400pt}}
\multiput(1305.00,430.17)(1.547,-2.000){2}{\rule{0.350pt}{0.400pt}}
\put(1308,427.67){\rule{0.723pt}{0.400pt}}
\multiput(1308.00,428.17)(1.500,-1.000){2}{\rule{0.361pt}{0.400pt}}
\put(1311,426.67){\rule{0.723pt}{0.400pt}}
\multiput(1311.00,427.17)(1.500,-1.000){2}{\rule{0.361pt}{0.400pt}}
\put(1314,425.67){\rule{0.723pt}{0.400pt}}
\multiput(1314.00,426.17)(1.500,-1.000){2}{\rule{0.361pt}{0.400pt}}
\put(1317,424.67){\rule{0.723pt}{0.400pt}}
\multiput(1317.00,425.17)(1.500,-1.000){2}{\rule{0.361pt}{0.400pt}}
\put(1320,423.67){\rule{0.723pt}{0.400pt}}
\multiput(1320.00,424.17)(1.500,-1.000){2}{\rule{0.361pt}{0.400pt}}
\put(1323,422.17){\rule{0.700pt}{0.400pt}}
\multiput(1323.00,423.17)(1.547,-2.000){2}{\rule{0.350pt}{0.400pt}}
\put(1326,420.67){\rule{0.723pt}{0.400pt}}
\multiput(1326.00,421.17)(1.500,-1.000){2}{\rule{0.361pt}{0.400pt}}
\put(1329,419.67){\rule{0.723pt}{0.400pt}}
\multiput(1329.00,420.17)(1.500,-1.000){2}{\rule{0.361pt}{0.400pt}}
\put(1332,418.67){\rule{0.723pt}{0.400pt}}
\multiput(1332.00,419.17)(1.500,-1.000){2}{\rule{0.361pt}{0.400pt}}
\put(1335,417.67){\rule{0.723pt}{0.400pt}}
\multiput(1335.00,418.17)(1.500,-1.000){2}{\rule{0.361pt}{0.400pt}}
\put(1338,416.67){\rule{0.723pt}{0.400pt}}
\multiput(1338.00,417.17)(1.500,-1.000){2}{\rule{0.361pt}{0.400pt}}
\put(1341,415.17){\rule{0.700pt}{0.400pt}}
\multiput(1341.00,416.17)(1.547,-2.000){2}{\rule{0.350pt}{0.400pt}}
\put(1344,413.67){\rule{0.723pt}{0.400pt}}
\multiput(1344.00,414.17)(1.500,-1.000){2}{\rule{0.361pt}{0.400pt}}
\put(1347,412.67){\rule{0.723pt}{0.400pt}}
\multiput(1347.00,413.17)(1.500,-1.000){2}{\rule{0.361pt}{0.400pt}}
\put(1350,411.67){\rule{0.723pt}{0.400pt}}
\multiput(1350.00,412.17)(1.500,-1.000){2}{\rule{0.361pt}{0.400pt}}
\put(1353,410.67){\rule{0.964pt}{0.400pt}}
\multiput(1353.00,411.17)(2.000,-1.000){2}{\rule{0.482pt}{0.400pt}}
\put(1357,409.67){\rule{0.723pt}{0.400pt}}
\multiput(1357.00,410.17)(1.500,-1.000){2}{\rule{0.361pt}{0.400pt}}
\put(1360,408.17){\rule{0.700pt}{0.400pt}}
\multiput(1360.00,409.17)(1.547,-2.000){2}{\rule{0.350pt}{0.400pt}}
\put(1363,406.67){\rule{0.723pt}{0.400pt}}
\multiput(1363.00,407.17)(1.500,-1.000){2}{\rule{0.361pt}{0.400pt}}
\put(1366,405.67){\rule{0.723pt}{0.400pt}}
\multiput(1366.00,406.17)(1.500,-1.000){2}{\rule{0.361pt}{0.400pt}}
\put(1369,404.67){\rule{0.723pt}{0.400pt}}
\multiput(1369.00,405.17)(1.500,-1.000){2}{\rule{0.361pt}{0.400pt}}
\put(1372,403.67){\rule{0.723pt}{0.400pt}}
\multiput(1372.00,404.17)(1.500,-1.000){2}{\rule{0.361pt}{0.400pt}}
\put(1375,402.67){\rule{0.723pt}{0.400pt}}
\multiput(1375.00,403.17)(1.500,-1.000){2}{\rule{0.361pt}{0.400pt}}
\put(1378,401.17){\rule{0.700pt}{0.400pt}}
\multiput(1378.00,402.17)(1.547,-2.000){2}{\rule{0.350pt}{0.400pt}}
\put(1381,399.67){\rule{0.723pt}{0.400pt}}
\multiput(1381.00,400.17)(1.500,-1.000){2}{\rule{0.361pt}{0.400pt}}
\put(1384,398.67){\rule{0.723pt}{0.400pt}}
\multiput(1384.00,399.17)(1.500,-1.000){2}{\rule{0.361pt}{0.400pt}}
\put(1387,397.67){\rule{0.723pt}{0.400pt}}
\multiput(1387.00,398.17)(1.500,-1.000){2}{\rule{0.361pt}{0.400pt}}
\put(1390,396.67){\rule{0.723pt}{0.400pt}}
\multiput(1390.00,397.17)(1.500,-1.000){2}{\rule{0.361pt}{0.400pt}}
\put(1393,395.67){\rule{0.723pt}{0.400pt}}
\multiput(1393.00,396.17)(1.500,-1.000){2}{\rule{0.361pt}{0.400pt}}
\put(1396,394.17){\rule{0.700pt}{0.400pt}}
\multiput(1396.00,395.17)(1.547,-2.000){2}{\rule{0.350pt}{0.400pt}}
\put(1399,392.67){\rule{0.723pt}{0.400pt}}
\multiput(1399.00,393.17)(1.500,-1.000){2}{\rule{0.361pt}{0.400pt}}
\put(1402,391.67){\rule{0.723pt}{0.400pt}}
\multiput(1402.00,392.17)(1.500,-1.000){2}{\rule{0.361pt}{0.400pt}}
\put(1405,390.67){\rule{0.723pt}{0.400pt}}
\multiput(1405.00,391.17)(1.500,-1.000){2}{\rule{0.361pt}{0.400pt}}
\put(1408,389.67){\rule{0.723pt}{0.400pt}}
\multiput(1408.00,390.17)(1.500,-1.000){2}{\rule{0.361pt}{0.400pt}}
\put(1411,388.67){\rule{0.964pt}{0.400pt}}
\multiput(1411.00,389.17)(2.000,-1.000){2}{\rule{0.482pt}{0.400pt}}
\put(1411,387.17){\rule{0.900pt}{0.400pt}}
\multiput(1413.13,388.17)(-2.132,-2.000){2}{\rule{0.450pt}{0.400pt}}
\put(1408,385.67){\rule{0.723pt}{0.400pt}}
\multiput(1409.50,386.17)(-1.500,-1.000){2}{\rule{0.361pt}{0.400pt}}
\put(1405,384.67){\rule{0.723pt}{0.400pt}}
\multiput(1406.50,385.17)(-1.500,-1.000){2}{\rule{0.361pt}{0.400pt}}
\put(1402,383.67){\rule{0.723pt}{0.400pt}}
\multiput(1403.50,384.17)(-1.500,-1.000){2}{\rule{0.361pt}{0.400pt}}
\put(1399,382.67){\rule{0.723pt}{0.400pt}}
\multiput(1400.50,383.17)(-1.500,-1.000){2}{\rule{0.361pt}{0.400pt}}
\put(1396,381.67){\rule{0.723pt}{0.400pt}}
\multiput(1397.50,382.17)(-1.500,-1.000){2}{\rule{0.361pt}{0.400pt}}
\put(1393,380.17){\rule{0.700pt}{0.400pt}}
\multiput(1394.55,381.17)(-1.547,-2.000){2}{\rule{0.350pt}{0.400pt}}
\put(1390,378.67){\rule{0.723pt}{0.400pt}}
\multiput(1391.50,379.17)(-1.500,-1.000){2}{\rule{0.361pt}{0.400pt}}
\put(1387,377.67){\rule{0.723pt}{0.400pt}}
\multiput(1388.50,378.17)(-1.500,-1.000){2}{\rule{0.361pt}{0.400pt}}
\put(1384,376.67){\rule{0.723pt}{0.400pt}}
\multiput(1385.50,377.17)(-1.500,-1.000){2}{\rule{0.361pt}{0.400pt}}
\put(1381,375.67){\rule{0.723pt}{0.400pt}}
\multiput(1382.50,376.17)(-1.500,-1.000){2}{\rule{0.361pt}{0.400pt}}
\put(1378,374.67){\rule{0.723pt}{0.400pt}}
\multiput(1379.50,375.17)(-1.500,-1.000){2}{\rule{0.361pt}{0.400pt}}
\put(1375,373.17){\rule{0.700pt}{0.400pt}}
\multiput(1376.55,374.17)(-1.547,-2.000){2}{\rule{0.350pt}{0.400pt}}
\put(1372,371.67){\rule{0.723pt}{0.400pt}}
\multiput(1373.50,372.17)(-1.500,-1.000){2}{\rule{0.361pt}{0.400pt}}
\put(1369,370.67){\rule{0.723pt}{0.400pt}}
\multiput(1370.50,371.17)(-1.500,-1.000){2}{\rule{0.361pt}{0.400pt}}
\put(1366,369.67){\rule{0.723pt}{0.400pt}}
\multiput(1367.50,370.17)(-1.500,-1.000){2}{\rule{0.361pt}{0.400pt}}
\put(1363,368.67){\rule{0.723pt}{0.400pt}}
\multiput(1364.50,369.17)(-1.500,-1.000){2}{\rule{0.361pt}{0.400pt}}
\put(1360,367.67){\rule{0.723pt}{0.400pt}}
\multiput(1361.50,368.17)(-1.500,-1.000){2}{\rule{0.361pt}{0.400pt}}
\put(1357,366.17){\rule{0.700pt}{0.400pt}}
\multiput(1358.55,367.17)(-1.547,-2.000){2}{\rule{0.350pt}{0.400pt}}
\put(1353,364.67){\rule{0.964pt}{0.400pt}}
\multiput(1355.00,365.17)(-2.000,-1.000){2}{\rule{0.482pt}{0.400pt}}
\put(1350,363.67){\rule{0.723pt}{0.400pt}}
\multiput(1351.50,364.17)(-1.500,-1.000){2}{\rule{0.361pt}{0.400pt}}
\put(1347,362.67){\rule{0.723pt}{0.400pt}}
\multiput(1348.50,363.17)(-1.500,-1.000){2}{\rule{0.361pt}{0.400pt}}
\put(1344,361.67){\rule{0.723pt}{0.400pt}}
\multiput(1345.50,362.17)(-1.500,-1.000){2}{\rule{0.361pt}{0.400pt}}
\put(1341,360.67){\rule{0.723pt}{0.400pt}}
\multiput(1342.50,361.17)(-1.500,-1.000){2}{\rule{0.361pt}{0.400pt}}
\put(1338,359.17){\rule{0.700pt}{0.400pt}}
\multiput(1339.55,360.17)(-1.547,-2.000){2}{\rule{0.350pt}{0.400pt}}
\put(1335,357.67){\rule{0.723pt}{0.400pt}}
\multiput(1336.50,358.17)(-1.500,-1.000){2}{\rule{0.361pt}{0.400pt}}
\put(1332,356.67){\rule{0.723pt}{0.400pt}}
\multiput(1333.50,357.17)(-1.500,-1.000){2}{\rule{0.361pt}{0.400pt}}
\put(1329,355.67){\rule{0.723pt}{0.400pt}}
\multiput(1330.50,356.17)(-1.500,-1.000){2}{\rule{0.361pt}{0.400pt}}
\put(1326,354.67){\rule{0.723pt}{0.400pt}}
\multiput(1327.50,355.17)(-1.500,-1.000){2}{\rule{0.361pt}{0.400pt}}
\put(1323,353.67){\rule{0.723pt}{0.400pt}}
\multiput(1324.50,354.17)(-1.500,-1.000){2}{\rule{0.361pt}{0.400pt}}
\put(1320,352.17){\rule{0.700pt}{0.400pt}}
\multiput(1321.55,353.17)(-1.547,-2.000){2}{\rule{0.350pt}{0.400pt}}
\put(1317,350.67){\rule{0.723pt}{0.400pt}}
\multiput(1318.50,351.17)(-1.500,-1.000){2}{\rule{0.361pt}{0.400pt}}
\put(1314,349.67){\rule{0.723pt}{0.400pt}}
\multiput(1315.50,350.17)(-1.500,-1.000){2}{\rule{0.361pt}{0.400pt}}
\put(1311,348.67){\rule{0.723pt}{0.400pt}}
\multiput(1312.50,349.17)(-1.500,-1.000){2}{\rule{0.361pt}{0.400pt}}
\put(1308,347.67){\rule{0.723pt}{0.400pt}}
\multiput(1309.50,348.17)(-1.500,-1.000){2}{\rule{0.361pt}{0.400pt}}
\put(1305,346.67){\rule{0.723pt}{0.400pt}}
\multiput(1306.50,347.17)(-1.500,-1.000){2}{\rule{0.361pt}{0.400pt}}
\put(1302,345.17){\rule{0.700pt}{0.400pt}}
\multiput(1303.55,346.17)(-1.547,-2.000){2}{\rule{0.350pt}{0.400pt}}
\put(1299,343.67){\rule{0.723pt}{0.400pt}}
\multiput(1300.50,344.17)(-1.500,-1.000){2}{\rule{0.361pt}{0.400pt}}
\put(1295,342.67){\rule{0.964pt}{0.400pt}}
\multiput(1297.00,343.17)(-2.000,-1.000){2}{\rule{0.482pt}{0.400pt}}
\put(1292,341.67){\rule{0.723pt}{0.400pt}}
\multiput(1293.50,342.17)(-1.500,-1.000){2}{\rule{0.361pt}{0.400pt}}
\put(1289,340.67){\rule{0.723pt}{0.400pt}}
\multiput(1290.50,341.17)(-1.500,-1.000){2}{\rule{0.361pt}{0.400pt}}
\put(1286,339.67){\rule{0.723pt}{0.400pt}}
\multiput(1287.50,340.17)(-1.500,-1.000){2}{\rule{0.361pt}{0.400pt}}
\put(1283,338.17){\rule{0.700pt}{0.400pt}}
\multiput(1284.55,339.17)(-1.547,-2.000){2}{\rule{0.350pt}{0.400pt}}
\put(1280,336.67){\rule{0.723pt}{0.400pt}}
\multiput(1281.50,337.17)(-1.500,-1.000){2}{\rule{0.361pt}{0.400pt}}
\put(1277,335.67){\rule{0.723pt}{0.400pt}}
\multiput(1278.50,336.17)(-1.500,-1.000){2}{\rule{0.361pt}{0.400pt}}
\put(1274,334.67){\rule{0.723pt}{0.400pt}}
\multiput(1275.50,335.17)(-1.500,-1.000){2}{\rule{0.361pt}{0.400pt}}
\put(1271,333.67){\rule{0.723pt}{0.400pt}}
\multiput(1272.50,334.17)(-1.500,-1.000){2}{\rule{0.361pt}{0.400pt}}
\put(1268,332.67){\rule{0.723pt}{0.400pt}}
\multiput(1269.50,333.17)(-1.500,-1.000){2}{\rule{0.361pt}{0.400pt}}
\put(1265,331.17){\rule{0.700pt}{0.400pt}}
\multiput(1266.55,332.17)(-1.547,-2.000){2}{\rule{0.350pt}{0.400pt}}
\put(1262,329.67){\rule{0.723pt}{0.400pt}}
\multiput(1263.50,330.17)(-1.500,-1.000){2}{\rule{0.361pt}{0.400pt}}
\put(1259,328.67){\rule{0.723pt}{0.400pt}}
\multiput(1260.50,329.17)(-1.500,-1.000){2}{\rule{0.361pt}{0.400pt}}
\put(1256,327.67){\rule{0.723pt}{0.400pt}}
\multiput(1257.50,328.17)(-1.500,-1.000){2}{\rule{0.361pt}{0.400pt}}
\put(1253,326.67){\rule{0.723pt}{0.400pt}}
\multiput(1254.50,327.17)(-1.500,-1.000){2}{\rule{0.361pt}{0.400pt}}
\put(1250,325.67){\rule{0.723pt}{0.400pt}}
\multiput(1251.50,326.17)(-1.500,-1.000){2}{\rule{0.361pt}{0.400pt}}
\put(1247,324.17){\rule{0.700pt}{0.400pt}}
\multiput(1248.55,325.17)(-1.547,-2.000){2}{\rule{0.350pt}{0.400pt}}
\put(1244,322.67){\rule{0.723pt}{0.400pt}}
\multiput(1245.50,323.17)(-1.500,-1.000){2}{\rule{0.361pt}{0.400pt}}
\put(1241,321.67){\rule{0.723pt}{0.400pt}}
\multiput(1242.50,322.17)(-1.500,-1.000){2}{\rule{0.361pt}{0.400pt}}
\put(1238,320.67){\rule{0.723pt}{0.400pt}}
\multiput(1239.50,321.17)(-1.500,-1.000){2}{\rule{0.361pt}{0.400pt}}
\put(1234,319.67){\rule{0.964pt}{0.400pt}}
\multiput(1236.00,320.17)(-2.000,-1.000){2}{\rule{0.482pt}{0.400pt}}
\put(1231,318.67){\rule{0.723pt}{0.400pt}}
\multiput(1232.50,319.17)(-1.500,-1.000){2}{\rule{0.361pt}{0.400pt}}
\put(1228,317.17){\rule{0.700pt}{0.400pt}}
\multiput(1229.55,318.17)(-1.547,-2.000){2}{\rule{0.350pt}{0.400pt}}
\put(1225,315.67){\rule{0.723pt}{0.400pt}}
\multiput(1226.50,316.17)(-1.500,-1.000){2}{\rule{0.361pt}{0.400pt}}
\put(1222,314.67){\rule{0.723pt}{0.400pt}}
\multiput(1223.50,315.17)(-1.500,-1.000){2}{\rule{0.361pt}{0.400pt}}
\put(1219,313.67){\rule{0.723pt}{0.400pt}}
\multiput(1220.50,314.17)(-1.500,-1.000){2}{\rule{0.361pt}{0.400pt}}
\put(1216,312.67){\rule{0.723pt}{0.400pt}}
\multiput(1217.50,313.17)(-1.500,-1.000){2}{\rule{0.361pt}{0.400pt}}
\put(1213,311.67){\rule{0.723pt}{0.400pt}}
\multiput(1214.50,312.17)(-1.500,-1.000){2}{\rule{0.361pt}{0.400pt}}
\put(1210,310.67){\rule{0.723pt}{0.400pt}}
\multiput(1211.50,311.17)(-1.500,-1.000){2}{\rule{0.361pt}{0.400pt}}
\put(1207,309.17){\rule{0.700pt}{0.400pt}}
\multiput(1208.55,310.17)(-1.547,-2.000){2}{\rule{0.350pt}{0.400pt}}
\put(1204,307.67){\rule{0.723pt}{0.400pt}}
\multiput(1205.50,308.17)(-1.500,-1.000){2}{\rule{0.361pt}{0.400pt}}
\put(1201,306.67){\rule{0.723pt}{0.400pt}}
\multiput(1202.50,307.17)(-1.500,-1.000){2}{\rule{0.361pt}{0.400pt}}
\put(1198,305.67){\rule{0.723pt}{0.400pt}}
\multiput(1199.50,306.17)(-1.500,-1.000){2}{\rule{0.361pt}{0.400pt}}
\put(1195,304.67){\rule{0.723pt}{0.400pt}}
\multiput(1196.50,305.17)(-1.500,-1.000){2}{\rule{0.361pt}{0.400pt}}
\put(1192,303.67){\rule{0.723pt}{0.400pt}}
\multiput(1193.50,304.17)(-1.500,-1.000){2}{\rule{0.361pt}{0.400pt}}
\put(1189,302.17){\rule{0.700pt}{0.400pt}}
\multiput(1190.55,303.17)(-1.547,-2.000){2}{\rule{0.350pt}{0.400pt}}
\put(1186,300.67){\rule{0.723pt}{0.400pt}}
\multiput(1187.50,301.17)(-1.500,-1.000){2}{\rule{0.361pt}{0.400pt}}
\put(1183,299.67){\rule{0.723pt}{0.400pt}}
\multiput(1184.50,300.17)(-1.500,-1.000){2}{\rule{0.361pt}{0.400pt}}
\put(1180,298.67){\rule{0.723pt}{0.400pt}}
\multiput(1181.50,299.17)(-1.500,-1.000){2}{\rule{0.361pt}{0.400pt}}
\put(1176,297.67){\rule{0.964pt}{0.400pt}}
\multiput(1178.00,298.17)(-2.000,-1.000){2}{\rule{0.482pt}{0.400pt}}
\put(1173,296.67){\rule{0.723pt}{0.400pt}}
\multiput(1174.50,297.17)(-1.500,-1.000){2}{\rule{0.361pt}{0.400pt}}
\put(1170,295.17){\rule{0.700pt}{0.400pt}}
\multiput(1171.55,296.17)(-1.547,-2.000){2}{\rule{0.350pt}{0.400pt}}
\put(1167,293.67){\rule{0.723pt}{0.400pt}}
\multiput(1168.50,294.17)(-1.500,-1.000){2}{\rule{0.361pt}{0.400pt}}
\put(1164,292.67){\rule{0.723pt}{0.400pt}}
\multiput(1165.50,293.17)(-1.500,-1.000){2}{\rule{0.361pt}{0.400pt}}
\put(1161,291.67){\rule{0.723pt}{0.400pt}}
\multiput(1162.50,292.17)(-1.500,-1.000){2}{\rule{0.361pt}{0.400pt}}
\put(1158,290.67){\rule{0.723pt}{0.400pt}}
\multiput(1159.50,291.17)(-1.500,-1.000){2}{\rule{0.361pt}{0.400pt}}
\put(1155,289.67){\rule{0.723pt}{0.400pt}}
\multiput(1156.50,290.17)(-1.500,-1.000){2}{\rule{0.361pt}{0.400pt}}
\put(1152,288.17){\rule{0.700pt}{0.400pt}}
\multiput(1153.55,289.17)(-1.547,-2.000){2}{\rule{0.350pt}{0.400pt}}
\put(1149,286.67){\rule{0.723pt}{0.400pt}}
\multiput(1150.50,287.17)(-1.500,-1.000){2}{\rule{0.361pt}{0.400pt}}
\put(1146,285.67){\rule{0.723pt}{0.400pt}}
\multiput(1147.50,286.17)(-1.500,-1.000){2}{\rule{0.361pt}{0.400pt}}
\put(1143,284.67){\rule{0.723pt}{0.400pt}}
\multiput(1144.50,285.17)(-1.500,-1.000){2}{\rule{0.361pt}{0.400pt}}
\put(1140,283.67){\rule{0.723pt}{0.400pt}}
\multiput(1141.50,284.17)(-1.500,-1.000){2}{\rule{0.361pt}{0.400pt}}
\put(1137,282.67){\rule{0.723pt}{0.400pt}}
\multiput(1138.50,283.17)(-1.500,-1.000){2}{\rule{0.361pt}{0.400pt}}
\put(1134,281.17){\rule{0.700pt}{0.400pt}}
\multiput(1135.55,282.17)(-1.547,-2.000){2}{\rule{0.350pt}{0.400pt}}
\put(1131,279.67){\rule{0.723pt}{0.400pt}}
\multiput(1132.50,280.17)(-1.500,-1.000){2}{\rule{0.361pt}{0.400pt}}
\put(1128,278.67){\rule{0.723pt}{0.400pt}}
\multiput(1129.50,279.17)(-1.500,-1.000){2}{\rule{0.361pt}{0.400pt}}
\put(1125,277.67){\rule{0.723pt}{0.400pt}}
\multiput(1126.50,278.17)(-1.500,-1.000){2}{\rule{0.361pt}{0.400pt}}
\put(1122,276.67){\rule{0.723pt}{0.400pt}}
\multiput(1123.50,277.17)(-1.500,-1.000){2}{\rule{0.361pt}{0.400pt}}
\put(1118,275.67){\rule{0.964pt}{0.400pt}}
\multiput(1120.00,276.17)(-2.000,-1.000){2}{\rule{0.482pt}{0.400pt}}
\put(1115,274.17){\rule{0.700pt}{0.400pt}}
\multiput(1116.55,275.17)(-1.547,-2.000){2}{\rule{0.350pt}{0.400pt}}
\put(1112,272.67){\rule{0.723pt}{0.400pt}}
\multiput(1113.50,273.17)(-1.500,-1.000){2}{\rule{0.361pt}{0.400pt}}
\put(1109,271.67){\rule{0.723pt}{0.400pt}}
\multiput(1110.50,272.17)(-1.500,-1.000){2}{\rule{0.361pt}{0.400pt}}
\put(1106,270.67){\rule{0.723pt}{0.400pt}}
\multiput(1107.50,271.17)(-1.500,-1.000){2}{\rule{0.361pt}{0.400pt}}
\put(1103,269.67){\rule{0.723pt}{0.400pt}}
\multiput(1104.50,270.17)(-1.500,-1.000){2}{\rule{0.361pt}{0.400pt}}
\put(1100,268.67){\rule{0.723pt}{0.400pt}}
\multiput(1101.50,269.17)(-1.500,-1.000){2}{\rule{0.361pt}{0.400pt}}
\put(1097,267.17){\rule{0.700pt}{0.400pt}}
\multiput(1098.55,268.17)(-1.547,-2.000){2}{\rule{0.350pt}{0.400pt}}
\put(1094,265.67){\rule{0.723pt}{0.400pt}}
\multiput(1095.50,266.17)(-1.500,-1.000){2}{\rule{0.361pt}{0.400pt}}
\put(1091,264.67){\rule{0.723pt}{0.400pt}}
\multiput(1092.50,265.17)(-1.500,-1.000){2}{\rule{0.361pt}{0.400pt}}
\put(1088,263.67){\rule{0.723pt}{0.400pt}}
\multiput(1089.50,264.17)(-1.500,-1.000){2}{\rule{0.361pt}{0.400pt}}
\put(1085,262.67){\rule{0.723pt}{0.400pt}}
\multiput(1086.50,263.17)(-1.500,-1.000){2}{\rule{0.361pt}{0.400pt}}
\put(1082,261.67){\rule{0.723pt}{0.400pt}}
\multiput(1083.50,262.17)(-1.500,-1.000){2}{\rule{0.361pt}{0.400pt}}
\put(1079,260.17){\rule{0.700pt}{0.400pt}}
\multiput(1080.55,261.17)(-1.547,-2.000){2}{\rule{0.350pt}{0.400pt}}
\put(1076,258.67){\rule{0.723pt}{0.400pt}}
\multiput(1077.50,259.17)(-1.500,-1.000){2}{\rule{0.361pt}{0.400pt}}
\put(1073,257.67){\rule{0.723pt}{0.400pt}}
\multiput(1074.50,258.17)(-1.500,-1.000){2}{\rule{0.361pt}{0.400pt}}
\put(1070,256.67){\rule{0.723pt}{0.400pt}}
\multiput(1071.50,257.17)(-1.500,-1.000){2}{\rule{0.361pt}{0.400pt}}
\put(1067,255.67){\rule{0.723pt}{0.400pt}}
\multiput(1068.50,256.17)(-1.500,-1.000){2}{\rule{0.361pt}{0.400pt}}
\put(1064,254.67){\rule{0.723pt}{0.400pt}}
\multiput(1065.50,255.17)(-1.500,-1.000){2}{\rule{0.361pt}{0.400pt}}
\put(1060,253.17){\rule{0.900pt}{0.400pt}}
\multiput(1062.13,254.17)(-2.132,-2.000){2}{\rule{0.450pt}{0.400pt}}
\put(1057,251.67){\rule{0.723pt}{0.400pt}}
\multiput(1058.50,252.17)(-1.500,-1.000){2}{\rule{0.361pt}{0.400pt}}
\put(1054,250.67){\rule{0.723pt}{0.400pt}}
\multiput(1055.50,251.17)(-1.500,-1.000){2}{\rule{0.361pt}{0.400pt}}
\put(1051,249.67){\rule{0.723pt}{0.400pt}}
\multiput(1052.50,250.17)(-1.500,-1.000){2}{\rule{0.361pt}{0.400pt}}
\put(1048,248.67){\rule{0.723pt}{0.400pt}}
\multiput(1049.50,249.17)(-1.500,-1.000){2}{\rule{0.361pt}{0.400pt}}
\put(1045,247.67){\rule{0.723pt}{0.400pt}}
\multiput(1046.50,248.17)(-1.500,-1.000){2}{\rule{0.361pt}{0.400pt}}
\put(1042,246.17){\rule{0.700pt}{0.400pt}}
\multiput(1043.55,247.17)(-1.547,-2.000){2}{\rule{0.350pt}{0.400pt}}
\put(1039,244.67){\rule{0.723pt}{0.400pt}}
\multiput(1040.50,245.17)(-1.500,-1.000){2}{\rule{0.361pt}{0.400pt}}
\put(1036,243.67){\rule{0.723pt}{0.400pt}}
\multiput(1037.50,244.17)(-1.500,-1.000){2}{\rule{0.361pt}{0.400pt}}
\put(1033,242.67){\rule{0.723pt}{0.400pt}}
\multiput(1034.50,243.17)(-1.500,-1.000){2}{\rule{0.361pt}{0.400pt}}
\put(1030,241.67){\rule{0.723pt}{0.400pt}}
\multiput(1031.50,242.17)(-1.500,-1.000){2}{\rule{0.361pt}{0.400pt}}
\put(1027,240.67){\rule{0.723pt}{0.400pt}}
\multiput(1028.50,241.17)(-1.500,-1.000){2}{\rule{0.361pt}{0.400pt}}
\put(1024,239.17){\rule{0.700pt}{0.400pt}}
\multiput(1025.55,240.17)(-1.547,-2.000){2}{\rule{0.350pt}{0.400pt}}
\put(1021,237.67){\rule{0.723pt}{0.400pt}}
\multiput(1022.50,238.17)(-1.500,-1.000){2}{\rule{0.361pt}{0.400pt}}
\put(1018,236.67){\rule{0.723pt}{0.400pt}}
\multiput(1019.50,237.17)(-1.500,-1.000){2}{\rule{0.361pt}{0.400pt}}
\put(1015,235.67){\rule{0.723pt}{0.400pt}}
\multiput(1016.50,236.17)(-1.500,-1.000){2}{\rule{0.361pt}{0.400pt}}
\put(1012,234.67){\rule{0.723pt}{0.400pt}}
\multiput(1013.50,235.17)(-1.500,-1.000){2}{\rule{0.361pt}{0.400pt}}
\put(1009,233.67){\rule{0.723pt}{0.400pt}}
\multiput(1010.50,234.17)(-1.500,-1.000){2}{\rule{0.361pt}{0.400pt}}
\put(1006,232.17){\rule{0.700pt}{0.400pt}}
\multiput(1007.55,233.17)(-1.547,-2.000){2}{\rule{0.350pt}{0.400pt}}
\put(1002,230.67){\rule{0.964pt}{0.400pt}}
\multiput(1004.00,231.17)(-2.000,-1.000){2}{\rule{0.482pt}{0.400pt}}
\put(999,229.67){\rule{0.723pt}{0.400pt}}
\multiput(1000.50,230.17)(-1.500,-1.000){2}{\rule{0.361pt}{0.400pt}}
\put(996,228.67){\rule{0.723pt}{0.400pt}}
\multiput(997.50,229.17)(-1.500,-1.000){2}{\rule{0.361pt}{0.400pt}}
\put(993,227.67){\rule{0.723pt}{0.400pt}}
\multiput(994.50,228.17)(-1.500,-1.000){2}{\rule{0.361pt}{0.400pt}}
\put(990,226.67){\rule{0.723pt}{0.400pt}}
\multiput(991.50,227.17)(-1.500,-1.000){2}{\rule{0.361pt}{0.400pt}}
\put(987,225.17){\rule{0.700pt}{0.400pt}}
\multiput(988.55,226.17)(-1.547,-2.000){2}{\rule{0.350pt}{0.400pt}}
\put(984,223.67){\rule{0.723pt}{0.400pt}}
\multiput(985.50,224.17)(-1.500,-1.000){2}{\rule{0.361pt}{0.400pt}}
\put(981,222.67){\rule{0.723pt}{0.400pt}}
\multiput(982.50,223.17)(-1.500,-1.000){2}{\rule{0.361pt}{0.400pt}}
\put(978,221.67){\rule{0.723pt}{0.400pt}}
\multiput(979.50,222.17)(-1.500,-1.000){2}{\rule{0.361pt}{0.400pt}}
\put(975,220.67){\rule{0.723pt}{0.400pt}}
\multiput(976.50,221.17)(-1.500,-1.000){2}{\rule{0.361pt}{0.400pt}}
\put(972,219.67){\rule{0.723pt}{0.400pt}}
\multiput(973.50,220.17)(-1.500,-1.000){2}{\rule{0.361pt}{0.400pt}}
\put(969,218.17){\rule{0.700pt}{0.400pt}}
\multiput(970.55,219.17)(-1.547,-2.000){2}{\rule{0.350pt}{0.400pt}}
\put(966,216.67){\rule{0.723pt}{0.400pt}}
\multiput(967.50,217.17)(-1.500,-1.000){2}{\rule{0.361pt}{0.400pt}}
\put(963,215.67){\rule{0.723pt}{0.400pt}}
\multiput(964.50,216.17)(-1.500,-1.000){2}{\rule{0.361pt}{0.400pt}}
\put(960,214.67){\rule{0.723pt}{0.400pt}}
\multiput(961.50,215.17)(-1.500,-1.000){2}{\rule{0.361pt}{0.400pt}}
\put(957,213.67){\rule{0.723pt}{0.400pt}}
\multiput(958.50,214.17)(-1.500,-1.000){2}{\rule{0.361pt}{0.400pt}}
\put(954,212.67){\rule{0.723pt}{0.400pt}}
\multiput(955.50,213.17)(-1.500,-1.000){2}{\rule{0.361pt}{0.400pt}}
\put(951,211.17){\rule{0.700pt}{0.400pt}}
\multiput(952.55,212.17)(-1.547,-2.000){2}{\rule{0.350pt}{0.400pt}}
\put(948,209.67){\rule{0.723pt}{0.400pt}}
\multiput(949.50,210.17)(-1.500,-1.000){2}{\rule{0.361pt}{0.400pt}}
\put(944,208.67){\rule{0.964pt}{0.400pt}}
\multiput(946.00,209.17)(-2.000,-1.000){2}{\rule{0.482pt}{0.400pt}}
\put(941,207.67){\rule{0.723pt}{0.400pt}}
\multiput(942.50,208.17)(-1.500,-1.000){2}{\rule{0.361pt}{0.400pt}}
\put(938,206.67){\rule{0.723pt}{0.400pt}}
\multiput(939.50,207.17)(-1.500,-1.000){2}{\rule{0.361pt}{0.400pt}}
\put(935,205.67){\rule{0.723pt}{0.400pt}}
\multiput(936.50,206.17)(-1.500,-1.000){2}{\rule{0.361pt}{0.400pt}}
\put(932,204.17){\rule{0.700pt}{0.400pt}}
\multiput(933.55,205.17)(-1.547,-2.000){2}{\rule{0.350pt}{0.400pt}}
\put(929,202.67){\rule{0.723pt}{0.400pt}}
\multiput(930.50,203.17)(-1.500,-1.000){2}{\rule{0.361pt}{0.400pt}}
\put(926,201.67){\rule{0.723pt}{0.400pt}}
\multiput(927.50,202.17)(-1.500,-1.000){2}{\rule{0.361pt}{0.400pt}}
\put(923,200.67){\rule{0.723pt}{0.400pt}}
\multiput(924.50,201.17)(-1.500,-1.000){2}{\rule{0.361pt}{0.400pt}}
\put(920,199.67){\rule{0.723pt}{0.400pt}}
\multiput(921.50,200.17)(-1.500,-1.000){2}{\rule{0.361pt}{0.400pt}}
\put(917,198.67){\rule{0.723pt}{0.400pt}}
\multiput(918.50,199.17)(-1.500,-1.000){2}{\rule{0.361pt}{0.400pt}}
\put(914,197.17){\rule{0.700pt}{0.400pt}}
\multiput(915.55,198.17)(-1.547,-2.000){2}{\rule{0.350pt}{0.400pt}}
\put(911,195.67){\rule{0.723pt}{0.400pt}}
\multiput(912.50,196.17)(-1.500,-1.000){2}{\rule{0.361pt}{0.400pt}}
\put(908,194.67){\rule{0.723pt}{0.400pt}}
\multiput(909.50,195.17)(-1.500,-1.000){2}{\rule{0.361pt}{0.400pt}}
\put(905,193.67){\rule{0.723pt}{0.400pt}}
\multiput(906.50,194.17)(-1.500,-1.000){2}{\rule{0.361pt}{0.400pt}}
\put(902,192.67){\rule{0.723pt}{0.400pt}}
\multiput(903.50,193.17)(-1.500,-1.000){2}{\rule{0.361pt}{0.400pt}}
\put(899,191.67){\rule{0.723pt}{0.400pt}}
\multiput(900.50,192.17)(-1.500,-1.000){2}{\rule{0.361pt}{0.400pt}}
\put(896,190.17){\rule{0.700pt}{0.400pt}}
\multiput(897.55,191.17)(-1.547,-2.000){2}{\rule{0.350pt}{0.400pt}}
\put(893,188.67){\rule{0.723pt}{0.400pt}}
\multiput(894.50,189.17)(-1.500,-1.000){2}{\rule{0.361pt}{0.400pt}}
\put(890,187.67){\rule{0.723pt}{0.400pt}}
\multiput(891.50,188.17)(-1.500,-1.000){2}{\rule{0.361pt}{0.400pt}}
\put(887,186.67){\rule{0.723pt}{0.400pt}}
\multiput(888.50,187.17)(-1.500,-1.000){2}{\rule{0.361pt}{0.400pt}}
\put(883,185.67){\rule{0.964pt}{0.400pt}}
\multiput(885.00,186.17)(-2.000,-1.000){2}{\rule{0.482pt}{0.400pt}}
\put(880,184.67){\rule{0.723pt}{0.400pt}}
\multiput(881.50,185.17)(-1.500,-1.000){2}{\rule{0.361pt}{0.400pt}}
\put(877,183.17){\rule{0.700pt}{0.400pt}}
\multiput(878.55,184.17)(-1.547,-2.000){2}{\rule{0.350pt}{0.400pt}}
\put(874,181.67){\rule{0.723pt}{0.400pt}}
\multiput(875.50,182.17)(-1.500,-1.000){2}{\rule{0.361pt}{0.400pt}}
\put(871,180.67){\rule{0.723pt}{0.400pt}}
\multiput(872.50,181.17)(-1.500,-1.000){2}{\rule{0.361pt}{0.400pt}}
\put(868,179.67){\rule{0.723pt}{0.400pt}}
\multiput(869.50,180.17)(-1.500,-1.000){2}{\rule{0.361pt}{0.400pt}}
\put(865,178.67){\rule{0.723pt}{0.400pt}}
\multiput(866.50,179.17)(-1.500,-1.000){2}{\rule{0.361pt}{0.400pt}}
\put(862,177.67){\rule{0.723pt}{0.400pt}}
\multiput(863.50,178.17)(-1.500,-1.000){2}{\rule{0.361pt}{0.400pt}}
\put(859,176.17){\rule{0.700pt}{0.400pt}}
\multiput(860.55,177.17)(-1.547,-2.000){2}{\rule{0.350pt}{0.400pt}}
\put(856,174.67){\rule{0.723pt}{0.400pt}}
\multiput(857.50,175.17)(-1.500,-1.000){2}{\rule{0.361pt}{0.400pt}}
\put(853,173.67){\rule{0.723pt}{0.400pt}}
\multiput(854.50,174.17)(-1.500,-1.000){2}{\rule{0.361pt}{0.400pt}}
\put(850,172.67){\rule{0.723pt}{0.400pt}}
\multiput(851.50,173.17)(-1.500,-1.000){2}{\rule{0.361pt}{0.400pt}}
\put(847,171.67){\rule{0.723pt}{0.400pt}}
\multiput(848.50,172.17)(-1.500,-1.000){2}{\rule{0.361pt}{0.400pt}}
\put(844,170.67){\rule{0.723pt}{0.400pt}}
\multiput(845.50,171.17)(-1.500,-1.000){2}{\rule{0.361pt}{0.400pt}}
\put(841,169.17){\rule{0.700pt}{0.400pt}}
\multiput(842.55,170.17)(-1.547,-2.000){2}{\rule{0.350pt}{0.400pt}}
\put(838,167.67){\rule{0.723pt}{0.400pt}}
\multiput(839.50,168.17)(-1.500,-1.000){2}{\rule{0.361pt}{0.400pt}}
\put(835,166.67){\rule{0.723pt}{0.400pt}}
\multiput(836.50,167.17)(-1.500,-1.000){2}{\rule{0.361pt}{0.400pt}}
\put(832,165.67){\rule{0.723pt}{0.400pt}}
\multiput(833.50,166.17)(-1.500,-1.000){2}{\rule{0.361pt}{0.400pt}}
\put(829,164.67){\rule{0.723pt}{0.400pt}}
\multiput(830.50,165.17)(-1.500,-1.000){2}{\rule{0.361pt}{0.400pt}}
\put(825,163.67){\rule{0.964pt}{0.400pt}}
\multiput(827.00,164.17)(-2.000,-1.000){2}{\rule{0.482pt}{0.400pt}}
\put(822,162.17){\rule{0.700pt}{0.400pt}}
\multiput(823.55,163.17)(-1.547,-2.000){2}{\rule{0.350pt}{0.400pt}}
\put(819,160.67){\rule{0.723pt}{0.400pt}}
\multiput(820.50,161.17)(-1.500,-1.000){2}{\rule{0.361pt}{0.400pt}}
\put(816,159.67){\rule{0.723pt}{0.400pt}}
\multiput(817.50,160.17)(-1.500,-1.000){2}{\rule{0.361pt}{0.400pt}}
\put(813,158.67){\rule{0.723pt}{0.400pt}}
\multiput(814.50,159.17)(-1.500,-1.000){2}{\rule{0.361pt}{0.400pt}}
\put(810,157.67){\rule{0.723pt}{0.400pt}}
\multiput(811.50,158.17)(-1.500,-1.000){2}{\rule{0.361pt}{0.400pt}}
\put(807,156.67){\rule{0.723pt}{0.400pt}}
\multiput(808.50,157.17)(-1.500,-1.000){2}{\rule{0.361pt}{0.400pt}}
\put(804,155.17){\rule{0.700pt}{0.400pt}}
\multiput(805.55,156.17)(-1.547,-2.000){2}{\rule{0.350pt}{0.400pt}}
\put(801,153.67){\rule{0.723pt}{0.400pt}}
\multiput(802.50,154.17)(-1.500,-1.000){2}{\rule{0.361pt}{0.400pt}}
\put(798,152.67){\rule{0.723pt}{0.400pt}}
\multiput(799.50,153.17)(-1.500,-1.000){2}{\rule{0.361pt}{0.400pt}}
\put(795,151.67){\rule{0.723pt}{0.400pt}}
\multiput(796.50,152.17)(-1.500,-1.000){2}{\rule{0.361pt}{0.400pt}}
\put(792,150.67){\rule{0.723pt}{0.400pt}}
\multiput(793.50,151.17)(-1.500,-1.000){2}{\rule{0.361pt}{0.400pt}}
\put(789,149.67){\rule{0.723pt}{0.400pt}}
\multiput(790.50,150.17)(-1.500,-1.000){2}{\rule{0.361pt}{0.400pt}}
\put(786,148.17){\rule{0.700pt}{0.400pt}}
\multiput(787.55,149.17)(-1.547,-2.000){2}{\rule{0.350pt}{0.400pt}}
\put(783,146.67){\rule{0.723pt}{0.400pt}}
\multiput(784.50,147.17)(-1.500,-1.000){2}{\rule{0.361pt}{0.400pt}}
\put(780,145.67){\rule{0.723pt}{0.400pt}}
\multiput(781.50,146.17)(-1.500,-1.000){2}{\rule{0.361pt}{0.400pt}}
\put(777,144.67){\rule{0.723pt}{0.400pt}}
\multiput(778.50,145.17)(-1.500,-1.000){2}{\rule{0.361pt}{0.400pt}}
\put(774,143.67){\rule{0.723pt}{0.400pt}}
\multiput(775.50,144.17)(-1.500,-1.000){2}{\rule{0.361pt}{0.400pt}}
\put(771,142.67){\rule{0.723pt}{0.400pt}}
\multiput(772.50,143.17)(-1.500,-1.000){2}{\rule{0.361pt}{0.400pt}}
\put(767,141.17){\rule{0.900pt}{0.400pt}}
\multiput(769.13,142.17)(-2.132,-2.000){2}{\rule{0.450pt}{0.400pt}}
\put(764,139.67){\rule{0.723pt}{0.400pt}}
\multiput(765.50,140.17)(-1.500,-1.000){2}{\rule{0.361pt}{0.400pt}}
\put(761,138.67){\rule{0.723pt}{0.400pt}}
\multiput(762.50,139.17)(-1.500,-1.000){2}{\rule{0.361pt}{0.400pt}}
\put(758,137.67){\rule{0.723pt}{0.400pt}}
\multiput(759.50,138.17)(-1.500,-1.000){2}{\rule{0.361pt}{0.400pt}}
\put(755,136.67){\rule{0.723pt}{0.400pt}}
\multiput(756.50,137.17)(-1.500,-1.000){2}{\rule{0.361pt}{0.400pt}}
\put(752,135.67){\rule{0.723pt}{0.400pt}}
\multiput(753.50,136.17)(-1.500,-1.000){2}{\rule{0.361pt}{0.400pt}}
\put(749,134.17){\rule{0.700pt}{0.400pt}}
\multiput(750.55,135.17)(-1.547,-2.000){2}{\rule{0.350pt}{0.400pt}}
\put(746,132.67){\rule{0.723pt}{0.400pt}}
\multiput(747.50,133.17)(-1.500,-1.000){2}{\rule{0.361pt}{0.400pt}}
\put(743,131.67){\rule{0.723pt}{0.400pt}}
\multiput(744.50,132.17)(-1.500,-1.000){2}{\rule{0.361pt}{0.400pt}}
\put(740,130.67){\rule{0.723pt}{0.400pt}}
\multiput(741.50,131.17)(-1.500,-1.000){2}{\rule{0.361pt}{0.400pt}}
\put(737,129.67){\rule{0.723pt}{0.400pt}}
\multiput(738.50,130.17)(-1.500,-1.000){2}{\rule{0.361pt}{0.400pt}}
\put(734,128.67){\rule{0.723pt}{0.400pt}}
\multiput(735.50,129.17)(-1.500,-1.000){2}{\rule{0.361pt}{0.400pt}}
\put(731,127.17){\rule{0.700pt}{0.400pt}}
\multiput(732.55,128.17)(-1.547,-2.000){2}{\rule{0.350pt}{0.400pt}}
\put(728,125.67){\rule{0.723pt}{0.400pt}}
\multiput(729.50,126.17)(-1.500,-1.000){2}{\rule{0.361pt}{0.400pt}}
\put(725,124.67){\rule{0.723pt}{0.400pt}}
\multiput(726.50,125.17)(-1.500,-1.000){2}{\rule{0.361pt}{0.400pt}}
\put(722,123.67){\rule{0.723pt}{0.400pt}}
\multiput(723.50,124.17)(-1.500,-1.000){2}{\rule{0.361pt}{0.400pt}}
\put(719,122.67){\rule{0.723pt}{0.400pt}}
\multiput(720.50,123.17)(-1.500,-1.000){2}{\rule{0.361pt}{0.400pt}}
\put(716,121.67){\rule{0.723pt}{0.400pt}}
\multiput(717.50,122.17)(-1.500,-1.000){2}{\rule{0.361pt}{0.400pt}}
\put(713,120.17){\rule{0.700pt}{0.400pt}}
\multiput(714.55,121.17)(-1.547,-2.000){2}{\rule{0.350pt}{0.400pt}}
\put(709,118.67){\rule{0.964pt}{0.400pt}}
\multiput(711.00,119.17)(-2.000,-1.000){2}{\rule{0.482pt}{0.400pt}}
\put(706,117.67){\rule{0.723pt}{0.400pt}}
\multiput(707.50,118.17)(-1.500,-1.000){2}{\rule{0.361pt}{0.400pt}}
\put(703,116.67){\rule{0.723pt}{0.400pt}}
\multiput(704.50,117.17)(-1.500,-1.000){2}{\rule{0.361pt}{0.400pt}}
\put(700,115.67){\rule{0.723pt}{0.400pt}}
\multiput(701.50,116.17)(-1.500,-1.000){2}{\rule{0.361pt}{0.400pt}}
\put(697,114.67){\rule{0.723pt}{0.400pt}}
\multiput(698.50,115.17)(-1.500,-1.000){2}{\rule{0.361pt}{0.400pt}}
\put(694,113.67){\rule{0.723pt}{0.400pt}}
\multiput(695.50,114.17)(-1.500,-1.000){2}{\rule{0.361pt}{0.400pt}}
\put(691,112.17){\rule{0.700pt}{0.400pt}}
\multiput(692.55,113.17)(-1.547,-2.000){2}{\rule{0.350pt}{0.400pt}}
\put(688,110.67){\rule{0.723pt}{0.400pt}}
\multiput(689.50,111.17)(-1.500,-1.000){2}{\rule{0.361pt}{0.400pt}}
\put(685,109.67){\rule{0.723pt}{0.400pt}}
\multiput(686.50,110.17)(-1.500,-1.000){2}{\rule{0.361pt}{0.400pt}}
\put(682,108.67){\rule{0.723pt}{0.400pt}}
\multiput(683.50,109.17)(-1.500,-1.000){2}{\rule{0.361pt}{0.400pt}}
\put(679,107.67){\rule{0.723pt}{0.400pt}}
\multiput(680.50,108.17)(-1.500,-1.000){2}{\rule{0.361pt}{0.400pt}}
\put(676,106.67){\rule{0.723pt}{0.400pt}}
\multiput(677.50,107.17)(-1.500,-1.000){2}{\rule{0.361pt}{0.400pt}}
\put(673,105.17){\rule{0.700pt}{0.400pt}}
\multiput(674.55,106.17)(-1.547,-2.000){2}{\rule{0.350pt}{0.400pt}}
\put(670,103.67){\rule{0.723pt}{0.400pt}}
\multiput(671.50,104.17)(-1.500,-1.000){2}{\rule{0.361pt}{0.400pt}}
\put(667,103.67){\rule{0.723pt}{0.400pt}}
\multiput(668.50,103.17)(-1.500,1.000){2}{\rule{0.361pt}{0.400pt}}
\put(664,105.17){\rule{0.700pt}{0.400pt}}
\multiput(665.55,104.17)(-1.547,2.000){2}{\rule{0.350pt}{0.400pt}}
\put(661,106.67){\rule{0.723pt}{0.400pt}}
\multiput(662.50,106.17)(-1.500,1.000){2}{\rule{0.361pt}{0.400pt}}
\put(658,107.67){\rule{0.723pt}{0.400pt}}
\multiput(659.50,107.17)(-1.500,1.000){2}{\rule{0.361pt}{0.400pt}}
\put(655,108.67){\rule{0.723pt}{0.400pt}}
\multiput(656.50,108.17)(-1.500,1.000){2}{\rule{0.361pt}{0.400pt}}
\put(651,109.67){\rule{0.964pt}{0.400pt}}
\multiput(653.00,109.17)(-2.000,1.000){2}{\rule{0.482pt}{0.400pt}}
\put(648,110.67){\rule{0.723pt}{0.400pt}}
\multiput(649.50,110.17)(-1.500,1.000){2}{\rule{0.361pt}{0.400pt}}
\put(645,112.17){\rule{0.700pt}{0.400pt}}
\multiput(646.55,111.17)(-1.547,2.000){2}{\rule{0.350pt}{0.400pt}}
\put(642,113.67){\rule{0.723pt}{0.400pt}}
\multiput(643.50,113.17)(-1.500,1.000){2}{\rule{0.361pt}{0.400pt}}
\put(639,114.67){\rule{0.723pt}{0.400pt}}
\multiput(640.50,114.17)(-1.500,1.000){2}{\rule{0.361pt}{0.400pt}}
\put(636,115.67){\rule{0.723pt}{0.400pt}}
\multiput(637.50,115.17)(-1.500,1.000){2}{\rule{0.361pt}{0.400pt}}
\put(633,116.67){\rule{0.723pt}{0.400pt}}
\multiput(634.50,116.17)(-1.500,1.000){2}{\rule{0.361pt}{0.400pt}}
\put(630,117.67){\rule{0.723pt}{0.400pt}}
\multiput(631.50,117.17)(-1.500,1.000){2}{\rule{0.361pt}{0.400pt}}
\put(627,118.67){\rule{0.723pt}{0.400pt}}
\multiput(628.50,118.17)(-1.500,1.000){2}{\rule{0.361pt}{0.400pt}}
\put(624,120.17){\rule{0.700pt}{0.400pt}}
\multiput(625.55,119.17)(-1.547,2.000){2}{\rule{0.350pt}{0.400pt}}
\put(621,121.67){\rule{0.723pt}{0.400pt}}
\multiput(622.50,121.17)(-1.500,1.000){2}{\rule{0.361pt}{0.400pt}}
\put(618,122.67){\rule{0.723pt}{0.400pt}}
\multiput(619.50,122.17)(-1.500,1.000){2}{\rule{0.361pt}{0.400pt}}
\put(615,123.67){\rule{0.723pt}{0.400pt}}
\multiput(616.50,123.17)(-1.500,1.000){2}{\rule{0.361pt}{0.400pt}}
\put(612,124.67){\rule{0.723pt}{0.400pt}}
\multiput(613.50,124.17)(-1.500,1.000){2}{\rule{0.361pt}{0.400pt}}
\put(609,125.67){\rule{0.723pt}{0.400pt}}
\multiput(610.50,125.17)(-1.500,1.000){2}{\rule{0.361pt}{0.400pt}}
\put(606,127.17){\rule{0.700pt}{0.400pt}}
\multiput(607.55,126.17)(-1.547,2.000){2}{\rule{0.350pt}{0.400pt}}
\put(603,128.67){\rule{0.723pt}{0.400pt}}
\multiput(604.50,128.17)(-1.500,1.000){2}{\rule{0.361pt}{0.400pt}}
\put(600,129.67){\rule{0.723pt}{0.400pt}}
\multiput(601.50,129.17)(-1.500,1.000){2}{\rule{0.361pt}{0.400pt}}
\put(597,130.67){\rule{0.723pt}{0.400pt}}
\multiput(598.50,130.17)(-1.500,1.000){2}{\rule{0.361pt}{0.400pt}}
\put(593,131.67){\rule{0.964pt}{0.400pt}}
\multiput(595.00,131.17)(-2.000,1.000){2}{\rule{0.482pt}{0.400pt}}
\put(590,132.67){\rule{0.723pt}{0.400pt}}
\multiput(591.50,132.17)(-1.500,1.000){2}{\rule{0.361pt}{0.400pt}}
\put(587,134.17){\rule{0.700pt}{0.400pt}}
\multiput(588.55,133.17)(-1.547,2.000){2}{\rule{0.350pt}{0.400pt}}
\put(584,135.67){\rule{0.723pt}{0.400pt}}
\multiput(585.50,135.17)(-1.500,1.000){2}{\rule{0.361pt}{0.400pt}}
\put(581,136.67){\rule{0.723pt}{0.400pt}}
\multiput(582.50,136.17)(-1.500,1.000){2}{\rule{0.361pt}{0.400pt}}
\put(578,137.67){\rule{0.723pt}{0.400pt}}
\multiput(579.50,137.17)(-1.500,1.000){2}{\rule{0.361pt}{0.400pt}}
\put(575,138.67){\rule{0.723pt}{0.400pt}}
\multiput(576.50,138.17)(-1.500,1.000){2}{\rule{0.361pt}{0.400pt}}
\put(572,139.67){\rule{0.723pt}{0.400pt}}
\multiput(573.50,139.17)(-1.500,1.000){2}{\rule{0.361pt}{0.400pt}}
\put(569,141.17){\rule{0.700pt}{0.400pt}}
\multiput(570.55,140.17)(-1.547,2.000){2}{\rule{0.350pt}{0.400pt}}
\put(566,142.67){\rule{0.723pt}{0.400pt}}
\multiput(567.50,142.17)(-1.500,1.000){2}{\rule{0.361pt}{0.400pt}}
\put(563,143.67){\rule{0.723pt}{0.400pt}}
\multiput(564.50,143.17)(-1.500,1.000){2}{\rule{0.361pt}{0.400pt}}
\put(560,144.67){\rule{0.723pt}{0.400pt}}
\multiput(561.50,144.17)(-1.500,1.000){2}{\rule{0.361pt}{0.400pt}}
\put(557,145.67){\rule{0.723pt}{0.400pt}}
\multiput(558.50,145.17)(-1.500,1.000){2}{\rule{0.361pt}{0.400pt}}
\put(554,146.67){\rule{0.723pt}{0.400pt}}
\multiput(555.50,146.17)(-1.500,1.000){2}{\rule{0.361pt}{0.400pt}}
\put(551,148.17){\rule{0.700pt}{0.400pt}}
\multiput(552.55,147.17)(-1.547,2.000){2}{\rule{0.350pt}{0.400pt}}
\put(548,149.67){\rule{0.723pt}{0.400pt}}
\multiput(549.50,149.17)(-1.500,1.000){2}{\rule{0.361pt}{0.400pt}}
\put(545,150.67){\rule{0.723pt}{0.400pt}}
\multiput(546.50,150.17)(-1.500,1.000){2}{\rule{0.361pt}{0.400pt}}
\put(542,151.67){\rule{0.723pt}{0.400pt}}
\multiput(543.50,151.17)(-1.500,1.000){2}{\rule{0.361pt}{0.400pt}}
\put(539,152.67){\rule{0.723pt}{0.400pt}}
\multiput(540.50,152.17)(-1.500,1.000){2}{\rule{0.361pt}{0.400pt}}
\put(536,153.67){\rule{0.723pt}{0.400pt}}
\multiput(537.50,153.17)(-1.500,1.000){2}{\rule{0.361pt}{0.400pt}}
\put(532,155.17){\rule{0.900pt}{0.400pt}}
\multiput(534.13,154.17)(-2.132,2.000){2}{\rule{0.450pt}{0.400pt}}
\put(529,156.67){\rule{0.723pt}{0.400pt}}
\multiput(530.50,156.17)(-1.500,1.000){2}{\rule{0.361pt}{0.400pt}}
\put(526,157.67){\rule{0.723pt}{0.400pt}}
\multiput(527.50,157.17)(-1.500,1.000){2}{\rule{0.361pt}{0.400pt}}
\put(523,158.67){\rule{0.723pt}{0.400pt}}
\multiput(524.50,158.17)(-1.500,1.000){2}{\rule{0.361pt}{0.400pt}}
\put(520,159.67){\rule{0.723pt}{0.400pt}}
\multiput(521.50,159.17)(-1.500,1.000){2}{\rule{0.361pt}{0.400pt}}
\put(517,160.67){\rule{0.723pt}{0.400pt}}
\multiput(518.50,160.17)(-1.500,1.000){2}{\rule{0.361pt}{0.400pt}}
\put(514,162.17){\rule{0.700pt}{0.400pt}}
\multiput(515.55,161.17)(-1.547,2.000){2}{\rule{0.350pt}{0.400pt}}
\put(511,163.67){\rule{0.723pt}{0.400pt}}
\multiput(512.50,163.17)(-1.500,1.000){2}{\rule{0.361pt}{0.400pt}}
\put(508,164.67){\rule{0.723pt}{0.400pt}}
\multiput(509.50,164.17)(-1.500,1.000){2}{\rule{0.361pt}{0.400pt}}
\put(505,165.67){\rule{0.723pt}{0.400pt}}
\multiput(506.50,165.17)(-1.500,1.000){2}{\rule{0.361pt}{0.400pt}}
\put(502,166.67){\rule{0.723pt}{0.400pt}}
\multiput(503.50,166.17)(-1.500,1.000){2}{\rule{0.361pt}{0.400pt}}
\put(499,167.67){\rule{0.723pt}{0.400pt}}
\multiput(500.50,167.17)(-1.500,1.000){2}{\rule{0.361pt}{0.400pt}}
\put(496,169.17){\rule{0.700pt}{0.400pt}}
\multiput(497.55,168.17)(-1.547,2.000){2}{\rule{0.350pt}{0.400pt}}
\put(493,170.67){\rule{0.723pt}{0.400pt}}
\multiput(494.50,170.17)(-1.500,1.000){2}{\rule{0.361pt}{0.400pt}}
\put(490,171.67){\rule{0.723pt}{0.400pt}}
\multiput(491.50,171.17)(-1.500,1.000){2}{\rule{0.361pt}{0.400pt}}
\put(487,172.67){\rule{0.723pt}{0.400pt}}
\multiput(488.50,172.17)(-1.500,1.000){2}{\rule{0.361pt}{0.400pt}}
\put(484,173.67){\rule{0.723pt}{0.400pt}}
\multiput(485.50,173.17)(-1.500,1.000){2}{\rule{0.361pt}{0.400pt}}
\put(481,174.67){\rule{0.723pt}{0.400pt}}
\multiput(482.50,174.17)(-1.500,1.000){2}{\rule{0.361pt}{0.400pt}}
\put(478,176.17){\rule{0.700pt}{0.400pt}}
\multiput(479.55,175.17)(-1.547,2.000){2}{\rule{0.350pt}{0.400pt}}
\put(474,177.67){\rule{0.964pt}{0.400pt}}
\multiput(476.00,177.17)(-2.000,1.000){2}{\rule{0.482pt}{0.400pt}}
\put(471,178.67){\rule{0.723pt}{0.400pt}}
\multiput(472.50,178.17)(-1.500,1.000){2}{\rule{0.361pt}{0.400pt}}
\put(468,179.67){\rule{0.723pt}{0.400pt}}
\multiput(469.50,179.17)(-1.500,1.000){2}{\rule{0.361pt}{0.400pt}}
\put(465,180.67){\rule{0.723pt}{0.400pt}}
\multiput(466.50,180.17)(-1.500,1.000){2}{\rule{0.361pt}{0.400pt}}
\put(462,181.67){\rule{0.723pt}{0.400pt}}
\multiput(463.50,181.17)(-1.500,1.000){2}{\rule{0.361pt}{0.400pt}}
\put(459,183.17){\rule{0.700pt}{0.400pt}}
\multiput(460.55,182.17)(-1.547,2.000){2}{\rule{0.350pt}{0.400pt}}
\put(456,184.67){\rule{0.723pt}{0.400pt}}
\multiput(457.50,184.17)(-1.500,1.000){2}{\rule{0.361pt}{0.400pt}}
\put(453,185.67){\rule{0.723pt}{0.400pt}}
\multiput(454.50,185.17)(-1.500,1.000){2}{\rule{0.361pt}{0.400pt}}
\put(450,186.67){\rule{0.723pt}{0.400pt}}
\multiput(451.50,186.17)(-1.500,1.000){2}{\rule{0.361pt}{0.400pt}}
\put(447,187.67){\rule{0.723pt}{0.400pt}}
\multiput(448.50,187.17)(-1.500,1.000){2}{\rule{0.361pt}{0.400pt}}
\put(444,188.67){\rule{0.723pt}{0.400pt}}
\multiput(445.50,188.17)(-1.500,1.000){2}{\rule{0.361pt}{0.400pt}}
\put(441,190.17){\rule{0.700pt}{0.400pt}}
\multiput(442.55,189.17)(-1.547,2.000){2}{\rule{0.350pt}{0.400pt}}
\put(438,191.67){\rule{0.723pt}{0.400pt}}
\multiput(439.50,191.17)(-1.500,1.000){2}{\rule{0.361pt}{0.400pt}}
\put(435,192.67){\rule{0.723pt}{0.400pt}}
\multiput(436.50,192.17)(-1.500,1.000){2}{\rule{0.361pt}{0.400pt}}
\put(432,193.67){\rule{0.723pt}{0.400pt}}
\multiput(433.50,193.17)(-1.500,1.000){2}{\rule{0.361pt}{0.400pt}}
\put(429,194.67){\rule{0.723pt}{0.400pt}}
\multiput(430.50,194.17)(-1.500,1.000){2}{\rule{0.361pt}{0.400pt}}
\put(426,195.67){\rule{0.723pt}{0.400pt}}
\multiput(427.50,195.17)(-1.500,1.000){2}{\rule{0.361pt}{0.400pt}}
\put(423,197.17){\rule{0.700pt}{0.400pt}}
\multiput(424.55,196.17)(-1.547,2.000){2}{\rule{0.350pt}{0.400pt}}
\put(420,198.67){\rule{0.723pt}{0.400pt}}
\multiput(421.50,198.17)(-1.500,1.000){2}{\rule{0.361pt}{0.400pt}}
\put(416,199.67){\rule{0.964pt}{0.400pt}}
\multiput(418.00,199.17)(-2.000,1.000){2}{\rule{0.482pt}{0.400pt}}
\put(413,200.67){\rule{0.723pt}{0.400pt}}
\multiput(414.50,200.17)(-1.500,1.000){2}{\rule{0.361pt}{0.400pt}}
\put(410,201.67){\rule{0.723pt}{0.400pt}}
\multiput(411.50,201.17)(-1.500,1.000){2}{\rule{0.361pt}{0.400pt}}
\put(407,202.67){\rule{0.723pt}{0.400pt}}
\multiput(408.50,202.17)(-1.500,1.000){2}{\rule{0.361pt}{0.400pt}}
\put(404,204.17){\rule{0.700pt}{0.400pt}}
\multiput(405.55,203.17)(-1.547,2.000){2}{\rule{0.350pt}{0.400pt}}
\put(401,205.67){\rule{0.723pt}{0.400pt}}
\multiput(402.50,205.17)(-1.500,1.000){2}{\rule{0.361pt}{0.400pt}}
\put(398,206.67){\rule{0.723pt}{0.400pt}}
\multiput(399.50,206.17)(-1.500,1.000){2}{\rule{0.361pt}{0.400pt}}
\put(395,207.67){\rule{0.723pt}{0.400pt}}
\multiput(396.50,207.17)(-1.500,1.000){2}{\rule{0.361pt}{0.400pt}}
\put(392,208.67){\rule{0.723pt}{0.400pt}}
\multiput(393.50,208.17)(-1.500,1.000){2}{\rule{0.361pt}{0.400pt}}
\put(389,209.67){\rule{0.723pt}{0.400pt}}
\multiput(390.50,209.17)(-1.500,1.000){2}{\rule{0.361pt}{0.400pt}}
\put(386,211.17){\rule{0.700pt}{0.400pt}}
\multiput(387.55,210.17)(-1.547,2.000){2}{\rule{0.350pt}{0.400pt}}
\put(383,212.67){\rule{0.723pt}{0.400pt}}
\multiput(384.50,212.17)(-1.500,1.000){2}{\rule{0.361pt}{0.400pt}}
\put(380,213.67){\rule{0.723pt}{0.400pt}}
\multiput(381.50,213.17)(-1.500,1.000){2}{\rule{0.361pt}{0.400pt}}
\put(377,214.67){\rule{0.723pt}{0.400pt}}
\multiput(378.50,214.17)(-1.500,1.000){2}{\rule{0.361pt}{0.400pt}}
\put(374,215.67){\rule{0.723pt}{0.400pt}}
\multiput(375.50,215.17)(-1.500,1.000){2}{\rule{0.361pt}{0.400pt}}
\put(371,216.67){\rule{0.723pt}{0.400pt}}
\multiput(372.50,216.17)(-1.500,1.000){2}{\rule{0.361pt}{0.400pt}}
\put(368,218.17){\rule{0.700pt}{0.400pt}}
\multiput(369.55,217.17)(-1.547,2.000){2}{\rule{0.350pt}{0.400pt}}
\put(365,219.67){\rule{0.723pt}{0.400pt}}
\multiput(366.50,219.17)(-1.500,1.000){2}{\rule{0.361pt}{0.400pt}}
\put(362,220.67){\rule{0.723pt}{0.400pt}}
\multiput(363.50,220.17)(-1.500,1.000){2}{\rule{0.361pt}{0.400pt}}
\put(358,221.67){\rule{0.964pt}{0.400pt}}
\multiput(360.00,221.17)(-2.000,1.000){2}{\rule{0.482pt}{0.400pt}}
\put(355,222.67){\rule{0.723pt}{0.400pt}}
\multiput(356.50,222.17)(-1.500,1.000){2}{\rule{0.361pt}{0.400pt}}
\put(352,223.67){\rule{0.723pt}{0.400pt}}
\multiput(353.50,223.17)(-1.500,1.000){2}{\rule{0.361pt}{0.400pt}}
\put(349,225.17){\rule{0.700pt}{0.400pt}}
\multiput(350.55,224.17)(-1.547,2.000){2}{\rule{0.350pt}{0.400pt}}
\put(346,226.67){\rule{0.723pt}{0.400pt}}
\multiput(347.50,226.17)(-1.500,1.000){2}{\rule{0.361pt}{0.400pt}}
\put(343,227.67){\rule{0.723pt}{0.400pt}}
\multiput(344.50,227.17)(-1.500,1.000){2}{\rule{0.361pt}{0.400pt}}
\put(340,228.67){\rule{0.723pt}{0.400pt}}
\multiput(341.50,228.17)(-1.500,1.000){2}{\rule{0.361pt}{0.400pt}}
\put(337,229.67){\rule{0.723pt}{0.400pt}}
\multiput(338.50,229.17)(-1.500,1.000){2}{\rule{0.361pt}{0.400pt}}
\put(334,230.67){\rule{0.723pt}{0.400pt}}
\multiput(335.50,230.17)(-1.500,1.000){2}{\rule{0.361pt}{0.400pt}}
\put(331,232.17){\rule{0.700pt}{0.400pt}}
\multiput(332.55,231.17)(-1.547,2.000){2}{\rule{0.350pt}{0.400pt}}
\put(328,233.67){\rule{0.723pt}{0.400pt}}
\multiput(329.50,233.17)(-1.500,1.000){2}{\rule{0.361pt}{0.400pt}}
\put(325,234.67){\rule{0.723pt}{0.400pt}}
\multiput(326.50,234.17)(-1.500,1.000){2}{\rule{0.361pt}{0.400pt}}
\put(322,235.67){\rule{0.723pt}{0.400pt}}
\multiput(323.50,235.17)(-1.500,1.000){2}{\rule{0.361pt}{0.400pt}}
\put(319,236.67){\rule{0.723pt}{0.400pt}}
\multiput(320.50,236.17)(-1.500,1.000){2}{\rule{0.361pt}{0.400pt}}
\put(316,237.67){\rule{0.723pt}{0.400pt}}
\multiput(317.50,237.17)(-1.500,1.000){2}{\rule{0.361pt}{0.400pt}}
\put(313,239.17){\rule{0.700pt}{0.400pt}}
\multiput(314.55,238.17)(-1.547,2.000){2}{\rule{0.350pt}{0.400pt}}
\put(310,240.67){\rule{0.723pt}{0.400pt}}
\multiput(311.50,240.17)(-1.500,1.000){2}{\rule{0.361pt}{0.400pt}}
\put(307,241.67){\rule{0.723pt}{0.400pt}}
\multiput(308.50,241.17)(-1.500,1.000){2}{\rule{0.361pt}{0.400pt}}
\put(304,242.67){\rule{0.723pt}{0.400pt}}
\multiput(305.50,242.17)(-1.500,1.000){2}{\rule{0.361pt}{0.400pt}}
\put(300,243.67){\rule{0.964pt}{0.400pt}}
\multiput(302.00,243.17)(-2.000,1.000){2}{\rule{0.482pt}{0.400pt}}
\put(297,244.67){\rule{0.723pt}{0.400pt}}
\multiput(298.50,244.17)(-1.500,1.000){2}{\rule{0.361pt}{0.400pt}}
\put(294,246.17){\rule{0.700pt}{0.400pt}}
\multiput(295.55,245.17)(-1.547,2.000){2}{\rule{0.350pt}{0.400pt}}
\put(291,247.67){\rule{0.723pt}{0.400pt}}
\multiput(292.50,247.17)(-1.500,1.000){2}{\rule{0.361pt}{0.400pt}}
\put(288,248.67){\rule{0.723pt}{0.400pt}}
\multiput(289.50,248.17)(-1.500,1.000){2}{\rule{0.361pt}{0.400pt}}
\put(285,249.67){\rule{0.723pt}{0.400pt}}
\multiput(286.50,249.17)(-1.500,1.000){2}{\rule{0.361pt}{0.400pt}}
\put(282,250.67){\rule{0.723pt}{0.400pt}}
\multiput(283.50,250.17)(-1.500,1.000){2}{\rule{0.361pt}{0.400pt}}
\put(279,251.67){\rule{0.723pt}{0.400pt}}
\multiput(280.50,251.17)(-1.500,1.000){2}{\rule{0.361pt}{0.400pt}}
\put(276,253.17){\rule{0.700pt}{0.400pt}}
\multiput(277.55,252.17)(-1.547,2.000){2}{\rule{0.350pt}{0.400pt}}
\put(273,254.67){\rule{0.723pt}{0.400pt}}
\multiput(274.50,254.17)(-1.500,1.000){2}{\rule{0.361pt}{0.400pt}}
\put(270,255.67){\rule{0.723pt}{0.400pt}}
\multiput(271.50,255.17)(-1.500,1.000){2}{\rule{0.361pt}{0.400pt}}
\put(267,256.67){\rule{0.723pt}{0.400pt}}
\multiput(268.50,256.17)(-1.500,1.000){2}{\rule{0.361pt}{0.400pt}}
\put(264,257.67){\rule{0.723pt}{0.400pt}}
\multiput(265.50,257.17)(-1.500,1.000){2}{\rule{0.361pt}{0.400pt}}
\put(261,258.67){\rule{0.723pt}{0.400pt}}
\multiput(262.50,258.17)(-1.500,1.000){2}{\rule{0.361pt}{0.400pt}}
\put(258,260.17){\rule{0.700pt}{0.400pt}}
\multiput(259.55,259.17)(-1.547,2.000){2}{\rule{0.350pt}{0.400pt}}
\put(255,261.67){\rule{0.723pt}{0.400pt}}
\multiput(256.50,261.17)(-1.500,1.000){2}{\rule{0.361pt}{0.400pt}}
\put(252,262.67){\rule{0.723pt}{0.400pt}}
\multiput(253.50,262.17)(-1.500,1.000){2}{\rule{0.361pt}{0.400pt}}
\put(249,263.67){\rule{0.723pt}{0.400pt}}
\multiput(250.50,263.17)(-1.500,1.000){2}{\rule{0.361pt}{0.400pt}}
\put(246,264.67){\rule{0.723pt}{0.400pt}}
\multiput(247.50,264.17)(-1.500,1.000){2}{\rule{0.361pt}{0.400pt}}
\put(242,265.67){\rule{0.964pt}{0.400pt}}
\multiput(244.00,265.17)(-2.000,1.000){2}{\rule{0.482pt}{0.400pt}}
\put(239,267.17){\rule{0.700pt}{0.400pt}}
\multiput(240.55,266.17)(-1.547,2.000){2}{\rule{0.350pt}{0.400pt}}
\put(236,268.67){\rule{0.723pt}{0.400pt}}
\multiput(237.50,268.17)(-1.500,1.000){2}{\rule{0.361pt}{0.400pt}}
\put(233,269.67){\rule{0.723pt}{0.400pt}}
\multiput(234.50,269.17)(-1.500,1.000){2}{\rule{0.361pt}{0.400pt}}
\put(230,270.67){\rule{0.723pt}{0.400pt}}
\multiput(231.50,270.17)(-1.500,1.000){2}{\rule{0.361pt}{0.400pt}}
\put(227,271.67){\rule{0.723pt}{0.400pt}}
\multiput(228.50,271.17)(-1.500,1.000){2}{\rule{0.361pt}{0.400pt}}
\put(224,272.67){\rule{0.723pt}{0.400pt}}
\multiput(225.50,272.17)(-1.500,1.000){2}{\rule{0.361pt}{0.400pt}}
\put(221,274.17){\rule{0.700pt}{0.400pt}}
\multiput(222.55,273.17)(-1.547,2.000){2}{\rule{0.350pt}{0.400pt}}
\put(218,275.67){\rule{0.723pt}{0.400pt}}
\multiput(219.50,275.17)(-1.500,1.000){2}{\rule{0.361pt}{0.400pt}}
\put(215,276.67){\rule{0.723pt}{0.400pt}}
\multiput(216.50,276.17)(-1.500,1.000){2}{\rule{0.361pt}{0.400pt}}
\put(212,277.67){\rule{0.723pt}{0.400pt}}
\multiput(213.50,277.17)(-1.500,1.000){2}{\rule{0.361pt}{0.400pt}}
\put(209,278.67){\rule{0.723pt}{0.400pt}}
\multiput(210.50,278.17)(-1.500,1.000){2}{\rule{0.361pt}{0.400pt}}
\put(206,279.67){\rule{0.723pt}{0.400pt}}
\multiput(207.50,279.17)(-1.500,1.000){2}{\rule{0.361pt}{0.400pt}}
\put(203,281.17){\rule{0.700pt}{0.400pt}}
\multiput(204.55,280.17)(-1.547,2.000){2}{\rule{0.350pt}{0.400pt}}
\put(200,282.67){\rule{0.723pt}{0.400pt}}
\multiput(201.50,282.17)(-1.500,1.000){2}{\rule{0.361pt}{0.400pt}}
\put(197,283.67){\rule{0.723pt}{0.400pt}}
\multiput(198.50,283.17)(-1.500,1.000){2}{\rule{0.361pt}{0.400pt}}
\put(194,284.67){\rule{0.723pt}{0.400pt}}
\multiput(195.50,284.17)(-1.500,1.000){2}{\rule{0.361pt}{0.400pt}}
\put(191,285.67){\rule{0.723pt}{0.400pt}}
\multiput(192.50,285.17)(-1.500,1.000){2}{\rule{0.361pt}{0.400pt}}
\put(188,286.67){\rule{0.723pt}{0.400pt}}
\multiput(189.50,286.17)(-1.500,1.000){2}{\rule{0.361pt}{0.400pt}}
\put(185,288.17){\rule{0.700pt}{0.400pt}}
\multiput(186.55,287.17)(-1.547,2.000){2}{\rule{0.350pt}{0.400pt}}
\put(181,289.67){\rule{0.964pt}{0.400pt}}
\multiput(183.00,289.17)(-2.000,1.000){2}{\rule{0.482pt}{0.400pt}}
\put(178,290.67){\rule{0.723pt}{0.400pt}}
\multiput(179.50,290.17)(-1.500,1.000){2}{\rule{0.361pt}{0.400pt}}
\put(175,291.67){\rule{0.723pt}{0.400pt}}
\multiput(176.50,291.17)(-1.500,1.000){2}{\rule{0.361pt}{0.400pt}}
\put(172,292.67){\rule{0.723pt}{0.400pt}}
\multiput(173.50,292.17)(-1.500,1.000){2}{\rule{0.361pt}{0.400pt}}
\put(169,293.67){\rule{0.723pt}{0.400pt}}
\multiput(170.50,293.17)(-1.500,1.000){2}{\rule{0.361pt}{0.400pt}}
\put(166,295.17){\rule{0.700pt}{0.400pt}}
\multiput(167.55,294.17)(-1.547,2.000){2}{\rule{0.350pt}{0.400pt}}
\put(163,296.67){\rule{0.723pt}{0.400pt}}
\multiput(164.50,296.17)(-1.500,1.000){2}{\rule{0.361pt}{0.400pt}}
\put(160,297.67){\rule{0.723pt}{0.400pt}}
\multiput(161.50,297.17)(-1.500,1.000){2}{\rule{0.361pt}{0.400pt}}
\put(157,298.67){\rule{0.723pt}{0.400pt}}
\multiput(158.50,298.17)(-1.500,1.000){2}{\rule{0.361pt}{0.400pt}}
\put(154,299.67){\rule{0.723pt}{0.400pt}}
\multiput(155.50,299.17)(-1.500,1.000){2}{\rule{0.361pt}{0.400pt}}
\put(154,300.67){\rule{0.723pt}{0.400pt}}
\multiput(154.00,300.17)(1.500,1.000){2}{\rule{0.361pt}{0.400pt}}
\put(157,302.17){\rule{0.700pt}{0.400pt}}
\multiput(157.00,301.17)(1.547,2.000){2}{\rule{0.350pt}{0.400pt}}
\put(160,303.67){\rule{0.723pt}{0.400pt}}
\multiput(160.00,303.17)(1.500,1.000){2}{\rule{0.361pt}{0.400pt}}
\put(163,304.67){\rule{0.723pt}{0.400pt}}
\multiput(163.00,304.17)(1.500,1.000){2}{\rule{0.361pt}{0.400pt}}
\put(166,305.67){\rule{0.723pt}{0.400pt}}
\multiput(166.00,305.17)(1.500,1.000){2}{\rule{0.361pt}{0.400pt}}
\put(169,306.67){\rule{0.723pt}{0.400pt}}
\multiput(169.00,306.17)(1.500,1.000){2}{\rule{0.361pt}{0.400pt}}
\put(172,307.67){\rule{0.723pt}{0.400pt}}
\multiput(172.00,307.17)(1.500,1.000){2}{\rule{0.361pt}{0.400pt}}
\put(175,309.17){\rule{0.700pt}{0.400pt}}
\multiput(175.00,308.17)(1.547,2.000){2}{\rule{0.350pt}{0.400pt}}
\put(178,310.67){\rule{0.723pt}{0.400pt}}
\multiput(178.00,310.17)(1.500,1.000){2}{\rule{0.361pt}{0.400pt}}
\put(181,311.67){\rule{0.964pt}{0.400pt}}
\multiput(181.00,311.17)(2.000,1.000){2}{\rule{0.482pt}{0.400pt}}
\put(185,312.67){\rule{0.723pt}{0.400pt}}
\multiput(185.00,312.17)(1.500,1.000){2}{\rule{0.361pt}{0.400pt}}
\put(188,313.67){\rule{0.723pt}{0.400pt}}
\multiput(188.00,313.17)(1.500,1.000){2}{\rule{0.361pt}{0.400pt}}
\put(191,314.67){\rule{0.723pt}{0.400pt}}
\multiput(191.00,314.17)(1.500,1.000){2}{\rule{0.361pt}{0.400pt}}
\put(194,315.67){\rule{0.723pt}{0.400pt}}
\multiput(194.00,315.17)(1.500,1.000){2}{\rule{0.361pt}{0.400pt}}
\put(197,317.17){\rule{0.700pt}{0.400pt}}
\multiput(197.00,316.17)(1.547,2.000){2}{\rule{0.350pt}{0.400pt}}
\put(200,318.67){\rule{0.723pt}{0.400pt}}
\multiput(200.00,318.17)(1.500,1.000){2}{\rule{0.361pt}{0.400pt}}
\put(203,319.67){\rule{0.723pt}{0.400pt}}
\multiput(203.00,319.17)(1.500,1.000){2}{\rule{0.361pt}{0.400pt}}
\put(206,320.67){\rule{0.723pt}{0.400pt}}
\multiput(206.00,320.17)(1.500,1.000){2}{\rule{0.361pt}{0.400pt}}
\put(209,321.67){\rule{0.723pt}{0.400pt}}
\multiput(209.00,321.17)(1.500,1.000){2}{\rule{0.361pt}{0.400pt}}
\put(212,322.67){\rule{0.723pt}{0.400pt}}
\multiput(212.00,322.17)(1.500,1.000){2}{\rule{0.361pt}{0.400pt}}
\put(215,324.17){\rule{0.700pt}{0.400pt}}
\multiput(215.00,323.17)(1.547,2.000){2}{\rule{0.350pt}{0.400pt}}
\put(218,325.67){\rule{0.723pt}{0.400pt}}
\multiput(218.00,325.17)(1.500,1.000){2}{\rule{0.361pt}{0.400pt}}
\put(221,326.67){\rule{0.723pt}{0.400pt}}
\multiput(221.00,326.17)(1.500,1.000){2}{\rule{0.361pt}{0.400pt}}
\put(224,327.67){\rule{0.723pt}{0.400pt}}
\multiput(224.00,327.17)(1.500,1.000){2}{\rule{0.361pt}{0.400pt}}
\put(227,328.67){\rule{0.723pt}{0.400pt}}
\multiput(227.00,328.17)(1.500,1.000){2}{\rule{0.361pt}{0.400pt}}
\put(230,329.67){\rule{0.723pt}{0.400pt}}
\multiput(230.00,329.17)(1.500,1.000){2}{\rule{0.361pt}{0.400pt}}
\put(233,331.17){\rule{0.700pt}{0.400pt}}
\multiput(233.00,330.17)(1.547,2.000){2}{\rule{0.350pt}{0.400pt}}
\put(236,332.67){\rule{0.723pt}{0.400pt}}
\multiput(236.00,332.17)(1.500,1.000){2}{\rule{0.361pt}{0.400pt}}
\put(239,333.67){\rule{0.723pt}{0.400pt}}
\multiput(239.00,333.17)(1.500,1.000){2}{\rule{0.361pt}{0.400pt}}
\put(242,334.67){\rule{0.964pt}{0.400pt}}
\multiput(242.00,334.17)(2.000,1.000){2}{\rule{0.482pt}{0.400pt}}
\put(246,335.67){\rule{0.723pt}{0.400pt}}
\multiput(246.00,335.17)(1.500,1.000){2}{\rule{0.361pt}{0.400pt}}
\put(249,336.67){\rule{0.723pt}{0.400pt}}
\multiput(249.00,336.17)(1.500,1.000){2}{\rule{0.361pt}{0.400pt}}
\put(252,338.17){\rule{0.700pt}{0.400pt}}
\multiput(252.00,337.17)(1.547,2.000){2}{\rule{0.350pt}{0.400pt}}
\put(255,339.67){\rule{0.723pt}{0.400pt}}
\multiput(255.00,339.17)(1.500,1.000){2}{\rule{0.361pt}{0.400pt}}
\put(258,340.67){\rule{0.723pt}{0.400pt}}
\multiput(258.00,340.17)(1.500,1.000){2}{\rule{0.361pt}{0.400pt}}
\put(261,341.67){\rule{0.723pt}{0.400pt}}
\multiput(261.00,341.17)(1.500,1.000){2}{\rule{0.361pt}{0.400pt}}
\put(264,342.67){\rule{0.723pt}{0.400pt}}
\multiput(264.00,342.17)(1.500,1.000){2}{\rule{0.361pt}{0.400pt}}
\put(267,343.67){\rule{0.723pt}{0.400pt}}
\multiput(267.00,343.17)(1.500,1.000){2}{\rule{0.361pt}{0.400pt}}
\put(270,345.17){\rule{0.700pt}{0.400pt}}
\multiput(270.00,344.17)(1.547,2.000){2}{\rule{0.350pt}{0.400pt}}
\put(273,346.67){\rule{0.723pt}{0.400pt}}
\multiput(273.00,346.17)(1.500,1.000){2}{\rule{0.361pt}{0.400pt}}
\put(276,347.67){\rule{0.723pt}{0.400pt}}
\multiput(276.00,347.17)(1.500,1.000){2}{\rule{0.361pt}{0.400pt}}
\put(279,348.67){\rule{0.723pt}{0.400pt}}
\multiput(279.00,348.17)(1.500,1.000){2}{\rule{0.361pt}{0.400pt}}
\put(282,349.67){\rule{0.723pt}{0.400pt}}
\multiput(282.00,349.17)(1.500,1.000){2}{\rule{0.361pt}{0.400pt}}
\put(285,350.67){\rule{0.723pt}{0.400pt}}
\multiput(285.00,350.17)(1.500,1.000){2}{\rule{0.361pt}{0.400pt}}
\put(288,352.17){\rule{0.700pt}{0.400pt}}
\multiput(288.00,351.17)(1.547,2.000){2}{\rule{0.350pt}{0.400pt}}
\put(291,353.67){\rule{0.723pt}{0.400pt}}
\multiput(291.00,353.17)(1.500,1.000){2}{\rule{0.361pt}{0.400pt}}
\put(294,354.67){\rule{0.723pt}{0.400pt}}
\multiput(294.00,354.17)(1.500,1.000){2}{\rule{0.361pt}{0.400pt}}
\put(297,355.67){\rule{0.723pt}{0.400pt}}
\multiput(297.00,355.17)(1.500,1.000){2}{\rule{0.361pt}{0.400pt}}
\put(300,356.67){\rule{0.964pt}{0.400pt}}
\multiput(300.00,356.17)(2.000,1.000){2}{\rule{0.482pt}{0.400pt}}
\put(304,357.67){\rule{0.723pt}{0.400pt}}
\multiput(304.00,357.17)(1.500,1.000){2}{\rule{0.361pt}{0.400pt}}
\put(307,359.17){\rule{0.700pt}{0.400pt}}
\multiput(307.00,358.17)(1.547,2.000){2}{\rule{0.350pt}{0.400pt}}
\put(310,360.67){\rule{0.723pt}{0.400pt}}
\multiput(310.00,360.17)(1.500,1.000){2}{\rule{0.361pt}{0.400pt}}
\put(313,361.67){\rule{0.723pt}{0.400pt}}
\multiput(313.00,361.17)(1.500,1.000){2}{\rule{0.361pt}{0.400pt}}
\put(316,362.67){\rule{0.723pt}{0.400pt}}
\multiput(316.00,362.17)(1.500,1.000){2}{\rule{0.361pt}{0.400pt}}
\put(319,363.67){\rule{0.723pt}{0.400pt}}
\multiput(319.00,363.17)(1.500,1.000){2}{\rule{0.361pt}{0.400pt}}
\put(322,364.67){\rule{0.723pt}{0.400pt}}
\multiput(322.00,364.17)(1.500,1.000){2}{\rule{0.361pt}{0.400pt}}
\put(325,366.17){\rule{0.700pt}{0.400pt}}
\multiput(325.00,365.17)(1.547,2.000){2}{\rule{0.350pt}{0.400pt}}
\put(328,367.67){\rule{0.723pt}{0.400pt}}
\multiput(328.00,367.17)(1.500,1.000){2}{\rule{0.361pt}{0.400pt}}
\put(331,368.67){\rule{0.723pt}{0.400pt}}
\multiput(331.00,368.17)(1.500,1.000){2}{\rule{0.361pt}{0.400pt}}
\put(334,369.67){\rule{0.723pt}{0.400pt}}
\multiput(334.00,369.17)(1.500,1.000){2}{\rule{0.361pt}{0.400pt}}
\put(337,370.67){\rule{0.723pt}{0.400pt}}
\multiput(337.00,370.17)(1.500,1.000){2}{\rule{0.361pt}{0.400pt}}
\put(340,371.67){\rule{0.723pt}{0.400pt}}
\multiput(340.00,371.17)(1.500,1.000){2}{\rule{0.361pt}{0.400pt}}
\put(343,373.17){\rule{0.700pt}{0.400pt}}
\multiput(343.00,372.17)(1.547,2.000){2}{\rule{0.350pt}{0.400pt}}
\put(346,374.67){\rule{0.723pt}{0.400pt}}
\multiput(346.00,374.17)(1.500,1.000){2}{\rule{0.361pt}{0.400pt}}
\put(349,375.67){\rule{0.723pt}{0.400pt}}
\multiput(349.00,375.17)(1.500,1.000){2}{\rule{0.361pt}{0.400pt}}
\put(352,376.67){\rule{0.723pt}{0.400pt}}
\multiput(352.00,376.17)(1.500,1.000){2}{\rule{0.361pt}{0.400pt}}
\put(355,377.67){\rule{0.723pt}{0.400pt}}
\multiput(355.00,377.17)(1.500,1.000){2}{\rule{0.361pt}{0.400pt}}
\put(358,378.67){\rule{0.964pt}{0.400pt}}
\multiput(358.00,378.17)(2.000,1.000){2}{\rule{0.482pt}{0.400pt}}
\put(362,380.17){\rule{0.700pt}{0.400pt}}
\multiput(362.00,379.17)(1.547,2.000){2}{\rule{0.350pt}{0.400pt}}
\put(365,381.67){\rule{0.723pt}{0.400pt}}
\multiput(365.00,381.17)(1.500,1.000){2}{\rule{0.361pt}{0.400pt}}
\put(368,382.67){\rule{0.723pt}{0.400pt}}
\multiput(368.00,382.17)(1.500,1.000){2}{\rule{0.361pt}{0.400pt}}
\put(371,383.67){\rule{0.723pt}{0.400pt}}
\multiput(371.00,383.17)(1.500,1.000){2}{\rule{0.361pt}{0.400pt}}
\put(374,384.67){\rule{0.723pt}{0.400pt}}
\multiput(374.00,384.17)(1.500,1.000){2}{\rule{0.361pt}{0.400pt}}
\put(377,385.67){\rule{0.723pt}{0.400pt}}
\multiput(377.00,385.17)(1.500,1.000){2}{\rule{0.361pt}{0.400pt}}
\put(380,387.17){\rule{0.700pt}{0.400pt}}
\multiput(380.00,386.17)(1.547,2.000){2}{\rule{0.350pt}{0.400pt}}
\put(383,388.67){\rule{0.723pt}{0.400pt}}
\multiput(383.00,388.17)(1.500,1.000){2}{\rule{0.361pt}{0.400pt}}
\put(386,389.67){\rule{0.723pt}{0.400pt}}
\multiput(386.00,389.17)(1.500,1.000){2}{\rule{0.361pt}{0.400pt}}
\put(389,390.67){\rule{0.723pt}{0.400pt}}
\multiput(389.00,390.17)(1.500,1.000){2}{\rule{0.361pt}{0.400pt}}
\put(392,391.67){\rule{0.723pt}{0.400pt}}
\multiput(392.00,391.17)(1.500,1.000){2}{\rule{0.361pt}{0.400pt}}
\put(395,392.67){\rule{0.723pt}{0.400pt}}
\multiput(395.00,392.17)(1.500,1.000){2}{\rule{0.361pt}{0.400pt}}
\put(398,394.17){\rule{0.700pt}{0.400pt}}
\multiput(398.00,393.17)(1.547,2.000){2}{\rule{0.350pt}{0.400pt}}
\put(401,395.67){\rule{0.723pt}{0.400pt}}
\multiput(401.00,395.17)(1.500,1.000){2}{\rule{0.361pt}{0.400pt}}
\put(404,396.67){\rule{0.723pt}{0.400pt}}
\multiput(404.00,396.17)(1.500,1.000){2}{\rule{0.361pt}{0.400pt}}
\put(407,397.67){\rule{0.723pt}{0.400pt}}
\multiput(407.00,397.17)(1.500,1.000){2}{\rule{0.361pt}{0.400pt}}
\put(410,398.67){\rule{0.723pt}{0.400pt}}
\multiput(410.00,398.17)(1.500,1.000){2}{\rule{0.361pt}{0.400pt}}
\put(413,399.67){\rule{0.723pt}{0.400pt}}
\multiput(413.00,399.17)(1.500,1.000){2}{\rule{0.361pt}{0.400pt}}
\put(416,401.17){\rule{0.900pt}{0.400pt}}
\multiput(416.00,400.17)(2.132,2.000){2}{\rule{0.450pt}{0.400pt}}
\put(420,402.67){\rule{0.723pt}{0.400pt}}
\multiput(420.00,402.17)(1.500,1.000){2}{\rule{0.361pt}{0.400pt}}
\put(423,403.67){\rule{0.723pt}{0.400pt}}
\multiput(423.00,403.17)(1.500,1.000){2}{\rule{0.361pt}{0.400pt}}
\put(426,404.67){\rule{0.723pt}{0.400pt}}
\multiput(426.00,404.17)(1.500,1.000){2}{\rule{0.361pt}{0.400pt}}
\put(429,405.67){\rule{0.723pt}{0.400pt}}
\multiput(429.00,405.17)(1.500,1.000){2}{\rule{0.361pt}{0.400pt}}
\put(432,406.67){\rule{0.723pt}{0.400pt}}
\multiput(432.00,406.17)(1.500,1.000){2}{\rule{0.361pt}{0.400pt}}
\put(435,408.17){\rule{0.700pt}{0.400pt}}
\multiput(435.00,407.17)(1.547,2.000){2}{\rule{0.350pt}{0.400pt}}
\put(438,409.67){\rule{0.723pt}{0.400pt}}
\multiput(438.00,409.17)(1.500,1.000){2}{\rule{0.361pt}{0.400pt}}
\put(441,410.67){\rule{0.723pt}{0.400pt}}
\multiput(441.00,410.17)(1.500,1.000){2}{\rule{0.361pt}{0.400pt}}
\put(444,411.67){\rule{0.723pt}{0.400pt}}
\multiput(444.00,411.17)(1.500,1.000){2}{\rule{0.361pt}{0.400pt}}
\put(447,412.67){\rule{0.723pt}{0.400pt}}
\multiput(447.00,412.17)(1.500,1.000){2}{\rule{0.361pt}{0.400pt}}
\put(450,413.67){\rule{0.723pt}{0.400pt}}
\multiput(450.00,413.17)(1.500,1.000){2}{\rule{0.361pt}{0.400pt}}
\put(453,415.17){\rule{0.700pt}{0.400pt}}
\multiput(453.00,414.17)(1.547,2.000){2}{\rule{0.350pt}{0.400pt}}
\put(456,416.67){\rule{0.723pt}{0.400pt}}
\multiput(456.00,416.17)(1.500,1.000){2}{\rule{0.361pt}{0.400pt}}
\put(459,417.67){\rule{0.723pt}{0.400pt}}
\multiput(459.00,417.17)(1.500,1.000){2}{\rule{0.361pt}{0.400pt}}
\put(462,418.67){\rule{0.723pt}{0.400pt}}
\multiput(462.00,418.17)(1.500,1.000){2}{\rule{0.361pt}{0.400pt}}
\put(465,419.67){\rule{0.723pt}{0.400pt}}
\multiput(465.00,419.17)(1.500,1.000){2}{\rule{0.361pt}{0.400pt}}
\put(468,420.67){\rule{0.723pt}{0.400pt}}
\multiput(468.00,420.17)(1.500,1.000){2}{\rule{0.361pt}{0.400pt}}
\put(471,422.17){\rule{0.700pt}{0.400pt}}
\multiput(471.00,421.17)(1.547,2.000){2}{\rule{0.350pt}{0.400pt}}
\put(474,423.67){\rule{0.964pt}{0.400pt}}
\multiput(474.00,423.17)(2.000,1.000){2}{\rule{0.482pt}{0.400pt}}
\put(478,424.67){\rule{0.723pt}{0.400pt}}
\multiput(478.00,424.17)(1.500,1.000){2}{\rule{0.361pt}{0.400pt}}
\put(481,425.67){\rule{0.723pt}{0.400pt}}
\multiput(481.00,425.17)(1.500,1.000){2}{\rule{0.361pt}{0.400pt}}
\put(484,426.67){\rule{0.723pt}{0.400pt}}
\multiput(484.00,426.17)(1.500,1.000){2}{\rule{0.361pt}{0.400pt}}
\put(487,427.67){\rule{0.723pt}{0.400pt}}
\multiput(487.00,427.17)(1.500,1.000){2}{\rule{0.361pt}{0.400pt}}
\put(490,429.17){\rule{0.700pt}{0.400pt}}
\multiput(490.00,428.17)(1.547,2.000){2}{\rule{0.350pt}{0.400pt}}
\put(493,430.67){\rule{0.723pt}{0.400pt}}
\multiput(493.00,430.17)(1.500,1.000){2}{\rule{0.361pt}{0.400pt}}
\put(496,431.67){\rule{0.723pt}{0.400pt}}
\multiput(496.00,431.17)(1.500,1.000){2}{\rule{0.361pt}{0.400pt}}
\put(499,432.67){\rule{0.723pt}{0.400pt}}
\multiput(499.00,432.17)(1.500,1.000){2}{\rule{0.361pt}{0.400pt}}
\put(502,433.67){\rule{0.723pt}{0.400pt}}
\multiput(502.00,433.17)(1.500,1.000){2}{\rule{0.361pt}{0.400pt}}
\put(505,434.67){\rule{0.723pt}{0.400pt}}
\multiput(505.00,434.17)(1.500,1.000){2}{\rule{0.361pt}{0.400pt}}
\put(508,436.17){\rule{0.700pt}{0.400pt}}
\multiput(508.00,435.17)(1.547,2.000){2}{\rule{0.350pt}{0.400pt}}
\put(511,437.67){\rule{0.723pt}{0.400pt}}
\multiput(511.00,437.17)(1.500,1.000){2}{\rule{0.361pt}{0.400pt}}
\put(514,438.67){\rule{0.723pt}{0.400pt}}
\multiput(514.00,438.17)(1.500,1.000){2}{\rule{0.361pt}{0.400pt}}
\put(517,439.67){\rule{0.723pt}{0.400pt}}
\multiput(517.00,439.17)(1.500,1.000){2}{\rule{0.361pt}{0.400pt}}
\put(520,440.67){\rule{0.723pt}{0.400pt}}
\multiput(520.00,440.17)(1.500,1.000){2}{\rule{0.361pt}{0.400pt}}
\put(523,441.67){\rule{0.723pt}{0.400pt}}
\multiput(523.00,441.17)(1.500,1.000){2}{\rule{0.361pt}{0.400pt}}
\put(526,443.17){\rule{0.700pt}{0.400pt}}
\multiput(526.00,442.17)(1.547,2.000){2}{\rule{0.350pt}{0.400pt}}
\put(529,444.67){\rule{0.723pt}{0.400pt}}
\multiput(529.00,444.17)(1.500,1.000){2}{\rule{0.361pt}{0.400pt}}
\put(532,445.67){\rule{0.964pt}{0.400pt}}
\multiput(532.00,445.17)(2.000,1.000){2}{\rule{0.482pt}{0.400pt}}
\put(536,446.67){\rule{0.723pt}{0.400pt}}
\multiput(536.00,446.17)(1.500,1.000){2}{\rule{0.361pt}{0.400pt}}
\put(539,447.67){\rule{0.723pt}{0.400pt}}
\multiput(539.00,447.17)(1.500,1.000){2}{\rule{0.361pt}{0.400pt}}
\put(542,448.67){\rule{0.723pt}{0.400pt}}
\multiput(542.00,448.17)(1.500,1.000){2}{\rule{0.361pt}{0.400pt}}
\put(545,450.17){\rule{0.700pt}{0.400pt}}
\multiput(545.00,449.17)(1.547,2.000){2}{\rule{0.350pt}{0.400pt}}
\put(548,451.67){\rule{0.723pt}{0.400pt}}
\multiput(548.00,451.17)(1.500,1.000){2}{\rule{0.361pt}{0.400pt}}
\put(551,452.67){\rule{0.723pt}{0.400pt}}
\multiput(551.00,452.17)(1.500,1.000){2}{\rule{0.361pt}{0.400pt}}
\put(554,453.67){\rule{0.723pt}{0.400pt}}
\multiput(554.00,453.17)(1.500,1.000){2}{\rule{0.361pt}{0.400pt}}
\put(557,454.67){\rule{0.723pt}{0.400pt}}
\multiput(557.00,454.17)(1.500,1.000){2}{\rule{0.361pt}{0.400pt}}
\put(560,455.67){\rule{0.723pt}{0.400pt}}
\multiput(560.00,455.17)(1.500,1.000){2}{\rule{0.361pt}{0.400pt}}
\put(563,457.17){\rule{0.700pt}{0.400pt}}
\multiput(563.00,456.17)(1.547,2.000){2}{\rule{0.350pt}{0.400pt}}
\put(566,458.67){\rule{0.723pt}{0.400pt}}
\multiput(566.00,458.17)(1.500,1.000){2}{\rule{0.361pt}{0.400pt}}
\put(569,459.67){\rule{0.723pt}{0.400pt}}
\multiput(569.00,459.17)(1.500,1.000){2}{\rule{0.361pt}{0.400pt}}
\put(572,460.67){\rule{0.723pt}{0.400pt}}
\multiput(572.00,460.17)(1.500,1.000){2}{\rule{0.361pt}{0.400pt}}
\put(575,461.67){\rule{0.723pt}{0.400pt}}
\multiput(575.00,461.17)(1.500,1.000){2}{\rule{0.361pt}{0.400pt}}
\put(578,462.67){\rule{0.723pt}{0.400pt}}
\multiput(578.00,462.17)(1.500,1.000){2}{\rule{0.361pt}{0.400pt}}
\put(581,464.17){\rule{0.700pt}{0.400pt}}
\multiput(581.00,463.17)(1.547,2.000){2}{\rule{0.350pt}{0.400pt}}
\put(584,465.67){\rule{0.723pt}{0.400pt}}
\multiput(584.00,465.17)(1.500,1.000){2}{\rule{0.361pt}{0.400pt}}
\put(587,466.67){\rule{0.723pt}{0.400pt}}
\multiput(587.00,466.17)(1.500,1.000){2}{\rule{0.361pt}{0.400pt}}
\put(590,467.67){\rule{0.723pt}{0.400pt}}
\multiput(590.00,467.17)(1.500,1.000){2}{\rule{0.361pt}{0.400pt}}
\put(593,468.67){\rule{0.964pt}{0.400pt}}
\multiput(593.00,468.17)(2.000,1.000){2}{\rule{0.482pt}{0.400pt}}
\put(597,469.67){\rule{0.723pt}{0.400pt}}
\multiput(597.00,469.17)(1.500,1.000){2}{\rule{0.361pt}{0.400pt}}
\put(600,471.17){\rule{0.700pt}{0.400pt}}
\multiput(600.00,470.17)(1.547,2.000){2}{\rule{0.350pt}{0.400pt}}
\put(603,472.67){\rule{0.723pt}{0.400pt}}
\multiput(603.00,472.17)(1.500,1.000){2}{\rule{0.361pt}{0.400pt}}
\put(606,473.67){\rule{0.723pt}{0.400pt}}
\multiput(606.00,473.17)(1.500,1.000){2}{\rule{0.361pt}{0.400pt}}
\put(609,474.67){\rule{0.723pt}{0.400pt}}
\multiput(609.00,474.17)(1.500,1.000){2}{\rule{0.361pt}{0.400pt}}
\put(612,475.67){\rule{0.723pt}{0.400pt}}
\multiput(612.00,475.17)(1.500,1.000){2}{\rule{0.361pt}{0.400pt}}
\put(615,476.67){\rule{0.723pt}{0.400pt}}
\multiput(615.00,476.17)(1.500,1.000){2}{\rule{0.361pt}{0.400pt}}
\put(618,478.17){\rule{0.700pt}{0.400pt}}
\multiput(618.00,477.17)(1.547,2.000){2}{\rule{0.350pt}{0.400pt}}
\put(621,479.67){\rule{0.723pt}{0.400pt}}
\multiput(621.00,479.17)(1.500,1.000){2}{\rule{0.361pt}{0.400pt}}
\put(624,480.67){\rule{0.723pt}{0.400pt}}
\multiput(624.00,480.17)(1.500,1.000){2}{\rule{0.361pt}{0.400pt}}
\put(627,481.67){\rule{0.723pt}{0.400pt}}
\multiput(627.00,481.17)(1.500,1.000){2}{\rule{0.361pt}{0.400pt}}
\put(630,482.67){\rule{0.723pt}{0.400pt}}
\multiput(630.00,482.17)(1.500,1.000){2}{\rule{0.361pt}{0.400pt}}
\put(633,483.67){\rule{0.723pt}{0.400pt}}
\multiput(633.00,483.17)(1.500,1.000){2}{\rule{0.361pt}{0.400pt}}
\put(636,485.17){\rule{0.700pt}{0.400pt}}
\multiput(636.00,484.17)(1.547,2.000){2}{\rule{0.350pt}{0.400pt}}
\put(639,486.67){\rule{0.723pt}{0.400pt}}
\multiput(639.00,486.17)(1.500,1.000){2}{\rule{0.361pt}{0.400pt}}
\put(642,487.67){\rule{0.723pt}{0.400pt}}
\multiput(642.00,487.17)(1.500,1.000){2}{\rule{0.361pt}{0.400pt}}
\put(645,488.67){\rule{0.723pt}{0.400pt}}
\multiput(645.00,488.17)(1.500,1.000){2}{\rule{0.361pt}{0.400pt}}
\put(648,489.67){\rule{0.723pt}{0.400pt}}
\multiput(648.00,489.17)(1.500,1.000){2}{\rule{0.361pt}{0.400pt}}
\put(651,490.67){\rule{0.964pt}{0.400pt}}
\multiput(651.00,490.17)(2.000,1.000){2}{\rule{0.482pt}{0.400pt}}
\put(655,492.17){\rule{0.700pt}{0.400pt}}
\multiput(655.00,491.17)(1.547,2.000){2}{\rule{0.350pt}{0.400pt}}
\put(658,493.67){\rule{0.723pt}{0.400pt}}
\multiput(658.00,493.17)(1.500,1.000){2}{\rule{0.361pt}{0.400pt}}
\put(661,494.67){\rule{0.723pt}{0.400pt}}
\multiput(661.00,494.17)(1.500,1.000){2}{\rule{0.361pt}{0.400pt}}
\put(664,495.67){\rule{0.723pt}{0.400pt}}
\multiput(664.00,495.17)(1.500,1.000){2}{\rule{0.361pt}{0.400pt}}
\put(667,496.67){\rule{0.723pt}{0.400pt}}
\multiput(667.00,496.17)(1.500,1.000){2}{\rule{0.361pt}{0.400pt}}
\put(670,497.67){\rule{0.723pt}{0.400pt}}
\multiput(670.00,497.17)(1.500,1.000){2}{\rule{0.361pt}{0.400pt}}
\put(673,499.17){\rule{0.700pt}{0.400pt}}
\multiput(673.00,498.17)(1.547,2.000){2}{\rule{0.350pt}{0.400pt}}
\put(676,500.67){\rule{0.723pt}{0.400pt}}
\multiput(676.00,500.17)(1.500,1.000){2}{\rule{0.361pt}{0.400pt}}
\put(679,501.67){\rule{0.723pt}{0.400pt}}
\multiput(679.00,501.17)(1.500,1.000){2}{\rule{0.361pt}{0.400pt}}
\put(682,502.67){\rule{0.723pt}{0.400pt}}
\multiput(682.00,502.17)(1.500,1.000){2}{\rule{0.361pt}{0.400pt}}
\put(685,503.67){\rule{0.723pt}{0.400pt}}
\multiput(685.00,503.17)(1.500,1.000){2}{\rule{0.361pt}{0.400pt}}
\put(688,504.67){\rule{0.723pt}{0.400pt}}
\multiput(688.00,504.17)(1.500,1.000){2}{\rule{0.361pt}{0.400pt}}
\put(691,506.17){\rule{0.700pt}{0.400pt}}
\multiput(691.00,505.17)(1.547,2.000){2}{\rule{0.350pt}{0.400pt}}
\put(694,507.67){\rule{0.723pt}{0.400pt}}
\multiput(694.00,507.17)(1.500,1.000){2}{\rule{0.361pt}{0.400pt}}
\put(697,508.67){\rule{0.723pt}{0.400pt}}
\multiput(697.00,508.17)(1.500,1.000){2}{\rule{0.361pt}{0.400pt}}
\put(700,509.67){\rule{0.723pt}{0.400pt}}
\multiput(700.00,509.17)(1.500,1.000){2}{\rule{0.361pt}{0.400pt}}
\put(703,510.67){\rule{0.723pt}{0.400pt}}
\multiput(703.00,510.17)(1.500,1.000){2}{\rule{0.361pt}{0.400pt}}
\put(706,511.67){\rule{0.723pt}{0.400pt}}
\multiput(706.00,511.17)(1.500,1.000){2}{\rule{0.361pt}{0.400pt}}
\put(709,512.67){\rule{0.964pt}{0.400pt}}
\multiput(709.00,512.17)(2.000,1.000){2}{\rule{0.482pt}{0.400pt}}
\put(713,514.17){\rule{0.700pt}{0.400pt}}
\multiput(713.00,513.17)(1.547,2.000){2}{\rule{0.350pt}{0.400pt}}
\put(716,515.67){\rule{0.723pt}{0.400pt}}
\multiput(716.00,515.17)(1.500,1.000){2}{\rule{0.361pt}{0.400pt}}
\put(719,516.67){\rule{0.723pt}{0.400pt}}
\multiput(719.00,516.17)(1.500,1.000){2}{\rule{0.361pt}{0.400pt}}
\put(722,517.67){\rule{0.723pt}{0.400pt}}
\multiput(722.00,517.17)(1.500,1.000){2}{\rule{0.361pt}{0.400pt}}
\put(725,518.67){\rule{0.723pt}{0.400pt}}
\multiput(725.00,518.17)(1.500,1.000){2}{\rule{0.361pt}{0.400pt}}
\put(728,519.67){\rule{0.723pt}{0.400pt}}
\multiput(728.00,519.17)(1.500,1.000){2}{\rule{0.361pt}{0.400pt}}
\put(731,521.17){\rule{0.700pt}{0.400pt}}
\multiput(731.00,520.17)(1.547,2.000){2}{\rule{0.350pt}{0.400pt}}
\put(734,522.67){\rule{0.723pt}{0.400pt}}
\multiput(734.00,522.17)(1.500,1.000){2}{\rule{0.361pt}{0.400pt}}
\put(737,523.67){\rule{0.723pt}{0.400pt}}
\multiput(737.00,523.17)(1.500,1.000){2}{\rule{0.361pt}{0.400pt}}
\put(740,524.67){\rule{0.723pt}{0.400pt}}
\multiput(740.00,524.17)(1.500,1.000){2}{\rule{0.361pt}{0.400pt}}
\put(743,525.67){\rule{0.723pt}{0.400pt}}
\multiput(743.00,525.17)(1.500,1.000){2}{\rule{0.361pt}{0.400pt}}
\put(746,526.67){\rule{0.723pt}{0.400pt}}
\multiput(746.00,526.17)(1.500,1.000){2}{\rule{0.361pt}{0.400pt}}
\put(749,528.17){\rule{0.700pt}{0.400pt}}
\multiput(749.00,527.17)(1.547,2.000){2}{\rule{0.350pt}{0.400pt}}
\put(752,529.67){\rule{0.723pt}{0.400pt}}
\multiput(752.00,529.17)(1.500,1.000){2}{\rule{0.361pt}{0.400pt}}
\put(755,530.67){\rule{0.723pt}{0.400pt}}
\multiput(755.00,530.17)(1.500,1.000){2}{\rule{0.361pt}{0.400pt}}
\put(758,531.67){\rule{0.723pt}{0.400pt}}
\multiput(758.00,531.17)(1.500,1.000){2}{\rule{0.361pt}{0.400pt}}
\put(761,532.67){\rule{0.723pt}{0.400pt}}
\multiput(761.00,532.17)(1.500,1.000){2}{\rule{0.361pt}{0.400pt}}
\put(764,533.67){\rule{0.723pt}{0.400pt}}
\multiput(764.00,533.17)(1.500,1.000){2}{\rule{0.361pt}{0.400pt}}
\put(767,535.17){\rule{0.900pt}{0.400pt}}
\multiput(767.00,534.17)(2.132,2.000){2}{\rule{0.450pt}{0.400pt}}
\put(771,536.67){\rule{0.723pt}{0.400pt}}
\multiput(771.00,536.17)(1.500,1.000){2}{\rule{0.361pt}{0.400pt}}
\put(774,537.67){\rule{0.723pt}{0.400pt}}
\multiput(774.00,537.17)(1.500,1.000){2}{\rule{0.361pt}{0.400pt}}
\put(777,538.67){\rule{0.723pt}{0.400pt}}
\multiput(777.00,538.17)(1.500,1.000){2}{\rule{0.361pt}{0.400pt}}
\put(780,539.67){\rule{0.723pt}{0.400pt}}
\multiput(780.00,539.17)(1.500,1.000){2}{\rule{0.361pt}{0.400pt}}
\put(783,540.67){\rule{0.723pt}{0.400pt}}
\multiput(783.00,540.17)(1.500,1.000){2}{\rule{0.361pt}{0.400pt}}
\put(786,542.17){\rule{0.700pt}{0.400pt}}
\multiput(786.00,541.17)(1.547,2.000){2}{\rule{0.350pt}{0.400pt}}
\put(789,543.67){\rule{0.723pt}{0.400pt}}
\multiput(789.00,543.17)(1.500,1.000){2}{\rule{0.361pt}{0.400pt}}
\put(792,544.67){\rule{0.723pt}{0.400pt}}
\multiput(792.00,544.17)(1.500,1.000){2}{\rule{0.361pt}{0.400pt}}
\put(795,545.67){\rule{0.723pt}{0.400pt}}
\multiput(795.00,545.17)(1.500,1.000){2}{\rule{0.361pt}{0.400pt}}
\put(798,546.67){\rule{0.723pt}{0.400pt}}
\multiput(798.00,546.17)(1.500,1.000){2}{\rule{0.361pt}{0.400pt}}
\put(801,547.67){\rule{0.723pt}{0.400pt}}
\multiput(801.00,547.17)(1.500,1.000){2}{\rule{0.361pt}{0.400pt}}
\put(804,549.17){\rule{0.700pt}{0.400pt}}
\multiput(804.00,548.17)(1.547,2.000){2}{\rule{0.350pt}{0.400pt}}
\put(807,550.67){\rule{0.723pt}{0.400pt}}
\multiput(807.00,550.17)(1.500,1.000){2}{\rule{0.361pt}{0.400pt}}
\put(810,551.67){\rule{0.723pt}{0.400pt}}
\multiput(810.00,551.17)(1.500,1.000){2}{\rule{0.361pt}{0.400pt}}
\put(813,552.67){\rule{0.723pt}{0.400pt}}
\multiput(813.00,552.17)(1.500,1.000){2}{\rule{0.361pt}{0.400pt}}
\put(816,553.67){\rule{0.723pt}{0.400pt}}
\multiput(816.00,553.17)(1.500,1.000){2}{\rule{0.361pt}{0.400pt}}
\put(819,554.67){\rule{0.723pt}{0.400pt}}
\multiput(819.00,554.17)(1.500,1.000){2}{\rule{0.361pt}{0.400pt}}
\put(822,556.17){\rule{0.700pt}{0.400pt}}
\multiput(822.00,555.17)(1.547,2.000){2}{\rule{0.350pt}{0.400pt}}
\put(825,557.67){\rule{0.964pt}{0.400pt}}
\multiput(825.00,557.17)(2.000,1.000){2}{\rule{0.482pt}{0.400pt}}
\put(829,558.67){\rule{0.723pt}{0.400pt}}
\multiput(829.00,558.17)(1.500,1.000){2}{\rule{0.361pt}{0.400pt}}
\put(832,559.67){\rule{0.723pt}{0.400pt}}
\multiput(832.00,559.17)(1.500,1.000){2}{\rule{0.361pt}{0.400pt}}
\put(835,560.67){\rule{0.723pt}{0.400pt}}
\multiput(835.00,560.17)(1.500,1.000){2}{\rule{0.361pt}{0.400pt}}
\put(838,561.67){\rule{0.723pt}{0.400pt}}
\multiput(838.00,561.17)(1.500,1.000){2}{\rule{0.361pt}{0.400pt}}
\put(841,563.17){\rule{0.700pt}{0.400pt}}
\multiput(841.00,562.17)(1.547,2.000){2}{\rule{0.350pt}{0.400pt}}
\put(844,564.67){\rule{0.723pt}{0.400pt}}
\multiput(844.00,564.17)(1.500,1.000){2}{\rule{0.361pt}{0.400pt}}
\put(847,565.67){\rule{0.723pt}{0.400pt}}
\multiput(847.00,565.17)(1.500,1.000){2}{\rule{0.361pt}{0.400pt}}
\put(850,566.67){\rule{0.723pt}{0.400pt}}
\multiput(850.00,566.17)(1.500,1.000){2}{\rule{0.361pt}{0.400pt}}
\put(853,567.67){\rule{0.723pt}{0.400pt}}
\multiput(853.00,567.17)(1.500,1.000){2}{\rule{0.361pt}{0.400pt}}
\put(856,568.67){\rule{0.723pt}{0.400pt}}
\multiput(856.00,568.17)(1.500,1.000){2}{\rule{0.361pt}{0.400pt}}
\put(859,570.17){\rule{0.700pt}{0.400pt}}
\multiput(859.00,569.17)(1.547,2.000){2}{\rule{0.350pt}{0.400pt}}
\put(862,571.67){\rule{0.723pt}{0.400pt}}
\multiput(862.00,571.17)(1.500,1.000){2}{\rule{0.361pt}{0.400pt}}
\put(865,572.67){\rule{0.723pt}{0.400pt}}
\multiput(865.00,572.17)(1.500,1.000){2}{\rule{0.361pt}{0.400pt}}
\put(868,573.67){\rule{0.723pt}{0.400pt}}
\multiput(868.00,573.17)(1.500,1.000){2}{\rule{0.361pt}{0.400pt}}
\put(871,574.67){\rule{0.723pt}{0.400pt}}
\multiput(871.00,574.17)(1.500,1.000){2}{\rule{0.361pt}{0.400pt}}
\put(874,575.67){\rule{0.723pt}{0.400pt}}
\multiput(874.00,575.17)(1.500,1.000){2}{\rule{0.361pt}{0.400pt}}
\put(877,577.17){\rule{0.700pt}{0.400pt}}
\multiput(877.00,576.17)(1.547,2.000){2}{\rule{0.350pt}{0.400pt}}
\put(880,578.67){\rule{0.723pt}{0.400pt}}
\multiput(880.00,578.17)(1.500,1.000){2}{\rule{0.361pt}{0.400pt}}
\put(883,579.67){\rule{0.964pt}{0.400pt}}
\multiput(883.00,579.17)(2.000,1.000){2}{\rule{0.482pt}{0.400pt}}
\put(887,580.67){\rule{0.723pt}{0.400pt}}
\multiput(887.00,580.17)(1.500,1.000){2}{\rule{0.361pt}{0.400pt}}
\put(890,581.67){\rule{0.723pt}{0.400pt}}
\multiput(890.00,581.17)(1.500,1.000){2}{\rule{0.361pt}{0.400pt}}
\put(893,582.67){\rule{0.723pt}{0.400pt}}
\multiput(893.00,582.17)(1.500,1.000){2}{\rule{0.361pt}{0.400pt}}
\put(896,584.17){\rule{0.700pt}{0.400pt}}
\multiput(896.00,583.17)(1.547,2.000){2}{\rule{0.350pt}{0.400pt}}
\put(899,585.67){\rule{0.723pt}{0.400pt}}
\multiput(899.00,585.17)(1.500,1.000){2}{\rule{0.361pt}{0.400pt}}
\put(902,586.67){\rule{0.723pt}{0.400pt}}
\multiput(902.00,586.17)(1.500,1.000){2}{\rule{0.361pt}{0.400pt}}
\put(905,587.67){\rule{0.723pt}{0.400pt}}
\multiput(905.00,587.17)(1.500,1.000){2}{\rule{0.361pt}{0.400pt}}
\put(908,588.67){\rule{0.723pt}{0.400pt}}
\multiput(908.00,588.17)(1.500,1.000){2}{\rule{0.361pt}{0.400pt}}
\put(911,589.67){\rule{0.723pt}{0.400pt}}
\multiput(911.00,589.17)(1.500,1.000){2}{\rule{0.361pt}{0.400pt}}
\put(914,591.17){\rule{0.700pt}{0.400pt}}
\multiput(914.00,590.17)(1.547,2.000){2}{\rule{0.350pt}{0.400pt}}
\put(917,592.67){\rule{0.723pt}{0.400pt}}
\multiput(917.00,592.17)(1.500,1.000){2}{\rule{0.361pt}{0.400pt}}
\put(920,593.67){\rule{0.723pt}{0.400pt}}
\multiput(920.00,593.17)(1.500,1.000){2}{\rule{0.361pt}{0.400pt}}
\put(923,594.67){\rule{0.723pt}{0.400pt}}
\multiput(923.00,594.17)(1.500,1.000){2}{\rule{0.361pt}{0.400pt}}
\put(926,595.67){\rule{0.723pt}{0.400pt}}
\multiput(926.00,595.17)(1.500,1.000){2}{\rule{0.361pt}{0.400pt}}
\put(929,596.67){\rule{0.723pt}{0.400pt}}
\multiput(929.00,596.17)(1.500,1.000){2}{\rule{0.361pt}{0.400pt}}
\put(932,598.17){\rule{0.700pt}{0.400pt}}
\multiput(932.00,597.17)(1.547,2.000){2}{\rule{0.350pt}{0.400pt}}
\put(935,599.67){\rule{0.723pt}{0.400pt}}
\multiput(935.00,599.17)(1.500,1.000){2}{\rule{0.361pt}{0.400pt}}
\put(938,600.67){\rule{0.723pt}{0.400pt}}
\multiput(938.00,600.17)(1.500,1.000){2}{\rule{0.361pt}{0.400pt}}
\put(941,601.67){\rule{0.723pt}{0.400pt}}
\multiput(941.00,601.17)(1.500,1.000){2}{\rule{0.361pt}{0.400pt}}
\put(944,602.67){\rule{0.964pt}{0.400pt}}
\multiput(944.00,602.17)(2.000,1.000){2}{\rule{0.482pt}{0.400pt}}
\put(948,603.67){\rule{0.723pt}{0.400pt}}
\multiput(948.00,603.17)(1.500,1.000){2}{\rule{0.361pt}{0.400pt}}
\put(951,605.17){\rule{0.700pt}{0.400pt}}
\multiput(951.00,604.17)(1.547,2.000){2}{\rule{0.350pt}{0.400pt}}
\put(954,606.67){\rule{0.723pt}{0.400pt}}
\multiput(954.00,606.17)(1.500,1.000){2}{\rule{0.361pt}{0.400pt}}
\put(957,607.67){\rule{0.723pt}{0.400pt}}
\multiput(957.00,607.17)(1.500,1.000){2}{\rule{0.361pt}{0.400pt}}
\put(960,608.67){\rule{0.723pt}{0.400pt}}
\multiput(960.00,608.17)(1.500,1.000){2}{\rule{0.361pt}{0.400pt}}
\put(963,609.67){\rule{0.723pt}{0.400pt}}
\multiput(963.00,609.17)(1.500,1.000){2}{\rule{0.361pt}{0.400pt}}
\put(966,610.67){\rule{0.723pt}{0.400pt}}
\multiput(966.00,610.17)(1.500,1.000){2}{\rule{0.361pt}{0.400pt}}
\put(969,612.17){\rule{0.700pt}{0.400pt}}
\multiput(969.00,611.17)(1.547,2.000){2}{\rule{0.350pt}{0.400pt}}
\put(972,613.67){\rule{0.723pt}{0.400pt}}
\multiput(972.00,613.17)(1.500,1.000){2}{\rule{0.361pt}{0.400pt}}
\put(975,614.67){\rule{0.723pt}{0.400pt}}
\multiput(975.00,614.17)(1.500,1.000){2}{\rule{0.361pt}{0.400pt}}
\put(978,615.67){\rule{0.723pt}{0.400pt}}
\multiput(978.00,615.17)(1.500,1.000){2}{\rule{0.361pt}{0.400pt}}
\put(981,616.67){\rule{0.723pt}{0.400pt}}
\multiput(981.00,616.17)(1.500,1.000){2}{\rule{0.361pt}{0.400pt}}
\put(984,617.67){\rule{0.723pt}{0.400pt}}
\multiput(984.00,617.17)(1.500,1.000){2}{\rule{0.361pt}{0.400pt}}
\put(987,619.17){\rule{0.700pt}{0.400pt}}
\multiput(987.00,618.17)(1.547,2.000){2}{\rule{0.350pt}{0.400pt}}
\put(990,620.67){\rule{0.723pt}{0.400pt}}
\multiput(990.00,620.17)(1.500,1.000){2}{\rule{0.361pt}{0.400pt}}
\put(993,621.67){\rule{0.723pt}{0.400pt}}
\multiput(993.00,621.17)(1.500,1.000){2}{\rule{0.361pt}{0.400pt}}
\put(996,622.67){\rule{0.723pt}{0.400pt}}
\multiput(996.00,622.17)(1.500,1.000){2}{\rule{0.361pt}{0.400pt}}
\put(999,623.67){\rule{0.723pt}{0.400pt}}
\multiput(999.00,623.17)(1.500,1.000){2}{\rule{0.361pt}{0.400pt}}
\put(1002,624.67){\rule{0.964pt}{0.400pt}}
\multiput(1002.00,624.17)(2.000,1.000){2}{\rule{0.482pt}{0.400pt}}
\put(1006,626.17){\rule{0.700pt}{0.400pt}}
\multiput(1006.00,625.17)(1.547,2.000){2}{\rule{0.350pt}{0.400pt}}
\put(1009,627.67){\rule{0.723pt}{0.400pt}}
\multiput(1009.00,627.17)(1.500,1.000){2}{\rule{0.361pt}{0.400pt}}
\put(1012,628.67){\rule{0.723pt}{0.400pt}}
\multiput(1012.00,628.17)(1.500,1.000){2}{\rule{0.361pt}{0.400pt}}
\put(1015,629.67){\rule{0.723pt}{0.400pt}}
\multiput(1015.00,629.17)(1.500,1.000){2}{\rule{0.361pt}{0.400pt}}
\put(1018,630.67){\rule{0.723pt}{0.400pt}}
\multiput(1018.00,630.17)(1.500,1.000){2}{\rule{0.361pt}{0.400pt}}
\put(1021,631.67){\rule{0.723pt}{0.400pt}}
\multiput(1021.00,631.17)(1.500,1.000){2}{\rule{0.361pt}{0.400pt}}
\put(1024,633.17){\rule{0.700pt}{0.400pt}}
\multiput(1024.00,632.17)(1.547,2.000){2}{\rule{0.350pt}{0.400pt}}
\put(1027,634.67){\rule{0.723pt}{0.400pt}}
\multiput(1027.00,634.17)(1.500,1.000){2}{\rule{0.361pt}{0.400pt}}
\put(1030,635.67){\rule{0.723pt}{0.400pt}}
\multiput(1030.00,635.17)(1.500,1.000){2}{\rule{0.361pt}{0.400pt}}
\put(1033,636.67){\rule{0.723pt}{0.400pt}}
\multiput(1033.00,636.17)(1.500,1.000){2}{\rule{0.361pt}{0.400pt}}
\put(1036,637.67){\rule{0.723pt}{0.400pt}}
\multiput(1036.00,637.17)(1.500,1.000){2}{\rule{0.361pt}{0.400pt}}
\put(1039,638.67){\rule{0.723pt}{0.400pt}}
\multiput(1039.00,638.17)(1.500,1.000){2}{\rule{0.361pt}{0.400pt}}
\put(1042,640.17){\rule{0.700pt}{0.400pt}}
\multiput(1042.00,639.17)(1.547,2.000){2}{\rule{0.350pt}{0.400pt}}
\put(1045,641.67){\rule{0.723pt}{0.400pt}}
\multiput(1045.00,641.17)(1.500,1.000){2}{\rule{0.361pt}{0.400pt}}
\put(1048,642.67){\rule{0.723pt}{0.400pt}}
\multiput(1048.00,642.17)(1.500,1.000){2}{\rule{0.361pt}{0.400pt}}
\put(1051,643.67){\rule{0.723pt}{0.400pt}}
\multiput(1051.00,643.17)(1.500,1.000){2}{\rule{0.361pt}{0.400pt}}
\put(1054,644.67){\rule{0.723pt}{0.400pt}}
\multiput(1054.00,644.17)(1.500,1.000){2}{\rule{0.361pt}{0.400pt}}
\put(1057,645.67){\rule{0.723pt}{0.400pt}}
\multiput(1057.00,645.17)(1.500,1.000){2}{\rule{0.361pt}{0.400pt}}
\put(1060,647.17){\rule{0.900pt}{0.400pt}}
\multiput(1060.00,646.17)(2.132,2.000){2}{\rule{0.450pt}{0.400pt}}
\put(1064,648.67){\rule{0.723pt}{0.400pt}}
\multiput(1064.00,648.17)(1.500,1.000){2}{\rule{0.361pt}{0.400pt}}
\put(1067,649.67){\rule{0.723pt}{0.400pt}}
\multiput(1067.00,649.17)(1.500,1.000){2}{\rule{0.361pt}{0.400pt}}
\put(1070,650.67){\rule{0.723pt}{0.400pt}}
\multiput(1070.00,650.17)(1.500,1.000){2}{\rule{0.361pt}{0.400pt}}
\put(1073,651.67){\rule{0.723pt}{0.400pt}}
\multiput(1073.00,651.17)(1.500,1.000){2}{\rule{0.361pt}{0.400pt}}
\put(1076,652.67){\rule{0.723pt}{0.400pt}}
\multiput(1076.00,652.17)(1.500,1.000){2}{\rule{0.361pt}{0.400pt}}
\put(1079,654.17){\rule{0.700pt}{0.400pt}}
\multiput(1079.00,653.17)(1.547,2.000){2}{\rule{0.350pt}{0.400pt}}
\put(1082,655.67){\rule{0.723pt}{0.400pt}}
\multiput(1082.00,655.17)(1.500,1.000){2}{\rule{0.361pt}{0.400pt}}
\put(1085,656.67){\rule{0.723pt}{0.400pt}}
\multiput(1085.00,656.17)(1.500,1.000){2}{\rule{0.361pt}{0.400pt}}
\put(1088,657.67){\rule{0.723pt}{0.400pt}}
\multiput(1088.00,657.17)(1.500,1.000){2}{\rule{0.361pt}{0.400pt}}
\put(1091,658.67){\rule{0.723pt}{0.400pt}}
\multiput(1091.00,658.17)(1.500,1.000){2}{\rule{0.361pt}{0.400pt}}
\put(1094,659.67){\rule{0.723pt}{0.400pt}}
\multiput(1094.00,659.17)(1.500,1.000){2}{\rule{0.361pt}{0.400pt}}
\put(1097,661.17){\rule{0.700pt}{0.400pt}}
\multiput(1097.00,660.17)(1.547,2.000){2}{\rule{0.350pt}{0.400pt}}
\put(1100,662.67){\rule{0.723pt}{0.400pt}}
\multiput(1100.00,662.17)(1.500,1.000){2}{\rule{0.361pt}{0.400pt}}
\put(1103,663.67){\rule{0.723pt}{0.400pt}}
\multiput(1103.00,663.17)(1.500,1.000){2}{\rule{0.361pt}{0.400pt}}
\put(1106,664.67){\rule{0.723pt}{0.400pt}}
\multiput(1106.00,664.17)(1.500,1.000){2}{\rule{0.361pt}{0.400pt}}
\put(1109,665.67){\rule{0.723pt}{0.400pt}}
\multiput(1109.00,665.17)(1.500,1.000){2}{\rule{0.361pt}{0.400pt}}
\put(1112,666.67){\rule{0.723pt}{0.400pt}}
\multiput(1112.00,666.17)(1.500,1.000){2}{\rule{0.361pt}{0.400pt}}
\put(1115,668.17){\rule{0.700pt}{0.400pt}}
\multiput(1115.00,667.17)(1.547,2.000){2}{\rule{0.350pt}{0.400pt}}
\put(1118,669.67){\rule{0.964pt}{0.400pt}}
\multiput(1118.00,669.17)(2.000,1.000){2}{\rule{0.482pt}{0.400pt}}
\put(1122,670.67){\rule{0.723pt}{0.400pt}}
\multiput(1122.00,670.17)(1.500,1.000){2}{\rule{0.361pt}{0.400pt}}
\put(1125,671.67){\rule{0.723pt}{0.400pt}}
\multiput(1125.00,671.17)(1.500,1.000){2}{\rule{0.361pt}{0.400pt}}
\put(1128,672.67){\rule{0.723pt}{0.400pt}}
\multiput(1128.00,672.17)(1.500,1.000){2}{\rule{0.361pt}{0.400pt}}
\put(1131,673.67){\rule{0.723pt}{0.400pt}}
\multiput(1131.00,673.17)(1.500,1.000){2}{\rule{0.361pt}{0.400pt}}
\put(1134,675.17){\rule{0.700pt}{0.400pt}}
\multiput(1134.00,674.17)(1.547,2.000){2}{\rule{0.350pt}{0.400pt}}
\put(1137,676.67){\rule{0.723pt}{0.400pt}}
\multiput(1137.00,676.17)(1.500,1.000){2}{\rule{0.361pt}{0.400pt}}
\put(1140,677.67){\rule{0.723pt}{0.400pt}}
\multiput(1140.00,677.17)(1.500,1.000){2}{\rule{0.361pt}{0.400pt}}
\put(1143,678.67){\rule{0.723pt}{0.400pt}}
\multiput(1143.00,678.17)(1.500,1.000){2}{\rule{0.361pt}{0.400pt}}
\put(1146,679.67){\rule{0.723pt}{0.400pt}}
\multiput(1146.00,679.17)(1.500,1.000){2}{\rule{0.361pt}{0.400pt}}
\put(1149,680.67){\rule{0.723pt}{0.400pt}}
\multiput(1149.00,680.17)(1.500,1.000){2}{\rule{0.361pt}{0.400pt}}
\put(1152,682.17){\rule{0.700pt}{0.400pt}}
\multiput(1152.00,681.17)(1.547,2.000){2}{\rule{0.350pt}{0.400pt}}
\put(1155,683.67){\rule{0.723pt}{0.400pt}}
\multiput(1155.00,683.17)(1.500,1.000){2}{\rule{0.361pt}{0.400pt}}
\put(1158,684.67){\rule{0.723pt}{0.400pt}}
\multiput(1158.00,684.17)(1.500,1.000){2}{\rule{0.361pt}{0.400pt}}
\put(1161,685.67){\rule{0.723pt}{0.400pt}}
\multiput(1161.00,685.17)(1.500,1.000){2}{\rule{0.361pt}{0.400pt}}
\put(1164,686.67){\rule{0.723pt}{0.400pt}}
\multiput(1164.00,686.17)(1.500,1.000){2}{\rule{0.361pt}{0.400pt}}
\put(1167,687.67){\rule{0.723pt}{0.400pt}}
\multiput(1167.00,687.17)(1.500,1.000){2}{\rule{0.361pt}{0.400pt}}
\put(1170,689.17){\rule{0.700pt}{0.400pt}}
\multiput(1170.00,688.17)(1.547,2.000){2}{\rule{0.350pt}{0.400pt}}
\put(1173,690.67){\rule{0.723pt}{0.400pt}}
\multiput(1173.00,690.17)(1.500,1.000){2}{\rule{0.361pt}{0.400pt}}
\put(1176,691.67){\rule{0.964pt}{0.400pt}}
\multiput(1176.00,691.17)(2.000,1.000){2}{\rule{0.482pt}{0.400pt}}
\put(1180,692.67){\rule{0.723pt}{0.400pt}}
\multiput(1180.00,692.17)(1.500,1.000){2}{\rule{0.361pt}{0.400pt}}
\put(1183,693.67){\rule{0.723pt}{0.400pt}}
\multiput(1183.00,693.17)(1.500,1.000){2}{\rule{0.361pt}{0.400pt}}
\put(1186,694.67){\rule{0.723pt}{0.400pt}}
\multiput(1186.00,694.17)(1.500,1.000){2}{\rule{0.361pt}{0.400pt}}
\put(1189,696.17){\rule{0.700pt}{0.400pt}}
\multiput(1189.00,695.17)(1.547,2.000){2}{\rule{0.350pt}{0.400pt}}
\put(1192,697.67){\rule{0.723pt}{0.400pt}}
\multiput(1192.00,697.17)(1.500,1.000){2}{\rule{0.361pt}{0.400pt}}
\put(1195,698.67){\rule{0.723pt}{0.400pt}}
\multiput(1195.00,698.17)(1.500,1.000){2}{\rule{0.361pt}{0.400pt}}
\put(1198,699.67){\rule{0.723pt}{0.400pt}}
\multiput(1198.00,699.17)(1.500,1.000){2}{\rule{0.361pt}{0.400pt}}
\put(1201,700.67){\rule{0.723pt}{0.400pt}}
\multiput(1201.00,700.17)(1.500,1.000){2}{\rule{0.361pt}{0.400pt}}
\put(1204,701.67){\rule{0.723pt}{0.400pt}}
\multiput(1204.00,701.17)(1.500,1.000){2}{\rule{0.361pt}{0.400pt}}
\put(1207,703.17){\rule{0.700pt}{0.400pt}}
\multiput(1207.00,702.17)(1.547,2.000){2}{\rule{0.350pt}{0.400pt}}
\put(1210,704.67){\rule{0.723pt}{0.400pt}}
\multiput(1210.00,704.17)(1.500,1.000){2}{\rule{0.361pt}{0.400pt}}
\put(1213,705.67){\rule{0.723pt}{0.400pt}}
\multiput(1213.00,705.17)(1.500,1.000){2}{\rule{0.361pt}{0.400pt}}
\put(1216,706.67){\rule{0.723pt}{0.400pt}}
\multiput(1216.00,706.17)(1.500,1.000){2}{\rule{0.361pt}{0.400pt}}
\put(1219,707.67){\rule{0.723pt}{0.400pt}}
\multiput(1219.00,707.17)(1.500,1.000){2}{\rule{0.361pt}{0.400pt}}
\put(1222,708.67){\rule{0.723pt}{0.400pt}}
\multiput(1222.00,708.17)(1.500,1.000){2}{\rule{0.361pt}{0.400pt}}
\put(1225,709.67){\rule{0.723pt}{0.400pt}}
\multiput(1225.00,709.17)(1.500,1.000){2}{\rule{0.361pt}{0.400pt}}
\put(1228,711.17){\rule{0.700pt}{0.400pt}}
\multiput(1228.00,710.17)(1.547,2.000){2}{\rule{0.350pt}{0.400pt}}
\put(1231,712.67){\rule{0.723pt}{0.400pt}}
\multiput(1231.00,712.17)(1.500,1.000){2}{\rule{0.361pt}{0.400pt}}
\put(1234,713.67){\rule{0.964pt}{0.400pt}}
\multiput(1234.00,713.17)(2.000,1.000){2}{\rule{0.482pt}{0.400pt}}
\put(1238,714.67){\rule{0.723pt}{0.400pt}}
\multiput(1238.00,714.17)(1.500,1.000){2}{\rule{0.361pt}{0.400pt}}
\put(1241,715.67){\rule{0.723pt}{0.400pt}}
\multiput(1241.00,715.17)(1.500,1.000){2}{\rule{0.361pt}{0.400pt}}
\put(1244,716.67){\rule{0.723pt}{0.400pt}}
\multiput(1244.00,716.17)(1.500,1.000){2}{\rule{0.361pt}{0.400pt}}
\put(1247,718.17){\rule{0.700pt}{0.400pt}}
\multiput(1247.00,717.17)(1.547,2.000){2}{\rule{0.350pt}{0.400pt}}
\put(1250,719.67){\rule{0.723pt}{0.400pt}}
\multiput(1250.00,719.17)(1.500,1.000){2}{\rule{0.361pt}{0.400pt}}
\put(1253,720.67){\rule{0.723pt}{0.400pt}}
\multiput(1253.00,720.17)(1.500,1.000){2}{\rule{0.361pt}{0.400pt}}
\put(1256,721.67){\rule{0.723pt}{0.400pt}}
\multiput(1256.00,721.17)(1.500,1.000){2}{\rule{0.361pt}{0.400pt}}
\put(1259,722.67){\rule{0.723pt}{0.400pt}}
\multiput(1259.00,722.17)(1.500,1.000){2}{\rule{0.361pt}{0.400pt}}
\put(1262,723.67){\rule{0.723pt}{0.400pt}}
\multiput(1262.00,723.17)(1.500,1.000){2}{\rule{0.361pt}{0.400pt}}
\put(1265,725.17){\rule{0.700pt}{0.400pt}}
\multiput(1265.00,724.17)(1.547,2.000){2}{\rule{0.350pt}{0.400pt}}
\put(1268,726.67){\rule{0.723pt}{0.400pt}}
\multiput(1268.00,726.17)(1.500,1.000){2}{\rule{0.361pt}{0.400pt}}
\put(1271,727.67){\rule{0.723pt}{0.400pt}}
\multiput(1271.00,727.17)(1.500,1.000){2}{\rule{0.361pt}{0.400pt}}
\put(1274,728.67){\rule{0.723pt}{0.400pt}}
\multiput(1274.00,728.17)(1.500,1.000){2}{\rule{0.361pt}{0.400pt}}
\put(1277,729.67){\rule{0.723pt}{0.400pt}}
\multiput(1277.00,729.17)(1.500,1.000){2}{\rule{0.361pt}{0.400pt}}
\put(1280,730.67){\rule{0.723pt}{0.400pt}}
\multiput(1280.00,730.17)(1.500,1.000){2}{\rule{0.361pt}{0.400pt}}
\put(1283,732.17){\rule{0.700pt}{0.400pt}}
\multiput(1283.00,731.17)(1.547,2.000){2}{\rule{0.350pt}{0.400pt}}
\put(1286,733.67){\rule{0.723pt}{0.400pt}}
\multiput(1286.00,733.17)(1.500,1.000){2}{\rule{0.361pt}{0.400pt}}
\put(1289,734.67){\rule{0.723pt}{0.400pt}}
\multiput(1289.00,734.17)(1.500,1.000){2}{\rule{0.361pt}{0.400pt}}
\put(1292,735.67){\rule{0.723pt}{0.400pt}}
\multiput(1292.00,735.17)(1.500,1.000){2}{\rule{0.361pt}{0.400pt}}
\put(1295,736.67){\rule{0.964pt}{0.400pt}}
\multiput(1295.00,736.17)(2.000,1.000){2}{\rule{0.482pt}{0.400pt}}
\put(1299,737.67){\rule{0.723pt}{0.400pt}}
\multiput(1299.00,737.17)(1.500,1.000){2}{\rule{0.361pt}{0.400pt}}
\put(1302,739.17){\rule{0.700pt}{0.400pt}}
\multiput(1302.00,738.17)(1.547,2.000){2}{\rule{0.350pt}{0.400pt}}
\put(1305,740.67){\rule{0.723pt}{0.400pt}}
\multiput(1305.00,740.17)(1.500,1.000){2}{\rule{0.361pt}{0.400pt}}
\put(1308,741.67){\rule{0.723pt}{0.400pt}}
\multiput(1308.00,741.17)(1.500,1.000){2}{\rule{0.361pt}{0.400pt}}
\put(1311,742.67){\rule{0.723pt}{0.400pt}}
\multiput(1311.00,742.17)(1.500,1.000){2}{\rule{0.361pt}{0.400pt}}
\put(1314,743.67){\rule{0.723pt}{0.400pt}}
\multiput(1314.00,743.17)(1.500,1.000){2}{\rule{0.361pt}{0.400pt}}
\put(1317,744.67){\rule{0.723pt}{0.400pt}}
\multiput(1317.00,744.17)(1.500,1.000){2}{\rule{0.361pt}{0.400pt}}
\put(1320,746.17){\rule{0.700pt}{0.400pt}}
\multiput(1320.00,745.17)(1.547,2.000){2}{\rule{0.350pt}{0.400pt}}
\put(1323,747.67){\rule{0.723pt}{0.400pt}}
\multiput(1323.00,747.17)(1.500,1.000){2}{\rule{0.361pt}{0.400pt}}
\put(1326,748.67){\rule{0.723pt}{0.400pt}}
\multiput(1326.00,748.17)(1.500,1.000){2}{\rule{0.361pt}{0.400pt}}
\put(1329,749.67){\rule{0.723pt}{0.400pt}}
\multiput(1329.00,749.17)(1.500,1.000){2}{\rule{0.361pt}{0.400pt}}
\put(1332,750.67){\rule{0.723pt}{0.400pt}}
\multiput(1332.00,750.17)(1.500,1.000){2}{\rule{0.361pt}{0.400pt}}
\put(1335,751.67){\rule{0.723pt}{0.400pt}}
\multiput(1335.00,751.17)(1.500,1.000){2}{\rule{0.361pt}{0.400pt}}
\put(1338,753.17){\rule{0.700pt}{0.400pt}}
\multiput(1338.00,752.17)(1.547,2.000){2}{\rule{0.350pt}{0.400pt}}
\put(1341,754.67){\rule{0.723pt}{0.400pt}}
\multiput(1341.00,754.17)(1.500,1.000){2}{\rule{0.361pt}{0.400pt}}
\put(1344,755.67){\rule{0.723pt}{0.400pt}}
\multiput(1344.00,755.17)(1.500,1.000){2}{\rule{0.361pt}{0.400pt}}
\put(1347,756.67){\rule{0.723pt}{0.400pt}}
\multiput(1347.00,756.17)(1.500,1.000){2}{\rule{0.361pt}{0.400pt}}
\put(1350,757.67){\rule{0.723pt}{0.400pt}}
\multiput(1350.00,757.17)(1.500,1.000){2}{\rule{0.361pt}{0.400pt}}
\put(1353,758.67){\rule{0.964pt}{0.400pt}}
\multiput(1353.00,758.17)(2.000,1.000){2}{\rule{0.482pt}{0.400pt}}
\put(1357,760.17){\rule{0.700pt}{0.400pt}}
\multiput(1357.00,759.17)(1.547,2.000){2}{\rule{0.350pt}{0.400pt}}
\put(1360,761.67){\rule{0.723pt}{0.400pt}}
\multiput(1360.00,761.17)(1.500,1.000){2}{\rule{0.361pt}{0.400pt}}
\put(1363,762.67){\rule{0.723pt}{0.400pt}}
\multiput(1363.00,762.17)(1.500,1.000){2}{\rule{0.361pt}{0.400pt}}
\put(1366,763.67){\rule{0.723pt}{0.400pt}}
\multiput(1366.00,763.17)(1.500,1.000){2}{\rule{0.361pt}{0.400pt}}
\put(1369,764.67){\rule{0.723pt}{0.400pt}}
\multiput(1369.00,764.17)(1.500,1.000){2}{\rule{0.361pt}{0.400pt}}
\put(1372,765.67){\rule{0.723pt}{0.400pt}}
\multiput(1372.00,765.17)(1.500,1.000){2}{\rule{0.361pt}{0.400pt}}
\put(1375,767.17){\rule{0.700pt}{0.400pt}}
\multiput(1375.00,766.17)(1.547,2.000){2}{\rule{0.350pt}{0.400pt}}
\put(1378,768.67){\rule{0.723pt}{0.400pt}}
\multiput(1378.00,768.17)(1.500,1.000){2}{\rule{0.361pt}{0.400pt}}
\put(1381,769.67){\rule{0.723pt}{0.400pt}}
\multiput(1381.00,769.17)(1.500,1.000){2}{\rule{0.361pt}{0.400pt}}
\put(1384,770.67){\rule{0.723pt}{0.400pt}}
\multiput(1384.00,770.17)(1.500,1.000){2}{\rule{0.361pt}{0.400pt}}
\put(1387,771.67){\rule{0.723pt}{0.400pt}}
\multiput(1387.00,771.17)(1.500,1.000){2}{\rule{0.361pt}{0.400pt}}
\put(1390,772.67){\rule{0.723pt}{0.400pt}}
\multiput(1390.00,772.17)(1.500,1.000){2}{\rule{0.361pt}{0.400pt}}
\put(1393,774.17){\rule{0.700pt}{0.400pt}}
\multiput(1393.00,773.17)(1.547,2.000){2}{\rule{0.350pt}{0.400pt}}
\put(1396,775.67){\rule{0.723pt}{0.400pt}}
\multiput(1396.00,775.17)(1.500,1.000){2}{\rule{0.361pt}{0.400pt}}
\put(130.0,82.0){\rule[-0.200pt]{0.400pt}{187.179pt}}
\put(130.0,82.0){\rule[-0.200pt]{315.338pt}{0.400pt}}
\put(1439.0,82.0){\rule[-0.200pt]{0.400pt}{187.179pt}}
\put(130.0,859.0){\rule[-0.200pt]{315.338pt}{0.400pt}}
\end{picture}

Plot for Ball 2:\\
% GNUPLOT: LaTeX picture
\setlength{\unitlength}{0.240900pt}
\ifx\plotpoint\undefined\newsavebox{\plotpoint}\fi
\sbox{\plotpoint}{\rule[-0.200pt]{0.400pt}{0.400pt}}%
\begin{picture}(1500,900)(0,0)
\sbox{\plotpoint}{\rule[-0.200pt]{0.400pt}{0.400pt}}%
\put(130.0,90.0){\rule[-0.200pt]{4.818pt}{0.400pt}}
\put(110,90){\makebox(0,0)[r]{ 0}}
\put(1419.0,90.0){\rule[-0.200pt]{4.818pt}{0.400pt}}
\put(130.0,242.0){\rule[-0.200pt]{4.818pt}{0.400pt}}
\put(110,242){\makebox(0,0)[r]{ 0.2}}
\put(1419.0,242.0){\rule[-0.200pt]{4.818pt}{0.400pt}}
\put(130.0,394.0){\rule[-0.200pt]{4.818pt}{0.400pt}}
\put(110,394){\makebox(0,0)[r]{ 0.4}}
\put(1419.0,394.0){\rule[-0.200pt]{4.818pt}{0.400pt}}
\put(130.0,547.0){\rule[-0.200pt]{4.818pt}{0.400pt}}
\put(110,547){\makebox(0,0)[r]{ 0.6}}
\put(1419.0,547.0){\rule[-0.200pt]{4.818pt}{0.400pt}}
\put(130.0,699.0){\rule[-0.200pt]{4.818pt}{0.400pt}}
\put(110,699){\makebox(0,0)[r]{ 0.8}}
\put(1419.0,699.0){\rule[-0.200pt]{4.818pt}{0.400pt}}
\put(130.0,851.0){\rule[-0.200pt]{4.818pt}{0.400pt}}
\put(110,851){\makebox(0,0)[r]{ 1}}
\put(1419.0,851.0){\rule[-0.200pt]{4.818pt}{0.400pt}}
\put(130.0,82.0){\rule[-0.200pt]{0.400pt}{4.818pt}}
\put(130,41){\makebox(0,0){ 0}}
\put(130.0,839.0){\rule[-0.200pt]{0.400pt}{4.818pt}}
\put(392.0,82.0){\rule[-0.200pt]{0.400pt}{4.818pt}}
\put(392,41){\makebox(0,0){ 0.2}}
\put(392.0,839.0){\rule[-0.200pt]{0.400pt}{4.818pt}}
\put(654.0,82.0){\rule[-0.200pt]{0.400pt}{4.818pt}}
\put(654,41){\makebox(0,0){ 0.4}}
\put(654.0,839.0){\rule[-0.200pt]{0.400pt}{4.818pt}}
\put(915.0,82.0){\rule[-0.200pt]{0.400pt}{4.818pt}}
\put(915,41){\makebox(0,0){ 0.6}}
\put(915.0,839.0){\rule[-0.200pt]{0.400pt}{4.818pt}}
\put(1177.0,82.0){\rule[-0.200pt]{0.400pt}{4.818pt}}
\put(1177,41){\makebox(0,0){ 0.8}}
\put(1177.0,839.0){\rule[-0.200pt]{0.400pt}{4.818pt}}
\put(1439.0,82.0){\rule[-0.200pt]{0.400pt}{4.818pt}}
\put(1439,41){\makebox(0,0){ 1}}
\put(1439.0,839.0){\rule[-0.200pt]{0.400pt}{4.818pt}}
\put(130.0,82.0){\rule[-0.200pt]{0.400pt}{187.179pt}}
\put(130.0,82.0){\rule[-0.200pt]{315.338pt}{0.400pt}}
\put(1439.0,82.0){\rule[-0.200pt]{0.400pt}{187.179pt}}
\put(130.0,859.0){\rule[-0.200pt]{315.338pt}{0.400pt}}
\put(1279,819){\makebox(0,0)[r]{'-'}}
\put(1299.0,819.0){\rule[-0.200pt]{24.090pt}{0.400pt}}
\put(955,532){\usebox{\plotpoint}}
\put(955,531.67){\rule{1.204pt}{0.400pt}}
\multiput(955.00,531.17)(2.500,1.000){2}{\rule{0.602pt}{0.400pt}}
\put(965,532.67){\rule{1.204pt}{0.400pt}}
\multiput(965.00,532.17)(2.500,1.000){2}{\rule{0.602pt}{0.400pt}}
\put(960.0,533.0){\rule[-0.200pt]{1.204pt}{0.400pt}}
\put(975,533.67){\rule{1.204pt}{0.400pt}}
\multiput(975.00,533.17)(2.500,1.000){2}{\rule{0.602pt}{0.400pt}}
\put(970.0,534.0){\rule[-0.200pt]{1.204pt}{0.400pt}}
\put(985,534.67){\rule{1.204pt}{0.400pt}}
\multiput(985.00,534.17)(2.500,1.000){2}{\rule{0.602pt}{0.400pt}}
\put(980.0,535.0){\rule[-0.200pt]{1.204pt}{0.400pt}}
\put(1000,535.67){\rule{1.204pt}{0.400pt}}
\multiput(1000.00,535.17)(2.500,1.000){2}{\rule{0.602pt}{0.400pt}}
\put(990.0,536.0){\rule[-0.200pt]{2.409pt}{0.400pt}}
\put(1010,536.67){\rule{1.204pt}{0.400pt}}
\multiput(1010.00,536.17)(2.500,1.000){2}{\rule{0.602pt}{0.400pt}}
\put(1005.0,537.0){\rule[-0.200pt]{1.204pt}{0.400pt}}
\put(1020,537.67){\rule{1.204pt}{0.400pt}}
\multiput(1020.00,537.17)(2.500,1.000){2}{\rule{0.602pt}{0.400pt}}
\put(1015.0,538.0){\rule[-0.200pt]{1.204pt}{0.400pt}}
\put(1030,538.67){\rule{1.204pt}{0.400pt}}
\multiput(1030.00,538.17)(2.500,1.000){2}{\rule{0.602pt}{0.400pt}}
\put(1025.0,539.0){\rule[-0.200pt]{1.204pt}{0.400pt}}
\put(1040,539.67){\rule{1.204pt}{0.400pt}}
\multiput(1040.00,539.17)(2.500,1.000){2}{\rule{0.602pt}{0.400pt}}
\put(1035.0,540.0){\rule[-0.200pt]{1.204pt}{0.400pt}}
\put(1055,540.67){\rule{0.964pt}{0.400pt}}
\multiput(1055.00,540.17)(2.000,1.000){2}{\rule{0.482pt}{0.400pt}}
\put(1045.0,541.0){\rule[-0.200pt]{2.409pt}{0.400pt}}
\put(1064,541.67){\rule{1.204pt}{0.400pt}}
\multiput(1064.00,541.17)(2.500,1.000){2}{\rule{0.602pt}{0.400pt}}
\put(1059.0,542.0){\rule[-0.200pt]{1.204pt}{0.400pt}}
\put(1074,542.67){\rule{1.204pt}{0.400pt}}
\multiput(1074.00,542.17)(2.500,1.000){2}{\rule{0.602pt}{0.400pt}}
\put(1069.0,543.0){\rule[-0.200pt]{1.204pt}{0.400pt}}
\put(1084,543.67){\rule{1.204pt}{0.400pt}}
\multiput(1084.00,543.17)(2.500,1.000){2}{\rule{0.602pt}{0.400pt}}
\put(1079.0,544.0){\rule[-0.200pt]{1.204pt}{0.400pt}}
\put(1094,544.67){\rule{1.204pt}{0.400pt}}
\multiput(1094.00,544.17)(2.500,1.000){2}{\rule{0.602pt}{0.400pt}}
\put(1089.0,545.0){\rule[-0.200pt]{1.204pt}{0.400pt}}
\put(1104,545.67){\rule{1.204pt}{0.400pt}}
\multiput(1104.00,545.17)(2.500,1.000){2}{\rule{0.602pt}{0.400pt}}
\put(1099.0,546.0){\rule[-0.200pt]{1.204pt}{0.400pt}}
\put(1119,546.67){\rule{1.204pt}{0.400pt}}
\multiput(1119.00,546.17)(2.500,1.000){2}{\rule{0.602pt}{0.400pt}}
\put(1109.0,547.0){\rule[-0.200pt]{2.409pt}{0.400pt}}
\put(1129,547.67){\rule{1.204pt}{0.400pt}}
\multiput(1129.00,547.17)(2.500,1.000){2}{\rule{0.602pt}{0.400pt}}
\put(1124.0,548.0){\rule[-0.200pt]{1.204pt}{0.400pt}}
\put(1139,548.67){\rule{1.204pt}{0.400pt}}
\multiput(1139.00,548.17)(2.500,1.000){2}{\rule{0.602pt}{0.400pt}}
\put(1134.0,549.0){\rule[-0.200pt]{1.204pt}{0.400pt}}
\put(1149,549.67){\rule{1.204pt}{0.400pt}}
\multiput(1149.00,549.17)(2.500,1.000){2}{\rule{0.602pt}{0.400pt}}
\put(1144.0,550.0){\rule[-0.200pt]{1.204pt}{0.400pt}}
\put(1165,549.67){\rule{0.482pt}{0.400pt}}
\multiput(1165.00,550.17)(1.000,-1.000){2}{\rule{0.241pt}{0.400pt}}
\put(1154.0,551.0){\rule[-0.200pt]{2.650pt}{0.400pt}}
\put(1179,548.67){\rule{0.723pt}{0.400pt}}
\multiput(1179.00,549.17)(1.500,-1.000){2}{\rule{0.361pt}{0.400pt}}
\put(1167.0,550.0){\rule[-0.200pt]{2.891pt}{0.400pt}}
\put(1193,547.67){\rule{0.723pt}{0.400pt}}
\multiput(1193.00,548.17)(1.500,-1.000){2}{\rule{0.361pt}{0.400pt}}
\put(1182.0,549.0){\rule[-0.200pt]{2.650pt}{0.400pt}}
\put(1208,546.67){\rule{0.482pt}{0.400pt}}
\multiput(1208.00,547.17)(1.000,-1.000){2}{\rule{0.241pt}{0.400pt}}
\put(1196.0,548.0){\rule[-0.200pt]{2.891pt}{0.400pt}}
\put(1222,545.67){\rule{0.723pt}{0.400pt}}
\multiput(1222.00,546.17)(1.500,-1.000){2}{\rule{0.361pt}{0.400pt}}
\put(1210.0,547.0){\rule[-0.200pt]{2.891pt}{0.400pt}}
\put(1236,544.67){\rule{0.723pt}{0.400pt}}
\multiput(1236.00,545.17)(1.500,-1.000){2}{\rule{0.361pt}{0.400pt}}
\put(1225.0,546.0){\rule[-0.200pt]{2.650pt}{0.400pt}}
\put(1251,543.67){\rule{0.723pt}{0.400pt}}
\multiput(1251.00,544.17)(1.500,-1.000){2}{\rule{0.361pt}{0.400pt}}
\put(1239.0,545.0){\rule[-0.200pt]{2.891pt}{0.400pt}}
\put(1265,542.67){\rule{0.723pt}{0.400pt}}
\multiput(1265.00,543.17)(1.500,-1.000){2}{\rule{0.361pt}{0.400pt}}
\put(1254.0,544.0){\rule[-0.200pt]{2.650pt}{0.400pt}}
\put(1279,541.67){\rule{0.723pt}{0.400pt}}
\multiput(1279.00,542.17)(1.500,-1.000){2}{\rule{0.361pt}{0.400pt}}
\put(1268.0,543.0){\rule[-0.200pt]{2.650pt}{0.400pt}}
\put(1294,540.67){\rule{0.723pt}{0.400pt}}
\multiput(1294.00,541.17)(1.500,-1.000){2}{\rule{0.361pt}{0.400pt}}
\put(1282.0,542.0){\rule[-0.200pt]{2.891pt}{0.400pt}}
\put(1308,539.67){\rule{0.723pt}{0.400pt}}
\multiput(1308.00,540.17)(1.500,-1.000){2}{\rule{0.361pt}{0.400pt}}
\put(1297.0,541.0){\rule[-0.200pt]{2.650pt}{0.400pt}}
\put(1323,538.67){\rule{0.482pt}{0.400pt}}
\multiput(1323.00,539.17)(1.000,-1.000){2}{\rule{0.241pt}{0.400pt}}
\put(1311.0,540.0){\rule[-0.200pt]{2.891pt}{0.400pt}}
\put(1337,537.67){\rule{0.723pt}{0.400pt}}
\multiput(1337.00,538.17)(1.500,-1.000){2}{\rule{0.361pt}{0.400pt}}
\put(1325.0,539.0){\rule[-0.200pt]{2.891pt}{0.400pt}}
\put(1351,536.67){\rule{0.723pt}{0.400pt}}
\multiput(1351.00,537.17)(1.500,-1.000){2}{\rule{0.361pt}{0.400pt}}
\put(1340.0,538.0){\rule[-0.200pt]{2.650pt}{0.400pt}}
\put(1366,535.67){\rule{0.723pt}{0.400pt}}
\multiput(1366.00,536.17)(1.500,-1.000){2}{\rule{0.361pt}{0.400pt}}
\put(1354.0,537.0){\rule[-0.200pt]{2.891pt}{0.400pt}}
\put(1380,534.67){\rule{0.723pt}{0.400pt}}
\multiput(1380.00,535.17)(1.500,-1.000){2}{\rule{0.361pt}{0.400pt}}
\put(1369.0,536.0){\rule[-0.200pt]{2.650pt}{0.400pt}}
\put(1394,533.67){\rule{0.723pt}{0.400pt}}
\multiput(1394.00,534.17)(1.500,-1.000){2}{\rule{0.361pt}{0.400pt}}
\put(1383.0,535.0){\rule[-0.200pt]{2.650pt}{0.400pt}}
\put(1409,532.67){\rule{0.723pt}{0.400pt}}
\multiput(1409.00,533.17)(1.500,-1.000){2}{\rule{0.361pt}{0.400pt}}
\put(1397.0,534.0){\rule[-0.200pt]{2.891pt}{0.400pt}}
\put(1412.0,533.0){\rule[-0.200pt]{0.482pt}{0.400pt}}
\put(1403,531.67){\rule{0.723pt}{0.400pt}}
\multiput(1404.50,532.17)(-1.500,-1.000){2}{\rule{0.361pt}{0.400pt}}
\put(1406.0,533.0){\rule[-0.200pt]{1.927pt}{0.400pt}}
\put(1389,530.67){\rule{0.723pt}{0.400pt}}
\multiput(1390.50,531.17)(-1.500,-1.000){2}{\rule{0.361pt}{0.400pt}}
\put(1392.0,532.0){\rule[-0.200pt]{2.650pt}{0.400pt}}
\put(1374,529.67){\rule{0.723pt}{0.400pt}}
\multiput(1375.50,530.17)(-1.500,-1.000){2}{\rule{0.361pt}{0.400pt}}
\put(1377.0,531.0){\rule[-0.200pt]{2.891pt}{0.400pt}}
\put(1360,528.67){\rule{0.723pt}{0.400pt}}
\multiput(1361.50,529.17)(-1.500,-1.000){2}{\rule{0.361pt}{0.400pt}}
\put(1363.0,530.0){\rule[-0.200pt]{2.650pt}{0.400pt}}
\put(1346,527.67){\rule{0.482pt}{0.400pt}}
\multiput(1347.00,528.17)(-1.000,-1.000){2}{\rule{0.241pt}{0.400pt}}
\put(1348.0,529.0){\rule[-0.200pt]{2.891pt}{0.400pt}}
\put(1331,526.67){\rule{0.723pt}{0.400pt}}
\multiput(1332.50,527.17)(-1.500,-1.000){2}{\rule{0.361pt}{0.400pt}}
\put(1334.0,528.0){\rule[-0.200pt]{2.891pt}{0.400pt}}
\put(1317,525.67){\rule{0.723pt}{0.400pt}}
\multiput(1318.50,526.17)(-1.500,-1.000){2}{\rule{0.361pt}{0.400pt}}
\put(1320.0,527.0){\rule[-0.200pt]{2.650pt}{0.400pt}}
\put(1302,524.67){\rule{0.723pt}{0.400pt}}
\multiput(1303.50,525.17)(-1.500,-1.000){2}{\rule{0.361pt}{0.400pt}}
\put(1305.0,526.0){\rule[-0.200pt]{2.891pt}{0.400pt}}
\put(1288,523.67){\rule{0.723pt}{0.400pt}}
\multiput(1289.50,524.17)(-1.500,-1.000){2}{\rule{0.361pt}{0.400pt}}
\put(1291.0,525.0){\rule[-0.200pt]{2.650pt}{0.400pt}}
\put(1274,522.67){\rule{0.723pt}{0.400pt}}
\multiput(1275.50,523.17)(-1.500,-1.000){2}{\rule{0.361pt}{0.400pt}}
\put(1277.0,524.0){\rule[-0.200pt]{2.650pt}{0.400pt}}
\put(1259,521.67){\rule{0.723pt}{0.400pt}}
\multiput(1260.50,522.17)(-1.500,-1.000){2}{\rule{0.361pt}{0.400pt}}
\put(1262.0,523.0){\rule[-0.200pt]{2.891pt}{0.400pt}}
\put(1245,520.67){\rule{0.723pt}{0.400pt}}
\multiput(1246.50,521.17)(-1.500,-1.000){2}{\rule{0.361pt}{0.400pt}}
\put(1248.0,522.0){\rule[-0.200pt]{2.650pt}{0.400pt}}
\put(1231,519.67){\rule{0.482pt}{0.400pt}}
\multiput(1232.00,520.17)(-1.000,-1.000){2}{\rule{0.241pt}{0.400pt}}
\put(1233.0,521.0){\rule[-0.200pt]{2.891pt}{0.400pt}}
\put(1216,518.67){\rule{0.723pt}{0.400pt}}
\multiput(1217.50,519.17)(-1.500,-1.000){2}{\rule{0.361pt}{0.400pt}}
\put(1219.0,520.0){\rule[-0.200pt]{2.891pt}{0.400pt}}
\put(1202,517.67){\rule{0.723pt}{0.400pt}}
\multiput(1203.50,518.17)(-1.500,-1.000){2}{\rule{0.361pt}{0.400pt}}
\put(1205.0,519.0){\rule[-0.200pt]{2.650pt}{0.400pt}}
\put(1187,516.67){\rule{0.723pt}{0.400pt}}
\multiput(1188.50,517.17)(-1.500,-1.000){2}{\rule{0.361pt}{0.400pt}}
\put(1190.0,518.0){\rule[-0.200pt]{2.891pt}{0.400pt}}
\put(1173,515.67){\rule{0.723pt}{0.400pt}}
\multiput(1174.50,516.17)(-1.500,-1.000){2}{\rule{0.361pt}{0.400pt}}
\put(1176.0,517.0){\rule[-0.200pt]{2.650pt}{0.400pt}}
\put(1159,514.67){\rule{0.723pt}{0.400pt}}
\multiput(1160.50,515.17)(-1.500,-1.000){2}{\rule{0.361pt}{0.400pt}}
\put(1162.0,516.0){\rule[-0.200pt]{2.650pt}{0.400pt}}
\put(1144,513.67){\rule{0.723pt}{0.400pt}}
\multiput(1145.50,514.17)(-1.500,-1.000){2}{\rule{0.361pt}{0.400pt}}
\put(1147.0,515.0){\rule[-0.200pt]{2.891pt}{0.400pt}}
\put(1130,512.67){\rule{0.723pt}{0.400pt}}
\multiput(1131.50,513.17)(-1.500,-1.000){2}{\rule{0.361pt}{0.400pt}}
\put(1133.0,514.0){\rule[-0.200pt]{2.650pt}{0.400pt}}
\put(1116,511.67){\rule{0.723pt}{0.400pt}}
\multiput(1117.50,512.17)(-1.500,-1.000){2}{\rule{0.361pt}{0.400pt}}
\put(1119.0,513.0){\rule[-0.200pt]{2.650pt}{0.400pt}}
\put(1101,510.67){\rule{0.723pt}{0.400pt}}
\multiput(1102.50,511.17)(-1.500,-1.000){2}{\rule{0.361pt}{0.400pt}}
\put(1104.0,512.0){\rule[-0.200pt]{2.891pt}{0.400pt}}
\put(1087,509.67){\rule{0.723pt}{0.400pt}}
\multiput(1088.50,510.17)(-1.500,-1.000){2}{\rule{0.361pt}{0.400pt}}
\put(1090.0,511.0){\rule[-0.200pt]{2.650pt}{0.400pt}}
\put(1073,508.67){\rule{0.482pt}{0.400pt}}
\multiput(1074.00,509.17)(-1.000,-1.000){2}{\rule{0.241pt}{0.400pt}}
\put(1075.0,510.0){\rule[-0.200pt]{2.891pt}{0.400pt}}
\put(1058,507.67){\rule{0.723pt}{0.400pt}}
\multiput(1059.50,508.17)(-1.500,-1.000){2}{\rule{0.361pt}{0.400pt}}
\put(1061.0,509.0){\rule[-0.200pt]{2.891pt}{0.400pt}}
\put(1044,506.67){\rule{0.723pt}{0.400pt}}
\multiput(1045.50,507.17)(-1.500,-1.000){2}{\rule{0.361pt}{0.400pt}}
\put(1047.0,508.0){\rule[-0.200pt]{2.650pt}{0.400pt}}
\put(1029,505.67){\rule{0.723pt}{0.400pt}}
\multiput(1030.50,506.17)(-1.500,-1.000){2}{\rule{0.361pt}{0.400pt}}
\put(1032.0,507.0){\rule[-0.200pt]{2.891pt}{0.400pt}}
\put(1015,504.67){\rule{0.723pt}{0.400pt}}
\multiput(1016.50,505.17)(-1.500,-1.000){2}{\rule{0.361pt}{0.400pt}}
\put(1018.0,506.0){\rule[-0.200pt]{2.650pt}{0.400pt}}
\put(1001,503.67){\rule{0.723pt}{0.400pt}}
\multiput(1002.50,504.17)(-1.500,-1.000){2}{\rule{0.361pt}{0.400pt}}
\put(1004.0,505.0){\rule[-0.200pt]{2.650pt}{0.400pt}}
\put(986,502.67){\rule{0.723pt}{0.400pt}}
\multiput(987.50,503.17)(-1.500,-1.000){2}{\rule{0.361pt}{0.400pt}}
\put(989.0,504.0){\rule[-0.200pt]{2.891pt}{0.400pt}}
\put(972,501.67){\rule{0.723pt}{0.400pt}}
\multiput(973.50,502.17)(-1.500,-1.000){2}{\rule{0.361pt}{0.400pt}}
\put(975.0,503.0){\rule[-0.200pt]{2.650pt}{0.400pt}}
\put(958,500.67){\rule{0.723pt}{0.400pt}}
\multiput(959.50,501.17)(-1.500,-1.000){2}{\rule{0.361pt}{0.400pt}}
\put(961.0,502.0){\rule[-0.200pt]{2.650pt}{0.400pt}}
\put(943,499.67){\rule{0.723pt}{0.400pt}}
\multiput(944.50,500.17)(-1.500,-1.000){2}{\rule{0.361pt}{0.400pt}}
\put(946.0,501.0){\rule[-0.200pt]{2.891pt}{0.400pt}}
\put(929,498.67){\rule{0.723pt}{0.400pt}}
\multiput(930.50,499.17)(-1.500,-1.000){2}{\rule{0.361pt}{0.400pt}}
\put(932.0,500.0){\rule[-0.200pt]{2.650pt}{0.400pt}}
\put(915,497.67){\rule{0.482pt}{0.400pt}}
\multiput(916.00,498.17)(-1.000,-1.000){2}{\rule{0.241pt}{0.400pt}}
\put(917.0,499.0){\rule[-0.200pt]{2.891pt}{0.400pt}}
\put(900,496.67){\rule{0.723pt}{0.400pt}}
\multiput(901.50,497.17)(-1.500,-1.000){2}{\rule{0.361pt}{0.400pt}}
\put(903.0,498.0){\rule[-0.200pt]{2.891pt}{0.400pt}}
\put(892.0,497.0){\rule[-0.200pt]{1.927pt}{0.400pt}}
\put(891.67,501){\rule{0.400pt}{0.964pt}}
\multiput(891.17,501.00)(1.000,2.000){2}{\rule{0.400pt}{0.482pt}}
\put(892.0,497.0){\rule[-0.200pt]{0.400pt}{0.964pt}}
\put(892.67,510){\rule{0.400pt}{0.964pt}}
\multiput(892.17,510.00)(1.000,2.000){2}{\rule{0.400pt}{0.482pt}}
\put(893.0,505.0){\rule[-0.200pt]{0.400pt}{1.204pt}}
\put(893.67,518){\rule{0.400pt}{1.204pt}}
\multiput(893.17,518.00)(1.000,2.500){2}{\rule{0.400pt}{0.602pt}}
\put(894.0,514.0){\rule[-0.200pt]{0.400pt}{0.964pt}}
\put(894.67,527){\rule{0.400pt}{0.964pt}}
\multiput(894.17,527.00)(1.000,2.000){2}{\rule{0.400pt}{0.482pt}}
\put(895.0,523.0){\rule[-0.200pt]{0.400pt}{0.964pt}}
\put(895.67,535){\rule{0.400pt}{1.204pt}}
\multiput(895.17,535.00)(1.000,2.500){2}{\rule{0.400pt}{0.602pt}}
\put(896.67,540){\rule{0.400pt}{0.964pt}}
\multiput(896.17,540.00)(1.000,2.000){2}{\rule{0.400pt}{0.482pt}}
\put(896.0,531.0){\rule[-0.200pt]{0.400pt}{0.964pt}}
\put(897.67,548){\rule{0.400pt}{1.204pt}}
\multiput(897.17,548.00)(1.000,2.500){2}{\rule{0.400pt}{0.602pt}}
\put(898.0,544.0){\rule[-0.200pt]{0.400pt}{0.964pt}}
\put(898.67,557){\rule{0.400pt}{0.964pt}}
\multiput(898.17,557.00)(1.000,2.000){2}{\rule{0.400pt}{0.482pt}}
\put(899.0,553.0){\rule[-0.200pt]{0.400pt}{0.964pt}}
\put(899.67,566){\rule{0.400pt}{0.964pt}}
\multiput(899.17,566.00)(1.000,2.000){2}{\rule{0.400pt}{0.482pt}}
\put(900.0,561.0){\rule[-0.200pt]{0.400pt}{1.204pt}}
\put(900.67,574){\rule{0.400pt}{1.204pt}}
\multiput(900.17,574.00)(1.000,2.500){2}{\rule{0.400pt}{0.602pt}}
\put(901.0,570.0){\rule[-0.200pt]{0.400pt}{0.964pt}}
\put(901.67,583){\rule{0.400pt}{0.964pt}}
\multiput(901.17,583.00)(1.000,2.000){2}{\rule{0.400pt}{0.482pt}}
\put(902.67,587){\rule{0.400pt}{0.964pt}}
\multiput(902.17,587.00)(1.000,2.000){2}{\rule{0.400pt}{0.482pt}}
\put(902.0,579.0){\rule[-0.200pt]{0.400pt}{0.964pt}}
\put(903.67,596){\rule{0.400pt}{0.964pt}}
\multiput(903.17,596.00)(1.000,2.000){2}{\rule{0.400pt}{0.482pt}}
\put(904.0,591.0){\rule[-0.200pt]{0.400pt}{1.204pt}}
\put(904.67,604){\rule{0.400pt}{1.204pt}}
\multiput(904.17,604.00)(1.000,2.500){2}{\rule{0.400pt}{0.602pt}}
\put(905.0,600.0){\rule[-0.200pt]{0.400pt}{0.964pt}}
\put(905.67,613){\rule{0.400pt}{0.964pt}}
\multiput(905.17,613.00)(1.000,2.000){2}{\rule{0.400pt}{0.482pt}}
\put(906.0,609.0){\rule[-0.200pt]{0.400pt}{0.964pt}}
\put(906.67,622){\rule{0.400pt}{0.964pt}}
\multiput(906.17,622.00)(1.000,2.000){2}{\rule{0.400pt}{0.482pt}}
\put(907.67,626){\rule{0.400pt}{0.964pt}}
\multiput(907.17,626.00)(1.000,2.000){2}{\rule{0.400pt}{0.482pt}}
\put(907.0,617.0){\rule[-0.200pt]{0.400pt}{1.204pt}}
\put(908.67,635){\rule{0.400pt}{0.964pt}}
\multiput(908.17,635.00)(1.000,2.000){2}{\rule{0.400pt}{0.482pt}}
\put(909.0,630.0){\rule[-0.200pt]{0.400pt}{1.204pt}}
\put(909.67,643){\rule{0.400pt}{0.964pt}}
\multiput(909.17,643.00)(1.000,2.000){2}{\rule{0.400pt}{0.482pt}}
\put(910.0,639.0){\rule[-0.200pt]{0.400pt}{0.964pt}}
\put(910.67,652){\rule{0.400pt}{0.964pt}}
\multiput(910.17,652.00)(1.000,2.000){2}{\rule{0.400pt}{0.482pt}}
\put(911.0,647.0){\rule[-0.200pt]{0.400pt}{1.204pt}}
\put(911.67,660){\rule{0.400pt}{1.204pt}}
\multiput(911.17,660.00)(1.000,2.500){2}{\rule{0.400pt}{0.602pt}}
\put(912.0,656.0){\rule[-0.200pt]{0.400pt}{0.964pt}}
\put(912.67,669){\rule{0.400pt}{0.964pt}}
\multiput(912.17,669.00)(1.000,2.000){2}{\rule{0.400pt}{0.482pt}}
\put(913.67,673){\rule{0.400pt}{1.204pt}}
\multiput(913.17,673.00)(1.000,2.500){2}{\rule{0.400pt}{0.602pt}}
\put(913.0,665.0){\rule[-0.200pt]{0.400pt}{0.964pt}}
\put(914.67,682){\rule{0.400pt}{0.964pt}}
\multiput(914.17,682.00)(1.000,2.000){2}{\rule{0.400pt}{0.482pt}}
\put(915.0,678.0){\rule[-0.200pt]{0.400pt}{0.964pt}}
\put(915.67,691){\rule{0.400pt}{0.964pt}}
\multiput(915.17,691.00)(1.000,2.000){2}{\rule{0.400pt}{0.482pt}}
\put(916.0,686.0){\rule[-0.200pt]{0.400pt}{1.204pt}}
\put(916.67,699){\rule{0.400pt}{0.964pt}}
\multiput(916.17,699.00)(1.000,2.000){2}{\rule{0.400pt}{0.482pt}}
\put(917.0,695.0){\rule[-0.200pt]{0.400pt}{0.964pt}}
\put(917.67,708){\rule{0.400pt}{0.964pt}}
\multiput(917.17,708.00)(1.000,2.000){2}{\rule{0.400pt}{0.482pt}}
\put(918.0,703.0){\rule[-0.200pt]{0.400pt}{1.204pt}}
\put(918.67,716){\rule{0.400pt}{1.204pt}}
\multiput(918.17,716.00)(1.000,2.500){2}{\rule{0.400pt}{0.602pt}}
\put(919.67,721){\rule{0.400pt}{0.964pt}}
\multiput(919.17,721.00)(1.000,2.000){2}{\rule{0.400pt}{0.482pt}}
\put(919.0,712.0){\rule[-0.200pt]{0.400pt}{0.964pt}}
\put(920.67,729){\rule{0.400pt}{1.204pt}}
\multiput(920.17,729.00)(1.000,2.500){2}{\rule{0.400pt}{0.602pt}}
\put(921.0,725.0){\rule[-0.200pt]{0.400pt}{0.964pt}}
\put(921.67,738){\rule{0.400pt}{0.964pt}}
\multiput(921.17,738.00)(1.000,2.000){2}{\rule{0.400pt}{0.482pt}}
\put(922.0,734.0){\rule[-0.200pt]{0.400pt}{0.964pt}}
\put(922.67,747){\rule{0.400pt}{0.964pt}}
\multiput(922.17,747.00)(1.000,2.000){2}{\rule{0.400pt}{0.482pt}}
\put(923.0,742.0){\rule[-0.200pt]{0.400pt}{1.204pt}}
\put(923.67,755){\rule{0.400pt}{0.964pt}}
\multiput(923.17,755.00)(1.000,2.000){2}{\rule{0.400pt}{0.482pt}}
\put(924.0,751.0){\rule[-0.200pt]{0.400pt}{0.964pt}}
\put(924.67,764){\rule{0.400pt}{0.964pt}}
\multiput(924.17,764.00)(1.000,2.000){2}{\rule{0.400pt}{0.482pt}}
\put(925.67,768){\rule{0.400pt}{0.964pt}}
\multiput(925.17,768.00)(1.000,2.000){2}{\rule{0.400pt}{0.482pt}}
\put(925.0,759.0){\rule[-0.200pt]{0.400pt}{1.204pt}}
\put(926.67,777){\rule{0.400pt}{0.964pt}}
\multiput(926.17,777.00)(1.000,2.000){2}{\rule{0.400pt}{0.482pt}}
\put(927.0,772.0){\rule[-0.200pt]{0.400pt}{1.204pt}}
\put(927.67,785){\rule{0.400pt}{1.204pt}}
\multiput(927.17,785.00)(1.000,2.500){2}{\rule{0.400pt}{0.602pt}}
\put(928.0,781.0){\rule[-0.200pt]{0.400pt}{0.964pt}}
\put(928.67,794){\rule{0.400pt}{0.964pt}}
\multiput(928.17,794.00)(1.000,2.000){2}{\rule{0.400pt}{0.482pt}}
\put(929.0,790.0){\rule[-0.200pt]{0.400pt}{0.964pt}}
\put(929.67,803){\rule{0.400pt}{0.964pt}}
\multiput(929.17,803.00)(1.000,2.000){2}{\rule{0.400pt}{0.482pt}}
\put(930.67,807){\rule{0.400pt}{0.964pt}}
\multiput(930.17,807.00)(1.000,2.000){2}{\rule{0.400pt}{0.482pt}}
\put(930.0,798.0){\rule[-0.200pt]{0.400pt}{1.204pt}}
\put(931.67,815){\rule{0.400pt}{1.204pt}}
\multiput(931.17,815.00)(1.000,2.500){2}{\rule{0.400pt}{0.602pt}}
\put(932.0,811.0){\rule[-0.200pt]{0.400pt}{0.964pt}}
\put(932.67,824){\rule{0.400pt}{0.964pt}}
\multiput(932.17,824.00)(1.000,2.000){2}{\rule{0.400pt}{0.482pt}}
\put(933.0,820.0){\rule[-0.200pt]{0.400pt}{0.964pt}}
\put(933.67,833){\rule{0.400pt}{0.964pt}}
\multiput(933.17,833.00)(1.000,2.000){2}{\rule{0.400pt}{0.482pt}}
\put(934.0,828.0){\rule[-0.200pt]{0.400pt}{1.204pt}}
\put(934.67,828){\rule{0.400pt}{1.204pt}}
\multiput(934.17,830.50)(1.000,-2.500){2}{\rule{0.400pt}{0.602pt}}
\put(935.0,833.0){\rule[-0.200pt]{0.400pt}{0.964pt}}
\put(935.67,820){\rule{0.400pt}{0.964pt}}
\multiput(935.17,822.00)(1.000,-2.000){2}{\rule{0.400pt}{0.482pt}}
\put(936.67,815){\rule{0.400pt}{1.204pt}}
\multiput(936.17,817.50)(1.000,-2.500){2}{\rule{0.400pt}{0.602pt}}
\put(936.0,824.0){\rule[-0.200pt]{0.400pt}{0.964pt}}
\put(937.67,807){\rule{0.400pt}{0.964pt}}
\multiput(937.17,809.00)(1.000,-2.000){2}{\rule{0.400pt}{0.482pt}}
\put(938.0,811.0){\rule[-0.200pt]{0.400pt}{0.964pt}}
\put(938.67,798){\rule{0.400pt}{1.204pt}}
\multiput(938.17,800.50)(1.000,-2.500){2}{\rule{0.400pt}{0.602pt}}
\put(939.0,803.0){\rule[-0.200pt]{0.400pt}{0.964pt}}
\put(939.67,790){\rule{0.400pt}{0.964pt}}
\multiput(939.17,792.00)(1.000,-2.000){2}{\rule{0.400pt}{0.482pt}}
\put(940.0,794.0){\rule[-0.200pt]{0.400pt}{0.964pt}}
\put(940.67,781){\rule{0.400pt}{0.964pt}}
\multiput(940.17,783.00)(1.000,-2.000){2}{\rule{0.400pt}{0.482pt}}
\put(941.0,785.0){\rule[-0.200pt]{0.400pt}{1.204pt}}
\put(941.67,772){\rule{0.400pt}{1.204pt}}
\multiput(941.17,774.50)(1.000,-2.500){2}{\rule{0.400pt}{0.602pt}}
\put(942.67,768){\rule{0.400pt}{0.964pt}}
\multiput(942.17,770.00)(1.000,-2.000){2}{\rule{0.400pt}{0.482pt}}
\put(942.0,777.0){\rule[-0.200pt]{0.400pt}{0.964pt}}
\put(943.67,759){\rule{0.400pt}{1.204pt}}
\multiput(943.17,761.50)(1.000,-2.500){2}{\rule{0.400pt}{0.602pt}}
\put(944.0,764.0){\rule[-0.200pt]{0.400pt}{0.964pt}}
\put(944.67,751){\rule{0.400pt}{0.964pt}}
\multiput(944.17,753.00)(1.000,-2.000){2}{\rule{0.400pt}{0.482pt}}
\put(945.0,755.0){\rule[-0.200pt]{0.400pt}{0.964pt}}
\put(945.67,742){\rule{0.400pt}{1.204pt}}
\multiput(945.17,744.50)(1.000,-2.500){2}{\rule{0.400pt}{0.602pt}}
\put(946.0,747.0){\rule[-0.200pt]{0.400pt}{0.964pt}}
\put(946.67,734){\rule{0.400pt}{0.964pt}}
\multiput(946.17,736.00)(1.000,-2.000){2}{\rule{0.400pt}{0.482pt}}
\put(947.67,729){\rule{0.400pt}{1.204pt}}
\multiput(947.17,731.50)(1.000,-2.500){2}{\rule{0.400pt}{0.602pt}}
\put(947.0,738.0){\rule[-0.200pt]{0.400pt}{0.964pt}}
\put(948.67,721){\rule{0.400pt}{0.964pt}}
\multiput(948.17,723.00)(1.000,-2.000){2}{\rule{0.400pt}{0.482pt}}
\put(949.0,725.0){\rule[-0.200pt]{0.400pt}{0.964pt}}
\put(949.67,712){\rule{0.400pt}{0.964pt}}
\multiput(949.17,714.00)(1.000,-2.000){2}{\rule{0.400pt}{0.482pt}}
\put(950.0,716.0){\rule[-0.200pt]{0.400pt}{1.204pt}}
\put(950.67,703){\rule{0.400pt}{1.204pt}}
\multiput(950.17,705.50)(1.000,-2.500){2}{\rule{0.400pt}{0.602pt}}
\put(951.0,708.0){\rule[-0.200pt]{0.400pt}{0.964pt}}
\put(951.67,695){\rule{0.400pt}{0.964pt}}
\multiput(951.17,697.00)(1.000,-2.000){2}{\rule{0.400pt}{0.482pt}}
\put(952.0,699.0){\rule[-0.200pt]{0.400pt}{0.964pt}}
\put(952.67,686){\rule{0.400pt}{1.204pt}}
\multiput(952.17,688.50)(1.000,-2.500){2}{\rule{0.400pt}{0.602pt}}
\put(953.67,682){\rule{0.400pt}{0.964pt}}
\multiput(953.17,684.00)(1.000,-2.000){2}{\rule{0.400pt}{0.482pt}}
\put(953.0,691.0){\rule[-0.200pt]{0.400pt}{0.964pt}}
\put(954.67,673){\rule{0.400pt}{1.204pt}}
\multiput(954.17,675.50)(1.000,-2.500){2}{\rule{0.400pt}{0.602pt}}
\put(955.0,678.0){\rule[-0.200pt]{0.400pt}{0.964pt}}
\put(955.67,665){\rule{0.400pt}{0.964pt}}
\multiput(955.17,667.00)(1.000,-2.000){2}{\rule{0.400pt}{0.482pt}}
\put(956.0,669.0){\rule[-0.200pt]{0.400pt}{0.964pt}}
\put(956.67,656){\rule{0.400pt}{0.964pt}}
\multiput(956.17,658.00)(1.000,-2.000){2}{\rule{0.400pt}{0.482pt}}
\put(957.0,660.0){\rule[-0.200pt]{0.400pt}{1.204pt}}
\put(957.67,647){\rule{0.400pt}{1.204pt}}
\multiput(957.17,649.50)(1.000,-2.500){2}{\rule{0.400pt}{0.602pt}}
\put(958.0,652.0){\rule[-0.200pt]{0.400pt}{0.964pt}}
\put(958.67,639){\rule{0.400pt}{0.964pt}}
\multiput(958.17,641.00)(1.000,-2.000){2}{\rule{0.400pt}{0.482pt}}
\put(959.67,635){\rule{0.400pt}{0.964pt}}
\multiput(959.17,637.00)(1.000,-2.000){2}{\rule{0.400pt}{0.482pt}}
\put(959.0,643.0){\rule[-0.200pt]{0.400pt}{0.964pt}}
\put(960.67,626){\rule{0.400pt}{0.964pt}}
\multiput(960.17,628.00)(1.000,-2.000){2}{\rule{0.400pt}{0.482pt}}
\put(961.0,630.0){\rule[-0.200pt]{0.400pt}{1.204pt}}
\put(961.67,617){\rule{0.400pt}{1.204pt}}
\multiput(961.17,619.50)(1.000,-2.500){2}{\rule{0.400pt}{0.602pt}}
\put(962.0,622.0){\rule[-0.200pt]{0.400pt}{0.964pt}}
\put(962.67,609){\rule{0.400pt}{0.964pt}}
\multiput(962.17,611.00)(1.000,-2.000){2}{\rule{0.400pt}{0.482pt}}
\put(963.0,613.0){\rule[-0.200pt]{0.400pt}{0.964pt}}
\put(963.67,600){\rule{0.400pt}{0.964pt}}
\multiput(963.17,602.00)(1.000,-2.000){2}{\rule{0.400pt}{0.482pt}}
\put(964.0,604.0){\rule[-0.200pt]{0.400pt}{1.204pt}}
\put(964.67,591){\rule{0.400pt}{1.204pt}}
\multiput(964.17,593.50)(1.000,-2.500){2}{\rule{0.400pt}{0.602pt}}
\put(965.67,587){\rule{0.400pt}{0.964pt}}
\multiput(965.17,589.00)(1.000,-2.000){2}{\rule{0.400pt}{0.482pt}}
\put(965.0,596.0){\rule[-0.200pt]{0.400pt}{0.964pt}}
\put(966.67,579){\rule{0.400pt}{0.964pt}}
\multiput(966.17,581.00)(1.000,-2.000){2}{\rule{0.400pt}{0.482pt}}
\put(967.0,583.0){\rule[-0.200pt]{0.400pt}{0.964pt}}
\put(967.67,570){\rule{0.400pt}{0.964pt}}
\multiput(967.17,572.00)(1.000,-2.000){2}{\rule{0.400pt}{0.482pt}}
\put(968.0,574.0){\rule[-0.200pt]{0.400pt}{1.204pt}}
\put(968.67,561){\rule{0.400pt}{1.204pt}}
\multiput(968.17,563.50)(1.000,-2.500){2}{\rule{0.400pt}{0.602pt}}
\put(969.0,566.0){\rule[-0.200pt]{0.400pt}{0.964pt}}
\put(969.67,553){\rule{0.400pt}{0.964pt}}
\multiput(969.17,555.00)(1.000,-2.000){2}{\rule{0.400pt}{0.482pt}}
\put(970.67,548){\rule{0.400pt}{1.204pt}}
\multiput(970.17,550.50)(1.000,-2.500){2}{\rule{0.400pt}{0.602pt}}
\put(970.0,557.0){\rule[-0.200pt]{0.400pt}{0.964pt}}
\put(971.67,540){\rule{0.400pt}{0.964pt}}
\multiput(971.17,542.00)(1.000,-2.000){2}{\rule{0.400pt}{0.482pt}}
\put(972.0,544.0){\rule[-0.200pt]{0.400pt}{0.964pt}}
\put(972.67,531){\rule{0.400pt}{0.964pt}}
\multiput(972.17,533.00)(1.000,-2.000){2}{\rule{0.400pt}{0.482pt}}
\put(973.0,535.0){\rule[-0.200pt]{0.400pt}{1.204pt}}
\put(973.67,523){\rule{0.400pt}{0.964pt}}
\multiput(973.17,525.00)(1.000,-2.000){2}{\rule{0.400pt}{0.482pt}}
\put(974.0,527.0){\rule[-0.200pt]{0.400pt}{0.964pt}}
\put(974.67,514){\rule{0.400pt}{0.964pt}}
\multiput(974.17,516.00)(1.000,-2.000){2}{\rule{0.400pt}{0.482pt}}
\put(975.0,518.0){\rule[-0.200pt]{0.400pt}{1.204pt}}
\put(975.67,505){\rule{0.400pt}{1.204pt}}
\multiput(975.17,507.50)(1.000,-2.500){2}{\rule{0.400pt}{0.602pt}}
\put(976.67,501){\rule{0.400pt}{0.964pt}}
\multiput(976.17,503.00)(1.000,-2.000){2}{\rule{0.400pt}{0.482pt}}
\put(976.0,510.0){\rule[-0.200pt]{0.400pt}{0.964pt}}
\put(977.67,492){\rule{0.400pt}{1.204pt}}
\multiput(977.17,494.50)(1.000,-2.500){2}{\rule{0.400pt}{0.602pt}}
\put(978.0,497.0){\rule[-0.200pt]{0.400pt}{0.964pt}}
\put(978.67,484){\rule{0.400pt}{0.964pt}}
\multiput(978.17,486.00)(1.000,-2.000){2}{\rule{0.400pt}{0.482pt}}
\put(979.0,488.0){\rule[-0.200pt]{0.400pt}{0.964pt}}
\put(979.67,475){\rule{0.400pt}{0.964pt}}
\multiput(979.17,477.00)(1.000,-2.000){2}{\rule{0.400pt}{0.482pt}}
\put(980.0,479.0){\rule[-0.200pt]{0.400pt}{1.204pt}}
\put(980.67,467){\rule{0.400pt}{0.964pt}}
\multiput(980.17,469.00)(1.000,-2.000){2}{\rule{0.400pt}{0.482pt}}
\put(981.0,471.0){\rule[-0.200pt]{0.400pt}{0.964pt}}
\put(981.67,458){\rule{0.400pt}{0.964pt}}
\multiput(981.17,460.00)(1.000,-2.000){2}{\rule{0.400pt}{0.482pt}}
\put(982.67,454){\rule{0.400pt}{0.964pt}}
\multiput(982.17,456.00)(1.000,-2.000){2}{\rule{0.400pt}{0.482pt}}
\put(982.0,462.0){\rule[-0.200pt]{0.400pt}{1.204pt}}
\put(983.67,445){\rule{0.400pt}{0.964pt}}
\multiput(983.17,447.00)(1.000,-2.000){2}{\rule{0.400pt}{0.482pt}}
\put(984.0,449.0){\rule[-0.200pt]{0.400pt}{1.204pt}}
\put(984.67,436){\rule{0.400pt}{1.204pt}}
\multiput(984.17,438.50)(1.000,-2.500){2}{\rule{0.400pt}{0.602pt}}
\put(985.0,441.0){\rule[-0.200pt]{0.400pt}{0.964pt}}
\put(985.67,428){\rule{0.400pt}{0.964pt}}
\multiput(985.17,430.00)(1.000,-2.000){2}{\rule{0.400pt}{0.482pt}}
\put(986.0,432.0){\rule[-0.200pt]{0.400pt}{0.964pt}}
\put(986.67,419){\rule{0.400pt}{0.964pt}}
\multiput(986.17,421.00)(1.000,-2.000){2}{\rule{0.400pt}{0.482pt}}
\put(987.0,423.0){\rule[-0.200pt]{0.400pt}{1.204pt}}
\put(987.67,411){\rule{0.400pt}{0.964pt}}
\multiput(987.17,413.00)(1.000,-2.000){2}{\rule{0.400pt}{0.482pt}}
\put(988.67,406){\rule{0.400pt}{1.204pt}}
\multiput(988.17,408.50)(1.000,-2.500){2}{\rule{0.400pt}{0.602pt}}
\put(988.0,415.0){\rule[-0.200pt]{0.400pt}{0.964pt}}
\put(989.67,398){\rule{0.400pt}{0.964pt}}
\multiput(989.17,400.00)(1.000,-2.000){2}{\rule{0.400pt}{0.482pt}}
\put(990.0,402.0){\rule[-0.200pt]{0.400pt}{0.964pt}}
\put(990.67,389){\rule{0.400pt}{0.964pt}}
\multiput(990.17,391.00)(1.000,-2.000){2}{\rule{0.400pt}{0.482pt}}
\put(991.0,393.0){\rule[-0.200pt]{0.400pt}{1.204pt}}
\put(991.67,380){\rule{0.400pt}{1.204pt}}
\multiput(991.17,382.50)(1.000,-2.500){2}{\rule{0.400pt}{0.602pt}}
\put(992.0,385.0){\rule[-0.200pt]{0.400pt}{0.964pt}}
\put(992.67,372){\rule{0.400pt}{0.964pt}}
\multiput(992.17,374.00)(1.000,-2.000){2}{\rule{0.400pt}{0.482pt}}
\put(993.67,367){\rule{0.400pt}{1.204pt}}
\multiput(993.17,369.50)(1.000,-2.500){2}{\rule{0.400pt}{0.602pt}}
\put(993.0,376.0){\rule[-0.200pt]{0.400pt}{0.964pt}}
\put(994.67,359){\rule{0.400pt}{0.964pt}}
\multiput(994.17,361.00)(1.000,-2.000){2}{\rule{0.400pt}{0.482pt}}
\put(995.0,363.0){\rule[-0.200pt]{0.400pt}{0.964pt}}
\put(995.67,350){\rule{0.400pt}{1.204pt}}
\multiput(995.17,352.50)(1.000,-2.500){2}{\rule{0.400pt}{0.602pt}}
\put(996.0,355.0){\rule[-0.200pt]{0.400pt}{0.964pt}}
\put(996.67,342){\rule{0.400pt}{0.964pt}}
\multiput(996.17,344.00)(1.000,-2.000){2}{\rule{0.400pt}{0.482pt}}
\put(997.0,346.0){\rule[-0.200pt]{0.400pt}{0.964pt}}
\put(997.67,333){\rule{0.400pt}{0.964pt}}
\multiput(997.17,335.00)(1.000,-2.000){2}{\rule{0.400pt}{0.482pt}}
\put(998.0,337.0){\rule[-0.200pt]{0.400pt}{1.204pt}}
\put(998.67,324){\rule{0.400pt}{1.204pt}}
\multiput(998.17,326.50)(1.000,-2.500){2}{\rule{0.400pt}{0.602pt}}
\put(999.67,320){\rule{0.400pt}{0.964pt}}
\multiput(999.17,322.00)(1.000,-2.000){2}{\rule{0.400pt}{0.482pt}}
\put(999.0,329.0){\rule[-0.200pt]{0.400pt}{0.964pt}}
\put(1000.67,311){\rule{0.400pt}{1.204pt}}
\multiput(1000.17,313.50)(1.000,-2.500){2}{\rule{0.400pt}{0.602pt}}
\put(1001.0,316.0){\rule[-0.200pt]{0.400pt}{0.964pt}}
\put(1001.67,303){\rule{0.400pt}{0.964pt}}
\multiput(1001.17,305.00)(1.000,-2.000){2}{\rule{0.400pt}{0.482pt}}
\put(1002.0,307.0){\rule[-0.200pt]{0.400pt}{0.964pt}}
\put(1002.67,294){\rule{0.400pt}{1.204pt}}
\multiput(1002.17,296.50)(1.000,-2.500){2}{\rule{0.400pt}{0.602pt}}
\put(1003.0,299.0){\rule[-0.200pt]{0.400pt}{0.964pt}}
\put(1003.67,286){\rule{0.400pt}{0.964pt}}
\multiput(1003.17,288.00)(1.000,-2.000){2}{\rule{0.400pt}{0.482pt}}
\put(1004.0,290.0){\rule[-0.200pt]{0.400pt}{0.964pt}}
\put(1004.67,277){\rule{0.400pt}{0.964pt}}
\multiput(1004.17,279.00)(1.000,-2.000){2}{\rule{0.400pt}{0.482pt}}
\put(1005.67,273){\rule{0.400pt}{0.964pt}}
\multiput(1005.17,275.00)(1.000,-2.000){2}{\rule{0.400pt}{0.482pt}}
\put(1005.0,281.0){\rule[-0.200pt]{0.400pt}{1.204pt}}
\put(1006.67,264){\rule{0.400pt}{0.964pt}}
\multiput(1006.17,266.00)(1.000,-2.000){2}{\rule{0.400pt}{0.482pt}}
\put(1007.0,268.0){\rule[-0.200pt]{0.400pt}{1.204pt}}
\put(1007.67,255){\rule{0.400pt}{1.204pt}}
\multiput(1007.17,257.50)(1.000,-2.500){2}{\rule{0.400pt}{0.602pt}}
\put(1008.0,260.0){\rule[-0.200pt]{0.400pt}{0.964pt}}
\put(1008.67,247){\rule{0.400pt}{0.964pt}}
\multiput(1008.17,249.00)(1.000,-2.000){2}{\rule{0.400pt}{0.482pt}}
\put(1009.0,251.0){\rule[-0.200pt]{0.400pt}{0.964pt}}
\put(1009.67,238){\rule{0.400pt}{1.204pt}}
\multiput(1009.17,240.50)(1.000,-2.500){2}{\rule{0.400pt}{0.602pt}}
\put(1010.0,243.0){\rule[-0.200pt]{0.400pt}{0.964pt}}
\put(1010.67,230){\rule{0.400pt}{0.964pt}}
\multiput(1010.17,232.00)(1.000,-2.000){2}{\rule{0.400pt}{0.482pt}}
\put(1011.67,225){\rule{0.400pt}{1.204pt}}
\multiput(1011.17,227.50)(1.000,-2.500){2}{\rule{0.400pt}{0.602pt}}
\put(1011.0,234.0){\rule[-0.200pt]{0.400pt}{0.964pt}}
\put(1012.67,217){\rule{0.400pt}{0.964pt}}
\multiput(1012.17,219.00)(1.000,-2.000){2}{\rule{0.400pt}{0.482pt}}
\put(1013.0,221.0){\rule[-0.200pt]{0.400pt}{0.964pt}}
\put(1013.67,208){\rule{0.400pt}{0.964pt}}
\multiput(1013.17,210.00)(1.000,-2.000){2}{\rule{0.400pt}{0.482pt}}
\put(1014.0,212.0){\rule[-0.200pt]{0.400pt}{1.204pt}}
\put(1014.67,199){\rule{0.400pt}{1.204pt}}
\multiput(1014.17,201.50)(1.000,-2.500){2}{\rule{0.400pt}{0.602pt}}
\put(1015.0,204.0){\rule[-0.200pt]{0.400pt}{0.964pt}}
\put(1015.67,191){\rule{0.400pt}{0.964pt}}
\multiput(1015.17,193.00)(1.000,-2.000){2}{\rule{0.400pt}{0.482pt}}
\put(1016.67,187){\rule{0.400pt}{0.964pt}}
\multiput(1016.17,189.00)(1.000,-2.000){2}{\rule{0.400pt}{0.482pt}}
\put(1016.0,195.0){\rule[-0.200pt]{0.400pt}{0.964pt}}
\put(1017.67,178){\rule{0.400pt}{0.964pt}}
\multiput(1017.17,180.00)(1.000,-2.000){2}{\rule{0.400pt}{0.482pt}}
\put(1018.0,182.0){\rule[-0.200pt]{0.400pt}{1.204pt}}
\put(1018.67,169){\rule{0.400pt}{1.204pt}}
\multiput(1018.17,171.50)(1.000,-2.500){2}{\rule{0.400pt}{0.602pt}}
\put(1019.0,174.0){\rule[-0.200pt]{0.400pt}{0.964pt}}
\put(1019.67,161){\rule{0.400pt}{0.964pt}}
\multiput(1019.17,163.00)(1.000,-2.000){2}{\rule{0.400pt}{0.482pt}}
\put(1020.0,165.0){\rule[-0.200pt]{0.400pt}{0.964pt}}
\put(1020.67,152){\rule{0.400pt}{0.964pt}}
\multiput(1020.17,154.00)(1.000,-2.000){2}{\rule{0.400pt}{0.482pt}}
\put(1021.0,156.0){\rule[-0.200pt]{0.400pt}{1.204pt}}
\put(1021.67,143){\rule{0.400pt}{1.204pt}}
\multiput(1021.17,145.50)(1.000,-2.500){2}{\rule{0.400pt}{0.602pt}}
\put(1022.67,139){\rule{0.400pt}{0.964pt}}
\multiput(1022.17,141.00)(1.000,-2.000){2}{\rule{0.400pt}{0.482pt}}
\put(1022.0,148.0){\rule[-0.200pt]{0.400pt}{0.964pt}}
\put(1023.67,131){\rule{0.400pt}{0.964pt}}
\multiput(1023.17,133.00)(1.000,-2.000){2}{\rule{0.400pt}{0.482pt}}
\put(1024.0,135.0){\rule[-0.200pt]{0.400pt}{0.964pt}}
\put(1024.67,122){\rule{0.400pt}{0.964pt}}
\multiput(1024.17,124.00)(1.000,-2.000){2}{\rule{0.400pt}{0.482pt}}
\put(1025.0,126.0){\rule[-0.200pt]{0.400pt}{1.204pt}}
\put(1025.67,113){\rule{0.400pt}{1.204pt}}
\multiput(1025.17,115.50)(1.000,-2.500){2}{\rule{0.400pt}{0.602pt}}
\put(1026.0,118.0){\rule[-0.200pt]{0.400pt}{0.964pt}}
\put(1026.67,105){\rule{0.400pt}{0.964pt}}
\multiput(1026.17,107.00)(1.000,-2.000){2}{\rule{0.400pt}{0.482pt}}
\put(1027.0,109.0){\rule[-0.200pt]{0.400pt}{0.964pt}}
\put(1027.67,109){\rule{0.400pt}{0.964pt}}
\multiput(1027.17,109.00)(1.000,2.000){2}{\rule{0.400pt}{0.482pt}}
\put(1028.67,113){\rule{0.400pt}{1.204pt}}
\multiput(1028.17,113.00)(1.000,2.500){2}{\rule{0.400pt}{0.602pt}}
\put(1028.0,105.0){\rule[-0.200pt]{0.400pt}{0.964pt}}
\put(1029.67,122){\rule{0.400pt}{0.964pt}}
\multiput(1029.17,122.00)(1.000,2.000){2}{\rule{0.400pt}{0.482pt}}
\put(1030.0,118.0){\rule[-0.200pt]{0.400pt}{0.964pt}}
\put(1030.67,131){\rule{0.400pt}{0.964pt}}
\multiput(1030.17,131.00)(1.000,2.000){2}{\rule{0.400pt}{0.482pt}}
\put(1031.0,126.0){\rule[-0.200pt]{0.400pt}{1.204pt}}
\put(1031.67,139){\rule{0.400pt}{0.964pt}}
\multiput(1031.17,139.00)(1.000,2.000){2}{\rule{0.400pt}{0.482pt}}
\put(1032.0,135.0){\rule[-0.200pt]{0.400pt}{0.964pt}}
\put(1032.67,148){\rule{0.400pt}{0.964pt}}
\multiput(1032.17,148.00)(1.000,2.000){2}{\rule{0.400pt}{0.482pt}}
\put(1033.0,143.0){\rule[-0.200pt]{0.400pt}{1.204pt}}
\put(1033.67,156){\rule{0.400pt}{1.204pt}}
\multiput(1033.17,156.00)(1.000,2.500){2}{\rule{0.400pt}{0.602pt}}
\put(1034.67,161){\rule{0.400pt}{0.964pt}}
\multiput(1034.17,161.00)(1.000,2.000){2}{\rule{0.400pt}{0.482pt}}
\put(1034.0,152.0){\rule[-0.200pt]{0.400pt}{0.964pt}}
\put(1035.67,169){\rule{0.400pt}{1.204pt}}
\multiput(1035.17,169.00)(1.000,2.500){2}{\rule{0.400pt}{0.602pt}}
\put(1036.0,165.0){\rule[-0.200pt]{0.400pt}{0.964pt}}
\put(1036.67,178){\rule{0.400pt}{0.964pt}}
\multiput(1036.17,178.00)(1.000,2.000){2}{\rule{0.400pt}{0.482pt}}
\put(1037.0,174.0){\rule[-0.200pt]{0.400pt}{0.964pt}}
\put(1037.67,187){\rule{0.400pt}{0.964pt}}
\multiput(1037.17,187.00)(1.000,2.000){2}{\rule{0.400pt}{0.482pt}}
\put(1038.0,182.0){\rule[-0.200pt]{0.400pt}{1.204pt}}
\put(1038.67,195){\rule{0.400pt}{0.964pt}}
\multiput(1038.17,195.00)(1.000,2.000){2}{\rule{0.400pt}{0.482pt}}
\put(1039.67,199){\rule{0.400pt}{1.204pt}}
\multiput(1039.17,199.00)(1.000,2.500){2}{\rule{0.400pt}{0.602pt}}
\put(1039.0,191.0){\rule[-0.200pt]{0.400pt}{0.964pt}}
\put(1040.67,208){\rule{0.400pt}{0.964pt}}
\multiput(1040.17,208.00)(1.000,2.000){2}{\rule{0.400pt}{0.482pt}}
\put(1041.0,204.0){\rule[-0.200pt]{0.400pt}{0.964pt}}
\put(1041.67,217){\rule{0.400pt}{0.964pt}}
\multiput(1041.17,217.00)(1.000,2.000){2}{\rule{0.400pt}{0.482pt}}
\put(1042.0,212.0){\rule[-0.200pt]{0.400pt}{1.204pt}}
\put(1042.67,225){\rule{0.400pt}{1.204pt}}
\multiput(1042.17,225.00)(1.000,2.500){2}{\rule{0.400pt}{0.602pt}}
\put(1043.0,221.0){\rule[-0.200pt]{0.400pt}{0.964pt}}
\put(1043.67,234){\rule{0.400pt}{0.964pt}}
\multiput(1043.17,234.00)(1.000,2.000){2}{\rule{0.400pt}{0.482pt}}
\put(1044.0,230.0){\rule[-0.200pt]{0.400pt}{0.964pt}}
\put(1044.67,243){\rule{0.400pt}{0.964pt}}
\multiput(1044.17,243.00)(1.000,2.000){2}{\rule{0.400pt}{0.482pt}}
\put(1045.67,247){\rule{0.400pt}{0.964pt}}
\multiput(1045.17,247.00)(1.000,2.000){2}{\rule{0.400pt}{0.482pt}}
\put(1045.0,238.0){\rule[-0.200pt]{0.400pt}{1.204pt}}
\put(1046.67,255){\rule{0.400pt}{1.204pt}}
\multiput(1046.17,255.00)(1.000,2.500){2}{\rule{0.400pt}{0.602pt}}
\put(1047.0,251.0){\rule[-0.200pt]{0.400pt}{0.964pt}}
\put(1047.67,264){\rule{0.400pt}{0.964pt}}
\multiput(1047.17,264.00)(1.000,2.000){2}{\rule{0.400pt}{0.482pt}}
\put(1048.0,260.0){\rule[-0.200pt]{0.400pt}{0.964pt}}
\put(1048.67,273){\rule{0.400pt}{0.964pt}}
\multiput(1048.17,273.00)(1.000,2.000){2}{\rule{0.400pt}{0.482pt}}
\put(1049.0,268.0){\rule[-0.200pt]{0.400pt}{1.204pt}}
\put(1049.67,281){\rule{0.400pt}{1.204pt}}
\multiput(1049.17,281.00)(1.000,2.500){2}{\rule{0.400pt}{0.602pt}}
\put(1050.0,277.0){\rule[-0.200pt]{0.400pt}{0.964pt}}
\put(1050.67,290){\rule{0.400pt}{0.964pt}}
\multiput(1050.17,290.00)(1.000,2.000){2}{\rule{0.400pt}{0.482pt}}
\put(1051.67,294){\rule{0.400pt}{1.204pt}}
\multiput(1051.17,294.00)(1.000,2.500){2}{\rule{0.400pt}{0.602pt}}
\put(1051.0,286.0){\rule[-0.200pt]{0.400pt}{0.964pt}}
\put(1052.67,303){\rule{0.400pt}{0.964pt}}
\multiput(1052.17,303.00)(1.000,2.000){2}{\rule{0.400pt}{0.482pt}}
\put(1053.0,299.0){\rule[-0.200pt]{0.400pt}{0.964pt}}
\put(1053.67,311){\rule{0.400pt}{1.204pt}}
\multiput(1053.17,311.00)(1.000,2.500){2}{\rule{0.400pt}{0.602pt}}
\put(1054.0,307.0){\rule[-0.200pt]{0.400pt}{0.964pt}}
\put(1054.67,320){\rule{0.400pt}{0.964pt}}
\multiput(1054.17,320.00)(1.000,2.000){2}{\rule{0.400pt}{0.482pt}}
\put(1055.0,316.0){\rule[-0.200pt]{0.400pt}{0.964pt}}
\put(1055.67,329){\rule{0.400pt}{0.964pt}}
\multiput(1055.17,329.00)(1.000,2.000){2}{\rule{0.400pt}{0.482pt}}
\put(1056.0,324.0){\rule[-0.200pt]{0.400pt}{1.204pt}}
\put(1056.67,337){\rule{0.400pt}{1.204pt}}
\multiput(1056.17,337.00)(1.000,2.500){2}{\rule{0.400pt}{0.602pt}}
\put(1057.67,342){\rule{0.400pt}{0.964pt}}
\multiput(1057.17,342.00)(1.000,2.000){2}{\rule{0.400pt}{0.482pt}}
\put(1057.0,333.0){\rule[-0.200pt]{0.400pt}{0.964pt}}
\put(1058.67,350){\rule{0.400pt}{1.204pt}}
\multiput(1058.17,350.00)(1.000,2.500){2}{\rule{0.400pt}{0.602pt}}
\put(1059.0,346.0){\rule[-0.200pt]{0.400pt}{0.964pt}}
\put(1059.67,359){\rule{0.400pt}{0.964pt}}
\multiput(1059.17,359.00)(1.000,2.000){2}{\rule{0.400pt}{0.482pt}}
\put(1060.0,355.0){\rule[-0.200pt]{0.400pt}{0.964pt}}
\put(1060.67,367){\rule{0.400pt}{1.204pt}}
\multiput(1060.17,367.00)(1.000,2.500){2}{\rule{0.400pt}{0.602pt}}
\put(1061.0,363.0){\rule[-0.200pt]{0.400pt}{0.964pt}}
\put(1061.67,376){\rule{0.400pt}{0.964pt}}
\multiput(1061.17,376.00)(1.000,2.000){2}{\rule{0.400pt}{0.482pt}}
\put(1062.67,380){\rule{0.400pt}{1.204pt}}
\multiput(1062.17,380.00)(1.000,2.500){2}{\rule{0.400pt}{0.602pt}}
\put(1062.0,372.0){\rule[-0.200pt]{0.400pt}{0.964pt}}
\put(1063.67,389){\rule{0.400pt}{0.964pt}}
\multiput(1063.17,389.00)(1.000,2.000){2}{\rule{0.400pt}{0.482pt}}
\put(1064.0,385.0){\rule[-0.200pt]{0.400pt}{0.964pt}}
\put(1064.67,398){\rule{0.400pt}{0.964pt}}
\multiput(1064.17,398.00)(1.000,2.000){2}{\rule{0.400pt}{0.482pt}}
\put(1065.0,393.0){\rule[-0.200pt]{0.400pt}{1.204pt}}
\put(1065.67,406){\rule{0.400pt}{1.204pt}}
\multiput(1065.17,406.00)(1.000,2.500){2}{\rule{0.400pt}{0.602pt}}
\put(1066.0,402.0){\rule[-0.200pt]{0.400pt}{0.964pt}}
\put(1066.67,415){\rule{0.400pt}{0.964pt}}
\multiput(1066.17,415.00)(1.000,2.000){2}{\rule{0.400pt}{0.482pt}}
\put(1067.0,411.0){\rule[-0.200pt]{0.400pt}{0.964pt}}
\put(1067.67,423){\rule{0.400pt}{1.204pt}}
\multiput(1067.17,423.00)(1.000,2.500){2}{\rule{0.400pt}{0.602pt}}
\put(1068.67,428){\rule{0.400pt}{0.964pt}}
\multiput(1068.17,428.00)(1.000,2.000){2}{\rule{0.400pt}{0.482pt}}
\put(1068.0,419.0){\rule[-0.200pt]{0.400pt}{0.964pt}}
\put(1069.67,436){\rule{0.400pt}{1.204pt}}
\multiput(1069.17,436.00)(1.000,2.500){2}{\rule{0.400pt}{0.602pt}}
\put(1070.0,432.0){\rule[-0.200pt]{0.400pt}{0.964pt}}
\put(1070.67,445){\rule{0.400pt}{0.964pt}}
\multiput(1070.17,445.00)(1.000,2.000){2}{\rule{0.400pt}{0.482pt}}
\put(1071.0,441.0){\rule[-0.200pt]{0.400pt}{0.964pt}}
\put(1071.67,454){\rule{0.400pt}{0.964pt}}
\multiput(1071.17,454.00)(1.000,2.000){2}{\rule{0.400pt}{0.482pt}}
\put(1072.0,449.0){\rule[-0.200pt]{0.400pt}{1.204pt}}
\put(1072.67,462){\rule{0.400pt}{1.204pt}}
\multiput(1072.17,462.00)(1.000,2.500){2}{\rule{0.400pt}{0.602pt}}
\put(1073.0,458.0){\rule[-0.200pt]{0.400pt}{0.964pt}}
\put(1073.67,471){\rule{0.400pt}{0.964pt}}
\multiput(1073.17,471.00)(1.000,2.000){2}{\rule{0.400pt}{0.482pt}}
\put(1074.67,475){\rule{0.400pt}{0.964pt}}
\multiput(1074.17,475.00)(1.000,2.000){2}{\rule{0.400pt}{0.482pt}}
\put(1074.0,467.0){\rule[-0.200pt]{0.400pt}{0.964pt}}
\put(1075.67,484){\rule{0.400pt}{0.964pt}}
\multiput(1075.17,484.00)(1.000,2.000){2}{\rule{0.400pt}{0.482pt}}
\put(1076.0,479.0){\rule[-0.200pt]{0.400pt}{1.204pt}}
\put(1076.67,492){\rule{0.400pt}{1.204pt}}
\multiput(1076.17,492.00)(1.000,2.500){2}{\rule{0.400pt}{0.602pt}}
\put(1077.0,488.0){\rule[-0.200pt]{0.400pt}{0.964pt}}
\put(1077.67,501){\rule{0.400pt}{0.964pt}}
\multiput(1077.17,501.00)(1.000,2.000){2}{\rule{0.400pt}{0.482pt}}
\put(1078.0,497.0){\rule[-0.200pt]{0.400pt}{0.964pt}}
\put(1078.67,510){\rule{0.400pt}{0.964pt}}
\multiput(1078.17,510.00)(1.000,2.000){2}{\rule{0.400pt}{0.482pt}}
\put(1079.67,514){\rule{0.400pt}{0.964pt}}
\multiput(1079.17,514.00)(1.000,2.000){2}{\rule{0.400pt}{0.482pt}}
\put(1079.0,505.0){\rule[-0.200pt]{0.400pt}{1.204pt}}
\put(1080.67,523){\rule{0.400pt}{0.964pt}}
\multiput(1080.17,523.00)(1.000,2.000){2}{\rule{0.400pt}{0.482pt}}
\put(1081.0,518.0){\rule[-0.200pt]{0.400pt}{1.204pt}}
\put(1081.67,531){\rule{0.400pt}{0.964pt}}
\multiput(1081.17,531.00)(1.000,2.000){2}{\rule{0.400pt}{0.482pt}}
\put(1082.0,527.0){\rule[-0.200pt]{0.400pt}{0.964pt}}
\put(1082.67,540){\rule{0.400pt}{0.964pt}}
\multiput(1082.17,540.00)(1.000,2.000){2}{\rule{0.400pt}{0.482pt}}
\put(1083.0,535.0){\rule[-0.200pt]{0.400pt}{1.204pt}}
\put(1083.67,548){\rule{0.400pt}{1.204pt}}
\multiput(1083.17,548.00)(1.000,2.500){2}{\rule{0.400pt}{0.602pt}}
\put(1084.0,544.0){\rule[-0.200pt]{0.400pt}{0.964pt}}
\put(1084.67,557){\rule{0.400pt}{0.964pt}}
\multiput(1084.17,557.00)(1.000,2.000){2}{\rule{0.400pt}{0.482pt}}
\put(1085.67,561){\rule{0.400pt}{1.204pt}}
\multiput(1085.17,561.00)(1.000,2.500){2}{\rule{0.400pt}{0.602pt}}
\put(1085.0,553.0){\rule[-0.200pt]{0.400pt}{0.964pt}}
\put(1086.67,570){\rule{0.400pt}{0.964pt}}
\multiput(1086.17,570.00)(1.000,2.000){2}{\rule{0.400pt}{0.482pt}}
\put(1087.0,566.0){\rule[-0.200pt]{0.400pt}{0.964pt}}
\put(1087.67,579){\rule{0.400pt}{0.964pt}}
\multiput(1087.17,579.00)(1.000,2.000){2}{\rule{0.400pt}{0.482pt}}
\put(1088.0,574.0){\rule[-0.200pt]{0.400pt}{1.204pt}}
\put(1088.67,587){\rule{0.400pt}{0.964pt}}
\multiput(1088.17,587.00)(1.000,2.000){2}{\rule{0.400pt}{0.482pt}}
\put(1089.0,583.0){\rule[-0.200pt]{0.400pt}{0.964pt}}
\put(1089.67,596){\rule{0.400pt}{0.964pt}}
\multiput(1089.17,596.00)(1.000,2.000){2}{\rule{0.400pt}{0.482pt}}
\put(1090.0,591.0){\rule[-0.200pt]{0.400pt}{1.204pt}}
\put(1090.67,604){\rule{0.400pt}{1.204pt}}
\multiput(1090.17,604.00)(1.000,2.500){2}{\rule{0.400pt}{0.602pt}}
\put(1091.67,609){\rule{0.400pt}{0.964pt}}
\multiput(1091.17,609.00)(1.000,2.000){2}{\rule{0.400pt}{0.482pt}}
\put(1091.0,600.0){\rule[-0.200pt]{0.400pt}{0.964pt}}
\put(1092.67,617){\rule{0.400pt}{1.204pt}}
\multiput(1092.17,617.00)(1.000,2.500){2}{\rule{0.400pt}{0.602pt}}
\put(1093.0,613.0){\rule[-0.200pt]{0.400pt}{0.964pt}}
\put(1093.67,626){\rule{0.400pt}{0.964pt}}
\multiput(1093.17,626.00)(1.000,2.000){2}{\rule{0.400pt}{0.482pt}}
\put(1094.0,622.0){\rule[-0.200pt]{0.400pt}{0.964pt}}
\put(1094.67,635){\rule{0.400pt}{0.964pt}}
\multiput(1094.17,635.00)(1.000,2.000){2}{\rule{0.400pt}{0.482pt}}
\put(1095.0,630.0){\rule[-0.200pt]{0.400pt}{1.204pt}}
\put(1095.67,643){\rule{0.400pt}{0.964pt}}
\multiput(1095.17,643.00)(1.000,2.000){2}{\rule{0.400pt}{0.482pt}}
\put(1096.0,639.0){\rule[-0.200pt]{0.400pt}{0.964pt}}
\put(1096.67,652){\rule{0.400pt}{0.964pt}}
\multiput(1096.17,652.00)(1.000,2.000){2}{\rule{0.400pt}{0.482pt}}
\put(1097.67,656){\rule{0.400pt}{0.964pt}}
\multiput(1097.17,656.00)(1.000,2.000){2}{\rule{0.400pt}{0.482pt}}
\put(1097.0,647.0){\rule[-0.200pt]{0.400pt}{1.204pt}}
\put(1098.67,665){\rule{0.400pt}{0.964pt}}
\multiput(1098.17,665.00)(1.000,2.000){2}{\rule{0.400pt}{0.482pt}}
\put(1099.0,660.0){\rule[-0.200pt]{0.400pt}{1.204pt}}
\put(1099.67,673){\rule{0.400pt}{1.204pt}}
\multiput(1099.17,673.00)(1.000,2.500){2}{\rule{0.400pt}{0.602pt}}
\put(1100.0,669.0){\rule[-0.200pt]{0.400pt}{0.964pt}}
\put(1100.67,682){\rule{0.400pt}{0.964pt}}
\multiput(1100.17,682.00)(1.000,2.000){2}{\rule{0.400pt}{0.482pt}}
\put(1101.0,678.0){\rule[-0.200pt]{0.400pt}{0.964pt}}
\put(1101.67,691){\rule{0.400pt}{0.964pt}}
\multiput(1101.17,691.00)(1.000,2.000){2}{\rule{0.400pt}{0.482pt}}
\put(1102.67,695){\rule{0.400pt}{0.964pt}}
\multiput(1102.17,695.00)(1.000,2.000){2}{\rule{0.400pt}{0.482pt}}
\put(1102.0,686.0){\rule[-0.200pt]{0.400pt}{1.204pt}}
\put(1103.67,703){\rule{0.400pt}{1.204pt}}
\multiput(1103.17,703.00)(1.000,2.500){2}{\rule{0.400pt}{0.602pt}}
\put(1104.0,699.0){\rule[-0.200pt]{0.400pt}{0.964pt}}
\put(1104.67,712){\rule{0.400pt}{0.964pt}}
\multiput(1104.17,712.00)(1.000,2.000){2}{\rule{0.400pt}{0.482pt}}
\put(1105.0,708.0){\rule[-0.200pt]{0.400pt}{0.964pt}}
\put(1105.67,721){\rule{0.400pt}{0.964pt}}
\multiput(1105.17,721.00)(1.000,2.000){2}{\rule{0.400pt}{0.482pt}}
\put(1106.0,716.0){\rule[-0.200pt]{0.400pt}{1.204pt}}
\put(1106.67,729){\rule{0.400pt}{1.204pt}}
\multiput(1106.17,729.00)(1.000,2.500){2}{\rule{0.400pt}{0.602pt}}
\put(1107.0,725.0){\rule[-0.200pt]{0.400pt}{0.964pt}}
\put(1107.67,738){\rule{0.400pt}{0.964pt}}
\multiput(1107.17,738.00)(1.000,2.000){2}{\rule{0.400pt}{0.482pt}}
\put(1108.67,742){\rule{0.400pt}{1.204pt}}
\multiput(1108.17,742.00)(1.000,2.500){2}{\rule{0.400pt}{0.602pt}}
\put(1108.0,734.0){\rule[-0.200pt]{0.400pt}{0.964pt}}
\put(1109.67,751){\rule{0.400pt}{0.964pt}}
\multiput(1109.17,751.00)(1.000,2.000){2}{\rule{0.400pt}{0.482pt}}
\put(1110.0,747.0){\rule[-0.200pt]{0.400pt}{0.964pt}}
\put(1110.67,759){\rule{0.400pt}{1.204pt}}
\multiput(1110.17,759.00)(1.000,2.500){2}{\rule{0.400pt}{0.602pt}}
\put(1111.0,755.0){\rule[-0.200pt]{0.400pt}{0.964pt}}
\put(1111.67,768){\rule{0.400pt}{0.964pt}}
\multiput(1111.17,768.00)(1.000,2.000){2}{\rule{0.400pt}{0.482pt}}
\put(1112.0,764.0){\rule[-0.200pt]{0.400pt}{0.964pt}}
\put(1112.67,777){\rule{0.400pt}{0.964pt}}
\multiput(1112.17,777.00)(1.000,2.000){2}{\rule{0.400pt}{0.482pt}}
\put(1113.0,772.0){\rule[-0.200pt]{0.400pt}{1.204pt}}
\put(1113.67,785){\rule{0.400pt}{1.204pt}}
\multiput(1113.17,785.00)(1.000,2.500){2}{\rule{0.400pt}{0.602pt}}
\put(1114.67,790){\rule{0.400pt}{0.964pt}}
\multiput(1114.17,790.00)(1.000,2.000){2}{\rule{0.400pt}{0.482pt}}
\put(1112,794.17){\rule{0.900pt}{0.400pt}}
\multiput(1114.13,793.17)(-2.132,2.000){2}{\rule{0.450pt}{0.400pt}}
\put(1109,796.17){\rule{0.700pt}{0.400pt}}
\multiput(1110.55,795.17)(-1.547,2.000){2}{\rule{0.350pt}{0.400pt}}
\put(1105,797.67){\rule{0.964pt}{0.400pt}}
\multiput(1107.00,797.17)(-2.000,1.000){2}{\rule{0.482pt}{0.400pt}}
\put(1102,799.17){\rule{0.700pt}{0.400pt}}
\multiput(1103.55,798.17)(-1.547,2.000){2}{\rule{0.350pt}{0.400pt}}
\put(1099,801.17){\rule{0.700pt}{0.400pt}}
\multiput(1100.55,800.17)(-1.547,2.000){2}{\rule{0.350pt}{0.400pt}}
\put(1095,803.17){\rule{0.900pt}{0.400pt}}
\multiput(1097.13,802.17)(-2.132,2.000){2}{\rule{0.450pt}{0.400pt}}
\put(1092,805.17){\rule{0.700pt}{0.400pt}}
\multiput(1093.55,804.17)(-1.547,2.000){2}{\rule{0.350pt}{0.400pt}}
\put(1088,807.17){\rule{0.900pt}{0.400pt}}
\multiput(1090.13,806.17)(-2.132,2.000){2}{\rule{0.450pt}{0.400pt}}
\put(1085,808.67){\rule{0.723pt}{0.400pt}}
\multiput(1086.50,808.17)(-1.500,1.000){2}{\rule{0.361pt}{0.400pt}}
\put(1082,810.17){\rule{0.700pt}{0.400pt}}
\multiput(1083.55,809.17)(-1.547,2.000){2}{\rule{0.350pt}{0.400pt}}
\put(1078,812.17){\rule{0.900pt}{0.400pt}}
\multiput(1080.13,811.17)(-2.132,2.000){2}{\rule{0.450pt}{0.400pt}}
\put(1075,814.17){\rule{0.700pt}{0.400pt}}
\multiput(1076.55,813.17)(-1.547,2.000){2}{\rule{0.350pt}{0.400pt}}
\put(1071,816.17){\rule{0.900pt}{0.400pt}}
\multiput(1073.13,815.17)(-2.132,2.000){2}{\rule{0.450pt}{0.400pt}}
\put(1068,818.17){\rule{0.700pt}{0.400pt}}
\multiput(1069.55,817.17)(-1.547,2.000){2}{\rule{0.350pt}{0.400pt}}
\put(1065,819.67){\rule{0.723pt}{0.400pt}}
\multiput(1066.50,819.17)(-1.500,1.000){2}{\rule{0.361pt}{0.400pt}}
\put(1061,821.17){\rule{0.900pt}{0.400pt}}
\multiput(1063.13,820.17)(-2.132,2.000){2}{\rule{0.450pt}{0.400pt}}
\put(1058,823.17){\rule{0.700pt}{0.400pt}}
\multiput(1059.55,822.17)(-1.547,2.000){2}{\rule{0.350pt}{0.400pt}}
\put(1054,825.17){\rule{0.900pt}{0.400pt}}
\multiput(1056.13,824.17)(-2.132,2.000){2}{\rule{0.450pt}{0.400pt}}
\put(1051,827.17){\rule{0.700pt}{0.400pt}}
\multiput(1052.55,826.17)(-1.547,2.000){2}{\rule{0.350pt}{0.400pt}}
\put(1048,829.17){\rule{0.700pt}{0.400pt}}
\multiput(1049.55,828.17)(-1.547,2.000){2}{\rule{0.350pt}{0.400pt}}
\put(1044,830.67){\rule{0.964pt}{0.400pt}}
\multiput(1046.00,830.17)(-2.000,1.000){2}{\rule{0.482pt}{0.400pt}}
\put(1041,832.17){\rule{0.700pt}{0.400pt}}
\multiput(1042.55,831.17)(-1.547,2.000){2}{\rule{0.350pt}{0.400pt}}
\put(1037,834.17){\rule{0.900pt}{0.400pt}}
\multiput(1039.13,833.17)(-2.132,2.000){2}{\rule{0.450pt}{0.400pt}}
\put(1034,836.17){\rule{0.700pt}{0.400pt}}
\multiput(1035.55,835.17)(-1.547,2.000){2}{\rule{0.350pt}{0.400pt}}
\put(1031,836.17){\rule{0.700pt}{0.400pt}}
\multiput(1032.55,837.17)(-1.547,-2.000){2}{\rule{0.350pt}{0.400pt}}
\put(1027,834.17){\rule{0.900pt}{0.400pt}}
\multiput(1029.13,835.17)(-2.132,-2.000){2}{\rule{0.450pt}{0.400pt}}
\put(1024,832.17){\rule{0.700pt}{0.400pt}}
\multiput(1025.55,833.17)(-1.547,-2.000){2}{\rule{0.350pt}{0.400pt}}
\put(1020,830.67){\rule{0.964pt}{0.400pt}}
\multiput(1022.00,831.17)(-2.000,-1.000){2}{\rule{0.482pt}{0.400pt}}
\put(1017,829.17){\rule{0.700pt}{0.400pt}}
\multiput(1018.55,830.17)(-1.547,-2.000){2}{\rule{0.350pt}{0.400pt}}
\put(1014,827.17){\rule{0.700pt}{0.400pt}}
\multiput(1015.55,828.17)(-1.547,-2.000){2}{\rule{0.350pt}{0.400pt}}
\put(1010,825.17){\rule{0.900pt}{0.400pt}}
\multiput(1012.13,826.17)(-2.132,-2.000){2}{\rule{0.450pt}{0.400pt}}
\put(1007,823.17){\rule{0.700pt}{0.400pt}}
\multiput(1008.55,824.17)(-1.547,-2.000){2}{\rule{0.350pt}{0.400pt}}
\put(1003,821.17){\rule{0.900pt}{0.400pt}}
\multiput(1005.13,822.17)(-2.132,-2.000){2}{\rule{0.450pt}{0.400pt}}
\put(1000,819.67){\rule{0.723pt}{0.400pt}}
\multiput(1001.50,820.17)(-1.500,-1.000){2}{\rule{0.361pt}{0.400pt}}
\put(997,818.17){\rule{0.700pt}{0.400pt}}
\multiput(998.55,819.17)(-1.547,-2.000){2}{\rule{0.350pt}{0.400pt}}
\put(993,816.17){\rule{0.900pt}{0.400pt}}
\multiput(995.13,817.17)(-2.132,-2.000){2}{\rule{0.450pt}{0.400pt}}
\put(990,814.17){\rule{0.700pt}{0.400pt}}
\multiput(991.55,815.17)(-1.547,-2.000){2}{\rule{0.350pt}{0.400pt}}
\put(986,812.17){\rule{0.900pt}{0.400pt}}
\multiput(988.13,813.17)(-2.132,-2.000){2}{\rule{0.450pt}{0.400pt}}
\put(983,810.17){\rule{0.700pt}{0.400pt}}
\multiput(984.55,811.17)(-1.547,-2.000){2}{\rule{0.350pt}{0.400pt}}
\put(980,808.67){\rule{0.723pt}{0.400pt}}
\multiput(981.50,809.17)(-1.500,-1.000){2}{\rule{0.361pt}{0.400pt}}
\put(976,807.17){\rule{0.900pt}{0.400pt}}
\multiput(978.13,808.17)(-2.132,-2.000){2}{\rule{0.450pt}{0.400pt}}
\put(973,805.17){\rule{0.700pt}{0.400pt}}
\multiput(974.55,806.17)(-1.547,-2.000){2}{\rule{0.350pt}{0.400pt}}
\put(969,803.17){\rule{0.900pt}{0.400pt}}
\multiput(971.13,804.17)(-2.132,-2.000){2}{\rule{0.450pt}{0.400pt}}
\put(966,801.17){\rule{0.700pt}{0.400pt}}
\multiput(967.55,802.17)(-1.547,-2.000){2}{\rule{0.350pt}{0.400pt}}
\put(963,799.17){\rule{0.700pt}{0.400pt}}
\multiput(964.55,800.17)(-1.547,-2.000){2}{\rule{0.350pt}{0.400pt}}
\put(959,797.67){\rule{0.964pt}{0.400pt}}
\multiput(961.00,798.17)(-2.000,-1.000){2}{\rule{0.482pt}{0.400pt}}
\put(956,796.17){\rule{0.700pt}{0.400pt}}
\multiput(957.55,797.17)(-1.547,-2.000){2}{\rule{0.350pt}{0.400pt}}
\put(952,794.17){\rule{0.900pt}{0.400pt}}
\multiput(954.13,795.17)(-2.132,-2.000){2}{\rule{0.450pt}{0.400pt}}
\put(949,792.17){\rule{0.700pt}{0.400pt}}
\multiput(950.55,793.17)(-1.547,-2.000){2}{\rule{0.350pt}{0.400pt}}
\put(946,790.17){\rule{0.700pt}{0.400pt}}
\multiput(947.55,791.17)(-1.547,-2.000){2}{\rule{0.350pt}{0.400pt}}
\put(942,788.17){\rule{0.900pt}{0.400pt}}
\multiput(944.13,789.17)(-2.132,-2.000){2}{\rule{0.450pt}{0.400pt}}
\put(939,786.67){\rule{0.723pt}{0.400pt}}
\multiput(940.50,787.17)(-1.500,-1.000){2}{\rule{0.361pt}{0.400pt}}
\put(935,785.17){\rule{0.900pt}{0.400pt}}
\multiput(937.13,786.17)(-2.132,-2.000){2}{\rule{0.450pt}{0.400pt}}
\put(932,783.17){\rule{0.700pt}{0.400pt}}
\multiput(933.55,784.17)(-1.547,-2.000){2}{\rule{0.350pt}{0.400pt}}
\put(928,781.17){\rule{0.900pt}{0.400pt}}
\multiput(930.13,782.17)(-2.132,-2.000){2}{\rule{0.450pt}{0.400pt}}
\put(925,779.17){\rule{0.700pt}{0.400pt}}
\multiput(926.55,780.17)(-1.547,-2.000){2}{\rule{0.350pt}{0.400pt}}
\put(922,777.17){\rule{0.700pt}{0.400pt}}
\multiput(923.55,778.17)(-1.547,-2.000){2}{\rule{0.350pt}{0.400pt}}
\put(918,775.67){\rule{0.964pt}{0.400pt}}
\multiput(920.00,776.17)(-2.000,-1.000){2}{\rule{0.482pt}{0.400pt}}
\put(915,774.17){\rule{0.700pt}{0.400pt}}
\multiput(916.55,775.17)(-1.547,-2.000){2}{\rule{0.350pt}{0.400pt}}
\put(911,772.17){\rule{0.900pt}{0.400pt}}
\multiput(913.13,773.17)(-2.132,-2.000){2}{\rule{0.450pt}{0.400pt}}
\put(908,770.17){\rule{0.700pt}{0.400pt}}
\multiput(909.55,771.17)(-1.547,-2.000){2}{\rule{0.350pt}{0.400pt}}
\put(905,768.17){\rule{0.700pt}{0.400pt}}
\multiput(906.55,769.17)(-1.547,-2.000){2}{\rule{0.350pt}{0.400pt}}
\put(901,766.67){\rule{0.964pt}{0.400pt}}
\multiput(903.00,767.17)(-2.000,-1.000){2}{\rule{0.482pt}{0.400pt}}
\put(898,765.17){\rule{0.700pt}{0.400pt}}
\multiput(899.55,766.17)(-1.547,-2.000){2}{\rule{0.350pt}{0.400pt}}
\put(894,763.17){\rule{0.900pt}{0.400pt}}
\multiput(896.13,764.17)(-2.132,-2.000){2}{\rule{0.450pt}{0.400pt}}
\put(891,761.17){\rule{0.700pt}{0.400pt}}
\multiput(892.55,762.17)(-1.547,-2.000){2}{\rule{0.350pt}{0.400pt}}
\put(888,759.17){\rule{0.700pt}{0.400pt}}
\multiput(889.55,760.17)(-1.547,-2.000){2}{\rule{0.350pt}{0.400pt}}
\put(884,757.17){\rule{0.900pt}{0.400pt}}
\multiput(886.13,758.17)(-2.132,-2.000){2}{\rule{0.450pt}{0.400pt}}
\put(881,755.67){\rule{0.723pt}{0.400pt}}
\multiput(882.50,756.17)(-1.500,-1.000){2}{\rule{0.361pt}{0.400pt}}
\put(877,754.17){\rule{0.900pt}{0.400pt}}
\multiput(879.13,755.17)(-2.132,-2.000){2}{\rule{0.450pt}{0.400pt}}
\put(874,752.17){\rule{0.700pt}{0.400pt}}
\multiput(875.55,753.17)(-1.547,-2.000){2}{\rule{0.350pt}{0.400pt}}
\put(871,750.17){\rule{0.700pt}{0.400pt}}
\multiput(872.55,751.17)(-1.547,-2.000){2}{\rule{0.350pt}{0.400pt}}
\put(867,748.17){\rule{0.900pt}{0.400pt}}
\multiput(869.13,749.17)(-2.132,-2.000){2}{\rule{0.450pt}{0.400pt}}
\put(864,746.17){\rule{0.700pt}{0.400pt}}
\multiput(865.55,747.17)(-1.547,-2.000){2}{\rule{0.350pt}{0.400pt}}
\put(860,744.67){\rule{0.964pt}{0.400pt}}
\multiput(862.00,745.17)(-2.000,-1.000){2}{\rule{0.482pt}{0.400pt}}
\put(857,743.17){\rule{0.700pt}{0.400pt}}
\multiput(858.55,744.17)(-1.547,-2.000){2}{\rule{0.350pt}{0.400pt}}
\put(854,741.17){\rule{0.700pt}{0.400pt}}
\multiput(855.55,742.17)(-1.547,-2.000){2}{\rule{0.350pt}{0.400pt}}
\put(850,739.17){\rule{0.900pt}{0.400pt}}
\multiput(852.13,740.17)(-2.132,-2.000){2}{\rule{0.450pt}{0.400pt}}
\put(847,737.17){\rule{0.700pt}{0.400pt}}
\multiput(848.55,738.17)(-1.547,-2.000){2}{\rule{0.350pt}{0.400pt}}
\put(843,735.17){\rule{0.900pt}{0.400pt}}
\multiput(845.13,736.17)(-2.132,-2.000){2}{\rule{0.450pt}{0.400pt}}
\put(840,733.67){\rule{0.723pt}{0.400pt}}
\multiput(841.50,734.17)(-1.500,-1.000){2}{\rule{0.361pt}{0.400pt}}
\put(837,732.17){\rule{0.700pt}{0.400pt}}
\multiput(838.55,733.17)(-1.547,-2.000){2}{\rule{0.350pt}{0.400pt}}
\put(833,730.17){\rule{0.900pt}{0.400pt}}
\multiput(835.13,731.17)(-2.132,-2.000){2}{\rule{0.450pt}{0.400pt}}
\put(830,728.17){\rule{0.700pt}{0.400pt}}
\multiput(831.55,729.17)(-1.547,-2.000){2}{\rule{0.350pt}{0.400pt}}
\put(826,726.17){\rule{0.900pt}{0.400pt}}
\multiput(828.13,727.17)(-2.132,-2.000){2}{\rule{0.450pt}{0.400pt}}
\put(823,724.17){\rule{0.700pt}{0.400pt}}
\multiput(824.55,725.17)(-1.547,-2.000){2}{\rule{0.350pt}{0.400pt}}
\put(820,722.67){\rule{0.723pt}{0.400pt}}
\multiput(821.50,723.17)(-1.500,-1.000){2}{\rule{0.361pt}{0.400pt}}
\put(816,721.17){\rule{0.900pt}{0.400pt}}
\multiput(818.13,722.17)(-2.132,-2.000){2}{\rule{0.450pt}{0.400pt}}
\put(813,719.17){\rule{0.700pt}{0.400pt}}
\multiput(814.55,720.17)(-1.547,-2.000){2}{\rule{0.350pt}{0.400pt}}
\put(809,717.17){\rule{0.900pt}{0.400pt}}
\multiput(811.13,718.17)(-2.132,-2.000){2}{\rule{0.450pt}{0.400pt}}
\put(806,715.17){\rule{0.700pt}{0.400pt}}
\multiput(807.55,716.17)(-1.547,-2.000){2}{\rule{0.350pt}{0.400pt}}
\put(803,713.17){\rule{0.700pt}{0.400pt}}
\multiput(804.55,714.17)(-1.547,-2.000){2}{\rule{0.350pt}{0.400pt}}
\put(799,711.67){\rule{0.964pt}{0.400pt}}
\multiput(801.00,712.17)(-2.000,-1.000){2}{\rule{0.482pt}{0.400pt}}
\put(796,710.17){\rule{0.700pt}{0.400pt}}
\multiput(797.55,711.17)(-1.547,-2.000){2}{\rule{0.350pt}{0.400pt}}
\put(792,708.17){\rule{0.900pt}{0.400pt}}
\multiput(794.13,709.17)(-2.132,-2.000){2}{\rule{0.450pt}{0.400pt}}
\put(789,706.17){\rule{0.700pt}{0.400pt}}
\multiput(790.55,707.17)(-1.547,-2.000){2}{\rule{0.350pt}{0.400pt}}
\put(786,704.17){\rule{0.700pt}{0.400pt}}
\multiput(787.55,705.17)(-1.547,-2.000){2}{\rule{0.350pt}{0.400pt}}
\put(782,702.67){\rule{0.964pt}{0.400pt}}
\multiput(784.00,703.17)(-2.000,-1.000){2}{\rule{0.482pt}{0.400pt}}
\put(779,701.17){\rule{0.700pt}{0.400pt}}
\multiput(780.55,702.17)(-1.547,-2.000){2}{\rule{0.350pt}{0.400pt}}
\put(775,699.17){\rule{0.900pt}{0.400pt}}
\multiput(777.13,700.17)(-2.132,-2.000){2}{\rule{0.450pt}{0.400pt}}
\put(772,697.17){\rule{0.700pt}{0.400pt}}
\multiput(773.55,698.17)(-1.547,-2.000){2}{\rule{0.350pt}{0.400pt}}
\put(769,695.17){\rule{0.700pt}{0.400pt}}
\multiput(770.55,696.17)(-1.547,-2.000){2}{\rule{0.350pt}{0.400pt}}
\put(765,693.17){\rule{0.900pt}{0.400pt}}
\multiput(767.13,694.17)(-2.132,-2.000){2}{\rule{0.450pt}{0.400pt}}
\put(762,691.67){\rule{0.723pt}{0.400pt}}
\multiput(763.50,692.17)(-1.500,-1.000){2}{\rule{0.361pt}{0.400pt}}
\put(758,690.17){\rule{0.900pt}{0.400pt}}
\multiput(760.13,691.17)(-2.132,-2.000){2}{\rule{0.450pt}{0.400pt}}
\put(755,688.17){\rule{0.700pt}{0.400pt}}
\multiput(756.55,689.17)(-1.547,-2.000){2}{\rule{0.350pt}{0.400pt}}
\put(752,686.17){\rule{0.700pt}{0.400pt}}
\multiput(753.55,687.17)(-1.547,-2.000){2}{\rule{0.350pt}{0.400pt}}
\put(748,684.17){\rule{0.900pt}{0.400pt}}
\multiput(750.13,685.17)(-2.132,-2.000){2}{\rule{0.450pt}{0.400pt}}
\put(745,682.17){\rule{0.700pt}{0.400pt}}
\multiput(746.55,683.17)(-1.547,-2.000){2}{\rule{0.350pt}{0.400pt}}
\put(741,680.67){\rule{0.964pt}{0.400pt}}
\multiput(743.00,681.17)(-2.000,-1.000){2}{\rule{0.482pt}{0.400pt}}
\put(738,679.17){\rule{0.700pt}{0.400pt}}
\multiput(739.55,680.17)(-1.547,-2.000){2}{\rule{0.350pt}{0.400pt}}
\put(735,677.17){\rule{0.700pt}{0.400pt}}
\multiput(736.55,678.17)(-1.547,-2.000){2}{\rule{0.350pt}{0.400pt}}
\put(731,675.17){\rule{0.900pt}{0.400pt}}
\multiput(733.13,676.17)(-2.132,-2.000){2}{\rule{0.450pt}{0.400pt}}
\put(728,673.17){\rule{0.700pt}{0.400pt}}
\multiput(729.55,674.17)(-1.547,-2.000){2}{\rule{0.350pt}{0.400pt}}
\put(724,671.17){\rule{0.900pt}{0.400pt}}
\multiput(726.13,672.17)(-2.132,-2.000){2}{\rule{0.450pt}{0.400pt}}
\put(721,669.67){\rule{0.723pt}{0.400pt}}
\multiput(722.50,670.17)(-1.500,-1.000){2}{\rule{0.361pt}{0.400pt}}
\put(718,668.17){\rule{0.700pt}{0.400pt}}
\multiput(719.55,669.17)(-1.547,-2.000){2}{\rule{0.350pt}{0.400pt}}
\put(714,666.17){\rule{0.900pt}{0.400pt}}
\multiput(716.13,667.17)(-2.132,-2.000){2}{\rule{0.450pt}{0.400pt}}
\put(711,664.17){\rule{0.700pt}{0.400pt}}
\multiput(712.55,665.17)(-1.547,-2.000){2}{\rule{0.350pt}{0.400pt}}
\put(707,662.17){\rule{0.900pt}{0.400pt}}
\multiput(709.13,663.17)(-2.132,-2.000){2}{\rule{0.450pt}{0.400pt}}
\put(704,660.17){\rule{0.700pt}{0.400pt}}
\multiput(705.55,661.17)(-1.547,-2.000){2}{\rule{0.350pt}{0.400pt}}
\put(701,658.67){\rule{0.723pt}{0.400pt}}
\multiput(702.50,659.17)(-1.500,-1.000){2}{\rule{0.361pt}{0.400pt}}
\put(697,657.17){\rule{0.900pt}{0.400pt}}
\multiput(699.13,658.17)(-2.132,-2.000){2}{\rule{0.450pt}{0.400pt}}
\put(694,655.17){\rule{0.700pt}{0.400pt}}
\multiput(695.55,656.17)(-1.547,-2.000){2}{\rule{0.350pt}{0.400pt}}
\put(690,653.17){\rule{0.900pt}{0.400pt}}
\multiput(692.13,654.17)(-2.132,-2.000){2}{\rule{0.450pt}{0.400pt}}
\put(687,651.17){\rule{0.700pt}{0.400pt}}
\multiput(688.55,652.17)(-1.547,-2.000){2}{\rule{0.350pt}{0.400pt}}
\put(684,649.17){\rule{0.700pt}{0.400pt}}
\multiput(685.55,650.17)(-1.547,-2.000){2}{\rule{0.350pt}{0.400pt}}
\put(680,647.67){\rule{0.964pt}{0.400pt}}
\multiput(682.00,648.17)(-2.000,-1.000){2}{\rule{0.482pt}{0.400pt}}
\put(677,646.17){\rule{0.700pt}{0.400pt}}
\multiput(678.55,647.17)(-1.547,-2.000){2}{\rule{0.350pt}{0.400pt}}
\put(673,644.17){\rule{0.900pt}{0.400pt}}
\multiput(675.13,645.17)(-2.132,-2.000){2}{\rule{0.450pt}{0.400pt}}
\put(670,642.17){\rule{0.700pt}{0.400pt}}
\multiput(671.55,643.17)(-1.547,-2.000){2}{\rule{0.350pt}{0.400pt}}
\put(667,640.17){\rule{0.700pt}{0.400pt}}
\multiput(668.55,641.17)(-1.547,-2.000){2}{\rule{0.350pt}{0.400pt}}
\put(663,638.17){\rule{0.900pt}{0.400pt}}
\multiput(665.13,639.17)(-2.132,-2.000){2}{\rule{0.450pt}{0.400pt}}
\put(660,636.67){\rule{0.723pt}{0.400pt}}
\multiput(661.50,637.17)(-1.500,-1.000){2}{\rule{0.361pt}{0.400pt}}
\put(656,635.17){\rule{0.900pt}{0.400pt}}
\multiput(658.13,636.17)(-2.132,-2.000){2}{\rule{0.450pt}{0.400pt}}
\put(653,633.17){\rule{0.700pt}{0.400pt}}
\multiput(654.55,634.17)(-1.547,-2.000){2}{\rule{0.350pt}{0.400pt}}
\put(650,631.17){\rule{0.700pt}{0.400pt}}
\multiput(651.55,632.17)(-1.547,-2.000){2}{\rule{0.350pt}{0.400pt}}
\put(646,629.17){\rule{0.900pt}{0.400pt}}
\multiput(648.13,630.17)(-2.132,-2.000){2}{\rule{0.450pt}{0.400pt}}
\put(643,627.67){\rule{0.723pt}{0.400pt}}
\multiput(644.50,628.17)(-1.500,-1.000){2}{\rule{0.361pt}{0.400pt}}
\put(639,626.17){\rule{0.900pt}{0.400pt}}
\multiput(641.13,627.17)(-2.132,-2.000){2}{\rule{0.450pt}{0.400pt}}
\put(636,624.17){\rule{0.700pt}{0.400pt}}
\multiput(637.55,625.17)(-1.547,-2.000){2}{\rule{0.350pt}{0.400pt}}
\put(633,622.17){\rule{0.700pt}{0.400pt}}
\multiput(634.55,623.17)(-1.547,-2.000){2}{\rule{0.350pt}{0.400pt}}
\put(629,620.17){\rule{0.900pt}{0.400pt}}
\multiput(631.13,621.17)(-2.132,-2.000){2}{\rule{0.450pt}{0.400pt}}
\put(626,618.17){\rule{0.700pt}{0.400pt}}
\multiput(627.55,619.17)(-1.547,-2.000){2}{\rule{0.350pt}{0.400pt}}
\put(622,616.67){\rule{0.964pt}{0.400pt}}
\multiput(624.00,617.17)(-2.000,-1.000){2}{\rule{0.482pt}{0.400pt}}
\put(619,615.17){\rule{0.700pt}{0.400pt}}
\multiput(620.55,616.17)(-1.547,-2.000){2}{\rule{0.350pt}{0.400pt}}
\put(616,613.17){\rule{0.700pt}{0.400pt}}
\multiput(617.55,614.17)(-1.547,-2.000){2}{\rule{0.350pt}{0.400pt}}
\put(612,611.17){\rule{0.900pt}{0.400pt}}
\multiput(614.13,612.17)(-2.132,-2.000){2}{\rule{0.450pt}{0.400pt}}
\put(609,609.17){\rule{0.700pt}{0.400pt}}
\multiput(610.55,610.17)(-1.547,-2.000){2}{\rule{0.350pt}{0.400pt}}
\put(605,607.17){\rule{0.900pt}{0.400pt}}
\multiput(607.13,608.17)(-2.132,-2.000){2}{\rule{0.450pt}{0.400pt}}
\put(602,605.67){\rule{0.723pt}{0.400pt}}
\multiput(603.50,606.17)(-1.500,-1.000){2}{\rule{0.361pt}{0.400pt}}
\put(599,604.17){\rule{0.700pt}{0.400pt}}
\multiput(600.55,605.17)(-1.547,-2.000){2}{\rule{0.350pt}{0.400pt}}
\put(595,602.17){\rule{0.900pt}{0.400pt}}
\multiput(597.13,603.17)(-2.132,-2.000){2}{\rule{0.450pt}{0.400pt}}
\put(592,600.17){\rule{0.700pt}{0.400pt}}
\multiput(593.55,601.17)(-1.547,-2.000){2}{\rule{0.350pt}{0.400pt}}
\put(588,598.17){\rule{0.900pt}{0.400pt}}
\multiput(590.13,599.17)(-2.132,-2.000){2}{\rule{0.450pt}{0.400pt}}
\put(585,596.17){\rule{0.700pt}{0.400pt}}
\multiput(586.55,597.17)(-1.547,-2.000){2}{\rule{0.350pt}{0.400pt}}
\put(582,594.67){\rule{0.723pt}{0.400pt}}
\multiput(583.50,595.17)(-1.500,-1.000){2}{\rule{0.361pt}{0.400pt}}
\put(578,593.17){\rule{0.900pt}{0.400pt}}
\multiput(580.13,594.17)(-2.132,-2.000){2}{\rule{0.450pt}{0.400pt}}
\put(575,591.17){\rule{0.700pt}{0.400pt}}
\multiput(576.55,592.17)(-1.547,-2.000){2}{\rule{0.350pt}{0.400pt}}
\put(571,589.17){\rule{0.900pt}{0.400pt}}
\multiput(573.13,590.17)(-2.132,-2.000){2}{\rule{0.450pt}{0.400pt}}
\put(568,587.17){\rule{0.700pt}{0.400pt}}
\multiput(569.55,588.17)(-1.547,-2.000){2}{\rule{0.350pt}{0.400pt}}
\put(565,585.17){\rule{0.700pt}{0.400pt}}
\multiput(566.55,586.17)(-1.547,-2.000){2}{\rule{0.350pt}{0.400pt}}
\put(561,583.67){\rule{0.964pt}{0.400pt}}
\multiput(563.00,584.17)(-2.000,-1.000){2}{\rule{0.482pt}{0.400pt}}
\put(558,582.17){\rule{0.700pt}{0.400pt}}
\multiput(559.55,583.17)(-1.547,-2.000){2}{\rule{0.350pt}{0.400pt}}
\put(554,580.17){\rule{0.900pt}{0.400pt}}
\multiput(556.13,581.17)(-2.132,-2.000){2}{\rule{0.450pt}{0.400pt}}
\put(551,578.17){\rule{0.700pt}{0.400pt}}
\multiput(552.55,579.17)(-1.547,-2.000){2}{\rule{0.350pt}{0.400pt}}
\put(548,576.17){\rule{0.700pt}{0.400pt}}
\multiput(549.55,577.17)(-1.547,-2.000){2}{\rule{0.350pt}{0.400pt}}
\put(544,574.17){\rule{0.900pt}{0.400pt}}
\multiput(546.13,575.17)(-2.132,-2.000){2}{\rule{0.450pt}{0.400pt}}
\put(541,572.67){\rule{0.723pt}{0.400pt}}
\multiput(542.50,573.17)(-1.500,-1.000){2}{\rule{0.361pt}{0.400pt}}
\put(537,571.17){\rule{0.900pt}{0.400pt}}
\multiput(539.13,572.17)(-2.132,-2.000){2}{\rule{0.450pt}{0.400pt}}
\put(534,569.17){\rule{0.700pt}{0.400pt}}
\multiput(535.55,570.17)(-1.547,-2.000){2}{\rule{0.350pt}{0.400pt}}
\put(531,567.17){\rule{0.700pt}{0.400pt}}
\multiput(532.55,568.17)(-1.547,-2.000){2}{\rule{0.350pt}{0.400pt}}
\put(527,565.17){\rule{0.900pt}{0.400pt}}
\multiput(529.13,566.17)(-2.132,-2.000){2}{\rule{0.450pt}{0.400pt}}
\put(524,563.67){\rule{0.723pt}{0.400pt}}
\multiput(525.50,564.17)(-1.500,-1.000){2}{\rule{0.361pt}{0.400pt}}
\put(520,562.17){\rule{0.900pt}{0.400pt}}
\multiput(522.13,563.17)(-2.132,-2.000){2}{\rule{0.450pt}{0.400pt}}
\put(517,560.17){\rule{0.700pt}{0.400pt}}
\multiput(518.55,561.17)(-1.547,-2.000){2}{\rule{0.350pt}{0.400pt}}
\put(514,558.17){\rule{0.700pt}{0.400pt}}
\multiput(515.55,559.17)(-1.547,-2.000){2}{\rule{0.350pt}{0.400pt}}
\put(510,556.17){\rule{0.900pt}{0.400pt}}
\multiput(512.13,557.17)(-2.132,-2.000){2}{\rule{0.450pt}{0.400pt}}
\put(507,554.17){\rule{0.700pt}{0.400pt}}
\multiput(508.55,555.17)(-1.547,-2.000){2}{\rule{0.350pt}{0.400pt}}
\put(503,552.67){\rule{0.964pt}{0.400pt}}
\multiput(505.00,553.17)(-2.000,-1.000){2}{\rule{0.482pt}{0.400pt}}
\put(500,551.17){\rule{0.700pt}{0.400pt}}
\multiput(501.55,552.17)(-1.547,-2.000){2}{\rule{0.350pt}{0.400pt}}
\put(497,549.17){\rule{0.700pt}{0.400pt}}
\multiput(498.55,550.17)(-1.547,-2.000){2}{\rule{0.350pt}{0.400pt}}
\put(493,547.17){\rule{0.900pt}{0.400pt}}
\multiput(495.13,548.17)(-2.132,-2.000){2}{\rule{0.450pt}{0.400pt}}
\put(490,545.17){\rule{0.700pt}{0.400pt}}
\multiput(491.55,546.17)(-1.547,-2.000){2}{\rule{0.350pt}{0.400pt}}
\put(486,543.17){\rule{0.900pt}{0.400pt}}
\multiput(488.13,544.17)(-2.132,-2.000){2}{\rule{0.450pt}{0.400pt}}
\put(483,541.67){\rule{0.723pt}{0.400pt}}
\multiput(484.50,542.17)(-1.500,-1.000){2}{\rule{0.361pt}{0.400pt}}
\put(480,540.17){\rule{0.700pt}{0.400pt}}
\multiput(481.55,541.17)(-1.547,-2.000){2}{\rule{0.350pt}{0.400pt}}
\put(476,538.17){\rule{0.900pt}{0.400pt}}
\multiput(478.13,539.17)(-2.132,-2.000){2}{\rule{0.450pt}{0.400pt}}
\put(473,536.17){\rule{0.700pt}{0.400pt}}
\multiput(474.55,537.17)(-1.547,-2.000){2}{\rule{0.350pt}{0.400pt}}
\put(469,534.17){\rule{0.900pt}{0.400pt}}
\multiput(471.13,535.17)(-2.132,-2.000){2}{\rule{0.450pt}{0.400pt}}
\put(466,532.17){\rule{0.700pt}{0.400pt}}
\multiput(467.55,533.17)(-1.547,-2.000){2}{\rule{0.350pt}{0.400pt}}
\put(463,530.67){\rule{0.723pt}{0.400pt}}
\multiput(464.50,531.17)(-1.500,-1.000){2}{\rule{0.361pt}{0.400pt}}
\put(459,529.17){\rule{0.900pt}{0.400pt}}
\multiput(461.13,530.17)(-2.132,-2.000){2}{\rule{0.450pt}{0.400pt}}
\put(456,527.17){\rule{0.700pt}{0.400pt}}
\multiput(457.55,528.17)(-1.547,-2.000){2}{\rule{0.350pt}{0.400pt}}
\put(452,525.17){\rule{0.900pt}{0.400pt}}
\multiput(454.13,526.17)(-2.132,-2.000){2}{\rule{0.450pt}{0.400pt}}
\put(449,523.17){\rule{0.700pt}{0.400pt}}
\multiput(450.55,524.17)(-1.547,-2.000){2}{\rule{0.350pt}{0.400pt}}
\put(446,521.17){\rule{0.700pt}{0.400pt}}
\multiput(447.55,522.17)(-1.547,-2.000){2}{\rule{0.350pt}{0.400pt}}
\put(442,519.67){\rule{0.964pt}{0.400pt}}
\multiput(444.00,520.17)(-2.000,-1.000){2}{\rule{0.482pt}{0.400pt}}
\put(439,518.17){\rule{0.700pt}{0.400pt}}
\multiput(440.55,519.17)(-1.547,-2.000){2}{\rule{0.350pt}{0.400pt}}
\put(435,516.17){\rule{0.900pt}{0.400pt}}
\multiput(437.13,517.17)(-2.132,-2.000){2}{\rule{0.450pt}{0.400pt}}
\put(432,514.17){\rule{0.700pt}{0.400pt}}
\multiput(433.55,515.17)(-1.547,-2.000){2}{\rule{0.350pt}{0.400pt}}
\put(429,512.17){\rule{0.700pt}{0.400pt}}
\multiput(430.55,513.17)(-1.547,-2.000){2}{\rule{0.350pt}{0.400pt}}
\put(425,510.17){\rule{0.900pt}{0.400pt}}
\multiput(427.13,511.17)(-2.132,-2.000){2}{\rule{0.450pt}{0.400pt}}
\put(422,508.67){\rule{0.723pt}{0.400pt}}
\multiput(423.50,509.17)(-1.500,-1.000){2}{\rule{0.361pt}{0.400pt}}
\put(418,507.17){\rule{0.900pt}{0.400pt}}
\multiput(420.13,508.17)(-2.132,-2.000){2}{\rule{0.450pt}{0.400pt}}
\put(415,505.17){\rule{0.700pt}{0.400pt}}
\multiput(416.55,506.17)(-1.547,-2.000){2}{\rule{0.350pt}{0.400pt}}
\put(412,503.17){\rule{0.700pt}{0.400pt}}
\multiput(413.55,504.17)(-1.547,-2.000){2}{\rule{0.350pt}{0.400pt}}
\put(408,501.17){\rule{0.900pt}{0.400pt}}
\multiput(410.13,502.17)(-2.132,-2.000){2}{\rule{0.450pt}{0.400pt}}
\put(405,499.67){\rule{0.723pt}{0.400pt}}
\multiput(406.50,500.17)(-1.500,-1.000){2}{\rule{0.361pt}{0.400pt}}
\put(401,498.17){\rule{0.900pt}{0.400pt}}
\multiput(403.13,499.17)(-2.132,-2.000){2}{\rule{0.450pt}{0.400pt}}
\put(398,496.17){\rule{0.700pt}{0.400pt}}
\multiput(399.55,497.17)(-1.547,-2.000){2}{\rule{0.350pt}{0.400pt}}
\put(395,494.17){\rule{0.700pt}{0.400pt}}
\multiput(396.55,495.17)(-1.547,-2.000){2}{\rule{0.350pt}{0.400pt}}
\put(391,492.17){\rule{0.900pt}{0.400pt}}
\multiput(393.13,493.17)(-2.132,-2.000){2}{\rule{0.450pt}{0.400pt}}
\put(388,490.17){\rule{0.700pt}{0.400pt}}
\multiput(389.55,491.17)(-1.547,-2.000){2}{\rule{0.350pt}{0.400pt}}
\put(384,488.67){\rule{0.964pt}{0.400pt}}
\multiput(386.00,489.17)(-2.000,-1.000){2}{\rule{0.482pt}{0.400pt}}
\put(381,487.17){\rule{0.700pt}{0.400pt}}
\multiput(382.55,488.17)(-1.547,-2.000){2}{\rule{0.350pt}{0.400pt}}
\put(378,485.17){\rule{0.700pt}{0.400pt}}
\multiput(379.55,486.17)(-1.547,-2.000){2}{\rule{0.350pt}{0.400pt}}
\put(374,483.17){\rule{0.900pt}{0.400pt}}
\multiput(376.13,484.17)(-2.132,-2.000){2}{\rule{0.450pt}{0.400pt}}
\put(371,481.17){\rule{0.700pt}{0.400pt}}
\multiput(372.55,482.17)(-1.547,-2.000){2}{\rule{0.350pt}{0.400pt}}
\put(367,479.17){\rule{0.900pt}{0.400pt}}
\multiput(369.13,480.17)(-2.132,-2.000){2}{\rule{0.450pt}{0.400pt}}
\put(364,477.67){\rule{0.723pt}{0.400pt}}
\multiput(365.50,478.17)(-1.500,-1.000){2}{\rule{0.361pt}{0.400pt}}
\put(361,476.17){\rule{0.700pt}{0.400pt}}
\multiput(362.55,477.17)(-1.547,-2.000){2}{\rule{0.350pt}{0.400pt}}
\put(357,474.17){\rule{0.900pt}{0.400pt}}
\multiput(359.13,475.17)(-2.132,-2.000){2}{\rule{0.450pt}{0.400pt}}
\put(354,472.17){\rule{0.700pt}{0.400pt}}
\multiput(355.55,473.17)(-1.547,-2.000){2}{\rule{0.350pt}{0.400pt}}
\put(350,470.17){\rule{0.900pt}{0.400pt}}
\multiput(352.13,471.17)(-2.132,-2.000){2}{\rule{0.450pt}{0.400pt}}
\put(347,468.17){\rule{0.700pt}{0.400pt}}
\multiput(348.55,469.17)(-1.547,-2.000){2}{\rule{0.350pt}{0.400pt}}
\put(344,466.67){\rule{0.723pt}{0.400pt}}
\multiput(345.50,467.17)(-1.500,-1.000){2}{\rule{0.361pt}{0.400pt}}
\put(340,465.17){\rule{0.900pt}{0.400pt}}
\multiput(342.13,466.17)(-2.132,-2.000){2}{\rule{0.450pt}{0.400pt}}
\put(337,463.17){\rule{0.700pt}{0.400pt}}
\multiput(338.55,464.17)(-1.547,-2.000){2}{\rule{0.350pt}{0.400pt}}
\put(333,461.17){\rule{0.900pt}{0.400pt}}
\multiput(335.13,462.17)(-2.132,-2.000){2}{\rule{0.450pt}{0.400pt}}
\put(330,459.17){\rule{0.700pt}{0.400pt}}
\multiput(331.55,460.17)(-1.547,-2.000){2}{\rule{0.350pt}{0.400pt}}
\put(327,457.17){\rule{0.700pt}{0.400pt}}
\multiput(328.55,458.17)(-1.547,-2.000){2}{\rule{0.350pt}{0.400pt}}
\put(323,455.67){\rule{0.964pt}{0.400pt}}
\multiput(325.00,456.17)(-2.000,-1.000){2}{\rule{0.482pt}{0.400pt}}
\put(320,454.17){\rule{0.700pt}{0.400pt}}
\multiput(321.55,455.17)(-1.547,-2.000){2}{\rule{0.350pt}{0.400pt}}
\put(316,452.17){\rule{0.900pt}{0.400pt}}
\multiput(318.13,453.17)(-2.132,-2.000){2}{\rule{0.450pt}{0.400pt}}
\put(313,450.17){\rule{0.700pt}{0.400pt}}
\multiput(314.55,451.17)(-1.547,-2.000){2}{\rule{0.350pt}{0.400pt}}
\put(310,448.17){\rule{0.700pt}{0.400pt}}
\multiput(311.55,449.17)(-1.547,-2.000){2}{\rule{0.350pt}{0.400pt}}
\put(306,446.17){\rule{0.900pt}{0.400pt}}
\multiput(308.13,447.17)(-2.132,-2.000){2}{\rule{0.450pt}{0.400pt}}
\put(303,444.67){\rule{0.723pt}{0.400pt}}
\multiput(304.50,445.17)(-1.500,-1.000){2}{\rule{0.361pt}{0.400pt}}
\put(299,443.17){\rule{0.900pt}{0.400pt}}
\multiput(301.13,444.17)(-2.132,-2.000){2}{\rule{0.450pt}{0.400pt}}
\put(296,441.17){\rule{0.700pt}{0.400pt}}
\multiput(297.55,442.17)(-1.547,-2.000){2}{\rule{0.350pt}{0.400pt}}
\put(292,439.17){\rule{0.900pt}{0.400pt}}
\multiput(294.13,440.17)(-2.132,-2.000){2}{\rule{0.450pt}{0.400pt}}
\put(289,437.17){\rule{0.700pt}{0.400pt}}
\multiput(290.55,438.17)(-1.547,-2.000){2}{\rule{0.350pt}{0.400pt}}
\put(286,435.67){\rule{0.723pt}{0.400pt}}
\multiput(287.50,436.17)(-1.500,-1.000){2}{\rule{0.361pt}{0.400pt}}
\put(282,434.17){\rule{0.900pt}{0.400pt}}
\multiput(284.13,435.17)(-2.132,-2.000){2}{\rule{0.450pt}{0.400pt}}
\put(279,432.17){\rule{0.700pt}{0.400pt}}
\multiput(280.55,433.17)(-1.547,-2.000){2}{\rule{0.350pt}{0.400pt}}
\put(275,430.17){\rule{0.900pt}{0.400pt}}
\multiput(277.13,431.17)(-2.132,-2.000){2}{\rule{0.450pt}{0.400pt}}
\put(272,428.17){\rule{0.700pt}{0.400pt}}
\multiput(273.55,429.17)(-1.547,-2.000){2}{\rule{0.350pt}{0.400pt}}
\put(269,426.17){\rule{0.700pt}{0.400pt}}
\multiput(270.55,427.17)(-1.547,-2.000){2}{\rule{0.350pt}{0.400pt}}
\put(265,424.67){\rule{0.964pt}{0.400pt}}
\multiput(267.00,425.17)(-2.000,-1.000){2}{\rule{0.482pt}{0.400pt}}
\put(262,423.17){\rule{0.700pt}{0.400pt}}
\multiput(263.55,424.17)(-1.547,-2.000){2}{\rule{0.350pt}{0.400pt}}
\put(258,421.17){\rule{0.900pt}{0.400pt}}
\multiput(260.13,422.17)(-2.132,-2.000){2}{\rule{0.450pt}{0.400pt}}
\put(255,419.17){\rule{0.700pt}{0.400pt}}
\multiput(256.55,420.17)(-1.547,-2.000){2}{\rule{0.350pt}{0.400pt}}
\put(252,417.17){\rule{0.700pt}{0.400pt}}
\multiput(253.55,418.17)(-1.547,-2.000){2}{\rule{0.350pt}{0.400pt}}
\put(248,415.17){\rule{0.900pt}{0.400pt}}
\multiput(250.13,416.17)(-2.132,-2.000){2}{\rule{0.450pt}{0.400pt}}
\put(245,413.67){\rule{0.723pt}{0.400pt}}
\multiput(246.50,414.17)(-1.500,-1.000){2}{\rule{0.361pt}{0.400pt}}
\put(241,412.17){\rule{0.900pt}{0.400pt}}
\multiput(243.13,413.17)(-2.132,-2.000){2}{\rule{0.450pt}{0.400pt}}
\put(238,410.17){\rule{0.700pt}{0.400pt}}
\multiput(239.55,411.17)(-1.547,-2.000){2}{\rule{0.350pt}{0.400pt}}
\put(235,408.17){\rule{0.700pt}{0.400pt}}
\multiput(236.55,409.17)(-1.547,-2.000){2}{\rule{0.350pt}{0.400pt}}
\put(231,406.17){\rule{0.900pt}{0.400pt}}
\multiput(233.13,407.17)(-2.132,-2.000){2}{\rule{0.450pt}{0.400pt}}
\put(228,404.17){\rule{0.700pt}{0.400pt}}
\multiput(229.55,405.17)(-1.547,-2.000){2}{\rule{0.350pt}{0.400pt}}
\put(224,402.67){\rule{0.964pt}{0.400pt}}
\multiput(226.00,403.17)(-2.000,-1.000){2}{\rule{0.482pt}{0.400pt}}
\put(221,401.17){\rule{0.700pt}{0.400pt}}
\multiput(222.55,402.17)(-1.547,-2.000){2}{\rule{0.350pt}{0.400pt}}
\put(218,399.17){\rule{0.700pt}{0.400pt}}
\multiput(219.55,400.17)(-1.547,-2.000){2}{\rule{0.350pt}{0.400pt}}
\put(214,397.17){\rule{0.900pt}{0.400pt}}
\multiput(216.13,398.17)(-2.132,-2.000){2}{\rule{0.450pt}{0.400pt}}
\put(211,395.17){\rule{0.700pt}{0.400pt}}
\multiput(212.55,396.17)(-1.547,-2.000){2}{\rule{0.350pt}{0.400pt}}
\put(207,393.17){\rule{0.900pt}{0.400pt}}
\multiput(209.13,394.17)(-2.132,-2.000){2}{\rule{0.450pt}{0.400pt}}
\put(204,391.67){\rule{0.723pt}{0.400pt}}
\multiput(205.50,392.17)(-1.500,-1.000){2}{\rule{0.361pt}{0.400pt}}
\put(201,390.17){\rule{0.700pt}{0.400pt}}
\multiput(202.55,391.17)(-1.547,-2.000){2}{\rule{0.350pt}{0.400pt}}
\put(197,388.17){\rule{0.900pt}{0.400pt}}
\multiput(199.13,389.17)(-2.132,-2.000){2}{\rule{0.450pt}{0.400pt}}
\put(194,386.17){\rule{0.700pt}{0.400pt}}
\multiput(195.55,387.17)(-1.547,-2.000){2}{\rule{0.350pt}{0.400pt}}
\put(190,384.17){\rule{0.900pt}{0.400pt}}
\multiput(192.13,385.17)(-2.132,-2.000){2}{\rule{0.450pt}{0.400pt}}
\put(187,382.17){\rule{0.700pt}{0.400pt}}
\multiput(188.55,383.17)(-1.547,-2.000){2}{\rule{0.350pt}{0.400pt}}
\put(184,380.67){\rule{0.723pt}{0.400pt}}
\multiput(185.50,381.17)(-1.500,-1.000){2}{\rule{0.361pt}{0.400pt}}
\put(180,379.17){\rule{0.900pt}{0.400pt}}
\multiput(182.13,380.17)(-2.132,-2.000){2}{\rule{0.450pt}{0.400pt}}
\put(177,377.17){\rule{0.700pt}{0.400pt}}
\multiput(178.55,378.17)(-1.547,-2.000){2}{\rule{0.350pt}{0.400pt}}
\put(173,375.17){\rule{0.900pt}{0.400pt}}
\multiput(175.13,376.17)(-2.132,-2.000){2}{\rule{0.450pt}{0.400pt}}
\put(170,373.17){\rule{0.700pt}{0.400pt}}
\multiput(171.55,374.17)(-1.547,-2.000){2}{\rule{0.350pt}{0.400pt}}
\put(167,371.17){\rule{0.700pt}{0.400pt}}
\multiput(168.55,372.17)(-1.547,-2.000){2}{\rule{0.350pt}{0.400pt}}
\put(163,369.67){\rule{0.964pt}{0.400pt}}
\multiput(165.00,370.17)(-2.000,-1.000){2}{\rule{0.482pt}{0.400pt}}
\put(160,368.17){\rule{0.700pt}{0.400pt}}
\multiput(161.55,369.17)(-1.547,-2.000){2}{\rule{0.350pt}{0.400pt}}
\put(156,366.17){\rule{0.900pt}{0.400pt}}
\multiput(158.13,367.17)(-2.132,-2.000){2}{\rule{0.450pt}{0.400pt}}
\put(153,364.17){\rule{0.700pt}{0.400pt}}
\multiput(154.55,365.17)(-1.547,-2.000){2}{\rule{0.350pt}{0.400pt}}
\put(1114.0,781.0){\rule[-0.200pt]{0.400pt}{0.964pt}}
\put(130.0,82.0){\rule[-0.200pt]{0.400pt}{187.179pt}}
\put(130.0,82.0){\rule[-0.200pt]{315.338pt}{0.400pt}}
\put(1439.0,82.0){\rule[-0.200pt]{0.400pt}{187.179pt}}
\put(130.0,859.0){\rule[-0.200pt]{315.338pt}{0.400pt}}
\end{picture}

Plot for Ball 3:\\
% GNUPLOT: LaTeX picture
\setlength{\unitlength}{0.240900pt}
\ifx\plotpoint\undefined\newsavebox{\plotpoint}\fi
\sbox{\plotpoint}{\rule[-0.200pt]{0.400pt}{0.400pt}}%
\begin{picture}(1500,900)(0,0)
\sbox{\plotpoint}{\rule[-0.200pt]{0.400pt}{0.400pt}}%
\put(130.0,90.0){\rule[-0.200pt]{4.818pt}{0.400pt}}
\put(110,90){\makebox(0,0)[r]{ 0}}
\put(1419.0,90.0){\rule[-0.200pt]{4.818pt}{0.400pt}}
\put(130.0,242.0){\rule[-0.200pt]{4.818pt}{0.400pt}}
\put(110,242){\makebox(0,0)[r]{ 0.2}}
\put(1419.0,242.0){\rule[-0.200pt]{4.818pt}{0.400pt}}
\put(130.0,394.0){\rule[-0.200pt]{4.818pt}{0.400pt}}
\put(110,394){\makebox(0,0)[r]{ 0.4}}
\put(1419.0,394.0){\rule[-0.200pt]{4.818pt}{0.400pt}}
\put(130.0,547.0){\rule[-0.200pt]{4.818pt}{0.400pt}}
\put(110,547){\makebox(0,0)[r]{ 0.6}}
\put(1419.0,547.0){\rule[-0.200pt]{4.818pt}{0.400pt}}
\put(130.0,699.0){\rule[-0.200pt]{4.818pt}{0.400pt}}
\put(110,699){\makebox(0,0)[r]{ 0.8}}
\put(1419.0,699.0){\rule[-0.200pt]{4.818pt}{0.400pt}}
\put(130.0,851.0){\rule[-0.200pt]{4.818pt}{0.400pt}}
\put(110,851){\makebox(0,0)[r]{ 1}}
\put(1419.0,851.0){\rule[-0.200pt]{4.818pt}{0.400pt}}
\put(130.0,82.0){\rule[-0.200pt]{0.400pt}{4.818pt}}
\put(130,41){\makebox(0,0){ 0}}
\put(130.0,839.0){\rule[-0.200pt]{0.400pt}{4.818pt}}
\put(392.0,82.0){\rule[-0.200pt]{0.400pt}{4.818pt}}
\put(392,41){\makebox(0,0){ 0.2}}
\put(392.0,839.0){\rule[-0.200pt]{0.400pt}{4.818pt}}
\put(654.0,82.0){\rule[-0.200pt]{0.400pt}{4.818pt}}
\put(654,41){\makebox(0,0){ 0.4}}
\put(654.0,839.0){\rule[-0.200pt]{0.400pt}{4.818pt}}
\put(915.0,82.0){\rule[-0.200pt]{0.400pt}{4.818pt}}
\put(915,41){\makebox(0,0){ 0.6}}
\put(915.0,839.0){\rule[-0.200pt]{0.400pt}{4.818pt}}
\put(1177.0,82.0){\rule[-0.200pt]{0.400pt}{4.818pt}}
\put(1177,41){\makebox(0,0){ 0.8}}
\put(1177.0,839.0){\rule[-0.200pt]{0.400pt}{4.818pt}}
\put(1439.0,82.0){\rule[-0.200pt]{0.400pt}{4.818pt}}
\put(1439,41){\makebox(0,0){ 1}}
\put(1439.0,839.0){\rule[-0.200pt]{0.400pt}{4.818pt}}
\put(130.0,82.0){\rule[-0.200pt]{0.400pt}{187.179pt}}
\put(130.0,82.0){\rule[-0.200pt]{315.338pt}{0.400pt}}
\put(1439.0,82.0){\rule[-0.200pt]{0.400pt}{187.179pt}}
\put(130.0,859.0){\rule[-0.200pt]{315.338pt}{0.400pt}}
\put(1279,819){\makebox(0,0)[r]{'-'}}
\put(1299.0,819.0){\rule[-0.200pt]{24.090pt}{0.400pt}}
\put(1056,703){\usebox{\plotpoint}}
\multiput(1053.92,703.61)(-0.462,0.447){3}{\rule{0.500pt}{0.108pt}}
\multiput(1054.96,702.17)(-1.962,3.000){2}{\rule{0.250pt}{0.400pt}}
\put(1050,706.17){\rule{0.700pt}{0.400pt}}
\multiput(1051.55,705.17)(-1.547,2.000){2}{\rule{0.350pt}{0.400pt}}
\multiput(1047.37,708.61)(-0.685,0.447){3}{\rule{0.633pt}{0.108pt}}
\multiput(1048.69,707.17)(-2.685,3.000){2}{\rule{0.317pt}{0.400pt}}
\put(1043,711.17){\rule{0.700pt}{0.400pt}}
\multiput(1044.55,710.17)(-1.547,2.000){2}{\rule{0.350pt}{0.400pt}}
\multiput(1040.92,713.61)(-0.462,0.447){3}{\rule{0.500pt}{0.108pt}}
\multiput(1041.96,712.17)(-1.962,3.000){2}{\rule{0.250pt}{0.400pt}}
\put(1036,716.17){\rule{0.900pt}{0.400pt}}
\multiput(1038.13,715.17)(-2.132,2.000){2}{\rule{0.450pt}{0.400pt}}
\multiput(1033.92,718.61)(-0.462,0.447){3}{\rule{0.500pt}{0.108pt}}
\multiput(1034.96,717.17)(-1.962,3.000){2}{\rule{0.250pt}{0.400pt}}
\put(1029,721.17){\rule{0.900pt}{0.400pt}}
\multiput(1031.13,720.17)(-2.132,2.000){2}{\rule{0.450pt}{0.400pt}}
\multiput(1026.92,723.61)(-0.462,0.447){3}{\rule{0.500pt}{0.108pt}}
\multiput(1027.96,722.17)(-1.962,3.000){2}{\rule{0.250pt}{0.400pt}}
\put(1023,726.17){\rule{0.700pt}{0.400pt}}
\multiput(1024.55,725.17)(-1.547,2.000){2}{\rule{0.350pt}{0.400pt}}
\multiput(1020.37,728.61)(-0.685,0.447){3}{\rule{0.633pt}{0.108pt}}
\multiput(1021.69,727.17)(-2.685,3.000){2}{\rule{0.317pt}{0.400pt}}
\put(1016,731.17){\rule{0.700pt}{0.400pt}}
\multiput(1017.55,730.17)(-1.547,2.000){2}{\rule{0.350pt}{0.400pt}}
\multiput(1013.92,733.61)(-0.462,0.447){3}{\rule{0.500pt}{0.108pt}}
\multiput(1014.96,732.17)(-1.962,3.000){2}{\rule{0.250pt}{0.400pt}}
\put(1009,736.17){\rule{0.900pt}{0.400pt}}
\multiput(1011.13,735.17)(-2.132,2.000){2}{\rule{0.450pt}{0.400pt}}
\multiput(1006.92,738.61)(-0.462,0.447){3}{\rule{0.500pt}{0.108pt}}
\multiput(1007.96,737.17)(-1.962,3.000){2}{\rule{0.250pt}{0.400pt}}
\put(1003,741.17){\rule{0.700pt}{0.400pt}}
\multiput(1004.55,740.17)(-1.547,2.000){2}{\rule{0.350pt}{0.400pt}}
\multiput(1000.37,743.61)(-0.685,0.447){3}{\rule{0.633pt}{0.108pt}}
\multiput(1001.69,742.17)(-2.685,3.000){2}{\rule{0.317pt}{0.400pt}}
\put(996,746.17){\rule{0.700pt}{0.400pt}}
\multiput(997.55,745.17)(-1.547,2.000){2}{\rule{0.350pt}{0.400pt}}
\multiput(993.37,748.61)(-0.685,0.447){3}{\rule{0.633pt}{0.108pt}}
\multiput(994.69,747.17)(-2.685,3.000){2}{\rule{0.317pt}{0.400pt}}
\multiput(989.92,751.61)(-0.462,0.447){3}{\rule{0.500pt}{0.108pt}}
\multiput(990.96,750.17)(-1.962,3.000){2}{\rule{0.250pt}{0.400pt}}
\put(986,754.17){\rule{0.700pt}{0.400pt}}
\multiput(987.55,753.17)(-1.547,2.000){2}{\rule{0.350pt}{0.400pt}}
\multiput(983.37,756.61)(-0.685,0.447){3}{\rule{0.633pt}{0.108pt}}
\multiput(984.69,755.17)(-2.685,3.000){2}{\rule{0.317pt}{0.400pt}}
\put(979,759.17){\rule{0.700pt}{0.400pt}}
\multiput(980.55,758.17)(-1.547,2.000){2}{\rule{0.350pt}{0.400pt}}
\multiput(976.92,761.61)(-0.462,0.447){3}{\rule{0.500pt}{0.108pt}}
\multiput(977.96,760.17)(-1.962,3.000){2}{\rule{0.250pt}{0.400pt}}
\put(972,764.17){\rule{0.900pt}{0.400pt}}
\multiput(974.13,763.17)(-2.132,2.000){2}{\rule{0.450pt}{0.400pt}}
\multiput(969.92,766.61)(-0.462,0.447){3}{\rule{0.500pt}{0.108pt}}
\multiput(970.96,765.17)(-1.962,3.000){2}{\rule{0.250pt}{0.400pt}}
\put(966,769.17){\rule{0.700pt}{0.400pt}}
\multiput(967.55,768.17)(-1.547,2.000){2}{\rule{0.350pt}{0.400pt}}
\multiput(963.37,771.61)(-0.685,0.447){3}{\rule{0.633pt}{0.108pt}}
\multiput(964.69,770.17)(-2.685,3.000){2}{\rule{0.317pt}{0.400pt}}
\put(959,774.17){\rule{0.700pt}{0.400pt}}
\multiput(960.55,773.17)(-1.547,2.000){2}{\rule{0.350pt}{0.400pt}}
\multiput(956.37,776.61)(-0.685,0.447){3}{\rule{0.633pt}{0.108pt}}
\multiput(957.69,775.17)(-2.685,3.000){2}{\rule{0.317pt}{0.400pt}}
\put(952,779.17){\rule{0.700pt}{0.400pt}}
\multiput(953.55,778.17)(-1.547,2.000){2}{\rule{0.350pt}{0.400pt}}
\multiput(949.92,781.61)(-0.462,0.447){3}{\rule{0.500pt}{0.108pt}}
\multiput(950.96,780.17)(-1.962,3.000){2}{\rule{0.250pt}{0.400pt}}
\put(945,784.17){\rule{0.900pt}{0.400pt}}
\multiput(947.13,783.17)(-2.132,2.000){2}{\rule{0.450pt}{0.400pt}}
\multiput(942.92,786.61)(-0.462,0.447){3}{\rule{0.500pt}{0.108pt}}
\multiput(943.96,785.17)(-1.962,3.000){2}{\rule{0.250pt}{0.400pt}}
\put(939,789.17){\rule{0.700pt}{0.400pt}}
\multiput(940.55,788.17)(-1.547,2.000){2}{\rule{0.350pt}{0.400pt}}
\multiput(936.37,791.61)(-0.685,0.447){3}{\rule{0.633pt}{0.108pt}}
\multiput(937.69,790.17)(-2.685,3.000){2}{\rule{0.317pt}{0.400pt}}
\put(932,794.17){\rule{0.700pt}{0.400pt}}
\multiput(933.55,793.17)(-1.547,2.000){2}{\rule{0.350pt}{0.400pt}}
\multiput(929.92,796.61)(-0.462,0.447){3}{\rule{0.500pt}{0.108pt}}
\multiput(930.96,795.17)(-1.962,3.000){2}{\rule{0.250pt}{0.400pt}}
\put(925,799.17){\rule{0.900pt}{0.400pt}}
\multiput(927.13,798.17)(-2.132,2.000){2}{\rule{0.450pt}{0.400pt}}
\multiput(922.92,801.61)(-0.462,0.447){3}{\rule{0.500pt}{0.108pt}}
\multiput(923.96,800.17)(-1.962,3.000){2}{\rule{0.250pt}{0.400pt}}
\put(918,804.17){\rule{0.900pt}{0.400pt}}
\multiput(920.13,803.17)(-2.132,2.000){2}{\rule{0.450pt}{0.400pt}}
\multiput(915.92,806.61)(-0.462,0.447){3}{\rule{0.500pt}{0.108pt}}
\multiput(916.96,805.17)(-1.962,3.000){2}{\rule{0.250pt}{0.400pt}}
\put(912,809.17){\rule{0.700pt}{0.400pt}}
\multiput(913.55,808.17)(-1.547,2.000){2}{\rule{0.350pt}{0.400pt}}
\multiput(909.37,811.61)(-0.685,0.447){3}{\rule{0.633pt}{0.108pt}}
\multiput(910.69,810.17)(-2.685,3.000){2}{\rule{0.317pt}{0.400pt}}
\put(905,814.17){\rule{0.700pt}{0.400pt}}
\multiput(906.55,813.17)(-1.547,2.000){2}{\rule{0.350pt}{0.400pt}}
\multiput(902.92,816.61)(-0.462,0.447){3}{\rule{0.500pt}{0.108pt}}
\multiput(903.96,815.17)(-1.962,3.000){2}{\rule{0.250pt}{0.400pt}}
\put(898,819.17){\rule{0.900pt}{0.400pt}}
\multiput(900.13,818.17)(-2.132,2.000){2}{\rule{0.450pt}{0.400pt}}
\multiput(895.92,821.61)(-0.462,0.447){3}{\rule{0.500pt}{0.108pt}}
\multiput(896.96,820.17)(-1.962,3.000){2}{\rule{0.250pt}{0.400pt}}
\put(892,824.17){\rule{0.700pt}{0.400pt}}
\multiput(893.55,823.17)(-1.547,2.000){2}{\rule{0.350pt}{0.400pt}}
\multiput(889.37,826.61)(-0.685,0.447){3}{\rule{0.633pt}{0.108pt}}
\multiput(890.69,825.17)(-2.685,3.000){2}{\rule{0.317pt}{0.400pt}}
\put(885,829.17){\rule{0.700pt}{0.400pt}}
\multiput(886.55,828.17)(-1.547,2.000){2}{\rule{0.350pt}{0.400pt}}
\multiput(882.37,831.61)(-0.685,0.447){3}{\rule{0.633pt}{0.108pt}}
\multiput(883.69,830.17)(-2.685,3.000){2}{\rule{0.317pt}{0.400pt}}
\put(878,834.17){\rule{0.700pt}{0.400pt}}
\multiput(879.55,833.17)(-1.547,2.000){2}{\rule{0.350pt}{0.400pt}}
\put(875,834.17){\rule{0.700pt}{0.400pt}}
\multiput(876.55,835.17)(-1.547,-2.000){2}{\rule{0.350pt}{0.400pt}}
\multiput(872.37,832.95)(-0.685,-0.447){3}{\rule{0.633pt}{0.108pt}}
\multiput(873.69,833.17)(-2.685,-3.000){2}{\rule{0.317pt}{0.400pt}}
\put(868,829.17){\rule{0.700pt}{0.400pt}}
\multiput(869.55,830.17)(-1.547,-2.000){2}{\rule{0.350pt}{0.400pt}}
\multiput(865.92,827.95)(-0.462,-0.447){3}{\rule{0.500pt}{0.108pt}}
\multiput(866.96,828.17)(-1.962,-3.000){2}{\rule{0.250pt}{0.400pt}}
\put(861,824.17){\rule{0.900pt}{0.400pt}}
\multiput(863.13,825.17)(-2.132,-2.000){2}{\rule{0.450pt}{0.400pt}}
\multiput(858.92,822.95)(-0.462,-0.447){3}{\rule{0.500pt}{0.108pt}}
\multiput(859.96,823.17)(-1.962,-3.000){2}{\rule{0.250pt}{0.400pt}}
\put(855,819.17){\rule{0.700pt}{0.400pt}}
\multiput(856.55,820.17)(-1.547,-2.000){2}{\rule{0.350pt}{0.400pt}}
\multiput(852.37,817.95)(-0.685,-0.447){3}{\rule{0.633pt}{0.108pt}}
\multiput(853.69,818.17)(-2.685,-3.000){2}{\rule{0.317pt}{0.400pt}}
\put(848,814.17){\rule{0.700pt}{0.400pt}}
\multiput(849.55,815.17)(-1.547,-2.000){2}{\rule{0.350pt}{0.400pt}}
\multiput(845.37,812.95)(-0.685,-0.447){3}{\rule{0.633pt}{0.108pt}}
\multiput(846.69,813.17)(-2.685,-3.000){2}{\rule{0.317pt}{0.400pt}}
\put(841,809.17){\rule{0.700pt}{0.400pt}}
\multiput(842.55,810.17)(-1.547,-2.000){2}{\rule{0.350pt}{0.400pt}}
\multiput(838.92,807.95)(-0.462,-0.447){3}{\rule{0.500pt}{0.108pt}}
\multiput(839.96,808.17)(-1.962,-3.000){2}{\rule{0.250pt}{0.400pt}}
\put(834,804.17){\rule{0.900pt}{0.400pt}}
\multiput(836.13,805.17)(-2.132,-2.000){2}{\rule{0.450pt}{0.400pt}}
\multiput(831.92,802.95)(-0.462,-0.447){3}{\rule{0.500pt}{0.108pt}}
\multiput(832.96,803.17)(-1.962,-3.000){2}{\rule{0.250pt}{0.400pt}}
\put(828,799.17){\rule{0.700pt}{0.400pt}}
\multiput(829.55,800.17)(-1.547,-2.000){2}{\rule{0.350pt}{0.400pt}}
\multiput(825.37,797.95)(-0.685,-0.447){3}{\rule{0.633pt}{0.108pt}}
\multiput(826.69,798.17)(-2.685,-3.000){2}{\rule{0.317pt}{0.400pt}}
\put(821,794.17){\rule{0.700pt}{0.400pt}}
\multiput(822.55,795.17)(-1.547,-2.000){2}{\rule{0.350pt}{0.400pt}}
\multiput(818.92,792.95)(-0.462,-0.447){3}{\rule{0.500pt}{0.108pt}}
\multiput(819.96,793.17)(-1.962,-3.000){2}{\rule{0.250pt}{0.400pt}}
\put(814,789.17){\rule{0.900pt}{0.400pt}}
\multiput(816.13,790.17)(-2.132,-2.000){2}{\rule{0.450pt}{0.400pt}}
\multiput(811.92,787.95)(-0.462,-0.447){3}{\rule{0.500pt}{0.108pt}}
\multiput(812.96,788.17)(-1.962,-3.000){2}{\rule{0.250pt}{0.400pt}}
\put(807,784.17){\rule{0.900pt}{0.400pt}}
\multiput(809.13,785.17)(-2.132,-2.000){2}{\rule{0.450pt}{0.400pt}}
\multiput(804.92,782.95)(-0.462,-0.447){3}{\rule{0.500pt}{0.108pt}}
\multiput(805.96,783.17)(-1.962,-3.000){2}{\rule{0.250pt}{0.400pt}}
\put(801,779.17){\rule{0.700pt}{0.400pt}}
\multiput(802.55,780.17)(-1.547,-2.000){2}{\rule{0.350pt}{0.400pt}}
\multiput(798.37,777.95)(-0.685,-0.447){3}{\rule{0.633pt}{0.108pt}}
\multiput(799.69,778.17)(-2.685,-3.000){2}{\rule{0.317pt}{0.400pt}}
\put(794,774.17){\rule{0.700pt}{0.400pt}}
\multiput(795.55,775.17)(-1.547,-2.000){2}{\rule{0.350pt}{0.400pt}}
\multiput(791.92,772.95)(-0.462,-0.447){3}{\rule{0.500pt}{0.108pt}}
\multiput(792.96,773.17)(-1.962,-3.000){2}{\rule{0.250pt}{0.400pt}}
\put(787,769.17){\rule{0.900pt}{0.400pt}}
\multiput(789.13,770.17)(-2.132,-2.000){2}{\rule{0.450pt}{0.400pt}}
\multiput(784.92,767.95)(-0.462,-0.447){3}{\rule{0.500pt}{0.108pt}}
\multiput(785.96,768.17)(-1.962,-3.000){2}{\rule{0.250pt}{0.400pt}}
\put(781,764.17){\rule{0.700pt}{0.400pt}}
\multiput(782.55,765.17)(-1.547,-2.000){2}{\rule{0.350pt}{0.400pt}}
\multiput(778.37,762.95)(-0.685,-0.447){3}{\rule{0.633pt}{0.108pt}}
\multiput(779.69,763.17)(-2.685,-3.000){2}{\rule{0.317pt}{0.400pt}}
\put(774,759.17){\rule{0.700pt}{0.400pt}}
\multiput(775.55,760.17)(-1.547,-2.000){2}{\rule{0.350pt}{0.400pt}}
\multiput(771.37,757.95)(-0.685,-0.447){3}{\rule{0.633pt}{0.108pt}}
\multiput(772.69,758.17)(-2.685,-3.000){2}{\rule{0.317pt}{0.400pt}}
\put(767,754.17){\rule{0.700pt}{0.400pt}}
\multiput(768.55,755.17)(-1.547,-2.000){2}{\rule{0.350pt}{0.400pt}}
\multiput(764.92,752.95)(-0.462,-0.447){3}{\rule{0.500pt}{0.108pt}}
\multiput(765.96,753.17)(-1.962,-3.000){2}{\rule{0.250pt}{0.400pt}}
\multiput(761.37,749.95)(-0.685,-0.447){3}{\rule{0.633pt}{0.108pt}}
\multiput(762.69,750.17)(-2.685,-3.000){2}{\rule{0.317pt}{0.400pt}}
\put(757,746.17){\rule{0.700pt}{0.400pt}}
\multiput(758.55,747.17)(-1.547,-2.000){2}{\rule{0.350pt}{0.400pt}}
\multiput(754.92,744.95)(-0.462,-0.447){3}{\rule{0.500pt}{0.108pt}}
\multiput(755.96,745.17)(-1.962,-3.000){2}{\rule{0.250pt}{0.400pt}}
\put(750,741.17){\rule{0.900pt}{0.400pt}}
\multiput(752.13,742.17)(-2.132,-2.000){2}{\rule{0.450pt}{0.400pt}}
\multiput(747.92,739.95)(-0.462,-0.447){3}{\rule{0.500pt}{0.108pt}}
\multiput(748.96,740.17)(-1.962,-3.000){2}{\rule{0.250pt}{0.400pt}}
\put(744,736.17){\rule{0.700pt}{0.400pt}}
\multiput(745.55,737.17)(-1.547,-2.000){2}{\rule{0.350pt}{0.400pt}}
\multiput(741.37,734.95)(-0.685,-0.447){3}{\rule{0.633pt}{0.108pt}}
\multiput(742.69,735.17)(-2.685,-3.000){2}{\rule{0.317pt}{0.400pt}}
\put(737,731.17){\rule{0.700pt}{0.400pt}}
\multiput(738.55,732.17)(-1.547,-2.000){2}{\rule{0.350pt}{0.400pt}}
\multiput(734.37,729.95)(-0.685,-0.447){3}{\rule{0.633pt}{0.108pt}}
\multiput(735.69,730.17)(-2.685,-3.000){2}{\rule{0.317pt}{0.400pt}}
\put(730,726.17){\rule{0.700pt}{0.400pt}}
\multiput(731.55,727.17)(-1.547,-2.000){2}{\rule{0.350pt}{0.400pt}}
\multiput(727.92,724.95)(-0.462,-0.447){3}{\rule{0.500pt}{0.108pt}}
\multiput(728.96,725.17)(-1.962,-3.000){2}{\rule{0.250pt}{0.400pt}}
\put(723,721.17){\rule{0.900pt}{0.400pt}}
\multiput(725.13,722.17)(-2.132,-2.000){2}{\rule{0.450pt}{0.400pt}}
\multiput(720.92,719.95)(-0.462,-0.447){3}{\rule{0.500pt}{0.108pt}}
\multiput(721.96,720.17)(-1.962,-3.000){2}{\rule{0.250pt}{0.400pt}}
\put(717,716.17){\rule{0.700pt}{0.400pt}}
\multiput(718.55,717.17)(-1.547,-2.000){2}{\rule{0.350pt}{0.400pt}}
\multiput(714.37,714.95)(-0.685,-0.447){3}{\rule{0.633pt}{0.108pt}}
\multiput(715.69,715.17)(-2.685,-3.000){2}{\rule{0.317pt}{0.400pt}}
\put(710,711.17){\rule{0.700pt}{0.400pt}}
\multiput(711.55,712.17)(-1.547,-2.000){2}{\rule{0.350pt}{0.400pt}}
\multiput(707.92,709.95)(-0.462,-0.447){3}{\rule{0.500pt}{0.108pt}}
\multiput(708.96,710.17)(-1.962,-3.000){2}{\rule{0.250pt}{0.400pt}}
\put(703,706.17){\rule{0.900pt}{0.400pt}}
\multiput(705.13,707.17)(-2.132,-2.000){2}{\rule{0.450pt}{0.400pt}}
\multiput(700.92,704.95)(-0.462,-0.447){3}{\rule{0.500pt}{0.108pt}}
\multiput(701.96,705.17)(-1.962,-3.000){2}{\rule{0.250pt}{0.400pt}}
\put(696,701.17){\rule{0.900pt}{0.400pt}}
\multiput(698.13,702.17)(-2.132,-2.000){2}{\rule{0.450pt}{0.400pt}}
\multiput(693.92,699.95)(-0.462,-0.447){3}{\rule{0.500pt}{0.108pt}}
\multiput(694.96,700.17)(-1.962,-3.000){2}{\rule{0.250pt}{0.400pt}}
\put(690,696.17){\rule{0.700pt}{0.400pt}}
\multiput(691.55,697.17)(-1.547,-2.000){2}{\rule{0.350pt}{0.400pt}}
\multiput(687.37,694.95)(-0.685,-0.447){3}{\rule{0.633pt}{0.108pt}}
\multiput(688.69,695.17)(-2.685,-3.000){2}{\rule{0.317pt}{0.400pt}}
\put(683,691.17){\rule{0.700pt}{0.400pt}}
\multiput(684.55,692.17)(-1.547,-2.000){2}{\rule{0.350pt}{0.400pt}}
\multiput(680.92,689.95)(-0.462,-0.447){3}{\rule{0.500pt}{0.108pt}}
\multiput(681.96,690.17)(-1.962,-3.000){2}{\rule{0.250pt}{0.400pt}}
\put(676,686.17){\rule{0.900pt}{0.400pt}}
\multiput(678.13,687.17)(-2.132,-2.000){2}{\rule{0.450pt}{0.400pt}}
\multiput(673.92,684.95)(-0.462,-0.447){3}{\rule{0.500pt}{0.108pt}}
\multiput(674.96,685.17)(-1.962,-3.000){2}{\rule{0.250pt}{0.400pt}}
\put(670,681.17){\rule{0.700pt}{0.400pt}}
\multiput(671.55,682.17)(-1.547,-2.000){2}{\rule{0.350pt}{0.400pt}}
\multiput(667.37,679.95)(-0.685,-0.447){3}{\rule{0.633pt}{0.108pt}}
\multiput(668.69,680.17)(-2.685,-3.000){2}{\rule{0.317pt}{0.400pt}}
\put(663,676.17){\rule{0.700pt}{0.400pt}}
\multiput(664.55,677.17)(-1.547,-2.000){2}{\rule{0.350pt}{0.400pt}}
\multiput(660.37,674.95)(-0.685,-0.447){3}{\rule{0.633pt}{0.108pt}}
\multiput(661.69,675.17)(-2.685,-3.000){2}{\rule{0.317pt}{0.400pt}}
\put(656,671.17){\rule{0.700pt}{0.400pt}}
\multiput(657.55,672.17)(-1.547,-2.000){2}{\rule{0.350pt}{0.400pt}}
\multiput(653.92,669.95)(-0.462,-0.447){3}{\rule{0.500pt}{0.108pt}}
\multiput(654.96,670.17)(-1.962,-3.000){2}{\rule{0.250pt}{0.400pt}}
\put(649,666.17){\rule{0.900pt}{0.400pt}}
\multiput(651.13,667.17)(-2.132,-2.000){2}{\rule{0.450pt}{0.400pt}}
\multiput(646.92,664.95)(-0.462,-0.447){3}{\rule{0.500pt}{0.108pt}}
\multiput(647.96,665.17)(-1.962,-3.000){2}{\rule{0.250pt}{0.400pt}}
\put(643,661.17){\rule{0.700pt}{0.400pt}}
\multiput(644.55,662.17)(-1.547,-2.000){2}{\rule{0.350pt}{0.400pt}}
\multiput(640.37,659.95)(-0.685,-0.447){3}{\rule{0.633pt}{0.108pt}}
\multiput(641.69,660.17)(-2.685,-3.000){2}{\rule{0.317pt}{0.400pt}}
\put(636,656.17){\rule{0.700pt}{0.400pt}}
\multiput(637.55,657.17)(-1.547,-2.000){2}{\rule{0.350pt}{0.400pt}}
\multiput(633.92,654.95)(-0.462,-0.447){3}{\rule{0.500pt}{0.108pt}}
\multiput(634.96,655.17)(-1.962,-3.000){2}{\rule{0.250pt}{0.400pt}}
\put(629,651.17){\rule{0.900pt}{0.400pt}}
\multiput(631.13,652.17)(-2.132,-2.000){2}{\rule{0.450pt}{0.400pt}}
\multiput(626.92,649.95)(-0.462,-0.447){3}{\rule{0.500pt}{0.108pt}}
\multiput(627.96,650.17)(-1.962,-3.000){2}{\rule{0.250pt}{0.400pt}}
\multiput(623.37,646.95)(-0.685,-0.447){3}{\rule{0.633pt}{0.108pt}}
\multiput(624.69,647.17)(-2.685,-3.000){2}{\rule{0.317pt}{0.400pt}}
\put(619,643.17){\rule{0.700pt}{0.400pt}}
\multiput(620.55,644.17)(-1.547,-2.000){2}{\rule{0.350pt}{0.400pt}}
\multiput(616.92,641.95)(-0.462,-0.447){3}{\rule{0.500pt}{0.108pt}}
\multiput(617.96,642.17)(-1.962,-3.000){2}{\rule{0.250pt}{0.400pt}}
\put(612,638.17){\rule{0.900pt}{0.400pt}}
\multiput(614.13,639.17)(-2.132,-2.000){2}{\rule{0.450pt}{0.400pt}}
\multiput(609.92,636.95)(-0.462,-0.447){3}{\rule{0.500pt}{0.108pt}}
\multiput(610.96,637.17)(-1.962,-3.000){2}{\rule{0.250pt}{0.400pt}}
\put(606,633.17){\rule{0.700pt}{0.400pt}}
\multiput(607.55,634.17)(-1.547,-2.000){2}{\rule{0.350pt}{0.400pt}}
\multiput(603.37,631.95)(-0.685,-0.447){3}{\rule{0.633pt}{0.108pt}}
\multiput(604.69,632.17)(-2.685,-3.000){2}{\rule{0.317pt}{0.400pt}}
\put(599,628.17){\rule{0.700pt}{0.400pt}}
\multiput(600.55,629.17)(-1.547,-2.000){2}{\rule{0.350pt}{0.400pt}}
\multiput(596.92,626.95)(-0.462,-0.447){3}{\rule{0.500pt}{0.108pt}}
\multiput(597.96,627.17)(-1.962,-3.000){2}{\rule{0.250pt}{0.400pt}}
\put(592,623.17){\rule{0.900pt}{0.400pt}}
\multiput(594.13,624.17)(-2.132,-2.000){2}{\rule{0.450pt}{0.400pt}}
\multiput(589.92,621.95)(-0.462,-0.447){3}{\rule{0.500pt}{0.108pt}}
\multiput(590.96,622.17)(-1.962,-3.000){2}{\rule{0.250pt}{0.400pt}}
\put(585,618.17){\rule{0.900pt}{0.400pt}}
\multiput(587.13,619.17)(-2.132,-2.000){2}{\rule{0.450pt}{0.400pt}}
\multiput(582.92,616.95)(-0.462,-0.447){3}{\rule{0.500pt}{0.108pt}}
\multiput(583.96,617.17)(-1.962,-3.000){2}{\rule{0.250pt}{0.400pt}}
\put(579,613.17){\rule{0.700pt}{0.400pt}}
\multiput(580.55,614.17)(-1.547,-2.000){2}{\rule{0.350pt}{0.400pt}}
\multiput(576.37,611.95)(-0.685,-0.447){3}{\rule{0.633pt}{0.108pt}}
\multiput(577.69,612.17)(-2.685,-3.000){2}{\rule{0.317pt}{0.400pt}}
\put(572,608.17){\rule{0.700pt}{0.400pt}}
\multiput(573.55,609.17)(-1.547,-2.000){2}{\rule{0.350pt}{0.400pt}}
\multiput(569.92,606.95)(-0.462,-0.447){3}{\rule{0.500pt}{0.108pt}}
\multiput(570.96,607.17)(-1.962,-3.000){2}{\rule{0.250pt}{0.400pt}}
\put(565,603.17){\rule{0.900pt}{0.400pt}}
\multiput(567.13,604.17)(-2.132,-2.000){2}{\rule{0.450pt}{0.400pt}}
\multiput(562.92,601.95)(-0.462,-0.447){3}{\rule{0.500pt}{0.108pt}}
\multiput(563.96,602.17)(-1.962,-3.000){2}{\rule{0.250pt}{0.400pt}}
\put(559,598.17){\rule{0.700pt}{0.400pt}}
\multiput(560.55,599.17)(-1.547,-2.000){2}{\rule{0.350pt}{0.400pt}}
\multiput(556.37,596.95)(-0.685,-0.447){3}{\rule{0.633pt}{0.108pt}}
\multiput(557.69,597.17)(-2.685,-3.000){2}{\rule{0.317pt}{0.400pt}}
\put(552,593.17){\rule{0.700pt}{0.400pt}}
\multiput(553.55,594.17)(-1.547,-2.000){2}{\rule{0.350pt}{0.400pt}}
\multiput(549.37,591.95)(-0.685,-0.447){3}{\rule{0.633pt}{0.108pt}}
\multiput(550.69,592.17)(-2.685,-3.000){2}{\rule{0.317pt}{0.400pt}}
\put(545,588.17){\rule{0.700pt}{0.400pt}}
\multiput(546.55,589.17)(-1.547,-2.000){2}{\rule{0.350pt}{0.400pt}}
\multiput(542.92,586.95)(-0.462,-0.447){3}{\rule{0.500pt}{0.108pt}}
\multiput(543.96,587.17)(-1.962,-3.000){2}{\rule{0.250pt}{0.400pt}}
\put(538,583.17){\rule{0.900pt}{0.400pt}}
\multiput(540.13,584.17)(-2.132,-2.000){2}{\rule{0.450pt}{0.400pt}}
\multiput(535.92,581.95)(-0.462,-0.447){3}{\rule{0.500pt}{0.108pt}}
\multiput(536.96,582.17)(-1.962,-3.000){2}{\rule{0.250pt}{0.400pt}}
\put(532,578.17){\rule{0.700pt}{0.400pt}}
\multiput(533.55,579.17)(-1.547,-2.000){2}{\rule{0.350pt}{0.400pt}}
\multiput(529.37,576.95)(-0.685,-0.447){3}{\rule{0.633pt}{0.108pt}}
\multiput(530.69,577.17)(-2.685,-3.000){2}{\rule{0.317pt}{0.400pt}}
\put(525,573.17){\rule{0.700pt}{0.400pt}}
\multiput(526.55,574.17)(-1.547,-2.000){2}{\rule{0.350pt}{0.400pt}}
\multiput(522.92,571.95)(-0.462,-0.447){3}{\rule{0.500pt}{0.108pt}}
\multiput(523.96,572.17)(-1.962,-3.000){2}{\rule{0.250pt}{0.400pt}}
\put(518,568.17){\rule{0.900pt}{0.400pt}}
\multiput(520.13,569.17)(-2.132,-2.000){2}{\rule{0.450pt}{0.400pt}}
\multiput(515.92,566.95)(-0.462,-0.447){3}{\rule{0.500pt}{0.108pt}}
\multiput(516.96,567.17)(-1.962,-3.000){2}{\rule{0.250pt}{0.400pt}}
\put(511,563.17){\rule{0.900pt}{0.400pt}}
\multiput(513.13,564.17)(-2.132,-2.000){2}{\rule{0.450pt}{0.400pt}}
\multiput(508.92,561.95)(-0.462,-0.447){3}{\rule{0.500pt}{0.108pt}}
\multiput(509.96,562.17)(-1.962,-3.000){2}{\rule{0.250pt}{0.400pt}}
\put(505,558.17){\rule{0.700pt}{0.400pt}}
\multiput(506.55,559.17)(-1.547,-2.000){2}{\rule{0.350pt}{0.400pt}}
\multiput(502.37,556.95)(-0.685,-0.447){3}{\rule{0.633pt}{0.108pt}}
\multiput(503.69,557.17)(-2.685,-3.000){2}{\rule{0.317pt}{0.400pt}}
\put(498,553.17){\rule{0.700pt}{0.400pt}}
\multiput(499.55,554.17)(-1.547,-2.000){2}{\rule{0.350pt}{0.400pt}}
\multiput(495.92,551.95)(-0.462,-0.447){3}{\rule{0.500pt}{0.108pt}}
\multiput(496.96,552.17)(-1.962,-3.000){2}{\rule{0.250pt}{0.400pt}}
\multiput(492.37,548.95)(-0.685,-0.447){3}{\rule{0.633pt}{0.108pt}}
\multiput(493.69,549.17)(-2.685,-3.000){2}{\rule{0.317pt}{0.400pt}}
\put(488,545.17){\rule{0.700pt}{0.400pt}}
\multiput(489.55,546.17)(-1.547,-2.000){2}{\rule{0.350pt}{0.400pt}}
\multiput(485.92,543.95)(-0.462,-0.447){3}{\rule{0.500pt}{0.108pt}}
\multiput(486.96,544.17)(-1.962,-3.000){2}{\rule{0.250pt}{0.400pt}}
\put(481,540.17){\rule{0.900pt}{0.400pt}}
\multiput(483.13,541.17)(-2.132,-2.000){2}{\rule{0.450pt}{0.400pt}}
\multiput(478.92,538.95)(-0.462,-0.447){3}{\rule{0.500pt}{0.108pt}}
\multiput(479.96,539.17)(-1.962,-3.000){2}{\rule{0.250pt}{0.400pt}}
\put(474,535.17){\rule{0.900pt}{0.400pt}}
\multiput(476.13,536.17)(-2.132,-2.000){2}{\rule{0.450pt}{0.400pt}}
\multiput(471.92,533.95)(-0.462,-0.447){3}{\rule{0.500pt}{0.108pt}}
\multiput(472.96,534.17)(-1.962,-3.000){2}{\rule{0.250pt}{0.400pt}}
\put(468,530.17){\rule{0.700pt}{0.400pt}}
\multiput(469.55,531.17)(-1.547,-2.000){2}{\rule{0.350pt}{0.400pt}}
\multiput(465.37,528.95)(-0.685,-0.447){3}{\rule{0.633pt}{0.108pt}}
\multiput(466.69,529.17)(-2.685,-3.000){2}{\rule{0.317pt}{0.400pt}}
\put(461,525.17){\rule{0.700pt}{0.400pt}}
\multiput(462.55,526.17)(-1.547,-2.000){2}{\rule{0.350pt}{0.400pt}}
\multiput(458.92,523.95)(-0.462,-0.447){3}{\rule{0.500pt}{0.108pt}}
\multiput(459.96,524.17)(-1.962,-3.000){2}{\rule{0.250pt}{0.400pt}}
\put(454,520.17){\rule{0.900pt}{0.400pt}}
\multiput(456.13,521.17)(-2.132,-2.000){2}{\rule{0.450pt}{0.400pt}}
\multiput(451.92,518.95)(-0.462,-0.447){3}{\rule{0.500pt}{0.108pt}}
\multiput(452.96,519.17)(-1.962,-3.000){2}{\rule{0.250pt}{0.400pt}}
\put(448,515.17){\rule{0.700pt}{0.400pt}}
\multiput(449.55,516.17)(-1.547,-2.000){2}{\rule{0.350pt}{0.400pt}}
\multiput(445.37,513.95)(-0.685,-0.447){3}{\rule{0.633pt}{0.108pt}}
\multiput(446.69,514.17)(-2.685,-3.000){2}{\rule{0.317pt}{0.400pt}}
\put(441,510.17){\rule{0.700pt}{0.400pt}}
\multiput(442.55,511.17)(-1.547,-2.000){2}{\rule{0.350pt}{0.400pt}}
\multiput(438.37,508.95)(-0.685,-0.447){3}{\rule{0.633pt}{0.108pt}}
\multiput(439.69,509.17)(-2.685,-3.000){2}{\rule{0.317pt}{0.400pt}}
\put(434,505.17){\rule{0.700pt}{0.400pt}}
\multiput(435.55,506.17)(-1.547,-2.000){2}{\rule{0.350pt}{0.400pt}}
\multiput(431.92,503.95)(-0.462,-0.447){3}{\rule{0.500pt}{0.108pt}}
\multiput(432.96,504.17)(-1.962,-3.000){2}{\rule{0.250pt}{0.400pt}}
\put(427,500.17){\rule{0.900pt}{0.400pt}}
\multiput(429.13,501.17)(-2.132,-2.000){2}{\rule{0.450pt}{0.400pt}}
\multiput(424.92,498.95)(-0.462,-0.447){3}{\rule{0.500pt}{0.108pt}}
\multiput(425.96,499.17)(-1.962,-3.000){2}{\rule{0.250pt}{0.400pt}}
\put(421,495.17){\rule{0.700pt}{0.400pt}}
\multiput(422.55,496.17)(-1.547,-2.000){2}{\rule{0.350pt}{0.400pt}}
\multiput(418.37,493.95)(-0.685,-0.447){3}{\rule{0.633pt}{0.108pt}}
\multiput(419.69,494.17)(-2.685,-3.000){2}{\rule{0.317pt}{0.400pt}}
\put(414,490.17){\rule{0.700pt}{0.400pt}}
\multiput(415.55,491.17)(-1.547,-2.000){2}{\rule{0.350pt}{0.400pt}}
\multiput(411.92,488.95)(-0.462,-0.447){3}{\rule{0.500pt}{0.108pt}}
\multiput(412.96,489.17)(-1.962,-3.000){2}{\rule{0.250pt}{0.400pt}}
\put(407,485.17){\rule{0.900pt}{0.400pt}}
\multiput(409.13,486.17)(-2.132,-2.000){2}{\rule{0.450pt}{0.400pt}}
\multiput(404.92,483.95)(-0.462,-0.447){3}{\rule{0.500pt}{0.108pt}}
\multiput(405.96,484.17)(-1.962,-3.000){2}{\rule{0.250pt}{0.400pt}}
\put(400,480.17){\rule{0.900pt}{0.400pt}}
\multiput(402.13,481.17)(-2.132,-2.000){2}{\rule{0.450pt}{0.400pt}}
\multiput(397.92,478.95)(-0.462,-0.447){3}{\rule{0.500pt}{0.108pt}}
\multiput(398.96,479.17)(-1.962,-3.000){2}{\rule{0.250pt}{0.400pt}}
\put(394,475.17){\rule{0.700pt}{0.400pt}}
\multiput(395.55,476.17)(-1.547,-2.000){2}{\rule{0.350pt}{0.400pt}}
\multiput(391.37,473.95)(-0.685,-0.447){3}{\rule{0.633pt}{0.108pt}}
\multiput(392.69,474.17)(-2.685,-3.000){2}{\rule{0.317pt}{0.400pt}}
\put(387,470.17){\rule{0.700pt}{0.400pt}}
\multiput(388.55,471.17)(-1.547,-2.000){2}{\rule{0.350pt}{0.400pt}}
\multiput(384.92,468.95)(-0.462,-0.447){3}{\rule{0.500pt}{0.108pt}}
\multiput(385.96,469.17)(-1.962,-3.000){2}{\rule{0.250pt}{0.400pt}}
\put(380,465.17){\rule{0.900pt}{0.400pt}}
\multiput(382.13,466.17)(-2.132,-2.000){2}{\rule{0.450pt}{0.400pt}}
\multiput(377.92,463.95)(-0.462,-0.447){3}{\rule{0.500pt}{0.108pt}}
\multiput(378.96,464.17)(-1.962,-3.000){2}{\rule{0.250pt}{0.400pt}}
\put(374,460.17){\rule{0.700pt}{0.400pt}}
\multiput(375.55,461.17)(-1.547,-2.000){2}{\rule{0.350pt}{0.400pt}}
\multiput(371.37,458.95)(-0.685,-0.447){3}{\rule{0.633pt}{0.108pt}}
\multiput(372.69,459.17)(-2.685,-3.000){2}{\rule{0.317pt}{0.400pt}}
\put(367,455.17){\rule{0.700pt}{0.400pt}}
\multiput(368.55,456.17)(-1.547,-2.000){2}{\rule{0.350pt}{0.400pt}}
\multiput(364.37,453.95)(-0.685,-0.447){3}{\rule{0.633pt}{0.108pt}}
\multiput(365.69,454.17)(-2.685,-3.000){2}{\rule{0.317pt}{0.400pt}}
\put(360,450.17){\rule{0.700pt}{0.400pt}}
\multiput(361.55,451.17)(-1.547,-2.000){2}{\rule{0.350pt}{0.400pt}}
\multiput(357.92,448.95)(-0.462,-0.447){3}{\rule{0.500pt}{0.108pt}}
\multiput(358.96,449.17)(-1.962,-3.000){2}{\rule{0.250pt}{0.400pt}}
\multiput(354.37,445.95)(-0.685,-0.447){3}{\rule{0.633pt}{0.108pt}}
\multiput(355.69,446.17)(-2.685,-3.000){2}{\rule{0.317pt}{0.400pt}}
\put(350,442.17){\rule{0.700pt}{0.400pt}}
\multiput(351.55,443.17)(-1.547,-2.000){2}{\rule{0.350pt}{0.400pt}}
\multiput(347.92,440.95)(-0.462,-0.447){3}{\rule{0.500pt}{0.108pt}}
\multiput(348.96,441.17)(-1.962,-3.000){2}{\rule{0.250pt}{0.400pt}}
\put(343,437.17){\rule{0.900pt}{0.400pt}}
\multiput(345.13,438.17)(-2.132,-2.000){2}{\rule{0.450pt}{0.400pt}}
\multiput(340.92,435.95)(-0.462,-0.447){3}{\rule{0.500pt}{0.108pt}}
\multiput(341.96,436.17)(-1.962,-3.000){2}{\rule{0.250pt}{0.400pt}}
\put(337,432.17){\rule{0.700pt}{0.400pt}}
\multiput(338.55,433.17)(-1.547,-2.000){2}{\rule{0.350pt}{0.400pt}}
\multiput(334.37,430.95)(-0.685,-0.447){3}{\rule{0.633pt}{0.108pt}}
\multiput(335.69,431.17)(-2.685,-3.000){2}{\rule{0.317pt}{0.400pt}}
\put(330,427.17){\rule{0.700pt}{0.400pt}}
\multiput(331.55,428.17)(-1.547,-2.000){2}{\rule{0.350pt}{0.400pt}}
\multiput(327.37,425.95)(-0.685,-0.447){3}{\rule{0.633pt}{0.108pt}}
\multiput(328.69,426.17)(-2.685,-3.000){2}{\rule{0.317pt}{0.400pt}}
\put(323,422.17){\rule{0.700pt}{0.400pt}}
\multiput(324.55,423.17)(-1.547,-2.000){2}{\rule{0.350pt}{0.400pt}}
\multiput(320.92,420.95)(-0.462,-0.447){3}{\rule{0.500pt}{0.108pt}}
\multiput(321.96,421.17)(-1.962,-3.000){2}{\rule{0.250pt}{0.400pt}}
\put(316,417.17){\rule{0.900pt}{0.400pt}}
\multiput(318.13,418.17)(-2.132,-2.000){2}{\rule{0.450pt}{0.400pt}}
\multiput(313.92,415.95)(-0.462,-0.447){3}{\rule{0.500pt}{0.108pt}}
\multiput(314.96,416.17)(-1.962,-3.000){2}{\rule{0.250pt}{0.400pt}}
\put(310,412.17){\rule{0.700pt}{0.400pt}}
\multiput(311.55,413.17)(-1.547,-2.000){2}{\rule{0.350pt}{0.400pt}}
\multiput(307.37,410.95)(-0.685,-0.447){3}{\rule{0.633pt}{0.108pt}}
\multiput(308.69,411.17)(-2.685,-3.000){2}{\rule{0.317pt}{0.400pt}}
\put(303,407.17){\rule{0.700pt}{0.400pt}}
\multiput(304.55,408.17)(-1.547,-2.000){2}{\rule{0.350pt}{0.400pt}}
\multiput(300.92,405.95)(-0.462,-0.447){3}{\rule{0.500pt}{0.108pt}}
\multiput(301.96,406.17)(-1.962,-3.000){2}{\rule{0.250pt}{0.400pt}}
\put(296,402.17){\rule{0.900pt}{0.400pt}}
\multiput(298.13,403.17)(-2.132,-2.000){2}{\rule{0.450pt}{0.400pt}}
\multiput(293.92,400.95)(-0.462,-0.447){3}{\rule{0.500pt}{0.108pt}}
\multiput(294.96,401.17)(-1.962,-3.000){2}{\rule{0.250pt}{0.400pt}}
\put(289,397.17){\rule{0.900pt}{0.400pt}}
\multiput(291.13,398.17)(-2.132,-2.000){2}{\rule{0.450pt}{0.400pt}}
\multiput(286.92,395.95)(-0.462,-0.447){3}{\rule{0.500pt}{0.108pt}}
\multiput(287.96,396.17)(-1.962,-3.000){2}{\rule{0.250pt}{0.400pt}}
\put(283,392.17){\rule{0.700pt}{0.400pt}}
\multiput(284.55,393.17)(-1.547,-2.000){2}{\rule{0.350pt}{0.400pt}}
\multiput(280.37,390.95)(-0.685,-0.447){3}{\rule{0.633pt}{0.108pt}}
\multiput(281.69,391.17)(-2.685,-3.000){2}{\rule{0.317pt}{0.400pt}}
\put(276,387.17){\rule{0.700pt}{0.400pt}}
\multiput(277.55,388.17)(-1.547,-2.000){2}{\rule{0.350pt}{0.400pt}}
\multiput(273.92,385.95)(-0.462,-0.447){3}{\rule{0.500pt}{0.108pt}}
\multiput(274.96,386.17)(-1.962,-3.000){2}{\rule{0.250pt}{0.400pt}}
\put(269,382.17){\rule{0.900pt}{0.400pt}}
\multiput(271.13,383.17)(-2.132,-2.000){2}{\rule{0.450pt}{0.400pt}}
\multiput(266.92,380.95)(-0.462,-0.447){3}{\rule{0.500pt}{0.108pt}}
\multiput(267.96,381.17)(-1.962,-3.000){2}{\rule{0.250pt}{0.400pt}}
\put(263,377.17){\rule{0.700pt}{0.400pt}}
\multiput(264.55,378.17)(-1.547,-2.000){2}{\rule{0.350pt}{0.400pt}}
\multiput(260.37,375.95)(-0.685,-0.447){3}{\rule{0.633pt}{0.108pt}}
\multiput(261.69,376.17)(-2.685,-3.000){2}{\rule{0.317pt}{0.400pt}}
\put(256,372.17){\rule{0.700pt}{0.400pt}}
\multiput(257.55,373.17)(-1.547,-2.000){2}{\rule{0.350pt}{0.400pt}}
\multiput(253.37,370.95)(-0.685,-0.447){3}{\rule{0.633pt}{0.108pt}}
\multiput(254.69,371.17)(-2.685,-3.000){2}{\rule{0.317pt}{0.400pt}}
\put(249,367.17){\rule{0.700pt}{0.400pt}}
\multiput(250.55,368.17)(-1.547,-2.000){2}{\rule{0.350pt}{0.400pt}}
\multiput(246.92,365.95)(-0.462,-0.447){3}{\rule{0.500pt}{0.108pt}}
\multiput(247.96,366.17)(-1.962,-3.000){2}{\rule{0.250pt}{0.400pt}}
\put(242,362.17){\rule{0.900pt}{0.400pt}}
\multiput(244.13,363.17)(-2.132,-2.000){2}{\rule{0.450pt}{0.400pt}}
\multiput(239.92,360.95)(-0.462,-0.447){3}{\rule{0.500pt}{0.108pt}}
\multiput(240.96,361.17)(-1.962,-3.000){2}{\rule{0.250pt}{0.400pt}}
\put(236,357.17){\rule{0.700pt}{0.400pt}}
\multiput(237.55,358.17)(-1.547,-2.000){2}{\rule{0.350pt}{0.400pt}}
\multiput(233.37,355.95)(-0.685,-0.447){3}{\rule{0.633pt}{0.108pt}}
\multiput(234.69,356.17)(-2.685,-3.000){2}{\rule{0.317pt}{0.400pt}}
\put(229,352.17){\rule{0.700pt}{0.400pt}}
\multiput(230.55,353.17)(-1.547,-2.000){2}{\rule{0.350pt}{0.400pt}}
\multiput(226.92,350.95)(-0.462,-0.447){3}{\rule{0.500pt}{0.108pt}}
\multiput(227.96,351.17)(-1.962,-3.000){2}{\rule{0.250pt}{0.400pt}}
\put(222,347.17){\rule{0.900pt}{0.400pt}}
\multiput(224.13,348.17)(-2.132,-2.000){2}{\rule{0.450pt}{0.400pt}}
\multiput(219.92,345.95)(-0.462,-0.447){3}{\rule{0.500pt}{0.108pt}}
\multiput(220.96,346.17)(-1.962,-3.000){2}{\rule{0.250pt}{0.400pt}}
\multiput(216.37,342.95)(-0.685,-0.447){3}{\rule{0.633pt}{0.108pt}}
\multiput(217.69,343.17)(-2.685,-3.000){2}{\rule{0.317pt}{0.400pt}}
\put(212,339.17){\rule{0.700pt}{0.400pt}}
\multiput(213.55,340.17)(-1.547,-2.000){2}{\rule{0.350pt}{0.400pt}}
\multiput(209.92,337.95)(-0.462,-0.447){3}{\rule{0.500pt}{0.108pt}}
\multiput(210.96,338.17)(-1.962,-3.000){2}{\rule{0.250pt}{0.400pt}}
\put(205,334.17){\rule{0.900pt}{0.400pt}}
\multiput(207.13,335.17)(-2.132,-2.000){2}{\rule{0.450pt}{0.400pt}}
\multiput(202.92,332.95)(-0.462,-0.447){3}{\rule{0.500pt}{0.108pt}}
\multiput(203.96,333.17)(-1.962,-3.000){2}{\rule{0.250pt}{0.400pt}}
\put(199,329.17){\rule{0.700pt}{0.400pt}}
\multiput(200.55,330.17)(-1.547,-2.000){2}{\rule{0.350pt}{0.400pt}}
\multiput(196.37,327.95)(-0.685,-0.447){3}{\rule{0.633pt}{0.108pt}}
\multiput(197.69,328.17)(-2.685,-3.000){2}{\rule{0.317pt}{0.400pt}}
\put(192,324.17){\rule{0.700pt}{0.400pt}}
\multiput(193.55,325.17)(-1.547,-2.000){2}{\rule{0.350pt}{0.400pt}}
\multiput(189.92,322.95)(-0.462,-0.447){3}{\rule{0.500pt}{0.108pt}}
\multiput(190.96,323.17)(-1.962,-3.000){2}{\rule{0.250pt}{0.400pt}}
\put(185,319.17){\rule{0.900pt}{0.400pt}}
\multiput(187.13,320.17)(-2.132,-2.000){2}{\rule{0.450pt}{0.400pt}}
\multiput(182.92,317.95)(-0.462,-0.447){3}{\rule{0.500pt}{0.108pt}}
\multiput(183.96,318.17)(-1.962,-3.000){2}{\rule{0.250pt}{0.400pt}}
\put(178,314.17){\rule{0.900pt}{0.400pt}}
\multiput(180.13,315.17)(-2.132,-2.000){2}{\rule{0.450pt}{0.400pt}}
\multiput(175.92,312.95)(-0.462,-0.447){3}{\rule{0.500pt}{0.108pt}}
\multiput(176.96,313.17)(-1.962,-3.000){2}{\rule{0.250pt}{0.400pt}}
\put(172,309.17){\rule{0.700pt}{0.400pt}}
\multiput(173.55,310.17)(-1.547,-2.000){2}{\rule{0.350pt}{0.400pt}}
\multiput(169.37,307.95)(-0.685,-0.447){3}{\rule{0.633pt}{0.108pt}}
\multiput(170.69,308.17)(-2.685,-3.000){2}{\rule{0.317pt}{0.400pt}}
\put(165,304.17){\rule{0.700pt}{0.400pt}}
\multiput(166.55,305.17)(-1.547,-2.000){2}{\rule{0.350pt}{0.400pt}}
\multiput(162.92,302.95)(-0.462,-0.447){3}{\rule{0.500pt}{0.108pt}}
\multiput(163.96,303.17)(-1.962,-3.000){2}{\rule{0.250pt}{0.400pt}}
\put(158,299.17){\rule{0.900pt}{0.400pt}}
\multiput(160.13,300.17)(-2.132,-2.000){2}{\rule{0.450pt}{0.400pt}}
\multiput(155.92,297.95)(-0.462,-0.447){3}{\rule{0.500pt}{0.108pt}}
\multiput(156.96,298.17)(-1.962,-3.000){2}{\rule{0.250pt}{0.400pt}}
\put(155,294.17){\rule{0.700pt}{0.400pt}}
\multiput(155.00,295.17)(1.547,-2.000){2}{\rule{0.350pt}{0.400pt}}
\multiput(158.00,292.95)(0.685,-0.447){3}{\rule{0.633pt}{0.108pt}}
\multiput(158.00,293.17)(2.685,-3.000){2}{\rule{0.317pt}{0.400pt}}
\put(162,289.17){\rule{0.700pt}{0.400pt}}
\multiput(162.00,290.17)(1.547,-2.000){2}{\rule{0.350pt}{0.400pt}}
\multiput(165.00,287.95)(0.462,-0.447){3}{\rule{0.500pt}{0.108pt}}
\multiput(165.00,288.17)(1.962,-3.000){2}{\rule{0.250pt}{0.400pt}}
\put(168,284.17){\rule{0.900pt}{0.400pt}}
\multiput(168.00,285.17)(2.132,-2.000){2}{\rule{0.450pt}{0.400pt}}
\multiput(172.00,282.95)(0.462,-0.447){3}{\rule{0.500pt}{0.108pt}}
\multiput(172.00,283.17)(1.962,-3.000){2}{\rule{0.250pt}{0.400pt}}
\put(175,279.17){\rule{0.700pt}{0.400pt}}
\multiput(175.00,280.17)(1.547,-2.000){2}{\rule{0.350pt}{0.400pt}}
\multiput(178.00,277.95)(0.685,-0.447){3}{\rule{0.633pt}{0.108pt}}
\multiput(178.00,278.17)(2.685,-3.000){2}{\rule{0.317pt}{0.400pt}}
\put(182,274.17){\rule{0.700pt}{0.400pt}}
\multiput(182.00,275.17)(1.547,-2.000){2}{\rule{0.350pt}{0.400pt}}
\multiput(185.00,272.95)(0.685,-0.447){3}{\rule{0.633pt}{0.108pt}}
\multiput(185.00,273.17)(2.685,-3.000){2}{\rule{0.317pt}{0.400pt}}
\put(189,269.17){\rule{0.700pt}{0.400pt}}
\multiput(189.00,270.17)(1.547,-2.000){2}{\rule{0.350pt}{0.400pt}}
\multiput(192.00,267.95)(0.462,-0.447){3}{\rule{0.500pt}{0.108pt}}
\multiput(192.00,268.17)(1.962,-3.000){2}{\rule{0.250pt}{0.400pt}}
\put(195,264.17){\rule{0.900pt}{0.400pt}}
\multiput(195.00,265.17)(2.132,-2.000){2}{\rule{0.450pt}{0.400pt}}
\multiput(199.00,262.95)(0.462,-0.447){3}{\rule{0.500pt}{0.108pt}}
\multiput(199.00,263.17)(1.962,-3.000){2}{\rule{0.250pt}{0.400pt}}
\put(202,259.17){\rule{0.700pt}{0.400pt}}
\multiput(202.00,260.17)(1.547,-2.000){2}{\rule{0.350pt}{0.400pt}}
\multiput(205.00,257.95)(0.685,-0.447){3}{\rule{0.633pt}{0.108pt}}
\multiput(205.00,258.17)(2.685,-3.000){2}{\rule{0.317pt}{0.400pt}}
\put(209,254.17){\rule{0.700pt}{0.400pt}}
\multiput(209.00,255.17)(1.547,-2.000){2}{\rule{0.350pt}{0.400pt}}
\multiput(212.00,252.95)(0.462,-0.447){3}{\rule{0.500pt}{0.108pt}}
\multiput(212.00,253.17)(1.962,-3.000){2}{\rule{0.250pt}{0.400pt}}
\put(215,249.17){\rule{0.900pt}{0.400pt}}
\multiput(215.00,250.17)(2.132,-2.000){2}{\rule{0.450pt}{0.400pt}}
\multiput(219.00,247.95)(0.462,-0.447){3}{\rule{0.500pt}{0.108pt}}
\multiput(219.00,248.17)(1.962,-3.000){2}{\rule{0.250pt}{0.400pt}}
\put(222,244.17){\rule{0.900pt}{0.400pt}}
\multiput(222.00,245.17)(2.132,-2.000){2}{\rule{0.450pt}{0.400pt}}
\multiput(226.00,242.95)(0.462,-0.447){3}{\rule{0.500pt}{0.108pt}}
\multiput(226.00,243.17)(1.962,-3.000){2}{\rule{0.250pt}{0.400pt}}
\multiput(229.00,239.95)(0.462,-0.447){3}{\rule{0.500pt}{0.108pt}}
\multiput(229.00,240.17)(1.962,-3.000){2}{\rule{0.250pt}{0.400pt}}
\put(232,236.17){\rule{0.900pt}{0.400pt}}
\multiput(232.00,237.17)(2.132,-2.000){2}{\rule{0.450pt}{0.400pt}}
\multiput(236.00,234.95)(0.462,-0.447){3}{\rule{0.500pt}{0.108pt}}
\multiput(236.00,235.17)(1.962,-3.000){2}{\rule{0.250pt}{0.400pt}}
\put(239,231.17){\rule{0.700pt}{0.400pt}}
\multiput(239.00,232.17)(1.547,-2.000){2}{\rule{0.350pt}{0.400pt}}
\multiput(242.00,229.95)(0.685,-0.447){3}{\rule{0.633pt}{0.108pt}}
\multiput(242.00,230.17)(2.685,-3.000){2}{\rule{0.317pt}{0.400pt}}
\put(246,226.17){\rule{0.700pt}{0.400pt}}
\multiput(246.00,227.17)(1.547,-2.000){2}{\rule{0.350pt}{0.400pt}}
\multiput(249.00,224.95)(0.462,-0.447){3}{\rule{0.500pt}{0.108pt}}
\multiput(249.00,225.17)(1.962,-3.000){2}{\rule{0.250pt}{0.400pt}}
\put(252,221.17){\rule{0.900pt}{0.400pt}}
\multiput(252.00,222.17)(2.132,-2.000){2}{\rule{0.450pt}{0.400pt}}
\multiput(256.00,219.95)(0.462,-0.447){3}{\rule{0.500pt}{0.108pt}}
\multiput(256.00,220.17)(1.962,-3.000){2}{\rule{0.250pt}{0.400pt}}
\put(259,216.17){\rule{0.900pt}{0.400pt}}
\multiput(259.00,217.17)(2.132,-2.000){2}{\rule{0.450pt}{0.400pt}}
\multiput(263.00,214.95)(0.462,-0.447){3}{\rule{0.500pt}{0.108pt}}
\multiput(263.00,215.17)(1.962,-3.000){2}{\rule{0.250pt}{0.400pt}}
\put(266,211.17){\rule{0.700pt}{0.400pt}}
\multiput(266.00,212.17)(1.547,-2.000){2}{\rule{0.350pt}{0.400pt}}
\multiput(269.00,209.95)(0.685,-0.447){3}{\rule{0.633pt}{0.108pt}}
\multiput(269.00,210.17)(2.685,-3.000){2}{\rule{0.317pt}{0.400pt}}
\put(273,206.17){\rule{0.700pt}{0.400pt}}
\multiput(273.00,207.17)(1.547,-2.000){2}{\rule{0.350pt}{0.400pt}}
\multiput(276.00,204.95)(0.462,-0.447){3}{\rule{0.500pt}{0.108pt}}
\multiput(276.00,205.17)(1.962,-3.000){2}{\rule{0.250pt}{0.400pt}}
\put(279,201.17){\rule{0.900pt}{0.400pt}}
\multiput(279.00,202.17)(2.132,-2.000){2}{\rule{0.450pt}{0.400pt}}
\multiput(283.00,199.95)(0.462,-0.447){3}{\rule{0.500pt}{0.108pt}}
\multiput(283.00,200.17)(1.962,-3.000){2}{\rule{0.250pt}{0.400pt}}
\put(286,196.17){\rule{0.700pt}{0.400pt}}
\multiput(286.00,197.17)(1.547,-2.000){2}{\rule{0.350pt}{0.400pt}}
\multiput(289.00,194.95)(0.685,-0.447){3}{\rule{0.633pt}{0.108pt}}
\multiput(289.00,195.17)(2.685,-3.000){2}{\rule{0.317pt}{0.400pt}}
\put(293,191.17){\rule{0.700pt}{0.400pt}}
\multiput(293.00,192.17)(1.547,-2.000){2}{\rule{0.350pt}{0.400pt}}
\multiput(296.00,189.95)(0.685,-0.447){3}{\rule{0.633pt}{0.108pt}}
\multiput(296.00,190.17)(2.685,-3.000){2}{\rule{0.317pt}{0.400pt}}
\put(300,186.17){\rule{0.700pt}{0.400pt}}
\multiput(300.00,187.17)(1.547,-2.000){2}{\rule{0.350pt}{0.400pt}}
\multiput(303.00,184.95)(0.462,-0.447){3}{\rule{0.500pt}{0.108pt}}
\multiput(303.00,185.17)(1.962,-3.000){2}{\rule{0.250pt}{0.400pt}}
\put(306,181.17){\rule{0.900pt}{0.400pt}}
\multiput(306.00,182.17)(2.132,-2.000){2}{\rule{0.450pt}{0.400pt}}
\multiput(310.00,179.95)(0.462,-0.447){3}{\rule{0.500pt}{0.108pt}}
\multiput(310.00,180.17)(1.962,-3.000){2}{\rule{0.250pt}{0.400pt}}
\put(313,176.17){\rule{0.700pt}{0.400pt}}
\multiput(313.00,177.17)(1.547,-2.000){2}{\rule{0.350pt}{0.400pt}}
\multiput(316.00,174.95)(0.685,-0.447){3}{\rule{0.633pt}{0.108pt}}
\multiput(316.00,175.17)(2.685,-3.000){2}{\rule{0.317pt}{0.400pt}}
\put(320,171.17){\rule{0.700pt}{0.400pt}}
\multiput(320.00,172.17)(1.547,-2.000){2}{\rule{0.350pt}{0.400pt}}
\multiput(323.00,169.95)(0.462,-0.447){3}{\rule{0.500pt}{0.108pt}}
\multiput(323.00,170.17)(1.962,-3.000){2}{\rule{0.250pt}{0.400pt}}
\put(326,166.17){\rule{0.900pt}{0.400pt}}
\multiput(326.00,167.17)(2.132,-2.000){2}{\rule{0.450pt}{0.400pt}}
\multiput(330.00,164.95)(0.462,-0.447){3}{\rule{0.500pt}{0.108pt}}
\multiput(330.00,165.17)(1.962,-3.000){2}{\rule{0.250pt}{0.400pt}}
\put(333,161.17){\rule{0.900pt}{0.400pt}}
\multiput(333.00,162.17)(2.132,-2.000){2}{\rule{0.450pt}{0.400pt}}
\multiput(337.00,159.95)(0.462,-0.447){3}{\rule{0.500pt}{0.108pt}}
\multiput(337.00,160.17)(1.962,-3.000){2}{\rule{0.250pt}{0.400pt}}
\put(340,156.17){\rule{0.700pt}{0.400pt}}
\multiput(340.00,157.17)(1.547,-2.000){2}{\rule{0.350pt}{0.400pt}}
\multiput(343.00,154.95)(0.685,-0.447){3}{\rule{0.633pt}{0.108pt}}
\multiput(343.00,155.17)(2.685,-3.000){2}{\rule{0.317pt}{0.400pt}}
\put(347,151.17){\rule{0.700pt}{0.400pt}}
\multiput(347.00,152.17)(1.547,-2.000){2}{\rule{0.350pt}{0.400pt}}
\multiput(350.00,149.95)(0.462,-0.447){3}{\rule{0.500pt}{0.108pt}}
\multiput(350.00,150.17)(1.962,-3.000){2}{\rule{0.250pt}{0.400pt}}
\put(353,146.17){\rule{0.900pt}{0.400pt}}
\multiput(353.00,147.17)(2.132,-2.000){2}{\rule{0.450pt}{0.400pt}}
\multiput(357.00,144.95)(0.462,-0.447){3}{\rule{0.500pt}{0.108pt}}
\multiput(357.00,145.17)(1.962,-3.000){2}{\rule{0.250pt}{0.400pt}}
\put(360,141.17){\rule{0.700pt}{0.400pt}}
\multiput(360.00,142.17)(1.547,-2.000){2}{\rule{0.350pt}{0.400pt}}
\multiput(363.00,139.95)(0.685,-0.447){3}{\rule{0.633pt}{0.108pt}}
\multiput(363.00,140.17)(2.685,-3.000){2}{\rule{0.317pt}{0.400pt}}
\multiput(367.00,136.95)(0.462,-0.447){3}{\rule{0.500pt}{0.108pt}}
\multiput(367.00,137.17)(1.962,-3.000){2}{\rule{0.250pt}{0.400pt}}
\put(370,133.17){\rule{0.900pt}{0.400pt}}
\multiput(370.00,134.17)(2.132,-2.000){2}{\rule{0.450pt}{0.400pt}}
\multiput(374.00,131.95)(0.462,-0.447){3}{\rule{0.500pt}{0.108pt}}
\multiput(374.00,132.17)(1.962,-3.000){2}{\rule{0.250pt}{0.400pt}}
\put(377,128.17){\rule{0.700pt}{0.400pt}}
\multiput(377.00,129.17)(1.547,-2.000){2}{\rule{0.350pt}{0.400pt}}
\multiput(380.00,126.95)(0.685,-0.447){3}{\rule{0.633pt}{0.108pt}}
\multiput(380.00,127.17)(2.685,-3.000){2}{\rule{0.317pt}{0.400pt}}
\put(384,123.17){\rule{0.700pt}{0.400pt}}
\multiput(384.00,124.17)(1.547,-2.000){2}{\rule{0.350pt}{0.400pt}}
\multiput(387.00,121.95)(0.462,-0.447){3}{\rule{0.500pt}{0.108pt}}
\multiput(387.00,122.17)(1.962,-3.000){2}{\rule{0.250pt}{0.400pt}}
\put(390,118.17){\rule{0.900pt}{0.400pt}}
\multiput(390.00,119.17)(2.132,-2.000){2}{\rule{0.450pt}{0.400pt}}
\multiput(394.00,116.95)(0.462,-0.447){3}{\rule{0.500pt}{0.108pt}}
\multiput(394.00,117.17)(1.962,-3.000){2}{\rule{0.250pt}{0.400pt}}
\put(397,113.17){\rule{0.700pt}{0.400pt}}
\multiput(397.00,114.17)(1.547,-2.000){2}{\rule{0.350pt}{0.400pt}}
\multiput(400.00,111.95)(0.685,-0.447){3}{\rule{0.633pt}{0.108pt}}
\multiput(400.00,112.17)(2.685,-3.000){2}{\rule{0.317pt}{0.400pt}}
\put(404,108.17){\rule{0.700pt}{0.400pt}}
\multiput(404.00,109.17)(1.547,-2.000){2}{\rule{0.350pt}{0.400pt}}
\multiput(407.00,106.95)(0.685,-0.447){3}{\rule{0.633pt}{0.108pt}}
\multiput(407.00,107.17)(2.685,-3.000){2}{\rule{0.317pt}{0.400pt}}
\put(411,103.17){\rule{0.700pt}{0.400pt}}
\multiput(411.00,104.17)(1.547,-2.000){2}{\rule{0.350pt}{0.400pt}}
\put(414,103.17){\rule{0.700pt}{0.400pt}}
\multiput(414.00,102.17)(1.547,2.000){2}{\rule{0.350pt}{0.400pt}}
\multiput(417.00,105.61)(0.685,0.447){3}{\rule{0.633pt}{0.108pt}}
\multiput(417.00,104.17)(2.685,3.000){2}{\rule{0.317pt}{0.400pt}}
\put(421,108.17){\rule{0.700pt}{0.400pt}}
\multiput(421.00,107.17)(1.547,2.000){2}{\rule{0.350pt}{0.400pt}}
\multiput(424.00,110.61)(0.462,0.447){3}{\rule{0.500pt}{0.108pt}}
\multiput(424.00,109.17)(1.962,3.000){2}{\rule{0.250pt}{0.400pt}}
\put(427,113.17){\rule{0.900pt}{0.400pt}}
\multiput(427.00,112.17)(2.132,2.000){2}{\rule{0.450pt}{0.400pt}}
\multiput(431.00,115.61)(0.462,0.447){3}{\rule{0.500pt}{0.108pt}}
\multiput(431.00,114.17)(1.962,3.000){2}{\rule{0.250pt}{0.400pt}}
\put(434,118.17){\rule{0.700pt}{0.400pt}}
\multiput(434.00,117.17)(1.547,2.000){2}{\rule{0.350pt}{0.400pt}}
\multiput(437.00,120.61)(0.685,0.447){3}{\rule{0.633pt}{0.108pt}}
\multiput(437.00,119.17)(2.685,3.000){2}{\rule{0.317pt}{0.400pt}}
\put(441,123.17){\rule{0.700pt}{0.400pt}}
\multiput(441.00,122.17)(1.547,2.000){2}{\rule{0.350pt}{0.400pt}}
\multiput(444.00,125.61)(0.685,0.447){3}{\rule{0.633pt}{0.108pt}}
\multiput(444.00,124.17)(2.685,3.000){2}{\rule{0.317pt}{0.400pt}}
\put(448,128.17){\rule{0.700pt}{0.400pt}}
\multiput(448.00,127.17)(1.547,2.000){2}{\rule{0.350pt}{0.400pt}}
\multiput(451.00,130.61)(0.462,0.447){3}{\rule{0.500pt}{0.108pt}}
\multiput(451.00,129.17)(1.962,3.000){2}{\rule{0.250pt}{0.400pt}}
\put(454,133.17){\rule{0.900pt}{0.400pt}}
\multiput(454.00,132.17)(2.132,2.000){2}{\rule{0.450pt}{0.400pt}}
\multiput(458.00,135.61)(0.462,0.447){3}{\rule{0.500pt}{0.108pt}}
\multiput(458.00,134.17)(1.962,3.000){2}{\rule{0.250pt}{0.400pt}}
\multiput(461.00,138.61)(0.462,0.447){3}{\rule{0.500pt}{0.108pt}}
\multiput(461.00,137.17)(1.962,3.000){2}{\rule{0.250pt}{0.400pt}}
\put(464,141.17){\rule{0.900pt}{0.400pt}}
\multiput(464.00,140.17)(2.132,2.000){2}{\rule{0.450pt}{0.400pt}}
\multiput(468.00,143.61)(0.462,0.447){3}{\rule{0.500pt}{0.108pt}}
\multiput(468.00,142.17)(1.962,3.000){2}{\rule{0.250pt}{0.400pt}}
\put(471,146.17){\rule{0.700pt}{0.400pt}}
\multiput(471.00,145.17)(1.547,2.000){2}{\rule{0.350pt}{0.400pt}}
\multiput(474.00,148.61)(0.685,0.447){3}{\rule{0.633pt}{0.108pt}}
\multiput(474.00,147.17)(2.685,3.000){2}{\rule{0.317pt}{0.400pt}}
\put(478,151.17){\rule{0.700pt}{0.400pt}}
\multiput(478.00,150.17)(1.547,2.000){2}{\rule{0.350pt}{0.400pt}}
\multiput(481.00,153.61)(0.685,0.447){3}{\rule{0.633pt}{0.108pt}}
\multiput(481.00,152.17)(2.685,3.000){2}{\rule{0.317pt}{0.400pt}}
\put(485,156.17){\rule{0.700pt}{0.400pt}}
\multiput(485.00,155.17)(1.547,2.000){2}{\rule{0.350pt}{0.400pt}}
\multiput(488.00,158.61)(0.462,0.447){3}{\rule{0.500pt}{0.108pt}}
\multiput(488.00,157.17)(1.962,3.000){2}{\rule{0.250pt}{0.400pt}}
\put(491,161.17){\rule{0.900pt}{0.400pt}}
\multiput(491.00,160.17)(2.132,2.000){2}{\rule{0.450pt}{0.400pt}}
\multiput(495.00,163.61)(0.462,0.447){3}{\rule{0.500pt}{0.108pt}}
\multiput(495.00,162.17)(1.962,3.000){2}{\rule{0.250pt}{0.400pt}}
\put(498,166.17){\rule{0.700pt}{0.400pt}}
\multiput(498.00,165.17)(1.547,2.000){2}{\rule{0.350pt}{0.400pt}}
\multiput(501.00,168.61)(0.685,0.447){3}{\rule{0.633pt}{0.108pt}}
\multiput(501.00,167.17)(2.685,3.000){2}{\rule{0.317pt}{0.400pt}}
\put(505,171.17){\rule{0.700pt}{0.400pt}}
\multiput(505.00,170.17)(1.547,2.000){2}{\rule{0.350pt}{0.400pt}}
\multiput(508.00,173.61)(0.462,0.447){3}{\rule{0.500pt}{0.108pt}}
\multiput(508.00,172.17)(1.962,3.000){2}{\rule{0.250pt}{0.400pt}}
\put(511,176.17){\rule{0.900pt}{0.400pt}}
\multiput(511.00,175.17)(2.132,2.000){2}{\rule{0.450pt}{0.400pt}}
\multiput(515.00,178.61)(0.462,0.447){3}{\rule{0.500pt}{0.108pt}}
\multiput(515.00,177.17)(1.962,3.000){2}{\rule{0.250pt}{0.400pt}}
\put(518,181.17){\rule{0.900pt}{0.400pt}}
\multiput(518.00,180.17)(2.132,2.000){2}{\rule{0.450pt}{0.400pt}}
\multiput(522.00,183.61)(0.462,0.447){3}{\rule{0.500pt}{0.108pt}}
\multiput(522.00,182.17)(1.962,3.000){2}{\rule{0.250pt}{0.400pt}}
\put(525,186.17){\rule{0.700pt}{0.400pt}}
\multiput(525.00,185.17)(1.547,2.000){2}{\rule{0.350pt}{0.400pt}}
\multiput(528.00,188.61)(0.685,0.447){3}{\rule{0.633pt}{0.108pt}}
\multiput(528.00,187.17)(2.685,3.000){2}{\rule{0.317pt}{0.400pt}}
\put(532,191.17){\rule{0.700pt}{0.400pt}}
\multiput(532.00,190.17)(1.547,2.000){2}{\rule{0.350pt}{0.400pt}}
\put(535,191.67){\rule{0.723pt}{0.400pt}}
\multiput(535.00,192.17)(1.500,-1.000){2}{\rule{0.361pt}{0.400pt}}
\put(538,190.67){\rule{0.723pt}{0.400pt}}
\multiput(538.00,191.17)(1.500,-1.000){2}{\rule{0.361pt}{0.400pt}}
\put(541,189.67){\rule{0.723pt}{0.400pt}}
\multiput(541.00,190.17)(1.500,-1.000){2}{\rule{0.361pt}{0.400pt}}
\put(544,188.17){\rule{0.482pt}{0.400pt}}
\multiput(544.00,189.17)(1.000,-2.000){2}{\rule{0.241pt}{0.400pt}}
\put(546,186.67){\rule{0.723pt}{0.400pt}}
\multiput(546.00,187.17)(1.500,-1.000){2}{\rule{0.361pt}{0.400pt}}
\put(549,185.67){\rule{0.723pt}{0.400pt}}
\multiput(549.00,186.17)(1.500,-1.000){2}{\rule{0.361pt}{0.400pt}}
\put(552,184.67){\rule{0.723pt}{0.400pt}}
\multiput(552.00,185.17)(1.500,-1.000){2}{\rule{0.361pt}{0.400pt}}
\put(555,183.17){\rule{0.700pt}{0.400pt}}
\multiput(555.00,184.17)(1.547,-2.000){2}{\rule{0.350pt}{0.400pt}}
\put(558,181.67){\rule{0.723pt}{0.400pt}}
\multiput(558.00,182.17)(1.500,-1.000){2}{\rule{0.361pt}{0.400pt}}
\put(561,180.67){\rule{0.482pt}{0.400pt}}
\multiput(561.00,181.17)(1.000,-1.000){2}{\rule{0.241pt}{0.400pt}}
\put(563,179.67){\rule{0.723pt}{0.400pt}}
\multiput(563.00,180.17)(1.500,-1.000){2}{\rule{0.361pt}{0.400pt}}
\put(566,178.17){\rule{0.700pt}{0.400pt}}
\multiput(566.00,179.17)(1.547,-2.000){2}{\rule{0.350pt}{0.400pt}}
\put(569,176.67){\rule{0.723pt}{0.400pt}}
\multiput(569.00,177.17)(1.500,-1.000){2}{\rule{0.361pt}{0.400pt}}
\put(572,175.67){\rule{0.723pt}{0.400pt}}
\multiput(572.00,176.17)(1.500,-1.000){2}{\rule{0.361pt}{0.400pt}}
\put(575,174.67){\rule{0.723pt}{0.400pt}}
\multiput(575.00,175.17)(1.500,-1.000){2}{\rule{0.361pt}{0.400pt}}
\put(578,173.67){\rule{0.482pt}{0.400pt}}
\multiput(578.00,174.17)(1.000,-1.000){2}{\rule{0.241pt}{0.400pt}}
\put(580,172.17){\rule{0.700pt}{0.400pt}}
\multiput(580.00,173.17)(1.547,-2.000){2}{\rule{0.350pt}{0.400pt}}
\put(583,170.67){\rule{0.723pt}{0.400pt}}
\multiput(583.00,171.17)(1.500,-1.000){2}{\rule{0.361pt}{0.400pt}}
\put(586,169.67){\rule{0.723pt}{0.400pt}}
\multiput(586.00,170.17)(1.500,-1.000){2}{\rule{0.361pt}{0.400pt}}
\put(589,168.67){\rule{0.723pt}{0.400pt}}
\multiput(589.00,169.17)(1.500,-1.000){2}{\rule{0.361pt}{0.400pt}}
\put(592,167.17){\rule{0.700pt}{0.400pt}}
\multiput(592.00,168.17)(1.547,-2.000){2}{\rule{0.350pt}{0.400pt}}
\put(595,165.67){\rule{0.723pt}{0.400pt}}
\multiput(595.00,166.17)(1.500,-1.000){2}{\rule{0.361pt}{0.400pt}}
\put(598,164.67){\rule{0.482pt}{0.400pt}}
\multiput(598.00,165.17)(1.000,-1.000){2}{\rule{0.241pt}{0.400pt}}
\put(600,163.67){\rule{0.723pt}{0.400pt}}
\multiput(600.00,164.17)(1.500,-1.000){2}{\rule{0.361pt}{0.400pt}}
\put(603,162.17){\rule{0.700pt}{0.400pt}}
\multiput(603.00,163.17)(1.547,-2.000){2}{\rule{0.350pt}{0.400pt}}
\put(606,160.67){\rule{0.723pt}{0.400pt}}
\multiput(606.00,161.17)(1.500,-1.000){2}{\rule{0.361pt}{0.400pt}}
\put(609,159.67){\rule{0.723pt}{0.400pt}}
\multiput(609.00,160.17)(1.500,-1.000){2}{\rule{0.361pt}{0.400pt}}
\put(612,158.67){\rule{0.723pt}{0.400pt}}
\multiput(612.00,159.17)(1.500,-1.000){2}{\rule{0.361pt}{0.400pt}}
\put(615,157.67){\rule{0.482pt}{0.400pt}}
\multiput(615.00,158.17)(1.000,-1.000){2}{\rule{0.241pt}{0.400pt}}
\put(617,156.17){\rule{0.700pt}{0.400pt}}
\multiput(617.00,157.17)(1.547,-2.000){2}{\rule{0.350pt}{0.400pt}}
\put(620,154.67){\rule{0.723pt}{0.400pt}}
\multiput(620.00,155.17)(1.500,-1.000){2}{\rule{0.361pt}{0.400pt}}
\put(623,153.67){\rule{0.723pt}{0.400pt}}
\multiput(623.00,154.17)(1.500,-1.000){2}{\rule{0.361pt}{0.400pt}}
\put(626,152.67){\rule{0.723pt}{0.400pt}}
\multiput(626.00,153.17)(1.500,-1.000){2}{\rule{0.361pt}{0.400pt}}
\put(629,151.17){\rule{0.700pt}{0.400pt}}
\multiput(629.00,152.17)(1.547,-2.000){2}{\rule{0.350pt}{0.400pt}}
\put(632,149.67){\rule{0.482pt}{0.400pt}}
\multiput(632.00,150.17)(1.000,-1.000){2}{\rule{0.241pt}{0.400pt}}
\put(634,148.67){\rule{0.723pt}{0.400pt}}
\multiput(634.00,149.17)(1.500,-1.000){2}{\rule{0.361pt}{0.400pt}}
\put(637,147.67){\rule{0.723pt}{0.400pt}}
\multiput(637.00,148.17)(1.500,-1.000){2}{\rule{0.361pt}{0.400pt}}
\put(640,146.17){\rule{0.700pt}{0.400pt}}
\multiput(640.00,147.17)(1.547,-2.000){2}{\rule{0.350pt}{0.400pt}}
\put(643,144.67){\rule{0.723pt}{0.400pt}}
\multiput(643.00,145.17)(1.500,-1.000){2}{\rule{0.361pt}{0.400pt}}
\put(646,143.67){\rule{0.723pt}{0.400pt}}
\multiput(646.00,144.17)(1.500,-1.000){2}{\rule{0.361pt}{0.400pt}}
\put(649,142.67){\rule{0.723pt}{0.400pt}}
\multiput(649.00,143.17)(1.500,-1.000){2}{\rule{0.361pt}{0.400pt}}
\put(652,141.67){\rule{0.482pt}{0.400pt}}
\multiput(652.00,142.17)(1.000,-1.000){2}{\rule{0.241pt}{0.400pt}}
\put(654,140.17){\rule{0.700pt}{0.400pt}}
\multiput(654.00,141.17)(1.547,-2.000){2}{\rule{0.350pt}{0.400pt}}
\put(657,138.67){\rule{0.723pt}{0.400pt}}
\multiput(657.00,139.17)(1.500,-1.000){2}{\rule{0.361pt}{0.400pt}}
\put(660,137.67){\rule{0.723pt}{0.400pt}}
\multiput(660.00,138.17)(1.500,-1.000){2}{\rule{0.361pt}{0.400pt}}
\put(663,136.67){\rule{0.723pt}{0.400pt}}
\multiput(663.00,137.17)(1.500,-1.000){2}{\rule{0.361pt}{0.400pt}}
\put(666,135.17){\rule{0.700pt}{0.400pt}}
\multiput(666.00,136.17)(1.547,-2.000){2}{\rule{0.350pt}{0.400pt}}
\put(669,133.67){\rule{0.482pt}{0.400pt}}
\multiput(669.00,134.17)(1.000,-1.000){2}{\rule{0.241pt}{0.400pt}}
\put(671,132.67){\rule{0.723pt}{0.400pt}}
\multiput(671.00,133.17)(1.500,-1.000){2}{\rule{0.361pt}{0.400pt}}
\put(674,131.67){\rule{0.723pt}{0.400pt}}
\multiput(674.00,132.17)(1.500,-1.000){2}{\rule{0.361pt}{0.400pt}}
\put(677,130.17){\rule{0.700pt}{0.400pt}}
\multiput(677.00,131.17)(1.547,-2.000){2}{\rule{0.350pt}{0.400pt}}
\put(680,128.67){\rule{0.723pt}{0.400pt}}
\multiput(680.00,129.17)(1.500,-1.000){2}{\rule{0.361pt}{0.400pt}}
\put(683,127.67){\rule{0.723pt}{0.400pt}}
\multiput(683.00,128.17)(1.500,-1.000){2}{\rule{0.361pt}{0.400pt}}
\put(686,126.67){\rule{0.482pt}{0.400pt}}
\multiput(686.00,127.17)(1.000,-1.000){2}{\rule{0.241pt}{0.400pt}}
\put(688,125.17){\rule{0.700pt}{0.400pt}}
\multiput(688.00,126.17)(1.547,-2.000){2}{\rule{0.350pt}{0.400pt}}
\put(691,123.67){\rule{0.723pt}{0.400pt}}
\multiput(691.00,124.17)(1.500,-1.000){2}{\rule{0.361pt}{0.400pt}}
\put(694,122.67){\rule{0.723pt}{0.400pt}}
\multiput(694.00,123.17)(1.500,-1.000){2}{\rule{0.361pt}{0.400pt}}
\put(697,121.67){\rule{0.723pt}{0.400pt}}
\multiput(697.00,122.17)(1.500,-1.000){2}{\rule{0.361pt}{0.400pt}}
\put(700,120.67){\rule{0.723pt}{0.400pt}}
\multiput(700.00,121.17)(1.500,-1.000){2}{\rule{0.361pt}{0.400pt}}
\put(703,119.17){\rule{0.700pt}{0.400pt}}
\multiput(703.00,120.17)(1.547,-2.000){2}{\rule{0.350pt}{0.400pt}}
\put(706,117.67){\rule{0.482pt}{0.400pt}}
\multiput(706.00,118.17)(1.000,-1.000){2}{\rule{0.241pt}{0.400pt}}
\put(708,116.67){\rule{0.723pt}{0.400pt}}
\multiput(708.00,117.17)(1.500,-1.000){2}{\rule{0.361pt}{0.400pt}}
\put(711,115.67){\rule{0.723pt}{0.400pt}}
\multiput(711.00,116.17)(1.500,-1.000){2}{\rule{0.361pt}{0.400pt}}
\put(714,114.17){\rule{0.700pt}{0.400pt}}
\multiput(714.00,115.17)(1.547,-2.000){2}{\rule{0.350pt}{0.400pt}}
\put(717,112.67){\rule{0.723pt}{0.400pt}}
\multiput(717.00,113.17)(1.500,-1.000){2}{\rule{0.361pt}{0.400pt}}
\put(720,111.67){\rule{0.723pt}{0.400pt}}
\multiput(720.00,112.17)(1.500,-1.000){2}{\rule{0.361pt}{0.400pt}}
\put(723,110.67){\rule{0.482pt}{0.400pt}}
\multiput(723.00,111.17)(1.000,-1.000){2}{\rule{0.241pt}{0.400pt}}
\put(725,109.17){\rule{0.700pt}{0.400pt}}
\multiput(725.00,110.17)(1.547,-2.000){2}{\rule{0.350pt}{0.400pt}}
\put(728,107.67){\rule{0.723pt}{0.400pt}}
\multiput(728.00,108.17)(1.500,-1.000){2}{\rule{0.361pt}{0.400pt}}
\put(731,106.67){\rule{0.723pt}{0.400pt}}
\multiput(731.00,107.17)(1.500,-1.000){2}{\rule{0.361pt}{0.400pt}}
\put(734,105.67){\rule{0.723pt}{0.400pt}}
\multiput(734.00,106.17)(1.500,-1.000){2}{\rule{0.361pt}{0.400pt}}
\put(737,104.67){\rule{0.723pt}{0.400pt}}
\multiput(737.00,105.17)(1.500,-1.000){2}{\rule{0.361pt}{0.400pt}}
\put(740,104.67){\rule{0.482pt}{0.400pt}}
\multiput(740.00,104.17)(1.000,1.000){2}{\rule{0.241pt}{0.400pt}}
\put(742,105.67){\rule{0.723pt}{0.400pt}}
\multiput(742.00,105.17)(1.500,1.000){2}{\rule{0.361pt}{0.400pt}}
\put(745,106.67){\rule{0.723pt}{0.400pt}}
\multiput(745.00,106.17)(1.500,1.000){2}{\rule{0.361pt}{0.400pt}}
\put(748,107.67){\rule{0.723pt}{0.400pt}}
\multiput(748.00,107.17)(1.500,1.000){2}{\rule{0.361pt}{0.400pt}}
\put(751,109.17){\rule{0.700pt}{0.400pt}}
\multiput(751.00,108.17)(1.547,2.000){2}{\rule{0.350pt}{0.400pt}}
\put(754,110.67){\rule{0.723pt}{0.400pt}}
\multiput(754.00,110.17)(1.500,1.000){2}{\rule{0.361pt}{0.400pt}}
\put(757,111.67){\rule{0.723pt}{0.400pt}}
\multiput(757.00,111.17)(1.500,1.000){2}{\rule{0.361pt}{0.400pt}}
\put(760,112.67){\rule{0.482pt}{0.400pt}}
\multiput(760.00,112.17)(1.000,1.000){2}{\rule{0.241pt}{0.400pt}}
\put(762,114.17){\rule{0.700pt}{0.400pt}}
\multiput(762.00,113.17)(1.547,2.000){2}{\rule{0.350pt}{0.400pt}}
\put(765,115.67){\rule{0.723pt}{0.400pt}}
\multiput(765.00,115.17)(1.500,1.000){2}{\rule{0.361pt}{0.400pt}}
\put(768,116.67){\rule{0.723pt}{0.400pt}}
\multiput(768.00,116.17)(1.500,1.000){2}{\rule{0.361pt}{0.400pt}}
\put(771,117.67){\rule{0.723pt}{0.400pt}}
\multiput(771.00,117.17)(1.500,1.000){2}{\rule{0.361pt}{0.400pt}}
\put(774,119.17){\rule{0.700pt}{0.400pt}}
\multiput(774.00,118.17)(1.547,2.000){2}{\rule{0.350pt}{0.400pt}}
\put(777,120.67){\rule{0.482pt}{0.400pt}}
\multiput(777.00,120.17)(1.000,1.000){2}{\rule{0.241pt}{0.400pt}}
\put(779,121.67){\rule{0.723pt}{0.400pt}}
\multiput(779.00,121.17)(1.500,1.000){2}{\rule{0.361pt}{0.400pt}}
\put(782,122.67){\rule{0.723pt}{0.400pt}}
\multiput(782.00,122.17)(1.500,1.000){2}{\rule{0.361pt}{0.400pt}}
\put(785,123.67){\rule{0.723pt}{0.400pt}}
\multiput(785.00,123.17)(1.500,1.000){2}{\rule{0.361pt}{0.400pt}}
\put(788,125.17){\rule{0.700pt}{0.400pt}}
\multiput(788.00,124.17)(1.547,2.000){2}{\rule{0.350pt}{0.400pt}}
\put(791,126.67){\rule{0.723pt}{0.400pt}}
\multiput(791.00,126.17)(1.500,1.000){2}{\rule{0.361pt}{0.400pt}}
\put(794,127.67){\rule{0.482pt}{0.400pt}}
\multiput(794.00,127.17)(1.000,1.000){2}{\rule{0.241pt}{0.400pt}}
\put(796,128.67){\rule{0.723pt}{0.400pt}}
\multiput(796.00,128.17)(1.500,1.000){2}{\rule{0.361pt}{0.400pt}}
\put(799,130.17){\rule{0.700pt}{0.400pt}}
\multiput(799.00,129.17)(1.547,2.000){2}{\rule{0.350pt}{0.400pt}}
\put(802,131.67){\rule{0.723pt}{0.400pt}}
\multiput(802.00,131.17)(1.500,1.000){2}{\rule{0.361pt}{0.400pt}}
\put(805,132.67){\rule{0.723pt}{0.400pt}}
\multiput(805.00,132.17)(1.500,1.000){2}{\rule{0.361pt}{0.400pt}}
\put(808,133.67){\rule{0.723pt}{0.400pt}}
\multiput(808.00,133.17)(1.500,1.000){2}{\rule{0.361pt}{0.400pt}}
\put(811,135.17){\rule{0.700pt}{0.400pt}}
\multiput(811.00,134.17)(1.547,2.000){2}{\rule{0.350pt}{0.400pt}}
\put(814,136.67){\rule{0.482pt}{0.400pt}}
\multiput(814.00,136.17)(1.000,1.000){2}{\rule{0.241pt}{0.400pt}}
\put(816,137.67){\rule{0.723pt}{0.400pt}}
\multiput(816.00,137.17)(1.500,1.000){2}{\rule{0.361pt}{0.400pt}}
\put(819,138.67){\rule{0.723pt}{0.400pt}}
\multiput(819.00,138.17)(1.500,1.000){2}{\rule{0.361pt}{0.400pt}}
\put(822,140.17){\rule{0.700pt}{0.400pt}}
\multiput(822.00,139.17)(1.547,2.000){2}{\rule{0.350pt}{0.400pt}}
\put(825,141.67){\rule{0.723pt}{0.400pt}}
\multiput(825.00,141.17)(1.500,1.000){2}{\rule{0.361pt}{0.400pt}}
\put(828,142.67){\rule{0.723pt}{0.400pt}}
\multiput(828.00,142.17)(1.500,1.000){2}{\rule{0.361pt}{0.400pt}}
\put(831,143.67){\rule{0.482pt}{0.400pt}}
\multiput(831.00,143.17)(1.000,1.000){2}{\rule{0.241pt}{0.400pt}}
\put(833,144.67){\rule{0.723pt}{0.400pt}}
\multiput(833.00,144.17)(1.500,1.000){2}{\rule{0.361pt}{0.400pt}}
\put(836,146.17){\rule{0.700pt}{0.400pt}}
\multiput(836.00,145.17)(1.547,2.000){2}{\rule{0.350pt}{0.400pt}}
\put(839,147.67){\rule{0.723pt}{0.400pt}}
\multiput(839.00,147.17)(1.500,1.000){2}{\rule{0.361pt}{0.400pt}}
\put(842,148.67){\rule{0.723pt}{0.400pt}}
\multiput(842.00,148.17)(1.500,1.000){2}{\rule{0.361pt}{0.400pt}}
\put(845,149.67){\rule{0.723pt}{0.400pt}}
\multiput(845.00,149.17)(1.500,1.000){2}{\rule{0.361pt}{0.400pt}}
\put(848,151.17){\rule{0.482pt}{0.400pt}}
\multiput(848.00,150.17)(1.000,2.000){2}{\rule{0.241pt}{0.400pt}}
\put(850,152.67){\rule{0.723pt}{0.400pt}}
\multiput(850.00,152.17)(1.500,1.000){2}{\rule{0.361pt}{0.400pt}}
\put(853,153.67){\rule{0.723pt}{0.400pt}}
\multiput(853.00,153.17)(1.500,1.000){2}{\rule{0.361pt}{0.400pt}}
\put(856,154.67){\rule{0.723pt}{0.400pt}}
\multiput(856.00,154.17)(1.500,1.000){2}{\rule{0.361pt}{0.400pt}}
\put(859,156.17){\rule{0.700pt}{0.400pt}}
\multiput(859.00,155.17)(1.547,2.000){2}{\rule{0.350pt}{0.400pt}}
\put(862,157.67){\rule{0.723pt}{0.400pt}}
\multiput(862.00,157.17)(1.500,1.000){2}{\rule{0.361pt}{0.400pt}}
\put(865,158.67){\rule{0.723pt}{0.400pt}}
\multiput(865.00,158.17)(1.500,1.000){2}{\rule{0.361pt}{0.400pt}}
\put(868,159.67){\rule{0.482pt}{0.400pt}}
\multiput(868.00,159.17)(1.000,1.000){2}{\rule{0.241pt}{0.400pt}}
\put(870,160.67){\rule{0.723pt}{0.400pt}}
\multiput(870.00,160.17)(1.500,1.000){2}{\rule{0.361pt}{0.400pt}}
\put(873,162.17){\rule{0.700pt}{0.400pt}}
\multiput(873.00,161.17)(1.547,2.000){2}{\rule{0.350pt}{0.400pt}}
\put(876,163.67){\rule{0.723pt}{0.400pt}}
\multiput(876.00,163.17)(1.500,1.000){2}{\rule{0.361pt}{0.400pt}}
\put(879,164.67){\rule{0.723pt}{0.400pt}}
\multiput(879.00,164.17)(1.500,1.000){2}{\rule{0.361pt}{0.400pt}}
\put(882,165.67){\rule{0.723pt}{0.400pt}}
\multiput(882.00,165.17)(1.500,1.000){2}{\rule{0.361pt}{0.400pt}}
\put(885,167.17){\rule{0.482pt}{0.400pt}}
\multiput(885.00,166.17)(1.000,2.000){2}{\rule{0.241pt}{0.400pt}}
\put(887,168.67){\rule{0.723pt}{0.400pt}}
\multiput(887.00,168.17)(1.500,1.000){2}{\rule{0.361pt}{0.400pt}}
\put(890,169.67){\rule{0.723pt}{0.400pt}}
\multiput(890.00,169.17)(1.500,1.000){2}{\rule{0.361pt}{0.400pt}}
\put(893,170.67){\rule{0.723pt}{0.400pt}}
\multiput(893.00,170.17)(1.500,1.000){2}{\rule{0.361pt}{0.400pt}}
\put(896,172.17){\rule{0.700pt}{0.400pt}}
\multiput(896.00,171.17)(1.547,2.000){2}{\rule{0.350pt}{0.400pt}}
\put(899,173.67){\rule{0.723pt}{0.400pt}}
\multiput(899.00,173.17)(1.500,1.000){2}{\rule{0.361pt}{0.400pt}}
\put(902,174.67){\rule{0.482pt}{0.400pt}}
\multiput(902.00,174.17)(1.000,1.000){2}{\rule{0.241pt}{0.400pt}}
\put(904,175.67){\rule{0.723pt}{0.400pt}}
\multiput(904.00,175.17)(1.500,1.000){2}{\rule{0.361pt}{0.400pt}}
\put(907,176.67){\rule{0.723pt}{0.400pt}}
\multiput(907.00,176.17)(1.500,1.000){2}{\rule{0.361pt}{0.400pt}}
\put(910,178.17){\rule{0.700pt}{0.400pt}}
\multiput(910.00,177.17)(1.547,2.000){2}{\rule{0.350pt}{0.400pt}}
\put(913,179.67){\rule{0.723pt}{0.400pt}}
\multiput(913.00,179.17)(1.500,1.000){2}{\rule{0.361pt}{0.400pt}}
\put(916,180.67){\rule{0.723pt}{0.400pt}}
\multiput(916.00,180.17)(1.500,1.000){2}{\rule{0.361pt}{0.400pt}}
\put(919,181.67){\rule{0.723pt}{0.400pt}}
\multiput(919.00,181.17)(1.500,1.000){2}{\rule{0.361pt}{0.400pt}}
\put(922,183.17){\rule{0.482pt}{0.400pt}}
\multiput(922.00,182.17)(1.000,2.000){2}{\rule{0.241pt}{0.400pt}}
\put(924,184.67){\rule{0.723pt}{0.400pt}}
\multiput(924.00,184.17)(1.500,1.000){2}{\rule{0.361pt}{0.400pt}}
\put(927,185.67){\rule{0.723pt}{0.400pt}}
\multiput(927.00,185.17)(1.500,1.000){2}{\rule{0.361pt}{0.400pt}}
\put(930,186.67){\rule{0.723pt}{0.400pt}}
\multiput(930.00,186.17)(1.500,1.000){2}{\rule{0.361pt}{0.400pt}}
\put(933,188.17){\rule{0.700pt}{0.400pt}}
\multiput(933.00,187.17)(1.547,2.000){2}{\rule{0.350pt}{0.400pt}}
\put(936,189.67){\rule{0.723pt}{0.400pt}}
\multiput(936.00,189.17)(1.500,1.000){2}{\rule{0.361pt}{0.400pt}}
\put(939,190.67){\rule{0.482pt}{0.400pt}}
\multiput(939.00,190.17)(1.000,1.000){2}{\rule{0.241pt}{0.400pt}}
\put(941,191.67){\rule{0.723pt}{0.400pt}}
\multiput(941.00,191.17)(1.500,1.000){2}{\rule{0.361pt}{0.400pt}}
\put(944,192.67){\rule{0.723pt}{0.400pt}}
\multiput(944.00,192.17)(1.500,1.000){2}{\rule{0.361pt}{0.400pt}}
\put(947,194.17){\rule{0.700pt}{0.400pt}}
\multiput(947.00,193.17)(1.547,2.000){2}{\rule{0.350pt}{0.400pt}}
\put(950,195.67){\rule{0.723pt}{0.400pt}}
\multiput(950.00,195.17)(1.500,1.000){2}{\rule{0.361pt}{0.400pt}}
\put(953,196.67){\rule{0.723pt}{0.400pt}}
\multiput(953.00,196.17)(1.500,1.000){2}{\rule{0.361pt}{0.400pt}}
\put(956,197.67){\rule{0.482pt}{0.400pt}}
\multiput(956.00,197.17)(1.000,1.000){2}{\rule{0.241pt}{0.400pt}}
\put(958,199.17){\rule{0.700pt}{0.400pt}}
\multiput(958.00,198.17)(1.547,2.000){2}{\rule{0.350pt}{0.400pt}}
\put(961,200.67){\rule{0.723pt}{0.400pt}}
\multiput(961.00,200.17)(1.500,1.000){2}{\rule{0.361pt}{0.400pt}}
\put(964,201.67){\rule{0.723pt}{0.400pt}}
\multiput(964.00,201.17)(1.500,1.000){2}{\rule{0.361pt}{0.400pt}}
\put(967,202.67){\rule{0.723pt}{0.400pt}}
\multiput(967.00,202.17)(1.500,1.000){2}{\rule{0.361pt}{0.400pt}}
\put(970,204.17){\rule{0.700pt}{0.400pt}}
\multiput(970.00,203.17)(1.547,2.000){2}{\rule{0.350pt}{0.400pt}}
\put(973,205.67){\rule{0.723pt}{0.400pt}}
\multiput(973.00,205.17)(1.500,1.000){2}{\rule{0.361pt}{0.400pt}}
\put(976,206.67){\rule{0.482pt}{0.400pt}}
\multiput(976.00,206.17)(1.000,1.000){2}{\rule{0.241pt}{0.400pt}}
\put(978,207.67){\rule{0.723pt}{0.400pt}}
\multiput(978.00,207.17)(1.500,1.000){2}{\rule{0.361pt}{0.400pt}}
\put(981,209.17){\rule{0.700pt}{0.400pt}}
\multiput(981.00,208.17)(1.547,2.000){2}{\rule{0.350pt}{0.400pt}}
\put(984,210.67){\rule{0.723pt}{0.400pt}}
\multiput(984.00,210.17)(1.500,1.000){2}{\rule{0.361pt}{0.400pt}}
\put(987,211.67){\rule{0.723pt}{0.400pt}}
\multiput(987.00,211.17)(1.500,1.000){2}{\rule{0.361pt}{0.400pt}}
\put(990,212.67){\rule{0.723pt}{0.400pt}}
\multiput(990.00,212.17)(1.500,1.000){2}{\rule{0.361pt}{0.400pt}}
\put(993,213.67){\rule{0.482pt}{0.400pt}}
\multiput(993.00,213.17)(1.000,1.000){2}{\rule{0.241pt}{0.400pt}}
\put(995,215.17){\rule{0.700pt}{0.400pt}}
\multiput(995.00,214.17)(1.547,2.000){2}{\rule{0.350pt}{0.400pt}}
\put(998,216.67){\rule{0.723pt}{0.400pt}}
\multiput(998.00,216.17)(1.500,1.000){2}{\rule{0.361pt}{0.400pt}}
\put(1001,217.67){\rule{0.723pt}{0.400pt}}
\multiput(1001.00,217.17)(1.500,1.000){2}{\rule{0.361pt}{0.400pt}}
\put(1004,218.67){\rule{0.723pt}{0.400pt}}
\multiput(1004.00,218.17)(1.500,1.000){2}{\rule{0.361pt}{0.400pt}}
\put(1007,220.17){\rule{0.700pt}{0.400pt}}
\multiput(1007.00,219.17)(1.547,2.000){2}{\rule{0.350pt}{0.400pt}}
\put(1010,221.67){\rule{0.482pt}{0.400pt}}
\multiput(1010.00,221.17)(1.000,1.000){2}{\rule{0.241pt}{0.400pt}}
\put(1012,222.67){\rule{0.723pt}{0.400pt}}
\multiput(1012.00,222.17)(1.500,1.000){2}{\rule{0.361pt}{0.400pt}}
\put(1015,223.67){\rule{0.723pt}{0.400pt}}
\multiput(1015.00,223.17)(1.500,1.000){2}{\rule{0.361pt}{0.400pt}}
\put(1018,225.17){\rule{0.700pt}{0.400pt}}
\multiput(1018.00,224.17)(1.547,2.000){2}{\rule{0.350pt}{0.400pt}}
\put(1021,226.67){\rule{0.723pt}{0.400pt}}
\multiput(1021.00,226.17)(1.500,1.000){2}{\rule{0.361pt}{0.400pt}}
\put(1024,227.67){\rule{0.723pt}{0.400pt}}
\multiput(1024.00,227.17)(1.500,1.000){2}{\rule{0.361pt}{0.400pt}}
\put(1027,228.67){\rule{0.723pt}{0.400pt}}
\multiput(1027.00,228.17)(1.500,1.000){2}{\rule{0.361pt}{0.400pt}}
\put(1030,229.67){\rule{0.482pt}{0.400pt}}
\multiput(1030.00,229.17)(1.000,1.000){2}{\rule{0.241pt}{0.400pt}}
\put(1032,231.17){\rule{0.700pt}{0.400pt}}
\multiput(1032.00,230.17)(1.547,2.000){2}{\rule{0.350pt}{0.400pt}}
\put(1035,232.67){\rule{0.723pt}{0.400pt}}
\multiput(1035.00,232.17)(1.500,1.000){2}{\rule{0.361pt}{0.400pt}}
\put(1038,233.67){\rule{0.723pt}{0.400pt}}
\multiput(1038.00,233.17)(1.500,1.000){2}{\rule{0.361pt}{0.400pt}}
\put(1041,234.67){\rule{0.723pt}{0.400pt}}
\multiput(1041.00,234.17)(1.500,1.000){2}{\rule{0.361pt}{0.400pt}}
\put(1044,236.17){\rule{0.700pt}{0.400pt}}
\multiput(1044.00,235.17)(1.547,2.000){2}{\rule{0.350pt}{0.400pt}}
\put(1047,237.67){\rule{0.482pt}{0.400pt}}
\multiput(1047.00,237.17)(1.000,1.000){2}{\rule{0.241pt}{0.400pt}}
\put(1049,238.67){\rule{0.723pt}{0.400pt}}
\multiput(1049.00,238.17)(1.500,1.000){2}{\rule{0.361pt}{0.400pt}}
\put(1052,239.67){\rule{0.723pt}{0.400pt}}
\multiput(1052.00,239.17)(1.500,1.000){2}{\rule{0.361pt}{0.400pt}}
\put(1055,241.17){\rule{0.700pt}{0.400pt}}
\multiput(1055.00,240.17)(1.547,2.000){2}{\rule{0.350pt}{0.400pt}}
\put(1058,242.67){\rule{0.723pt}{0.400pt}}
\multiput(1058.00,242.17)(1.500,1.000){2}{\rule{0.361pt}{0.400pt}}
\put(1061,243.67){\rule{0.723pt}{0.400pt}}
\multiput(1061.00,243.17)(1.500,1.000){2}{\rule{0.361pt}{0.400pt}}
\put(1064,244.67){\rule{0.482pt}{0.400pt}}
\multiput(1064.00,244.17)(1.000,1.000){2}{\rule{0.241pt}{0.400pt}}
\put(1066,245.67){\rule{0.723pt}{0.400pt}}
\multiput(1066.00,245.17)(1.500,1.000){2}{\rule{0.361pt}{0.400pt}}
\put(1069,247.17){\rule{0.700pt}{0.400pt}}
\multiput(1069.00,246.17)(1.547,2.000){2}{\rule{0.350pt}{0.400pt}}
\put(1072,248.67){\rule{0.723pt}{0.400pt}}
\multiput(1072.00,248.17)(1.500,1.000){2}{\rule{0.361pt}{0.400pt}}
\put(1075,249.67){\rule{0.723pt}{0.400pt}}
\multiput(1075.00,249.17)(1.500,1.000){2}{\rule{0.361pt}{0.400pt}}
\put(1078,250.67){\rule{0.723pt}{0.400pt}}
\multiput(1078.00,250.17)(1.500,1.000){2}{\rule{0.361pt}{0.400pt}}
\put(1081,252.17){\rule{0.700pt}{0.400pt}}
\multiput(1081.00,251.17)(1.547,2.000){2}{\rule{0.350pt}{0.400pt}}
\put(1084,253.67){\rule{0.482pt}{0.400pt}}
\multiput(1084.00,253.17)(1.000,1.000){2}{\rule{0.241pt}{0.400pt}}
\put(1086,254.67){\rule{0.723pt}{0.400pt}}
\multiput(1086.00,254.17)(1.500,1.000){2}{\rule{0.361pt}{0.400pt}}
\put(1089,255.67){\rule{0.723pt}{0.400pt}}
\multiput(1089.00,255.17)(1.500,1.000){2}{\rule{0.361pt}{0.400pt}}
\put(1092,257.17){\rule{0.700pt}{0.400pt}}
\multiput(1092.00,256.17)(1.547,2.000){2}{\rule{0.350pt}{0.400pt}}
\put(1095,258.67){\rule{0.723pt}{0.400pt}}
\multiput(1095.00,258.17)(1.500,1.000){2}{\rule{0.361pt}{0.400pt}}
\put(1098,259.67){\rule{0.723pt}{0.400pt}}
\multiput(1098.00,259.17)(1.500,1.000){2}{\rule{0.361pt}{0.400pt}}
\put(1101,260.67){\rule{0.482pt}{0.400pt}}
\multiput(1101.00,260.17)(1.000,1.000){2}{\rule{0.241pt}{0.400pt}}
\put(1103,261.67){\rule{0.723pt}{0.400pt}}
\multiput(1103.00,261.17)(1.500,1.000){2}{\rule{0.361pt}{0.400pt}}
\put(1106,263.17){\rule{0.700pt}{0.400pt}}
\multiput(1106.00,262.17)(1.547,2.000){2}{\rule{0.350pt}{0.400pt}}
\put(1109,264.67){\rule{0.723pt}{0.400pt}}
\multiput(1109.00,264.17)(1.500,1.000){2}{\rule{0.361pt}{0.400pt}}
\put(1112,265.67){\rule{0.723pt}{0.400pt}}
\multiput(1112.00,265.17)(1.500,1.000){2}{\rule{0.361pt}{0.400pt}}
\put(1115,266.67){\rule{0.723pt}{0.400pt}}
\multiput(1115.00,266.17)(1.500,1.000){2}{\rule{0.361pt}{0.400pt}}
\put(1118,268.17){\rule{0.482pt}{0.400pt}}
\multiput(1118.00,267.17)(1.000,2.000){2}{\rule{0.241pt}{0.400pt}}
\put(1120,269.67){\rule{0.723pt}{0.400pt}}
\multiput(1120.00,269.17)(1.500,1.000){2}{\rule{0.361pt}{0.400pt}}
\put(1123,270.67){\rule{0.723pt}{0.400pt}}
\multiput(1123.00,270.17)(1.500,1.000){2}{\rule{0.361pt}{0.400pt}}
\put(1126,271.67){\rule{0.723pt}{0.400pt}}
\multiput(1126.00,271.17)(1.500,1.000){2}{\rule{0.361pt}{0.400pt}}
\put(1129,273.17){\rule{0.700pt}{0.400pt}}
\multiput(1129.00,272.17)(1.547,2.000){2}{\rule{0.350pt}{0.400pt}}
\put(1132,274.67){\rule{0.723pt}{0.400pt}}
\multiput(1132.00,274.17)(1.500,1.000){2}{\rule{0.361pt}{0.400pt}}
\put(1135,275.67){\rule{0.723pt}{0.400pt}}
\multiput(1135.00,275.17)(1.500,1.000){2}{\rule{0.361pt}{0.400pt}}
\put(1138,276.67){\rule{0.482pt}{0.400pt}}
\multiput(1138.00,276.17)(1.000,1.000){2}{\rule{0.241pt}{0.400pt}}
\put(1140,278.17){\rule{0.700pt}{0.400pt}}
\multiput(1140.00,277.17)(1.547,2.000){2}{\rule{0.350pt}{0.400pt}}
\put(1143,279.67){\rule{0.723pt}{0.400pt}}
\multiput(1143.00,279.17)(1.500,1.000){2}{\rule{0.361pt}{0.400pt}}
\put(1146,280.67){\rule{0.723pt}{0.400pt}}
\multiput(1146.00,280.17)(1.500,1.000){2}{\rule{0.361pt}{0.400pt}}
\put(1149,281.67){\rule{0.723pt}{0.400pt}}
\multiput(1149.00,281.17)(1.500,1.000){2}{\rule{0.361pt}{0.400pt}}
\put(1152,282.67){\rule{0.723pt}{0.400pt}}
\multiput(1152.00,282.17)(1.500,1.000){2}{\rule{0.361pt}{0.400pt}}
\put(1155,284.17){\rule{0.482pt}{0.400pt}}
\multiput(1155.00,283.17)(1.000,2.000){2}{\rule{0.241pt}{0.400pt}}
\put(1157,285.67){\rule{0.723pt}{0.400pt}}
\multiput(1157.00,285.17)(1.500,1.000){2}{\rule{0.361pt}{0.400pt}}
\put(1160,286.67){\rule{0.723pt}{0.400pt}}
\multiput(1160.00,286.17)(1.500,1.000){2}{\rule{0.361pt}{0.400pt}}
\put(1163,287.67){\rule{0.723pt}{0.400pt}}
\multiput(1163.00,287.17)(1.500,1.000){2}{\rule{0.361pt}{0.400pt}}
\put(1166,289.17){\rule{0.700pt}{0.400pt}}
\multiput(1166.00,288.17)(1.547,2.000){2}{\rule{0.350pt}{0.400pt}}
\put(1169,290.67){\rule{0.723pt}{0.400pt}}
\multiput(1169.00,290.17)(1.500,1.000){2}{\rule{0.361pt}{0.400pt}}
\put(1172,291.67){\rule{0.482pt}{0.400pt}}
\multiput(1172.00,291.17)(1.000,1.000){2}{\rule{0.241pt}{0.400pt}}
\put(1174,292.67){\rule{0.723pt}{0.400pt}}
\multiput(1174.00,292.17)(1.500,1.000){2}{\rule{0.361pt}{0.400pt}}
\put(1177,294.17){\rule{0.700pt}{0.400pt}}
\multiput(1177.00,293.17)(1.547,2.000){2}{\rule{0.350pt}{0.400pt}}
\put(1180,295.67){\rule{0.723pt}{0.400pt}}
\multiput(1180.00,295.17)(1.500,1.000){2}{\rule{0.361pt}{0.400pt}}
\put(1183,296.67){\rule{0.723pt}{0.400pt}}
\multiput(1183.00,296.17)(1.500,1.000){2}{\rule{0.361pt}{0.400pt}}
\put(1186,297.67){\rule{0.723pt}{0.400pt}}
\multiput(1186.00,297.17)(1.500,1.000){2}{\rule{0.361pt}{0.400pt}}
\put(1189,298.67){\rule{0.723pt}{0.400pt}}
\multiput(1189.00,298.17)(1.500,1.000){2}{\rule{0.361pt}{0.400pt}}
\put(1192,300.17){\rule{0.482pt}{0.400pt}}
\multiput(1192.00,299.17)(1.000,2.000){2}{\rule{0.241pt}{0.400pt}}
\put(1194,301.67){\rule{0.723pt}{0.400pt}}
\multiput(1194.00,301.17)(1.500,1.000){2}{\rule{0.361pt}{0.400pt}}
\put(1197,302.67){\rule{0.723pt}{0.400pt}}
\multiput(1197.00,302.17)(1.500,1.000){2}{\rule{0.361pt}{0.400pt}}
\put(1200,303.67){\rule{0.723pt}{0.400pt}}
\multiput(1200.00,303.17)(1.500,1.000){2}{\rule{0.361pt}{0.400pt}}
\put(1203,305.17){\rule{0.700pt}{0.400pt}}
\multiput(1203.00,304.17)(1.547,2.000){2}{\rule{0.350pt}{0.400pt}}
\put(1206,306.67){\rule{0.723pt}{0.400pt}}
\multiput(1206.00,306.17)(1.500,1.000){2}{\rule{0.361pt}{0.400pt}}
\put(1209,307.67){\rule{0.482pt}{0.400pt}}
\multiput(1209.00,307.17)(1.000,1.000){2}{\rule{0.241pt}{0.400pt}}
\put(1211,308.67){\rule{0.723pt}{0.400pt}}
\multiput(1211.00,308.17)(1.500,1.000){2}{\rule{0.361pt}{0.400pt}}
\put(1214,310.17){\rule{0.700pt}{0.400pt}}
\multiput(1214.00,309.17)(1.547,2.000){2}{\rule{0.350pt}{0.400pt}}
\put(1217,311.67){\rule{0.723pt}{0.400pt}}
\multiput(1217.00,311.17)(1.500,1.000){2}{\rule{0.361pt}{0.400pt}}
\put(1220,312.67){\rule{0.723pt}{0.400pt}}
\multiput(1220.00,312.17)(1.500,1.000){2}{\rule{0.361pt}{0.400pt}}
\put(1223,313.67){\rule{0.723pt}{0.400pt}}
\multiput(1223.00,313.17)(1.500,1.000){2}{\rule{0.361pt}{0.400pt}}
\put(1226,314.67){\rule{0.482pt}{0.400pt}}
\multiput(1226.00,314.17)(1.000,1.000){2}{\rule{0.241pt}{0.400pt}}
\put(1228,316.17){\rule{0.700pt}{0.400pt}}
\multiput(1228.00,315.17)(1.547,2.000){2}{\rule{0.350pt}{0.400pt}}
\put(1231,317.67){\rule{0.723pt}{0.400pt}}
\multiput(1231.00,317.17)(1.500,1.000){2}{\rule{0.361pt}{0.400pt}}
\put(1234,318.67){\rule{0.723pt}{0.400pt}}
\multiput(1234.00,318.17)(1.500,1.000){2}{\rule{0.361pt}{0.400pt}}
\put(1237,319.67){\rule{0.723pt}{0.400pt}}
\multiput(1237.00,319.17)(1.500,1.000){2}{\rule{0.361pt}{0.400pt}}
\put(1240,321.17){\rule{0.700pt}{0.400pt}}
\multiput(1240.00,320.17)(1.547,2.000){2}{\rule{0.350pt}{0.400pt}}
\put(1243,322.67){\rule{0.723pt}{0.400pt}}
\multiput(1243.00,322.17)(1.500,1.000){2}{\rule{0.361pt}{0.400pt}}
\put(1246,323.67){\rule{0.482pt}{0.400pt}}
\multiput(1246.00,323.17)(1.000,1.000){2}{\rule{0.241pt}{0.400pt}}
\put(1248,324.67){\rule{0.723pt}{0.400pt}}
\multiput(1248.00,324.17)(1.500,1.000){2}{\rule{0.361pt}{0.400pt}}
\put(1251,326.17){\rule{0.700pt}{0.400pt}}
\multiput(1251.00,325.17)(1.547,2.000){2}{\rule{0.350pt}{0.400pt}}
\put(1254,327.67){\rule{0.723pt}{0.400pt}}
\multiput(1254.00,327.17)(1.500,1.000){2}{\rule{0.361pt}{0.400pt}}
\put(1257,328.67){\rule{0.723pt}{0.400pt}}
\multiput(1257.00,328.17)(1.500,1.000){2}{\rule{0.361pt}{0.400pt}}
\put(1260,329.67){\rule{0.723pt}{0.400pt}}
\multiput(1260.00,329.17)(1.500,1.000){2}{\rule{0.361pt}{0.400pt}}
\put(1263,330.67){\rule{0.482pt}{0.400pt}}
\multiput(1263.00,330.17)(1.000,1.000){2}{\rule{0.241pt}{0.400pt}}
\put(1265,332.17){\rule{0.700pt}{0.400pt}}
\multiput(1265.00,331.17)(1.547,2.000){2}{\rule{0.350pt}{0.400pt}}
\put(1268,333.67){\rule{0.723pt}{0.400pt}}
\multiput(1268.00,333.17)(1.500,1.000){2}{\rule{0.361pt}{0.400pt}}
\put(1271,334.67){\rule{0.723pt}{0.400pt}}
\multiput(1271.00,334.17)(1.500,1.000){2}{\rule{0.361pt}{0.400pt}}
\put(1274,335.67){\rule{0.723pt}{0.400pt}}
\multiput(1274.00,335.17)(1.500,1.000){2}{\rule{0.361pt}{0.400pt}}
\put(1277,337.17){\rule{0.700pt}{0.400pt}}
\multiput(1277.00,336.17)(1.547,2.000){2}{\rule{0.350pt}{0.400pt}}
\put(1280,338.67){\rule{0.482pt}{0.400pt}}
\multiput(1280.00,338.17)(1.000,1.000){2}{\rule{0.241pt}{0.400pt}}
\put(1282,339.67){\rule{0.723pt}{0.400pt}}
\multiput(1282.00,339.17)(1.500,1.000){2}{\rule{0.361pt}{0.400pt}}
\put(1285,340.67){\rule{0.723pt}{0.400pt}}
\multiput(1285.00,340.17)(1.500,1.000){2}{\rule{0.361pt}{0.400pt}}
\put(1288,342.17){\rule{0.700pt}{0.400pt}}
\multiput(1288.00,341.17)(1.547,2.000){2}{\rule{0.350pt}{0.400pt}}
\put(1291,343.67){\rule{0.723pt}{0.400pt}}
\multiput(1291.00,343.17)(1.500,1.000){2}{\rule{0.361pt}{0.400pt}}
\put(1294,344.67){\rule{0.723pt}{0.400pt}}
\multiput(1294.00,344.17)(1.500,1.000){2}{\rule{0.361pt}{0.400pt}}
\put(1297,345.67){\rule{0.723pt}{0.400pt}}
\multiput(1297.00,345.17)(1.500,1.000){2}{\rule{0.361pt}{0.400pt}}
\put(1300,346.67){\rule{0.482pt}{0.400pt}}
\multiput(1300.00,346.17)(1.000,1.000){2}{\rule{0.241pt}{0.400pt}}
\put(1302,348.17){\rule{0.700pt}{0.400pt}}
\multiput(1302.00,347.17)(1.547,2.000){2}{\rule{0.350pt}{0.400pt}}
\put(1305,349.67){\rule{0.723pt}{0.400pt}}
\multiput(1305.00,349.17)(1.500,1.000){2}{\rule{0.361pt}{0.400pt}}
\put(1308,350.67){\rule{0.723pt}{0.400pt}}
\multiput(1308.00,350.17)(1.500,1.000){2}{\rule{0.361pt}{0.400pt}}
\put(1311,351.67){\rule{0.723pt}{0.400pt}}
\multiput(1311.00,351.17)(1.500,1.000){2}{\rule{0.361pt}{0.400pt}}
\put(1314,353.17){\rule{0.700pt}{0.400pt}}
\multiput(1314.00,352.17)(1.547,2.000){2}{\rule{0.350pt}{0.400pt}}
\put(1317,354.67){\rule{0.482pt}{0.400pt}}
\multiput(1317.00,354.17)(1.000,1.000){2}{\rule{0.241pt}{0.400pt}}
\put(1319,355.67){\rule{0.723pt}{0.400pt}}
\multiput(1319.00,355.17)(1.500,1.000){2}{\rule{0.361pt}{0.400pt}}
\put(1322,356.67){\rule{0.723pt}{0.400pt}}
\multiput(1322.00,356.17)(1.500,1.000){2}{\rule{0.361pt}{0.400pt}}
\put(1325,358.17){\rule{0.700pt}{0.400pt}}
\multiput(1325.00,357.17)(1.547,2.000){2}{\rule{0.350pt}{0.400pt}}
\put(1328,359.67){\rule{0.723pt}{0.400pt}}
\multiput(1328.00,359.17)(1.500,1.000){2}{\rule{0.361pt}{0.400pt}}
\put(1331,360.67){\rule{0.723pt}{0.400pt}}
\multiput(1331.00,360.17)(1.500,1.000){2}{\rule{0.361pt}{0.400pt}}
\put(1334,361.67){\rule{0.482pt}{0.400pt}}
\multiput(1334.00,361.17)(1.000,1.000){2}{\rule{0.241pt}{0.400pt}}
\put(1336,363.17){\rule{0.700pt}{0.400pt}}
\multiput(1336.00,362.17)(1.547,2.000){2}{\rule{0.350pt}{0.400pt}}
\put(1339,364.67){\rule{0.723pt}{0.400pt}}
\multiput(1339.00,364.17)(1.500,1.000){2}{\rule{0.361pt}{0.400pt}}
\put(1342,365.67){\rule{0.723pt}{0.400pt}}
\multiput(1342.00,365.17)(1.500,1.000){2}{\rule{0.361pt}{0.400pt}}
\put(1345,366.67){\rule{0.723pt}{0.400pt}}
\multiput(1345.00,366.17)(1.500,1.000){2}{\rule{0.361pt}{0.400pt}}
\put(1348,367.67){\rule{0.723pt}{0.400pt}}
\multiput(1348.00,367.17)(1.500,1.000){2}{\rule{0.361pt}{0.400pt}}
\put(1351,369.17){\rule{0.700pt}{0.400pt}}
\multiput(1351.00,368.17)(1.547,2.000){2}{\rule{0.350pt}{0.400pt}}
\put(1354,370.67){\rule{0.482pt}{0.400pt}}
\multiput(1354.00,370.17)(1.000,1.000){2}{\rule{0.241pt}{0.400pt}}
\put(1356,371.67){\rule{0.723pt}{0.400pt}}
\multiput(1356.00,371.17)(1.500,1.000){2}{\rule{0.361pt}{0.400pt}}
\put(1359,372.67){\rule{0.723pt}{0.400pt}}
\multiput(1359.00,372.17)(1.500,1.000){2}{\rule{0.361pt}{0.400pt}}
\put(1362,374.17){\rule{0.700pt}{0.400pt}}
\multiput(1362.00,373.17)(1.547,2.000){2}{\rule{0.350pt}{0.400pt}}
\put(1365,375.67){\rule{0.723pt}{0.400pt}}
\multiput(1365.00,375.17)(1.500,1.000){2}{\rule{0.361pt}{0.400pt}}
\put(1368,376.67){\rule{0.723pt}{0.400pt}}
\multiput(1368.00,376.17)(1.500,1.000){2}{\rule{0.361pt}{0.400pt}}
\put(1371,377.67){\rule{0.482pt}{0.400pt}}
\multiput(1371.00,377.17)(1.000,1.000){2}{\rule{0.241pt}{0.400pt}}
\put(1373,379.17){\rule{0.700pt}{0.400pt}}
\multiput(1373.00,378.17)(1.547,2.000){2}{\rule{0.350pt}{0.400pt}}
\put(1376,380.67){\rule{0.723pt}{0.400pt}}
\multiput(1376.00,380.17)(1.500,1.000){2}{\rule{0.361pt}{0.400pt}}
\put(1379,381.67){\rule{0.723pt}{0.400pt}}
\multiput(1379.00,381.17)(1.500,1.000){2}{\rule{0.361pt}{0.400pt}}
\put(1382,382.67){\rule{0.723pt}{0.400pt}}
\multiput(1382.00,382.17)(1.500,1.000){2}{\rule{0.361pt}{0.400pt}}
\put(1385,383.67){\rule{0.723pt}{0.400pt}}
\multiput(1385.00,383.17)(1.500,1.000){2}{\rule{0.361pt}{0.400pt}}
\put(1388,385.17){\rule{0.482pt}{0.400pt}}
\multiput(1388.00,384.17)(1.000,2.000){2}{\rule{0.241pt}{0.400pt}}
\put(1390,386.67){\rule{0.723pt}{0.400pt}}
\multiput(1390.00,386.17)(1.500,1.000){2}{\rule{0.361pt}{0.400pt}}
\put(1393,387.67){\rule{0.723pt}{0.400pt}}
\multiput(1393.00,387.17)(1.500,1.000){2}{\rule{0.361pt}{0.400pt}}
\put(1396,388.67){\rule{0.723pt}{0.400pt}}
\multiput(1396.00,388.17)(1.500,1.000){2}{\rule{0.361pt}{0.400pt}}
\put(1399,390.17){\rule{0.700pt}{0.400pt}}
\multiput(1399.00,389.17)(1.547,2.000){2}{\rule{0.350pt}{0.400pt}}
\put(1402,391.67){\rule{0.723pt}{0.400pt}}
\multiput(1402.00,391.17)(1.500,1.000){2}{\rule{0.361pt}{0.400pt}}
\put(1405,392.67){\rule{0.723pt}{0.400pt}}
\multiput(1405.00,392.17)(1.500,1.000){2}{\rule{0.361pt}{0.400pt}}
\put(1408,393.67){\rule{0.482pt}{0.400pt}}
\multiput(1408.00,393.17)(1.000,1.000){2}{\rule{0.241pt}{0.400pt}}
\put(1410,395.17){\rule{0.700pt}{0.400pt}}
\multiput(1410.00,394.17)(1.547,2.000){2}{\rule{0.350pt}{0.400pt}}
\put(1410,396.67){\rule{0.723pt}{0.400pt}}
\multiput(1411.50,396.17)(-1.500,1.000){2}{\rule{0.361pt}{0.400pt}}
\put(1408,397.67){\rule{0.482pt}{0.400pt}}
\multiput(1409.00,397.17)(-1.000,1.000){2}{\rule{0.241pt}{0.400pt}}
\put(1405,398.67){\rule{0.723pt}{0.400pt}}
\multiput(1406.50,398.17)(-1.500,1.000){2}{\rule{0.361pt}{0.400pt}}
\put(1402,399.67){\rule{0.723pt}{0.400pt}}
\multiput(1403.50,399.17)(-1.500,1.000){2}{\rule{0.361pt}{0.400pt}}
\put(1399,401.17){\rule{0.700pt}{0.400pt}}
\multiput(1400.55,400.17)(-1.547,2.000){2}{\rule{0.350pt}{0.400pt}}
\put(1396,402.67){\rule{0.723pt}{0.400pt}}
\multiput(1397.50,402.17)(-1.500,1.000){2}{\rule{0.361pt}{0.400pt}}
\put(1393,403.67){\rule{0.723pt}{0.400pt}}
\multiput(1394.50,403.17)(-1.500,1.000){2}{\rule{0.361pt}{0.400pt}}
\put(1390,404.67){\rule{0.723pt}{0.400pt}}
\multiput(1391.50,404.17)(-1.500,1.000){2}{\rule{0.361pt}{0.400pt}}
\put(1388,406.17){\rule{0.482pt}{0.400pt}}
\multiput(1389.00,405.17)(-1.000,2.000){2}{\rule{0.241pt}{0.400pt}}
\put(1385,407.67){\rule{0.723pt}{0.400pt}}
\multiput(1386.50,407.17)(-1.500,1.000){2}{\rule{0.361pt}{0.400pt}}
\put(1382,408.67){\rule{0.723pt}{0.400pt}}
\multiput(1383.50,408.17)(-1.500,1.000){2}{\rule{0.361pt}{0.400pt}}
\put(1379,409.67){\rule{0.723pt}{0.400pt}}
\multiput(1380.50,409.17)(-1.500,1.000){2}{\rule{0.361pt}{0.400pt}}
\put(1376,411.17){\rule{0.700pt}{0.400pt}}
\multiput(1377.55,410.17)(-1.547,2.000){2}{\rule{0.350pt}{0.400pt}}
\put(1373,412.67){\rule{0.723pt}{0.400pt}}
\multiput(1374.50,412.17)(-1.500,1.000){2}{\rule{0.361pt}{0.400pt}}
\put(1371,413.67){\rule{0.482pt}{0.400pt}}
\multiput(1372.00,413.17)(-1.000,1.000){2}{\rule{0.241pt}{0.400pt}}
\put(1368,414.67){\rule{0.723pt}{0.400pt}}
\multiput(1369.50,414.17)(-1.500,1.000){2}{\rule{0.361pt}{0.400pt}}
\put(1365,415.67){\rule{0.723pt}{0.400pt}}
\multiput(1366.50,415.17)(-1.500,1.000){2}{\rule{0.361pt}{0.400pt}}
\put(1362,417.17){\rule{0.700pt}{0.400pt}}
\multiput(1363.55,416.17)(-1.547,2.000){2}{\rule{0.350pt}{0.400pt}}
\put(1359,418.67){\rule{0.723pt}{0.400pt}}
\multiput(1360.50,418.17)(-1.500,1.000){2}{\rule{0.361pt}{0.400pt}}
\put(1356,419.67){\rule{0.723pt}{0.400pt}}
\multiput(1357.50,419.17)(-1.500,1.000){2}{\rule{0.361pt}{0.400pt}}
\put(1354,420.67){\rule{0.482pt}{0.400pt}}
\multiput(1355.00,420.17)(-1.000,1.000){2}{\rule{0.241pt}{0.400pt}}
\put(1351,422.17){\rule{0.700pt}{0.400pt}}
\multiput(1352.55,421.17)(-1.547,2.000){2}{\rule{0.350pt}{0.400pt}}
\put(1348,423.67){\rule{0.723pt}{0.400pt}}
\multiput(1349.50,423.17)(-1.500,1.000){2}{\rule{0.361pt}{0.400pt}}
\put(1345,424.67){\rule{0.723pt}{0.400pt}}
\multiput(1346.50,424.17)(-1.500,1.000){2}{\rule{0.361pt}{0.400pt}}
\put(1342,425.67){\rule{0.723pt}{0.400pt}}
\multiput(1343.50,425.17)(-1.500,1.000){2}{\rule{0.361pt}{0.400pt}}
\put(1339,427.17){\rule{0.700pt}{0.400pt}}
\multiput(1340.55,426.17)(-1.547,2.000){2}{\rule{0.350pt}{0.400pt}}
\put(1336,428.67){\rule{0.723pt}{0.400pt}}
\multiput(1337.50,428.17)(-1.500,1.000){2}{\rule{0.361pt}{0.400pt}}
\put(1334,429.67){\rule{0.482pt}{0.400pt}}
\multiput(1335.00,429.17)(-1.000,1.000){2}{\rule{0.241pt}{0.400pt}}
\put(1331,430.67){\rule{0.723pt}{0.400pt}}
\multiput(1332.50,430.17)(-1.500,1.000){2}{\rule{0.361pt}{0.400pt}}
\put(1328,432.17){\rule{0.700pt}{0.400pt}}
\multiput(1329.55,431.17)(-1.547,2.000){2}{\rule{0.350pt}{0.400pt}}
\put(1325,433.67){\rule{0.723pt}{0.400pt}}
\multiput(1326.50,433.17)(-1.500,1.000){2}{\rule{0.361pt}{0.400pt}}
\put(1322,434.67){\rule{0.723pt}{0.400pt}}
\multiput(1323.50,434.17)(-1.500,1.000){2}{\rule{0.361pt}{0.400pt}}
\put(1319,435.67){\rule{0.723pt}{0.400pt}}
\multiput(1320.50,435.17)(-1.500,1.000){2}{\rule{0.361pt}{0.400pt}}
\put(1317,436.67){\rule{0.482pt}{0.400pt}}
\multiput(1318.00,436.17)(-1.000,1.000){2}{\rule{0.241pt}{0.400pt}}
\put(1314,438.17){\rule{0.700pt}{0.400pt}}
\multiput(1315.55,437.17)(-1.547,2.000){2}{\rule{0.350pt}{0.400pt}}
\put(1311,439.67){\rule{0.723pt}{0.400pt}}
\multiput(1312.50,439.17)(-1.500,1.000){2}{\rule{0.361pt}{0.400pt}}
\put(1308,440.67){\rule{0.723pt}{0.400pt}}
\multiput(1309.50,440.17)(-1.500,1.000){2}{\rule{0.361pt}{0.400pt}}
\put(1305,441.67){\rule{0.723pt}{0.400pt}}
\multiput(1306.50,441.17)(-1.500,1.000){2}{\rule{0.361pt}{0.400pt}}
\put(1302,443.17){\rule{0.700pt}{0.400pt}}
\multiput(1303.55,442.17)(-1.547,2.000){2}{\rule{0.350pt}{0.400pt}}
\put(1300,444.67){\rule{0.482pt}{0.400pt}}
\multiput(1301.00,444.17)(-1.000,1.000){2}{\rule{0.241pt}{0.400pt}}
\put(1297,445.67){\rule{0.723pt}{0.400pt}}
\multiput(1298.50,445.17)(-1.500,1.000){2}{\rule{0.361pt}{0.400pt}}
\put(1294,446.67){\rule{0.723pt}{0.400pt}}
\multiput(1295.50,446.17)(-1.500,1.000){2}{\rule{0.361pt}{0.400pt}}
\put(1291,448.17){\rule{0.700pt}{0.400pt}}
\multiput(1292.55,447.17)(-1.547,2.000){2}{\rule{0.350pt}{0.400pt}}
\put(1288,449.67){\rule{0.723pt}{0.400pt}}
\multiput(1289.50,449.17)(-1.500,1.000){2}{\rule{0.361pt}{0.400pt}}
\put(1285,450.67){\rule{0.723pt}{0.400pt}}
\multiput(1286.50,450.17)(-1.500,1.000){2}{\rule{0.361pt}{0.400pt}}
\put(1282,451.67){\rule{0.723pt}{0.400pt}}
\multiput(1283.50,451.17)(-1.500,1.000){2}{\rule{0.361pt}{0.400pt}}
\put(1280,452.67){\rule{0.482pt}{0.400pt}}
\multiput(1281.00,452.17)(-1.000,1.000){2}{\rule{0.241pt}{0.400pt}}
\put(1277,454.17){\rule{0.700pt}{0.400pt}}
\multiput(1278.55,453.17)(-1.547,2.000){2}{\rule{0.350pt}{0.400pt}}
\put(1274,455.67){\rule{0.723pt}{0.400pt}}
\multiput(1275.50,455.17)(-1.500,1.000){2}{\rule{0.361pt}{0.400pt}}
\put(1271,456.67){\rule{0.723pt}{0.400pt}}
\multiput(1272.50,456.17)(-1.500,1.000){2}{\rule{0.361pt}{0.400pt}}
\put(1268,457.67){\rule{0.723pt}{0.400pt}}
\multiput(1269.50,457.17)(-1.500,1.000){2}{\rule{0.361pt}{0.400pt}}
\put(1265,459.17){\rule{0.700pt}{0.400pt}}
\multiput(1266.55,458.17)(-1.547,2.000){2}{\rule{0.350pt}{0.400pt}}
\put(1263,460.67){\rule{0.482pt}{0.400pt}}
\multiput(1264.00,460.17)(-1.000,1.000){2}{\rule{0.241pt}{0.400pt}}
\put(1260,461.67){\rule{0.723pt}{0.400pt}}
\multiput(1261.50,461.17)(-1.500,1.000){2}{\rule{0.361pt}{0.400pt}}
\put(1257,462.67){\rule{0.723pt}{0.400pt}}
\multiput(1258.50,462.17)(-1.500,1.000){2}{\rule{0.361pt}{0.400pt}}
\put(1254,464.17){\rule{0.700pt}{0.400pt}}
\multiput(1255.55,463.17)(-1.547,2.000){2}{\rule{0.350pt}{0.400pt}}
\put(1251,465.67){\rule{0.723pt}{0.400pt}}
\multiput(1252.50,465.17)(-1.500,1.000){2}{\rule{0.361pt}{0.400pt}}
\put(1248,466.67){\rule{0.723pt}{0.400pt}}
\multiput(1249.50,466.17)(-1.500,1.000){2}{\rule{0.361pt}{0.400pt}}
\put(1246,467.67){\rule{0.482pt}{0.400pt}}
\multiput(1247.00,467.17)(-1.000,1.000){2}{\rule{0.241pt}{0.400pt}}
\put(1243,468.67){\rule{0.723pt}{0.400pt}}
\multiput(1244.50,468.17)(-1.500,1.000){2}{\rule{0.361pt}{0.400pt}}
\put(1240,470.17){\rule{0.700pt}{0.400pt}}
\multiput(1241.55,469.17)(-1.547,2.000){2}{\rule{0.350pt}{0.400pt}}
\put(1237,471.67){\rule{0.723pt}{0.400pt}}
\multiput(1238.50,471.17)(-1.500,1.000){2}{\rule{0.361pt}{0.400pt}}
\put(1234,472.67){\rule{0.723pt}{0.400pt}}
\multiput(1235.50,472.17)(-1.500,1.000){2}{\rule{0.361pt}{0.400pt}}
\put(1231,473.67){\rule{0.723pt}{0.400pt}}
\multiput(1232.50,473.17)(-1.500,1.000){2}{\rule{0.361pt}{0.400pt}}
\put(1228,475.17){\rule{0.700pt}{0.400pt}}
\multiput(1229.55,474.17)(-1.547,2.000){2}{\rule{0.350pt}{0.400pt}}
\put(1226,476.67){\rule{0.482pt}{0.400pt}}
\multiput(1227.00,476.17)(-1.000,1.000){2}{\rule{0.241pt}{0.400pt}}
\put(1223,477.67){\rule{0.723pt}{0.400pt}}
\multiput(1224.50,477.17)(-1.500,1.000){2}{\rule{0.361pt}{0.400pt}}
\put(1220,478.67){\rule{0.723pt}{0.400pt}}
\multiput(1221.50,478.17)(-1.500,1.000){2}{\rule{0.361pt}{0.400pt}}
\put(1217,480.17){\rule{0.700pt}{0.400pt}}
\multiput(1218.55,479.17)(-1.547,2.000){2}{\rule{0.350pt}{0.400pt}}
\put(1214,481.67){\rule{0.723pt}{0.400pt}}
\multiput(1215.50,481.17)(-1.500,1.000){2}{\rule{0.361pt}{0.400pt}}
\put(1211,482.67){\rule{0.723pt}{0.400pt}}
\multiput(1212.50,482.17)(-1.500,1.000){2}{\rule{0.361pt}{0.400pt}}
\put(1209,483.67){\rule{0.482pt}{0.400pt}}
\multiput(1210.00,483.17)(-1.000,1.000){2}{\rule{0.241pt}{0.400pt}}
\put(1206,484.67){\rule{0.723pt}{0.400pt}}
\multiput(1207.50,484.17)(-1.500,1.000){2}{\rule{0.361pt}{0.400pt}}
\put(1203,486.17){\rule{0.700pt}{0.400pt}}
\multiput(1204.55,485.17)(-1.547,2.000){2}{\rule{0.350pt}{0.400pt}}
\put(1200,487.67){\rule{0.723pt}{0.400pt}}
\multiput(1201.50,487.17)(-1.500,1.000){2}{\rule{0.361pt}{0.400pt}}
\put(1197,488.67){\rule{0.723pt}{0.400pt}}
\multiput(1198.50,488.17)(-1.500,1.000){2}{\rule{0.361pt}{0.400pt}}
\put(1194,489.67){\rule{0.723pt}{0.400pt}}
\multiput(1195.50,489.17)(-1.500,1.000){2}{\rule{0.361pt}{0.400pt}}
\put(1192,491.17){\rule{0.482pt}{0.400pt}}
\multiput(1193.00,490.17)(-1.000,2.000){2}{\rule{0.241pt}{0.400pt}}
\put(1189,492.67){\rule{0.723pt}{0.400pt}}
\multiput(1190.50,492.17)(-1.500,1.000){2}{\rule{0.361pt}{0.400pt}}
\put(1186,493.67){\rule{0.723pt}{0.400pt}}
\multiput(1187.50,493.17)(-1.500,1.000){2}{\rule{0.361pt}{0.400pt}}
\put(1183,494.67){\rule{0.723pt}{0.400pt}}
\multiput(1184.50,494.17)(-1.500,1.000){2}{\rule{0.361pt}{0.400pt}}
\put(1180,496.17){\rule{0.700pt}{0.400pt}}
\multiput(1181.55,495.17)(-1.547,2.000){2}{\rule{0.350pt}{0.400pt}}
\put(1177,497.67){\rule{0.723pt}{0.400pt}}
\multiput(1178.50,497.17)(-1.500,1.000){2}{\rule{0.361pt}{0.400pt}}
\put(1174,498.67){\rule{0.723pt}{0.400pt}}
\multiput(1175.50,498.17)(-1.500,1.000){2}{\rule{0.361pt}{0.400pt}}
\put(1172,499.67){\rule{0.482pt}{0.400pt}}
\multiput(1173.00,499.17)(-1.000,1.000){2}{\rule{0.241pt}{0.400pt}}
\put(1169,501.17){\rule{0.700pt}{0.400pt}}
\multiput(1170.55,500.17)(-1.547,2.000){2}{\rule{0.350pt}{0.400pt}}
\put(1166,502.67){\rule{0.723pt}{0.400pt}}
\multiput(1167.50,502.17)(-1.500,1.000){2}{\rule{0.361pt}{0.400pt}}
\put(1163,503.67){\rule{0.723pt}{0.400pt}}
\multiput(1164.50,503.17)(-1.500,1.000){2}{\rule{0.361pt}{0.400pt}}
\put(1160,504.67){\rule{0.723pt}{0.400pt}}
\multiput(1161.50,504.17)(-1.500,1.000){2}{\rule{0.361pt}{0.400pt}}
\put(1157,505.67){\rule{0.723pt}{0.400pt}}
\multiput(1158.50,505.17)(-1.500,1.000){2}{\rule{0.361pt}{0.400pt}}
\put(1155,507.17){\rule{0.482pt}{0.400pt}}
\multiput(1156.00,506.17)(-1.000,2.000){2}{\rule{0.241pt}{0.400pt}}
\put(1152,508.67){\rule{0.723pt}{0.400pt}}
\multiput(1153.50,508.17)(-1.500,1.000){2}{\rule{0.361pt}{0.400pt}}
\put(1149,509.67){\rule{0.723pt}{0.400pt}}
\multiput(1150.50,509.17)(-1.500,1.000){2}{\rule{0.361pt}{0.400pt}}
\put(1146,510.67){\rule{0.723pt}{0.400pt}}
\multiput(1147.50,510.17)(-1.500,1.000){2}{\rule{0.361pt}{0.400pt}}
\put(1143,512.17){\rule{0.700pt}{0.400pt}}
\multiput(1144.55,511.17)(-1.547,2.000){2}{\rule{0.350pt}{0.400pt}}
\put(1140,513.67){\rule{0.723pt}{0.400pt}}
\multiput(1141.50,513.17)(-1.500,1.000){2}{\rule{0.361pt}{0.400pt}}
\put(1138,514.67){\rule{0.482pt}{0.400pt}}
\multiput(1139.00,514.17)(-1.000,1.000){2}{\rule{0.241pt}{0.400pt}}
\put(1135,515.67){\rule{0.723pt}{0.400pt}}
\multiput(1136.50,515.17)(-1.500,1.000){2}{\rule{0.361pt}{0.400pt}}
\put(1132,517.17){\rule{0.700pt}{0.400pt}}
\multiput(1133.55,516.17)(-1.547,2.000){2}{\rule{0.350pt}{0.400pt}}
\put(1129,518.67){\rule{0.723pt}{0.400pt}}
\multiput(1130.50,518.17)(-1.500,1.000){2}{\rule{0.361pt}{0.400pt}}
\put(1126,519.67){\rule{0.723pt}{0.400pt}}
\multiput(1127.50,519.17)(-1.500,1.000){2}{\rule{0.361pt}{0.400pt}}
\put(1123,520.67){\rule{0.723pt}{0.400pt}}
\multiput(1124.50,520.17)(-1.500,1.000){2}{\rule{0.361pt}{0.400pt}}
\put(1120,521.67){\rule{0.723pt}{0.400pt}}
\multiput(1121.50,521.17)(-1.500,1.000){2}{\rule{0.361pt}{0.400pt}}
\put(1118,523.17){\rule{0.482pt}{0.400pt}}
\multiput(1119.00,522.17)(-1.000,2.000){2}{\rule{0.241pt}{0.400pt}}
\put(1115,524.67){\rule{0.723pt}{0.400pt}}
\multiput(1116.50,524.17)(-1.500,1.000){2}{\rule{0.361pt}{0.400pt}}
\put(1112,525.67){\rule{0.723pt}{0.400pt}}
\multiput(1113.50,525.17)(-1.500,1.000){2}{\rule{0.361pt}{0.400pt}}
\put(1109,526.67){\rule{0.723pt}{0.400pt}}
\multiput(1110.50,526.17)(-1.500,1.000){2}{\rule{0.361pt}{0.400pt}}
\put(1106,528.17){\rule{0.700pt}{0.400pt}}
\multiput(1107.55,527.17)(-1.547,2.000){2}{\rule{0.350pt}{0.400pt}}
\put(1103,529.67){\rule{0.723pt}{0.400pt}}
\multiput(1104.50,529.17)(-1.500,1.000){2}{\rule{0.361pt}{0.400pt}}
\put(1101,530.67){\rule{0.482pt}{0.400pt}}
\multiput(1102.00,530.17)(-1.000,1.000){2}{\rule{0.241pt}{0.400pt}}
\put(1098,531.67){\rule{0.723pt}{0.400pt}}
\multiput(1099.50,531.17)(-1.500,1.000){2}{\rule{0.361pt}{0.400pt}}
\put(1095,533.17){\rule{0.700pt}{0.400pt}}
\multiput(1096.55,532.17)(-1.547,2.000){2}{\rule{0.350pt}{0.400pt}}
\put(1092,534.67){\rule{0.723pt}{0.400pt}}
\multiput(1093.50,534.17)(-1.500,1.000){2}{\rule{0.361pt}{0.400pt}}
\put(1089,535.67){\rule{0.723pt}{0.400pt}}
\multiput(1090.50,535.17)(-1.500,1.000){2}{\rule{0.361pt}{0.400pt}}
\put(1086,536.67){\rule{0.723pt}{0.400pt}}
\multiput(1087.50,536.17)(-1.500,1.000){2}{\rule{0.361pt}{0.400pt}}
\put(1084,537.67){\rule{0.482pt}{0.400pt}}
\multiput(1085.00,537.17)(-1.000,1.000){2}{\rule{0.241pt}{0.400pt}}
\put(1081,539.17){\rule{0.700pt}{0.400pt}}
\multiput(1082.55,538.17)(-1.547,2.000){2}{\rule{0.350pt}{0.400pt}}
\put(1078,540.67){\rule{0.723pt}{0.400pt}}
\multiput(1079.50,540.17)(-1.500,1.000){2}{\rule{0.361pt}{0.400pt}}
\put(1075,541.67){\rule{0.723pt}{0.400pt}}
\multiput(1076.50,541.17)(-1.500,1.000){2}{\rule{0.361pt}{0.400pt}}
\put(1072,542.67){\rule{0.723pt}{0.400pt}}
\multiput(1073.50,542.17)(-1.500,1.000){2}{\rule{0.361pt}{0.400pt}}
\put(1069,544.17){\rule{0.700pt}{0.400pt}}
\multiput(1070.55,543.17)(-1.547,2.000){2}{\rule{0.350pt}{0.400pt}}
\put(1066,545.67){\rule{0.723pt}{0.400pt}}
\multiput(1067.50,545.17)(-1.500,1.000){2}{\rule{0.361pt}{0.400pt}}
\put(1064,546.67){\rule{0.482pt}{0.400pt}}
\multiput(1065.00,546.17)(-1.000,1.000){2}{\rule{0.241pt}{0.400pt}}
\put(1061,547.67){\rule{0.723pt}{0.400pt}}
\multiput(1062.50,547.17)(-1.500,1.000){2}{\rule{0.361pt}{0.400pt}}
\put(1058,549.17){\rule{0.700pt}{0.400pt}}
\multiput(1059.55,548.17)(-1.547,2.000){2}{\rule{0.350pt}{0.400pt}}
\put(1055,550.67){\rule{0.723pt}{0.400pt}}
\multiput(1056.50,550.17)(-1.500,1.000){2}{\rule{0.361pt}{0.400pt}}
\put(1052,551.67){\rule{0.723pt}{0.400pt}}
\multiput(1053.50,551.17)(-1.500,1.000){2}{\rule{0.361pt}{0.400pt}}
\put(1049,552.67){\rule{0.723pt}{0.400pt}}
\multiput(1050.50,552.17)(-1.500,1.000){2}{\rule{0.361pt}{0.400pt}}
\put(1047,553.67){\rule{0.482pt}{0.400pt}}
\multiput(1048.00,553.17)(-1.000,1.000){2}{\rule{0.241pt}{0.400pt}}
\put(1044,555.17){\rule{0.700pt}{0.400pt}}
\multiput(1045.55,554.17)(-1.547,2.000){2}{\rule{0.350pt}{0.400pt}}
\put(1041,556.67){\rule{0.723pt}{0.400pt}}
\multiput(1042.50,556.17)(-1.500,1.000){2}{\rule{0.361pt}{0.400pt}}
\put(1038,557.67){\rule{0.723pt}{0.400pt}}
\multiput(1039.50,557.17)(-1.500,1.000){2}{\rule{0.361pt}{0.400pt}}
\put(1035,558.67){\rule{0.723pt}{0.400pt}}
\multiput(1036.50,558.17)(-1.500,1.000){2}{\rule{0.361pt}{0.400pt}}
\put(1032,560.17){\rule{0.700pt}{0.400pt}}
\multiput(1033.55,559.17)(-1.547,2.000){2}{\rule{0.350pt}{0.400pt}}
\put(1030,561.67){\rule{0.482pt}{0.400pt}}
\multiput(1031.00,561.17)(-1.000,1.000){2}{\rule{0.241pt}{0.400pt}}
\put(1027,562.67){\rule{0.723pt}{0.400pt}}
\multiput(1028.50,562.17)(-1.500,1.000){2}{\rule{0.361pt}{0.400pt}}
\put(1024,563.67){\rule{0.723pt}{0.400pt}}
\multiput(1025.50,563.17)(-1.500,1.000){2}{\rule{0.361pt}{0.400pt}}
\put(1021,565.17){\rule{0.700pt}{0.400pt}}
\multiput(1022.55,564.17)(-1.547,2.000){2}{\rule{0.350pt}{0.400pt}}
\put(1018,566.67){\rule{0.723pt}{0.400pt}}
\multiput(1019.50,566.17)(-1.500,1.000){2}{\rule{0.361pt}{0.400pt}}
\put(1015,567.67){\rule{0.723pt}{0.400pt}}
\multiput(1016.50,567.17)(-1.500,1.000){2}{\rule{0.361pt}{0.400pt}}
\put(1012,568.67){\rule{0.723pt}{0.400pt}}
\multiput(1013.50,568.17)(-1.500,1.000){2}{\rule{0.361pt}{0.400pt}}
\put(1010,570.17){\rule{0.482pt}{0.400pt}}
\multiput(1011.00,569.17)(-1.000,2.000){2}{\rule{0.241pt}{0.400pt}}
\put(1007,571.67){\rule{0.723pt}{0.400pt}}
\multiput(1008.50,571.17)(-1.500,1.000){2}{\rule{0.361pt}{0.400pt}}
\put(1004,572.67){\rule{0.723pt}{0.400pt}}
\multiput(1005.50,572.17)(-1.500,1.000){2}{\rule{0.361pt}{0.400pt}}
\put(1001,573.67){\rule{0.723pt}{0.400pt}}
\multiput(1002.50,573.17)(-1.500,1.000){2}{\rule{0.361pt}{0.400pt}}
\put(998,574.67){\rule{0.723pt}{0.400pt}}
\multiput(999.50,574.17)(-1.500,1.000){2}{\rule{0.361pt}{0.400pt}}
\put(995,576.17){\rule{0.700pt}{0.400pt}}
\multiput(996.55,575.17)(-1.547,2.000){2}{\rule{0.350pt}{0.400pt}}
\put(993,577.67){\rule{0.482pt}{0.400pt}}
\multiput(994.00,577.17)(-1.000,1.000){2}{\rule{0.241pt}{0.400pt}}
\put(990,578.67){\rule{0.723pt}{0.400pt}}
\multiput(991.50,578.17)(-1.500,1.000){2}{\rule{0.361pt}{0.400pt}}
\put(987,579.67){\rule{0.723pt}{0.400pt}}
\multiput(988.50,579.17)(-1.500,1.000){2}{\rule{0.361pt}{0.400pt}}
\put(984,581.17){\rule{0.700pt}{0.400pt}}
\multiput(985.55,580.17)(-1.547,2.000){2}{\rule{0.350pt}{0.400pt}}
\put(981,582.67){\rule{0.723pt}{0.400pt}}
\multiput(982.50,582.17)(-1.500,1.000){2}{\rule{0.361pt}{0.400pt}}
\put(978,583.67){\rule{0.723pt}{0.400pt}}
\multiput(979.50,583.17)(-1.500,1.000){2}{\rule{0.361pt}{0.400pt}}
\put(976,584.67){\rule{0.482pt}{0.400pt}}
\multiput(977.00,584.17)(-1.000,1.000){2}{\rule{0.241pt}{0.400pt}}
\put(973,586.17){\rule{0.700pt}{0.400pt}}
\multiput(974.55,585.17)(-1.547,2.000){2}{\rule{0.350pt}{0.400pt}}
\put(970,587.67){\rule{0.723pt}{0.400pt}}
\multiput(971.50,587.17)(-1.500,1.000){2}{\rule{0.361pt}{0.400pt}}
\put(967,588.67){\rule{0.723pt}{0.400pt}}
\multiput(968.50,588.17)(-1.500,1.000){2}{\rule{0.361pt}{0.400pt}}
\put(964,589.67){\rule{0.723pt}{0.400pt}}
\multiput(965.50,589.17)(-1.500,1.000){2}{\rule{0.361pt}{0.400pt}}
\put(961,590.67){\rule{0.723pt}{0.400pt}}
\multiput(962.50,590.17)(-1.500,1.000){2}{\rule{0.361pt}{0.400pt}}
\put(958,592.17){\rule{0.700pt}{0.400pt}}
\multiput(959.55,591.17)(-1.547,2.000){2}{\rule{0.350pt}{0.400pt}}
\put(956,593.67){\rule{0.482pt}{0.400pt}}
\multiput(957.00,593.17)(-1.000,1.000){2}{\rule{0.241pt}{0.400pt}}
\put(953,594.67){\rule{0.723pt}{0.400pt}}
\multiput(954.50,594.17)(-1.500,1.000){2}{\rule{0.361pt}{0.400pt}}
\put(950,595.67){\rule{0.723pt}{0.400pt}}
\multiput(951.50,595.17)(-1.500,1.000){2}{\rule{0.361pt}{0.400pt}}
\put(947,597.17){\rule{0.700pt}{0.400pt}}
\multiput(948.55,596.17)(-1.547,2.000){2}{\rule{0.350pt}{0.400pt}}
\put(944,598.67){\rule{0.723pt}{0.400pt}}
\multiput(945.50,598.17)(-1.500,1.000){2}{\rule{0.361pt}{0.400pt}}
\put(941,599.67){\rule{0.723pt}{0.400pt}}
\multiput(942.50,599.17)(-1.500,1.000){2}{\rule{0.361pt}{0.400pt}}
\put(939,600.67){\rule{0.482pt}{0.400pt}}
\multiput(940.00,600.17)(-1.000,1.000){2}{\rule{0.241pt}{0.400pt}}
\put(936,602.17){\rule{0.700pt}{0.400pt}}
\multiput(937.55,601.17)(-1.547,2.000){2}{\rule{0.350pt}{0.400pt}}
\put(933,603.67){\rule{0.723pt}{0.400pt}}
\multiput(934.50,603.17)(-1.500,1.000){2}{\rule{0.361pt}{0.400pt}}
\put(930,604.67){\rule{0.723pt}{0.400pt}}
\multiput(931.50,604.17)(-1.500,1.000){2}{\rule{0.361pt}{0.400pt}}
\put(927,605.67){\rule{0.723pt}{0.400pt}}
\multiput(928.50,605.17)(-1.500,1.000){2}{\rule{0.361pt}{0.400pt}}
\put(924,606.67){\rule{0.723pt}{0.400pt}}
\multiput(925.50,606.17)(-1.500,1.000){2}{\rule{0.361pt}{0.400pt}}
\put(922,608.17){\rule{0.482pt}{0.400pt}}
\multiput(923.00,607.17)(-1.000,2.000){2}{\rule{0.241pt}{0.400pt}}
\put(919,609.67){\rule{0.723pt}{0.400pt}}
\multiput(920.50,609.17)(-1.500,1.000){2}{\rule{0.361pt}{0.400pt}}
\put(916,610.67){\rule{0.723pt}{0.400pt}}
\multiput(917.50,610.17)(-1.500,1.000){2}{\rule{0.361pt}{0.400pt}}
\put(913,611.67){\rule{0.723pt}{0.400pt}}
\multiput(914.50,611.17)(-1.500,1.000){2}{\rule{0.361pt}{0.400pt}}
\put(910,613.17){\rule{0.700pt}{0.400pt}}
\multiput(911.55,612.17)(-1.547,2.000){2}{\rule{0.350pt}{0.400pt}}
\put(907,614.67){\rule{0.723pt}{0.400pt}}
\multiput(908.50,614.17)(-1.500,1.000){2}{\rule{0.361pt}{0.400pt}}
\put(904,615.67){\rule{0.723pt}{0.400pt}}
\multiput(905.50,615.17)(-1.500,1.000){2}{\rule{0.361pt}{0.400pt}}
\put(902,616.67){\rule{0.482pt}{0.400pt}}
\multiput(903.00,616.17)(-1.000,1.000){2}{\rule{0.241pt}{0.400pt}}
\put(899,618.17){\rule{0.700pt}{0.400pt}}
\multiput(900.55,617.17)(-1.547,2.000){2}{\rule{0.350pt}{0.400pt}}
\put(896,619.67){\rule{0.723pt}{0.400pt}}
\multiput(897.50,619.17)(-1.500,1.000){2}{\rule{0.361pt}{0.400pt}}
\put(893,620.67){\rule{0.723pt}{0.400pt}}
\multiput(894.50,620.17)(-1.500,1.000){2}{\rule{0.361pt}{0.400pt}}
\put(890,621.67){\rule{0.723pt}{0.400pt}}
\multiput(891.50,621.17)(-1.500,1.000){2}{\rule{0.361pt}{0.400pt}}
\put(887,622.67){\rule{0.723pt}{0.400pt}}
\multiput(888.50,622.17)(-1.500,1.000){2}{\rule{0.361pt}{0.400pt}}
\put(885,624.17){\rule{0.482pt}{0.400pt}}
\multiput(886.00,623.17)(-1.000,2.000){2}{\rule{0.241pt}{0.400pt}}
\put(882,625.67){\rule{0.723pt}{0.400pt}}
\multiput(883.50,625.17)(-1.500,1.000){2}{\rule{0.361pt}{0.400pt}}
\put(879,626.67){\rule{0.723pt}{0.400pt}}
\multiput(880.50,626.17)(-1.500,1.000){2}{\rule{0.361pt}{0.400pt}}
\put(876,627.67){\rule{0.723pt}{0.400pt}}
\multiput(877.50,627.17)(-1.500,1.000){2}{\rule{0.361pt}{0.400pt}}
\put(873,629.17){\rule{0.700pt}{0.400pt}}
\multiput(874.55,628.17)(-1.547,2.000){2}{\rule{0.350pt}{0.400pt}}
\put(870,630.67){\rule{0.723pt}{0.400pt}}
\multiput(871.50,630.17)(-1.500,1.000){2}{\rule{0.361pt}{0.400pt}}
\put(868,631.67){\rule{0.482pt}{0.400pt}}
\multiput(869.00,631.17)(-1.000,1.000){2}{\rule{0.241pt}{0.400pt}}
\put(865,632.67){\rule{0.723pt}{0.400pt}}
\multiput(866.50,632.17)(-1.500,1.000){2}{\rule{0.361pt}{0.400pt}}
\put(862,634.17){\rule{0.700pt}{0.400pt}}
\multiput(863.55,633.17)(-1.547,2.000){2}{\rule{0.350pt}{0.400pt}}
\put(859,635.67){\rule{0.723pt}{0.400pt}}
\multiput(860.50,635.17)(-1.500,1.000){2}{\rule{0.361pt}{0.400pt}}
\put(856,636.67){\rule{0.723pt}{0.400pt}}
\multiput(857.50,636.17)(-1.500,1.000){2}{\rule{0.361pt}{0.400pt}}
\put(853,637.67){\rule{0.723pt}{0.400pt}}
\multiput(854.50,637.17)(-1.500,1.000){2}{\rule{0.361pt}{0.400pt}}
\put(850,638.67){\rule{0.723pt}{0.400pt}}
\multiput(851.50,638.17)(-1.500,1.000){2}{\rule{0.361pt}{0.400pt}}
\put(848,640.17){\rule{0.482pt}{0.400pt}}
\multiput(849.00,639.17)(-1.000,2.000){2}{\rule{0.241pt}{0.400pt}}
\put(845,641.67){\rule{0.723pt}{0.400pt}}
\multiput(846.50,641.17)(-1.500,1.000){2}{\rule{0.361pt}{0.400pt}}
\put(842,642.67){\rule{0.723pt}{0.400pt}}
\multiput(843.50,642.17)(-1.500,1.000){2}{\rule{0.361pt}{0.400pt}}
\put(839,643.67){\rule{0.723pt}{0.400pt}}
\multiput(840.50,643.17)(-1.500,1.000){2}{\rule{0.361pt}{0.400pt}}
\put(836,645.17){\rule{0.700pt}{0.400pt}}
\multiput(837.55,644.17)(-1.547,2.000){2}{\rule{0.350pt}{0.400pt}}
\put(833,646.67){\rule{0.723pt}{0.400pt}}
\multiput(834.50,646.17)(-1.500,1.000){2}{\rule{0.361pt}{0.400pt}}
\put(831,647.67){\rule{0.482pt}{0.400pt}}
\multiput(832.00,647.17)(-1.000,1.000){2}{\rule{0.241pt}{0.400pt}}
\put(828,648.67){\rule{0.723pt}{0.400pt}}
\multiput(829.50,648.17)(-1.500,1.000){2}{\rule{0.361pt}{0.400pt}}
\put(825,650.17){\rule{0.700pt}{0.400pt}}
\multiput(826.55,649.17)(-1.547,2.000){2}{\rule{0.350pt}{0.400pt}}
\put(822,651.67){\rule{0.723pt}{0.400pt}}
\multiput(823.50,651.17)(-1.500,1.000){2}{\rule{0.361pt}{0.400pt}}
\put(819,652.67){\rule{0.723pt}{0.400pt}}
\multiput(820.50,652.17)(-1.500,1.000){2}{\rule{0.361pt}{0.400pt}}
\put(816,653.67){\rule{0.723pt}{0.400pt}}
\multiput(817.50,653.17)(-1.500,1.000){2}{\rule{0.361pt}{0.400pt}}
\put(814,655.17){\rule{0.482pt}{0.400pt}}
\multiput(815.00,654.17)(-1.000,2.000){2}{\rule{0.241pt}{0.400pt}}
\put(811,656.67){\rule{0.723pt}{0.400pt}}
\multiput(812.50,656.17)(-1.500,1.000){2}{\rule{0.361pt}{0.400pt}}
\put(808,657.67){\rule{0.723pt}{0.400pt}}
\multiput(809.50,657.17)(-1.500,1.000){2}{\rule{0.361pt}{0.400pt}}
\put(805,658.67){\rule{0.723pt}{0.400pt}}
\multiput(806.50,658.17)(-1.500,1.000){2}{\rule{0.361pt}{0.400pt}}
\put(802,659.67){\rule{0.723pt}{0.400pt}}
\multiput(803.50,659.17)(-1.500,1.000){2}{\rule{0.361pt}{0.400pt}}
\put(799,661.17){\rule{0.700pt}{0.400pt}}
\multiput(800.55,660.17)(-1.547,2.000){2}{\rule{0.350pt}{0.400pt}}
\put(796,662.67){\rule{0.723pt}{0.400pt}}
\multiput(797.50,662.17)(-1.500,1.000){2}{\rule{0.361pt}{0.400pt}}
\put(794,663.67){\rule{0.482pt}{0.400pt}}
\multiput(795.00,663.17)(-1.000,1.000){2}{\rule{0.241pt}{0.400pt}}
\put(791,664.67){\rule{0.723pt}{0.400pt}}
\multiput(792.50,664.17)(-1.500,1.000){2}{\rule{0.361pt}{0.400pt}}
\put(788,666.17){\rule{0.700pt}{0.400pt}}
\multiput(789.55,665.17)(-1.547,2.000){2}{\rule{0.350pt}{0.400pt}}
\put(785,667.67){\rule{0.723pt}{0.400pt}}
\multiput(786.50,667.17)(-1.500,1.000){2}{\rule{0.361pt}{0.400pt}}
\put(782,668.67){\rule{0.723pt}{0.400pt}}
\multiput(783.50,668.17)(-1.500,1.000){2}{\rule{0.361pt}{0.400pt}}
\put(779,669.67){\rule{0.723pt}{0.400pt}}
\multiput(780.50,669.17)(-1.500,1.000){2}{\rule{0.361pt}{0.400pt}}
\put(777,671.17){\rule{0.482pt}{0.400pt}}
\multiput(778.00,670.17)(-1.000,2.000){2}{\rule{0.241pt}{0.400pt}}
\put(774,672.67){\rule{0.723pt}{0.400pt}}
\multiput(775.50,672.17)(-1.500,1.000){2}{\rule{0.361pt}{0.400pt}}
\put(771,673.67){\rule{0.723pt}{0.400pt}}
\multiput(772.50,673.17)(-1.500,1.000){2}{\rule{0.361pt}{0.400pt}}
\put(768,674.67){\rule{0.723pt}{0.400pt}}
\multiput(769.50,674.17)(-1.500,1.000){2}{\rule{0.361pt}{0.400pt}}
\put(765,675.67){\rule{0.723pt}{0.400pt}}
\multiput(766.50,675.17)(-1.500,1.000){2}{\rule{0.361pt}{0.400pt}}
\put(762,677.17){\rule{0.700pt}{0.400pt}}
\multiput(763.55,676.17)(-1.547,2.000){2}{\rule{0.350pt}{0.400pt}}
\put(760,678.67){\rule{0.482pt}{0.400pt}}
\multiput(761.00,678.17)(-1.000,1.000){2}{\rule{0.241pt}{0.400pt}}
\put(757,679.67){\rule{0.723pt}{0.400pt}}
\multiput(758.50,679.17)(-1.500,1.000){2}{\rule{0.361pt}{0.400pt}}
\put(754,680.67){\rule{0.723pt}{0.400pt}}
\multiput(755.50,680.17)(-1.500,1.000){2}{\rule{0.361pt}{0.400pt}}
\put(751,682.17){\rule{0.700pt}{0.400pt}}
\multiput(752.55,681.17)(-1.547,2.000){2}{\rule{0.350pt}{0.400pt}}
\put(748,683.67){\rule{0.723pt}{0.400pt}}
\multiput(749.50,683.17)(-1.500,1.000){2}{\rule{0.361pt}{0.400pt}}
\put(745,684.67){\rule{0.723pt}{0.400pt}}
\multiput(746.50,684.17)(-1.500,1.000){2}{\rule{0.361pt}{0.400pt}}
\put(742,685.67){\rule{0.723pt}{0.400pt}}
\multiput(743.50,685.17)(-1.500,1.000){2}{\rule{0.361pt}{0.400pt}}
\put(740,687.17){\rule{0.482pt}{0.400pt}}
\multiput(741.00,686.17)(-1.000,2.000){2}{\rule{0.241pt}{0.400pt}}
\put(737,688.67){\rule{0.723pt}{0.400pt}}
\multiput(738.50,688.17)(-1.500,1.000){2}{\rule{0.361pt}{0.400pt}}
\put(734,689.67){\rule{0.723pt}{0.400pt}}
\multiput(735.50,689.17)(-1.500,1.000){2}{\rule{0.361pt}{0.400pt}}
\put(731,690.67){\rule{0.723pt}{0.400pt}}
\multiput(732.50,690.17)(-1.500,1.000){2}{\rule{0.361pt}{0.400pt}}
\put(728,691.67){\rule{0.723pt}{0.400pt}}
\multiput(729.50,691.17)(-1.500,1.000){2}{\rule{0.361pt}{0.400pt}}
\put(725,693.17){\rule{0.700pt}{0.400pt}}
\multiput(726.55,692.17)(-1.547,2.000){2}{\rule{0.350pt}{0.400pt}}
\put(723,694.67){\rule{0.482pt}{0.400pt}}
\multiput(724.00,694.17)(-1.000,1.000){2}{\rule{0.241pt}{0.400pt}}
\put(720,695.67){\rule{0.723pt}{0.400pt}}
\multiput(721.50,695.17)(-1.500,1.000){2}{\rule{0.361pt}{0.400pt}}
\put(717,696.67){\rule{0.723pt}{0.400pt}}
\multiput(718.50,696.17)(-1.500,1.000){2}{\rule{0.361pt}{0.400pt}}
\put(714,698.17){\rule{0.700pt}{0.400pt}}
\multiput(715.55,697.17)(-1.547,2.000){2}{\rule{0.350pt}{0.400pt}}
\put(711,699.67){\rule{0.723pt}{0.400pt}}
\multiput(712.50,699.17)(-1.500,1.000){2}{\rule{0.361pt}{0.400pt}}
\put(708,700.67){\rule{0.723pt}{0.400pt}}
\multiput(709.50,700.17)(-1.500,1.000){2}{\rule{0.361pt}{0.400pt}}
\put(706,701.67){\rule{0.482pt}{0.400pt}}
\multiput(707.00,701.17)(-1.000,1.000){2}{\rule{0.241pt}{0.400pt}}
\put(703,703.17){\rule{0.700pt}{0.400pt}}
\multiput(704.55,702.17)(-1.547,2.000){2}{\rule{0.350pt}{0.400pt}}
\put(700,704.67){\rule{0.723pt}{0.400pt}}
\multiput(701.50,704.17)(-1.500,1.000){2}{\rule{0.361pt}{0.400pt}}
\put(697,705.67){\rule{0.723pt}{0.400pt}}
\multiput(698.50,705.17)(-1.500,1.000){2}{\rule{0.361pt}{0.400pt}}
\put(694,706.67){\rule{0.723pt}{0.400pt}}
\multiput(695.50,706.17)(-1.500,1.000){2}{\rule{0.361pt}{0.400pt}}
\put(691,707.67){\rule{0.723pt}{0.400pt}}
\multiput(692.50,707.17)(-1.500,1.000){2}{\rule{0.361pt}{0.400pt}}
\put(688,709.17){\rule{0.700pt}{0.400pt}}
\multiput(689.55,708.17)(-1.547,2.000){2}{\rule{0.350pt}{0.400pt}}
\put(686,710.67){\rule{0.482pt}{0.400pt}}
\multiput(687.00,710.17)(-1.000,1.000){2}{\rule{0.241pt}{0.400pt}}
\put(683,711.67){\rule{0.723pt}{0.400pt}}
\multiput(684.50,711.17)(-1.500,1.000){2}{\rule{0.361pt}{0.400pt}}
\put(680,712.67){\rule{0.723pt}{0.400pt}}
\multiput(681.50,712.17)(-1.500,1.000){2}{\rule{0.361pt}{0.400pt}}
\put(677,714.17){\rule{0.700pt}{0.400pt}}
\multiput(678.55,713.17)(-1.547,2.000){2}{\rule{0.350pt}{0.400pt}}
\put(674,715.67){\rule{0.723pt}{0.400pt}}
\multiput(675.50,715.17)(-1.500,1.000){2}{\rule{0.361pt}{0.400pt}}
\put(671,716.67){\rule{0.723pt}{0.400pt}}
\multiput(672.50,716.17)(-1.500,1.000){2}{\rule{0.361pt}{0.400pt}}
\put(669,717.67){\rule{0.482pt}{0.400pt}}
\multiput(670.00,717.17)(-1.000,1.000){2}{\rule{0.241pt}{0.400pt}}
\put(666,719.17){\rule{0.700pt}{0.400pt}}
\multiput(667.55,718.17)(-1.547,2.000){2}{\rule{0.350pt}{0.400pt}}
\put(663,720.67){\rule{0.723pt}{0.400pt}}
\multiput(664.50,720.17)(-1.500,1.000){2}{\rule{0.361pt}{0.400pt}}
\put(660,721.67){\rule{0.723pt}{0.400pt}}
\multiput(661.50,721.17)(-1.500,1.000){2}{\rule{0.361pt}{0.400pt}}
\put(657,722.67){\rule{0.723pt}{0.400pt}}
\multiput(658.50,722.17)(-1.500,1.000){2}{\rule{0.361pt}{0.400pt}}
\put(654,724.17){\rule{0.700pt}{0.400pt}}
\multiput(655.55,723.17)(-1.547,2.000){2}{\rule{0.350pt}{0.400pt}}
\put(652,725.67){\rule{0.482pt}{0.400pt}}
\multiput(653.00,725.17)(-1.000,1.000){2}{\rule{0.241pt}{0.400pt}}
\put(649,726.67){\rule{0.723pt}{0.400pt}}
\multiput(650.50,726.17)(-1.500,1.000){2}{\rule{0.361pt}{0.400pt}}
\put(646,727.67){\rule{0.723pt}{0.400pt}}
\multiput(647.50,727.17)(-1.500,1.000){2}{\rule{0.361pt}{0.400pt}}
\put(643,728.67){\rule{0.723pt}{0.400pt}}
\multiput(644.50,728.17)(-1.500,1.000){2}{\rule{0.361pt}{0.400pt}}
\put(640,730.17){\rule{0.700pt}{0.400pt}}
\multiput(641.55,729.17)(-1.547,2.000){2}{\rule{0.350pt}{0.400pt}}
\put(637,731.67){\rule{0.723pt}{0.400pt}}
\multiput(638.50,731.17)(-1.500,1.000){2}{\rule{0.361pt}{0.400pt}}
\put(634,732.67){\rule{0.723pt}{0.400pt}}
\multiput(635.50,732.17)(-1.500,1.000){2}{\rule{0.361pt}{0.400pt}}
\put(632,733.67){\rule{0.482pt}{0.400pt}}
\multiput(633.00,733.17)(-1.000,1.000){2}{\rule{0.241pt}{0.400pt}}
\put(629,735.17){\rule{0.700pt}{0.400pt}}
\multiput(630.55,734.17)(-1.547,2.000){2}{\rule{0.350pt}{0.400pt}}
\put(626,736.67){\rule{0.723pt}{0.400pt}}
\multiput(627.50,736.17)(-1.500,1.000){2}{\rule{0.361pt}{0.400pt}}
\put(623,737.67){\rule{0.723pt}{0.400pt}}
\multiput(624.50,737.17)(-1.500,1.000){2}{\rule{0.361pt}{0.400pt}}
\put(620,738.67){\rule{0.723pt}{0.400pt}}
\multiput(621.50,738.17)(-1.500,1.000){2}{\rule{0.361pt}{0.400pt}}
\put(617,740.17){\rule{0.700pt}{0.400pt}}
\multiput(618.55,739.17)(-1.547,2.000){2}{\rule{0.350pt}{0.400pt}}
\put(615,741.67){\rule{0.482pt}{0.400pt}}
\multiput(616.00,741.17)(-1.000,1.000){2}{\rule{0.241pt}{0.400pt}}
\put(612,742.67){\rule{0.723pt}{0.400pt}}
\multiput(613.50,742.17)(-1.500,1.000){2}{\rule{0.361pt}{0.400pt}}
\put(609,743.67){\rule{0.723pt}{0.400pt}}
\multiput(610.50,743.17)(-1.500,1.000){2}{\rule{0.361pt}{0.400pt}}
\put(606,744.67){\rule{0.723pt}{0.400pt}}
\multiput(607.50,744.17)(-1.500,1.000){2}{\rule{0.361pt}{0.400pt}}
\put(603,746.17){\rule{0.700pt}{0.400pt}}
\multiput(604.55,745.17)(-1.547,2.000){2}{\rule{0.350pt}{0.400pt}}
\put(600,747.67){\rule{0.723pt}{0.400pt}}
\multiput(601.50,747.17)(-1.500,1.000){2}{\rule{0.361pt}{0.400pt}}
\put(598,748.67){\rule{0.482pt}{0.400pt}}
\multiput(599.00,748.17)(-1.000,1.000){2}{\rule{0.241pt}{0.400pt}}
\put(595,749.67){\rule{0.723pt}{0.400pt}}
\multiput(596.50,749.17)(-1.500,1.000){2}{\rule{0.361pt}{0.400pt}}
\put(592,751.17){\rule{0.700pt}{0.400pt}}
\multiput(593.55,750.17)(-1.547,2.000){2}{\rule{0.350pt}{0.400pt}}
\put(589,752.67){\rule{0.723pt}{0.400pt}}
\multiput(590.50,752.17)(-1.500,1.000){2}{\rule{0.361pt}{0.400pt}}
\put(586,753.67){\rule{0.723pt}{0.400pt}}
\multiput(587.50,753.17)(-1.500,1.000){2}{\rule{0.361pt}{0.400pt}}
\put(583,754.67){\rule{0.723pt}{0.400pt}}
\multiput(584.50,754.17)(-1.500,1.000){2}{\rule{0.361pt}{0.400pt}}
\put(580,756.17){\rule{0.700pt}{0.400pt}}
\multiput(581.55,755.17)(-1.547,2.000){2}{\rule{0.350pt}{0.400pt}}
\put(578,757.67){\rule{0.482pt}{0.400pt}}
\multiput(579.00,757.17)(-1.000,1.000){2}{\rule{0.241pt}{0.400pt}}
\put(575,758.67){\rule{0.723pt}{0.400pt}}
\multiput(576.50,758.17)(-1.500,1.000){2}{\rule{0.361pt}{0.400pt}}
\put(572,759.67){\rule{0.723pt}{0.400pt}}
\multiput(573.50,759.17)(-1.500,1.000){2}{\rule{0.361pt}{0.400pt}}
\put(569,760.67){\rule{0.723pt}{0.400pt}}
\multiput(570.50,760.17)(-1.500,1.000){2}{\rule{0.361pt}{0.400pt}}
\put(566,762.17){\rule{0.700pt}{0.400pt}}
\multiput(567.55,761.17)(-1.547,2.000){2}{\rule{0.350pt}{0.400pt}}
\put(563,763.67){\rule{0.723pt}{0.400pt}}
\multiput(564.50,763.17)(-1.500,1.000){2}{\rule{0.361pt}{0.400pt}}
\put(561,764.67){\rule{0.482pt}{0.400pt}}
\multiput(562.00,764.17)(-1.000,1.000){2}{\rule{0.241pt}{0.400pt}}
\put(558,765.67){\rule{0.723pt}{0.400pt}}
\multiput(559.50,765.17)(-1.500,1.000){2}{\rule{0.361pt}{0.400pt}}
\put(555,767.17){\rule{0.700pt}{0.400pt}}
\multiput(556.55,766.17)(-1.547,2.000){2}{\rule{0.350pt}{0.400pt}}
\put(552,768.67){\rule{0.723pt}{0.400pt}}
\multiput(553.50,768.17)(-1.500,1.000){2}{\rule{0.361pt}{0.400pt}}
\put(549,769.67){\rule{0.723pt}{0.400pt}}
\multiput(550.50,769.17)(-1.500,1.000){2}{\rule{0.361pt}{0.400pt}}
\put(546,770.67){\rule{0.723pt}{0.400pt}}
\multiput(547.50,770.17)(-1.500,1.000){2}{\rule{0.361pt}{0.400pt}}
\put(544,772.17){\rule{0.482pt}{0.400pt}}
\multiput(545.00,771.17)(-1.000,2.000){2}{\rule{0.241pt}{0.400pt}}
\put(541,773.67){\rule{0.723pt}{0.400pt}}
\multiput(542.50,773.17)(-1.500,1.000){2}{\rule{0.361pt}{0.400pt}}
\put(538,774.67){\rule{0.723pt}{0.400pt}}
\multiput(539.50,774.17)(-1.500,1.000){2}{\rule{0.361pt}{0.400pt}}
\put(535,775.67){\rule{0.723pt}{0.400pt}}
\multiput(536.50,775.17)(-1.500,1.000){2}{\rule{0.361pt}{0.400pt}}
\put(532,776.67){\rule{0.723pt}{0.400pt}}
\multiput(533.50,776.17)(-1.500,1.000){2}{\rule{0.361pt}{0.400pt}}
\put(529,778.17){\rule{0.700pt}{0.400pt}}
\multiput(530.55,777.17)(-1.547,2.000){2}{\rule{0.350pt}{0.400pt}}
\put(526,779.67){\rule{0.723pt}{0.400pt}}
\multiput(527.50,779.17)(-1.500,1.000){2}{\rule{0.361pt}{0.400pt}}
\put(524,780.67){\rule{0.482pt}{0.400pt}}
\multiput(525.00,780.17)(-1.000,1.000){2}{\rule{0.241pt}{0.400pt}}
\put(521,781.67){\rule{0.723pt}{0.400pt}}
\multiput(522.50,781.17)(-1.500,1.000){2}{\rule{0.361pt}{0.400pt}}
\put(518,783.17){\rule{0.700pt}{0.400pt}}
\multiput(519.55,782.17)(-1.547,2.000){2}{\rule{0.350pt}{0.400pt}}
\put(515,784.67){\rule{0.723pt}{0.400pt}}
\multiput(516.50,784.17)(-1.500,1.000){2}{\rule{0.361pt}{0.400pt}}
\put(512,785.67){\rule{0.723pt}{0.400pt}}
\multiput(513.50,785.17)(-1.500,1.000){2}{\rule{0.361pt}{0.400pt}}
\put(509,786.67){\rule{0.723pt}{0.400pt}}
\multiput(510.50,786.17)(-1.500,1.000){2}{\rule{0.361pt}{0.400pt}}
\put(507,788.17){\rule{0.482pt}{0.400pt}}
\multiput(508.00,787.17)(-1.000,2.000){2}{\rule{0.241pt}{0.400pt}}
\put(504,789.67){\rule{0.723pt}{0.400pt}}
\multiput(505.50,789.17)(-1.500,1.000){2}{\rule{0.361pt}{0.400pt}}
\put(501,790.67){\rule{0.723pt}{0.400pt}}
\multiput(502.50,790.17)(-1.500,1.000){2}{\rule{0.361pt}{0.400pt}}
\put(498,791.67){\rule{0.723pt}{0.400pt}}
\multiput(499.50,791.17)(-1.500,1.000){2}{\rule{0.361pt}{0.400pt}}
\put(495,793.17){\rule{0.700pt}{0.400pt}}
\multiput(496.55,792.17)(-1.547,2.000){2}{\rule{0.350pt}{0.400pt}}
\put(492,794.67){\rule{0.723pt}{0.400pt}}
\multiput(493.50,794.17)(-1.500,1.000){2}{\rule{0.361pt}{0.400pt}}
\put(490,795.67){\rule{0.482pt}{0.400pt}}
\multiput(491.00,795.17)(-1.000,1.000){2}{\rule{0.241pt}{0.400pt}}
\put(487,796.67){\rule{0.723pt}{0.400pt}}
\multiput(488.50,796.17)(-1.500,1.000){2}{\rule{0.361pt}{0.400pt}}
\put(484,797.67){\rule{0.723pt}{0.400pt}}
\multiput(485.50,797.17)(-1.500,1.000){2}{\rule{0.361pt}{0.400pt}}
\put(481,799.17){\rule{0.700pt}{0.400pt}}
\multiput(482.55,798.17)(-1.547,2.000){2}{\rule{0.350pt}{0.400pt}}
\put(478,800.67){\rule{0.723pt}{0.400pt}}
\multiput(479.50,800.17)(-1.500,1.000){2}{\rule{0.361pt}{0.400pt}}
\put(475,801.67){\rule{0.723pt}{0.400pt}}
\multiput(476.50,801.17)(-1.500,1.000){2}{\rule{0.361pt}{0.400pt}}
\put(472,802.67){\rule{0.723pt}{0.400pt}}
\multiput(473.50,802.17)(-1.500,1.000){2}{\rule{0.361pt}{0.400pt}}
\put(470,804.17){\rule{0.482pt}{0.400pt}}
\multiput(471.00,803.17)(-1.000,2.000){2}{\rule{0.241pt}{0.400pt}}
\put(467,805.67){\rule{0.723pt}{0.400pt}}
\multiput(468.50,805.17)(-1.500,1.000){2}{\rule{0.361pt}{0.400pt}}
\put(464,806.67){\rule{0.723pt}{0.400pt}}
\multiput(465.50,806.17)(-1.500,1.000){2}{\rule{0.361pt}{0.400pt}}
\put(461,807.67){\rule{0.723pt}{0.400pt}}
\multiput(462.50,807.17)(-1.500,1.000){2}{\rule{0.361pt}{0.400pt}}
\put(458,809.17){\rule{0.700pt}{0.400pt}}
\multiput(459.55,808.17)(-1.547,2.000){2}{\rule{0.350pt}{0.400pt}}
\put(455,810.67){\rule{0.723pt}{0.400pt}}
\multiput(456.50,810.17)(-1.500,1.000){2}{\rule{0.361pt}{0.400pt}}
\put(453,811.67){\rule{0.482pt}{0.400pt}}
\multiput(454.00,811.17)(-1.000,1.000){2}{\rule{0.241pt}{0.400pt}}
\put(450,812.67){\rule{0.723pt}{0.400pt}}
\multiput(451.50,812.17)(-1.500,1.000){2}{\rule{0.361pt}{0.400pt}}
\put(447,813.67){\rule{0.723pt}{0.400pt}}
\multiput(448.50,813.17)(-1.500,1.000){2}{\rule{0.361pt}{0.400pt}}
\put(444,815.17){\rule{0.700pt}{0.400pt}}
\multiput(445.55,814.17)(-1.547,2.000){2}{\rule{0.350pt}{0.400pt}}
\put(441,816.67){\rule{0.723pt}{0.400pt}}
\multiput(442.50,816.17)(-1.500,1.000){2}{\rule{0.361pt}{0.400pt}}
\put(438,817.67){\rule{0.723pt}{0.400pt}}
\multiput(439.50,817.17)(-1.500,1.000){2}{\rule{0.361pt}{0.400pt}}
\put(436,818.67){\rule{0.482pt}{0.400pt}}
\multiput(437.00,818.17)(-1.000,1.000){2}{\rule{0.241pt}{0.400pt}}
\put(433,820.17){\rule{0.700pt}{0.400pt}}
\multiput(434.55,819.17)(-1.547,2.000){2}{\rule{0.350pt}{0.400pt}}
\put(430,821.67){\rule{0.723pt}{0.400pt}}
\multiput(431.50,821.17)(-1.500,1.000){2}{\rule{0.361pt}{0.400pt}}
\put(427,822.67){\rule{0.723pt}{0.400pt}}
\multiput(428.50,822.17)(-1.500,1.000){2}{\rule{0.361pt}{0.400pt}}
\put(424,823.67){\rule{0.723pt}{0.400pt}}
\multiput(425.50,823.17)(-1.500,1.000){2}{\rule{0.361pt}{0.400pt}}
\put(421,825.17){\rule{0.700pt}{0.400pt}}
\multiput(422.55,824.17)(-1.547,2.000){2}{\rule{0.350pt}{0.400pt}}
\put(418,826.67){\rule{0.723pt}{0.400pt}}
\multiput(419.50,826.17)(-1.500,1.000){2}{\rule{0.361pt}{0.400pt}}
\put(416,827.67){\rule{0.482pt}{0.400pt}}
\multiput(417.00,827.17)(-1.000,1.000){2}{\rule{0.241pt}{0.400pt}}
\put(413,828.67){\rule{0.723pt}{0.400pt}}
\multiput(414.50,828.17)(-1.500,1.000){2}{\rule{0.361pt}{0.400pt}}
\put(410,829.67){\rule{0.723pt}{0.400pt}}
\multiput(411.50,829.17)(-1.500,1.000){2}{\rule{0.361pt}{0.400pt}}
\put(407,831.17){\rule{0.700pt}{0.400pt}}
\multiput(408.55,830.17)(-1.547,2.000){2}{\rule{0.350pt}{0.400pt}}
\put(404,832.67){\rule{0.723pt}{0.400pt}}
\multiput(405.50,832.17)(-1.500,1.000){2}{\rule{0.361pt}{0.400pt}}
\put(401,833.67){\rule{0.723pt}{0.400pt}}
\multiput(402.50,833.17)(-1.500,1.000){2}{\rule{0.361pt}{0.400pt}}
\put(399,834.67){\rule{0.482pt}{0.400pt}}
\multiput(400.00,834.17)(-1.000,1.000){2}{\rule{0.241pt}{0.400pt}}
\put(130.0,82.0){\rule[-0.200pt]{0.400pt}{187.179pt}}
\put(130.0,82.0){\rule[-0.200pt]{315.338pt}{0.400pt}}
\put(1439.0,82.0){\rule[-0.200pt]{0.400pt}{187.179pt}}
\put(130.0,859.0){\rule[-0.200pt]{315.338pt}{0.400pt}}
\end{picture}

Plot for Ball 4:\\
% GNUPLOT: LaTeX picture
\setlength{\unitlength}{0.240900pt}
\ifx\plotpoint\undefined\newsavebox{\plotpoint}\fi
\sbox{\plotpoint}{\rule[-0.200pt]{0.400pt}{0.400pt}}%
\begin{picture}(1500,900)(0,0)
\sbox{\plotpoint}{\rule[-0.200pt]{0.400pt}{0.400pt}}%
\put(130.0,90.0){\rule[-0.200pt]{4.818pt}{0.400pt}}
\put(110,90){\makebox(0,0)[r]{ 0}}
\put(1419.0,90.0){\rule[-0.200pt]{4.818pt}{0.400pt}}
\put(130.0,242.0){\rule[-0.200pt]{4.818pt}{0.400pt}}
\put(110,242){\makebox(0,0)[r]{ 0.2}}
\put(1419.0,242.0){\rule[-0.200pt]{4.818pt}{0.400pt}}
\put(130.0,394.0){\rule[-0.200pt]{4.818pt}{0.400pt}}
\put(110,394){\makebox(0,0)[r]{ 0.4}}
\put(1419.0,394.0){\rule[-0.200pt]{4.818pt}{0.400pt}}
\put(130.0,547.0){\rule[-0.200pt]{4.818pt}{0.400pt}}
\put(110,547){\makebox(0,0)[r]{ 0.6}}
\put(1419.0,547.0){\rule[-0.200pt]{4.818pt}{0.400pt}}
\put(130.0,699.0){\rule[-0.200pt]{4.818pt}{0.400pt}}
\put(110,699){\makebox(0,0)[r]{ 0.8}}
\put(1419.0,699.0){\rule[-0.200pt]{4.818pt}{0.400pt}}
\put(130.0,851.0){\rule[-0.200pt]{4.818pt}{0.400pt}}
\put(110,851){\makebox(0,0)[r]{ 1}}
\put(1419.0,851.0){\rule[-0.200pt]{4.818pt}{0.400pt}}
\put(130.0,82.0){\rule[-0.200pt]{0.400pt}{4.818pt}}
\put(130,41){\makebox(0,0){ 0}}
\put(130.0,839.0){\rule[-0.200pt]{0.400pt}{4.818pt}}
\put(392.0,82.0){\rule[-0.200pt]{0.400pt}{4.818pt}}
\put(392,41){\makebox(0,0){ 0.2}}
\put(392.0,839.0){\rule[-0.200pt]{0.400pt}{4.818pt}}
\put(654.0,82.0){\rule[-0.200pt]{0.400pt}{4.818pt}}
\put(654,41){\makebox(0,0){ 0.4}}
\put(654.0,839.0){\rule[-0.200pt]{0.400pt}{4.818pt}}
\put(915.0,82.0){\rule[-0.200pt]{0.400pt}{4.818pt}}
\put(915,41){\makebox(0,0){ 0.6}}
\put(915.0,839.0){\rule[-0.200pt]{0.400pt}{4.818pt}}
\put(1177.0,82.0){\rule[-0.200pt]{0.400pt}{4.818pt}}
\put(1177,41){\makebox(0,0){ 0.8}}
\put(1177.0,839.0){\rule[-0.200pt]{0.400pt}{4.818pt}}
\put(1439.0,82.0){\rule[-0.200pt]{0.400pt}{4.818pt}}
\put(1439,41){\makebox(0,0){ 1}}
\put(1439.0,839.0){\rule[-0.200pt]{0.400pt}{4.818pt}}
\put(130.0,82.0){\rule[-0.200pt]{0.400pt}{187.179pt}}
\put(130.0,82.0){\rule[-0.200pt]{315.338pt}{0.400pt}}
\put(1439.0,82.0){\rule[-0.200pt]{0.400pt}{187.179pt}}
\put(130.0,859.0){\rule[-0.200pt]{315.338pt}{0.400pt}}
\put(1279,819){\makebox(0,0)[r]{'-'}}
\put(1299.0,819.0){\rule[-0.200pt]{24.090pt}{0.400pt}}
\put(551,488){\usebox{\plotpoint}}
\put(551,487.67){\rule{0.964pt}{0.400pt}}
\multiput(551.00,487.17)(2.000,1.000){2}{\rule{0.482pt}{0.400pt}}
\put(558,488.67){\rule{0.964pt}{0.400pt}}
\multiput(558.00,488.17)(2.000,1.000){2}{\rule{0.482pt}{0.400pt}}
\put(562,489.67){\rule{0.964pt}{0.400pt}}
\multiput(562.00,489.17)(2.000,1.000){2}{\rule{0.482pt}{0.400pt}}
\put(555.0,489.0){\rule[-0.200pt]{0.723pt}{0.400pt}}
\put(570,490.67){\rule{0.964pt}{0.400pt}}
\multiput(570.00,490.17)(2.000,1.000){2}{\rule{0.482pt}{0.400pt}}
\put(572.67,489){\rule{0.400pt}{0.723pt}}
\multiput(573.17,490.50)(-1.000,-1.500){2}{\rule{0.400pt}{0.361pt}}
\put(571.67,487){\rule{0.400pt}{0.482pt}}
\multiput(572.17,488.00)(-1.000,-1.000){2}{\rule{0.400pt}{0.241pt}}
\put(570.67,485){\rule{0.400pt}{0.482pt}}
\multiput(571.17,486.00)(-1.000,-1.000){2}{\rule{0.400pt}{0.241pt}}
\put(569.67,483){\rule{0.400pt}{0.482pt}}
\multiput(570.17,484.00)(-1.000,-1.000){2}{\rule{0.400pt}{0.241pt}}
\put(568.67,480){\rule{0.400pt}{0.723pt}}
\multiput(569.17,481.50)(-1.000,-1.500){2}{\rule{0.400pt}{0.361pt}}
\put(567.67,478){\rule{0.400pt}{0.482pt}}
\multiput(568.17,479.00)(-1.000,-1.000){2}{\rule{0.400pt}{0.241pt}}
\put(566.67,476){\rule{0.400pt}{0.482pt}}
\multiput(567.17,477.00)(-1.000,-1.000){2}{\rule{0.400pt}{0.241pt}}
\put(565.67,473){\rule{0.400pt}{0.723pt}}
\multiput(566.17,474.50)(-1.000,-1.500){2}{\rule{0.400pt}{0.361pt}}
\put(564.67,471){\rule{0.400pt}{0.482pt}}
\multiput(565.17,472.00)(-1.000,-1.000){2}{\rule{0.400pt}{0.241pt}}
\put(563.67,469){\rule{0.400pt}{0.482pt}}
\multiput(564.17,470.00)(-1.000,-1.000){2}{\rule{0.400pt}{0.241pt}}
\put(562.67,466){\rule{0.400pt}{0.723pt}}
\multiput(563.17,467.50)(-1.000,-1.500){2}{\rule{0.400pt}{0.361pt}}
\put(561.67,464){\rule{0.400pt}{0.482pt}}
\multiput(562.17,465.00)(-1.000,-1.000){2}{\rule{0.400pt}{0.241pt}}
\put(560.67,462){\rule{0.400pt}{0.482pt}}
\multiput(561.17,463.00)(-1.000,-1.000){2}{\rule{0.400pt}{0.241pt}}
\put(559.67,460){\rule{0.400pt}{0.482pt}}
\multiput(560.17,461.00)(-1.000,-1.000){2}{\rule{0.400pt}{0.241pt}}
\put(566.0,491.0){\rule[-0.200pt]{0.964pt}{0.400pt}}
\put(558.67,455){\rule{0.400pt}{0.482pt}}
\multiput(559.17,456.00)(-1.000,-1.000){2}{\rule{0.400pt}{0.241pt}}
\put(557.67,453){\rule{0.400pt}{0.482pt}}
\multiput(558.17,454.00)(-1.000,-1.000){2}{\rule{0.400pt}{0.241pt}}
\put(556.67,450){\rule{0.400pt}{0.723pt}}
\multiput(557.17,451.50)(-1.000,-1.500){2}{\rule{0.400pt}{0.361pt}}
\put(555.67,448){\rule{0.400pt}{0.482pt}}
\multiput(556.17,449.00)(-1.000,-1.000){2}{\rule{0.400pt}{0.241pt}}
\put(554.67,446){\rule{0.400pt}{0.482pt}}
\multiput(555.17,447.00)(-1.000,-1.000){2}{\rule{0.400pt}{0.241pt}}
\put(553.67,443){\rule{0.400pt}{0.723pt}}
\multiput(554.17,444.50)(-1.000,-1.500){2}{\rule{0.400pt}{0.361pt}}
\put(552.67,441){\rule{0.400pt}{0.482pt}}
\multiput(553.17,442.00)(-1.000,-1.000){2}{\rule{0.400pt}{0.241pt}}
\put(551.67,439){\rule{0.400pt}{0.482pt}}
\multiput(552.17,440.00)(-1.000,-1.000){2}{\rule{0.400pt}{0.241pt}}
\put(550.67,437){\rule{0.400pt}{0.482pt}}
\multiput(551.17,438.00)(-1.000,-1.000){2}{\rule{0.400pt}{0.241pt}}
\put(549.67,434){\rule{0.400pt}{0.723pt}}
\multiput(550.17,435.50)(-1.000,-1.500){2}{\rule{0.400pt}{0.361pt}}
\put(548.67,432){\rule{0.400pt}{0.482pt}}
\multiput(549.17,433.00)(-1.000,-1.000){2}{\rule{0.400pt}{0.241pt}}
\put(547.67,430){\rule{0.400pt}{0.482pt}}
\multiput(548.17,431.00)(-1.000,-1.000){2}{\rule{0.400pt}{0.241pt}}
\put(546.67,427){\rule{0.400pt}{0.723pt}}
\multiput(547.17,428.50)(-1.000,-1.500){2}{\rule{0.400pt}{0.361pt}}
\put(545.67,425){\rule{0.400pt}{0.482pt}}
\multiput(546.17,426.00)(-1.000,-1.000){2}{\rule{0.400pt}{0.241pt}}
\put(560.0,457.0){\rule[-0.200pt]{0.400pt}{0.723pt}}
\put(544.67,420){\rule{0.400pt}{0.723pt}}
\multiput(545.17,421.50)(-1.000,-1.500){2}{\rule{0.400pt}{0.361pt}}
\put(543.67,418){\rule{0.400pt}{0.482pt}}
\multiput(544.17,419.00)(-1.000,-1.000){2}{\rule{0.400pt}{0.241pt}}
\put(542.67,416){\rule{0.400pt}{0.482pt}}
\multiput(543.17,417.00)(-1.000,-1.000){2}{\rule{0.400pt}{0.241pt}}
\put(541.67,414){\rule{0.400pt}{0.482pt}}
\multiput(542.17,415.00)(-1.000,-1.000){2}{\rule{0.400pt}{0.241pt}}
\put(540.67,411){\rule{0.400pt}{0.723pt}}
\multiput(541.17,412.50)(-1.000,-1.500){2}{\rule{0.400pt}{0.361pt}}
\put(539.67,409){\rule{0.400pt}{0.482pt}}
\multiput(540.17,410.00)(-1.000,-1.000){2}{\rule{0.400pt}{0.241pt}}
\put(538.67,407){\rule{0.400pt}{0.482pt}}
\multiput(539.17,408.00)(-1.000,-1.000){2}{\rule{0.400pt}{0.241pt}}
\put(537.67,404){\rule{0.400pt}{0.723pt}}
\multiput(538.17,405.50)(-1.000,-1.500){2}{\rule{0.400pt}{0.361pt}}
\put(536.67,402){\rule{0.400pt}{0.482pt}}
\multiput(537.17,403.00)(-1.000,-1.000){2}{\rule{0.400pt}{0.241pt}}
\put(535.67,400){\rule{0.400pt}{0.482pt}}
\multiput(536.17,401.00)(-1.000,-1.000){2}{\rule{0.400pt}{0.241pt}}
\put(534.67,398){\rule{0.400pt}{0.482pt}}
\multiput(535.17,399.00)(-1.000,-1.000){2}{\rule{0.400pt}{0.241pt}}
\put(533.67,395){\rule{0.400pt}{0.723pt}}
\multiput(534.17,396.50)(-1.000,-1.500){2}{\rule{0.400pt}{0.361pt}}
\put(532.67,393){\rule{0.400pt}{0.482pt}}
\multiput(533.17,394.00)(-1.000,-1.000){2}{\rule{0.400pt}{0.241pt}}
\put(531.67,391){\rule{0.400pt}{0.482pt}}
\multiput(532.17,392.00)(-1.000,-1.000){2}{\rule{0.400pt}{0.241pt}}
\put(530.67,388){\rule{0.400pt}{0.723pt}}
\multiput(531.17,389.50)(-1.000,-1.500){2}{\rule{0.400pt}{0.361pt}}
\put(546.0,423.0){\rule[-0.200pt]{0.400pt}{0.482pt}}
\put(529.67,384){\rule{0.400pt}{0.482pt}}
\multiput(530.17,385.00)(-1.000,-1.000){2}{\rule{0.400pt}{0.241pt}}
\put(528.67,381){\rule{0.400pt}{0.723pt}}
\multiput(529.17,382.50)(-1.000,-1.500){2}{\rule{0.400pt}{0.361pt}}
\put(527.67,379){\rule{0.400pt}{0.482pt}}
\multiput(528.17,380.00)(-1.000,-1.000){2}{\rule{0.400pt}{0.241pt}}
\put(526.67,377){\rule{0.400pt}{0.482pt}}
\multiput(527.17,378.00)(-1.000,-1.000){2}{\rule{0.400pt}{0.241pt}}
\put(525.67,375){\rule{0.400pt}{0.482pt}}
\multiput(526.17,376.00)(-1.000,-1.000){2}{\rule{0.400pt}{0.241pt}}
\put(524.67,372){\rule{0.400pt}{0.723pt}}
\multiput(525.17,373.50)(-1.000,-1.500){2}{\rule{0.400pt}{0.361pt}}
\put(523.67,370){\rule{0.400pt}{0.482pt}}
\multiput(524.17,371.00)(-1.000,-1.000){2}{\rule{0.400pt}{0.241pt}}
\put(522.67,368){\rule{0.400pt}{0.482pt}}
\multiput(523.17,369.00)(-1.000,-1.000){2}{\rule{0.400pt}{0.241pt}}
\put(521.67,365){\rule{0.400pt}{0.723pt}}
\multiput(522.17,366.50)(-1.000,-1.500){2}{\rule{0.400pt}{0.361pt}}
\put(520.67,363){\rule{0.400pt}{0.482pt}}
\multiput(521.17,364.00)(-1.000,-1.000){2}{\rule{0.400pt}{0.241pt}}
\put(519.67,361){\rule{0.400pt}{0.482pt}}
\multiput(520.17,362.00)(-1.000,-1.000){2}{\rule{0.400pt}{0.241pt}}
\put(518.67,358){\rule{0.400pt}{0.723pt}}
\multiput(519.17,359.50)(-1.000,-1.500){2}{\rule{0.400pt}{0.361pt}}
\put(517.67,356){\rule{0.400pt}{0.482pt}}
\multiput(518.17,357.00)(-1.000,-1.000){2}{\rule{0.400pt}{0.241pt}}
\put(516.67,354){\rule{0.400pt}{0.482pt}}
\multiput(517.17,355.00)(-1.000,-1.000){2}{\rule{0.400pt}{0.241pt}}
\put(515.67,352){\rule{0.400pt}{0.482pt}}
\multiput(516.17,353.00)(-1.000,-1.000){2}{\rule{0.400pt}{0.241pt}}
\put(531.0,386.0){\rule[-0.200pt]{0.400pt}{0.482pt}}
\put(514.67,347){\rule{0.400pt}{0.482pt}}
\multiput(515.17,348.00)(-1.000,-1.000){2}{\rule{0.400pt}{0.241pt}}
\put(513.67,345){\rule{0.400pt}{0.482pt}}
\multiput(514.17,346.00)(-1.000,-1.000){2}{\rule{0.400pt}{0.241pt}}
\put(512.67,342){\rule{0.400pt}{0.723pt}}
\multiput(513.17,343.50)(-1.000,-1.500){2}{\rule{0.400pt}{0.361pt}}
\put(511.67,340){\rule{0.400pt}{0.482pt}}
\multiput(512.17,341.00)(-1.000,-1.000){2}{\rule{0.400pt}{0.241pt}}
\put(510.67,338){\rule{0.400pt}{0.482pt}}
\multiput(511.17,339.00)(-1.000,-1.000){2}{\rule{0.400pt}{0.241pt}}
\put(509.67,335){\rule{0.400pt}{0.723pt}}
\multiput(510.17,336.50)(-1.000,-1.500){2}{\rule{0.400pt}{0.361pt}}
\put(508.67,333){\rule{0.400pt}{0.482pt}}
\multiput(509.17,334.00)(-1.000,-1.000){2}{\rule{0.400pt}{0.241pt}}
\put(507.67,331){\rule{0.400pt}{0.482pt}}
\multiput(508.17,332.00)(-1.000,-1.000){2}{\rule{0.400pt}{0.241pt}}
\put(506.67,329){\rule{0.400pt}{0.482pt}}
\multiput(507.17,330.00)(-1.000,-1.000){2}{\rule{0.400pt}{0.241pt}}
\put(505.67,326){\rule{0.400pt}{0.723pt}}
\multiput(506.17,327.50)(-1.000,-1.500){2}{\rule{0.400pt}{0.361pt}}
\put(504.67,324){\rule{0.400pt}{0.482pt}}
\multiput(505.17,325.00)(-1.000,-1.000){2}{\rule{0.400pt}{0.241pt}}
\put(503.67,322){\rule{0.400pt}{0.482pt}}
\multiput(504.17,323.00)(-1.000,-1.000){2}{\rule{0.400pt}{0.241pt}}
\put(502.67,319){\rule{0.400pt}{0.723pt}}
\multiput(503.17,320.50)(-1.000,-1.500){2}{\rule{0.400pt}{0.361pt}}
\put(501.67,317){\rule{0.400pt}{0.482pt}}
\multiput(502.17,318.00)(-1.000,-1.000){2}{\rule{0.400pt}{0.241pt}}
\put(516.0,349.0){\rule[-0.200pt]{0.400pt}{0.723pt}}
\put(500.67,312){\rule{0.400pt}{0.723pt}}
\multiput(501.17,313.50)(-1.000,-1.500){2}{\rule{0.400pt}{0.361pt}}
\put(499.67,310){\rule{0.400pt}{0.482pt}}
\multiput(500.17,311.00)(-1.000,-1.000){2}{\rule{0.400pt}{0.241pt}}
\put(498.67,308){\rule{0.400pt}{0.482pt}}
\multiput(499.17,309.00)(-1.000,-1.000){2}{\rule{0.400pt}{0.241pt}}
\put(497.67,306){\rule{0.400pt}{0.482pt}}
\multiput(498.17,307.00)(-1.000,-1.000){2}{\rule{0.400pt}{0.241pt}}
\put(496.67,303){\rule{0.400pt}{0.723pt}}
\multiput(497.17,304.50)(-1.000,-1.500){2}{\rule{0.400pt}{0.361pt}}
\put(495.67,301){\rule{0.400pt}{0.482pt}}
\multiput(496.17,302.00)(-1.000,-1.000){2}{\rule{0.400pt}{0.241pt}}
\put(494.67,299){\rule{0.400pt}{0.482pt}}
\multiput(495.17,300.00)(-1.000,-1.000){2}{\rule{0.400pt}{0.241pt}}
\put(493.67,296){\rule{0.400pt}{0.723pt}}
\multiput(494.17,297.50)(-1.000,-1.500){2}{\rule{0.400pt}{0.361pt}}
\put(492.67,294){\rule{0.400pt}{0.482pt}}
\multiput(493.17,295.00)(-1.000,-1.000){2}{\rule{0.400pt}{0.241pt}}
\put(491.67,292){\rule{0.400pt}{0.482pt}}
\multiput(492.17,293.00)(-1.000,-1.000){2}{\rule{0.400pt}{0.241pt}}
\put(490.67,289){\rule{0.400pt}{0.723pt}}
\multiput(491.17,290.50)(-1.000,-1.500){2}{\rule{0.400pt}{0.361pt}}
\put(489.67,287){\rule{0.400pt}{0.482pt}}
\multiput(490.17,288.00)(-1.000,-1.000){2}{\rule{0.400pt}{0.241pt}}
\put(488.67,285){\rule{0.400pt}{0.482pt}}
\multiput(489.17,286.00)(-1.000,-1.000){2}{\rule{0.400pt}{0.241pt}}
\put(487.67,283){\rule{0.400pt}{0.482pt}}
\multiput(488.17,284.00)(-1.000,-1.000){2}{\rule{0.400pt}{0.241pt}}
\put(486.67,280){\rule{0.400pt}{0.723pt}}
\multiput(487.17,281.50)(-1.000,-1.500){2}{\rule{0.400pt}{0.361pt}}
\put(502.0,315.0){\rule[-0.200pt]{0.400pt}{0.482pt}}
\put(485.67,276){\rule{0.400pt}{0.482pt}}
\multiput(486.17,277.00)(-1.000,-1.000){2}{\rule{0.400pt}{0.241pt}}
\put(484.67,273){\rule{0.400pt}{0.723pt}}
\multiput(485.17,274.50)(-1.000,-1.500){2}{\rule{0.400pt}{0.361pt}}
\put(483.67,271){\rule{0.400pt}{0.482pt}}
\multiput(484.17,272.00)(-1.000,-1.000){2}{\rule{0.400pt}{0.241pt}}
\put(482.67,269){\rule{0.400pt}{0.482pt}}
\multiput(483.17,270.00)(-1.000,-1.000){2}{\rule{0.400pt}{0.241pt}}
\put(481.67,266){\rule{0.400pt}{0.723pt}}
\multiput(482.17,267.50)(-1.000,-1.500){2}{\rule{0.400pt}{0.361pt}}
\put(480.67,264){\rule{0.400pt}{0.482pt}}
\multiput(481.17,265.00)(-1.000,-1.000){2}{\rule{0.400pt}{0.241pt}}
\put(479.67,262){\rule{0.400pt}{0.482pt}}
\multiput(480.17,263.00)(-1.000,-1.000){2}{\rule{0.400pt}{0.241pt}}
\put(478.67,260){\rule{0.400pt}{0.482pt}}
\multiput(479.17,261.00)(-1.000,-1.000){2}{\rule{0.400pt}{0.241pt}}
\put(477.67,257){\rule{0.400pt}{0.723pt}}
\multiput(478.17,258.50)(-1.000,-1.500){2}{\rule{0.400pt}{0.361pt}}
\put(476.67,255){\rule{0.400pt}{0.482pt}}
\multiput(477.17,256.00)(-1.000,-1.000){2}{\rule{0.400pt}{0.241pt}}
\put(475.67,253){\rule{0.400pt}{0.482pt}}
\multiput(476.17,254.00)(-1.000,-1.000){2}{\rule{0.400pt}{0.241pt}}
\put(474.67,250){\rule{0.400pt}{0.723pt}}
\multiput(475.17,251.50)(-1.000,-1.500){2}{\rule{0.400pt}{0.361pt}}
\put(473.67,248){\rule{0.400pt}{0.482pt}}
\multiput(474.17,249.00)(-1.000,-1.000){2}{\rule{0.400pt}{0.241pt}}
\put(472.67,246){\rule{0.400pt}{0.482pt}}
\multiput(473.17,247.00)(-1.000,-1.000){2}{\rule{0.400pt}{0.241pt}}
\put(471.67,243){\rule{0.400pt}{0.723pt}}
\multiput(472.17,244.50)(-1.000,-1.500){2}{\rule{0.400pt}{0.361pt}}
\put(487.0,278.0){\rule[-0.200pt]{0.400pt}{0.482pt}}
\put(470.67,239){\rule{0.400pt}{0.482pt}}
\multiput(471.17,240.00)(-1.000,-1.000){2}{\rule{0.400pt}{0.241pt}}
\put(469.67,237){\rule{0.400pt}{0.482pt}}
\multiput(470.17,238.00)(-1.000,-1.000){2}{\rule{0.400pt}{0.241pt}}
\put(468.67,234){\rule{0.400pt}{0.723pt}}
\multiput(469.17,235.50)(-1.000,-1.500){2}{\rule{0.400pt}{0.361pt}}
\put(467.67,232){\rule{0.400pt}{0.482pt}}
\multiput(468.17,233.00)(-1.000,-1.000){2}{\rule{0.400pt}{0.241pt}}
\put(466.67,230){\rule{0.400pt}{0.482pt}}
\multiput(467.17,231.00)(-1.000,-1.000){2}{\rule{0.400pt}{0.241pt}}
\put(465.67,227){\rule{0.400pt}{0.723pt}}
\multiput(466.17,228.50)(-1.000,-1.500){2}{\rule{0.400pt}{0.361pt}}
\put(464.67,225){\rule{0.400pt}{0.482pt}}
\multiput(465.17,226.00)(-1.000,-1.000){2}{\rule{0.400pt}{0.241pt}}
\put(463.67,223){\rule{0.400pt}{0.482pt}}
\multiput(464.17,224.00)(-1.000,-1.000){2}{\rule{0.400pt}{0.241pt}}
\put(462.67,221){\rule{0.400pt}{0.482pt}}
\multiput(463.17,222.00)(-1.000,-1.000){2}{\rule{0.400pt}{0.241pt}}
\put(461.67,218){\rule{0.400pt}{0.723pt}}
\multiput(462.17,219.50)(-1.000,-1.500){2}{\rule{0.400pt}{0.361pt}}
\put(460.67,216){\rule{0.400pt}{0.482pt}}
\multiput(461.17,217.00)(-1.000,-1.000){2}{\rule{0.400pt}{0.241pt}}
\put(459.67,214){\rule{0.400pt}{0.482pt}}
\multiput(460.17,215.00)(-1.000,-1.000){2}{\rule{0.400pt}{0.241pt}}
\put(458.67,211){\rule{0.400pt}{0.723pt}}
\multiput(459.17,212.50)(-1.000,-1.500){2}{\rule{0.400pt}{0.361pt}}
\put(457.67,209){\rule{0.400pt}{0.482pt}}
\multiput(458.17,210.00)(-1.000,-1.000){2}{\rule{0.400pt}{0.241pt}}
\put(472.0,241.0){\rule[-0.200pt]{0.400pt}{0.482pt}}
\put(456.67,204){\rule{0.400pt}{0.723pt}}
\multiput(457.17,205.50)(-1.000,-1.500){2}{\rule{0.400pt}{0.361pt}}
\put(455.67,202){\rule{0.400pt}{0.482pt}}
\multiput(456.17,203.00)(-1.000,-1.000){2}{\rule{0.400pt}{0.241pt}}
\put(454.67,200){\rule{0.400pt}{0.482pt}}
\multiput(455.17,201.00)(-1.000,-1.000){2}{\rule{0.400pt}{0.241pt}}
\put(453.67,198){\rule{0.400pt}{0.482pt}}
\multiput(454.17,199.00)(-1.000,-1.000){2}{\rule{0.400pt}{0.241pt}}
\put(452.67,195){\rule{0.400pt}{0.723pt}}
\multiput(453.17,196.50)(-1.000,-1.500){2}{\rule{0.400pt}{0.361pt}}
\put(451.67,193){\rule{0.400pt}{0.482pt}}
\multiput(452.17,194.00)(-1.000,-1.000){2}{\rule{0.400pt}{0.241pt}}
\put(450.67,191){\rule{0.400pt}{0.482pt}}
\multiput(451.17,192.00)(-1.000,-1.000){2}{\rule{0.400pt}{0.241pt}}
\put(449.67,188){\rule{0.400pt}{0.723pt}}
\multiput(450.17,189.50)(-1.000,-1.500){2}{\rule{0.400pt}{0.361pt}}
\put(448.67,186){\rule{0.400pt}{0.482pt}}
\multiput(449.17,187.00)(-1.000,-1.000){2}{\rule{0.400pt}{0.241pt}}
\put(447.67,184){\rule{0.400pt}{0.482pt}}
\multiput(448.17,185.00)(-1.000,-1.000){2}{\rule{0.400pt}{0.241pt}}
\put(446.67,181){\rule{0.400pt}{0.723pt}}
\multiput(447.17,182.50)(-1.000,-1.500){2}{\rule{0.400pt}{0.361pt}}
\put(445.67,179){\rule{0.400pt}{0.482pt}}
\multiput(446.17,180.00)(-1.000,-1.000){2}{\rule{0.400pt}{0.241pt}}
\put(444.67,177){\rule{0.400pt}{0.482pt}}
\multiput(445.17,178.00)(-1.000,-1.000){2}{\rule{0.400pt}{0.241pt}}
\put(443.67,175){\rule{0.400pt}{0.482pt}}
\multiput(444.17,176.00)(-1.000,-1.000){2}{\rule{0.400pt}{0.241pt}}
\put(442.67,172){\rule{0.400pt}{0.723pt}}
\multiput(443.17,173.50)(-1.000,-1.500){2}{\rule{0.400pt}{0.361pt}}
\put(458.0,207.0){\rule[-0.200pt]{0.400pt}{0.482pt}}
\put(441.67,168){\rule{0.400pt}{0.482pt}}
\multiput(442.17,169.00)(-1.000,-1.000){2}{\rule{0.400pt}{0.241pt}}
\put(440.67,165){\rule{0.400pt}{0.723pt}}
\multiput(441.17,166.50)(-1.000,-1.500){2}{\rule{0.400pt}{0.361pt}}
\put(439.67,163){\rule{0.400pt}{0.482pt}}
\multiput(440.17,164.00)(-1.000,-1.000){2}{\rule{0.400pt}{0.241pt}}
\put(438.67,161){\rule{0.400pt}{0.482pt}}
\multiput(439.17,162.00)(-1.000,-1.000){2}{\rule{0.400pt}{0.241pt}}
\put(437.67,158){\rule{0.400pt}{0.723pt}}
\multiput(438.17,159.50)(-1.000,-1.500){2}{\rule{0.400pt}{0.361pt}}
\put(436.67,156){\rule{0.400pt}{0.482pt}}
\multiput(437.17,157.00)(-1.000,-1.000){2}{\rule{0.400pt}{0.241pt}}
\put(435.67,154){\rule{0.400pt}{0.482pt}}
\multiput(436.17,155.00)(-1.000,-1.000){2}{\rule{0.400pt}{0.241pt}}
\put(434.67,152){\rule{0.400pt}{0.482pt}}
\multiput(435.17,153.00)(-1.000,-1.000){2}{\rule{0.400pt}{0.241pt}}
\put(433.67,149){\rule{0.400pt}{0.723pt}}
\multiput(434.17,150.50)(-1.000,-1.500){2}{\rule{0.400pt}{0.361pt}}
\put(432.67,147){\rule{0.400pt}{0.482pt}}
\multiput(433.17,148.00)(-1.000,-1.000){2}{\rule{0.400pt}{0.241pt}}
\put(431.67,145){\rule{0.400pt}{0.482pt}}
\multiput(432.17,146.00)(-1.000,-1.000){2}{\rule{0.400pt}{0.241pt}}
\put(430.67,142){\rule{0.400pt}{0.723pt}}
\multiput(431.17,143.50)(-1.000,-1.500){2}{\rule{0.400pt}{0.361pt}}
\put(429.67,140){\rule{0.400pt}{0.482pt}}
\multiput(430.17,141.00)(-1.000,-1.000){2}{\rule{0.400pt}{0.241pt}}
\put(428.67,138){\rule{0.400pt}{0.482pt}}
\multiput(429.17,139.00)(-1.000,-1.000){2}{\rule{0.400pt}{0.241pt}}
\put(427.67,135){\rule{0.400pt}{0.723pt}}
\multiput(428.17,136.50)(-1.000,-1.500){2}{\rule{0.400pt}{0.361pt}}
\put(443.0,170.0){\rule[-0.200pt]{0.400pt}{0.482pt}}
\put(426.67,131){\rule{0.400pt}{0.482pt}}
\multiput(427.17,132.00)(-1.000,-1.000){2}{\rule{0.400pt}{0.241pt}}
\put(425.67,129){\rule{0.400pt}{0.482pt}}
\multiput(426.17,130.00)(-1.000,-1.000){2}{\rule{0.400pt}{0.241pt}}
\put(424.67,126){\rule{0.400pt}{0.723pt}}
\multiput(425.17,127.50)(-1.000,-1.500){2}{\rule{0.400pt}{0.361pt}}
\put(423.67,124){\rule{0.400pt}{0.482pt}}
\multiput(424.17,125.00)(-1.000,-1.000){2}{\rule{0.400pt}{0.241pt}}
\put(422.67,122){\rule{0.400pt}{0.482pt}}
\multiput(423.17,123.00)(-1.000,-1.000){2}{\rule{0.400pt}{0.241pt}}
\put(421.67,119){\rule{0.400pt}{0.723pt}}
\multiput(422.17,120.50)(-1.000,-1.500){2}{\rule{0.400pt}{0.361pt}}
\put(420.67,117){\rule{0.400pt}{0.482pt}}
\multiput(421.17,118.00)(-1.000,-1.000){2}{\rule{0.400pt}{0.241pt}}
\put(419.67,115){\rule{0.400pt}{0.482pt}}
\multiput(420.17,116.00)(-1.000,-1.000){2}{\rule{0.400pt}{0.241pt}}
\put(418.67,112){\rule{0.400pt}{0.723pt}}
\multiput(419.17,113.50)(-1.000,-1.500){2}{\rule{0.400pt}{0.361pt}}
\put(417.67,110){\rule{0.400pt}{0.482pt}}
\multiput(418.17,111.00)(-1.000,-1.000){2}{\rule{0.400pt}{0.241pt}}
\put(416.67,108){\rule{0.400pt}{0.482pt}}
\multiput(417.17,109.00)(-1.000,-1.000){2}{\rule{0.400pt}{0.241pt}}
\put(415.67,106){\rule{0.400pt}{0.482pt}}
\multiput(416.17,107.00)(-1.000,-1.000){2}{\rule{0.400pt}{0.241pt}}
\put(414.67,103){\rule{0.400pt}{0.723pt}}
\multiput(415.17,104.50)(-1.000,-1.500){2}{\rule{0.400pt}{0.361pt}}
\put(413.67,103){\rule{0.400pt}{0.723pt}}
\multiput(414.17,103.00)(-1.000,1.500){2}{\rule{0.400pt}{0.361pt}}
\put(428.0,133.0){\rule[-0.200pt]{0.400pt}{0.482pt}}
\put(412.67,108){\rule{0.400pt}{0.482pt}}
\multiput(413.17,108.00)(-1.000,1.000){2}{\rule{0.400pt}{0.241pt}}
\put(411.67,110){\rule{0.400pt}{0.482pt}}
\multiput(412.17,110.00)(-1.000,1.000){2}{\rule{0.400pt}{0.241pt}}
\put(410.67,112){\rule{0.400pt}{0.723pt}}
\multiput(411.17,112.00)(-1.000,1.500){2}{\rule{0.400pt}{0.361pt}}
\put(409.67,115){\rule{0.400pt}{0.482pt}}
\multiput(410.17,115.00)(-1.000,1.000){2}{\rule{0.400pt}{0.241pt}}
\put(408.67,117){\rule{0.400pt}{0.482pt}}
\multiput(409.17,117.00)(-1.000,1.000){2}{\rule{0.400pt}{0.241pt}}
\put(407.67,119){\rule{0.400pt}{0.723pt}}
\multiput(408.17,119.00)(-1.000,1.500){2}{\rule{0.400pt}{0.361pt}}
\put(406.67,122){\rule{0.400pt}{0.482pt}}
\multiput(407.17,122.00)(-1.000,1.000){2}{\rule{0.400pt}{0.241pt}}
\put(405.67,124){\rule{0.400pt}{0.482pt}}
\multiput(406.17,124.00)(-1.000,1.000){2}{\rule{0.400pt}{0.241pt}}
\put(404.67,126){\rule{0.400pt}{0.723pt}}
\multiput(405.17,126.00)(-1.000,1.500){2}{\rule{0.400pt}{0.361pt}}
\put(403.67,129){\rule{0.400pt}{0.482pt}}
\multiput(404.17,129.00)(-1.000,1.000){2}{\rule{0.400pt}{0.241pt}}
\put(402.67,131){\rule{0.400pt}{0.482pt}}
\multiput(403.17,131.00)(-1.000,1.000){2}{\rule{0.400pt}{0.241pt}}
\put(401.67,133){\rule{0.400pt}{0.482pt}}
\multiput(402.17,133.00)(-1.000,1.000){2}{\rule{0.400pt}{0.241pt}}
\put(400.67,135){\rule{0.400pt}{0.723pt}}
\multiput(401.17,135.00)(-1.000,1.500){2}{\rule{0.400pt}{0.361pt}}
\put(399.67,138){\rule{0.400pt}{0.482pt}}
\multiput(400.17,138.00)(-1.000,1.000){2}{\rule{0.400pt}{0.241pt}}
\put(398.67,140){\rule{0.400pt}{0.482pt}}
\multiput(399.17,140.00)(-1.000,1.000){2}{\rule{0.400pt}{0.241pt}}
\put(414.0,106.0){\rule[-0.200pt]{0.400pt}{0.482pt}}
\put(397.67,145){\rule{0.400pt}{0.482pt}}
\multiput(398.17,145.00)(-1.000,1.000){2}{\rule{0.400pt}{0.241pt}}
\put(396.67,147){\rule{0.400pt}{0.482pt}}
\multiput(397.17,147.00)(-1.000,1.000){2}{\rule{0.400pt}{0.241pt}}
\put(395.67,149){\rule{0.400pt}{0.723pt}}
\multiput(396.17,149.00)(-1.000,1.500){2}{\rule{0.400pt}{0.361pt}}
\put(394.67,152){\rule{0.400pt}{0.482pt}}
\multiput(395.17,152.00)(-1.000,1.000){2}{\rule{0.400pt}{0.241pt}}
\put(393.67,154){\rule{0.400pt}{0.482pt}}
\multiput(394.17,154.00)(-1.000,1.000){2}{\rule{0.400pt}{0.241pt}}
\put(392.67,156){\rule{0.400pt}{0.482pt}}
\multiput(393.17,156.00)(-1.000,1.000){2}{\rule{0.400pt}{0.241pt}}
\put(391.67,158){\rule{0.400pt}{0.723pt}}
\multiput(392.17,158.00)(-1.000,1.500){2}{\rule{0.400pt}{0.361pt}}
\put(390.67,161){\rule{0.400pt}{0.482pt}}
\multiput(391.17,161.00)(-1.000,1.000){2}{\rule{0.400pt}{0.241pt}}
\put(389.67,163){\rule{0.400pt}{0.482pt}}
\multiput(390.17,163.00)(-1.000,1.000){2}{\rule{0.400pt}{0.241pt}}
\put(388.67,165){\rule{0.400pt}{0.723pt}}
\multiput(389.17,165.00)(-1.000,1.500){2}{\rule{0.400pt}{0.361pt}}
\put(387.67,168){\rule{0.400pt}{0.482pt}}
\multiput(388.17,168.00)(-1.000,1.000){2}{\rule{0.400pt}{0.241pt}}
\put(386.67,170){\rule{0.400pt}{0.482pt}}
\multiput(387.17,170.00)(-1.000,1.000){2}{\rule{0.400pt}{0.241pt}}
\put(385.67,172){\rule{0.400pt}{0.723pt}}
\multiput(386.17,172.00)(-1.000,1.500){2}{\rule{0.400pt}{0.361pt}}
\put(384.67,175){\rule{0.400pt}{0.482pt}}
\multiput(385.17,175.00)(-1.000,1.000){2}{\rule{0.400pt}{0.241pt}}
\put(383.67,177){\rule{0.400pt}{0.482pt}}
\multiput(384.17,177.00)(-1.000,1.000){2}{\rule{0.400pt}{0.241pt}}
\put(399.0,142.0){\rule[-0.200pt]{0.400pt}{0.723pt}}
\put(382.67,181){\rule{0.400pt}{0.723pt}}
\multiput(383.17,181.00)(-1.000,1.500){2}{\rule{0.400pt}{0.361pt}}
\put(381.67,184){\rule{0.400pt}{0.482pt}}
\multiput(382.17,184.00)(-1.000,1.000){2}{\rule{0.400pt}{0.241pt}}
\put(380.67,186){\rule{0.400pt}{0.482pt}}
\multiput(381.17,186.00)(-1.000,1.000){2}{\rule{0.400pt}{0.241pt}}
\put(379.67,188){\rule{0.400pt}{0.723pt}}
\multiput(380.17,188.00)(-1.000,1.500){2}{\rule{0.400pt}{0.361pt}}
\put(378.67,191){\rule{0.400pt}{0.482pt}}
\multiput(379.17,191.00)(-1.000,1.000){2}{\rule{0.400pt}{0.241pt}}
\put(377.67,193){\rule{0.400pt}{0.482pt}}
\multiput(378.17,193.00)(-1.000,1.000){2}{\rule{0.400pt}{0.241pt}}
\put(376.67,195){\rule{0.400pt}{0.723pt}}
\multiput(377.17,195.00)(-1.000,1.500){2}{\rule{0.400pt}{0.361pt}}
\put(375.67,198){\rule{0.400pt}{0.482pt}}
\multiput(376.17,198.00)(-1.000,1.000){2}{\rule{0.400pt}{0.241pt}}
\put(374.67,200){\rule{0.400pt}{0.482pt}}
\multiput(375.17,200.00)(-1.000,1.000){2}{\rule{0.400pt}{0.241pt}}
\put(373.67,202){\rule{0.400pt}{0.482pt}}
\multiput(374.17,202.00)(-1.000,1.000){2}{\rule{0.400pt}{0.241pt}}
\put(372.67,204){\rule{0.400pt}{0.723pt}}
\multiput(373.17,204.00)(-1.000,1.500){2}{\rule{0.400pt}{0.361pt}}
\put(371.67,207){\rule{0.400pt}{0.482pt}}
\multiput(372.17,207.00)(-1.000,1.000){2}{\rule{0.400pt}{0.241pt}}
\put(370.67,209){\rule{0.400pt}{0.482pt}}
\multiput(371.17,209.00)(-1.000,1.000){2}{\rule{0.400pt}{0.241pt}}
\put(369.67,211){\rule{0.400pt}{0.723pt}}
\multiput(370.17,211.00)(-1.000,1.500){2}{\rule{0.400pt}{0.361pt}}
\put(368.67,214){\rule{0.400pt}{0.482pt}}
\multiput(369.17,214.00)(-1.000,1.000){2}{\rule{0.400pt}{0.241pt}}
\put(384.0,179.0){\rule[-0.200pt]{0.400pt}{0.482pt}}
\put(367.67,218){\rule{0.400pt}{0.723pt}}
\multiput(368.17,218.00)(-1.000,1.500){2}{\rule{0.400pt}{0.361pt}}
\put(366.67,221){\rule{0.400pt}{0.482pt}}
\multiput(367.17,221.00)(-1.000,1.000){2}{\rule{0.400pt}{0.241pt}}
\put(365.67,223){\rule{0.400pt}{0.482pt}}
\multiput(366.17,223.00)(-1.000,1.000){2}{\rule{0.400pt}{0.241pt}}
\put(364.67,225){\rule{0.400pt}{0.482pt}}
\multiput(365.17,225.00)(-1.000,1.000){2}{\rule{0.400pt}{0.241pt}}
\put(363.67,227){\rule{0.400pt}{0.723pt}}
\multiput(364.17,227.00)(-1.000,1.500){2}{\rule{0.400pt}{0.361pt}}
\put(362.67,230){\rule{0.400pt}{0.482pt}}
\multiput(363.17,230.00)(-1.000,1.000){2}{\rule{0.400pt}{0.241pt}}
\put(361.67,232){\rule{0.400pt}{0.482pt}}
\multiput(362.17,232.00)(-1.000,1.000){2}{\rule{0.400pt}{0.241pt}}
\put(360.67,234){\rule{0.400pt}{0.723pt}}
\multiput(361.17,234.00)(-1.000,1.500){2}{\rule{0.400pt}{0.361pt}}
\put(359.67,237){\rule{0.400pt}{0.482pt}}
\multiput(360.17,237.00)(-1.000,1.000){2}{\rule{0.400pt}{0.241pt}}
\put(358.67,239){\rule{0.400pt}{0.482pt}}
\multiput(359.17,239.00)(-1.000,1.000){2}{\rule{0.400pt}{0.241pt}}
\put(357.67,241){\rule{0.400pt}{0.482pt}}
\multiput(358.17,241.00)(-1.000,1.000){2}{\rule{0.400pt}{0.241pt}}
\put(356.67,243){\rule{0.400pt}{0.723pt}}
\multiput(357.17,243.00)(-1.000,1.500){2}{\rule{0.400pt}{0.361pt}}
\put(355.67,246){\rule{0.400pt}{0.482pt}}
\multiput(356.17,246.00)(-1.000,1.000){2}{\rule{0.400pt}{0.241pt}}
\put(354.67,248){\rule{0.400pt}{0.482pt}}
\multiput(355.17,248.00)(-1.000,1.000){2}{\rule{0.400pt}{0.241pt}}
\put(369.0,216.0){\rule[-0.200pt]{0.400pt}{0.482pt}}
\put(353.67,253){\rule{0.400pt}{0.482pt}}
\multiput(354.17,253.00)(-1.000,1.000){2}{\rule{0.400pt}{0.241pt}}
\put(352.67,255){\rule{0.400pt}{0.482pt}}
\multiput(353.17,255.00)(-1.000,1.000){2}{\rule{0.400pt}{0.241pt}}
\put(351.67,257){\rule{0.400pt}{0.723pt}}
\multiput(352.17,257.00)(-1.000,1.500){2}{\rule{0.400pt}{0.361pt}}
\put(350.67,260){\rule{0.400pt}{0.482pt}}
\multiput(351.17,260.00)(-1.000,1.000){2}{\rule{0.400pt}{0.241pt}}
\put(349.67,262){\rule{0.400pt}{0.482pt}}
\multiput(350.17,262.00)(-1.000,1.000){2}{\rule{0.400pt}{0.241pt}}
\put(348.67,264){\rule{0.400pt}{0.482pt}}
\multiput(349.17,264.00)(-1.000,1.000){2}{\rule{0.400pt}{0.241pt}}
\put(347.67,266){\rule{0.400pt}{0.723pt}}
\multiput(348.17,266.00)(-1.000,1.500){2}{\rule{0.400pt}{0.361pt}}
\put(346.67,269){\rule{0.400pt}{0.482pt}}
\multiput(347.17,269.00)(-1.000,1.000){2}{\rule{0.400pt}{0.241pt}}
\put(345.67,271){\rule{0.400pt}{0.482pt}}
\multiput(346.17,271.00)(-1.000,1.000){2}{\rule{0.400pt}{0.241pt}}
\put(344.67,273){\rule{0.400pt}{0.723pt}}
\multiput(345.17,273.00)(-1.000,1.500){2}{\rule{0.400pt}{0.361pt}}
\put(343.67,276){\rule{0.400pt}{0.482pt}}
\multiput(344.17,276.00)(-1.000,1.000){2}{\rule{0.400pt}{0.241pt}}
\put(342.67,278){\rule{0.400pt}{0.482pt}}
\multiput(343.17,278.00)(-1.000,1.000){2}{\rule{0.400pt}{0.241pt}}
\put(341.67,280){\rule{0.400pt}{0.723pt}}
\multiput(342.17,280.00)(-1.000,1.500){2}{\rule{0.400pt}{0.361pt}}
\put(340.67,283){\rule{0.400pt}{0.482pt}}
\multiput(341.17,283.00)(-1.000,1.000){2}{\rule{0.400pt}{0.241pt}}
\put(339.67,285){\rule{0.400pt}{0.482pt}}
\multiput(340.17,285.00)(-1.000,1.000){2}{\rule{0.400pt}{0.241pt}}
\put(355.0,250.0){\rule[-0.200pt]{0.400pt}{0.723pt}}
\put(338.67,289){\rule{0.400pt}{0.723pt}}
\multiput(339.17,289.00)(-1.000,1.500){2}{\rule{0.400pt}{0.361pt}}
\put(337.67,292){\rule{0.400pt}{0.482pt}}
\multiput(338.17,292.00)(-1.000,1.000){2}{\rule{0.400pt}{0.241pt}}
\put(336.67,294){\rule{0.400pt}{0.482pt}}
\multiput(337.17,294.00)(-1.000,1.000){2}{\rule{0.400pt}{0.241pt}}
\put(335.67,296){\rule{0.400pt}{0.723pt}}
\multiput(336.17,296.00)(-1.000,1.500){2}{\rule{0.400pt}{0.361pt}}
\put(334.67,299){\rule{0.400pt}{0.482pt}}
\multiput(335.17,299.00)(-1.000,1.000){2}{\rule{0.400pt}{0.241pt}}
\put(333.67,301){\rule{0.400pt}{0.482pt}}
\multiput(334.17,301.00)(-1.000,1.000){2}{\rule{0.400pt}{0.241pt}}
\put(332.67,303){\rule{0.400pt}{0.723pt}}
\multiput(333.17,303.00)(-1.000,1.500){2}{\rule{0.400pt}{0.361pt}}
\put(331.67,306){\rule{0.400pt}{0.482pt}}
\multiput(332.17,306.00)(-1.000,1.000){2}{\rule{0.400pt}{0.241pt}}
\put(330.67,308){\rule{0.400pt}{0.482pt}}
\multiput(331.17,308.00)(-1.000,1.000){2}{\rule{0.400pt}{0.241pt}}
\put(329.67,310){\rule{0.400pt}{0.482pt}}
\multiput(330.17,310.00)(-1.000,1.000){2}{\rule{0.400pt}{0.241pt}}
\put(328.67,312){\rule{0.400pt}{0.723pt}}
\multiput(329.17,312.00)(-1.000,1.500){2}{\rule{0.400pt}{0.361pt}}
\put(327.67,315){\rule{0.400pt}{0.482pt}}
\multiput(328.17,315.00)(-1.000,1.000){2}{\rule{0.400pt}{0.241pt}}
\put(326.67,317){\rule{0.400pt}{0.482pt}}
\multiput(327.17,317.00)(-1.000,1.000){2}{\rule{0.400pt}{0.241pt}}
\put(325.67,319){\rule{0.400pt}{0.723pt}}
\multiput(326.17,319.00)(-1.000,1.500){2}{\rule{0.400pt}{0.361pt}}
\put(324.67,322){\rule{0.400pt}{0.482pt}}
\multiput(325.17,322.00)(-1.000,1.000){2}{\rule{0.400pt}{0.241pt}}
\put(340.0,287.0){\rule[-0.200pt]{0.400pt}{0.482pt}}
\put(323.67,326){\rule{0.400pt}{0.723pt}}
\multiput(324.17,326.00)(-1.000,1.500){2}{\rule{0.400pt}{0.361pt}}
\put(322.67,329){\rule{0.400pt}{0.482pt}}
\multiput(323.17,329.00)(-1.000,1.000){2}{\rule{0.400pt}{0.241pt}}
\put(321.67,331){\rule{0.400pt}{0.482pt}}
\multiput(322.17,331.00)(-1.000,1.000){2}{\rule{0.400pt}{0.241pt}}
\put(320.67,333){\rule{0.400pt}{0.482pt}}
\multiput(321.17,333.00)(-1.000,1.000){2}{\rule{0.400pt}{0.241pt}}
\put(319.67,335){\rule{0.400pt}{0.723pt}}
\multiput(320.17,335.00)(-1.000,1.500){2}{\rule{0.400pt}{0.361pt}}
\put(320,336.67){\rule{0.482pt}{0.400pt}}
\multiput(320.00,337.17)(1.000,-1.000){2}{\rule{0.241pt}{0.400pt}}
\put(322,335.67){\rule{0.482pt}{0.400pt}}
\multiput(322.00,336.17)(1.000,-1.000){2}{\rule{0.241pt}{0.400pt}}
\put(324,334.17){\rule{0.482pt}{0.400pt}}
\multiput(324.00,335.17)(1.000,-2.000){2}{\rule{0.241pt}{0.400pt}}
\put(326,332.67){\rule{0.482pt}{0.400pt}}
\multiput(326.00,333.17)(1.000,-1.000){2}{\rule{0.241pt}{0.400pt}}
\put(328,331.67){\rule{0.482pt}{0.400pt}}
\multiput(328.00,332.17)(1.000,-1.000){2}{\rule{0.241pt}{0.400pt}}
\put(330,330.67){\rule{0.723pt}{0.400pt}}
\multiput(330.00,331.17)(1.500,-1.000){2}{\rule{0.361pt}{0.400pt}}
\put(333,329.67){\rule{0.482pt}{0.400pt}}
\multiput(333.00,330.17)(1.000,-1.000){2}{\rule{0.241pt}{0.400pt}}
\put(335,328.67){\rule{0.482pt}{0.400pt}}
\multiput(335.00,329.17)(1.000,-1.000){2}{\rule{0.241pt}{0.400pt}}
\put(337,327.67){\rule{0.482pt}{0.400pt}}
\multiput(337.00,328.17)(1.000,-1.000){2}{\rule{0.241pt}{0.400pt}}
\put(339,326.67){\rule{0.482pt}{0.400pt}}
\multiput(339.00,327.17)(1.000,-1.000){2}{\rule{0.241pt}{0.400pt}}
\put(341,325.67){\rule{0.482pt}{0.400pt}}
\multiput(341.00,326.17)(1.000,-1.000){2}{\rule{0.241pt}{0.400pt}}
\put(343,324.67){\rule{0.482pt}{0.400pt}}
\multiput(343.00,325.17)(1.000,-1.000){2}{\rule{0.241pt}{0.400pt}}
\put(345,323.67){\rule{0.482pt}{0.400pt}}
\multiput(345.00,324.17)(1.000,-1.000){2}{\rule{0.241pt}{0.400pt}}
\put(347,322.17){\rule{0.482pt}{0.400pt}}
\multiput(347.00,323.17)(1.000,-2.000){2}{\rule{0.241pt}{0.400pt}}
\put(349,320.67){\rule{0.723pt}{0.400pt}}
\multiput(349.00,321.17)(1.500,-1.000){2}{\rule{0.361pt}{0.400pt}}
\put(352,319.67){\rule{0.482pt}{0.400pt}}
\multiput(352.00,320.17)(1.000,-1.000){2}{\rule{0.241pt}{0.400pt}}
\put(354,318.67){\rule{0.482pt}{0.400pt}}
\multiput(354.00,319.17)(1.000,-1.000){2}{\rule{0.241pt}{0.400pt}}
\put(356,317.67){\rule{0.482pt}{0.400pt}}
\multiput(356.00,318.17)(1.000,-1.000){2}{\rule{0.241pt}{0.400pt}}
\put(358,316.67){\rule{0.482pt}{0.400pt}}
\multiput(358.00,317.17)(1.000,-1.000){2}{\rule{0.241pt}{0.400pt}}
\put(360,315.67){\rule{0.482pt}{0.400pt}}
\multiput(360.00,316.17)(1.000,-1.000){2}{\rule{0.241pt}{0.400pt}}
\put(362,314.67){\rule{0.482pt}{0.400pt}}
\multiput(362.00,315.17)(1.000,-1.000){2}{\rule{0.241pt}{0.400pt}}
\put(364,313.67){\rule{0.482pt}{0.400pt}}
\multiput(364.00,314.17)(1.000,-1.000){2}{\rule{0.241pt}{0.400pt}}
\put(366,312.67){\rule{0.482pt}{0.400pt}}
\multiput(366.00,313.17)(1.000,-1.000){2}{\rule{0.241pt}{0.400pt}}
\put(368,311.67){\rule{0.723pt}{0.400pt}}
\multiput(368.00,312.17)(1.500,-1.000){2}{\rule{0.361pt}{0.400pt}}
\put(371,310.17){\rule{0.482pt}{0.400pt}}
\multiput(371.00,311.17)(1.000,-2.000){2}{\rule{0.241pt}{0.400pt}}
\put(373,308.67){\rule{0.482pt}{0.400pt}}
\multiput(373.00,309.17)(1.000,-1.000){2}{\rule{0.241pt}{0.400pt}}
\put(375,307.67){\rule{0.482pt}{0.400pt}}
\multiput(375.00,308.17)(1.000,-1.000){2}{\rule{0.241pt}{0.400pt}}
\put(377,306.67){\rule{0.482pt}{0.400pt}}
\multiput(377.00,307.17)(1.000,-1.000){2}{\rule{0.241pt}{0.400pt}}
\put(379,305.67){\rule{0.482pt}{0.400pt}}
\multiput(379.00,306.17)(1.000,-1.000){2}{\rule{0.241pt}{0.400pt}}
\put(381,304.67){\rule{0.482pt}{0.400pt}}
\multiput(381.00,305.17)(1.000,-1.000){2}{\rule{0.241pt}{0.400pt}}
\put(383,303.67){\rule{0.482pt}{0.400pt}}
\multiput(383.00,304.17)(1.000,-1.000){2}{\rule{0.241pt}{0.400pt}}
\put(385,302.67){\rule{0.482pt}{0.400pt}}
\multiput(385.00,303.17)(1.000,-1.000){2}{\rule{0.241pt}{0.400pt}}
\put(387,301.67){\rule{0.723pt}{0.400pt}}
\multiput(387.00,302.17)(1.500,-1.000){2}{\rule{0.361pt}{0.400pt}}
\put(390,300.67){\rule{0.482pt}{0.400pt}}
\multiput(390.00,301.17)(1.000,-1.000){2}{\rule{0.241pt}{0.400pt}}
\put(392,299.17){\rule{0.482pt}{0.400pt}}
\multiput(392.00,300.17)(1.000,-2.000){2}{\rule{0.241pt}{0.400pt}}
\put(394,297.67){\rule{0.482pt}{0.400pt}}
\multiput(394.00,298.17)(1.000,-1.000){2}{\rule{0.241pt}{0.400pt}}
\put(396,296.67){\rule{0.482pt}{0.400pt}}
\multiput(396.00,297.17)(1.000,-1.000){2}{\rule{0.241pt}{0.400pt}}
\put(398,295.67){\rule{0.482pt}{0.400pt}}
\multiput(398.00,296.17)(1.000,-1.000){2}{\rule{0.241pt}{0.400pt}}
\put(400,294.67){\rule{0.482pt}{0.400pt}}
\multiput(400.00,295.17)(1.000,-1.000){2}{\rule{0.241pt}{0.400pt}}
\put(402,293.67){\rule{0.482pt}{0.400pt}}
\multiput(402.00,294.17)(1.000,-1.000){2}{\rule{0.241pt}{0.400pt}}
\put(404,292.67){\rule{0.482pt}{0.400pt}}
\multiput(404.00,293.17)(1.000,-1.000){2}{\rule{0.241pt}{0.400pt}}
\put(406,291.67){\rule{0.482pt}{0.400pt}}
\multiput(406.00,292.17)(1.000,-1.000){2}{\rule{0.241pt}{0.400pt}}
\put(408,290.67){\rule{0.723pt}{0.400pt}}
\multiput(408.00,291.17)(1.500,-1.000){2}{\rule{0.361pt}{0.400pt}}
\put(411,289.67){\rule{0.482pt}{0.400pt}}
\multiput(411.00,290.17)(1.000,-1.000){2}{\rule{0.241pt}{0.400pt}}
\put(413,288.67){\rule{0.482pt}{0.400pt}}
\multiput(413.00,289.17)(1.000,-1.000){2}{\rule{0.241pt}{0.400pt}}
\put(415,287.17){\rule{0.482pt}{0.400pt}}
\multiput(415.00,288.17)(1.000,-2.000){2}{\rule{0.241pt}{0.400pt}}
\put(417,285.67){\rule{0.482pt}{0.400pt}}
\multiput(417.00,286.17)(1.000,-1.000){2}{\rule{0.241pt}{0.400pt}}
\put(419,284.67){\rule{0.482pt}{0.400pt}}
\multiput(419.00,285.17)(1.000,-1.000){2}{\rule{0.241pt}{0.400pt}}
\put(421,283.67){\rule{0.482pt}{0.400pt}}
\multiput(421.00,284.17)(1.000,-1.000){2}{\rule{0.241pt}{0.400pt}}
\put(423,282.67){\rule{0.482pt}{0.400pt}}
\multiput(423.00,283.17)(1.000,-1.000){2}{\rule{0.241pt}{0.400pt}}
\put(425,281.67){\rule{0.482pt}{0.400pt}}
\multiput(425.00,282.17)(1.000,-1.000){2}{\rule{0.241pt}{0.400pt}}
\put(427,280.67){\rule{0.723pt}{0.400pt}}
\multiput(427.00,281.17)(1.500,-1.000){2}{\rule{0.361pt}{0.400pt}}
\put(430,279.67){\rule{0.482pt}{0.400pt}}
\multiput(430.00,280.17)(1.000,-1.000){2}{\rule{0.241pt}{0.400pt}}
\put(432,278.67){\rule{0.482pt}{0.400pt}}
\multiput(432.00,279.17)(1.000,-1.000){2}{\rule{0.241pt}{0.400pt}}
\put(434,277.67){\rule{0.482pt}{0.400pt}}
\multiput(434.00,278.17)(1.000,-1.000){2}{\rule{0.241pt}{0.400pt}}
\put(436,276.67){\rule{0.482pt}{0.400pt}}
\multiput(436.00,277.17)(1.000,-1.000){2}{\rule{0.241pt}{0.400pt}}
\put(438,275.17){\rule{0.482pt}{0.400pt}}
\multiput(438.00,276.17)(1.000,-2.000){2}{\rule{0.241pt}{0.400pt}}
\put(440,273.67){\rule{0.482pt}{0.400pt}}
\multiput(440.00,274.17)(1.000,-1.000){2}{\rule{0.241pt}{0.400pt}}
\put(442,272.67){\rule{0.482pt}{0.400pt}}
\multiput(442.00,273.17)(1.000,-1.000){2}{\rule{0.241pt}{0.400pt}}
\put(444,271.67){\rule{0.482pt}{0.400pt}}
\multiput(444.00,272.17)(1.000,-1.000){2}{\rule{0.241pt}{0.400pt}}
\put(446,270.67){\rule{0.723pt}{0.400pt}}
\multiput(446.00,271.17)(1.500,-1.000){2}{\rule{0.361pt}{0.400pt}}
\put(449,269.67){\rule{0.482pt}{0.400pt}}
\multiput(449.00,270.17)(1.000,-1.000){2}{\rule{0.241pt}{0.400pt}}
\put(451,268.67){\rule{0.482pt}{0.400pt}}
\multiput(451.00,269.17)(1.000,-1.000){2}{\rule{0.241pt}{0.400pt}}
\put(453,267.67){\rule{0.482pt}{0.400pt}}
\multiput(453.00,268.17)(1.000,-1.000){2}{\rule{0.241pt}{0.400pt}}
\put(455,266.67){\rule{0.482pt}{0.400pt}}
\multiput(455.00,267.17)(1.000,-1.000){2}{\rule{0.241pt}{0.400pt}}
\put(457,265.67){\rule{0.482pt}{0.400pt}}
\multiput(457.00,266.17)(1.000,-1.000){2}{\rule{0.241pt}{0.400pt}}
\put(459,264.67){\rule{0.482pt}{0.400pt}}
\multiput(459.00,265.17)(1.000,-1.000){2}{\rule{0.241pt}{0.400pt}}
\put(461,263.17){\rule{0.482pt}{0.400pt}}
\multiput(461.00,264.17)(1.000,-2.000){2}{\rule{0.241pt}{0.400pt}}
\put(463,261.67){\rule{0.482pt}{0.400pt}}
\multiput(463.00,262.17)(1.000,-1.000){2}{\rule{0.241pt}{0.400pt}}
\put(465,260.67){\rule{0.723pt}{0.400pt}}
\multiput(465.00,261.17)(1.500,-1.000){2}{\rule{0.361pt}{0.400pt}}
\put(468,259.67){\rule{0.482pt}{0.400pt}}
\multiput(468.00,260.17)(1.000,-1.000){2}{\rule{0.241pt}{0.400pt}}
\put(470,258.67){\rule{0.482pt}{0.400pt}}
\multiput(470.00,259.17)(1.000,-1.000){2}{\rule{0.241pt}{0.400pt}}
\put(472,257.67){\rule{0.482pt}{0.400pt}}
\multiput(472.00,258.17)(1.000,-1.000){2}{\rule{0.241pt}{0.400pt}}
\put(474,256.67){\rule{0.482pt}{0.400pt}}
\multiput(474.00,257.17)(1.000,-1.000){2}{\rule{0.241pt}{0.400pt}}
\put(476,255.67){\rule{0.482pt}{0.400pt}}
\multiput(476.00,256.17)(1.000,-1.000){2}{\rule{0.241pt}{0.400pt}}
\put(478,254.67){\rule{0.482pt}{0.400pt}}
\multiput(478.00,255.17)(1.000,-1.000){2}{\rule{0.241pt}{0.400pt}}
\put(480,253.67){\rule{0.482pt}{0.400pt}}
\multiput(480.00,254.17)(1.000,-1.000){2}{\rule{0.241pt}{0.400pt}}
\put(482,252.17){\rule{0.482pt}{0.400pt}}
\multiput(482.00,253.17)(1.000,-2.000){2}{\rule{0.241pt}{0.400pt}}
\put(484,250.67){\rule{0.723pt}{0.400pt}}
\multiput(484.00,251.17)(1.500,-1.000){2}{\rule{0.361pt}{0.400pt}}
\put(487,249.67){\rule{0.482pt}{0.400pt}}
\multiput(487.00,250.17)(1.000,-1.000){2}{\rule{0.241pt}{0.400pt}}
\put(489,248.67){\rule{0.482pt}{0.400pt}}
\multiput(489.00,249.17)(1.000,-1.000){2}{\rule{0.241pt}{0.400pt}}
\put(491,247.67){\rule{0.482pt}{0.400pt}}
\multiput(491.00,248.17)(1.000,-1.000){2}{\rule{0.241pt}{0.400pt}}
\put(493,246.67){\rule{0.482pt}{0.400pt}}
\multiput(493.00,247.17)(1.000,-1.000){2}{\rule{0.241pt}{0.400pt}}
\put(495,245.67){\rule{0.482pt}{0.400pt}}
\multiput(495.00,246.17)(1.000,-1.000){2}{\rule{0.241pt}{0.400pt}}
\put(497,244.67){\rule{0.482pt}{0.400pt}}
\multiput(497.00,245.17)(1.000,-1.000){2}{\rule{0.241pt}{0.400pt}}
\put(499,243.67){\rule{0.482pt}{0.400pt}}
\multiput(499.00,244.17)(1.000,-1.000){2}{\rule{0.241pt}{0.400pt}}
\put(501,242.67){\rule{0.482pt}{0.400pt}}
\multiput(501.00,243.17)(1.000,-1.000){2}{\rule{0.241pt}{0.400pt}}
\put(503,241.67){\rule{0.723pt}{0.400pt}}
\multiput(503.00,242.17)(1.500,-1.000){2}{\rule{0.361pt}{0.400pt}}
\put(506,240.17){\rule{0.482pt}{0.400pt}}
\multiput(506.00,241.17)(1.000,-2.000){2}{\rule{0.241pt}{0.400pt}}
\put(508,238.67){\rule{0.482pt}{0.400pt}}
\multiput(508.00,239.17)(1.000,-1.000){2}{\rule{0.241pt}{0.400pt}}
\put(510,237.67){\rule{0.482pt}{0.400pt}}
\multiput(510.00,238.17)(1.000,-1.000){2}{\rule{0.241pt}{0.400pt}}
\put(512,236.67){\rule{0.482pt}{0.400pt}}
\multiput(512.00,237.17)(1.000,-1.000){2}{\rule{0.241pt}{0.400pt}}
\put(514,235.67){\rule{0.482pt}{0.400pt}}
\multiput(514.00,236.17)(1.000,-1.000){2}{\rule{0.241pt}{0.400pt}}
\put(516,234.67){\rule{0.482pt}{0.400pt}}
\multiput(516.00,235.17)(1.000,-1.000){2}{\rule{0.241pt}{0.400pt}}
\put(518,233.67){\rule{0.482pt}{0.400pt}}
\multiput(518.00,234.17)(1.000,-1.000){2}{\rule{0.241pt}{0.400pt}}
\put(520,232.67){\rule{0.482pt}{0.400pt}}
\multiput(520.00,233.17)(1.000,-1.000){2}{\rule{0.241pt}{0.400pt}}
\put(522,231.67){\rule{0.723pt}{0.400pt}}
\multiput(522.00,232.17)(1.500,-1.000){2}{\rule{0.361pt}{0.400pt}}
\put(525,230.67){\rule{0.482pt}{0.400pt}}
\multiput(525.00,231.17)(1.000,-1.000){2}{\rule{0.241pt}{0.400pt}}
\put(527,229.67){\rule{0.482pt}{0.400pt}}
\multiput(527.00,230.17)(1.000,-1.000){2}{\rule{0.241pt}{0.400pt}}
\put(529,228.17){\rule{0.482pt}{0.400pt}}
\multiput(529.00,229.17)(1.000,-2.000){2}{\rule{0.241pt}{0.400pt}}
\put(531,226.67){\rule{0.482pt}{0.400pt}}
\multiput(531.00,227.17)(1.000,-1.000){2}{\rule{0.241pt}{0.400pt}}
\put(533,225.67){\rule{0.482pt}{0.400pt}}
\multiput(533.00,226.17)(1.000,-1.000){2}{\rule{0.241pt}{0.400pt}}
\put(535,224.67){\rule{0.482pt}{0.400pt}}
\multiput(535.00,225.17)(1.000,-1.000){2}{\rule{0.241pt}{0.400pt}}
\put(537,223.67){\rule{0.482pt}{0.400pt}}
\multiput(537.00,224.17)(1.000,-1.000){2}{\rule{0.241pt}{0.400pt}}
\multiput(539.00,224.61)(0.462,0.447){3}{\rule{0.500pt}{0.108pt}}
\multiput(539.00,223.17)(1.962,3.000){2}{\rule{0.250pt}{0.400pt}}
\put(542,227.17){\rule{0.700pt}{0.400pt}}
\multiput(542.00,226.17)(1.547,2.000){2}{\rule{0.350pt}{0.400pt}}
\put(545.17,229){\rule{0.400pt}{0.700pt}}
\multiput(544.17,229.00)(2.000,1.547){2}{\rule{0.400pt}{0.350pt}}
\multiput(547.00,232.61)(0.462,0.447){3}{\rule{0.500pt}{0.108pt}}
\multiput(547.00,231.17)(1.962,3.000){2}{\rule{0.250pt}{0.400pt}}
\put(550,235.17){\rule{0.482pt}{0.400pt}}
\multiput(550.00,234.17)(1.000,2.000){2}{\rule{0.241pt}{0.400pt}}
\multiput(552.00,237.61)(0.462,0.447){3}{\rule{0.500pt}{0.108pt}}
\multiput(552.00,236.17)(1.962,3.000){2}{\rule{0.250pt}{0.400pt}}
\multiput(555.00,240.61)(0.462,0.447){3}{\rule{0.500pt}{0.108pt}}
\multiput(555.00,239.17)(1.962,3.000){2}{\rule{0.250pt}{0.400pt}}
\put(558,243.17){\rule{0.482pt}{0.400pt}}
\multiput(558.00,242.17)(1.000,2.000){2}{\rule{0.241pt}{0.400pt}}
\multiput(560.00,245.61)(0.462,0.447){3}{\rule{0.500pt}{0.108pt}}
\multiput(560.00,244.17)(1.962,3.000){2}{\rule{0.250pt}{0.400pt}}
\multiput(563.00,248.61)(0.462,0.447){3}{\rule{0.500pt}{0.108pt}}
\multiput(563.00,247.17)(1.962,3.000){2}{\rule{0.250pt}{0.400pt}}
\put(566,251.17){\rule{0.482pt}{0.400pt}}
\multiput(566.00,250.17)(1.000,2.000){2}{\rule{0.241pt}{0.400pt}}
\multiput(568.00,253.61)(0.462,0.447){3}{\rule{0.500pt}{0.108pt}}
\multiput(568.00,252.17)(1.962,3.000){2}{\rule{0.250pt}{0.400pt}}
\multiput(571.00,256.61)(0.462,0.447){3}{\rule{0.500pt}{0.108pt}}
\multiput(571.00,255.17)(1.962,3.000){2}{\rule{0.250pt}{0.400pt}}
\put(574,259.17){\rule{0.482pt}{0.400pt}}
\multiput(574.00,258.17)(1.000,2.000){2}{\rule{0.241pt}{0.400pt}}
\multiput(576.00,261.61)(0.462,0.447){3}{\rule{0.500pt}{0.108pt}}
\multiput(576.00,260.17)(1.962,3.000){2}{\rule{0.250pt}{0.400pt}}
\put(579,264.17){\rule{0.482pt}{0.400pt}}
\multiput(579.00,263.17)(1.000,2.000){2}{\rule{0.241pt}{0.400pt}}
\multiput(581.00,266.61)(0.462,0.447){3}{\rule{0.500pt}{0.108pt}}
\multiput(581.00,265.17)(1.962,3.000){2}{\rule{0.250pt}{0.400pt}}
\multiput(584.00,269.61)(0.462,0.447){3}{\rule{0.500pt}{0.108pt}}
\multiput(584.00,268.17)(1.962,3.000){2}{\rule{0.250pt}{0.400pt}}
\put(587,272.17){\rule{0.482pt}{0.400pt}}
\multiput(587.00,271.17)(1.000,2.000){2}{\rule{0.241pt}{0.400pt}}
\multiput(589.00,274.61)(0.462,0.447){3}{\rule{0.500pt}{0.108pt}}
\multiput(589.00,273.17)(1.962,3.000){2}{\rule{0.250pt}{0.400pt}}
\multiput(592.00,277.61)(0.462,0.447){3}{\rule{0.500pt}{0.108pt}}
\multiput(592.00,276.17)(1.962,3.000){2}{\rule{0.250pt}{0.400pt}}
\put(595,280.17){\rule{0.482pt}{0.400pt}}
\multiput(595.00,279.17)(1.000,2.000){2}{\rule{0.241pt}{0.400pt}}
\multiput(597.00,282.61)(0.462,0.447){3}{\rule{0.500pt}{0.108pt}}
\multiput(597.00,281.17)(1.962,3.000){2}{\rule{0.250pt}{0.400pt}}
\put(600.17,285){\rule{0.400pt}{0.700pt}}
\multiput(599.17,285.00)(2.000,1.547){2}{\rule{0.400pt}{0.350pt}}
\put(602,288.17){\rule{0.700pt}{0.400pt}}
\multiput(602.00,287.17)(1.547,2.000){2}{\rule{0.350pt}{0.400pt}}
\multiput(605.00,290.61)(0.462,0.447){3}{\rule{0.500pt}{0.108pt}}
\multiput(605.00,289.17)(1.962,3.000){2}{\rule{0.250pt}{0.400pt}}
\put(608.17,293){\rule{0.400pt}{0.700pt}}
\multiput(607.17,293.00)(2.000,1.547){2}{\rule{0.400pt}{0.350pt}}
\put(610,296.17){\rule{0.700pt}{0.400pt}}
\multiput(610.00,295.17)(1.547,2.000){2}{\rule{0.350pt}{0.400pt}}
\multiput(613.00,298.61)(0.462,0.447){3}{\rule{0.500pt}{0.108pt}}
\multiput(613.00,297.17)(1.962,3.000){2}{\rule{0.250pt}{0.400pt}}
\put(616.17,301){\rule{0.400pt}{0.700pt}}
\multiput(615.17,301.00)(2.000,1.547){2}{\rule{0.400pt}{0.350pt}}
\put(618,304.17){\rule{0.700pt}{0.400pt}}
\multiput(618.00,303.17)(1.547,2.000){2}{\rule{0.350pt}{0.400pt}}
\multiput(621.00,306.61)(0.462,0.447){3}{\rule{0.500pt}{0.108pt}}
\multiput(621.00,305.17)(1.962,3.000){2}{\rule{0.250pt}{0.400pt}}
\put(624.17,309){\rule{0.400pt}{0.700pt}}
\multiput(623.17,309.00)(2.000,1.547){2}{\rule{0.400pt}{0.350pt}}
\put(626,312.17){\rule{0.700pt}{0.400pt}}
\multiput(626.00,311.17)(1.547,2.000){2}{\rule{0.350pt}{0.400pt}}
\put(629.17,314){\rule{0.400pt}{0.700pt}}
\multiput(628.17,314.00)(2.000,1.547){2}{\rule{0.400pt}{0.350pt}}
\multiput(631.00,317.61)(0.462,0.447){3}{\rule{0.500pt}{0.108pt}}
\multiput(631.00,316.17)(1.962,3.000){2}{\rule{0.250pt}{0.400pt}}
\put(634,320.17){\rule{0.700pt}{0.400pt}}
\multiput(634.00,319.17)(1.547,2.000){2}{\rule{0.350pt}{0.400pt}}
\put(637.17,322){\rule{0.400pt}{0.700pt}}
\multiput(636.17,322.00)(2.000,1.547){2}{\rule{0.400pt}{0.350pt}}
\put(639,325.17){\rule{0.700pt}{0.400pt}}
\multiput(639.00,324.17)(1.547,2.000){2}{\rule{0.350pt}{0.400pt}}
\multiput(642.00,327.61)(0.462,0.447){3}{\rule{0.500pt}{0.108pt}}
\multiput(642.00,326.17)(1.962,3.000){2}{\rule{0.250pt}{0.400pt}}
\put(645.17,330){\rule{0.400pt}{0.700pt}}
\multiput(644.17,330.00)(2.000,1.547){2}{\rule{0.400pt}{0.350pt}}
\put(647,333.17){\rule{0.700pt}{0.400pt}}
\multiput(647.00,332.17)(1.547,2.000){2}{\rule{0.350pt}{0.400pt}}
\put(650.17,335){\rule{0.400pt}{0.700pt}}
\multiput(649.17,335.00)(2.000,1.547){2}{\rule{0.400pt}{0.350pt}}
\multiput(652.00,338.61)(0.462,0.447){3}{\rule{0.500pt}{0.108pt}}
\multiput(652.00,337.17)(1.962,3.000){2}{\rule{0.250pt}{0.400pt}}
\put(655,341.17){\rule{0.700pt}{0.400pt}}
\multiput(655.00,340.17)(1.547,2.000){2}{\rule{0.350pt}{0.400pt}}
\put(658.17,343){\rule{0.400pt}{0.700pt}}
\multiput(657.17,343.00)(2.000,1.547){2}{\rule{0.400pt}{0.350pt}}
\multiput(660.00,346.61)(0.462,0.447){3}{\rule{0.500pt}{0.108pt}}
\multiput(660.00,345.17)(1.962,3.000){2}{\rule{0.250pt}{0.400pt}}
\put(663,349.17){\rule{0.700pt}{0.400pt}}
\multiput(663.00,348.17)(1.547,2.000){2}{\rule{0.350pt}{0.400pt}}
\put(666.17,351){\rule{0.400pt}{0.700pt}}
\multiput(665.17,351.00)(2.000,1.547){2}{\rule{0.400pt}{0.350pt}}
\multiput(668.00,354.61)(0.462,0.447){3}{\rule{0.500pt}{0.108pt}}
\multiput(668.00,353.17)(1.962,3.000){2}{\rule{0.250pt}{0.400pt}}
\put(671,357.17){\rule{0.700pt}{0.400pt}}
\multiput(671.00,356.17)(1.547,2.000){2}{\rule{0.350pt}{0.400pt}}
\put(674.17,359){\rule{0.400pt}{0.700pt}}
\multiput(673.17,359.00)(2.000,1.547){2}{\rule{0.400pt}{0.350pt}}
\multiput(676.00,362.61)(0.462,0.447){3}{\rule{0.500pt}{0.108pt}}
\multiput(676.00,361.17)(1.962,3.000){2}{\rule{0.250pt}{0.400pt}}
\put(679,365.17){\rule{0.482pt}{0.400pt}}
\multiput(679.00,364.17)(1.000,2.000){2}{\rule{0.241pt}{0.400pt}}
\multiput(681.00,367.61)(0.462,0.447){3}{\rule{0.500pt}{0.108pt}}
\multiput(681.00,366.17)(1.962,3.000){2}{\rule{0.250pt}{0.400pt}}
\multiput(684.00,370.61)(0.462,0.447){3}{\rule{0.500pt}{0.108pt}}
\multiput(684.00,369.17)(1.962,3.000){2}{\rule{0.250pt}{0.400pt}}
\put(687,373.17){\rule{0.482pt}{0.400pt}}
\multiput(687.00,372.17)(1.000,2.000){2}{\rule{0.241pt}{0.400pt}}
\multiput(689.00,375.61)(0.462,0.447){3}{\rule{0.500pt}{0.108pt}}
\multiput(689.00,374.17)(1.962,3.000){2}{\rule{0.250pt}{0.400pt}}
\put(692,378.17){\rule{0.700pt}{0.400pt}}
\multiput(692.00,377.17)(1.547,2.000){2}{\rule{0.350pt}{0.400pt}}
\put(695.17,380){\rule{0.400pt}{0.700pt}}
\multiput(694.17,380.00)(2.000,1.547){2}{\rule{0.400pt}{0.350pt}}
\multiput(697.00,383.61)(0.462,0.447){3}{\rule{0.500pt}{0.108pt}}
\multiput(697.00,382.17)(1.962,3.000){2}{\rule{0.250pt}{0.400pt}}
\put(700,386.17){\rule{0.482pt}{0.400pt}}
\multiput(700.00,385.17)(1.000,2.000){2}{\rule{0.241pt}{0.400pt}}
\multiput(702.00,388.61)(0.462,0.447){3}{\rule{0.500pt}{0.108pt}}
\multiput(702.00,387.17)(1.962,3.000){2}{\rule{0.250pt}{0.400pt}}
\multiput(705.00,391.61)(0.462,0.447){3}{\rule{0.500pt}{0.108pt}}
\multiput(705.00,390.17)(1.962,3.000){2}{\rule{0.250pt}{0.400pt}}
\put(708,394.17){\rule{0.482pt}{0.400pt}}
\multiput(708.00,393.17)(1.000,2.000){2}{\rule{0.241pt}{0.400pt}}
\multiput(710.00,396.61)(0.462,0.447){3}{\rule{0.500pt}{0.108pt}}
\multiput(710.00,395.17)(1.962,3.000){2}{\rule{0.250pt}{0.400pt}}
\multiput(713.00,399.61)(0.462,0.447){3}{\rule{0.500pt}{0.108pt}}
\multiput(713.00,398.17)(1.962,3.000){2}{\rule{0.250pt}{0.400pt}}
\put(716,402.17){\rule{0.482pt}{0.400pt}}
\multiput(716.00,401.17)(1.000,2.000){2}{\rule{0.241pt}{0.400pt}}
\multiput(718.00,404.61)(0.462,0.447){3}{\rule{0.500pt}{0.108pt}}
\multiput(718.00,403.17)(1.962,3.000){2}{\rule{0.250pt}{0.400pt}}
\multiput(721.00,407.61)(0.462,0.447){3}{\rule{0.500pt}{0.108pt}}
\multiput(721.00,406.17)(1.962,3.000){2}{\rule{0.250pt}{0.400pt}}
\put(724,409.67){\rule{0.241pt}{0.400pt}}
\multiput(724.00,409.17)(0.500,1.000){2}{\rule{0.120pt}{0.400pt}}
\put(725,410.67){\rule{0.482pt}{0.400pt}}
\multiput(725.00,410.17)(1.000,1.000){2}{\rule{0.241pt}{0.400pt}}
\put(727,411.67){\rule{0.241pt}{0.400pt}}
\multiput(727.00,411.17)(0.500,1.000){2}{\rule{0.120pt}{0.400pt}}
\put(728,412.67){\rule{0.482pt}{0.400pt}}
\multiput(728.00,412.17)(1.000,1.000){2}{\rule{0.241pt}{0.400pt}}
\put(325.0,324.0){\rule[-0.200pt]{0.400pt}{0.482pt}}
\put(731,413.67){\rule{0.482pt}{0.400pt}}
\multiput(731.00,413.17)(1.000,1.000){2}{\rule{0.241pt}{0.400pt}}
\put(733,414.67){\rule{0.241pt}{0.400pt}}
\multiput(733.00,414.17)(0.500,1.000){2}{\rule{0.120pt}{0.400pt}}
\put(734,415.67){\rule{0.482pt}{0.400pt}}
\multiput(734.00,415.17)(1.000,1.000){2}{\rule{0.241pt}{0.400pt}}
\put(736,416.67){\rule{0.241pt}{0.400pt}}
\multiput(736.00,416.17)(0.500,1.000){2}{\rule{0.120pt}{0.400pt}}
\put(737,417.67){\rule{0.482pt}{0.400pt}}
\multiput(737.00,417.17)(1.000,1.000){2}{\rule{0.241pt}{0.400pt}}
\put(739,418.67){\rule{0.241pt}{0.400pt}}
\multiput(739.00,418.17)(0.500,1.000){2}{\rule{0.120pt}{0.400pt}}
\put(740,419.67){\rule{0.482pt}{0.400pt}}
\multiput(740.00,419.17)(1.000,1.000){2}{\rule{0.241pt}{0.400pt}}
\put(742,420.67){\rule{0.482pt}{0.400pt}}
\multiput(742.00,420.17)(1.000,1.000){2}{\rule{0.241pt}{0.400pt}}
\put(744,421.67){\rule{0.241pt}{0.400pt}}
\multiput(744.00,421.17)(0.500,1.000){2}{\rule{0.120pt}{0.400pt}}
\put(745,422.67){\rule{0.482pt}{0.400pt}}
\multiput(745.00,422.17)(1.000,1.000){2}{\rule{0.241pt}{0.400pt}}
\put(747,423.67){\rule{0.241pt}{0.400pt}}
\multiput(747.00,423.17)(0.500,1.000){2}{\rule{0.120pt}{0.400pt}}
\put(748,424.67){\rule{0.482pt}{0.400pt}}
\multiput(748.00,424.17)(1.000,1.000){2}{\rule{0.241pt}{0.400pt}}
\put(750,425.67){\rule{0.241pt}{0.400pt}}
\multiput(750.00,425.17)(0.500,1.000){2}{\rule{0.120pt}{0.400pt}}
\put(751,426.67){\rule{0.482pt}{0.400pt}}
\multiput(751.00,426.17)(1.000,1.000){2}{\rule{0.241pt}{0.400pt}}
\put(753,427.67){\rule{0.241pt}{0.400pt}}
\multiput(753.00,427.17)(0.500,1.000){2}{\rule{0.120pt}{0.400pt}}
\put(754,428.67){\rule{0.482pt}{0.400pt}}
\multiput(754.00,428.17)(1.000,1.000){2}{\rule{0.241pt}{0.400pt}}
\put(756,429.67){\rule{0.241pt}{0.400pt}}
\multiput(756.00,429.17)(0.500,1.000){2}{\rule{0.120pt}{0.400pt}}
\put(757,430.67){\rule{0.482pt}{0.400pt}}
\multiput(757.00,430.17)(1.000,1.000){2}{\rule{0.241pt}{0.400pt}}
\put(759,431.67){\rule{0.241pt}{0.400pt}}
\multiput(759.00,431.17)(0.500,1.000){2}{\rule{0.120pt}{0.400pt}}
\put(760,432.67){\rule{0.482pt}{0.400pt}}
\multiput(760.00,432.17)(1.000,1.000){2}{\rule{0.241pt}{0.400pt}}
\put(762,433.67){\rule{0.482pt}{0.400pt}}
\multiput(762.00,433.17)(1.000,1.000){2}{\rule{0.241pt}{0.400pt}}
\put(764,434.67){\rule{0.241pt}{0.400pt}}
\multiput(764.00,434.17)(0.500,1.000){2}{\rule{0.120pt}{0.400pt}}
\put(765,435.67){\rule{0.482pt}{0.400pt}}
\multiput(765.00,435.17)(1.000,1.000){2}{\rule{0.241pt}{0.400pt}}
\put(767,436.67){\rule{0.241pt}{0.400pt}}
\multiput(767.00,436.17)(0.500,1.000){2}{\rule{0.120pt}{0.400pt}}
\put(768,437.67){\rule{0.482pt}{0.400pt}}
\multiput(768.00,437.17)(1.000,1.000){2}{\rule{0.241pt}{0.400pt}}
\put(770,438.67){\rule{0.241pt}{0.400pt}}
\multiput(770.00,438.17)(0.500,1.000){2}{\rule{0.120pt}{0.400pt}}
\put(730.0,414.0){\usebox{\plotpoint}}
\put(773,439.67){\rule{0.241pt}{0.400pt}}
\multiput(773.00,439.17)(0.500,1.000){2}{\rule{0.120pt}{0.400pt}}
\put(774,440.67){\rule{0.482pt}{0.400pt}}
\multiput(774.00,440.17)(1.000,1.000){2}{\rule{0.241pt}{0.400pt}}
\put(776,441.67){\rule{0.241pt}{0.400pt}}
\multiput(776.00,441.17)(0.500,1.000){2}{\rule{0.120pt}{0.400pt}}
\put(777,442.67){\rule{0.482pt}{0.400pt}}
\multiput(777.00,442.17)(1.000,1.000){2}{\rule{0.241pt}{0.400pt}}
\put(779,443.67){\rule{0.482pt}{0.400pt}}
\multiput(779.00,443.17)(1.000,1.000){2}{\rule{0.241pt}{0.400pt}}
\put(781,444.67){\rule{0.241pt}{0.400pt}}
\multiput(781.00,444.17)(0.500,1.000){2}{\rule{0.120pt}{0.400pt}}
\put(782,445.67){\rule{0.482pt}{0.400pt}}
\multiput(782.00,445.17)(1.000,1.000){2}{\rule{0.241pt}{0.400pt}}
\put(784,446.67){\rule{0.241pt}{0.400pt}}
\multiput(784.00,446.17)(0.500,1.000){2}{\rule{0.120pt}{0.400pt}}
\put(785,447.67){\rule{0.482pt}{0.400pt}}
\multiput(785.00,447.17)(1.000,1.000){2}{\rule{0.241pt}{0.400pt}}
\put(787,448.67){\rule{0.241pt}{0.400pt}}
\multiput(787.00,448.17)(0.500,1.000){2}{\rule{0.120pt}{0.400pt}}
\put(788,449.67){\rule{0.482pt}{0.400pt}}
\multiput(788.00,449.17)(1.000,1.000){2}{\rule{0.241pt}{0.400pt}}
\put(790,450.67){\rule{0.241pt}{0.400pt}}
\multiput(790.00,450.17)(0.500,1.000){2}{\rule{0.120pt}{0.400pt}}
\put(791,451.67){\rule{0.482pt}{0.400pt}}
\multiput(791.00,451.17)(1.000,1.000){2}{\rule{0.241pt}{0.400pt}}
\put(793,452.67){\rule{0.241pt}{0.400pt}}
\multiput(793.00,452.17)(0.500,1.000){2}{\rule{0.120pt}{0.400pt}}
\put(794,453.67){\rule{0.482pt}{0.400pt}}
\multiput(794.00,453.17)(1.000,1.000){2}{\rule{0.241pt}{0.400pt}}
\put(796,454.67){\rule{0.241pt}{0.400pt}}
\multiput(796.00,454.17)(0.500,1.000){2}{\rule{0.120pt}{0.400pt}}
\put(797,455.67){\rule{0.482pt}{0.400pt}}
\multiput(797.00,455.17)(1.000,1.000){2}{\rule{0.241pt}{0.400pt}}
\put(799,456.67){\rule{0.482pt}{0.400pt}}
\multiput(799.00,456.17)(1.000,1.000){2}{\rule{0.241pt}{0.400pt}}
\put(801,457.67){\rule{0.241pt}{0.400pt}}
\multiput(801.00,457.17)(0.500,1.000){2}{\rule{0.120pt}{0.400pt}}
\put(802,458.67){\rule{0.482pt}{0.400pt}}
\multiput(802.00,458.17)(1.000,1.000){2}{\rule{0.241pt}{0.400pt}}
\put(804,459.67){\rule{0.241pt}{0.400pt}}
\multiput(804.00,459.17)(0.500,1.000){2}{\rule{0.120pt}{0.400pt}}
\put(805,460.67){\rule{0.482pt}{0.400pt}}
\multiput(805.00,460.17)(1.000,1.000){2}{\rule{0.241pt}{0.400pt}}
\put(807,461.67){\rule{0.241pt}{0.400pt}}
\multiput(807.00,461.17)(0.500,1.000){2}{\rule{0.120pt}{0.400pt}}
\put(808,462.67){\rule{0.482pt}{0.400pt}}
\multiput(808.00,462.17)(1.000,1.000){2}{\rule{0.241pt}{0.400pt}}
\put(810,463.67){\rule{0.241pt}{0.400pt}}
\multiput(810.00,463.17)(0.500,1.000){2}{\rule{0.120pt}{0.400pt}}
\put(811,464.67){\rule{0.482pt}{0.400pt}}
\multiput(811.00,464.17)(1.000,1.000){2}{\rule{0.241pt}{0.400pt}}
\put(771.0,440.0){\rule[-0.200pt]{0.482pt}{0.400pt}}
\put(814,465.67){\rule{0.482pt}{0.400pt}}
\multiput(814.00,465.17)(1.000,1.000){2}{\rule{0.241pt}{0.400pt}}
\put(816,466.67){\rule{0.241pt}{0.400pt}}
\multiput(816.00,466.17)(0.500,1.000){2}{\rule{0.120pt}{0.400pt}}
\put(817,467.67){\rule{0.482pt}{0.400pt}}
\multiput(817.00,467.17)(1.000,1.000){2}{\rule{0.241pt}{0.400pt}}
\put(819,468.67){\rule{0.482pt}{0.400pt}}
\multiput(819.00,468.17)(1.000,1.000){2}{\rule{0.241pt}{0.400pt}}
\put(821,469.67){\rule{0.241pt}{0.400pt}}
\multiput(821.00,469.17)(0.500,1.000){2}{\rule{0.120pt}{0.400pt}}
\put(822,470.67){\rule{0.482pt}{0.400pt}}
\multiput(822.00,470.17)(1.000,1.000){2}{\rule{0.241pt}{0.400pt}}
\put(824,471.67){\rule{0.241pt}{0.400pt}}
\multiput(824.00,471.17)(0.500,1.000){2}{\rule{0.120pt}{0.400pt}}
\put(825,472.67){\rule{0.482pt}{0.400pt}}
\multiput(825.00,472.17)(1.000,1.000){2}{\rule{0.241pt}{0.400pt}}
\put(827,473.67){\rule{0.241pt}{0.400pt}}
\multiput(827.00,473.17)(0.500,1.000){2}{\rule{0.120pt}{0.400pt}}
\put(828,474.67){\rule{0.482pt}{0.400pt}}
\multiput(828.00,474.17)(1.000,1.000){2}{\rule{0.241pt}{0.400pt}}
\put(830,475.67){\rule{0.241pt}{0.400pt}}
\multiput(830.00,475.17)(0.500,1.000){2}{\rule{0.120pt}{0.400pt}}
\put(831,476.67){\rule{0.482pt}{0.400pt}}
\multiput(831.00,476.17)(1.000,1.000){2}{\rule{0.241pt}{0.400pt}}
\put(833,477.67){\rule{0.241pt}{0.400pt}}
\multiput(833.00,477.17)(0.500,1.000){2}{\rule{0.120pt}{0.400pt}}
\put(834,478.67){\rule{0.482pt}{0.400pt}}
\multiput(834.00,478.17)(1.000,1.000){2}{\rule{0.241pt}{0.400pt}}
\put(836,479.67){\rule{0.241pt}{0.400pt}}
\multiput(836.00,479.17)(0.500,1.000){2}{\rule{0.120pt}{0.400pt}}
\put(837,480.67){\rule{0.482pt}{0.400pt}}
\multiput(837.00,480.17)(1.000,1.000){2}{\rule{0.241pt}{0.400pt}}
\put(839,481.67){\rule{0.482pt}{0.400pt}}
\multiput(839.00,481.17)(1.000,1.000){2}{\rule{0.241pt}{0.400pt}}
\put(841,482.67){\rule{0.241pt}{0.400pt}}
\multiput(841.00,482.17)(0.500,1.000){2}{\rule{0.120pt}{0.400pt}}
\put(842,483.67){\rule{0.482pt}{0.400pt}}
\multiput(842.00,483.17)(1.000,1.000){2}{\rule{0.241pt}{0.400pt}}
\put(844,484.67){\rule{0.241pt}{0.400pt}}
\multiput(844.00,484.17)(0.500,1.000){2}{\rule{0.120pt}{0.400pt}}
\put(845,485.67){\rule{0.482pt}{0.400pt}}
\multiput(845.00,485.17)(1.000,1.000){2}{\rule{0.241pt}{0.400pt}}
\put(847,486.67){\rule{0.241pt}{0.400pt}}
\multiput(847.00,486.17)(0.500,1.000){2}{\rule{0.120pt}{0.400pt}}
\put(848,487.67){\rule{0.482pt}{0.400pt}}
\multiput(848.00,487.17)(1.000,1.000){2}{\rule{0.241pt}{0.400pt}}
\put(850,488.67){\rule{0.241pt}{0.400pt}}
\multiput(850.00,488.17)(0.500,1.000){2}{\rule{0.120pt}{0.400pt}}
\put(851,489.67){\rule{0.482pt}{0.400pt}}
\multiput(851.00,489.17)(1.000,1.000){2}{\rule{0.241pt}{0.400pt}}
\put(853,490.67){\rule{0.241pt}{0.400pt}}
\multiput(853.00,490.17)(0.500,1.000){2}{\rule{0.120pt}{0.400pt}}
\put(854,491.67){\rule{0.482pt}{0.400pt}}
\multiput(854.00,491.17)(1.000,1.000){2}{\rule{0.241pt}{0.400pt}}
\put(813.0,466.0){\usebox{\plotpoint}}
\put(857,492.67){\rule{0.482pt}{0.400pt}}
\multiput(857.00,492.17)(1.000,1.000){2}{\rule{0.241pt}{0.400pt}}
\put(859,493.67){\rule{0.482pt}{0.400pt}}
\multiput(859.00,493.17)(1.000,1.000){2}{\rule{0.241pt}{0.400pt}}
\put(861,494.67){\rule{0.241pt}{0.400pt}}
\multiput(861.00,494.17)(0.500,1.000){2}{\rule{0.120pt}{0.400pt}}
\put(862,495.67){\rule{0.482pt}{0.400pt}}
\multiput(862.00,495.17)(1.000,1.000){2}{\rule{0.241pt}{0.400pt}}
\put(864,496.67){\rule{0.241pt}{0.400pt}}
\multiput(864.00,496.17)(0.500,1.000){2}{\rule{0.120pt}{0.400pt}}
\put(865,497.67){\rule{0.482pt}{0.400pt}}
\multiput(865.00,497.17)(1.000,1.000){2}{\rule{0.241pt}{0.400pt}}
\put(867,498.67){\rule{0.241pt}{0.400pt}}
\multiput(867.00,498.17)(0.500,1.000){2}{\rule{0.120pt}{0.400pt}}
\put(868,499.67){\rule{0.482pt}{0.400pt}}
\multiput(868.00,499.17)(1.000,1.000){2}{\rule{0.241pt}{0.400pt}}
\put(870,500.67){\rule{0.241pt}{0.400pt}}
\multiput(870.00,500.17)(0.500,1.000){2}{\rule{0.120pt}{0.400pt}}
\put(871,501.67){\rule{0.482pt}{0.400pt}}
\multiput(871.00,501.17)(1.000,1.000){2}{\rule{0.241pt}{0.400pt}}
\put(873,502.67){\rule{0.241pt}{0.400pt}}
\multiput(873.00,502.17)(0.500,1.000){2}{\rule{0.120pt}{0.400pt}}
\put(874,503.67){\rule{0.482pt}{0.400pt}}
\multiput(874.00,503.17)(1.000,1.000){2}{\rule{0.241pt}{0.400pt}}
\put(876,504.67){\rule{0.482pt}{0.400pt}}
\multiput(876.00,504.17)(1.000,1.000){2}{\rule{0.241pt}{0.400pt}}
\put(878,505.67){\rule{0.241pt}{0.400pt}}
\multiput(878.00,505.17)(0.500,1.000){2}{\rule{0.120pt}{0.400pt}}
\put(879,506.67){\rule{0.482pt}{0.400pt}}
\multiput(879.00,506.17)(1.000,1.000){2}{\rule{0.241pt}{0.400pt}}
\put(881,507.67){\rule{0.241pt}{0.400pt}}
\multiput(881.00,507.17)(0.500,1.000){2}{\rule{0.120pt}{0.400pt}}
\put(882,508.67){\rule{0.482pt}{0.400pt}}
\multiput(882.00,508.17)(1.000,1.000){2}{\rule{0.241pt}{0.400pt}}
\put(884,509.67){\rule{0.241pt}{0.400pt}}
\multiput(884.00,509.17)(0.500,1.000){2}{\rule{0.120pt}{0.400pt}}
\put(885,510.67){\rule{0.482pt}{0.400pt}}
\multiput(885.00,510.17)(1.000,1.000){2}{\rule{0.241pt}{0.400pt}}
\put(887,511.67){\rule{0.241pt}{0.400pt}}
\multiput(887.00,511.17)(0.500,1.000){2}{\rule{0.120pt}{0.400pt}}
\put(888,512.67){\rule{0.482pt}{0.400pt}}
\multiput(888.00,512.17)(1.000,1.000){2}{\rule{0.241pt}{0.400pt}}
\put(890,513.67){\rule{0.241pt}{0.400pt}}
\multiput(890.00,513.17)(0.500,1.000){2}{\rule{0.120pt}{0.400pt}}
\put(891,514.67){\rule{0.482pt}{0.400pt}}
\multiput(891.00,514.17)(1.000,1.000){2}{\rule{0.241pt}{0.400pt}}
\put(893,515.67){\rule{0.241pt}{0.400pt}}
\multiput(893.00,515.17)(0.500,1.000){2}{\rule{0.120pt}{0.400pt}}
\put(894,516.67){\rule{0.482pt}{0.400pt}}
\multiput(894.00,516.17)(1.000,1.000){2}{\rule{0.241pt}{0.400pt}}
\put(896,517.67){\rule{0.482pt}{0.400pt}}
\multiput(896.00,517.17)(1.000,1.000){2}{\rule{0.241pt}{0.400pt}}
\put(856.0,493.0){\usebox{\plotpoint}}
\put(899,518.67){\rule{0.482pt}{0.400pt}}
\multiput(899.00,518.17)(1.000,1.000){2}{\rule{0.241pt}{0.400pt}}
\put(901,519.67){\rule{0.241pt}{0.400pt}}
\multiput(901.00,519.17)(0.500,1.000){2}{\rule{0.120pt}{0.400pt}}
\put(902,520.67){\rule{0.482pt}{0.400pt}}
\multiput(902.00,520.17)(1.000,1.000){2}{\rule{0.241pt}{0.400pt}}
\put(904,521.67){\rule{0.241pt}{0.400pt}}
\multiput(904.00,521.17)(0.500,1.000){2}{\rule{0.120pt}{0.400pt}}
\put(905,522.67){\rule{0.482pt}{0.400pt}}
\multiput(905.00,522.17)(1.000,1.000){2}{\rule{0.241pt}{0.400pt}}
\put(907,523.67){\rule{0.241pt}{0.400pt}}
\multiput(907.00,523.17)(0.500,1.000){2}{\rule{0.120pt}{0.400pt}}
\put(908,524.67){\rule{0.482pt}{0.400pt}}
\multiput(908.00,524.17)(1.000,1.000){2}{\rule{0.241pt}{0.400pt}}
\put(910,525.67){\rule{0.241pt}{0.400pt}}
\multiput(910.00,525.17)(0.500,1.000){2}{\rule{0.120pt}{0.400pt}}
\put(911,526.67){\rule{0.482pt}{0.400pt}}
\multiput(911.00,526.17)(1.000,1.000){2}{\rule{0.241pt}{0.400pt}}
\put(913,527.67){\rule{0.241pt}{0.400pt}}
\multiput(913.00,527.17)(0.500,1.000){2}{\rule{0.120pt}{0.400pt}}
\put(914,528.67){\rule{0.482pt}{0.400pt}}
\multiput(914.00,528.17)(1.000,1.000){2}{\rule{0.241pt}{0.400pt}}
\put(916,529.67){\rule{0.482pt}{0.400pt}}
\multiput(916.00,529.17)(1.000,1.000){2}{\rule{0.241pt}{0.400pt}}
\put(918,530.67){\rule{0.241pt}{0.400pt}}
\multiput(918.00,530.17)(0.500,1.000){2}{\rule{0.120pt}{0.400pt}}
\put(919,531.67){\rule{0.482pt}{0.400pt}}
\multiput(919.00,531.17)(1.000,1.000){2}{\rule{0.241pt}{0.400pt}}
\put(921,532.67){\rule{0.241pt}{0.400pt}}
\multiput(921.00,532.17)(0.500,1.000){2}{\rule{0.120pt}{0.400pt}}
\put(922,533.67){\rule{0.482pt}{0.400pt}}
\multiput(922.00,533.17)(1.000,1.000){2}{\rule{0.241pt}{0.400pt}}
\put(924,534.67){\rule{0.241pt}{0.400pt}}
\multiput(924.00,534.17)(0.500,1.000){2}{\rule{0.120pt}{0.400pt}}
\put(925,535.67){\rule{0.482pt}{0.400pt}}
\multiput(925.00,535.17)(1.000,1.000){2}{\rule{0.241pt}{0.400pt}}
\put(927,536.67){\rule{0.241pt}{0.400pt}}
\multiput(927.00,536.17)(0.500,1.000){2}{\rule{0.120pt}{0.400pt}}
\put(928,537.67){\rule{0.482pt}{0.400pt}}
\multiput(928.00,537.17)(1.000,1.000){2}{\rule{0.241pt}{0.400pt}}
\put(930,538.67){\rule{0.241pt}{0.400pt}}
\multiput(930.00,538.17)(0.500,1.000){2}{\rule{0.120pt}{0.400pt}}
\put(931,539.67){\rule{0.482pt}{0.400pt}}
\multiput(931.00,539.17)(1.000,1.000){2}{\rule{0.241pt}{0.400pt}}
\put(933,540.67){\rule{0.241pt}{0.400pt}}
\multiput(933.00,540.17)(0.500,1.000){2}{\rule{0.120pt}{0.400pt}}
\put(934,541.67){\rule{0.482pt}{0.400pt}}
\multiput(934.00,541.17)(1.000,1.000){2}{\rule{0.241pt}{0.400pt}}
\put(936,542.67){\rule{0.482pt}{0.400pt}}
\multiput(936.00,542.17)(1.000,1.000){2}{\rule{0.241pt}{0.400pt}}
\put(938,543.67){\rule{0.241pt}{0.400pt}}
\multiput(938.00,543.17)(0.500,1.000){2}{\rule{0.120pt}{0.400pt}}
\put(898.0,519.0){\usebox{\plotpoint}}
\put(941,544.67){\rule{0.241pt}{0.400pt}}
\multiput(941.00,544.17)(0.500,1.000){2}{\rule{0.120pt}{0.400pt}}
\put(942,545.67){\rule{0.482pt}{0.400pt}}
\multiput(942.00,545.17)(1.000,1.000){2}{\rule{0.241pt}{0.400pt}}
\put(944,546.67){\rule{0.241pt}{0.400pt}}
\multiput(944.00,546.17)(0.500,1.000){2}{\rule{0.120pt}{0.400pt}}
\put(945,547.67){\rule{0.482pt}{0.400pt}}
\multiput(945.00,547.17)(1.000,1.000){2}{\rule{0.241pt}{0.400pt}}
\put(947,548.67){\rule{0.241pt}{0.400pt}}
\multiput(947.00,548.17)(0.500,1.000){2}{\rule{0.120pt}{0.400pt}}
\put(948,549.67){\rule{0.482pt}{0.400pt}}
\multiput(948.00,549.17)(1.000,1.000){2}{\rule{0.241pt}{0.400pt}}
\put(950,550.67){\rule{0.241pt}{0.400pt}}
\multiput(950.00,550.17)(0.500,1.000){2}{\rule{0.120pt}{0.400pt}}
\put(951,551.67){\rule{0.482pt}{0.400pt}}
\multiput(951.00,551.17)(1.000,1.000){2}{\rule{0.241pt}{0.400pt}}
\put(953,552.67){\rule{0.241pt}{0.400pt}}
\multiput(953.00,552.17)(0.500,1.000){2}{\rule{0.120pt}{0.400pt}}
\put(954,553.67){\rule{0.482pt}{0.400pt}}
\multiput(954.00,553.17)(1.000,1.000){2}{\rule{0.241pt}{0.400pt}}
\put(956,554.67){\rule{0.482pt}{0.400pt}}
\multiput(956.00,554.17)(1.000,1.000){2}{\rule{0.241pt}{0.400pt}}
\put(958,555.67){\rule{0.241pt}{0.400pt}}
\multiput(958.00,555.17)(0.500,1.000){2}{\rule{0.120pt}{0.400pt}}
\put(959,556.67){\rule{0.482pt}{0.400pt}}
\multiput(959.00,556.17)(1.000,1.000){2}{\rule{0.241pt}{0.400pt}}
\put(961,557.67){\rule{0.241pt}{0.400pt}}
\multiput(961.00,557.17)(0.500,1.000){2}{\rule{0.120pt}{0.400pt}}
\put(962,558.67){\rule{0.482pt}{0.400pt}}
\multiput(962.00,558.17)(1.000,1.000){2}{\rule{0.241pt}{0.400pt}}
\put(964,559.67){\rule{0.241pt}{0.400pt}}
\multiput(964.00,559.17)(0.500,1.000){2}{\rule{0.120pt}{0.400pt}}
\put(965,560.67){\rule{0.482pt}{0.400pt}}
\multiput(965.00,560.17)(1.000,1.000){2}{\rule{0.241pt}{0.400pt}}
\put(967,561.67){\rule{0.241pt}{0.400pt}}
\multiput(967.00,561.17)(0.500,1.000){2}{\rule{0.120pt}{0.400pt}}
\put(968,562.67){\rule{0.482pt}{0.400pt}}
\multiput(968.00,562.17)(1.000,1.000){2}{\rule{0.241pt}{0.400pt}}
\put(970,563.67){\rule{0.241pt}{0.400pt}}
\multiput(970.00,563.17)(0.500,1.000){2}{\rule{0.120pt}{0.400pt}}
\put(971,564.67){\rule{0.482pt}{0.400pt}}
\multiput(971.00,564.17)(1.000,1.000){2}{\rule{0.241pt}{0.400pt}}
\put(973,565.67){\rule{0.482pt}{0.400pt}}
\multiput(973.00,565.17)(1.000,1.000){2}{\rule{0.241pt}{0.400pt}}
\put(975,566.67){\rule{0.241pt}{0.400pt}}
\multiput(975.00,566.17)(0.500,1.000){2}{\rule{0.120pt}{0.400pt}}
\put(976,567.67){\rule{0.482pt}{0.400pt}}
\multiput(976.00,567.17)(1.000,1.000){2}{\rule{0.241pt}{0.400pt}}
\put(978,568.67){\rule{0.241pt}{0.400pt}}
\multiput(978.00,568.17)(0.500,1.000){2}{\rule{0.120pt}{0.400pt}}
\put(979,569.67){\rule{0.482pt}{0.400pt}}
\multiput(979.00,569.17)(1.000,1.000){2}{\rule{0.241pt}{0.400pt}}
\put(939.0,545.0){\rule[-0.200pt]{0.482pt}{0.400pt}}
\put(982,570.67){\rule{0.482pt}{0.400pt}}
\multiput(982.00,570.17)(1.000,1.000){2}{\rule{0.241pt}{0.400pt}}
\put(984,571.67){\rule{0.241pt}{0.400pt}}
\multiput(984.00,571.17)(0.500,1.000){2}{\rule{0.120pt}{0.400pt}}
\put(985,572.67){\rule{0.482pt}{0.400pt}}
\multiput(985.00,572.17)(1.000,1.000){2}{\rule{0.241pt}{0.400pt}}
\put(987,573.67){\rule{0.241pt}{0.400pt}}
\multiput(987.00,573.17)(0.500,1.000){2}{\rule{0.120pt}{0.400pt}}
\put(988,574.67){\rule{0.482pt}{0.400pt}}
\multiput(988.00,574.17)(1.000,1.000){2}{\rule{0.241pt}{0.400pt}}
\put(990,575.67){\rule{0.241pt}{0.400pt}}
\multiput(990.00,575.17)(0.500,1.000){2}{\rule{0.120pt}{0.400pt}}
\put(991,576.67){\rule{0.482pt}{0.400pt}}
\multiput(991.00,576.17)(1.000,1.000){2}{\rule{0.241pt}{0.400pt}}
\put(993,577.67){\rule{0.482pt}{0.400pt}}
\multiput(993.00,577.17)(1.000,1.000){2}{\rule{0.241pt}{0.400pt}}
\put(995,578.67){\rule{0.241pt}{0.400pt}}
\multiput(995.00,578.17)(0.500,1.000){2}{\rule{0.120pt}{0.400pt}}
\put(996,579.67){\rule{0.482pt}{0.400pt}}
\multiput(996.00,579.17)(1.000,1.000){2}{\rule{0.241pt}{0.400pt}}
\put(998,580.67){\rule{0.241pt}{0.400pt}}
\multiput(998.00,580.17)(0.500,1.000){2}{\rule{0.120pt}{0.400pt}}
\put(999,581.67){\rule{0.482pt}{0.400pt}}
\multiput(999.00,581.17)(1.000,1.000){2}{\rule{0.241pt}{0.400pt}}
\put(1001,582.67){\rule{0.241pt}{0.400pt}}
\multiput(1001.00,582.17)(0.500,1.000){2}{\rule{0.120pt}{0.400pt}}
\put(1002,583.67){\rule{0.482pt}{0.400pt}}
\multiput(1002.00,583.17)(1.000,1.000){2}{\rule{0.241pt}{0.400pt}}
\put(1004,584.67){\rule{0.241pt}{0.400pt}}
\multiput(1004.00,584.17)(0.500,1.000){2}{\rule{0.120pt}{0.400pt}}
\put(1005,585.67){\rule{0.482pt}{0.400pt}}
\multiput(1005.00,585.17)(1.000,1.000){2}{\rule{0.241pt}{0.400pt}}
\put(1007,586.67){\rule{0.241pt}{0.400pt}}
\multiput(1007.00,586.17)(0.500,1.000){2}{\rule{0.120pt}{0.400pt}}
\put(1008,587.67){\rule{0.482pt}{0.400pt}}
\multiput(1008.00,587.17)(1.000,1.000){2}{\rule{0.241pt}{0.400pt}}
\put(1010,588.67){\rule{0.241pt}{0.400pt}}
\multiput(1010.00,588.17)(0.500,1.000){2}{\rule{0.120pt}{0.400pt}}
\put(1011,589.67){\rule{0.482pt}{0.400pt}}
\multiput(1011.00,589.17)(1.000,1.000){2}{\rule{0.241pt}{0.400pt}}
\put(1013,590.67){\rule{0.482pt}{0.400pt}}
\multiput(1013.00,590.17)(1.000,1.000){2}{\rule{0.241pt}{0.400pt}}
\put(1015,591.67){\rule{0.241pt}{0.400pt}}
\multiput(1015.00,591.17)(0.500,1.000){2}{\rule{0.120pt}{0.400pt}}
\put(1016,592.67){\rule{0.482pt}{0.400pt}}
\multiput(1016.00,592.17)(1.000,1.000){2}{\rule{0.241pt}{0.400pt}}
\put(1018,593.67){\rule{0.241pt}{0.400pt}}
\multiput(1018.00,593.17)(0.500,1.000){2}{\rule{0.120pt}{0.400pt}}
\put(1019,594.67){\rule{0.482pt}{0.400pt}}
\multiput(1019.00,594.17)(1.000,1.000){2}{\rule{0.241pt}{0.400pt}}
\put(1021,595.67){\rule{0.241pt}{0.400pt}}
\multiput(1021.00,595.17)(0.500,1.000){2}{\rule{0.120pt}{0.400pt}}
\put(1022,596.67){\rule{0.482pt}{0.400pt}}
\multiput(1022.00,596.17)(1.000,1.000){2}{\rule{0.241pt}{0.400pt}}
\put(981.0,571.0){\usebox{\plotpoint}}
\put(1025,597.67){\rule{0.482pt}{0.400pt}}
\multiput(1025.00,597.17)(1.000,1.000){2}{\rule{0.241pt}{0.400pt}}
\put(1027,598.67){\rule{0.241pt}{0.400pt}}
\multiput(1027.00,598.17)(0.500,1.000){2}{\rule{0.120pt}{0.400pt}}
\put(1028,599.67){\rule{0.482pt}{0.400pt}}
\multiput(1028.00,599.17)(1.000,1.000){2}{\rule{0.241pt}{0.400pt}}
\put(1030,600.67){\rule{0.241pt}{0.400pt}}
\multiput(1030.00,600.17)(0.500,1.000){2}{\rule{0.120pt}{0.400pt}}
\put(1031,601.67){\rule{0.482pt}{0.400pt}}
\multiput(1031.00,601.17)(1.000,1.000){2}{\rule{0.241pt}{0.400pt}}
\put(1033,602.67){\rule{0.482pt}{0.400pt}}
\multiput(1033.00,602.17)(1.000,1.000){2}{\rule{0.241pt}{0.400pt}}
\put(1035,603.67){\rule{0.241pt}{0.400pt}}
\multiput(1035.00,603.17)(0.500,1.000){2}{\rule{0.120pt}{0.400pt}}
\put(1036,604.67){\rule{0.482pt}{0.400pt}}
\multiput(1036.00,604.17)(1.000,1.000){2}{\rule{0.241pt}{0.400pt}}
\put(1038,605.67){\rule{0.241pt}{0.400pt}}
\multiput(1038.00,605.17)(0.500,1.000){2}{\rule{0.120pt}{0.400pt}}
\put(1039,606.67){\rule{0.482pt}{0.400pt}}
\multiput(1039.00,606.17)(1.000,1.000){2}{\rule{0.241pt}{0.400pt}}
\put(1041,607.67){\rule{0.241pt}{0.400pt}}
\multiput(1041.00,607.17)(0.500,1.000){2}{\rule{0.120pt}{0.400pt}}
\put(1042,608.67){\rule{0.482pt}{0.400pt}}
\multiput(1042.00,608.17)(1.000,1.000){2}{\rule{0.241pt}{0.400pt}}
\put(1044,609.67){\rule{0.241pt}{0.400pt}}
\multiput(1044.00,609.17)(0.500,1.000){2}{\rule{0.120pt}{0.400pt}}
\put(1045,610.67){\rule{0.482pt}{0.400pt}}
\multiput(1045.00,610.17)(1.000,1.000){2}{\rule{0.241pt}{0.400pt}}
\put(1047,611.67){\rule{0.241pt}{0.400pt}}
\multiput(1047.00,611.17)(0.500,1.000){2}{\rule{0.120pt}{0.400pt}}
\put(1048,612.67){\rule{0.482pt}{0.400pt}}
\multiput(1048.00,612.17)(1.000,1.000){2}{\rule{0.241pt}{0.400pt}}
\put(1050,613.67){\rule{0.482pt}{0.400pt}}
\multiput(1050.00,613.17)(1.000,1.000){2}{\rule{0.241pt}{0.400pt}}
\put(1052,614.67){\rule{0.241pt}{0.400pt}}
\multiput(1052.00,614.17)(0.500,1.000){2}{\rule{0.120pt}{0.400pt}}
\put(1053,615.67){\rule{0.482pt}{0.400pt}}
\multiput(1053.00,615.17)(1.000,1.000){2}{\rule{0.241pt}{0.400pt}}
\put(1055,616.67){\rule{0.241pt}{0.400pt}}
\multiput(1055.00,616.17)(0.500,1.000){2}{\rule{0.120pt}{0.400pt}}
\put(1056,617.67){\rule{0.482pt}{0.400pt}}
\multiput(1056.00,617.17)(1.000,1.000){2}{\rule{0.241pt}{0.400pt}}
\put(1058,618.67){\rule{0.241pt}{0.400pt}}
\multiput(1058.00,618.17)(0.500,1.000){2}{\rule{0.120pt}{0.400pt}}
\put(1059,619.67){\rule{0.482pt}{0.400pt}}
\multiput(1059.00,619.17)(1.000,1.000){2}{\rule{0.241pt}{0.400pt}}
\put(1061,620.67){\rule{0.241pt}{0.400pt}}
\multiput(1061.00,620.17)(0.500,1.000){2}{\rule{0.120pt}{0.400pt}}
\put(1062,621.67){\rule{0.482pt}{0.400pt}}
\multiput(1062.00,621.17)(1.000,1.000){2}{\rule{0.241pt}{0.400pt}}
\put(1064,622.67){\rule{0.241pt}{0.400pt}}
\multiput(1064.00,622.17)(0.500,1.000){2}{\rule{0.120pt}{0.400pt}}
\put(1024.0,598.0){\usebox{\plotpoint}}
\put(1067,623.67){\rule{0.241pt}{0.400pt}}
\multiput(1067.00,623.17)(0.500,1.000){2}{\rule{0.120pt}{0.400pt}}
\put(1068,624.67){\rule{0.482pt}{0.400pt}}
\multiput(1068.00,624.17)(1.000,1.000){2}{\rule{0.241pt}{0.400pt}}
\put(1070,625.67){\rule{0.482pt}{0.400pt}}
\multiput(1070.00,625.17)(1.000,1.000){2}{\rule{0.241pt}{0.400pt}}
\put(1072,626.67){\rule{0.241pt}{0.400pt}}
\multiput(1072.00,626.17)(0.500,1.000){2}{\rule{0.120pt}{0.400pt}}
\put(1073,627.67){\rule{0.482pt}{0.400pt}}
\multiput(1073.00,627.17)(1.000,1.000){2}{\rule{0.241pt}{0.400pt}}
\put(1075,628.67){\rule{0.241pt}{0.400pt}}
\multiput(1075.00,628.17)(0.500,1.000){2}{\rule{0.120pt}{0.400pt}}
\put(1076,629.67){\rule{0.482pt}{0.400pt}}
\multiput(1076.00,629.17)(1.000,1.000){2}{\rule{0.241pt}{0.400pt}}
\put(1078,630.67){\rule{0.241pt}{0.400pt}}
\multiput(1078.00,630.17)(0.500,1.000){2}{\rule{0.120pt}{0.400pt}}
\put(1079,631.67){\rule{0.482pt}{0.400pt}}
\multiput(1079.00,631.17)(1.000,1.000){2}{\rule{0.241pt}{0.400pt}}
\put(1081,632.67){\rule{0.241pt}{0.400pt}}
\multiput(1081.00,632.17)(0.500,1.000){2}{\rule{0.120pt}{0.400pt}}
\put(1082,633.67){\rule{0.482pt}{0.400pt}}
\multiput(1082.00,633.17)(1.000,1.000){2}{\rule{0.241pt}{0.400pt}}
\put(1084,634.67){\rule{0.241pt}{0.400pt}}
\multiput(1084.00,634.17)(0.500,1.000){2}{\rule{0.120pt}{0.400pt}}
\put(1085,635.67){\rule{0.482pt}{0.400pt}}
\multiput(1085.00,635.17)(1.000,1.000){2}{\rule{0.241pt}{0.400pt}}
\put(1087,636.67){\rule{0.241pt}{0.400pt}}
\multiput(1087.00,636.17)(0.500,1.000){2}{\rule{0.120pt}{0.400pt}}
\put(1088,637.67){\rule{0.482pt}{0.400pt}}
\multiput(1088.00,637.17)(1.000,1.000){2}{\rule{0.241pt}{0.400pt}}
\put(1090,638.67){\rule{0.482pt}{0.400pt}}
\multiput(1090.00,638.17)(1.000,1.000){2}{\rule{0.241pt}{0.400pt}}
\put(1092,639.67){\rule{0.241pt}{0.400pt}}
\multiput(1092.00,639.17)(0.500,1.000){2}{\rule{0.120pt}{0.400pt}}
\put(1093,640.67){\rule{0.482pt}{0.400pt}}
\multiput(1093.00,640.17)(1.000,1.000){2}{\rule{0.241pt}{0.400pt}}
\put(1095,641.67){\rule{0.241pt}{0.400pt}}
\multiput(1095.00,641.17)(0.500,1.000){2}{\rule{0.120pt}{0.400pt}}
\put(1096,642.67){\rule{0.482pt}{0.400pt}}
\multiput(1096.00,642.17)(1.000,1.000){2}{\rule{0.241pt}{0.400pt}}
\put(1098,643.67){\rule{0.241pt}{0.400pt}}
\multiput(1098.00,643.17)(0.500,1.000){2}{\rule{0.120pt}{0.400pt}}
\put(1099,644.67){\rule{0.482pt}{0.400pt}}
\multiput(1099.00,644.17)(1.000,1.000){2}{\rule{0.241pt}{0.400pt}}
\put(1101,645.67){\rule{0.241pt}{0.400pt}}
\multiput(1101.00,645.17)(0.500,1.000){2}{\rule{0.120pt}{0.400pt}}
\put(1102,646.67){\rule{0.482pt}{0.400pt}}
\multiput(1102.00,646.17)(1.000,1.000){2}{\rule{0.241pt}{0.400pt}}
\put(1104,647.67){\rule{0.241pt}{0.400pt}}
\multiput(1104.00,647.17)(0.500,1.000){2}{\rule{0.120pt}{0.400pt}}
\put(1105,648.67){\rule{0.482pt}{0.400pt}}
\multiput(1105.00,648.17)(1.000,1.000){2}{\rule{0.241pt}{0.400pt}}
\put(1065.0,624.0){\rule[-0.200pt]{0.482pt}{0.400pt}}
\put(1108,649.67){\rule{0.482pt}{0.400pt}}
\multiput(1108.00,649.17)(1.000,1.000){2}{\rule{0.241pt}{0.400pt}}
\put(1110,650.67){\rule{0.482pt}{0.400pt}}
\multiput(1110.00,650.17)(1.000,1.000){2}{\rule{0.241pt}{0.400pt}}
\put(1112,651.67){\rule{0.241pt}{0.400pt}}
\multiput(1112.00,651.17)(0.500,1.000){2}{\rule{0.120pt}{0.400pt}}
\put(1113,652.67){\rule{0.482pt}{0.400pt}}
\multiput(1113.00,652.17)(1.000,1.000){2}{\rule{0.241pt}{0.400pt}}
\put(1115,653.67){\rule{0.241pt}{0.400pt}}
\multiput(1115.00,653.17)(0.500,1.000){2}{\rule{0.120pt}{0.400pt}}
\put(1116,654.67){\rule{0.482pt}{0.400pt}}
\multiput(1116.00,654.17)(1.000,1.000){2}{\rule{0.241pt}{0.400pt}}
\put(1118,655.67){\rule{0.241pt}{0.400pt}}
\multiput(1118.00,655.17)(0.500,1.000){2}{\rule{0.120pt}{0.400pt}}
\put(1119,656.67){\rule{0.482pt}{0.400pt}}
\multiput(1119.00,656.17)(1.000,1.000){2}{\rule{0.241pt}{0.400pt}}
\put(1121,657.67){\rule{0.241pt}{0.400pt}}
\multiput(1121.00,657.17)(0.500,1.000){2}{\rule{0.120pt}{0.400pt}}
\put(1122,658.67){\rule{0.482pt}{0.400pt}}
\multiput(1122.00,658.17)(1.000,1.000){2}{\rule{0.241pt}{0.400pt}}
\put(1124,659.67){\rule{0.241pt}{0.400pt}}
\multiput(1124.00,659.17)(0.500,1.000){2}{\rule{0.120pt}{0.400pt}}
\put(1125,660.67){\rule{0.482pt}{0.400pt}}
\multiput(1125.00,660.17)(1.000,1.000){2}{\rule{0.241pt}{0.400pt}}
\put(1127,661.67){\rule{0.241pt}{0.400pt}}
\multiput(1127.00,661.17)(0.500,1.000){2}{\rule{0.120pt}{0.400pt}}
\put(1128,662.67){\rule{0.482pt}{0.400pt}}
\multiput(1128.00,662.17)(1.000,1.000){2}{\rule{0.241pt}{0.400pt}}
\put(1130,663.67){\rule{0.482pt}{0.400pt}}
\multiput(1130.00,663.17)(1.000,1.000){2}{\rule{0.241pt}{0.400pt}}
\put(1132,664.67){\rule{0.241pt}{0.400pt}}
\multiput(1132.00,664.17)(0.500,1.000){2}{\rule{0.120pt}{0.400pt}}
\put(1133,665.67){\rule{0.482pt}{0.400pt}}
\multiput(1133.00,665.17)(1.000,1.000){2}{\rule{0.241pt}{0.400pt}}
\put(1135,666.67){\rule{0.241pt}{0.400pt}}
\multiput(1135.00,666.17)(0.500,1.000){2}{\rule{0.120pt}{0.400pt}}
\put(1136,667.67){\rule{0.482pt}{0.400pt}}
\multiput(1136.00,667.17)(1.000,1.000){2}{\rule{0.241pt}{0.400pt}}
\put(1138,668.67){\rule{0.241pt}{0.400pt}}
\multiput(1138.00,668.17)(0.500,1.000){2}{\rule{0.120pt}{0.400pt}}
\put(1139,669.67){\rule{0.482pt}{0.400pt}}
\multiput(1139.00,669.17)(1.000,1.000){2}{\rule{0.241pt}{0.400pt}}
\put(1141,670.67){\rule{0.241pt}{0.400pt}}
\multiput(1141.00,670.17)(0.500,1.000){2}{\rule{0.120pt}{0.400pt}}
\put(1142,671.67){\rule{0.482pt}{0.400pt}}
\multiput(1142.00,671.17)(1.000,1.000){2}{\rule{0.241pt}{0.400pt}}
\put(1144,672.67){\rule{0.241pt}{0.400pt}}
\multiput(1144.00,672.17)(0.500,1.000){2}{\rule{0.120pt}{0.400pt}}
\put(1145,673.67){\rule{0.482pt}{0.400pt}}
\multiput(1145.00,673.17)(1.000,1.000){2}{\rule{0.241pt}{0.400pt}}
\put(1147,674.67){\rule{0.482pt}{0.400pt}}
\multiput(1147.00,674.17)(1.000,1.000){2}{\rule{0.241pt}{0.400pt}}
\put(1149,675.67){\rule{0.241pt}{0.400pt}}
\multiput(1149.00,675.17)(0.500,1.000){2}{\rule{0.120pt}{0.400pt}}
\put(1107.0,650.0){\usebox{\plotpoint}}
\put(1152,676.67){\rule{0.241pt}{0.400pt}}
\multiput(1152.00,676.17)(0.500,1.000){2}{\rule{0.120pt}{0.400pt}}
\put(1153,677.67){\rule{0.482pt}{0.400pt}}
\multiput(1153.00,677.17)(1.000,1.000){2}{\rule{0.241pt}{0.400pt}}
\put(1155,678.67){\rule{0.241pt}{0.400pt}}
\multiput(1155.00,678.17)(0.500,1.000){2}{\rule{0.120pt}{0.400pt}}
\put(1156,679.67){\rule{0.482pt}{0.400pt}}
\multiput(1156.00,679.17)(1.000,1.000){2}{\rule{0.241pt}{0.400pt}}
\put(1158,680.67){\rule{0.241pt}{0.400pt}}
\multiput(1158.00,680.17)(0.500,1.000){2}{\rule{0.120pt}{0.400pt}}
\put(1159,681.67){\rule{0.482pt}{0.400pt}}
\multiput(1159.00,681.17)(1.000,1.000){2}{\rule{0.241pt}{0.400pt}}
\put(1161,682.67){\rule{0.241pt}{0.400pt}}
\multiput(1161.00,682.17)(0.500,1.000){2}{\rule{0.120pt}{0.400pt}}
\put(1162,683.67){\rule{0.482pt}{0.400pt}}
\multiput(1162.00,683.17)(1.000,1.000){2}{\rule{0.241pt}{0.400pt}}
\put(1164,684.67){\rule{0.241pt}{0.400pt}}
\multiput(1164.00,684.17)(0.500,1.000){2}{\rule{0.120pt}{0.400pt}}
\put(1165,685.67){\rule{0.482pt}{0.400pt}}
\multiput(1165.00,685.17)(1.000,1.000){2}{\rule{0.241pt}{0.400pt}}
\put(1167,686.67){\rule{0.482pt}{0.400pt}}
\multiput(1167.00,686.17)(1.000,1.000){2}{\rule{0.241pt}{0.400pt}}
\put(1169,687.67){\rule{0.241pt}{0.400pt}}
\multiput(1169.00,687.17)(0.500,1.000){2}{\rule{0.120pt}{0.400pt}}
\put(1170,688.67){\rule{0.482pt}{0.400pt}}
\multiput(1170.00,688.17)(1.000,1.000){2}{\rule{0.241pt}{0.400pt}}
\put(1172,689.67){\rule{0.241pt}{0.400pt}}
\multiput(1172.00,689.17)(0.500,1.000){2}{\rule{0.120pt}{0.400pt}}
\put(1173,690.67){\rule{0.482pt}{0.400pt}}
\multiput(1173.00,690.17)(1.000,1.000){2}{\rule{0.241pt}{0.400pt}}
\put(1175,691.67){\rule{0.241pt}{0.400pt}}
\multiput(1175.00,691.17)(0.500,1.000){2}{\rule{0.120pt}{0.400pt}}
\put(1176,692.67){\rule{0.482pt}{0.400pt}}
\multiput(1176.00,692.17)(1.000,1.000){2}{\rule{0.241pt}{0.400pt}}
\put(1178,693.67){\rule{0.241pt}{0.400pt}}
\multiput(1178.00,693.17)(0.500,1.000){2}{\rule{0.120pt}{0.400pt}}
\put(1179,694.67){\rule{0.482pt}{0.400pt}}
\multiput(1179.00,694.17)(1.000,1.000){2}{\rule{0.241pt}{0.400pt}}
\put(1181,695.67){\rule{0.241pt}{0.400pt}}
\multiput(1181.00,695.17)(0.500,1.000){2}{\rule{0.120pt}{0.400pt}}
\put(1182,696.67){\rule{0.482pt}{0.400pt}}
\multiput(1182.00,696.17)(1.000,1.000){2}{\rule{0.241pt}{0.400pt}}
\put(1184,697.67){\rule{0.241pt}{0.400pt}}
\multiput(1184.00,697.17)(0.500,1.000){2}{\rule{0.120pt}{0.400pt}}
\put(1185,698.67){\rule{0.482pt}{0.400pt}}
\multiput(1185.00,698.17)(1.000,1.000){2}{\rule{0.241pt}{0.400pt}}
\put(1187,699.67){\rule{0.482pt}{0.400pt}}
\multiput(1187.00,699.17)(1.000,1.000){2}{\rule{0.241pt}{0.400pt}}
\put(1189,700.67){\rule{0.241pt}{0.400pt}}
\multiput(1189.00,700.17)(0.500,1.000){2}{\rule{0.120pt}{0.400pt}}
\put(1190,701.67){\rule{0.482pt}{0.400pt}}
\multiput(1190.00,701.17)(1.000,1.000){2}{\rule{0.241pt}{0.400pt}}
\put(1150.0,677.0){\rule[-0.200pt]{0.482pt}{0.400pt}}
\put(1193,702.67){\rule{0.482pt}{0.400pt}}
\multiput(1193.00,702.17)(1.000,1.000){2}{\rule{0.241pt}{0.400pt}}
\put(1195,703.67){\rule{0.241pt}{0.400pt}}
\multiput(1195.00,703.17)(0.500,1.000){2}{\rule{0.120pt}{0.400pt}}
\put(1196,704.67){\rule{0.482pt}{0.400pt}}
\multiput(1196.00,704.17)(1.000,1.000){2}{\rule{0.241pt}{0.400pt}}
\put(1198,705.67){\rule{0.241pt}{0.400pt}}
\multiput(1198.00,705.17)(0.500,1.000){2}{\rule{0.120pt}{0.400pt}}
\put(1199,706.67){\rule{0.482pt}{0.400pt}}
\multiput(1199.00,706.17)(1.000,1.000){2}{\rule{0.241pt}{0.400pt}}
\put(1201,707.67){\rule{0.241pt}{0.400pt}}
\multiput(1201.00,707.17)(0.500,1.000){2}{\rule{0.120pt}{0.400pt}}
\put(1202,708.67){\rule{0.482pt}{0.400pt}}
\multiput(1202.00,708.17)(1.000,1.000){2}{\rule{0.241pt}{0.400pt}}
\put(1204,709.67){\rule{0.241pt}{0.400pt}}
\multiput(1204.00,709.17)(0.500,1.000){2}{\rule{0.120pt}{0.400pt}}
\put(1205,710.67){\rule{0.482pt}{0.400pt}}
\multiput(1205.00,710.17)(1.000,1.000){2}{\rule{0.241pt}{0.400pt}}
\put(1207,711.67){\rule{0.482pt}{0.400pt}}
\multiput(1207.00,711.17)(1.000,1.000){2}{\rule{0.241pt}{0.400pt}}
\put(1209,712.67){\rule{0.241pt}{0.400pt}}
\multiput(1209.00,712.17)(0.500,1.000){2}{\rule{0.120pt}{0.400pt}}
\put(1210,713.67){\rule{0.482pt}{0.400pt}}
\multiput(1210.00,713.17)(1.000,1.000){2}{\rule{0.241pt}{0.400pt}}
\put(1212,714.67){\rule{0.241pt}{0.400pt}}
\multiput(1212.00,714.17)(0.500,1.000){2}{\rule{0.120pt}{0.400pt}}
\put(1213,715.67){\rule{0.482pt}{0.400pt}}
\multiput(1213.00,715.17)(1.000,1.000){2}{\rule{0.241pt}{0.400pt}}
\put(1215,716.67){\rule{0.241pt}{0.400pt}}
\multiput(1215.00,716.17)(0.500,1.000){2}{\rule{0.120pt}{0.400pt}}
\put(1216,717.67){\rule{0.482pt}{0.400pt}}
\multiput(1216.00,717.17)(1.000,1.000){2}{\rule{0.241pt}{0.400pt}}
\put(1218,718.67){\rule{0.241pt}{0.400pt}}
\multiput(1218.00,718.17)(0.500,1.000){2}{\rule{0.120pt}{0.400pt}}
\put(1219,719.67){\rule{0.482pt}{0.400pt}}
\multiput(1219.00,719.17)(1.000,1.000){2}{\rule{0.241pt}{0.400pt}}
\put(1221,720.67){\rule{0.241pt}{0.400pt}}
\multiput(1221.00,720.17)(0.500,1.000){2}{\rule{0.120pt}{0.400pt}}
\put(1222,721.67){\rule{0.482pt}{0.400pt}}
\multiput(1222.00,721.17)(1.000,1.000){2}{\rule{0.241pt}{0.400pt}}
\put(1224,722.67){\rule{0.482pt}{0.400pt}}
\multiput(1224.00,722.17)(1.000,1.000){2}{\rule{0.241pt}{0.400pt}}
\put(1226,723.67){\rule{0.241pt}{0.400pt}}
\multiput(1226.00,723.17)(0.500,1.000){2}{\rule{0.120pt}{0.400pt}}
\put(1227,724.67){\rule{0.482pt}{0.400pt}}
\multiput(1227.00,724.17)(1.000,1.000){2}{\rule{0.241pt}{0.400pt}}
\put(1229,725.67){\rule{0.241pt}{0.400pt}}
\multiput(1229.00,725.17)(0.500,1.000){2}{\rule{0.120pt}{0.400pt}}
\put(1230,726.67){\rule{0.482pt}{0.400pt}}
\multiput(1230.00,726.17)(1.000,1.000){2}{\rule{0.241pt}{0.400pt}}
\put(1232,727.67){\rule{0.241pt}{0.400pt}}
\multiput(1232.00,727.17)(0.500,1.000){2}{\rule{0.120pt}{0.400pt}}
\put(1192.0,703.0){\usebox{\plotpoint}}
\put(1235,728.67){\rule{0.241pt}{0.400pt}}
\multiput(1235.00,728.17)(0.500,1.000){2}{\rule{0.120pt}{0.400pt}}
\put(1236,729.67){\rule{0.482pt}{0.400pt}}
\multiput(1236.00,729.17)(1.000,1.000){2}{\rule{0.241pt}{0.400pt}}
\put(1238,730.67){\rule{0.241pt}{0.400pt}}
\multiput(1238.00,730.17)(0.500,1.000){2}{\rule{0.120pt}{0.400pt}}
\put(1239,731.67){\rule{0.482pt}{0.400pt}}
\multiput(1239.00,731.17)(1.000,1.000){2}{\rule{0.241pt}{0.400pt}}
\put(1241,732.67){\rule{0.241pt}{0.400pt}}
\multiput(1241.00,732.17)(0.500,1.000){2}{\rule{0.120pt}{0.400pt}}
\put(1242,733.67){\rule{0.482pt}{0.400pt}}
\multiput(1242.00,733.17)(1.000,1.000){2}{\rule{0.241pt}{0.400pt}}
\put(1244,734.67){\rule{0.482pt}{0.400pt}}
\multiput(1244.00,734.17)(1.000,1.000){2}{\rule{0.241pt}{0.400pt}}
\put(1246,735.67){\rule{0.241pt}{0.400pt}}
\multiput(1246.00,735.17)(0.500,1.000){2}{\rule{0.120pt}{0.400pt}}
\put(1247,736.67){\rule{0.482pt}{0.400pt}}
\multiput(1247.00,736.17)(1.000,1.000){2}{\rule{0.241pt}{0.400pt}}
\put(1249,737.67){\rule{0.241pt}{0.400pt}}
\multiput(1249.00,737.17)(0.500,1.000){2}{\rule{0.120pt}{0.400pt}}
\put(1250,738.67){\rule{0.482pt}{0.400pt}}
\multiput(1250.00,738.17)(1.000,1.000){2}{\rule{0.241pt}{0.400pt}}
\put(1252,739.67){\rule{0.241pt}{0.400pt}}
\multiput(1252.00,739.17)(0.500,1.000){2}{\rule{0.120pt}{0.400pt}}
\put(1253,740.67){\rule{0.482pt}{0.400pt}}
\multiput(1253.00,740.17)(1.000,1.000){2}{\rule{0.241pt}{0.400pt}}
\put(1255,741.67){\rule{0.241pt}{0.400pt}}
\multiput(1255.00,741.17)(0.500,1.000){2}{\rule{0.120pt}{0.400pt}}
\put(1256,742.67){\rule{0.482pt}{0.400pt}}
\multiput(1256.00,742.17)(1.000,1.000){2}{\rule{0.241pt}{0.400pt}}
\put(1258,743.67){\rule{0.241pt}{0.400pt}}
\multiput(1258.00,743.17)(0.500,1.000){2}{\rule{0.120pt}{0.400pt}}
\put(1259,744.67){\rule{0.482pt}{0.400pt}}
\multiput(1259.00,744.17)(1.000,1.000){2}{\rule{0.241pt}{0.400pt}}
\put(1261,745.67){\rule{0.241pt}{0.400pt}}
\multiput(1261.00,745.17)(0.500,1.000){2}{\rule{0.120pt}{0.400pt}}
\put(1262,746.67){\rule{0.482pt}{0.400pt}}
\multiput(1262.00,746.17)(1.000,1.000){2}{\rule{0.241pt}{0.400pt}}
\put(1264,747.67){\rule{0.482pt}{0.400pt}}
\multiput(1264.00,747.17)(1.000,1.000){2}{\rule{0.241pt}{0.400pt}}
\put(1266,748.67){\rule{0.241pt}{0.400pt}}
\multiput(1266.00,748.17)(0.500,1.000){2}{\rule{0.120pt}{0.400pt}}
\put(1267,749.67){\rule{0.482pt}{0.400pt}}
\multiput(1267.00,749.17)(1.000,1.000){2}{\rule{0.241pt}{0.400pt}}
\put(1269,750.67){\rule{0.241pt}{0.400pt}}
\multiput(1269.00,750.17)(0.500,1.000){2}{\rule{0.120pt}{0.400pt}}
\put(1270,751.67){\rule{0.482pt}{0.400pt}}
\multiput(1270.00,751.17)(1.000,1.000){2}{\rule{0.241pt}{0.400pt}}
\put(1272,752.67){\rule{0.241pt}{0.400pt}}
\multiput(1272.00,752.17)(0.500,1.000){2}{\rule{0.120pt}{0.400pt}}
\put(1273,753.67){\rule{0.482pt}{0.400pt}}
\multiput(1273.00,753.17)(1.000,1.000){2}{\rule{0.241pt}{0.400pt}}
\put(1233.0,729.0){\rule[-0.200pt]{0.482pt}{0.400pt}}
\put(1276,754.67){\rule{0.482pt}{0.400pt}}
\multiput(1276.00,754.17)(1.000,1.000){2}{\rule{0.241pt}{0.400pt}}
\put(1278,755.67){\rule{0.241pt}{0.400pt}}
\multiput(1278.00,755.17)(0.500,1.000){2}{\rule{0.120pt}{0.400pt}}
\put(1279,756.67){\rule{0.482pt}{0.400pt}}
\multiput(1279.00,756.17)(1.000,1.000){2}{\rule{0.241pt}{0.400pt}}
\put(1281,757.67){\rule{0.241pt}{0.400pt}}
\multiput(1281.00,757.17)(0.500,1.000){2}{\rule{0.120pt}{0.400pt}}
\put(1282,758.67){\rule{0.482pt}{0.400pt}}
\multiput(1282.00,758.17)(1.000,1.000){2}{\rule{0.241pt}{0.400pt}}
\put(1284,759.67){\rule{0.482pt}{0.400pt}}
\multiput(1284.00,759.17)(1.000,1.000){2}{\rule{0.241pt}{0.400pt}}
\put(1286,760.67){\rule{0.241pt}{0.400pt}}
\multiput(1286.00,760.17)(0.500,1.000){2}{\rule{0.120pt}{0.400pt}}
\put(1287,761.67){\rule{0.482pt}{0.400pt}}
\multiput(1287.00,761.17)(1.000,1.000){2}{\rule{0.241pt}{0.400pt}}
\put(1289,762.67){\rule{0.241pt}{0.400pt}}
\multiput(1289.00,762.17)(0.500,1.000){2}{\rule{0.120pt}{0.400pt}}
\put(1290,763.67){\rule{0.482pt}{0.400pt}}
\multiput(1290.00,763.17)(1.000,1.000){2}{\rule{0.241pt}{0.400pt}}
\put(1292,764.67){\rule{0.241pt}{0.400pt}}
\multiput(1292.00,764.17)(0.500,1.000){2}{\rule{0.120pt}{0.400pt}}
\put(1293,765.67){\rule{0.482pt}{0.400pt}}
\multiput(1293.00,765.17)(1.000,1.000){2}{\rule{0.241pt}{0.400pt}}
\put(1295,766.67){\rule{0.241pt}{0.400pt}}
\multiput(1295.00,766.17)(0.500,1.000){2}{\rule{0.120pt}{0.400pt}}
\put(1296,767.67){\rule{0.482pt}{0.400pt}}
\multiput(1296.00,767.17)(1.000,1.000){2}{\rule{0.241pt}{0.400pt}}
\put(1298,768.67){\rule{0.241pt}{0.400pt}}
\multiput(1298.00,768.17)(0.500,1.000){2}{\rule{0.120pt}{0.400pt}}
\put(1299,769.67){\rule{0.482pt}{0.400pt}}
\multiput(1299.00,769.17)(1.000,1.000){2}{\rule{0.241pt}{0.400pt}}
\put(1301,770.67){\rule{0.241pt}{0.400pt}}
\multiput(1301.00,770.17)(0.500,1.000){2}{\rule{0.120pt}{0.400pt}}
\put(1302,771.67){\rule{0.482pt}{0.400pt}}
\multiput(1302.00,771.17)(1.000,1.000){2}{\rule{0.241pt}{0.400pt}}
\put(1304,772.67){\rule{0.482pt}{0.400pt}}
\multiput(1304.00,772.17)(1.000,1.000){2}{\rule{0.241pt}{0.400pt}}
\put(1306,773.67){\rule{0.241pt}{0.400pt}}
\multiput(1306.00,773.17)(0.500,1.000){2}{\rule{0.120pt}{0.400pt}}
\put(1307,774.67){\rule{0.482pt}{0.400pt}}
\multiput(1307.00,774.17)(1.000,1.000){2}{\rule{0.241pt}{0.400pt}}
\put(1309,775.67){\rule{0.241pt}{0.400pt}}
\multiput(1309.00,775.17)(0.500,1.000){2}{\rule{0.120pt}{0.400pt}}
\put(1310,776.67){\rule{0.482pt}{0.400pt}}
\multiput(1310.00,776.17)(1.000,1.000){2}{\rule{0.241pt}{0.400pt}}
\put(1312,777.67){\rule{0.241pt}{0.400pt}}
\multiput(1312.00,777.17)(0.500,1.000){2}{\rule{0.120pt}{0.400pt}}
\put(1313,778.67){\rule{0.482pt}{0.400pt}}
\multiput(1313.00,778.17)(1.000,1.000){2}{\rule{0.241pt}{0.400pt}}
\put(1315,779.67){\rule{0.241pt}{0.400pt}}
\multiput(1315.00,779.17)(0.500,1.000){2}{\rule{0.120pt}{0.400pt}}
\put(1316,780.67){\rule{0.482pt}{0.400pt}}
\multiput(1316.00,780.17)(1.000,1.000){2}{\rule{0.241pt}{0.400pt}}
\put(1275.0,755.0){\usebox{\plotpoint}}
\put(1319,781.67){\rule{0.482pt}{0.400pt}}
\multiput(1319.00,781.17)(1.000,1.000){2}{\rule{0.241pt}{0.400pt}}
\put(1321,782.67){\rule{0.482pt}{0.400pt}}
\multiput(1321.00,782.17)(1.000,1.000){2}{\rule{0.241pt}{0.400pt}}
\put(1323,783.67){\rule{0.241pt}{0.400pt}}
\multiput(1323.00,783.17)(0.500,1.000){2}{\rule{0.120pt}{0.400pt}}
\put(1324,784.67){\rule{0.482pt}{0.400pt}}
\multiput(1324.00,784.17)(1.000,1.000){2}{\rule{0.241pt}{0.400pt}}
\put(1326,785.67){\rule{0.241pt}{0.400pt}}
\multiput(1326.00,785.17)(0.500,1.000){2}{\rule{0.120pt}{0.400pt}}
\put(1327,786.67){\rule{0.482pt}{0.400pt}}
\multiput(1327.00,786.17)(1.000,1.000){2}{\rule{0.241pt}{0.400pt}}
\put(1329,787.67){\rule{0.241pt}{0.400pt}}
\multiput(1329.00,787.17)(0.500,1.000){2}{\rule{0.120pt}{0.400pt}}
\put(1330,788.67){\rule{0.482pt}{0.400pt}}
\multiput(1330.00,788.17)(1.000,1.000){2}{\rule{0.241pt}{0.400pt}}
\put(1332,789.67){\rule{0.241pt}{0.400pt}}
\multiput(1332.00,789.17)(0.500,1.000){2}{\rule{0.120pt}{0.400pt}}
\put(1333,790.67){\rule{0.482pt}{0.400pt}}
\multiput(1333.00,790.17)(1.000,1.000){2}{\rule{0.241pt}{0.400pt}}
\put(1335,791.67){\rule{0.241pt}{0.400pt}}
\multiput(1335.00,791.17)(0.500,1.000){2}{\rule{0.120pt}{0.400pt}}
\put(1336,792.67){\rule{0.482pt}{0.400pt}}
\multiput(1336.00,792.17)(1.000,1.000){2}{\rule{0.241pt}{0.400pt}}
\put(1338,793.67){\rule{0.241pt}{0.400pt}}
\multiput(1338.00,793.17)(0.500,1.000){2}{\rule{0.120pt}{0.400pt}}
\put(1339,794.67){\rule{0.482pt}{0.400pt}}
\multiput(1339.00,794.17)(1.000,1.000){2}{\rule{0.241pt}{0.400pt}}
\put(1341,795.67){\rule{0.482pt}{0.400pt}}
\multiput(1341.00,795.17)(1.000,1.000){2}{\rule{0.241pt}{0.400pt}}
\put(1343,796.67){\rule{0.241pt}{0.400pt}}
\multiput(1343.00,796.17)(0.500,1.000){2}{\rule{0.120pt}{0.400pt}}
\put(1344,797.67){\rule{0.482pt}{0.400pt}}
\multiput(1344.00,797.17)(1.000,1.000){2}{\rule{0.241pt}{0.400pt}}
\put(1346,798.67){\rule{0.241pt}{0.400pt}}
\multiput(1346.00,798.17)(0.500,1.000){2}{\rule{0.120pt}{0.400pt}}
\put(1347,799.67){\rule{0.482pt}{0.400pt}}
\multiput(1347.00,799.17)(1.000,1.000){2}{\rule{0.241pt}{0.400pt}}
\put(1349,800.67){\rule{0.241pt}{0.400pt}}
\multiput(1349.00,800.17)(0.500,1.000){2}{\rule{0.120pt}{0.400pt}}
\put(1350,801.67){\rule{0.482pt}{0.400pt}}
\multiput(1350.00,801.17)(1.000,1.000){2}{\rule{0.241pt}{0.400pt}}
\put(1352,802.67){\rule{0.241pt}{0.400pt}}
\multiput(1352.00,802.17)(0.500,1.000){2}{\rule{0.120pt}{0.400pt}}
\put(1353,803.67){\rule{0.482pt}{0.400pt}}
\multiput(1353.00,803.17)(1.000,1.000){2}{\rule{0.241pt}{0.400pt}}
\put(1355,804.67){\rule{0.241pt}{0.400pt}}
\multiput(1355.00,804.17)(0.500,1.000){2}{\rule{0.120pt}{0.400pt}}
\put(1356,805.67){\rule{0.482pt}{0.400pt}}
\multiput(1356.00,805.17)(1.000,1.000){2}{\rule{0.241pt}{0.400pt}}
\put(1358,806.67){\rule{0.241pt}{0.400pt}}
\multiput(1358.00,806.17)(0.500,1.000){2}{\rule{0.120pt}{0.400pt}}
\put(1318.0,782.0){\usebox{\plotpoint}}
\put(1361,807.67){\rule{0.482pt}{0.400pt}}
\multiput(1361.00,807.17)(1.000,1.000){2}{\rule{0.241pt}{0.400pt}}
\put(1363,808.67){\rule{0.241pt}{0.400pt}}
\multiput(1363.00,808.17)(0.500,1.000){2}{\rule{0.120pt}{0.400pt}}
\put(1364,809.67){\rule{0.482pt}{0.400pt}}
\multiput(1364.00,809.17)(1.000,1.000){2}{\rule{0.241pt}{0.400pt}}
\put(1366,810.67){\rule{0.241pt}{0.400pt}}
\multiput(1366.00,810.17)(0.500,1.000){2}{\rule{0.120pt}{0.400pt}}
\put(1367,811.67){\rule{0.482pt}{0.400pt}}
\multiput(1367.00,811.17)(1.000,1.000){2}{\rule{0.241pt}{0.400pt}}
\put(1369,812.67){\rule{0.241pt}{0.400pt}}
\multiput(1369.00,812.17)(0.500,1.000){2}{\rule{0.120pt}{0.400pt}}
\put(1370,813.67){\rule{0.482pt}{0.400pt}}
\multiput(1370.00,813.17)(1.000,1.000){2}{\rule{0.241pt}{0.400pt}}
\put(1372,814.67){\rule{0.241pt}{0.400pt}}
\multiput(1372.00,814.17)(0.500,1.000){2}{\rule{0.120pt}{0.400pt}}
\put(1373,815.67){\rule{0.482pt}{0.400pt}}
\multiput(1373.00,815.17)(1.000,1.000){2}{\rule{0.241pt}{0.400pt}}
\put(1375,816.67){\rule{0.241pt}{0.400pt}}
\multiput(1375.00,816.17)(0.500,1.000){2}{\rule{0.120pt}{0.400pt}}
\put(1376,817.67){\rule{0.482pt}{0.400pt}}
\multiput(1376.00,817.17)(1.000,1.000){2}{\rule{0.241pt}{0.400pt}}
\put(1378,818.67){\rule{0.241pt}{0.400pt}}
\multiput(1378.00,818.17)(0.500,1.000){2}{\rule{0.120pt}{0.400pt}}
\put(1379,819.67){\rule{0.482pt}{0.400pt}}
\multiput(1379.00,819.17)(1.000,1.000){2}{\rule{0.241pt}{0.400pt}}
\put(1381,820.67){\rule{0.482pt}{0.400pt}}
\multiput(1381.00,820.17)(1.000,1.000){2}{\rule{0.241pt}{0.400pt}}
\put(1383,821.67){\rule{0.241pt}{0.400pt}}
\multiput(1383.00,821.17)(0.500,1.000){2}{\rule{0.120pt}{0.400pt}}
\put(1384,822.67){\rule{0.482pt}{0.400pt}}
\multiput(1384.00,822.17)(1.000,1.000){2}{\rule{0.241pt}{0.400pt}}
\put(1386,823.67){\rule{0.241pt}{0.400pt}}
\multiput(1386.00,823.17)(0.500,1.000){2}{\rule{0.120pt}{0.400pt}}
\put(1387,824.67){\rule{0.482pt}{0.400pt}}
\multiput(1387.00,824.17)(1.000,1.000){2}{\rule{0.241pt}{0.400pt}}
\put(1389,825.67){\rule{0.241pt}{0.400pt}}
\multiput(1389.00,825.17)(0.500,1.000){2}{\rule{0.120pt}{0.400pt}}
\put(1390,826.67){\rule{0.482pt}{0.400pt}}
\multiput(1390.00,826.17)(1.000,1.000){2}{\rule{0.241pt}{0.400pt}}
\put(1392,827.67){\rule{0.241pt}{0.400pt}}
\multiput(1392.00,827.17)(0.500,1.000){2}{\rule{0.120pt}{0.400pt}}
\put(1393,828.67){\rule{0.482pt}{0.400pt}}
\multiput(1393.00,828.17)(1.000,1.000){2}{\rule{0.241pt}{0.400pt}}
\put(1395,829.67){\rule{0.241pt}{0.400pt}}
\multiput(1395.00,829.17)(0.500,1.000){2}{\rule{0.120pt}{0.400pt}}
\put(1396,830.67){\rule{0.482pt}{0.400pt}}
\multiput(1396.00,830.17)(1.000,1.000){2}{\rule{0.241pt}{0.400pt}}
\put(1398,831.67){\rule{0.482pt}{0.400pt}}
\multiput(1398.00,831.17)(1.000,1.000){2}{\rule{0.241pt}{0.400pt}}
\put(1400,832.67){\rule{0.241pt}{0.400pt}}
\multiput(1400.00,832.17)(0.500,1.000){2}{\rule{0.120pt}{0.400pt}}
\put(1359.0,808.0){\rule[-0.200pt]{0.482pt}{0.400pt}}
\put(1403,833.67){\rule{0.241pt}{0.400pt}}
\multiput(1403.00,833.17)(0.500,1.000){2}{\rule{0.120pt}{0.400pt}}
\put(1404,834.67){\rule{0.482pt}{0.400pt}}
\multiput(1404.00,834.17)(1.000,1.000){2}{\rule{0.241pt}{0.400pt}}
\put(1406,834.67){\rule{0.241pt}{0.400pt}}
\multiput(1406.00,835.17)(0.500,-1.000){2}{\rule{0.120pt}{0.400pt}}
\put(1407,833.67){\rule{0.482pt}{0.400pt}}
\multiput(1407.00,834.17)(1.000,-1.000){2}{\rule{0.241pt}{0.400pt}}
\put(1401.0,834.0){\rule[-0.200pt]{0.482pt}{0.400pt}}
\put(1410,832.67){\rule{0.482pt}{0.400pt}}
\multiput(1410.00,833.17)(1.000,-1.000){2}{\rule{0.241pt}{0.400pt}}
\put(1412,831.67){\rule{0.241pt}{0.400pt}}
\multiput(1412.00,832.17)(0.500,-1.000){2}{\rule{0.120pt}{0.400pt}}
\put(1412,830.67){\rule{0.241pt}{0.400pt}}
\multiput(1412.50,831.17)(-0.500,-1.000){2}{\rule{0.120pt}{0.400pt}}
\put(1410,829.67){\rule{0.482pt}{0.400pt}}
\multiput(1411.00,830.17)(-1.000,-1.000){2}{\rule{0.241pt}{0.400pt}}
\put(1409,828.67){\rule{0.241pt}{0.400pt}}
\multiput(1409.50,829.17)(-0.500,-1.000){2}{\rule{0.120pt}{0.400pt}}
\put(1407,827.67){\rule{0.482pt}{0.400pt}}
\multiput(1408.00,828.17)(-1.000,-1.000){2}{\rule{0.241pt}{0.400pt}}
\put(1406,826.67){\rule{0.241pt}{0.400pt}}
\multiput(1406.50,827.17)(-0.500,-1.000){2}{\rule{0.120pt}{0.400pt}}
\put(1404,825.67){\rule{0.482pt}{0.400pt}}
\multiput(1405.00,826.17)(-1.000,-1.000){2}{\rule{0.241pt}{0.400pt}}
\put(1403,824.67){\rule{0.241pt}{0.400pt}}
\multiput(1403.50,825.17)(-0.500,-1.000){2}{\rule{0.120pt}{0.400pt}}
\put(1401,823.67){\rule{0.482pt}{0.400pt}}
\multiput(1402.00,824.17)(-1.000,-1.000){2}{\rule{0.241pt}{0.400pt}}
\put(1400,822.67){\rule{0.241pt}{0.400pt}}
\multiput(1400.50,823.17)(-0.500,-1.000){2}{\rule{0.120pt}{0.400pt}}
\put(1398,821.67){\rule{0.482pt}{0.400pt}}
\multiput(1399.00,822.17)(-1.000,-1.000){2}{\rule{0.241pt}{0.400pt}}
\put(1396,820.67){\rule{0.482pt}{0.400pt}}
\multiput(1397.00,821.17)(-1.000,-1.000){2}{\rule{0.241pt}{0.400pt}}
\put(1395,819.67){\rule{0.241pt}{0.400pt}}
\multiput(1395.50,820.17)(-0.500,-1.000){2}{\rule{0.120pt}{0.400pt}}
\put(1393,818.67){\rule{0.482pt}{0.400pt}}
\multiput(1394.00,819.17)(-1.000,-1.000){2}{\rule{0.241pt}{0.400pt}}
\put(1392,817.67){\rule{0.241pt}{0.400pt}}
\multiput(1392.50,818.17)(-0.500,-1.000){2}{\rule{0.120pt}{0.400pt}}
\put(1390,816.67){\rule{0.482pt}{0.400pt}}
\multiput(1391.00,817.17)(-1.000,-1.000){2}{\rule{0.241pt}{0.400pt}}
\put(1389,815.67){\rule{0.241pt}{0.400pt}}
\multiput(1389.50,816.17)(-0.500,-1.000){2}{\rule{0.120pt}{0.400pt}}
\put(1387,814.67){\rule{0.482pt}{0.400pt}}
\multiput(1388.00,815.17)(-1.000,-1.000){2}{\rule{0.241pt}{0.400pt}}
\put(1386,813.67){\rule{0.241pt}{0.400pt}}
\multiput(1386.50,814.17)(-0.500,-1.000){2}{\rule{0.120pt}{0.400pt}}
\put(1384,812.67){\rule{0.482pt}{0.400pt}}
\multiput(1385.00,813.17)(-1.000,-1.000){2}{\rule{0.241pt}{0.400pt}}
\put(1383,811.67){\rule{0.241pt}{0.400pt}}
\multiput(1383.50,812.17)(-0.500,-1.000){2}{\rule{0.120pt}{0.400pt}}
\put(1381,810.67){\rule{0.482pt}{0.400pt}}
\multiput(1382.00,811.17)(-1.000,-1.000){2}{\rule{0.241pt}{0.400pt}}
\put(1379,809.67){\rule{0.482pt}{0.400pt}}
\multiput(1380.00,810.17)(-1.000,-1.000){2}{\rule{0.241pt}{0.400pt}}
\put(1378,808.67){\rule{0.241pt}{0.400pt}}
\multiput(1378.50,809.17)(-0.500,-1.000){2}{\rule{0.120pt}{0.400pt}}
\put(1376,807.67){\rule{0.482pt}{0.400pt}}
\multiput(1377.00,808.17)(-1.000,-1.000){2}{\rule{0.241pt}{0.400pt}}
\put(1409.0,834.0){\usebox{\plotpoint}}
\put(1373,806.67){\rule{0.482pt}{0.400pt}}
\multiput(1374.00,807.17)(-1.000,-1.000){2}{\rule{0.241pt}{0.400pt}}
\put(1372,805.67){\rule{0.241pt}{0.400pt}}
\multiput(1372.50,806.17)(-0.500,-1.000){2}{\rule{0.120pt}{0.400pt}}
\put(1370,804.67){\rule{0.482pt}{0.400pt}}
\multiput(1371.00,805.17)(-1.000,-1.000){2}{\rule{0.241pt}{0.400pt}}
\put(1369,803.67){\rule{0.241pt}{0.400pt}}
\multiput(1369.50,804.17)(-0.500,-1.000){2}{\rule{0.120pt}{0.400pt}}
\put(1367,802.67){\rule{0.482pt}{0.400pt}}
\multiput(1368.00,803.17)(-1.000,-1.000){2}{\rule{0.241pt}{0.400pt}}
\put(1366,801.67){\rule{0.241pt}{0.400pt}}
\multiput(1366.50,802.17)(-0.500,-1.000){2}{\rule{0.120pt}{0.400pt}}
\put(1364,800.67){\rule{0.482pt}{0.400pt}}
\multiput(1365.00,801.17)(-1.000,-1.000){2}{\rule{0.241pt}{0.400pt}}
\put(1363,799.67){\rule{0.241pt}{0.400pt}}
\multiput(1363.50,800.17)(-0.500,-1.000){2}{\rule{0.120pt}{0.400pt}}
\put(1361,798.67){\rule{0.482pt}{0.400pt}}
\multiput(1362.00,799.17)(-1.000,-1.000){2}{\rule{0.241pt}{0.400pt}}
\put(1359,797.67){\rule{0.482pt}{0.400pt}}
\multiput(1360.00,798.17)(-1.000,-1.000){2}{\rule{0.241pt}{0.400pt}}
\put(1358,796.67){\rule{0.241pt}{0.400pt}}
\multiput(1358.50,797.17)(-0.500,-1.000){2}{\rule{0.120pt}{0.400pt}}
\put(1356,795.67){\rule{0.482pt}{0.400pt}}
\multiput(1357.00,796.17)(-1.000,-1.000){2}{\rule{0.241pt}{0.400pt}}
\put(1355,794.67){\rule{0.241pt}{0.400pt}}
\multiput(1355.50,795.17)(-0.500,-1.000){2}{\rule{0.120pt}{0.400pt}}
\put(1353,793.67){\rule{0.482pt}{0.400pt}}
\multiput(1354.00,794.17)(-1.000,-1.000){2}{\rule{0.241pt}{0.400pt}}
\put(1352,792.67){\rule{0.241pt}{0.400pt}}
\multiput(1352.50,793.17)(-0.500,-1.000){2}{\rule{0.120pt}{0.400pt}}
\put(1350,791.67){\rule{0.482pt}{0.400pt}}
\multiput(1351.00,792.17)(-1.000,-1.000){2}{\rule{0.241pt}{0.400pt}}
\put(1349,790.67){\rule{0.241pt}{0.400pt}}
\multiput(1349.50,791.17)(-0.500,-1.000){2}{\rule{0.120pt}{0.400pt}}
\put(1347,789.67){\rule{0.482pt}{0.400pt}}
\multiput(1348.00,790.17)(-1.000,-1.000){2}{\rule{0.241pt}{0.400pt}}
\put(1346,788.67){\rule{0.241pt}{0.400pt}}
\multiput(1346.50,789.17)(-0.500,-1.000){2}{\rule{0.120pt}{0.400pt}}
\put(1344,787.67){\rule{0.482pt}{0.400pt}}
\multiput(1345.00,788.17)(-1.000,-1.000){2}{\rule{0.241pt}{0.400pt}}
\put(1343,786.67){\rule{0.241pt}{0.400pt}}
\multiput(1343.50,787.17)(-0.500,-1.000){2}{\rule{0.120pt}{0.400pt}}
\put(1341,785.67){\rule{0.482pt}{0.400pt}}
\multiput(1342.00,786.17)(-1.000,-1.000){2}{\rule{0.241pt}{0.400pt}}
\put(1339,784.67){\rule{0.482pt}{0.400pt}}
\multiput(1340.00,785.17)(-1.000,-1.000){2}{\rule{0.241pt}{0.400pt}}
\put(1338,783.67){\rule{0.241pt}{0.400pt}}
\multiput(1338.50,784.17)(-0.500,-1.000){2}{\rule{0.120pt}{0.400pt}}
\put(1336,782.67){\rule{0.482pt}{0.400pt}}
\multiput(1337.00,783.17)(-1.000,-1.000){2}{\rule{0.241pt}{0.400pt}}
\put(1335,781.67){\rule{0.241pt}{0.400pt}}
\multiput(1335.50,782.17)(-0.500,-1.000){2}{\rule{0.120pt}{0.400pt}}
\put(1375.0,808.0){\usebox{\plotpoint}}
\put(1332,780.67){\rule{0.241pt}{0.400pt}}
\multiput(1332.50,781.17)(-0.500,-1.000){2}{\rule{0.120pt}{0.400pt}}
\put(1330,779.67){\rule{0.482pt}{0.400pt}}
\multiput(1331.00,780.17)(-1.000,-1.000){2}{\rule{0.241pt}{0.400pt}}
\put(1329,778.67){\rule{0.241pt}{0.400pt}}
\multiput(1329.50,779.17)(-0.500,-1.000){2}{\rule{0.120pt}{0.400pt}}
\put(1327,777.67){\rule{0.482pt}{0.400pt}}
\multiput(1328.00,778.17)(-1.000,-1.000){2}{\rule{0.241pt}{0.400pt}}
\put(1326,776.67){\rule{0.241pt}{0.400pt}}
\multiput(1326.50,777.17)(-0.500,-1.000){2}{\rule{0.120pt}{0.400pt}}
\put(1324,775.67){\rule{0.482pt}{0.400pt}}
\multiput(1325.00,776.17)(-1.000,-1.000){2}{\rule{0.241pt}{0.400pt}}
\put(1323,774.67){\rule{0.241pt}{0.400pt}}
\multiput(1323.50,775.17)(-0.500,-1.000){2}{\rule{0.120pt}{0.400pt}}
\put(1321,773.67){\rule{0.482pt}{0.400pt}}
\multiput(1322.00,774.17)(-1.000,-1.000){2}{\rule{0.241pt}{0.400pt}}
\put(1319,772.67){\rule{0.482pt}{0.400pt}}
\multiput(1320.00,773.17)(-1.000,-1.000){2}{\rule{0.241pt}{0.400pt}}
\put(1318,771.67){\rule{0.241pt}{0.400pt}}
\multiput(1318.50,772.17)(-0.500,-1.000){2}{\rule{0.120pt}{0.400pt}}
\put(1316,770.67){\rule{0.482pt}{0.400pt}}
\multiput(1317.00,771.17)(-1.000,-1.000){2}{\rule{0.241pt}{0.400pt}}
\put(1315,769.67){\rule{0.241pt}{0.400pt}}
\multiput(1315.50,770.17)(-0.500,-1.000){2}{\rule{0.120pt}{0.400pt}}
\put(1313,768.67){\rule{0.482pt}{0.400pt}}
\multiput(1314.00,769.17)(-1.000,-1.000){2}{\rule{0.241pt}{0.400pt}}
\put(1312,767.67){\rule{0.241pt}{0.400pt}}
\multiput(1312.50,768.17)(-0.500,-1.000){2}{\rule{0.120pt}{0.400pt}}
\put(1310,766.67){\rule{0.482pt}{0.400pt}}
\multiput(1311.00,767.17)(-1.000,-1.000){2}{\rule{0.241pt}{0.400pt}}
\put(1309,765.67){\rule{0.241pt}{0.400pt}}
\multiput(1309.50,766.17)(-0.500,-1.000){2}{\rule{0.120pt}{0.400pt}}
\put(1307,764.67){\rule{0.482pt}{0.400pt}}
\multiput(1308.00,765.17)(-1.000,-1.000){2}{\rule{0.241pt}{0.400pt}}
\put(1306,763.67){\rule{0.241pt}{0.400pt}}
\multiput(1306.50,764.17)(-0.500,-1.000){2}{\rule{0.120pt}{0.400pt}}
\put(1304,762.67){\rule{0.482pt}{0.400pt}}
\multiput(1305.00,763.17)(-1.000,-1.000){2}{\rule{0.241pt}{0.400pt}}
\put(1302,761.67){\rule{0.482pt}{0.400pt}}
\multiput(1303.00,762.17)(-1.000,-1.000){2}{\rule{0.241pt}{0.400pt}}
\put(1301,760.67){\rule{0.241pt}{0.400pt}}
\multiput(1301.50,761.17)(-0.500,-1.000){2}{\rule{0.120pt}{0.400pt}}
\put(1299,759.67){\rule{0.482pt}{0.400pt}}
\multiput(1300.00,760.17)(-1.000,-1.000){2}{\rule{0.241pt}{0.400pt}}
\put(1298,758.67){\rule{0.241pt}{0.400pt}}
\multiput(1298.50,759.17)(-0.500,-1.000){2}{\rule{0.120pt}{0.400pt}}
\put(1296,757.67){\rule{0.482pt}{0.400pt}}
\multiput(1297.00,758.17)(-1.000,-1.000){2}{\rule{0.241pt}{0.400pt}}
\put(1295,756.67){\rule{0.241pt}{0.400pt}}
\multiput(1295.50,757.17)(-0.500,-1.000){2}{\rule{0.120pt}{0.400pt}}
\put(1293,755.67){\rule{0.482pt}{0.400pt}}
\multiput(1294.00,756.17)(-1.000,-1.000){2}{\rule{0.241pt}{0.400pt}}
\put(1292,754.67){\rule{0.241pt}{0.400pt}}
\multiput(1292.50,755.17)(-0.500,-1.000){2}{\rule{0.120pt}{0.400pt}}
\put(1333.0,782.0){\rule[-0.200pt]{0.482pt}{0.400pt}}
\put(1289,753.67){\rule{0.241pt}{0.400pt}}
\multiput(1289.50,754.17)(-0.500,-1.000){2}{\rule{0.120pt}{0.400pt}}
\put(1287,752.67){\rule{0.482pt}{0.400pt}}
\multiput(1288.00,753.17)(-1.000,-1.000){2}{\rule{0.241pt}{0.400pt}}
\put(1286,751.67){\rule{0.241pt}{0.400pt}}
\multiput(1286.50,752.17)(-0.500,-1.000){2}{\rule{0.120pt}{0.400pt}}
\put(1284,750.67){\rule{0.482pt}{0.400pt}}
\multiput(1285.00,751.17)(-1.000,-1.000){2}{\rule{0.241pt}{0.400pt}}
\put(1282,749.67){\rule{0.482pt}{0.400pt}}
\multiput(1283.00,750.17)(-1.000,-1.000){2}{\rule{0.241pt}{0.400pt}}
\put(1281,748.67){\rule{0.241pt}{0.400pt}}
\multiput(1281.50,749.17)(-0.500,-1.000){2}{\rule{0.120pt}{0.400pt}}
\put(1279,747.67){\rule{0.482pt}{0.400pt}}
\multiput(1280.00,748.17)(-1.000,-1.000){2}{\rule{0.241pt}{0.400pt}}
\put(1278,746.67){\rule{0.241pt}{0.400pt}}
\multiput(1278.50,747.17)(-0.500,-1.000){2}{\rule{0.120pt}{0.400pt}}
\put(1276,745.67){\rule{0.482pt}{0.400pt}}
\multiput(1277.00,746.17)(-1.000,-1.000){2}{\rule{0.241pt}{0.400pt}}
\put(1275,744.67){\rule{0.241pt}{0.400pt}}
\multiput(1275.50,745.17)(-0.500,-1.000){2}{\rule{0.120pt}{0.400pt}}
\put(1273,743.67){\rule{0.482pt}{0.400pt}}
\multiput(1274.00,744.17)(-1.000,-1.000){2}{\rule{0.241pt}{0.400pt}}
\put(1272,742.67){\rule{0.241pt}{0.400pt}}
\multiput(1272.50,743.17)(-0.500,-1.000){2}{\rule{0.120pt}{0.400pt}}
\put(1270,741.67){\rule{0.482pt}{0.400pt}}
\multiput(1271.00,742.17)(-1.000,-1.000){2}{\rule{0.241pt}{0.400pt}}
\put(1269,740.67){\rule{0.241pt}{0.400pt}}
\multiput(1269.50,741.17)(-0.500,-1.000){2}{\rule{0.120pt}{0.400pt}}
\put(1267,739.67){\rule{0.482pt}{0.400pt}}
\multiput(1268.00,740.17)(-1.000,-1.000){2}{\rule{0.241pt}{0.400pt}}
\put(1266,738.67){\rule{0.241pt}{0.400pt}}
\multiput(1266.50,739.17)(-0.500,-1.000){2}{\rule{0.120pt}{0.400pt}}
\put(1264,737.67){\rule{0.482pt}{0.400pt}}
\multiput(1265.00,738.17)(-1.000,-1.000){2}{\rule{0.241pt}{0.400pt}}
\put(1262,736.67){\rule{0.482pt}{0.400pt}}
\multiput(1263.00,737.17)(-1.000,-1.000){2}{\rule{0.241pt}{0.400pt}}
\put(1261,735.67){\rule{0.241pt}{0.400pt}}
\multiput(1261.50,736.17)(-0.500,-1.000){2}{\rule{0.120pt}{0.400pt}}
\put(1259,734.67){\rule{0.482pt}{0.400pt}}
\multiput(1260.00,735.17)(-1.000,-1.000){2}{\rule{0.241pt}{0.400pt}}
\put(1258,733.67){\rule{0.241pt}{0.400pt}}
\multiput(1258.50,734.17)(-0.500,-1.000){2}{\rule{0.120pt}{0.400pt}}
\put(1256,732.67){\rule{0.482pt}{0.400pt}}
\multiput(1257.00,733.17)(-1.000,-1.000){2}{\rule{0.241pt}{0.400pt}}
\put(1255,731.67){\rule{0.241pt}{0.400pt}}
\multiput(1255.50,732.17)(-0.500,-1.000){2}{\rule{0.120pt}{0.400pt}}
\put(1253,730.67){\rule{0.482pt}{0.400pt}}
\multiput(1254.00,731.17)(-1.000,-1.000){2}{\rule{0.241pt}{0.400pt}}
\put(1252,729.67){\rule{0.241pt}{0.400pt}}
\multiput(1252.50,730.17)(-0.500,-1.000){2}{\rule{0.120pt}{0.400pt}}
\put(1250,728.67){\rule{0.482pt}{0.400pt}}
\multiput(1251.00,729.17)(-1.000,-1.000){2}{\rule{0.241pt}{0.400pt}}
\put(1290.0,755.0){\rule[-0.200pt]{0.482pt}{0.400pt}}
\put(1247,727.67){\rule{0.482pt}{0.400pt}}
\multiput(1248.00,728.17)(-1.000,-1.000){2}{\rule{0.241pt}{0.400pt}}
\put(1246,726.67){\rule{0.241pt}{0.400pt}}
\multiput(1246.50,727.17)(-0.500,-1.000){2}{\rule{0.120pt}{0.400pt}}
\put(1244,725.67){\rule{0.482pt}{0.400pt}}
\multiput(1245.00,726.17)(-1.000,-1.000){2}{\rule{0.241pt}{0.400pt}}
\put(1242,724.67){\rule{0.482pt}{0.400pt}}
\multiput(1243.00,725.17)(-1.000,-1.000){2}{\rule{0.241pt}{0.400pt}}
\put(1241,723.67){\rule{0.241pt}{0.400pt}}
\multiput(1241.50,724.17)(-0.500,-1.000){2}{\rule{0.120pt}{0.400pt}}
\put(1239,722.67){\rule{0.482pt}{0.400pt}}
\multiput(1240.00,723.17)(-1.000,-1.000){2}{\rule{0.241pt}{0.400pt}}
\put(1238,721.67){\rule{0.241pt}{0.400pt}}
\multiput(1238.50,722.17)(-0.500,-1.000){2}{\rule{0.120pt}{0.400pt}}
\put(1236,720.67){\rule{0.482pt}{0.400pt}}
\multiput(1237.00,721.17)(-1.000,-1.000){2}{\rule{0.241pt}{0.400pt}}
\put(1235,719.67){\rule{0.241pt}{0.400pt}}
\multiput(1235.50,720.17)(-0.500,-1.000){2}{\rule{0.120pt}{0.400pt}}
\put(1233,718.67){\rule{0.482pt}{0.400pt}}
\multiput(1234.00,719.17)(-1.000,-1.000){2}{\rule{0.241pt}{0.400pt}}
\put(1232,717.67){\rule{0.241pt}{0.400pt}}
\multiput(1232.50,718.17)(-0.500,-1.000){2}{\rule{0.120pt}{0.400pt}}
\put(1230,716.67){\rule{0.482pt}{0.400pt}}
\multiput(1231.00,717.17)(-1.000,-1.000){2}{\rule{0.241pt}{0.400pt}}
\put(1229,715.67){\rule{0.241pt}{0.400pt}}
\multiput(1229.50,716.17)(-0.500,-1.000){2}{\rule{0.120pt}{0.400pt}}
\put(1227,714.67){\rule{0.482pt}{0.400pt}}
\multiput(1228.00,715.17)(-1.000,-1.000){2}{\rule{0.241pt}{0.400pt}}
\put(1226,713.67){\rule{0.241pt}{0.400pt}}
\multiput(1226.50,714.17)(-0.500,-1.000){2}{\rule{0.120pt}{0.400pt}}
\put(1224,712.67){\rule{0.482pt}{0.400pt}}
\multiput(1225.00,713.17)(-1.000,-1.000){2}{\rule{0.241pt}{0.400pt}}
\put(1222,711.67){\rule{0.482pt}{0.400pt}}
\multiput(1223.00,712.17)(-1.000,-1.000){2}{\rule{0.241pt}{0.400pt}}
\put(1221,710.67){\rule{0.241pt}{0.400pt}}
\multiput(1221.50,711.17)(-0.500,-1.000){2}{\rule{0.120pt}{0.400pt}}
\put(1219,709.67){\rule{0.482pt}{0.400pt}}
\multiput(1220.00,710.17)(-1.000,-1.000){2}{\rule{0.241pt}{0.400pt}}
\put(1218,708.67){\rule{0.241pt}{0.400pt}}
\multiput(1218.50,709.17)(-0.500,-1.000){2}{\rule{0.120pt}{0.400pt}}
\put(1216,707.67){\rule{0.482pt}{0.400pt}}
\multiput(1217.00,708.17)(-1.000,-1.000){2}{\rule{0.241pt}{0.400pt}}
\put(1215,706.67){\rule{0.241pt}{0.400pt}}
\multiput(1215.50,707.17)(-0.500,-1.000){2}{\rule{0.120pt}{0.400pt}}
\put(1213,705.67){\rule{0.482pt}{0.400pt}}
\multiput(1214.00,706.17)(-1.000,-1.000){2}{\rule{0.241pt}{0.400pt}}
\put(1212,704.67){\rule{0.241pt}{0.400pt}}
\multiput(1212.50,705.17)(-0.500,-1.000){2}{\rule{0.120pt}{0.400pt}}
\put(1210,703.67){\rule{0.482pt}{0.400pt}}
\multiput(1211.00,704.17)(-1.000,-1.000){2}{\rule{0.241pt}{0.400pt}}
\put(1209,702.67){\rule{0.241pt}{0.400pt}}
\multiput(1209.50,703.17)(-0.500,-1.000){2}{\rule{0.120pt}{0.400pt}}
\put(1249.0,729.0){\usebox{\plotpoint}}
\put(1205,701.67){\rule{0.482pt}{0.400pt}}
\multiput(1206.00,702.17)(-1.000,-1.000){2}{\rule{0.241pt}{0.400pt}}
\put(1204,700.67){\rule{0.241pt}{0.400pt}}
\multiput(1204.50,701.17)(-0.500,-1.000){2}{\rule{0.120pt}{0.400pt}}
\put(1202,699.67){\rule{0.482pt}{0.400pt}}
\multiput(1203.00,700.17)(-1.000,-1.000){2}{\rule{0.241pt}{0.400pt}}
\put(1201,698.67){\rule{0.241pt}{0.400pt}}
\multiput(1201.50,699.17)(-0.500,-1.000){2}{\rule{0.120pt}{0.400pt}}
\put(1199,697.67){\rule{0.482pt}{0.400pt}}
\multiput(1200.00,698.17)(-1.000,-1.000){2}{\rule{0.241pt}{0.400pt}}
\put(1198,696.67){\rule{0.241pt}{0.400pt}}
\multiput(1198.50,697.17)(-0.500,-1.000){2}{\rule{0.120pt}{0.400pt}}
\put(1196,695.67){\rule{0.482pt}{0.400pt}}
\multiput(1197.00,696.17)(-1.000,-1.000){2}{\rule{0.241pt}{0.400pt}}
\put(1195,694.67){\rule{0.241pt}{0.400pt}}
\multiput(1195.50,695.17)(-0.500,-1.000){2}{\rule{0.120pt}{0.400pt}}
\put(1193,693.67){\rule{0.482pt}{0.400pt}}
\multiput(1194.00,694.17)(-1.000,-1.000){2}{\rule{0.241pt}{0.400pt}}
\put(1192,692.67){\rule{0.241pt}{0.400pt}}
\multiput(1192.50,693.17)(-0.500,-1.000){2}{\rule{0.120pt}{0.400pt}}
\put(1190,691.67){\rule{0.482pt}{0.400pt}}
\multiput(1191.00,692.17)(-1.000,-1.000){2}{\rule{0.241pt}{0.400pt}}
\put(1189,690.67){\rule{0.241pt}{0.400pt}}
\multiput(1189.50,691.17)(-0.500,-1.000){2}{\rule{0.120pt}{0.400pt}}
\put(1187,689.67){\rule{0.482pt}{0.400pt}}
\multiput(1188.00,690.17)(-1.000,-1.000){2}{\rule{0.241pt}{0.400pt}}
\put(1185,688.67){\rule{0.482pt}{0.400pt}}
\multiput(1186.00,689.17)(-1.000,-1.000){2}{\rule{0.241pt}{0.400pt}}
\put(1207.0,703.0){\rule[-0.200pt]{0.482pt}{0.400pt}}
\put(130.0,82.0){\rule[-0.200pt]{0.400pt}{187.179pt}}
\put(130.0,82.0){\rule[-0.200pt]{315.338pt}{0.400pt}}
\put(1439.0,82.0){\rule[-0.200pt]{0.400pt}{187.179pt}}
\put(130.0,859.0){\rule[-0.200pt]{315.338pt}{0.400pt}}
\end{picture}

Plot for Ball 5:\\
% GNUPLOT: LaTeX picture
\setlength{\unitlength}{0.240900pt}
\ifx\plotpoint\undefined\newsavebox{\plotpoint}\fi
\sbox{\plotpoint}{\rule[-0.200pt]{0.400pt}{0.400pt}}%
\begin{picture}(1500,900)(0,0)
\sbox{\plotpoint}{\rule[-0.200pt]{0.400pt}{0.400pt}}%
\put(130.0,90.0){\rule[-0.200pt]{4.818pt}{0.400pt}}
\put(110,90){\makebox(0,0)[r]{ 0}}
\put(1419.0,90.0){\rule[-0.200pt]{4.818pt}{0.400pt}}
\put(130.0,242.0){\rule[-0.200pt]{4.818pt}{0.400pt}}
\put(110,242){\makebox(0,0)[r]{ 0.2}}
\put(1419.0,242.0){\rule[-0.200pt]{4.818pt}{0.400pt}}
\put(130.0,394.0){\rule[-0.200pt]{4.818pt}{0.400pt}}
\put(110,394){\makebox(0,0)[r]{ 0.4}}
\put(1419.0,394.0){\rule[-0.200pt]{4.818pt}{0.400pt}}
\put(130.0,547.0){\rule[-0.200pt]{4.818pt}{0.400pt}}
\put(110,547){\makebox(0,0)[r]{ 0.6}}
\put(1419.0,547.0){\rule[-0.200pt]{4.818pt}{0.400pt}}
\put(130.0,699.0){\rule[-0.200pt]{4.818pt}{0.400pt}}
\put(110,699){\makebox(0,0)[r]{ 0.8}}
\put(1419.0,699.0){\rule[-0.200pt]{4.818pt}{0.400pt}}
\put(130.0,851.0){\rule[-0.200pt]{4.818pt}{0.400pt}}
\put(110,851){\makebox(0,0)[r]{ 1}}
\put(1419.0,851.0){\rule[-0.200pt]{4.818pt}{0.400pt}}
\put(130.0,82.0){\rule[-0.200pt]{0.400pt}{4.818pt}}
\put(130,41){\makebox(0,0){ 0}}
\put(130.0,839.0){\rule[-0.200pt]{0.400pt}{4.818pt}}
\put(392.0,82.0){\rule[-0.200pt]{0.400pt}{4.818pt}}
\put(392,41){\makebox(0,0){ 0.2}}
\put(392.0,839.0){\rule[-0.200pt]{0.400pt}{4.818pt}}
\put(654.0,82.0){\rule[-0.200pt]{0.400pt}{4.818pt}}
\put(654,41){\makebox(0,0){ 0.4}}
\put(654.0,839.0){\rule[-0.200pt]{0.400pt}{4.818pt}}
\put(915.0,82.0){\rule[-0.200pt]{0.400pt}{4.818pt}}
\put(915,41){\makebox(0,0){ 0.6}}
\put(915.0,839.0){\rule[-0.200pt]{0.400pt}{4.818pt}}
\put(1177.0,82.0){\rule[-0.200pt]{0.400pt}{4.818pt}}
\put(1177,41){\makebox(0,0){ 0.8}}
\put(1177.0,839.0){\rule[-0.200pt]{0.400pt}{4.818pt}}
\put(1439.0,82.0){\rule[-0.200pt]{0.400pt}{4.818pt}}
\put(1439,41){\makebox(0,0){ 1}}
\put(1439.0,839.0){\rule[-0.200pt]{0.400pt}{4.818pt}}
\put(130.0,82.0){\rule[-0.200pt]{0.400pt}{187.179pt}}
\put(130.0,82.0){\rule[-0.200pt]{315.338pt}{0.400pt}}
\put(1439.0,82.0){\rule[-0.200pt]{0.400pt}{187.179pt}}
\put(130.0,859.0){\rule[-0.200pt]{315.338pt}{0.400pt}}
\put(1279,819){\makebox(0,0)[r]{'-'}}
\put(1299.0,819.0){\rule[-0.200pt]{24.090pt}{0.400pt}}
\put(1045,419){\usebox{\plotpoint}}
\put(1043,419.17){\rule{0.482pt}{0.400pt}}
\multiput(1044.00,418.17)(-1.000,2.000){2}{\rule{0.241pt}{0.400pt}}
\multiput(1040.92,421.61)(-0.462,0.447){3}{\rule{0.500pt}{0.108pt}}
\multiput(1041.96,420.17)(-1.962,3.000){2}{\rule{0.250pt}{0.400pt}}
\multiput(1037.92,424.61)(-0.462,0.447){3}{\rule{0.500pt}{0.108pt}}
\multiput(1038.96,423.17)(-1.962,3.000){2}{\rule{0.250pt}{0.400pt}}
\put(1035,427.17){\rule{0.482pt}{0.400pt}}
\multiput(1036.00,426.17)(-1.000,2.000){2}{\rule{0.241pt}{0.400pt}}
\multiput(1032.92,429.61)(-0.462,0.447){3}{\rule{0.500pt}{0.108pt}}
\multiput(1033.96,428.17)(-1.962,3.000){2}{\rule{0.250pt}{0.400pt}}
\put(1030.17,432){\rule{0.400pt}{0.700pt}}
\multiput(1031.17,432.00)(-2.000,1.547){2}{\rule{0.400pt}{0.350pt}}
\put(1027,435.17){\rule{0.700pt}{0.400pt}}
\multiput(1028.55,434.17)(-1.547,2.000){2}{\rule{0.350pt}{0.400pt}}
\multiput(1024.92,437.61)(-0.462,0.447){3}{\rule{0.500pt}{0.108pt}}
\multiput(1025.96,436.17)(-1.962,3.000){2}{\rule{0.250pt}{0.400pt}}
\put(1022,440.17){\rule{0.482pt}{0.400pt}}
\multiput(1023.00,439.17)(-1.000,2.000){2}{\rule{0.241pt}{0.400pt}}
\multiput(1019.92,442.61)(-0.462,0.447){3}{\rule{0.500pt}{0.108pt}}
\multiput(1020.96,441.17)(-1.962,3.000){2}{\rule{0.250pt}{0.400pt}}
\multiput(1016.92,445.61)(-0.462,0.447){3}{\rule{0.500pt}{0.108pt}}
\multiput(1017.96,444.17)(-1.962,3.000){2}{\rule{0.250pt}{0.400pt}}
\put(1014,448.17){\rule{0.482pt}{0.400pt}}
\multiput(1015.00,447.17)(-1.000,2.000){2}{\rule{0.241pt}{0.400pt}}
\multiput(1011.92,450.61)(-0.462,0.447){3}{\rule{0.500pt}{0.108pt}}
\multiput(1012.96,449.17)(-1.962,3.000){2}{\rule{0.250pt}{0.400pt}}
\put(1009.17,453){\rule{0.400pt}{0.700pt}}
\multiput(1010.17,453.00)(-2.000,1.547){2}{\rule{0.400pt}{0.350pt}}
\put(1006,456.17){\rule{0.700pt}{0.400pt}}
\multiput(1007.55,455.17)(-1.547,2.000){2}{\rule{0.350pt}{0.400pt}}
\multiput(1003.92,458.61)(-0.462,0.447){3}{\rule{0.500pt}{0.108pt}}
\multiput(1004.96,457.17)(-1.962,3.000){2}{\rule{0.250pt}{0.400pt}}
\put(1001.17,461){\rule{0.400pt}{0.700pt}}
\multiput(1002.17,461.00)(-2.000,1.547){2}{\rule{0.400pt}{0.350pt}}
\put(998,464.17){\rule{0.700pt}{0.400pt}}
\multiput(999.55,463.17)(-1.547,2.000){2}{\rule{0.350pt}{0.400pt}}
\put(996.17,466){\rule{0.400pt}{0.700pt}}
\multiput(997.17,466.00)(-2.000,1.547){2}{\rule{0.400pt}{0.350pt}}
\multiput(993.92,469.61)(-0.462,0.447){3}{\rule{0.500pt}{0.108pt}}
\multiput(994.96,468.17)(-1.962,3.000){2}{\rule{0.250pt}{0.400pt}}
\put(990,472.17){\rule{0.700pt}{0.400pt}}
\multiput(991.55,471.17)(-1.547,2.000){2}{\rule{0.350pt}{0.400pt}}
\put(988.17,474){\rule{0.400pt}{0.700pt}}
\multiput(989.17,474.00)(-2.000,1.547){2}{\rule{0.400pt}{0.350pt}}
\multiput(985.92,477.61)(-0.462,0.447){3}{\rule{0.500pt}{0.108pt}}
\multiput(986.96,476.17)(-1.962,3.000){2}{\rule{0.250pt}{0.400pt}}
\put(983,480.17){\rule{0.482pt}{0.400pt}}
\multiput(984.00,479.17)(-1.000,2.000){2}{\rule{0.241pt}{0.400pt}}
\multiput(980.92,482.61)(-0.462,0.447){3}{\rule{0.500pt}{0.108pt}}
\multiput(981.96,481.17)(-1.962,3.000){2}{\rule{0.250pt}{0.400pt}}
\multiput(977.92,485.61)(-0.462,0.447){3}{\rule{0.500pt}{0.108pt}}
\multiput(978.96,484.17)(-1.962,3.000){2}{\rule{0.250pt}{0.400pt}}
\put(975,488.17){\rule{0.482pt}{0.400pt}}
\multiput(976.00,487.17)(-1.000,2.000){2}{\rule{0.241pt}{0.400pt}}
\multiput(972.92,490.61)(-0.462,0.447){3}{\rule{0.500pt}{0.108pt}}
\multiput(973.96,489.17)(-1.962,3.000){2}{\rule{0.250pt}{0.400pt}}
\multiput(969.92,493.61)(-0.462,0.447){3}{\rule{0.500pt}{0.108pt}}
\multiput(970.96,492.17)(-1.962,3.000){2}{\rule{0.250pt}{0.400pt}}
\put(967,496.17){\rule{0.482pt}{0.400pt}}
\multiput(968.00,495.17)(-1.000,2.000){2}{\rule{0.241pt}{0.400pt}}
\multiput(964.92,498.61)(-0.462,0.447){3}{\rule{0.500pt}{0.108pt}}
\multiput(965.96,497.17)(-1.962,3.000){2}{\rule{0.250pt}{0.400pt}}
\put(962,501.17){\rule{0.482pt}{0.400pt}}
\multiput(963.00,500.17)(-1.000,2.000){2}{\rule{0.241pt}{0.400pt}}
\multiput(959.92,503.61)(-0.462,0.447){3}{\rule{0.500pt}{0.108pt}}
\multiput(960.96,502.17)(-1.962,3.000){2}{\rule{0.250pt}{0.400pt}}
\multiput(956.92,506.61)(-0.462,0.447){3}{\rule{0.500pt}{0.108pt}}
\multiput(957.96,505.17)(-1.962,3.000){2}{\rule{0.250pt}{0.400pt}}
\put(954,509.17){\rule{0.482pt}{0.400pt}}
\multiput(955.00,508.17)(-1.000,2.000){2}{\rule{0.241pt}{0.400pt}}
\multiput(951.92,511.61)(-0.462,0.447){3}{\rule{0.500pt}{0.108pt}}
\multiput(952.96,510.17)(-1.962,3.000){2}{\rule{0.250pt}{0.400pt}}
\put(949.17,514){\rule{0.400pt}{0.700pt}}
\multiput(950.17,514.00)(-2.000,1.547){2}{\rule{0.400pt}{0.350pt}}
\put(946,517.17){\rule{0.700pt}{0.400pt}}
\multiput(947.55,516.17)(-1.547,2.000){2}{\rule{0.350pt}{0.400pt}}
\multiput(943.92,519.61)(-0.462,0.447){3}{\rule{0.500pt}{0.108pt}}
\multiput(944.96,518.17)(-1.962,3.000){2}{\rule{0.250pt}{0.400pt}}
\put(941.17,522){\rule{0.400pt}{0.700pt}}
\multiput(942.17,522.00)(-2.000,1.547){2}{\rule{0.400pt}{0.350pt}}
\put(938,525.17){\rule{0.700pt}{0.400pt}}
\multiput(939.55,524.17)(-1.547,2.000){2}{\rule{0.350pt}{0.400pt}}
\multiput(935.92,527.61)(-0.462,0.447){3}{\rule{0.500pt}{0.108pt}}
\multiput(936.96,526.17)(-1.962,3.000){2}{\rule{0.250pt}{0.400pt}}
\put(933.17,530){\rule{0.400pt}{0.700pt}}
\multiput(934.17,530.00)(-2.000,1.547){2}{\rule{0.400pt}{0.350pt}}
\put(930,533.17){\rule{0.700pt}{0.400pt}}
\multiput(931.55,532.17)(-1.547,2.000){2}{\rule{0.350pt}{0.400pt}}
\put(928.17,535){\rule{0.400pt}{0.700pt}}
\multiput(929.17,535.00)(-2.000,1.547){2}{\rule{0.400pt}{0.350pt}}
\multiput(925.92,538.61)(-0.462,0.447){3}{\rule{0.500pt}{0.108pt}}
\multiput(926.96,537.17)(-1.962,3.000){2}{\rule{0.250pt}{0.400pt}}
\put(922,541.17){\rule{0.700pt}{0.400pt}}
\multiput(923.55,540.17)(-1.547,2.000){2}{\rule{0.350pt}{0.400pt}}
\put(920.17,543){\rule{0.400pt}{0.700pt}}
\multiput(921.17,543.00)(-2.000,1.547){2}{\rule{0.400pt}{0.350pt}}
\multiput(917.92,546.61)(-0.462,0.447){3}{\rule{0.500pt}{0.108pt}}
\multiput(918.96,545.17)(-1.962,3.000){2}{\rule{0.250pt}{0.400pt}}
\put(915,549.17){\rule{0.482pt}{0.400pt}}
\multiput(916.00,548.17)(-1.000,2.000){2}{\rule{0.241pt}{0.400pt}}
\multiput(912.92,551.61)(-0.462,0.447){3}{\rule{0.500pt}{0.108pt}}
\multiput(913.96,550.17)(-1.962,3.000){2}{\rule{0.250pt}{0.400pt}}
\multiput(909.92,554.61)(-0.462,0.447){3}{\rule{0.500pt}{0.108pt}}
\multiput(910.96,553.17)(-1.962,3.000){2}{\rule{0.250pt}{0.400pt}}
\put(907,557.17){\rule{0.482pt}{0.400pt}}
\multiput(908.00,556.17)(-1.000,2.000){2}{\rule{0.241pt}{0.400pt}}
\multiput(904.92,559.61)(-0.462,0.447){3}{\rule{0.500pt}{0.108pt}}
\multiput(905.96,558.17)(-1.962,3.000){2}{\rule{0.250pt}{0.400pt}}
\put(901,562.17){\rule{0.700pt}{0.400pt}}
\multiput(902.55,561.17)(-1.547,2.000){2}{\rule{0.350pt}{0.400pt}}
\put(899.17,564){\rule{0.400pt}{0.700pt}}
\multiput(900.17,564.00)(-2.000,1.547){2}{\rule{0.400pt}{0.350pt}}
\multiput(896.92,567.61)(-0.462,0.447){3}{\rule{0.500pt}{0.108pt}}
\multiput(897.96,566.17)(-1.962,3.000){2}{\rule{0.250pt}{0.400pt}}
\put(894,570.17){\rule{0.482pt}{0.400pt}}
\multiput(895.00,569.17)(-1.000,2.000){2}{\rule{0.241pt}{0.400pt}}
\multiput(891.92,572.61)(-0.462,0.447){3}{\rule{0.500pt}{0.108pt}}
\multiput(892.96,571.17)(-1.962,3.000){2}{\rule{0.250pt}{0.400pt}}
\multiput(888.92,575.61)(-0.462,0.447){3}{\rule{0.500pt}{0.108pt}}
\multiput(889.96,574.17)(-1.962,3.000){2}{\rule{0.250pt}{0.400pt}}
\put(886,578.17){\rule{0.482pt}{0.400pt}}
\multiput(887.00,577.17)(-1.000,2.000){2}{\rule{0.241pt}{0.400pt}}
\multiput(883.92,580.61)(-0.462,0.447){3}{\rule{0.500pt}{0.108pt}}
\multiput(884.96,579.17)(-1.962,3.000){2}{\rule{0.250pt}{0.400pt}}
\put(881.17,583){\rule{0.400pt}{0.700pt}}
\multiput(882.17,583.00)(-2.000,1.547){2}{\rule{0.400pt}{0.350pt}}
\put(878,586.17){\rule{0.700pt}{0.400pt}}
\multiput(879.55,585.17)(-1.547,2.000){2}{\rule{0.350pt}{0.400pt}}
\multiput(875.92,588.61)(-0.462,0.447){3}{\rule{0.500pt}{0.108pt}}
\multiput(876.96,587.17)(-1.962,3.000){2}{\rule{0.250pt}{0.400pt}}
\put(873.17,591){\rule{0.400pt}{0.700pt}}
\multiput(874.17,591.00)(-2.000,1.547){2}{\rule{0.400pt}{0.350pt}}
\put(870,594.17){\rule{0.700pt}{0.400pt}}
\multiput(871.55,593.17)(-1.547,2.000){2}{\rule{0.350pt}{0.400pt}}
\put(868.17,596){\rule{0.400pt}{0.700pt}}
\multiput(869.17,596.00)(-2.000,1.547){2}{\rule{0.400pt}{0.350pt}}
\multiput(865.92,599.61)(-0.462,0.447){3}{\rule{0.500pt}{0.108pt}}
\multiput(866.96,598.17)(-1.962,3.000){2}{\rule{0.250pt}{0.400pt}}
\put(862,602.17){\rule{0.700pt}{0.400pt}}
\multiput(863.55,601.17)(-1.547,2.000){2}{\rule{0.350pt}{0.400pt}}
\put(860.17,604){\rule{0.400pt}{0.700pt}}
\multiput(861.17,604.00)(-2.000,1.547){2}{\rule{0.400pt}{0.350pt}}
\multiput(857.92,607.61)(-0.462,0.447){3}{\rule{0.500pt}{0.108pt}}
\multiput(858.96,606.17)(-1.962,3.000){2}{\rule{0.250pt}{0.400pt}}
\put(854,610.17){\rule{0.700pt}{0.400pt}}
\multiput(855.55,609.17)(-1.547,2.000){2}{\rule{0.350pt}{0.400pt}}
\put(852.17,612){\rule{0.400pt}{0.700pt}}
\multiput(853.17,612.00)(-2.000,1.547){2}{\rule{0.400pt}{0.350pt}}
\multiput(849.92,615.61)(-0.462,0.447){3}{\rule{0.500pt}{0.108pt}}
\multiput(850.96,614.17)(-1.962,3.000){2}{\rule{0.250pt}{0.400pt}}
\put(847,618.17){\rule{0.482pt}{0.400pt}}
\multiput(848.00,617.17)(-1.000,2.000){2}{\rule{0.241pt}{0.400pt}}
\multiput(844.92,620.61)(-0.462,0.447){3}{\rule{0.500pt}{0.108pt}}
\multiput(845.96,619.17)(-1.962,3.000){2}{\rule{0.250pt}{0.400pt}}
\put(841,623.17){\rule{0.700pt}{0.400pt}}
\multiput(842.55,622.17)(-1.547,2.000){2}{\rule{0.350pt}{0.400pt}}
\put(839.17,625){\rule{0.400pt}{0.700pt}}
\multiput(840.17,625.00)(-2.000,1.547){2}{\rule{0.400pt}{0.350pt}}
\multiput(836.92,628.61)(-0.462,0.447){3}{\rule{0.500pt}{0.108pt}}
\multiput(837.96,627.17)(-1.962,3.000){2}{\rule{0.250pt}{0.400pt}}
\put(834,631.17){\rule{0.482pt}{0.400pt}}
\multiput(835.00,630.17)(-1.000,2.000){2}{\rule{0.241pt}{0.400pt}}
\multiput(831.92,633.61)(-0.462,0.447){3}{\rule{0.500pt}{0.108pt}}
\multiput(832.96,632.17)(-1.962,3.000){2}{\rule{0.250pt}{0.400pt}}
\multiput(828.92,636.61)(-0.462,0.447){3}{\rule{0.500pt}{0.108pt}}
\multiput(829.96,635.17)(-1.962,3.000){2}{\rule{0.250pt}{0.400pt}}
\put(826,639.17){\rule{0.482pt}{0.400pt}}
\multiput(827.00,638.17)(-1.000,2.000){2}{\rule{0.241pt}{0.400pt}}
\multiput(823.92,641.61)(-0.462,0.447){3}{\rule{0.500pt}{0.108pt}}
\multiput(824.96,640.17)(-1.962,3.000){2}{\rule{0.250pt}{0.400pt}}
\multiput(820.92,644.61)(-0.462,0.447){3}{\rule{0.500pt}{0.108pt}}
\multiput(821.96,643.17)(-1.962,3.000){2}{\rule{0.250pt}{0.400pt}}
\put(818,647.17){\rule{0.482pt}{0.400pt}}
\multiput(819.00,646.17)(-1.000,2.000){2}{\rule{0.241pt}{0.400pt}}
\multiput(815.92,649.61)(-0.462,0.447){3}{\rule{0.500pt}{0.108pt}}
\multiput(816.96,648.17)(-1.962,3.000){2}{\rule{0.250pt}{0.400pt}}
\put(813.17,652){\rule{0.400pt}{0.700pt}}
\multiput(814.17,652.00)(-2.000,1.547){2}{\rule{0.400pt}{0.350pt}}
\put(810,655.17){\rule{0.700pt}{0.400pt}}
\multiput(811.55,654.17)(-1.547,2.000){2}{\rule{0.350pt}{0.400pt}}
\multiput(807.92,657.61)(-0.462,0.447){3}{\rule{0.500pt}{0.108pt}}
\multiput(808.96,656.17)(-1.962,3.000){2}{\rule{0.250pt}{0.400pt}}
\put(805.17,660){\rule{0.400pt}{0.700pt}}
\multiput(806.17,660.00)(-2.000,1.547){2}{\rule{0.400pt}{0.350pt}}
\put(802,663.17){\rule{0.700pt}{0.400pt}}
\multiput(803.55,662.17)(-1.547,2.000){2}{\rule{0.350pt}{0.400pt}}
\put(800.17,665){\rule{0.400pt}{0.700pt}}
\multiput(801.17,665.00)(-2.000,1.547){2}{\rule{0.400pt}{0.350pt}}
\multiput(797.92,668.61)(-0.462,0.447){3}{\rule{0.500pt}{0.108pt}}
\multiput(798.96,667.17)(-1.962,3.000){2}{\rule{0.250pt}{0.400pt}}
\put(794,671.17){\rule{0.700pt}{0.400pt}}
\multiput(795.55,670.17)(-1.547,2.000){2}{\rule{0.350pt}{0.400pt}}
\put(792.17,673){\rule{0.400pt}{0.700pt}}
\multiput(793.17,673.00)(-2.000,1.547){2}{\rule{0.400pt}{0.350pt}}
\multiput(789.92,676.61)(-0.462,0.447){3}{\rule{0.500pt}{0.108pt}}
\multiput(790.96,675.17)(-1.962,3.000){2}{\rule{0.250pt}{0.400pt}}
\put(786,679.17){\rule{0.700pt}{0.400pt}}
\multiput(787.55,678.17)(-1.547,2.000){2}{\rule{0.350pt}{0.400pt}}
\put(784.17,681){\rule{0.400pt}{0.700pt}}
\multiput(785.17,681.00)(-2.000,1.547){2}{\rule{0.400pt}{0.350pt}}
\put(781,684.17){\rule{0.700pt}{0.400pt}}
\multiput(782.55,683.17)(-1.547,2.000){2}{\rule{0.350pt}{0.400pt}}
\put(779.17,686){\rule{0.400pt}{0.700pt}}
\multiput(780.17,686.00)(-2.000,1.547){2}{\rule{0.400pt}{0.350pt}}
\multiput(776.92,689.61)(-0.462,0.447){3}{\rule{0.500pt}{0.108pt}}
\multiput(777.96,688.17)(-1.962,3.000){2}{\rule{0.250pt}{0.400pt}}
\put(773,692.17){\rule{0.700pt}{0.400pt}}
\multiput(774.55,691.17)(-1.547,2.000){2}{\rule{0.350pt}{0.400pt}}
\put(771.17,694){\rule{0.400pt}{0.700pt}}
\multiput(772.17,694.00)(-2.000,1.547){2}{\rule{0.400pt}{0.350pt}}
\multiput(768.92,697.61)(-0.462,0.447){3}{\rule{0.500pt}{0.108pt}}
\multiput(769.96,696.17)(-1.962,3.000){2}{\rule{0.250pt}{0.400pt}}
\put(766,700.17){\rule{0.482pt}{0.400pt}}
\multiput(767.00,699.17)(-1.000,2.000){2}{\rule{0.241pt}{0.400pt}}
\multiput(763.92,702.61)(-0.462,0.447){3}{\rule{0.500pt}{0.108pt}}
\multiput(764.96,701.17)(-1.962,3.000){2}{\rule{0.250pt}{0.400pt}}
\multiput(760.92,705.61)(-0.462,0.447){3}{\rule{0.500pt}{0.108pt}}
\multiput(761.96,704.17)(-1.962,3.000){2}{\rule{0.250pt}{0.400pt}}
\put(758,708.17){\rule{0.482pt}{0.400pt}}
\multiput(759.00,707.17)(-1.000,2.000){2}{\rule{0.241pt}{0.400pt}}
\multiput(755.92,710.61)(-0.462,0.447){3}{\rule{0.500pt}{0.108pt}}
\multiput(756.96,709.17)(-1.962,3.000){2}{\rule{0.250pt}{0.400pt}}
\put(753.17,713){\rule{0.400pt}{0.700pt}}
\multiput(754.17,713.00)(-2.000,1.547){2}{\rule{0.400pt}{0.350pt}}
\put(750,716.17){\rule{0.700pt}{0.400pt}}
\multiput(751.55,715.17)(-1.547,2.000){2}{\rule{0.350pt}{0.400pt}}
\multiput(747.92,718.61)(-0.462,0.447){3}{\rule{0.500pt}{0.108pt}}
\multiput(748.96,717.17)(-1.962,3.000){2}{\rule{0.250pt}{0.400pt}}
\put(745.17,721){\rule{0.400pt}{0.700pt}}
\multiput(746.17,721.00)(-2.000,1.547){2}{\rule{0.400pt}{0.350pt}}
\put(742,724.17){\rule{0.700pt}{0.400pt}}
\multiput(743.55,723.17)(-1.547,2.000){2}{\rule{0.350pt}{0.400pt}}
\multiput(739.92,726.61)(-0.462,0.447){3}{\rule{0.500pt}{0.108pt}}
\multiput(740.96,725.17)(-1.962,3.000){2}{\rule{0.250pt}{0.400pt}}
\put(737.17,729){\rule{0.400pt}{0.700pt}}
\multiput(738.17,729.00)(-2.000,1.547){2}{\rule{0.400pt}{0.350pt}}
\put(734,732.17){\rule{0.700pt}{0.400pt}}
\multiput(735.55,731.17)(-1.547,2.000){2}{\rule{0.350pt}{0.400pt}}
\put(732.17,734){\rule{0.400pt}{0.700pt}}
\multiput(733.17,734.00)(-2.000,1.547){2}{\rule{0.400pt}{0.350pt}}
\multiput(729.92,737.61)(-0.462,0.447){3}{\rule{0.500pt}{0.108pt}}
\multiput(730.96,736.17)(-1.962,3.000){2}{\rule{0.250pt}{0.400pt}}
\put(726,740.17){\rule{0.700pt}{0.400pt}}
\multiput(727.55,739.17)(-1.547,2.000){2}{\rule{0.350pt}{0.400pt}}
\put(724.17,742){\rule{0.400pt}{0.700pt}}
\multiput(725.17,742.00)(-2.000,1.547){2}{\rule{0.400pt}{0.350pt}}
\put(721,745.17){\rule{0.700pt}{0.400pt}}
\multiput(722.55,744.17)(-1.547,2.000){2}{\rule{0.350pt}{0.400pt}}
\put(719.17,747){\rule{0.400pt}{0.700pt}}
\multiput(720.17,747.00)(-2.000,1.547){2}{\rule{0.400pt}{0.350pt}}
\multiput(716.92,750.61)(-0.462,0.447){3}{\rule{0.500pt}{0.108pt}}
\multiput(717.96,749.17)(-1.962,3.000){2}{\rule{0.250pt}{0.400pt}}
\put(713,753.17){\rule{0.700pt}{0.400pt}}
\multiput(714.55,752.17)(-1.547,2.000){2}{\rule{0.350pt}{0.400pt}}
\put(711.17,755){\rule{0.400pt}{0.700pt}}
\multiput(712.17,755.00)(-2.000,1.547){2}{\rule{0.400pt}{0.350pt}}
\multiput(708.92,758.61)(-0.462,0.447){3}{\rule{0.500pt}{0.108pt}}
\multiput(709.96,757.17)(-1.962,3.000){2}{\rule{0.250pt}{0.400pt}}
\put(705,761.17){\rule{0.700pt}{0.400pt}}
\multiput(706.55,760.17)(-1.547,2.000){2}{\rule{0.350pt}{0.400pt}}
\put(703.17,763){\rule{0.400pt}{0.700pt}}
\multiput(704.17,763.00)(-2.000,1.547){2}{\rule{0.400pt}{0.350pt}}
\multiput(700.92,766.61)(-0.462,0.447){3}{\rule{0.500pt}{0.108pt}}
\multiput(701.96,765.17)(-1.962,3.000){2}{\rule{0.250pt}{0.400pt}}
\put(698,769.17){\rule{0.482pt}{0.400pt}}
\multiput(699.00,768.17)(-1.000,2.000){2}{\rule{0.241pt}{0.400pt}}
\multiput(695.92,771.61)(-0.462,0.447){3}{\rule{0.500pt}{0.108pt}}
\multiput(696.96,770.17)(-1.962,3.000){2}{\rule{0.250pt}{0.400pt}}
\multiput(692.92,774.61)(-0.462,0.447){3}{\rule{0.500pt}{0.108pt}}
\multiput(693.96,773.17)(-1.962,3.000){2}{\rule{0.250pt}{0.400pt}}
\put(690,777.17){\rule{0.482pt}{0.400pt}}
\multiput(691.00,776.17)(-1.000,2.000){2}{\rule{0.241pt}{0.400pt}}
\multiput(687.92,779.61)(-0.462,0.447){3}{\rule{0.500pt}{0.108pt}}
\multiput(688.96,778.17)(-1.962,3.000){2}{\rule{0.250pt}{0.400pt}}
\put(685.17,782){\rule{0.400pt}{0.700pt}}
\multiput(686.17,782.00)(-2.000,1.547){2}{\rule{0.400pt}{0.350pt}}
\put(682,785.17){\rule{0.700pt}{0.400pt}}
\multiput(683.55,784.17)(-1.547,2.000){2}{\rule{0.350pt}{0.400pt}}
\multiput(679.92,787.61)(-0.462,0.447){3}{\rule{0.500pt}{0.108pt}}
\multiput(680.96,786.17)(-1.962,3.000){2}{\rule{0.250pt}{0.400pt}}
\put(677.17,790){\rule{0.400pt}{0.700pt}}
\multiput(678.17,790.00)(-2.000,1.547){2}{\rule{0.400pt}{0.350pt}}
\put(674,793.17){\rule{0.700pt}{0.400pt}}
\multiput(675.55,792.17)(-1.547,2.000){2}{\rule{0.350pt}{0.400pt}}
\multiput(671.92,795.61)(-0.462,0.447){3}{\rule{0.500pt}{0.108pt}}
\multiput(672.96,794.17)(-1.962,3.000){2}{\rule{0.250pt}{0.400pt}}
\put(669.17,798){\rule{0.400pt}{0.700pt}}
\multiput(670.17,798.00)(-2.000,1.547){2}{\rule{0.400pt}{0.350pt}}
\put(666,801.17){\rule{0.700pt}{0.400pt}}
\multiput(667.55,800.17)(-1.547,2.000){2}{\rule{0.350pt}{0.400pt}}
\put(664.17,803){\rule{0.400pt}{0.700pt}}
\multiput(665.17,803.00)(-2.000,1.547){2}{\rule{0.400pt}{0.350pt}}
\put(661,806.17){\rule{0.700pt}{0.400pt}}
\multiput(662.55,805.17)(-1.547,2.000){2}{\rule{0.350pt}{0.400pt}}
\multiput(658.92,808.61)(-0.462,0.447){3}{\rule{0.500pt}{0.108pt}}
\multiput(659.96,807.17)(-1.962,3.000){2}{\rule{0.250pt}{0.400pt}}
\put(656.17,811){\rule{0.400pt}{0.700pt}}
\multiput(657.17,811.00)(-2.000,1.547){2}{\rule{0.400pt}{0.350pt}}
\put(653,814.17){\rule{0.700pt}{0.400pt}}
\multiput(654.55,813.17)(-1.547,2.000){2}{\rule{0.350pt}{0.400pt}}
\put(651.17,816){\rule{0.400pt}{0.700pt}}
\multiput(652.17,816.00)(-2.000,1.547){2}{\rule{0.400pt}{0.350pt}}
\multiput(648.92,819.61)(-0.462,0.447){3}{\rule{0.500pt}{0.108pt}}
\multiput(649.96,818.17)(-1.962,3.000){2}{\rule{0.250pt}{0.400pt}}
\put(645,822.17){\rule{0.700pt}{0.400pt}}
\multiput(646.55,821.17)(-1.547,2.000){2}{\rule{0.350pt}{0.400pt}}
\put(643.17,824){\rule{0.400pt}{0.700pt}}
\multiput(644.17,824.00)(-2.000,1.547){2}{\rule{0.400pt}{0.350pt}}
\multiput(640.92,827.61)(-0.462,0.447){3}{\rule{0.500pt}{0.108pt}}
\multiput(641.96,826.17)(-1.962,3.000){2}{\rule{0.250pt}{0.400pt}}
\put(638,830.17){\rule{0.482pt}{0.400pt}}
\multiput(639.00,829.17)(-1.000,2.000){2}{\rule{0.241pt}{0.400pt}}
\multiput(635.92,832.61)(-0.462,0.447){3}{\rule{0.500pt}{0.108pt}}
\multiput(636.96,831.17)(-1.962,3.000){2}{\rule{0.250pt}{0.400pt}}
\multiput(632.92,835.61)(-0.462,0.447){3}{\rule{0.500pt}{0.108pt}}
\multiput(633.96,834.17)(-1.962,3.000){2}{\rule{0.250pt}{0.400pt}}
\put(630.17,835){\rule{0.400pt}{0.700pt}}
\multiput(631.17,836.55)(-2.000,-1.547){2}{\rule{0.400pt}{0.350pt}}
\multiput(627.92,833.95)(-0.462,-0.447){3}{\rule{0.500pt}{0.108pt}}
\multiput(628.96,834.17)(-1.962,-3.000){2}{\rule{0.250pt}{0.400pt}}
\put(624,830.17){\rule{0.700pt}{0.400pt}}
\multiput(625.55,831.17)(-1.547,-2.000){2}{\rule{0.350pt}{0.400pt}}
\put(622.17,827){\rule{0.400pt}{0.700pt}}
\multiput(623.17,828.55)(-2.000,-1.547){2}{\rule{0.400pt}{0.350pt}}
\multiput(619.92,825.95)(-0.462,-0.447){3}{\rule{0.500pt}{0.108pt}}
\multiput(620.96,826.17)(-1.962,-3.000){2}{\rule{0.250pt}{0.400pt}}
\put(617,822.17){\rule{0.482pt}{0.400pt}}
\multiput(618.00,823.17)(-1.000,-2.000){2}{\rule{0.241pt}{0.400pt}}
\multiput(614.92,820.95)(-0.462,-0.447){3}{\rule{0.500pt}{0.108pt}}
\multiput(615.96,821.17)(-1.962,-3.000){2}{\rule{0.250pt}{0.400pt}}
\multiput(611.92,817.95)(-0.462,-0.447){3}{\rule{0.500pt}{0.108pt}}
\multiput(612.96,818.17)(-1.962,-3.000){2}{\rule{0.250pt}{0.400pt}}
\put(609,814.17){\rule{0.482pt}{0.400pt}}
\multiput(610.00,815.17)(-1.000,-2.000){2}{\rule{0.241pt}{0.400pt}}
\multiput(606.92,812.95)(-0.462,-0.447){3}{\rule{0.500pt}{0.108pt}}
\multiput(607.96,813.17)(-1.962,-3.000){2}{\rule{0.250pt}{0.400pt}}
\put(604.17,808){\rule{0.400pt}{0.700pt}}
\multiput(605.17,809.55)(-2.000,-1.547){2}{\rule{0.400pt}{0.350pt}}
\put(601,806.17){\rule{0.700pt}{0.400pt}}
\multiput(602.55,807.17)(-1.547,-2.000){2}{\rule{0.350pt}{0.400pt}}
\multiput(598.92,804.95)(-0.462,-0.447){3}{\rule{0.500pt}{0.108pt}}
\multiput(599.96,805.17)(-1.962,-3.000){2}{\rule{0.250pt}{0.400pt}}
\put(596,801.17){\rule{0.482pt}{0.400pt}}
\multiput(597.00,802.17)(-1.000,-2.000){2}{\rule{0.241pt}{0.400pt}}
\multiput(593.92,799.95)(-0.462,-0.447){3}{\rule{0.500pt}{0.108pt}}
\multiput(594.96,800.17)(-1.962,-3.000){2}{\rule{0.250pt}{0.400pt}}
\multiput(590.92,796.95)(-0.462,-0.447){3}{\rule{0.500pt}{0.108pt}}
\multiput(591.96,797.17)(-1.962,-3.000){2}{\rule{0.250pt}{0.400pt}}
\put(588,793.17){\rule{0.482pt}{0.400pt}}
\multiput(589.00,794.17)(-1.000,-2.000){2}{\rule{0.241pt}{0.400pt}}
\multiput(585.92,791.95)(-0.462,-0.447){3}{\rule{0.500pt}{0.108pt}}
\multiput(586.96,792.17)(-1.962,-3.000){2}{\rule{0.250pt}{0.400pt}}
\put(583.17,787){\rule{0.400pt}{0.700pt}}
\multiput(584.17,788.55)(-2.000,-1.547){2}{\rule{0.400pt}{0.350pt}}
\put(580,785.17){\rule{0.700pt}{0.400pt}}
\multiput(581.55,786.17)(-1.547,-2.000){2}{\rule{0.350pt}{0.400pt}}
\multiput(577.92,783.95)(-0.462,-0.447){3}{\rule{0.500pt}{0.108pt}}
\multiput(578.96,784.17)(-1.962,-3.000){2}{\rule{0.250pt}{0.400pt}}
\put(575.17,779){\rule{0.400pt}{0.700pt}}
\multiput(576.17,780.55)(-2.000,-1.547){2}{\rule{0.400pt}{0.350pt}}
\put(572,777.17){\rule{0.700pt}{0.400pt}}
\multiput(573.55,778.17)(-1.547,-2.000){2}{\rule{0.350pt}{0.400pt}}
\put(570.17,774){\rule{0.400pt}{0.700pt}}
\multiput(571.17,775.55)(-2.000,-1.547){2}{\rule{0.400pt}{0.350pt}}
\multiput(567.92,772.95)(-0.462,-0.447){3}{\rule{0.500pt}{0.108pt}}
\multiput(568.96,773.17)(-1.962,-3.000){2}{\rule{0.250pt}{0.400pt}}
\put(564,769.17){\rule{0.700pt}{0.400pt}}
\multiput(565.55,770.17)(-1.547,-2.000){2}{\rule{0.350pt}{0.400pt}}
\put(562.17,766){\rule{0.400pt}{0.700pt}}
\multiput(563.17,767.55)(-2.000,-1.547){2}{\rule{0.400pt}{0.350pt}}
\multiput(559.92,764.95)(-0.462,-0.447){3}{\rule{0.500pt}{0.108pt}}
\multiput(560.96,765.17)(-1.962,-3.000){2}{\rule{0.250pt}{0.400pt}}
\put(556,761.17){\rule{0.700pt}{0.400pt}}
\multiput(557.55,762.17)(-1.547,-2.000){2}{\rule{0.350pt}{0.400pt}}
\put(554.17,758){\rule{0.400pt}{0.700pt}}
\multiput(555.17,759.55)(-2.000,-1.547){2}{\rule{0.400pt}{0.350pt}}
\multiput(551.92,756.95)(-0.462,-0.447){3}{\rule{0.500pt}{0.108pt}}
\multiput(552.96,757.17)(-1.962,-3.000){2}{\rule{0.250pt}{0.400pt}}
\put(549,753.17){\rule{0.482pt}{0.400pt}}
\multiput(550.00,754.17)(-1.000,-2.000){2}{\rule{0.241pt}{0.400pt}}
\multiput(546.92,751.95)(-0.462,-0.447){3}{\rule{0.500pt}{0.108pt}}
\multiput(547.96,752.17)(-1.962,-3.000){2}{\rule{0.250pt}{0.400pt}}
\multiput(543.92,748.95)(-0.462,-0.447){3}{\rule{0.500pt}{0.108pt}}
\multiput(544.96,749.17)(-1.962,-3.000){2}{\rule{0.250pt}{0.400pt}}
\put(541,745.17){\rule{0.482pt}{0.400pt}}
\multiput(542.00,746.17)(-1.000,-2.000){2}{\rule{0.241pt}{0.400pt}}
\multiput(538.92,743.95)(-0.462,-0.447){3}{\rule{0.500pt}{0.108pt}}
\multiput(539.96,744.17)(-1.962,-3.000){2}{\rule{0.250pt}{0.400pt}}
\put(536,740.17){\rule{0.482pt}{0.400pt}}
\multiput(537.00,741.17)(-1.000,-2.000){2}{\rule{0.241pt}{0.400pt}}
\multiput(533.92,738.95)(-0.462,-0.447){3}{\rule{0.500pt}{0.108pt}}
\multiput(534.96,739.17)(-1.962,-3.000){2}{\rule{0.250pt}{0.400pt}}
\multiput(530.92,735.95)(-0.462,-0.447){3}{\rule{0.500pt}{0.108pt}}
\multiput(531.96,736.17)(-1.962,-3.000){2}{\rule{0.250pt}{0.400pt}}
\put(528,732.17){\rule{0.482pt}{0.400pt}}
\multiput(529.00,733.17)(-1.000,-2.000){2}{\rule{0.241pt}{0.400pt}}
\multiput(525.92,730.95)(-0.462,-0.447){3}{\rule{0.500pt}{0.108pt}}
\multiput(526.96,731.17)(-1.962,-3.000){2}{\rule{0.250pt}{0.400pt}}
\put(523.17,726){\rule{0.400pt}{0.700pt}}
\multiput(524.17,727.55)(-2.000,-1.547){2}{\rule{0.400pt}{0.350pt}}
\put(520,724.17){\rule{0.700pt}{0.400pt}}
\multiput(521.55,725.17)(-1.547,-2.000){2}{\rule{0.350pt}{0.400pt}}
\multiput(517.92,722.95)(-0.462,-0.447){3}{\rule{0.500pt}{0.108pt}}
\multiput(518.96,723.17)(-1.962,-3.000){2}{\rule{0.250pt}{0.400pt}}
\put(515.17,718){\rule{0.400pt}{0.700pt}}
\multiput(516.17,719.55)(-2.000,-1.547){2}{\rule{0.400pt}{0.350pt}}
\put(512,716.17){\rule{0.700pt}{0.400pt}}
\multiput(513.55,717.17)(-1.547,-2.000){2}{\rule{0.350pt}{0.400pt}}
\multiput(509.92,714.95)(-0.462,-0.447){3}{\rule{0.500pt}{0.108pt}}
\multiput(510.96,715.17)(-1.962,-3.000){2}{\rule{0.250pt}{0.400pt}}
\put(507.17,710){\rule{0.400pt}{0.700pt}}
\multiput(508.17,711.55)(-2.000,-1.547){2}{\rule{0.400pt}{0.350pt}}
\put(504,708.17){\rule{0.700pt}{0.400pt}}
\multiput(505.55,709.17)(-1.547,-2.000){2}{\rule{0.350pt}{0.400pt}}
\put(502.17,705){\rule{0.400pt}{0.700pt}}
\multiput(503.17,706.55)(-2.000,-1.547){2}{\rule{0.400pt}{0.350pt}}
\multiput(499.92,703.95)(-0.462,-0.447){3}{\rule{0.500pt}{0.108pt}}
\multiput(500.96,704.17)(-1.962,-3.000){2}{\rule{0.250pt}{0.400pt}}
\put(496,700.17){\rule{0.700pt}{0.400pt}}
\multiput(497.55,701.17)(-1.547,-2.000){2}{\rule{0.350pt}{0.400pt}}
\put(494.17,697){\rule{0.400pt}{0.700pt}}
\multiput(495.17,698.55)(-2.000,-1.547){2}{\rule{0.400pt}{0.350pt}}
\multiput(491.92,695.95)(-0.462,-0.447){3}{\rule{0.500pt}{0.108pt}}
\multiput(492.96,696.17)(-1.962,-3.000){2}{\rule{0.250pt}{0.400pt}}
\put(489,692.17){\rule{0.482pt}{0.400pt}}
\multiput(490.00,693.17)(-1.000,-2.000){2}{\rule{0.241pt}{0.400pt}}
\multiput(486.92,690.95)(-0.462,-0.447){3}{\rule{0.500pt}{0.108pt}}
\multiput(487.96,691.17)(-1.962,-3.000){2}{\rule{0.250pt}{0.400pt}}
\multiput(483.92,687.95)(-0.462,-0.447){3}{\rule{0.500pt}{0.108pt}}
\multiput(484.96,688.17)(-1.962,-3.000){2}{\rule{0.250pt}{0.400pt}}
\put(481,684.17){\rule{0.482pt}{0.400pt}}
\multiput(482.00,685.17)(-1.000,-2.000){2}{\rule{0.241pt}{0.400pt}}
\multiput(478.92,682.95)(-0.462,-0.447){3}{\rule{0.500pt}{0.108pt}}
\multiput(479.96,683.17)(-1.962,-3.000){2}{\rule{0.250pt}{0.400pt}}
\put(475,679.17){\rule{0.700pt}{0.400pt}}
\multiput(476.55,680.17)(-1.547,-2.000){2}{\rule{0.350pt}{0.400pt}}
\put(473.17,676){\rule{0.400pt}{0.700pt}}
\multiput(474.17,677.55)(-2.000,-1.547){2}{\rule{0.400pt}{0.350pt}}
\multiput(470.92,674.95)(-0.462,-0.447){3}{\rule{0.500pt}{0.108pt}}
\multiput(471.96,675.17)(-1.962,-3.000){2}{\rule{0.250pt}{0.400pt}}
\put(468,671.17){\rule{0.482pt}{0.400pt}}
\multiput(469.00,672.17)(-1.000,-2.000){2}{\rule{0.241pt}{0.400pt}}
\multiput(465.92,669.95)(-0.462,-0.447){3}{\rule{0.500pt}{0.108pt}}
\multiput(466.96,670.17)(-1.962,-3.000){2}{\rule{0.250pt}{0.400pt}}
\multiput(462.92,666.95)(-0.462,-0.447){3}{\rule{0.500pt}{0.108pt}}
\multiput(463.96,667.17)(-1.962,-3.000){2}{\rule{0.250pt}{0.400pt}}
\put(460,663.17){\rule{0.482pt}{0.400pt}}
\multiput(461.00,664.17)(-1.000,-2.000){2}{\rule{0.241pt}{0.400pt}}
\multiput(457.92,661.95)(-0.462,-0.447){3}{\rule{0.500pt}{0.108pt}}
\multiput(458.96,662.17)(-1.962,-3.000){2}{\rule{0.250pt}{0.400pt}}
\put(455.17,657){\rule{0.400pt}{0.700pt}}
\multiput(456.17,658.55)(-2.000,-1.547){2}{\rule{0.400pt}{0.350pt}}
\put(452,655.17){\rule{0.700pt}{0.400pt}}
\multiput(453.55,656.17)(-1.547,-2.000){2}{\rule{0.350pt}{0.400pt}}
\multiput(449.92,653.95)(-0.462,-0.447){3}{\rule{0.500pt}{0.108pt}}
\multiput(450.96,654.17)(-1.962,-3.000){2}{\rule{0.250pt}{0.400pt}}
\put(447.17,649){\rule{0.400pt}{0.700pt}}
\multiput(448.17,650.55)(-2.000,-1.547){2}{\rule{0.400pt}{0.350pt}}
\put(444,647.17){\rule{0.700pt}{0.400pt}}
\multiput(445.55,648.17)(-1.547,-2.000){2}{\rule{0.350pt}{0.400pt}}
\multiput(441.92,645.95)(-0.462,-0.447){3}{\rule{0.500pt}{0.108pt}}
\multiput(442.96,646.17)(-1.962,-3.000){2}{\rule{0.250pt}{0.400pt}}
\put(439.17,641){\rule{0.400pt}{0.700pt}}
\multiput(440.17,642.55)(-2.000,-1.547){2}{\rule{0.400pt}{0.350pt}}
\put(436,639.17){\rule{0.700pt}{0.400pt}}
\multiput(437.55,640.17)(-1.547,-2.000){2}{\rule{0.350pt}{0.400pt}}
\put(434.17,636){\rule{0.400pt}{0.700pt}}
\multiput(435.17,637.55)(-2.000,-1.547){2}{\rule{0.400pt}{0.350pt}}
\multiput(431.92,634.95)(-0.462,-0.447){3}{\rule{0.500pt}{0.108pt}}
\multiput(432.96,635.17)(-1.962,-3.000){2}{\rule{0.250pt}{0.400pt}}
\put(428,631.17){\rule{0.700pt}{0.400pt}}
\multiput(429.55,632.17)(-1.547,-2.000){2}{\rule{0.350pt}{0.400pt}}
\put(426.17,628){\rule{0.400pt}{0.700pt}}
\multiput(427.17,629.55)(-2.000,-1.547){2}{\rule{0.400pt}{0.350pt}}
\multiput(423.92,626.95)(-0.462,-0.447){3}{\rule{0.500pt}{0.108pt}}
\multiput(424.96,627.17)(-1.962,-3.000){2}{\rule{0.250pt}{0.400pt}}
\put(421,623.17){\rule{0.482pt}{0.400pt}}
\multiput(422.00,624.17)(-1.000,-2.000){2}{\rule{0.241pt}{0.400pt}}
\multiput(418.92,621.95)(-0.462,-0.447){3}{\rule{0.500pt}{0.108pt}}
\multiput(419.96,622.17)(-1.962,-3.000){2}{\rule{0.250pt}{0.400pt}}
\put(415,618.17){\rule{0.700pt}{0.400pt}}
\multiput(416.55,619.17)(-1.547,-2.000){2}{\rule{0.350pt}{0.400pt}}
\put(413.17,615){\rule{0.400pt}{0.700pt}}
\multiput(414.17,616.55)(-2.000,-1.547){2}{\rule{0.400pt}{0.350pt}}
\multiput(410.92,613.95)(-0.462,-0.447){3}{\rule{0.500pt}{0.108pt}}
\multiput(411.96,614.17)(-1.962,-3.000){2}{\rule{0.250pt}{0.400pt}}
\put(408,610.17){\rule{0.482pt}{0.400pt}}
\multiput(409.00,611.17)(-1.000,-2.000){2}{\rule{0.241pt}{0.400pt}}
\multiput(405.92,608.95)(-0.462,-0.447){3}{\rule{0.500pt}{0.108pt}}
\multiput(406.96,609.17)(-1.962,-3.000){2}{\rule{0.250pt}{0.400pt}}
\multiput(402.92,605.95)(-0.462,-0.447){3}{\rule{0.500pt}{0.108pt}}
\multiput(403.96,606.17)(-1.962,-3.000){2}{\rule{0.250pt}{0.400pt}}
\put(400,602.17){\rule{0.482pt}{0.400pt}}
\multiput(401.00,603.17)(-1.000,-2.000){2}{\rule{0.241pt}{0.400pt}}
\multiput(397.92,600.95)(-0.462,-0.447){3}{\rule{0.500pt}{0.108pt}}
\multiput(398.96,601.17)(-1.962,-3.000){2}{\rule{0.250pt}{0.400pt}}
\multiput(394.92,597.95)(-0.462,-0.447){3}{\rule{0.500pt}{0.108pt}}
\multiput(395.96,598.17)(-1.962,-3.000){2}{\rule{0.250pt}{0.400pt}}
\put(392,594.17){\rule{0.482pt}{0.400pt}}
\multiput(393.00,595.17)(-1.000,-2.000){2}{\rule{0.241pt}{0.400pt}}
\multiput(389.92,592.95)(-0.462,-0.447){3}{\rule{0.500pt}{0.108pt}}
\multiput(390.96,593.17)(-1.962,-3.000){2}{\rule{0.250pt}{0.400pt}}
\put(387.17,588){\rule{0.400pt}{0.700pt}}
\multiput(388.17,589.55)(-2.000,-1.547){2}{\rule{0.400pt}{0.350pt}}
\put(384,586.17){\rule{0.700pt}{0.400pt}}
\multiput(385.55,587.17)(-1.547,-2.000){2}{\rule{0.350pt}{0.400pt}}
\multiput(381.92,584.95)(-0.462,-0.447){3}{\rule{0.500pt}{0.108pt}}
\multiput(382.96,585.17)(-1.962,-3.000){2}{\rule{0.250pt}{0.400pt}}
\put(379.17,580){\rule{0.400pt}{0.700pt}}
\multiput(380.17,581.55)(-2.000,-1.547){2}{\rule{0.400pt}{0.350pt}}
\put(376,578.17){\rule{0.700pt}{0.400pt}}
\multiput(377.55,579.17)(-1.547,-2.000){2}{\rule{0.350pt}{0.400pt}}
\put(374.17,575){\rule{0.400pt}{0.700pt}}
\multiput(375.17,576.55)(-2.000,-1.547){2}{\rule{0.400pt}{0.350pt}}
\multiput(371.92,573.95)(-0.462,-0.447){3}{\rule{0.500pt}{0.108pt}}
\multiput(372.96,574.17)(-1.962,-3.000){2}{\rule{0.250pt}{0.400pt}}
\put(368,570.17){\rule{0.700pt}{0.400pt}}
\multiput(369.55,571.17)(-1.547,-2.000){2}{\rule{0.350pt}{0.400pt}}
\put(366.17,567){\rule{0.400pt}{0.700pt}}
\multiput(367.17,568.55)(-2.000,-1.547){2}{\rule{0.400pt}{0.350pt}}
\multiput(363.92,565.95)(-0.462,-0.447){3}{\rule{0.500pt}{0.108pt}}
\multiput(364.96,566.17)(-1.962,-3.000){2}{\rule{0.250pt}{0.400pt}}
\put(360,562.17){\rule{0.700pt}{0.400pt}}
\multiput(361.55,563.17)(-1.547,-2.000){2}{\rule{0.350pt}{0.400pt}}
\put(358.17,559){\rule{0.400pt}{0.700pt}}
\multiput(359.17,560.55)(-2.000,-1.547){2}{\rule{0.400pt}{0.350pt}}
\put(355,557.17){\rule{0.700pt}{0.400pt}}
\multiput(356.55,558.17)(-1.547,-2.000){2}{\rule{0.350pt}{0.400pt}}
\put(353.17,554){\rule{0.400pt}{0.700pt}}
\multiput(354.17,555.55)(-2.000,-1.547){2}{\rule{0.400pt}{0.350pt}}
\multiput(350.92,552.95)(-0.462,-0.447){3}{\rule{0.500pt}{0.108pt}}
\multiput(351.96,553.17)(-1.962,-3.000){2}{\rule{0.250pt}{0.400pt}}
\put(347,549.17){\rule{0.700pt}{0.400pt}}
\multiput(348.55,550.17)(-1.547,-2.000){2}{\rule{0.350pt}{0.400pt}}
\put(345.17,546){\rule{0.400pt}{0.700pt}}
\multiput(346.17,547.55)(-2.000,-1.547){2}{\rule{0.400pt}{0.350pt}}
\multiput(342.92,544.95)(-0.462,-0.447){3}{\rule{0.500pt}{0.108pt}}
\multiput(343.96,545.17)(-1.962,-3.000){2}{\rule{0.250pt}{0.400pt}}
\put(340,541.17){\rule{0.482pt}{0.400pt}}
\multiput(341.00,542.17)(-1.000,-2.000){2}{\rule{0.241pt}{0.400pt}}
\multiput(337.92,539.95)(-0.462,-0.447){3}{\rule{0.500pt}{0.108pt}}
\multiput(338.96,540.17)(-1.962,-3.000){2}{\rule{0.250pt}{0.400pt}}
\multiput(334.92,536.95)(-0.462,-0.447){3}{\rule{0.500pt}{0.108pt}}
\multiput(335.96,537.17)(-1.962,-3.000){2}{\rule{0.250pt}{0.400pt}}
\put(332,533.17){\rule{0.482pt}{0.400pt}}
\multiput(333.00,534.17)(-1.000,-2.000){2}{\rule{0.241pt}{0.400pt}}
\multiput(329.92,531.95)(-0.462,-0.447){3}{\rule{0.500pt}{0.108pt}}
\multiput(330.96,532.17)(-1.962,-3.000){2}{\rule{0.250pt}{0.400pt}}
\multiput(326.92,528.95)(-0.462,-0.447){3}{\rule{0.500pt}{0.108pt}}
\multiput(327.96,529.17)(-1.962,-3.000){2}{\rule{0.250pt}{0.400pt}}
\put(324,525.17){\rule{0.482pt}{0.400pt}}
\multiput(325.00,526.17)(-1.000,-2.000){2}{\rule{0.241pt}{0.400pt}}
\multiput(321.92,523.95)(-0.462,-0.447){3}{\rule{0.500pt}{0.108pt}}
\multiput(322.96,524.17)(-1.962,-3.000){2}{\rule{0.250pt}{0.400pt}}
\put(319.17,519){\rule{0.400pt}{0.700pt}}
\multiput(320.17,520.55)(-2.000,-1.547){2}{\rule{0.400pt}{0.350pt}}
\put(316,517.17){\rule{0.700pt}{0.400pt}}
\multiput(317.55,518.17)(-1.547,-2.000){2}{\rule{0.350pt}{0.400pt}}
\multiput(313.92,515.95)(-0.462,-0.447){3}{\rule{0.500pt}{0.108pt}}
\multiput(314.96,516.17)(-1.962,-3.000){2}{\rule{0.250pt}{0.400pt}}
\put(311.17,511){\rule{0.400pt}{0.700pt}}
\multiput(312.17,512.55)(-2.000,-1.547){2}{\rule{0.400pt}{0.350pt}}
\put(308,509.17){\rule{0.700pt}{0.400pt}}
\multiput(309.55,510.17)(-1.547,-2.000){2}{\rule{0.350pt}{0.400pt}}
\put(306.17,506){\rule{0.400pt}{0.700pt}}
\multiput(307.17,507.55)(-2.000,-1.547){2}{\rule{0.400pt}{0.350pt}}
\multiput(303.92,504.95)(-0.462,-0.447){3}{\rule{0.500pt}{0.108pt}}
\multiput(304.96,505.17)(-1.962,-3.000){2}{\rule{0.250pt}{0.400pt}}
\put(300,501.17){\rule{0.700pt}{0.400pt}}
\multiput(301.55,502.17)(-1.547,-2.000){2}{\rule{0.350pt}{0.400pt}}
\put(298.17,498){\rule{0.400pt}{0.700pt}}
\multiput(299.17,499.55)(-2.000,-1.547){2}{\rule{0.400pt}{0.350pt}}
\put(295,496.17){\rule{0.700pt}{0.400pt}}
\multiput(296.55,497.17)(-1.547,-2.000){2}{\rule{0.350pt}{0.400pt}}
\put(293.17,493){\rule{0.400pt}{0.700pt}}
\multiput(294.17,494.55)(-2.000,-1.547){2}{\rule{0.400pt}{0.350pt}}
\multiput(290.92,491.95)(-0.462,-0.447){3}{\rule{0.500pt}{0.108pt}}
\multiput(291.96,492.17)(-1.962,-3.000){2}{\rule{0.250pt}{0.400pt}}
\put(287,488.17){\rule{0.700pt}{0.400pt}}
\multiput(288.55,489.17)(-1.547,-2.000){2}{\rule{0.350pt}{0.400pt}}
\put(285.17,485){\rule{0.400pt}{0.700pt}}
\multiput(286.17,486.55)(-2.000,-1.547){2}{\rule{0.400pt}{0.350pt}}
\multiput(282.92,483.95)(-0.462,-0.447){3}{\rule{0.500pt}{0.108pt}}
\multiput(283.96,484.17)(-1.962,-3.000){2}{\rule{0.250pt}{0.400pt}}
\put(279,480.17){\rule{0.700pt}{0.400pt}}
\multiput(280.55,481.17)(-1.547,-2.000){2}{\rule{0.350pt}{0.400pt}}
\put(277.17,477){\rule{0.400pt}{0.700pt}}
\multiput(278.17,478.55)(-2.000,-1.547){2}{\rule{0.400pt}{0.350pt}}
\multiput(274.92,475.95)(-0.462,-0.447){3}{\rule{0.500pt}{0.108pt}}
\multiput(275.96,476.17)(-1.962,-3.000){2}{\rule{0.250pt}{0.400pt}}
\put(272,472.17){\rule{0.482pt}{0.400pt}}
\multiput(273.00,473.17)(-1.000,-2.000){2}{\rule{0.241pt}{0.400pt}}
\multiput(269.92,470.95)(-0.462,-0.447){3}{\rule{0.500pt}{0.108pt}}
\multiput(270.96,471.17)(-1.962,-3.000){2}{\rule{0.250pt}{0.400pt}}
\multiput(266.92,467.95)(-0.462,-0.447){3}{\rule{0.500pt}{0.108pt}}
\multiput(267.96,468.17)(-1.962,-3.000){2}{\rule{0.250pt}{0.400pt}}
\put(264,464.17){\rule{0.482pt}{0.400pt}}
\multiput(265.00,465.17)(-1.000,-2.000){2}{\rule{0.241pt}{0.400pt}}
\multiput(261.92,462.95)(-0.462,-0.447){3}{\rule{0.500pt}{0.108pt}}
\multiput(262.96,463.17)(-1.962,-3.000){2}{\rule{0.250pt}{0.400pt}}
\put(259.17,458){\rule{0.400pt}{0.700pt}}
\multiput(260.17,459.55)(-2.000,-1.547){2}{\rule{0.400pt}{0.350pt}}
\put(256,456.17){\rule{0.700pt}{0.400pt}}
\multiput(257.55,457.17)(-1.547,-2.000){2}{\rule{0.350pt}{0.400pt}}
\multiput(253.92,454.95)(-0.462,-0.447){3}{\rule{0.500pt}{0.108pt}}
\multiput(254.96,455.17)(-1.962,-3.000){2}{\rule{0.250pt}{0.400pt}}
\put(251.17,450){\rule{0.400pt}{0.700pt}}
\multiput(252.17,451.55)(-2.000,-1.547){2}{\rule{0.400pt}{0.350pt}}
\put(248,448.17){\rule{0.700pt}{0.400pt}}
\multiput(249.55,449.17)(-1.547,-2.000){2}{\rule{0.350pt}{0.400pt}}
\multiput(245.92,446.95)(-0.462,-0.447){3}{\rule{0.500pt}{0.108pt}}
\multiput(246.96,447.17)(-1.962,-3.000){2}{\rule{0.250pt}{0.400pt}}
\put(243.17,442){\rule{0.400pt}{0.700pt}}
\multiput(244.17,443.55)(-2.000,-1.547){2}{\rule{0.400pt}{0.350pt}}
\put(240,440.17){\rule{0.700pt}{0.400pt}}
\multiput(241.55,441.17)(-1.547,-2.000){2}{\rule{0.350pt}{0.400pt}}
\put(238.17,437){\rule{0.400pt}{0.700pt}}
\multiput(239.17,438.55)(-2.000,-1.547){2}{\rule{0.400pt}{0.350pt}}
\put(235,435.17){\rule{0.700pt}{0.400pt}}
\multiput(236.55,436.17)(-1.547,-2.000){2}{\rule{0.350pt}{0.400pt}}
\multiput(232.92,433.95)(-0.462,-0.447){3}{\rule{0.500pt}{0.108pt}}
\multiput(233.96,434.17)(-1.962,-3.000){2}{\rule{0.250pt}{0.400pt}}
\put(230.17,429){\rule{0.400pt}{0.700pt}}
\multiput(231.17,430.55)(-2.000,-1.547){2}{\rule{0.400pt}{0.350pt}}
\put(227,427.17){\rule{0.700pt}{0.400pt}}
\multiput(228.55,428.17)(-1.547,-2.000){2}{\rule{0.350pt}{0.400pt}}
\put(225.17,424){\rule{0.400pt}{0.700pt}}
\multiput(226.17,425.55)(-2.000,-1.547){2}{\rule{0.400pt}{0.350pt}}
\multiput(222.92,422.95)(-0.462,-0.447){3}{\rule{0.500pt}{0.108pt}}
\multiput(223.96,423.17)(-1.962,-3.000){2}{\rule{0.250pt}{0.400pt}}
\put(219,419.17){\rule{0.700pt}{0.400pt}}
\multiput(220.55,420.17)(-1.547,-2.000){2}{\rule{0.350pt}{0.400pt}}
\put(217.17,416){\rule{0.400pt}{0.700pt}}
\multiput(218.17,417.55)(-2.000,-1.547){2}{\rule{0.400pt}{0.350pt}}
\multiput(214.92,414.95)(-0.462,-0.447){3}{\rule{0.500pt}{0.108pt}}
\multiput(215.96,415.17)(-1.962,-3.000){2}{\rule{0.250pt}{0.400pt}}
\put(211,411.17){\rule{0.700pt}{0.400pt}}
\multiput(212.55,412.17)(-1.547,-2.000){2}{\rule{0.350pt}{0.400pt}}
\put(209.17,408){\rule{0.400pt}{0.700pt}}
\multiput(210.17,409.55)(-2.000,-1.547){2}{\rule{0.400pt}{0.350pt}}
\multiput(206.92,406.95)(-0.462,-0.447){3}{\rule{0.500pt}{0.108pt}}
\multiput(207.96,407.17)(-1.962,-3.000){2}{\rule{0.250pt}{0.400pt}}
\put(204,403.17){\rule{0.482pt}{0.400pt}}
\multiput(205.00,404.17)(-1.000,-2.000){2}{\rule{0.241pt}{0.400pt}}
\multiput(201.92,401.95)(-0.462,-0.447){3}{\rule{0.500pt}{0.108pt}}
\multiput(202.96,402.17)(-1.962,-3.000){2}{\rule{0.250pt}{0.400pt}}
\multiput(198.92,398.95)(-0.462,-0.447){3}{\rule{0.500pt}{0.108pt}}
\multiput(199.96,399.17)(-1.962,-3.000){2}{\rule{0.250pt}{0.400pt}}
\put(196,395.17){\rule{0.482pt}{0.400pt}}
\multiput(197.00,396.17)(-1.000,-2.000){2}{\rule{0.241pt}{0.400pt}}
\multiput(193.92,393.95)(-0.462,-0.447){3}{\rule{0.500pt}{0.108pt}}
\multiput(194.96,394.17)(-1.962,-3.000){2}{\rule{0.250pt}{0.400pt}}
\put(191.17,389){\rule{0.400pt}{0.700pt}}
\multiput(192.17,390.55)(-2.000,-1.547){2}{\rule{0.400pt}{0.350pt}}
\put(188,387.17){\rule{0.700pt}{0.400pt}}
\multiput(189.55,388.17)(-1.547,-2.000){2}{\rule{0.350pt}{0.400pt}}
\multiput(185.92,385.95)(-0.462,-0.447){3}{\rule{0.500pt}{0.108pt}}
\multiput(186.96,386.17)(-1.962,-3.000){2}{\rule{0.250pt}{0.400pt}}
\put(183.17,381){\rule{0.400pt}{0.700pt}}
\multiput(184.17,382.55)(-2.000,-1.547){2}{\rule{0.400pt}{0.350pt}}
\put(180,379.17){\rule{0.700pt}{0.400pt}}
\multiput(181.55,380.17)(-1.547,-2.000){2}{\rule{0.350pt}{0.400pt}}
\put(178.17,376){\rule{0.400pt}{0.700pt}}
\multiput(179.17,377.55)(-2.000,-1.547){2}{\rule{0.400pt}{0.350pt}}
\put(175,374.17){\rule{0.700pt}{0.400pt}}
\multiput(176.55,375.17)(-1.547,-2.000){2}{\rule{0.350pt}{0.400pt}}
\multiput(172.92,372.95)(-0.462,-0.447){3}{\rule{0.500pt}{0.108pt}}
\multiput(173.96,373.17)(-1.962,-3.000){2}{\rule{0.250pt}{0.400pt}}
\put(170.17,368){\rule{0.400pt}{0.700pt}}
\multiput(171.17,369.55)(-2.000,-1.547){2}{\rule{0.400pt}{0.350pt}}
\put(167,366.17){\rule{0.700pt}{0.400pt}}
\multiput(168.55,367.17)(-1.547,-2.000){2}{\rule{0.350pt}{0.400pt}}
\multiput(164.92,364.95)(-0.462,-0.447){3}{\rule{0.500pt}{0.108pt}}
\multiput(165.96,365.17)(-1.962,-3.000){2}{\rule{0.250pt}{0.400pt}}
\put(162.17,360){\rule{0.400pt}{0.700pt}}
\multiput(163.17,361.55)(-2.000,-1.547){2}{\rule{0.400pt}{0.350pt}}
\put(159,358.17){\rule{0.700pt}{0.400pt}}
\multiput(160.55,359.17)(-1.547,-2.000){2}{\rule{0.350pt}{0.400pt}}
\put(157.17,355){\rule{0.400pt}{0.700pt}}
\multiput(158.17,356.55)(-2.000,-1.547){2}{\rule{0.400pt}{0.350pt}}
\multiput(154.92,353.95)(-0.462,-0.447){3}{\rule{0.500pt}{0.108pt}}
\multiput(155.96,354.17)(-1.962,-3.000){2}{\rule{0.250pt}{0.400pt}}
\put(154,350.17){\rule{0.700pt}{0.400pt}}
\multiput(154.00,351.17)(1.547,-2.000){2}{\rule{0.350pt}{0.400pt}}
\put(157.17,347){\rule{0.400pt}{0.700pt}}
\multiput(156.17,348.55)(2.000,-1.547){2}{\rule{0.400pt}{0.350pt}}
\multiput(159.00,345.95)(0.462,-0.447){3}{\rule{0.500pt}{0.108pt}}
\multiput(159.00,346.17)(1.962,-3.000){2}{\rule{0.250pt}{0.400pt}}
\put(162,342.17){\rule{0.482pt}{0.400pt}}
\multiput(162.00,343.17)(1.000,-2.000){2}{\rule{0.241pt}{0.400pt}}
\multiput(164.00,340.95)(0.462,-0.447){3}{\rule{0.500pt}{0.108pt}}
\multiput(164.00,341.17)(1.962,-3.000){2}{\rule{0.250pt}{0.400pt}}
\multiput(167.00,337.95)(0.462,-0.447){3}{\rule{0.500pt}{0.108pt}}
\multiput(167.00,338.17)(1.962,-3.000){2}{\rule{0.250pt}{0.400pt}}
\put(170,334.17){\rule{0.482pt}{0.400pt}}
\multiput(170.00,335.17)(1.000,-2.000){2}{\rule{0.241pt}{0.400pt}}
\multiput(172.00,332.95)(0.462,-0.447){3}{\rule{0.500pt}{0.108pt}}
\multiput(172.00,333.17)(1.962,-3.000){2}{\rule{0.250pt}{0.400pt}}
\multiput(175.00,329.95)(0.462,-0.447){3}{\rule{0.500pt}{0.108pt}}
\multiput(175.00,330.17)(1.962,-3.000){2}{\rule{0.250pt}{0.400pt}}
\put(178,326.17){\rule{0.482pt}{0.400pt}}
\multiput(178.00,327.17)(1.000,-2.000){2}{\rule{0.241pt}{0.400pt}}
\multiput(180.00,324.95)(0.462,-0.447){3}{\rule{0.500pt}{0.108pt}}
\multiput(180.00,325.17)(1.962,-3.000){2}{\rule{0.250pt}{0.400pt}}
\put(183.17,320){\rule{0.400pt}{0.700pt}}
\multiput(182.17,321.55)(2.000,-1.547){2}{\rule{0.400pt}{0.350pt}}
\put(185,318.17){\rule{0.700pt}{0.400pt}}
\multiput(185.00,319.17)(1.547,-2.000){2}{\rule{0.350pt}{0.400pt}}
\multiput(188.00,316.95)(0.462,-0.447){3}{\rule{0.500pt}{0.108pt}}
\multiput(188.00,317.17)(1.962,-3.000){2}{\rule{0.250pt}{0.400pt}}
\put(191,313.17){\rule{0.482pt}{0.400pt}}
\multiput(191.00,314.17)(1.000,-2.000){2}{\rule{0.241pt}{0.400pt}}
\multiput(193.00,311.95)(0.462,-0.447){3}{\rule{0.500pt}{0.108pt}}
\multiput(193.00,312.17)(1.962,-3.000){2}{\rule{0.250pt}{0.400pt}}
\put(196.17,307){\rule{0.400pt}{0.700pt}}
\multiput(195.17,308.55)(2.000,-1.547){2}{\rule{0.400pt}{0.350pt}}
\put(198,305.17){\rule{0.700pt}{0.400pt}}
\multiput(198.00,306.17)(1.547,-2.000){2}{\rule{0.350pt}{0.400pt}}
\multiput(201.00,303.95)(0.462,-0.447){3}{\rule{0.500pt}{0.108pt}}
\multiput(201.00,304.17)(1.962,-3.000){2}{\rule{0.250pt}{0.400pt}}
\put(204.17,299){\rule{0.400pt}{0.700pt}}
\multiput(203.17,300.55)(2.000,-1.547){2}{\rule{0.400pt}{0.350pt}}
\put(206,297.17){\rule{0.700pt}{0.400pt}}
\multiput(206.00,298.17)(1.547,-2.000){2}{\rule{0.350pt}{0.400pt}}
\put(209.17,294){\rule{0.400pt}{0.700pt}}
\multiput(208.17,295.55)(2.000,-1.547){2}{\rule{0.400pt}{0.350pt}}
\multiput(211.00,292.95)(0.462,-0.447){3}{\rule{0.500pt}{0.108pt}}
\multiput(211.00,293.17)(1.962,-3.000){2}{\rule{0.250pt}{0.400pt}}
\put(214,289.17){\rule{0.700pt}{0.400pt}}
\multiput(214.00,290.17)(1.547,-2.000){2}{\rule{0.350pt}{0.400pt}}
\put(217.17,286){\rule{0.400pt}{0.700pt}}
\multiput(216.17,287.55)(2.000,-1.547){2}{\rule{0.400pt}{0.350pt}}
\multiput(219.00,284.95)(0.462,-0.447){3}{\rule{0.500pt}{0.108pt}}
\multiput(219.00,285.17)(1.962,-3.000){2}{\rule{0.250pt}{0.400pt}}
\put(222,281.17){\rule{0.700pt}{0.400pt}}
\multiput(222.00,282.17)(1.547,-2.000){2}{\rule{0.350pt}{0.400pt}}
\put(225.17,278){\rule{0.400pt}{0.700pt}}
\multiput(224.17,279.55)(2.000,-1.547){2}{\rule{0.400pt}{0.350pt}}
\multiput(227.00,276.95)(0.462,-0.447){3}{\rule{0.500pt}{0.108pt}}
\multiput(227.00,277.17)(1.962,-3.000){2}{\rule{0.250pt}{0.400pt}}
\put(230,273.17){\rule{0.482pt}{0.400pt}}
\multiput(230.00,274.17)(1.000,-2.000){2}{\rule{0.241pt}{0.400pt}}
\multiput(232.00,271.95)(0.462,-0.447){3}{\rule{0.500pt}{0.108pt}}
\multiput(232.00,272.17)(1.962,-3.000){2}{\rule{0.250pt}{0.400pt}}
\multiput(235.00,268.95)(0.462,-0.447){3}{\rule{0.500pt}{0.108pt}}
\multiput(235.00,269.17)(1.962,-3.000){2}{\rule{0.250pt}{0.400pt}}
\put(238,265.17){\rule{0.482pt}{0.400pt}}
\multiput(238.00,266.17)(1.000,-2.000){2}{\rule{0.241pt}{0.400pt}}
\multiput(240.00,263.95)(0.462,-0.447){3}{\rule{0.500pt}{0.108pt}}
\multiput(240.00,264.17)(1.962,-3.000){2}{\rule{0.250pt}{0.400pt}}
\put(243.17,259){\rule{0.400pt}{0.700pt}}
\multiput(242.17,260.55)(2.000,-1.547){2}{\rule{0.400pt}{0.350pt}}
\put(245,257.17){\rule{0.700pt}{0.400pt}}
\multiput(245.00,258.17)(1.547,-2.000){2}{\rule{0.350pt}{0.400pt}}
\multiput(248.00,255.95)(0.462,-0.447){3}{\rule{0.500pt}{0.108pt}}
\multiput(248.00,256.17)(1.962,-3.000){2}{\rule{0.250pt}{0.400pt}}
\put(251,252.17){\rule{0.482pt}{0.400pt}}
\multiput(251.00,253.17)(1.000,-2.000){2}{\rule{0.241pt}{0.400pt}}
\multiput(253.00,250.95)(0.462,-0.447){3}{\rule{0.500pt}{0.108pt}}
\multiput(253.00,251.17)(1.962,-3.000){2}{\rule{0.250pt}{0.400pt}}
\multiput(256.00,247.95)(0.462,-0.447){3}{\rule{0.500pt}{0.108pt}}
\multiput(256.00,248.17)(1.962,-3.000){2}{\rule{0.250pt}{0.400pt}}
\put(259,244.17){\rule{0.482pt}{0.400pt}}
\multiput(259.00,245.17)(1.000,-2.000){2}{\rule{0.241pt}{0.400pt}}
\multiput(261.00,242.95)(0.462,-0.447){3}{\rule{0.500pt}{0.108pt}}
\multiput(261.00,243.17)(1.962,-3.000){2}{\rule{0.250pt}{0.400pt}}
\put(264.17,238){\rule{0.400pt}{0.700pt}}
\multiput(263.17,239.55)(2.000,-1.547){2}{\rule{0.400pt}{0.350pt}}
\put(266,236.17){\rule{0.700pt}{0.400pt}}
\multiput(266.00,237.17)(1.547,-2.000){2}{\rule{0.350pt}{0.400pt}}
\multiput(269.00,234.95)(0.462,-0.447){3}{\rule{0.500pt}{0.108pt}}
\multiput(269.00,235.17)(1.962,-3.000){2}{\rule{0.250pt}{0.400pt}}
\put(272.17,230){\rule{0.400pt}{0.700pt}}
\multiput(271.17,231.55)(2.000,-1.547){2}{\rule{0.400pt}{0.350pt}}
\put(274,228.17){\rule{0.700pt}{0.400pt}}
\multiput(274.00,229.17)(1.547,-2.000){2}{\rule{0.350pt}{0.400pt}}
\put(277.17,225){\rule{0.400pt}{0.700pt}}
\multiput(276.17,226.55)(2.000,-1.547){2}{\rule{0.400pt}{0.350pt}}
\multiput(279.00,223.95)(0.462,-0.447){3}{\rule{0.500pt}{0.108pt}}
\multiput(279.00,224.17)(1.962,-3.000){2}{\rule{0.250pt}{0.400pt}}
\put(282,220.17){\rule{0.700pt}{0.400pt}}
\multiput(282.00,221.17)(1.547,-2.000){2}{\rule{0.350pt}{0.400pt}}
\put(285.17,217){\rule{0.400pt}{0.700pt}}
\multiput(284.17,218.55)(2.000,-1.547){2}{\rule{0.400pt}{0.350pt}}
\multiput(287.00,215.95)(0.462,-0.447){3}{\rule{0.500pt}{0.108pt}}
\multiput(287.00,216.17)(1.962,-3.000){2}{\rule{0.250pt}{0.400pt}}
\put(290,212.17){\rule{0.700pt}{0.400pt}}
\multiput(290.00,213.17)(1.547,-2.000){2}{\rule{0.350pt}{0.400pt}}
\put(293.17,209){\rule{0.400pt}{0.700pt}}
\multiput(292.17,210.55)(2.000,-1.547){2}{\rule{0.400pt}{0.350pt}}
\multiput(295.00,207.95)(0.462,-0.447){3}{\rule{0.500pt}{0.108pt}}
\multiput(295.00,208.17)(1.962,-3.000){2}{\rule{0.250pt}{0.400pt}}
\put(298,204.17){\rule{0.482pt}{0.400pt}}
\multiput(298.00,205.17)(1.000,-2.000){2}{\rule{0.241pt}{0.400pt}}
\multiput(300.00,202.95)(0.462,-0.447){3}{\rule{0.500pt}{0.108pt}}
\multiput(300.00,203.17)(1.962,-3.000){2}{\rule{0.250pt}{0.400pt}}
\multiput(303.00,199.95)(0.462,-0.447){3}{\rule{0.500pt}{0.108pt}}
\multiput(303.00,200.17)(1.962,-3.000){2}{\rule{0.250pt}{0.400pt}}
\put(306,196.17){\rule{0.482pt}{0.400pt}}
\multiput(306.00,197.17)(1.000,-2.000){2}{\rule{0.241pt}{0.400pt}}
\multiput(308.00,194.95)(0.462,-0.447){3}{\rule{0.500pt}{0.108pt}}
\multiput(308.00,195.17)(1.962,-3.000){2}{\rule{0.250pt}{0.400pt}}
\put(311,191.17){\rule{0.482pt}{0.400pt}}
\multiput(311.00,192.17)(1.000,-2.000){2}{\rule{0.241pt}{0.400pt}}
\multiput(313.00,189.95)(0.462,-0.447){3}{\rule{0.500pt}{0.108pt}}
\multiput(313.00,190.17)(1.962,-3.000){2}{\rule{0.250pt}{0.400pt}}
\multiput(316.00,186.95)(0.462,-0.447){3}{\rule{0.500pt}{0.108pt}}
\multiput(316.00,187.17)(1.962,-3.000){2}{\rule{0.250pt}{0.400pt}}
\put(319,183.17){\rule{0.482pt}{0.400pt}}
\multiput(319.00,184.17)(1.000,-2.000){2}{\rule{0.241pt}{0.400pt}}
\multiput(321.00,181.95)(0.462,-0.447){3}{\rule{0.500pt}{0.108pt}}
\multiput(321.00,182.17)(1.962,-3.000){2}{\rule{0.250pt}{0.400pt}}
\put(324.17,177){\rule{0.400pt}{0.700pt}}
\multiput(323.17,178.55)(2.000,-1.547){2}{\rule{0.400pt}{0.350pt}}
\put(326,175.17){\rule{0.700pt}{0.400pt}}
\multiput(326.00,176.17)(1.547,-2.000){2}{\rule{0.350pt}{0.400pt}}
\multiput(329.00,173.95)(0.462,-0.447){3}{\rule{0.500pt}{0.108pt}}
\multiput(329.00,174.17)(1.962,-3.000){2}{\rule{0.250pt}{0.400pt}}
\put(332.17,169){\rule{0.400pt}{0.700pt}}
\multiput(331.17,170.55)(2.000,-1.547){2}{\rule{0.400pt}{0.350pt}}
\put(334,167.17){\rule{0.700pt}{0.400pt}}
\multiput(334.00,168.17)(1.547,-2.000){2}{\rule{0.350pt}{0.400pt}}
\multiput(337.00,165.95)(0.462,-0.447){3}{\rule{0.500pt}{0.108pt}}
\multiput(337.00,166.17)(1.962,-3.000){2}{\rule{0.250pt}{0.400pt}}
\put(340.17,161){\rule{0.400pt}{0.700pt}}
\multiput(339.17,162.55)(2.000,-1.547){2}{\rule{0.400pt}{0.350pt}}
\put(342,159.17){\rule{0.700pt}{0.400pt}}
\multiput(342.00,160.17)(1.547,-2.000){2}{\rule{0.350pt}{0.400pt}}
\put(345.17,156){\rule{0.400pt}{0.700pt}}
\multiput(344.17,157.55)(2.000,-1.547){2}{\rule{0.400pt}{0.350pt}}
\multiput(347.00,154.95)(0.462,-0.447){3}{\rule{0.500pt}{0.108pt}}
\multiput(347.00,155.17)(1.962,-3.000){2}{\rule{0.250pt}{0.400pt}}
\put(350,151.17){\rule{0.700pt}{0.400pt}}
\multiput(350.00,152.17)(1.547,-2.000){2}{\rule{0.350pt}{0.400pt}}
\put(353.17,148){\rule{0.400pt}{0.700pt}}
\multiput(352.17,149.55)(2.000,-1.547){2}{\rule{0.400pt}{0.350pt}}
\multiput(355.00,146.95)(0.462,-0.447){3}{\rule{0.500pt}{0.108pt}}
\multiput(355.00,147.17)(1.962,-3.000){2}{\rule{0.250pt}{0.400pt}}
\put(358,143.17){\rule{0.482pt}{0.400pt}}
\multiput(358.00,144.17)(1.000,-2.000){2}{\rule{0.241pt}{0.400pt}}
\multiput(360.00,141.95)(0.462,-0.447){3}{\rule{0.500pt}{0.108pt}}
\multiput(360.00,142.17)(1.962,-3.000){2}{\rule{0.250pt}{0.400pt}}
\multiput(363.00,138.95)(0.462,-0.447){3}{\rule{0.500pt}{0.108pt}}
\multiput(363.00,139.17)(1.962,-3.000){2}{\rule{0.250pt}{0.400pt}}
\put(366,135.17){\rule{0.482pt}{0.400pt}}
\multiput(366.00,136.17)(1.000,-2.000){2}{\rule{0.241pt}{0.400pt}}
\multiput(368.00,133.95)(0.462,-0.447){3}{\rule{0.500pt}{0.108pt}}
\multiput(368.00,134.17)(1.962,-3.000){2}{\rule{0.250pt}{0.400pt}}
\put(371,130.17){\rule{0.700pt}{0.400pt}}
\multiput(371.00,131.17)(1.547,-2.000){2}{\rule{0.350pt}{0.400pt}}
\put(374.17,127){\rule{0.400pt}{0.700pt}}
\multiput(373.17,128.55)(2.000,-1.547){2}{\rule{0.400pt}{0.350pt}}
\multiput(376.00,125.95)(0.462,-0.447){3}{\rule{0.500pt}{0.108pt}}
\multiput(376.00,126.17)(1.962,-3.000){2}{\rule{0.250pt}{0.400pt}}
\put(379,122.17){\rule{0.482pt}{0.400pt}}
\multiput(379.00,123.17)(1.000,-2.000){2}{\rule{0.241pt}{0.400pt}}
\multiput(381.00,120.95)(0.462,-0.447){3}{\rule{0.500pt}{0.108pt}}
\multiput(381.00,121.17)(1.962,-3.000){2}{\rule{0.250pt}{0.400pt}}
\multiput(384.00,117.95)(0.462,-0.447){3}{\rule{0.500pt}{0.108pt}}
\multiput(384.00,118.17)(1.962,-3.000){2}{\rule{0.250pt}{0.400pt}}
\put(387,114.17){\rule{0.482pt}{0.400pt}}
\multiput(387.00,115.17)(1.000,-2.000){2}{\rule{0.241pt}{0.400pt}}
\multiput(389.00,112.95)(0.462,-0.447){3}{\rule{0.500pt}{0.108pt}}
\multiput(389.00,113.17)(1.962,-3.000){2}{\rule{0.250pt}{0.400pt}}
\put(392.17,108){\rule{0.400pt}{0.700pt}}
\multiput(391.17,109.55)(2.000,-1.547){2}{\rule{0.400pt}{0.350pt}}
\put(394,106.17){\rule{0.700pt}{0.400pt}}
\multiput(394.00,107.17)(1.547,-2.000){2}{\rule{0.350pt}{0.400pt}}
\multiput(397.00,104.95)(0.462,-0.447){3}{\rule{0.500pt}{0.108pt}}
\multiput(397.00,105.17)(1.962,-3.000){2}{\rule{0.250pt}{0.400pt}}
\put(400.17,103){\rule{0.400pt}{0.700pt}}
\multiput(399.17,103.00)(2.000,1.547){2}{\rule{0.400pt}{0.350pt}}
\put(402,106.17){\rule{0.700pt}{0.400pt}}
\multiput(402.00,105.17)(1.547,2.000){2}{\rule{0.350pt}{0.400pt}}
\multiput(405.00,108.61)(0.462,0.447){3}{\rule{0.500pt}{0.108pt}}
\multiput(405.00,107.17)(1.962,3.000){2}{\rule{0.250pt}{0.400pt}}
\put(408.17,111){\rule{0.400pt}{0.700pt}}
\multiput(407.17,111.00)(2.000,1.547){2}{\rule{0.400pt}{0.350pt}}
\put(410,114.17){\rule{0.700pt}{0.400pt}}
\multiput(410.00,113.17)(1.547,2.000){2}{\rule{0.350pt}{0.400pt}}
\put(413.17,116){\rule{0.400pt}{0.700pt}}
\multiput(412.17,116.00)(2.000,1.547){2}{\rule{0.400pt}{0.350pt}}
\multiput(415.00,119.61)(0.462,0.447){3}{\rule{0.500pt}{0.108pt}}
\multiput(415.00,118.17)(1.962,3.000){2}{\rule{0.250pt}{0.400pt}}
\put(418,122.17){\rule{0.700pt}{0.400pt}}
\multiput(418.00,121.17)(1.547,2.000){2}{\rule{0.350pt}{0.400pt}}
\put(421.17,124){\rule{0.400pt}{0.700pt}}
\multiput(420.17,124.00)(2.000,1.547){2}{\rule{0.400pt}{0.350pt}}
\multiput(423.00,127.61)(0.462,0.447){3}{\rule{0.500pt}{0.108pt}}
\multiput(423.00,126.17)(1.962,3.000){2}{\rule{0.250pt}{0.400pt}}
\put(426,130.17){\rule{0.482pt}{0.400pt}}
\multiput(426.00,129.17)(1.000,2.000){2}{\rule{0.241pt}{0.400pt}}
\multiput(428.00,132.61)(0.462,0.447){3}{\rule{0.500pt}{0.108pt}}
\multiput(428.00,131.17)(1.962,3.000){2}{\rule{0.250pt}{0.400pt}}
\put(431,135.17){\rule{0.700pt}{0.400pt}}
\multiput(431.00,134.17)(1.547,2.000){2}{\rule{0.350pt}{0.400pt}}
\put(434.17,137){\rule{0.400pt}{0.700pt}}
\multiput(433.17,137.00)(2.000,1.547){2}{\rule{0.400pt}{0.350pt}}
\multiput(436.00,140.61)(0.462,0.447){3}{\rule{0.500pt}{0.108pt}}
\multiput(436.00,139.17)(1.962,3.000){2}{\rule{0.250pt}{0.400pt}}
\put(439,143.17){\rule{0.482pt}{0.400pt}}
\multiput(439.00,142.17)(1.000,2.000){2}{\rule{0.241pt}{0.400pt}}
\multiput(441.00,145.61)(0.462,0.447){3}{\rule{0.500pt}{0.108pt}}
\multiput(441.00,144.17)(1.962,3.000){2}{\rule{0.250pt}{0.400pt}}
\multiput(444.00,148.61)(0.462,0.447){3}{\rule{0.500pt}{0.108pt}}
\multiput(444.00,147.17)(1.962,3.000){2}{\rule{0.250pt}{0.400pt}}
\put(447,151.17){\rule{0.482pt}{0.400pt}}
\multiput(447.00,150.17)(1.000,2.000){2}{\rule{0.241pt}{0.400pt}}
\multiput(449.00,153.61)(0.462,0.447){3}{\rule{0.500pt}{0.108pt}}
\multiput(449.00,152.17)(1.962,3.000){2}{\rule{0.250pt}{0.400pt}}
\multiput(452.00,156.61)(0.462,0.447){3}{\rule{0.500pt}{0.108pt}}
\multiput(452.00,155.17)(1.962,3.000){2}{\rule{0.250pt}{0.400pt}}
\put(455,159.17){\rule{0.482pt}{0.400pt}}
\multiput(455.00,158.17)(1.000,2.000){2}{\rule{0.241pt}{0.400pt}}
\multiput(457.00,161.61)(0.462,0.447){3}{\rule{0.500pt}{0.108pt}}
\multiput(457.00,160.17)(1.962,3.000){2}{\rule{0.250pt}{0.400pt}}
\put(460.17,164){\rule{0.400pt}{0.700pt}}
\multiput(459.17,164.00)(2.000,1.547){2}{\rule{0.400pt}{0.350pt}}
\put(462,167.17){\rule{0.700pt}{0.400pt}}
\multiput(462.00,166.17)(1.547,2.000){2}{\rule{0.350pt}{0.400pt}}
\multiput(465.00,169.61)(0.462,0.447){3}{\rule{0.500pt}{0.108pt}}
\multiput(465.00,168.17)(1.962,3.000){2}{\rule{0.250pt}{0.400pt}}
\put(468.17,172){\rule{0.400pt}{0.700pt}}
\multiput(467.17,172.00)(2.000,1.547){2}{\rule{0.400pt}{0.350pt}}
\put(470,175.17){\rule{0.700pt}{0.400pt}}
\multiput(470.00,174.17)(1.547,2.000){2}{\rule{0.350pt}{0.400pt}}
\put(473.17,177){\rule{0.400pt}{0.700pt}}
\multiput(472.17,177.00)(2.000,1.547){2}{\rule{0.400pt}{0.350pt}}
\multiput(475.00,180.61)(0.462,0.447){3}{\rule{0.500pt}{0.108pt}}
\multiput(475.00,179.17)(1.962,3.000){2}{\rule{0.250pt}{0.400pt}}
\put(478,183.17){\rule{0.700pt}{0.400pt}}
\multiput(478.00,182.17)(1.547,2.000){2}{\rule{0.350pt}{0.400pt}}
\put(481.17,185){\rule{0.400pt}{0.700pt}}
\multiput(480.17,185.00)(2.000,1.547){2}{\rule{0.400pt}{0.350pt}}
\multiput(483.00,188.61)(0.462,0.447){3}{\rule{0.500pt}{0.108pt}}
\multiput(483.00,187.17)(1.962,3.000){2}{\rule{0.250pt}{0.400pt}}
\put(486,191.17){\rule{0.700pt}{0.400pt}}
\multiput(486.00,190.17)(1.547,2.000){2}{\rule{0.350pt}{0.400pt}}
\put(489.17,193){\rule{0.400pt}{0.700pt}}
\multiput(488.17,193.00)(2.000,1.547){2}{\rule{0.400pt}{0.350pt}}
\put(491,196.17){\rule{0.700pt}{0.400pt}}
\multiput(491.00,195.17)(1.547,2.000){2}{\rule{0.350pt}{0.400pt}}
\put(494.17,198){\rule{0.400pt}{0.700pt}}
\multiput(493.17,198.00)(2.000,1.547){2}{\rule{0.400pt}{0.350pt}}
\multiput(496.00,201.61)(0.462,0.447){3}{\rule{0.500pt}{0.108pt}}
\multiput(496.00,200.17)(1.962,3.000){2}{\rule{0.250pt}{0.400pt}}
\put(499,204.17){\rule{0.700pt}{0.400pt}}
\multiput(499.00,203.17)(1.547,2.000){2}{\rule{0.350pt}{0.400pt}}
\put(502.17,206){\rule{0.400pt}{0.700pt}}
\multiput(501.17,206.00)(2.000,1.547){2}{\rule{0.400pt}{0.350pt}}
\multiput(504.00,209.61)(0.462,0.447){3}{\rule{0.500pt}{0.108pt}}
\multiput(504.00,208.17)(1.962,3.000){2}{\rule{0.250pt}{0.400pt}}
\put(507,212.17){\rule{0.482pt}{0.400pt}}
\multiput(507.00,211.17)(1.000,2.000){2}{\rule{0.241pt}{0.400pt}}
\multiput(509.00,214.61)(0.462,0.447){3}{\rule{0.500pt}{0.108pt}}
\multiput(509.00,213.17)(1.962,3.000){2}{\rule{0.250pt}{0.400pt}}
\multiput(512.00,217.61)(0.462,0.447){3}{\rule{0.500pt}{0.108pt}}
\multiput(512.00,216.17)(1.962,3.000){2}{\rule{0.250pt}{0.400pt}}
\put(515,220.17){\rule{0.482pt}{0.400pt}}
\multiput(515.00,219.17)(1.000,2.000){2}{\rule{0.241pt}{0.400pt}}
\multiput(517.00,222.61)(0.462,0.447){3}{\rule{0.500pt}{0.108pt}}
\multiput(517.00,221.17)(1.962,3.000){2}{\rule{0.250pt}{0.400pt}}
\multiput(520.00,225.61)(0.462,0.447){3}{\rule{0.500pt}{0.108pt}}
\multiput(520.00,224.17)(1.962,3.000){2}{\rule{0.250pt}{0.400pt}}
\put(523,228.17){\rule{0.482pt}{0.400pt}}
\multiput(523.00,227.17)(1.000,2.000){2}{\rule{0.241pt}{0.400pt}}
\multiput(525.00,230.61)(0.462,0.447){3}{\rule{0.500pt}{0.108pt}}
\multiput(525.00,229.17)(1.962,3.000){2}{\rule{0.250pt}{0.400pt}}
\put(528.17,233){\rule{0.400pt}{0.700pt}}
\multiput(527.17,233.00)(2.000,1.547){2}{\rule{0.400pt}{0.350pt}}
\put(530,236.17){\rule{0.700pt}{0.400pt}}
\multiput(530.00,235.17)(1.547,2.000){2}{\rule{0.350pt}{0.400pt}}
\multiput(533.00,238.61)(0.462,0.447){3}{\rule{0.500pt}{0.108pt}}
\multiput(533.00,237.17)(1.962,3.000){2}{\rule{0.250pt}{0.400pt}}
\put(536.17,241){\rule{0.400pt}{0.700pt}}
\multiput(535.17,241.00)(2.000,1.547){2}{\rule{0.400pt}{0.350pt}}
\put(538,244.17){\rule{0.700pt}{0.400pt}}
\multiput(538.00,243.17)(1.547,2.000){2}{\rule{0.350pt}{0.400pt}}
\put(541.17,246){\rule{0.400pt}{0.700pt}}
\multiput(540.17,246.00)(2.000,1.547){2}{\rule{0.400pt}{0.350pt}}
\multiput(543.00,249.61)(0.462,0.447){3}{\rule{0.500pt}{0.108pt}}
\multiput(543.00,248.17)(1.962,3.000){2}{\rule{0.250pt}{0.400pt}}
\put(546,252.17){\rule{0.700pt}{0.400pt}}
\multiput(546.00,251.17)(1.547,2.000){2}{\rule{0.350pt}{0.400pt}}
\put(549.17,254){\rule{0.400pt}{0.700pt}}
\multiput(548.17,254.00)(2.000,1.547){2}{\rule{0.400pt}{0.350pt}}
\put(551,257.17){\rule{0.700pt}{0.400pt}}
\multiput(551.00,256.17)(1.547,2.000){2}{\rule{0.350pt}{0.400pt}}
\put(554.17,259){\rule{0.400pt}{0.700pt}}
\multiput(553.17,259.00)(2.000,1.547){2}{\rule{0.400pt}{0.350pt}}
\multiput(556.00,262.61)(0.462,0.447){3}{\rule{0.500pt}{0.108pt}}
\multiput(556.00,261.17)(1.962,3.000){2}{\rule{0.250pt}{0.400pt}}
\put(559,265.17){\rule{0.700pt}{0.400pt}}
\multiput(559.00,264.17)(1.547,2.000){2}{\rule{0.350pt}{0.400pt}}
\put(562.17,267){\rule{0.400pt}{0.700pt}}
\multiput(561.17,267.00)(2.000,1.547){2}{\rule{0.400pt}{0.350pt}}
\multiput(564.00,270.61)(0.462,0.447){3}{\rule{0.500pt}{0.108pt}}
\multiput(564.00,269.17)(1.962,3.000){2}{\rule{0.250pt}{0.400pt}}
\put(567,273.17){\rule{0.700pt}{0.400pt}}
\multiput(567.00,272.17)(1.547,2.000){2}{\rule{0.350pt}{0.400pt}}
\put(570.17,275){\rule{0.400pt}{0.700pt}}
\multiput(569.17,275.00)(2.000,1.547){2}{\rule{0.400pt}{0.350pt}}
\multiput(572.00,278.61)(0.462,0.447){3}{\rule{0.500pt}{0.108pt}}
\multiput(572.00,277.17)(1.962,3.000){2}{\rule{0.250pt}{0.400pt}}
\put(575,281.17){\rule{0.482pt}{0.400pt}}
\multiput(575.00,280.17)(1.000,2.000){2}{\rule{0.241pt}{0.400pt}}
\multiput(577.00,283.61)(0.462,0.447){3}{\rule{0.500pt}{0.108pt}}
\multiput(577.00,282.17)(1.962,3.000){2}{\rule{0.250pt}{0.400pt}}
\multiput(580.00,286.61)(0.462,0.447){3}{\rule{0.500pt}{0.108pt}}
\multiput(580.00,285.17)(1.962,3.000){2}{\rule{0.250pt}{0.400pt}}
\put(583,289.17){\rule{0.482pt}{0.400pt}}
\multiput(583.00,288.17)(1.000,2.000){2}{\rule{0.241pt}{0.400pt}}
\multiput(585.00,291.61)(0.462,0.447){3}{\rule{0.500pt}{0.108pt}}
\multiput(585.00,290.17)(1.962,3.000){2}{\rule{0.250pt}{0.400pt}}
\put(588.17,294){\rule{0.400pt}{0.700pt}}
\multiput(587.17,294.00)(2.000,1.547){2}{\rule{0.400pt}{0.350pt}}
\put(590,297.17){\rule{0.700pt}{0.400pt}}
\multiput(590.00,296.17)(1.547,2.000){2}{\rule{0.350pt}{0.400pt}}
\multiput(593.00,299.61)(0.462,0.447){3}{\rule{0.500pt}{0.108pt}}
\multiput(593.00,298.17)(1.962,3.000){2}{\rule{0.250pt}{0.400pt}}
\put(596.17,302){\rule{0.400pt}{0.700pt}}
\multiput(595.17,302.00)(2.000,1.547){2}{\rule{0.400pt}{0.350pt}}
\put(598,305.17){\rule{0.700pt}{0.400pt}}
\multiput(598.00,304.17)(1.547,2.000){2}{\rule{0.350pt}{0.400pt}}
\multiput(601.00,307.61)(0.462,0.447){3}{\rule{0.500pt}{0.108pt}}
\multiput(601.00,306.17)(1.962,3.000){2}{\rule{0.250pt}{0.400pt}}
\put(604.17,310){\rule{0.400pt}{0.700pt}}
\multiput(603.17,310.00)(2.000,1.547){2}{\rule{0.400pt}{0.350pt}}
\put(606,313.17){\rule{0.700pt}{0.400pt}}
\multiput(606.00,312.17)(1.547,2.000){2}{\rule{0.350pt}{0.400pt}}
\put(609.17,315){\rule{0.400pt}{0.700pt}}
\multiput(608.17,315.00)(2.000,1.547){2}{\rule{0.400pt}{0.350pt}}
\put(611,318.17){\rule{0.700pt}{0.400pt}}
\multiput(611.00,317.17)(1.547,2.000){2}{\rule{0.350pt}{0.400pt}}
\multiput(614.00,320.61)(0.462,0.447){3}{\rule{0.500pt}{0.108pt}}
\multiput(614.00,319.17)(1.962,3.000){2}{\rule{0.250pt}{0.400pt}}
\put(617.17,323){\rule{0.400pt}{0.700pt}}
\multiput(616.17,323.00)(2.000,1.547){2}{\rule{0.400pt}{0.350pt}}
\put(619,326.17){\rule{0.700pt}{0.400pt}}
\multiput(619.00,325.17)(1.547,2.000){2}{\rule{0.350pt}{0.400pt}}
\put(622.17,328){\rule{0.400pt}{0.700pt}}
\multiput(621.17,328.00)(2.000,1.547){2}{\rule{0.400pt}{0.350pt}}
\multiput(624.00,331.61)(0.462,0.447){3}{\rule{0.500pt}{0.108pt}}
\multiput(624.00,330.17)(1.962,3.000){2}{\rule{0.250pt}{0.400pt}}
\put(627,334.17){\rule{0.700pt}{0.400pt}}
\multiput(627.00,333.17)(1.547,2.000){2}{\rule{0.350pt}{0.400pt}}
\put(630.17,336){\rule{0.400pt}{0.700pt}}
\multiput(629.17,336.00)(2.000,1.547){2}{\rule{0.400pt}{0.350pt}}
\multiput(632.00,339.61)(0.462,0.447){3}{\rule{0.500pt}{0.108pt}}
\multiput(632.00,338.17)(1.962,3.000){2}{\rule{0.250pt}{0.400pt}}
\put(635,342.17){\rule{0.700pt}{0.400pt}}
\multiput(635.00,341.17)(1.547,2.000){2}{\rule{0.350pt}{0.400pt}}
\put(638.17,344){\rule{0.400pt}{0.700pt}}
\multiput(637.17,344.00)(2.000,1.547){2}{\rule{0.400pt}{0.350pt}}
\multiput(640.00,347.61)(0.462,0.447){3}{\rule{0.500pt}{0.108pt}}
\multiput(640.00,346.17)(1.962,3.000){2}{\rule{0.250pt}{0.400pt}}
\put(643,350.17){\rule{0.482pt}{0.400pt}}
\multiput(643.00,349.17)(1.000,2.000){2}{\rule{0.241pt}{0.400pt}}
\multiput(645.00,352.61)(0.462,0.447){3}{\rule{0.500pt}{0.108pt}}
\multiput(645.00,351.17)(1.962,3.000){2}{\rule{0.250pt}{0.400pt}}
\multiput(648.00,355.61)(0.462,0.447){3}{\rule{0.500pt}{0.108pt}}
\multiput(648.00,354.17)(1.962,3.000){2}{\rule{0.250pt}{0.400pt}}
\put(651,358.17){\rule{0.482pt}{0.400pt}}
\multiput(651.00,357.17)(1.000,2.000){2}{\rule{0.241pt}{0.400pt}}
\multiput(653.00,360.61)(0.462,0.447){3}{\rule{0.500pt}{0.108pt}}
\multiput(653.00,359.17)(1.962,3.000){2}{\rule{0.250pt}{0.400pt}}
\put(656.17,363){\rule{0.400pt}{0.700pt}}
\multiput(655.17,363.00)(2.000,1.547){2}{\rule{0.400pt}{0.350pt}}
\put(658,366.17){\rule{0.700pt}{0.400pt}}
\multiput(658.00,365.17)(1.547,2.000){2}{\rule{0.350pt}{0.400pt}}
\multiput(661.00,368.61)(0.462,0.447){3}{\rule{0.500pt}{0.108pt}}
\multiput(661.00,367.17)(1.962,3.000){2}{\rule{0.250pt}{0.400pt}}
\put(664.17,371){\rule{0.400pt}{0.700pt}}
\multiput(663.17,371.00)(2.000,1.547){2}{\rule{0.400pt}{0.350pt}}
\put(666,374.17){\rule{0.700pt}{0.400pt}}
\multiput(666.00,373.17)(1.547,2.000){2}{\rule{0.350pt}{0.400pt}}
\put(669.17,376){\rule{0.400pt}{0.700pt}}
\multiput(668.17,376.00)(2.000,1.547){2}{\rule{0.400pt}{0.350pt}}
\put(671,379.17){\rule{0.700pt}{0.400pt}}
\multiput(671.00,378.17)(1.547,2.000){2}{\rule{0.350pt}{0.400pt}}
\multiput(674.00,381.61)(0.462,0.447){3}{\rule{0.500pt}{0.108pt}}
\multiput(674.00,380.17)(1.962,3.000){2}{\rule{0.250pt}{0.400pt}}
\put(677.17,384){\rule{0.400pt}{0.700pt}}
\multiput(676.17,384.00)(2.000,1.547){2}{\rule{0.400pt}{0.350pt}}
\put(679,387.17){\rule{0.700pt}{0.400pt}}
\multiput(679.00,386.17)(1.547,2.000){2}{\rule{0.350pt}{0.400pt}}
\multiput(682.00,389.61)(0.462,0.447){3}{\rule{0.500pt}{0.108pt}}
\multiput(682.00,388.17)(1.962,3.000){2}{\rule{0.250pt}{0.400pt}}
\put(685.17,392){\rule{0.400pt}{0.700pt}}
\multiput(684.17,392.00)(2.000,1.547){2}{\rule{0.400pt}{0.350pt}}
\put(687,395.17){\rule{0.700pt}{0.400pt}}
\multiput(687.00,394.17)(1.547,2.000){2}{\rule{0.350pt}{0.400pt}}
\put(690.17,397){\rule{0.400pt}{0.700pt}}
\multiput(689.17,397.00)(2.000,1.547){2}{\rule{0.400pt}{0.350pt}}
\multiput(692.00,400.61)(0.462,0.447){3}{\rule{0.500pt}{0.108pt}}
\multiput(692.00,399.17)(1.962,3.000){2}{\rule{0.250pt}{0.400pt}}
\put(695,403.17){\rule{0.700pt}{0.400pt}}
\multiput(695.00,402.17)(1.547,2.000){2}{\rule{0.350pt}{0.400pt}}
\put(698.17,405){\rule{0.400pt}{0.700pt}}
\multiput(697.17,405.00)(2.000,1.547){2}{\rule{0.400pt}{0.350pt}}
\multiput(700.00,408.61)(0.462,0.447){3}{\rule{0.500pt}{0.108pt}}
\multiput(700.00,407.17)(1.962,3.000){2}{\rule{0.250pt}{0.400pt}}
\put(703,411.17){\rule{0.482pt}{0.400pt}}
\multiput(703.00,410.17)(1.000,2.000){2}{\rule{0.241pt}{0.400pt}}
\multiput(705.00,413.61)(0.462,0.447){3}{\rule{0.500pt}{0.108pt}}
\multiput(705.00,412.17)(1.962,3.000){2}{\rule{0.250pt}{0.400pt}}
\multiput(708.00,416.61)(0.462,0.447){3}{\rule{0.500pt}{0.108pt}}
\multiput(708.00,415.17)(1.962,3.000){2}{\rule{0.250pt}{0.400pt}}
\put(711,419.17){\rule{0.482pt}{0.400pt}}
\multiput(711.00,418.17)(1.000,2.000){2}{\rule{0.241pt}{0.400pt}}
\multiput(713.00,421.61)(0.462,0.447){3}{\rule{0.500pt}{0.108pt}}
\multiput(713.00,420.17)(1.962,3.000){2}{\rule{0.250pt}{0.400pt}}
\multiput(716.00,424.61)(0.462,0.447){3}{\rule{0.500pt}{0.108pt}}
\multiput(716.00,423.17)(1.962,3.000){2}{\rule{0.250pt}{0.400pt}}
\put(719,427.17){\rule{0.482pt}{0.400pt}}
\multiput(719.00,426.17)(1.000,2.000){2}{\rule{0.241pt}{0.400pt}}
\multiput(721.00,429.61)(0.462,0.447){3}{\rule{0.500pt}{0.108pt}}
\multiput(721.00,428.17)(1.962,3.000){2}{\rule{0.250pt}{0.400pt}}
\put(724.17,432){\rule{0.400pt}{0.700pt}}
\multiput(723.17,432.00)(2.000,1.547){2}{\rule{0.400pt}{0.350pt}}
\put(726,435.17){\rule{0.700pt}{0.400pt}}
\multiput(726.00,434.17)(1.547,2.000){2}{\rule{0.350pt}{0.400pt}}
\multiput(729.00,437.61)(0.462,0.447){3}{\rule{0.500pt}{0.108pt}}
\multiput(729.00,436.17)(1.962,3.000){2}{\rule{0.250pt}{0.400pt}}
\put(732,440.17){\rule{0.482pt}{0.400pt}}
\multiput(732.00,439.17)(1.000,2.000){2}{\rule{0.241pt}{0.400pt}}
\multiput(734.00,442.61)(0.462,0.447){3}{\rule{0.500pt}{0.108pt}}
\multiput(734.00,441.17)(1.962,3.000){2}{\rule{0.250pt}{0.400pt}}
\put(737.17,445){\rule{0.400pt}{0.700pt}}
\multiput(736.17,445.00)(2.000,1.547){2}{\rule{0.400pt}{0.350pt}}
\put(739,448.17){\rule{0.700pt}{0.400pt}}
\multiput(739.00,447.17)(1.547,2.000){2}{\rule{0.350pt}{0.400pt}}
\multiput(742.00,450.61)(0.462,0.447){3}{\rule{0.500pt}{0.108pt}}
\multiput(742.00,449.17)(1.962,3.000){2}{\rule{0.250pt}{0.400pt}}
\put(745.17,453){\rule{0.400pt}{0.700pt}}
\multiput(744.17,453.00)(2.000,1.547){2}{\rule{0.400pt}{0.350pt}}
\put(747,456.17){\rule{0.700pt}{0.400pt}}
\multiput(747.00,455.17)(1.547,2.000){2}{\rule{0.350pt}{0.400pt}}
\multiput(750.00,458.61)(0.462,0.447){3}{\rule{0.500pt}{0.108pt}}
\multiput(750.00,457.17)(1.962,3.000){2}{\rule{0.250pt}{0.400pt}}
\put(753.17,461){\rule{0.400pt}{0.700pt}}
\multiput(752.17,461.00)(2.000,1.547){2}{\rule{0.400pt}{0.350pt}}
\put(755,464.17){\rule{0.700pt}{0.400pt}}
\multiput(755.00,463.17)(1.547,2.000){2}{\rule{0.350pt}{0.400pt}}
\put(758.17,466){\rule{0.400pt}{0.700pt}}
\multiput(757.17,466.00)(2.000,1.547){2}{\rule{0.400pt}{0.350pt}}
\multiput(760.00,469.61)(0.462,0.447){3}{\rule{0.500pt}{0.108pt}}
\multiput(760.00,468.17)(1.962,3.000){2}{\rule{0.250pt}{0.400pt}}
\put(763,472.17){\rule{0.700pt}{0.400pt}}
\multiput(763.00,471.17)(1.547,2.000){2}{\rule{0.350pt}{0.400pt}}
\put(766.17,474){\rule{0.400pt}{0.700pt}}
\multiput(765.17,474.00)(2.000,1.547){2}{\rule{0.400pt}{0.350pt}}
\multiput(768.00,477.61)(0.462,0.447){3}{\rule{0.500pt}{0.108pt}}
\multiput(768.00,476.17)(1.962,3.000){2}{\rule{0.250pt}{0.400pt}}
\put(771,480.17){\rule{0.482pt}{0.400pt}}
\multiput(771.00,479.17)(1.000,2.000){2}{\rule{0.241pt}{0.400pt}}
\multiput(773.00,482.61)(0.462,0.447){3}{\rule{0.500pt}{0.108pt}}
\multiput(773.00,481.17)(1.962,3.000){2}{\rule{0.250pt}{0.400pt}}
\multiput(776.00,485.61)(0.462,0.447){3}{\rule{0.500pt}{0.108pt}}
\multiput(776.00,484.17)(1.962,3.000){2}{\rule{0.250pt}{0.400pt}}
\put(779,488.17){\rule{0.482pt}{0.400pt}}
\multiput(779.00,487.17)(1.000,2.000){2}{\rule{0.241pt}{0.400pt}}
\multiput(781.00,490.61)(0.462,0.447){3}{\rule{0.500pt}{0.108pt}}
\multiput(781.00,489.17)(1.962,3.000){2}{\rule{0.250pt}{0.400pt}}
\put(784.17,493){\rule{0.400pt}{0.700pt}}
\multiput(783.17,493.00)(2.000,1.547){2}{\rule{0.400pt}{0.350pt}}
\put(786,496.17){\rule{0.700pt}{0.400pt}}
\multiput(786.00,495.17)(1.547,2.000){2}{\rule{0.350pt}{0.400pt}}
\multiput(789.00,498.61)(0.462,0.447){3}{\rule{0.500pt}{0.108pt}}
\multiput(789.00,497.17)(1.962,3.000){2}{\rule{0.250pt}{0.400pt}}
\put(792,501.17){\rule{0.482pt}{0.400pt}}
\multiput(792.00,500.17)(1.000,2.000){2}{\rule{0.241pt}{0.400pt}}
\multiput(794.00,503.61)(0.462,0.447){3}{\rule{0.500pt}{0.108pt}}
\multiput(794.00,502.17)(1.962,3.000){2}{\rule{0.250pt}{0.400pt}}
\multiput(797.00,506.61)(0.462,0.447){3}{\rule{0.500pt}{0.108pt}}
\multiput(797.00,505.17)(1.962,3.000){2}{\rule{0.250pt}{0.400pt}}
\put(800,509.17){\rule{0.482pt}{0.400pt}}
\multiput(800.00,508.17)(1.000,2.000){2}{\rule{0.241pt}{0.400pt}}
\multiput(802.00,511.61)(0.462,0.447){3}{\rule{0.500pt}{0.108pt}}
\multiput(802.00,510.17)(1.962,3.000){2}{\rule{0.250pt}{0.400pt}}
\put(805.17,514){\rule{0.400pt}{0.700pt}}
\multiput(804.17,514.00)(2.000,1.547){2}{\rule{0.400pt}{0.350pt}}
\put(807,517.17){\rule{0.700pt}{0.400pt}}
\multiput(807.00,516.17)(1.547,2.000){2}{\rule{0.350pt}{0.400pt}}
\multiput(810.00,519.61)(0.462,0.447){3}{\rule{0.500pt}{0.108pt}}
\multiput(810.00,518.17)(1.962,3.000){2}{\rule{0.250pt}{0.400pt}}
\put(813.17,522){\rule{0.400pt}{0.700pt}}
\multiput(812.17,522.00)(2.000,1.547){2}{\rule{0.400pt}{0.350pt}}
\put(815,525.17){\rule{0.700pt}{0.400pt}}
\multiput(815.00,524.17)(1.547,2.000){2}{\rule{0.350pt}{0.400pt}}
\put(818.17,527){\rule{0.400pt}{0.700pt}}
\multiput(817.17,527.00)(2.000,1.547){2}{\rule{0.400pt}{0.350pt}}
\multiput(820.00,530.61)(0.462,0.447){3}{\rule{0.500pt}{0.108pt}}
\multiput(820.00,529.17)(1.962,3.000){2}{\rule{0.250pt}{0.400pt}}
\put(823,533.17){\rule{0.700pt}{0.400pt}}
\multiput(823.00,532.17)(1.547,2.000){2}{\rule{0.350pt}{0.400pt}}
\put(826.17,535){\rule{0.400pt}{0.700pt}}
\multiput(825.17,535.00)(2.000,1.547){2}{\rule{0.400pt}{0.350pt}}
\multiput(828.00,538.61)(0.462,0.447){3}{\rule{0.500pt}{0.108pt}}
\multiput(828.00,537.17)(1.962,3.000){2}{\rule{0.250pt}{0.400pt}}
\put(831,541.17){\rule{0.700pt}{0.400pt}}
\multiput(831.00,540.17)(1.547,2.000){2}{\rule{0.350pt}{0.400pt}}
\put(834.17,543){\rule{0.400pt}{0.700pt}}
\multiput(833.17,543.00)(2.000,1.547){2}{\rule{0.400pt}{0.350pt}}
\multiput(836.00,546.61)(0.462,0.447){3}{\rule{0.500pt}{0.108pt}}
\multiput(836.00,545.17)(1.962,3.000){2}{\rule{0.250pt}{0.400pt}}
\put(839,549.17){\rule{0.482pt}{0.400pt}}
\multiput(839.00,548.17)(1.000,2.000){2}{\rule{0.241pt}{0.400pt}}
\multiput(841.00,551.61)(0.462,0.447){3}{\rule{0.500pt}{0.108pt}}
\multiput(841.00,550.17)(1.962,3.000){2}{\rule{0.250pt}{0.400pt}}
\multiput(844.00,554.61)(0.462,0.447){3}{\rule{0.500pt}{0.108pt}}
\multiput(844.00,553.17)(1.962,3.000){2}{\rule{0.250pt}{0.400pt}}
\put(847,557.17){\rule{0.482pt}{0.400pt}}
\multiput(847.00,556.17)(1.000,2.000){2}{\rule{0.241pt}{0.400pt}}
\multiput(849.00,559.61)(0.462,0.447){3}{\rule{0.500pt}{0.108pt}}
\multiput(849.00,558.17)(1.962,3.000){2}{\rule{0.250pt}{0.400pt}}
\put(852,562.17){\rule{0.482pt}{0.400pt}}
\multiput(852.00,561.17)(1.000,2.000){2}{\rule{0.241pt}{0.400pt}}
\multiput(854.00,564.61)(0.462,0.447){3}{\rule{0.500pt}{0.108pt}}
\multiput(854.00,563.17)(1.962,3.000){2}{\rule{0.250pt}{0.400pt}}
\multiput(857.00,567.61)(0.462,0.447){3}{\rule{0.500pt}{0.108pt}}
\multiput(857.00,566.17)(1.962,3.000){2}{\rule{0.250pt}{0.400pt}}
\put(860,570.17){\rule{0.482pt}{0.400pt}}
\multiput(860.00,569.17)(1.000,2.000){2}{\rule{0.241pt}{0.400pt}}
\multiput(862.00,572.61)(0.462,0.447){3}{\rule{0.500pt}{0.108pt}}
\multiput(862.00,571.17)(1.962,3.000){2}{\rule{0.250pt}{0.400pt}}
\multiput(865.00,575.61)(0.462,0.447){3}{\rule{0.500pt}{0.108pt}}
\multiput(865.00,574.17)(1.962,3.000){2}{\rule{0.250pt}{0.400pt}}
\put(868,578.17){\rule{0.482pt}{0.400pt}}
\multiput(868.00,577.17)(1.000,2.000){2}{\rule{0.241pt}{0.400pt}}
\multiput(870.00,580.61)(0.462,0.447){3}{\rule{0.500pt}{0.108pt}}
\multiput(870.00,579.17)(1.962,3.000){2}{\rule{0.250pt}{0.400pt}}
\put(873.17,583){\rule{0.400pt}{0.700pt}}
\multiput(872.17,583.00)(2.000,1.547){2}{\rule{0.400pt}{0.350pt}}
\put(875,586.17){\rule{0.700pt}{0.400pt}}
\multiput(875.00,585.17)(1.547,2.000){2}{\rule{0.350pt}{0.400pt}}
\multiput(878.00,588.61)(0.462,0.447){3}{\rule{0.500pt}{0.108pt}}
\multiput(878.00,587.17)(1.962,3.000){2}{\rule{0.250pt}{0.400pt}}
\put(881.17,591){\rule{0.400pt}{0.700pt}}
\multiput(880.17,591.00)(2.000,1.547){2}{\rule{0.400pt}{0.350pt}}
\put(883,594.17){\rule{0.700pt}{0.400pt}}
\multiput(883.00,593.17)(1.547,2.000){2}{\rule{0.350pt}{0.400pt}}
\put(886.17,596){\rule{0.400pt}{0.700pt}}
\multiput(885.17,596.00)(2.000,1.547){2}{\rule{0.400pt}{0.350pt}}
\multiput(888.00,599.61)(0.462,0.447){3}{\rule{0.500pt}{0.108pt}}
\multiput(888.00,598.17)(1.962,3.000){2}{\rule{0.250pt}{0.400pt}}
\put(891,602.17){\rule{0.700pt}{0.400pt}}
\multiput(891.00,601.17)(1.547,2.000){2}{\rule{0.350pt}{0.400pt}}
\put(894.17,604){\rule{0.400pt}{0.700pt}}
\multiput(893.17,604.00)(2.000,1.547){2}{\rule{0.400pt}{0.350pt}}
\multiput(896.00,607.61)(0.462,0.447){3}{\rule{0.500pt}{0.108pt}}
\multiput(896.00,606.17)(1.962,3.000){2}{\rule{0.250pt}{0.400pt}}
\put(899,610.17){\rule{0.482pt}{0.400pt}}
\multiput(899.00,609.17)(1.000,2.000){2}{\rule{0.241pt}{0.400pt}}
\multiput(901.00,612.61)(0.462,0.447){3}{\rule{0.500pt}{0.108pt}}
\multiput(901.00,611.17)(1.962,3.000){2}{\rule{0.250pt}{0.400pt}}
\multiput(904.00,615.61)(0.462,0.447){3}{\rule{0.500pt}{0.108pt}}
\multiput(904.00,614.17)(1.962,3.000){2}{\rule{0.250pt}{0.400pt}}
\put(907,618.17){\rule{0.482pt}{0.400pt}}
\multiput(907.00,617.17)(1.000,2.000){2}{\rule{0.241pt}{0.400pt}}
\multiput(909.00,620.61)(0.462,0.447){3}{\rule{0.500pt}{0.108pt}}
\multiput(909.00,619.17)(1.962,3.000){2}{\rule{0.250pt}{0.400pt}}
\put(912,623.17){\rule{0.700pt}{0.400pt}}
\multiput(912.00,622.17)(1.547,2.000){2}{\rule{0.350pt}{0.400pt}}
\put(915.17,625){\rule{0.400pt}{0.700pt}}
\multiput(914.17,625.00)(2.000,1.547){2}{\rule{0.400pt}{0.350pt}}
\multiput(917.00,628.61)(0.462,0.447){3}{\rule{0.500pt}{0.108pt}}
\multiput(917.00,627.17)(1.962,3.000){2}{\rule{0.250pt}{0.400pt}}
\put(920,631.17){\rule{0.482pt}{0.400pt}}
\multiput(920.00,630.17)(1.000,2.000){2}{\rule{0.241pt}{0.400pt}}
\multiput(922.00,633.61)(0.462,0.447){3}{\rule{0.500pt}{0.108pt}}
\multiput(922.00,632.17)(1.962,3.000){2}{\rule{0.250pt}{0.400pt}}
\multiput(925.00,636.61)(0.462,0.447){3}{\rule{0.500pt}{0.108pt}}
\multiput(925.00,635.17)(1.962,3.000){2}{\rule{0.250pt}{0.400pt}}
\put(928,639.17){\rule{0.482pt}{0.400pt}}
\multiput(928.00,638.17)(1.000,2.000){2}{\rule{0.241pt}{0.400pt}}
\multiput(930.00,641.61)(0.462,0.447){3}{\rule{0.500pt}{0.108pt}}
\multiput(930.00,640.17)(1.962,3.000){2}{\rule{0.250pt}{0.400pt}}
\put(933.17,644){\rule{0.400pt}{0.700pt}}
\multiput(932.17,644.00)(2.000,1.547){2}{\rule{0.400pt}{0.350pt}}
\put(935,647.17){\rule{0.700pt}{0.400pt}}
\multiput(935.00,646.17)(1.547,2.000){2}{\rule{0.350pt}{0.400pt}}
\multiput(938.00,649.61)(0.462,0.447){3}{\rule{0.500pt}{0.108pt}}
\multiput(938.00,648.17)(1.962,3.000){2}{\rule{0.250pt}{0.400pt}}
\put(941.17,652){\rule{0.400pt}{0.700pt}}
\multiput(940.17,652.00)(2.000,1.547){2}{\rule{0.400pt}{0.350pt}}
\put(943,655.17){\rule{0.700pt}{0.400pt}}
\multiput(943.00,654.17)(1.547,2.000){2}{\rule{0.350pt}{0.400pt}}
\multiput(946.00,657.61)(0.462,0.447){3}{\rule{0.500pt}{0.108pt}}
\multiput(946.00,656.17)(1.962,3.000){2}{\rule{0.250pt}{0.400pt}}
\put(949.17,660){\rule{0.400pt}{0.700pt}}
\multiput(948.17,660.00)(2.000,1.547){2}{\rule{0.400pt}{0.350pt}}
\put(951,663.17){\rule{0.700pt}{0.400pt}}
\multiput(951.00,662.17)(1.547,2.000){2}{\rule{0.350pt}{0.400pt}}
\put(954.17,665){\rule{0.400pt}{0.700pt}}
\multiput(953.17,665.00)(2.000,1.547){2}{\rule{0.400pt}{0.350pt}}
\multiput(956.00,668.61)(0.462,0.447){3}{\rule{0.500pt}{0.108pt}}
\multiput(956.00,667.17)(1.962,3.000){2}{\rule{0.250pt}{0.400pt}}
\put(959,671.17){\rule{0.700pt}{0.400pt}}
\multiput(959.00,670.17)(1.547,2.000){2}{\rule{0.350pt}{0.400pt}}
\put(962.17,673){\rule{0.400pt}{0.700pt}}
\multiput(961.17,673.00)(2.000,1.547){2}{\rule{0.400pt}{0.350pt}}
\multiput(964.00,676.61)(0.462,0.447){3}{\rule{0.500pt}{0.108pt}}
\multiput(964.00,675.17)(1.962,3.000){2}{\rule{0.250pt}{0.400pt}}
\put(967,679.17){\rule{0.482pt}{0.400pt}}
\multiput(967.00,678.17)(1.000,2.000){2}{\rule{0.241pt}{0.400pt}}
\multiput(969.00,681.61)(0.462,0.447){3}{\rule{0.500pt}{0.108pt}}
\multiput(969.00,680.17)(1.962,3.000){2}{\rule{0.250pt}{0.400pt}}
\put(972,684.17){\rule{0.700pt}{0.400pt}}
\multiput(972.00,683.17)(1.547,2.000){2}{\rule{0.350pt}{0.400pt}}
\put(975.17,686){\rule{0.400pt}{0.700pt}}
\multiput(974.17,686.00)(2.000,1.547){2}{\rule{0.400pt}{0.350pt}}
\multiput(977.00,689.61)(0.462,0.447){3}{\rule{0.500pt}{0.108pt}}
\multiput(977.00,688.17)(1.962,3.000){2}{\rule{0.250pt}{0.400pt}}
\put(980,692.17){\rule{0.700pt}{0.400pt}}
\multiput(980.00,691.17)(1.547,2.000){2}{\rule{0.350pt}{0.400pt}}
\put(983.17,694){\rule{0.400pt}{0.700pt}}
\multiput(982.17,694.00)(2.000,1.547){2}{\rule{0.400pt}{0.350pt}}
\multiput(985.00,697.61)(0.462,0.447){3}{\rule{0.500pt}{0.108pt}}
\multiput(985.00,696.17)(1.962,3.000){2}{\rule{0.250pt}{0.400pt}}
\put(988,700.17){\rule{0.482pt}{0.400pt}}
\multiput(988.00,699.17)(1.000,2.000){2}{\rule{0.241pt}{0.400pt}}
\multiput(990.00,702.61)(0.462,0.447){3}{\rule{0.500pt}{0.108pt}}
\multiput(990.00,701.17)(1.962,3.000){2}{\rule{0.250pt}{0.400pt}}
\multiput(993.00,705.61)(0.462,0.447){3}{\rule{0.500pt}{0.108pt}}
\multiput(993.00,704.17)(1.962,3.000){2}{\rule{0.250pt}{0.400pt}}
\put(996,708.17){\rule{0.482pt}{0.400pt}}
\multiput(996.00,707.17)(1.000,2.000){2}{\rule{0.241pt}{0.400pt}}
\multiput(998.00,710.61)(0.462,0.447){3}{\rule{0.500pt}{0.108pt}}
\multiput(998.00,709.17)(1.962,3.000){2}{\rule{0.250pt}{0.400pt}}
\put(1001.17,713){\rule{0.400pt}{0.700pt}}
\multiput(1000.17,713.00)(2.000,1.547){2}{\rule{0.400pt}{0.350pt}}
\put(1003,716.17){\rule{0.700pt}{0.400pt}}
\multiput(1003.00,715.17)(1.547,2.000){2}{\rule{0.350pt}{0.400pt}}
\multiput(1006.00,718.61)(0.462,0.447){3}{\rule{0.500pt}{0.108pt}}
\multiput(1006.00,717.17)(1.962,3.000){2}{\rule{0.250pt}{0.400pt}}
\put(1009.17,721){\rule{0.400pt}{0.700pt}}
\multiput(1008.17,721.00)(2.000,1.547){2}{\rule{0.400pt}{0.350pt}}
\put(1011,724.17){\rule{0.700pt}{0.400pt}}
\multiput(1011.00,723.17)(1.547,2.000){2}{\rule{0.350pt}{0.400pt}}
\put(1014.17,726){\rule{0.400pt}{0.700pt}}
\multiput(1013.17,726.00)(2.000,1.547){2}{\rule{0.400pt}{0.350pt}}
\multiput(1016.00,729.61)(0.462,0.447){3}{\rule{0.500pt}{0.108pt}}
\multiput(1016.00,728.17)(1.962,3.000){2}{\rule{0.250pt}{0.400pt}}
\put(1019,732.17){\rule{0.700pt}{0.400pt}}
\multiput(1019.00,731.17)(1.547,2.000){2}{\rule{0.350pt}{0.400pt}}
\put(1022.17,734){\rule{0.400pt}{0.700pt}}
\multiput(1021.17,734.00)(2.000,1.547){2}{\rule{0.400pt}{0.350pt}}
\multiput(1024.00,737.61)(0.462,0.447){3}{\rule{0.500pt}{0.108pt}}
\multiput(1024.00,736.17)(1.962,3.000){2}{\rule{0.250pt}{0.400pt}}
\put(1027,740.17){\rule{0.700pt}{0.400pt}}
\multiput(1027.00,739.17)(1.547,2.000){2}{\rule{0.350pt}{0.400pt}}
\put(1030.17,742){\rule{0.400pt}{0.700pt}}
\multiput(1029.17,742.00)(2.000,1.547){2}{\rule{0.400pt}{0.350pt}}
\put(1032,745.17){\rule{0.700pt}{0.400pt}}
\multiput(1032.00,744.17)(1.547,2.000){2}{\rule{0.350pt}{0.400pt}}
\put(1035.17,747){\rule{0.400pt}{0.700pt}}
\multiput(1034.17,747.00)(2.000,1.547){2}{\rule{0.400pt}{0.350pt}}
\multiput(1037.00,750.61)(0.462,0.447){3}{\rule{0.500pt}{0.108pt}}
\multiput(1037.00,749.17)(1.962,3.000){2}{\rule{0.250pt}{0.400pt}}
\put(1040,753.17){\rule{0.700pt}{0.400pt}}
\multiput(1040.00,752.17)(1.547,2.000){2}{\rule{0.350pt}{0.400pt}}
\put(1043.17,755){\rule{0.400pt}{0.700pt}}
\multiput(1042.17,755.00)(2.000,1.547){2}{\rule{0.400pt}{0.350pt}}
\multiput(1045.00,758.61)(0.462,0.447){3}{\rule{0.500pt}{0.108pt}}
\multiput(1045.00,757.17)(1.962,3.000){2}{\rule{0.250pt}{0.400pt}}
\put(1048,761.17){\rule{0.482pt}{0.400pt}}
\multiput(1048.00,760.17)(1.000,2.000){2}{\rule{0.241pt}{0.400pt}}
\multiput(1050.00,763.61)(0.462,0.447){3}{\rule{0.500pt}{0.108pt}}
\multiput(1050.00,762.17)(1.962,3.000){2}{\rule{0.250pt}{0.400pt}}
\multiput(1053.00,766.61)(0.462,0.447){3}{\rule{0.500pt}{0.108pt}}
\multiput(1053.00,765.17)(1.962,3.000){2}{\rule{0.250pt}{0.400pt}}
\put(1056,769.17){\rule{0.482pt}{0.400pt}}
\multiput(1056.00,768.17)(1.000,2.000){2}{\rule{0.241pt}{0.400pt}}
\multiput(1058.00,771.61)(0.462,0.447){3}{\rule{0.500pt}{0.108pt}}
\multiput(1058.00,770.17)(1.962,3.000){2}{\rule{0.250pt}{0.400pt}}
\multiput(1061.00,774.61)(0.462,0.447){3}{\rule{0.500pt}{0.108pt}}
\multiput(1061.00,773.17)(1.962,3.000){2}{\rule{0.250pt}{0.400pt}}
\put(1064,777.17){\rule{0.482pt}{0.400pt}}
\multiput(1064.00,776.17)(1.000,2.000){2}{\rule{0.241pt}{0.400pt}}
\multiput(1066.00,779.61)(0.462,0.447){3}{\rule{0.500pt}{0.108pt}}
\multiput(1066.00,778.17)(1.962,3.000){2}{\rule{0.250pt}{0.400pt}}
\put(1069.17,782){\rule{0.400pt}{0.700pt}}
\multiput(1068.17,782.00)(2.000,1.547){2}{\rule{0.400pt}{0.350pt}}
\put(1071,785.17){\rule{0.700pt}{0.400pt}}
\multiput(1071.00,784.17)(1.547,2.000){2}{\rule{0.350pt}{0.400pt}}
\multiput(1074.00,787.61)(0.462,0.447){3}{\rule{0.500pt}{0.108pt}}
\multiput(1074.00,786.17)(1.962,3.000){2}{\rule{0.250pt}{0.400pt}}
\put(1077.17,790){\rule{0.400pt}{0.700pt}}
\multiput(1076.17,790.00)(2.000,1.547){2}{\rule{0.400pt}{0.350pt}}
\put(1079,793.17){\rule{0.700pt}{0.400pt}}
\multiput(1079.00,792.17)(1.547,2.000){2}{\rule{0.350pt}{0.400pt}}
\put(1082.17,795){\rule{0.400pt}{0.700pt}}
\multiput(1081.17,795.00)(2.000,1.547){2}{\rule{0.400pt}{0.350pt}}
\multiput(1084.00,798.61)(0.462,0.447){3}{\rule{0.500pt}{0.108pt}}
\multiput(1084.00,797.17)(1.962,3.000){2}{\rule{0.250pt}{0.400pt}}
\put(1087,801.17){\rule{0.700pt}{0.400pt}}
\multiput(1087.00,800.17)(1.547,2.000){2}{\rule{0.350pt}{0.400pt}}
\put(1090.17,803){\rule{0.400pt}{0.700pt}}
\multiput(1089.17,803.00)(2.000,1.547){2}{\rule{0.400pt}{0.350pt}}
\put(1092,806.17){\rule{0.700pt}{0.400pt}}
\multiput(1092.00,805.17)(1.547,2.000){2}{\rule{0.350pt}{0.400pt}}
\multiput(1095.00,808.61)(0.462,0.447){3}{\rule{0.500pt}{0.108pt}}
\multiput(1095.00,807.17)(1.962,3.000){2}{\rule{0.250pt}{0.400pt}}
\put(1098.17,811){\rule{0.400pt}{0.700pt}}
\multiput(1097.17,811.00)(2.000,1.547){2}{\rule{0.400pt}{0.350pt}}
\put(1100,814.17){\rule{0.700pt}{0.400pt}}
\multiput(1100.00,813.17)(1.547,2.000){2}{\rule{0.350pt}{0.400pt}}
\put(1103.17,816){\rule{0.400pt}{0.700pt}}
\multiput(1102.17,816.00)(2.000,1.547){2}{\rule{0.400pt}{0.350pt}}
\multiput(1105.00,819.61)(0.462,0.447){3}{\rule{0.500pt}{0.108pt}}
\multiput(1105.00,818.17)(1.962,3.000){2}{\rule{0.250pt}{0.400pt}}
\put(1108,822.17){\rule{0.700pt}{0.400pt}}
\multiput(1108.00,821.17)(1.547,2.000){2}{\rule{0.350pt}{0.400pt}}
\put(1111.17,824){\rule{0.400pt}{0.700pt}}
\multiput(1110.17,824.00)(2.000,1.547){2}{\rule{0.400pt}{0.350pt}}
\multiput(1113.00,827.61)(0.462,0.447){3}{\rule{0.500pt}{0.108pt}}
\multiput(1113.00,826.17)(1.962,3.000){2}{\rule{0.250pt}{0.400pt}}
\put(1116,830.17){\rule{0.482pt}{0.400pt}}
\multiput(1116.00,829.17)(1.000,2.000){2}{\rule{0.241pt}{0.400pt}}
\multiput(1118.00,832.61)(0.462,0.447){3}{\rule{0.500pt}{0.108pt}}
\multiput(1118.00,831.17)(1.962,3.000){2}{\rule{0.250pt}{0.400pt}}
\multiput(1121.00,835.61)(0.462,0.447){3}{\rule{0.500pt}{0.108pt}}
\multiput(1121.00,834.17)(1.962,3.000){2}{\rule{0.250pt}{0.400pt}}
\put(1124.17,835){\rule{0.400pt}{0.700pt}}
\multiput(1123.17,836.55)(2.000,-1.547){2}{\rule{0.400pt}{0.350pt}}
\multiput(1126.00,833.95)(0.462,-0.447){3}{\rule{0.500pt}{0.108pt}}
\multiput(1126.00,834.17)(1.962,-3.000){2}{\rule{0.250pt}{0.400pt}}
\put(1129,830.17){\rule{0.482pt}{0.400pt}}
\multiput(1129.00,831.17)(1.000,-2.000){2}{\rule{0.241pt}{0.400pt}}
\multiput(1131.00,828.95)(0.462,-0.447){3}{\rule{0.500pt}{0.108pt}}
\multiput(1131.00,829.17)(1.962,-3.000){2}{\rule{0.250pt}{0.400pt}}
\multiput(1134.00,825.95)(0.462,-0.447){3}{\rule{0.500pt}{0.108pt}}
\multiput(1134.00,826.17)(1.962,-3.000){2}{\rule{0.250pt}{0.400pt}}
\put(1137,822.17){\rule{0.482pt}{0.400pt}}
\multiput(1137.00,823.17)(1.000,-2.000){2}{\rule{0.241pt}{0.400pt}}
\multiput(1139.00,820.95)(0.462,-0.447){3}{\rule{0.500pt}{0.108pt}}
\multiput(1139.00,821.17)(1.962,-3.000){2}{\rule{0.250pt}{0.400pt}}
\multiput(1142.00,817.95)(0.462,-0.447){3}{\rule{0.500pt}{0.108pt}}
\multiput(1142.00,818.17)(1.962,-3.000){2}{\rule{0.250pt}{0.400pt}}
\put(1145,814.17){\rule{0.482pt}{0.400pt}}
\multiput(1145.00,815.17)(1.000,-2.000){2}{\rule{0.241pt}{0.400pt}}
\put(1154,812.67){\rule{1.445pt}{0.400pt}}
\multiput(1154.00,813.17)(3.000,-1.000){2}{\rule{0.723pt}{0.400pt}}
\put(1147.0,814.0){\rule[-0.200pt]{1.686pt}{0.400pt}}
\put(1193,811.67){\rule{1.686pt}{0.400pt}}
\multiput(1193.00,812.17)(3.500,-1.000){2}{\rule{0.843pt}{0.400pt}}
\put(1160.0,813.0){\rule[-0.200pt]{7.950pt}{0.400pt}}
\put(1232,810.67){\rule{1.686pt}{0.400pt}}
\multiput(1232.00,811.17)(3.500,-1.000){2}{\rule{0.843pt}{0.400pt}}
\put(1200.0,812.0){\rule[-0.200pt]{7.709pt}{0.400pt}}
\put(1265,809.67){\rule{1.686pt}{0.400pt}}
\multiput(1265.00,810.17)(3.500,-1.000){2}{\rule{0.843pt}{0.400pt}}
\put(1239.0,811.0){\rule[-0.200pt]{6.263pt}{0.400pt}}
\put(1305,808.67){\rule{1.445pt}{0.400pt}}
\multiput(1305.00,809.17)(3.000,-1.000){2}{\rule{0.723pt}{0.400pt}}
\put(1272.0,810.0){\rule[-0.200pt]{7.950pt}{0.400pt}}
\put(1344,807.67){\rule{1.686pt}{0.400pt}}
\multiput(1344.00,808.17)(3.500,-1.000){2}{\rule{0.843pt}{0.400pt}}
\put(1311.0,809.0){\rule[-0.200pt]{7.950pt}{0.400pt}}
\put(1383,806.67){\rule{1.686pt}{0.400pt}}
\multiput(1383.00,807.17)(3.500,-1.000){2}{\rule{0.843pt}{0.400pt}}
\put(1351.0,808.0){\rule[-0.200pt]{7.709pt}{0.400pt}}
\put(1390.0,807.0){\rule[-0.200pt]{6.263pt}{0.400pt}}
\put(1403,805.67){\rule{1.686pt}{0.400pt}}
\multiput(1406.50,806.17)(-3.500,-1.000){2}{\rule{0.843pt}{0.400pt}}
\put(1410.0,807.0){\rule[-0.200pt]{1.445pt}{0.400pt}}
\put(1370,804.67){\rule{1.686pt}{0.400pt}}
\multiput(1373.50,805.17)(-3.500,-1.000){2}{\rule{0.843pt}{0.400pt}}
\put(1377.0,806.0){\rule[-0.200pt]{6.263pt}{0.400pt}}
\put(1331,803.67){\rule{1.445pt}{0.400pt}}
\multiput(1334.00,804.17)(-3.000,-1.000){2}{\rule{0.723pt}{0.400pt}}
\put(1337.0,805.0){\rule[-0.200pt]{7.950pt}{0.400pt}}
\put(1292,802.67){\rule{1.445pt}{0.400pt}}
\multiput(1295.00,803.17)(-3.000,-1.000){2}{\rule{0.723pt}{0.400pt}}
\put(1298.0,804.0){\rule[-0.200pt]{7.950pt}{0.400pt}}
\put(1252,801.67){\rule{1.686pt}{0.400pt}}
\multiput(1255.50,802.17)(-3.500,-1.000){2}{\rule{0.843pt}{0.400pt}}
\put(1259.0,803.0){\rule[-0.200pt]{7.950pt}{0.400pt}}
\put(1219,800.67){\rule{1.686pt}{0.400pt}}
\multiput(1222.50,801.17)(-3.500,-1.000){2}{\rule{0.843pt}{0.400pt}}
\put(1226.0,802.0){\rule[-0.200pt]{6.263pt}{0.400pt}}
\put(1180,799.67){\rule{1.686pt}{0.400pt}}
\multiput(1183.50,800.17)(-3.500,-1.000){2}{\rule{0.843pt}{0.400pt}}
\put(1187.0,801.0){\rule[-0.200pt]{7.709pt}{0.400pt}}
\put(1141,798.67){\rule{1.445pt}{0.400pt}}
\multiput(1144.00,799.17)(-3.000,-1.000){2}{\rule{0.723pt}{0.400pt}}
\put(1147.0,800.0){\rule[-0.200pt]{7.950pt}{0.400pt}}
\put(1101,797.67){\rule{1.686pt}{0.400pt}}
\multiput(1104.50,798.17)(-3.500,-1.000){2}{\rule{0.843pt}{0.400pt}}
\put(1108.0,799.0){\rule[-0.200pt]{7.950pt}{0.400pt}}
\put(1062,796.67){\rule{1.445pt}{0.400pt}}
\multiput(1065.00,797.17)(-3.000,-1.000){2}{\rule{0.723pt}{0.400pt}}
\put(1068.0,798.0){\rule[-0.200pt]{7.950pt}{0.400pt}}
\put(1029,795.67){\rule{1.686pt}{0.400pt}}
\multiput(1032.50,796.17)(-3.500,-1.000){2}{\rule{0.843pt}{0.400pt}}
\put(1036.0,797.0){\rule[-0.200pt]{6.263pt}{0.400pt}}
\put(990,794.67){\rule{1.445pt}{0.400pt}}
\multiput(993.00,795.17)(-3.000,-1.000){2}{\rule{0.723pt}{0.400pt}}
\put(996.0,796.0){\rule[-0.200pt]{7.950pt}{0.400pt}}
\put(950,793.67){\rule{1.686pt}{0.400pt}}
\multiput(953.50,794.17)(-3.500,-1.000){2}{\rule{0.843pt}{0.400pt}}
\put(957.0,795.0){\rule[-0.200pt]{7.950pt}{0.400pt}}
\put(911,792.67){\rule{1.445pt}{0.400pt}}
\multiput(914.00,793.17)(-3.000,-1.000){2}{\rule{0.723pt}{0.400pt}}
\put(917.0,794.0){\rule[-0.200pt]{7.950pt}{0.400pt}}
\put(878,791.67){\rule{1.686pt}{0.400pt}}
\multiput(881.50,792.17)(-3.500,-1.000){2}{\rule{0.843pt}{0.400pt}}
\put(885.0,793.0){\rule[-0.200pt]{6.263pt}{0.400pt}}
\put(839,790.67){\rule{1.445pt}{0.400pt}}
\multiput(842.00,791.17)(-3.000,-1.000){2}{\rule{0.723pt}{0.400pt}}
\put(845.0,792.0){\rule[-0.200pt]{7.950pt}{0.400pt}}
\put(799,789.67){\rule{1.686pt}{0.400pt}}
\multiput(802.50,790.17)(-3.500,-1.000){2}{\rule{0.843pt}{0.400pt}}
\put(806.0,791.0){\rule[-0.200pt]{7.950pt}{0.400pt}}
\put(760,788.67){\rule{1.686pt}{0.400pt}}
\multiput(763.50,789.17)(-3.500,-1.000){2}{\rule{0.843pt}{0.400pt}}
\put(767.0,790.0){\rule[-0.200pt]{7.709pt}{0.400pt}}
\put(721,787.67){\rule{1.445pt}{0.400pt}}
\multiput(724.00,788.17)(-3.000,-1.000){2}{\rule{0.723pt}{0.400pt}}
\put(727.0,789.0){\rule[-0.200pt]{7.950pt}{0.400pt}}
\put(688,786.67){\rule{1.445pt}{0.400pt}}
\multiput(691.00,787.17)(-3.000,-1.000){2}{\rule{0.723pt}{0.400pt}}
\put(694.0,788.0){\rule[-0.200pt]{6.504pt}{0.400pt}}
\put(648,785.67){\rule{1.686pt}{0.400pt}}
\multiput(651.50,786.17)(-3.500,-1.000){2}{\rule{0.843pt}{0.400pt}}
\put(655.0,787.0){\rule[-0.200pt]{7.950pt}{0.400pt}}
\put(609,784.67){\rule{1.686pt}{0.400pt}}
\multiput(612.50,785.17)(-3.500,-1.000){2}{\rule{0.843pt}{0.400pt}}
\put(616.0,786.0){\rule[-0.200pt]{7.709pt}{0.400pt}}
\put(570,783.67){\rule{1.445pt}{0.400pt}}
\multiput(573.00,784.17)(-3.000,-1.000){2}{\rule{0.723pt}{0.400pt}}
\put(576.0,785.0){\rule[-0.200pt]{7.950pt}{0.400pt}}
\put(537,782.67){\rule{1.445pt}{0.400pt}}
\multiput(540.00,783.17)(-3.000,-1.000){2}{\rule{0.723pt}{0.400pt}}
\put(543.0,784.0){\rule[-0.200pt]{6.504pt}{0.400pt}}
\put(497,781.67){\rule{1.686pt}{0.400pt}}
\multiput(500.50,782.17)(-3.500,-1.000){2}{\rule{0.843pt}{0.400pt}}
\put(504.0,783.0){\rule[-0.200pt]{7.950pt}{0.400pt}}
\put(458,780.67){\rule{1.686pt}{0.400pt}}
\multiput(461.50,781.17)(-3.500,-1.000){2}{\rule{0.843pt}{0.400pt}}
\put(465.0,782.0){\rule[-0.200pt]{7.709pt}{0.400pt}}
\put(419,779.67){\rule{1.445pt}{0.400pt}}
\multiput(422.00,780.17)(-3.000,-1.000){2}{\rule{0.723pt}{0.400pt}}
\put(425.0,781.0){\rule[-0.200pt]{7.950pt}{0.400pt}}
\put(379,778.67){\rule{1.686pt}{0.400pt}}
\multiput(382.50,779.17)(-3.500,-1.000){2}{\rule{0.843pt}{0.400pt}}
\put(386.0,780.0){\rule[-0.200pt]{7.950pt}{0.400pt}}
\put(347,777.67){\rule{1.445pt}{0.400pt}}
\multiput(350.00,778.17)(-3.000,-1.000){2}{\rule{0.723pt}{0.400pt}}
\put(353.0,779.0){\rule[-0.200pt]{6.263pt}{0.400pt}}
\put(307,776.67){\rule{1.686pt}{0.400pt}}
\multiput(310.50,777.17)(-3.500,-1.000){2}{\rule{0.843pt}{0.400pt}}
\put(314.0,778.0){\rule[-0.200pt]{7.950pt}{0.400pt}}
\put(268,775.67){\rule{1.445pt}{0.400pt}}
\multiput(271.00,776.17)(-3.000,-1.000){2}{\rule{0.723pt}{0.400pt}}
\put(274.0,777.0){\rule[-0.200pt]{7.950pt}{0.400pt}}
\put(228,774.67){\rule{1.686pt}{0.400pt}}
\multiput(231.50,775.17)(-3.500,-1.000){2}{\rule{0.843pt}{0.400pt}}
\put(235.0,776.0){\rule[-0.200pt]{7.950pt}{0.400pt}}
\put(196,773.67){\rule{1.445pt}{0.400pt}}
\multiput(199.00,774.17)(-3.000,-1.000){2}{\rule{0.723pt}{0.400pt}}
\put(202.0,775.0){\rule[-0.200pt]{6.263pt}{0.400pt}}
\put(156,772.67){\rule{1.686pt}{0.400pt}}
\multiput(159.50,773.17)(-3.500,-1.000){2}{\rule{0.843pt}{0.400pt}}
\put(163.0,774.0){\rule[-0.200pt]{7.950pt}{0.400pt}}
\put(150.0,773.0){\rule[-0.200pt]{1.445pt}{0.400pt}}
\put(176,771.67){\rule{1.445pt}{0.400pt}}
\multiput(176.00,772.17)(3.000,-1.000){2}{\rule{0.723pt}{0.400pt}}
\put(150.0,773.0){\rule[-0.200pt]{6.263pt}{0.400pt}}
\put(215,770.67){\rule{1.686pt}{0.400pt}}
\multiput(215.00,771.17)(3.500,-1.000){2}{\rule{0.843pt}{0.400pt}}
\put(182.0,772.0){\rule[-0.200pt]{7.950pt}{0.400pt}}
\put(255,769.67){\rule{1.445pt}{0.400pt}}
\multiput(255.00,770.17)(3.000,-1.000){2}{\rule{0.723pt}{0.400pt}}
\put(222.0,771.0){\rule[-0.200pt]{7.950pt}{0.400pt}}
\put(287,768.67){\rule{1.686pt}{0.400pt}}
\multiput(287.00,769.17)(3.500,-1.000){2}{\rule{0.843pt}{0.400pt}}
\put(261.0,770.0){\rule[-0.200pt]{6.263pt}{0.400pt}}
\put(327,767.67){\rule{1.445pt}{0.400pt}}
\multiput(327.00,768.17)(3.000,-1.000){2}{\rule{0.723pt}{0.400pt}}
\put(294.0,769.0){\rule[-0.200pt]{7.950pt}{0.400pt}}
\put(366,766.67){\rule{1.686pt}{0.400pt}}
\multiput(366.00,767.17)(3.500,-1.000){2}{\rule{0.843pt}{0.400pt}}
\put(333.0,768.0){\rule[-0.200pt]{7.950pt}{0.400pt}}
\put(406,765.67){\rule{1.445pt}{0.400pt}}
\multiput(406.00,766.17)(3.000,-1.000){2}{\rule{0.723pt}{0.400pt}}
\put(373.0,767.0){\rule[-0.200pt]{7.950pt}{0.400pt}}
\put(438,764.67){\rule{1.686pt}{0.400pt}}
\multiput(438.00,765.17)(3.500,-1.000){2}{\rule{0.843pt}{0.400pt}}
\put(412.0,766.0){\rule[-0.200pt]{6.263pt}{0.400pt}}
\put(445.0,765.0){\rule[-0.200pt]{6.263pt}{0.400pt}}
\put(130.0,82.0){\rule[-0.200pt]{0.400pt}{187.179pt}}
\put(130.0,82.0){\rule[-0.200pt]{315.338pt}{0.400pt}}
\put(1439.0,82.0){\rule[-0.200pt]{0.400pt}{187.179pt}}
\put(130.0,859.0){\rule[-0.200pt]{315.338pt}{0.400pt}}
\end{picture}

\end{document}
