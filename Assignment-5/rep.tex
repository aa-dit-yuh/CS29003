\title{Assignment 5 -CS29003}
\author{
        Aditya Narayan \\
        13EC30001\\
}
\date{\today}

\documentclass[12pt]{article}
\usepackage{pgfplots}
\usepackage[utf8]{inputenc}
\usepackage{times}

\pgfplotsset{width=10cm}

\begin{document}
\maketitle

\begin{abstract}
An analytical approach to Event-driven Simulation using heaps.
\end{abstract}

\section{Introduction}
This experiment intends to simulate the collision of  balls on a 2D planar region bounded by straight walls. The simulation is to be done efficiently using the application of the priority queue data structure. \\

\section{Event Driven Simulation}
The objective of the experiment is to devise the most efficient use of the Heap data structure to simulate the collision of particles using Event driven simulation. \\
Event driven simulation allows us to simulate events in a chronological order.\\

\section{Implementation}\label{Implementation}
The implementation of the priority queue is described as below:\\
\paragraph{Local Minima}
The priority queue, Local Minima (LM), used in this experiment is a compromise between having all events organized in a single priority queue (where invalidating events is not efficient) or having all the events in a set of linked lists, one per object (where finding the next event is not efficient). To achieve this balance, we define the local minimum associated to a ball i , as the
event of smallest time between all events E i ( x ) scheduled for object i .\\
The events E i ( x ) are stored in a simple linked list L i , which is used to obtain the local minimum for the object i . There are N linked lists L i ; one for every ball i . Note, that the events E j ( i ) are stored in the list L j and not in the list L i. To compare the N local minima, we use a heap (CBT) which performs a binary tournament between all local minima. That is, each leaf has a ball number and each internal node (recursively up to the root) has the ball number with smaller local minimum of its two children. So, the root of the tree has the smallest local minima. Every time that a new local minimum is computed for ball i , the tournament
is updated for all nodes in the path from the leaf labelled with i to the root. The tournament uses a complete binary tree because N is fixed. \\
An event E ( i, j ) can be invalidated if in a previous simulation step, Cancel( j ) was called. Not all events associated to an object are invalidated at the same time. Cancel( j ) invalidates
all events E ( j, x ), but events E (k, j ) will stay in the PQ until are invalidated by Cancel( k ) or they are detected and deleted when a local minimum for k is computed. \\
Every time that we extract a smallest event E ( i, j ) that has been invalidated, a new local minimum for i is computed and the HEAP updated, extracting again the next chronological event. We call this operation a Reschedule .\\

\paragraph{Insertion}
Pseudocode for insertion of events: \\ \\
\indent \indent  \textbf{Insert( E ( i, j ) )}\\
\indent \indent \indent \textbf{if} ( j is a moving ball)\\
\indent \indent \indent \indent E ( i, j ).c $\leftarrow$ O[ j ].c;\\
\indent \indent \indent O [ i ]. L $\leftarrow$ \textbf{InsertList} ( i , E ( i, j ) );\\
\indent \indent  \textbf{EndInsert}\\
\\
\paragraph{Next Event}
Pseudocode for extracting the next event from the Priority Queue:\\ \\
\indent \indent  \textbf{NextEvent ()}\\
\indent \indent \indent \textbf{while} (1)\\
\indent \indent \indent \indent i $\leftarrow$ HEAP [ root ]\\
\indent \indent \indent \indent E i ( j ) $\leftarrow$ O [ i ]. L ;\\
\indent \indent \indent \indent \textbf{if} ( j is a moving ball ) and ( E ( i, j ).c != O [ j ].c ) )\\
\indent \indent \indent \indent \indent /* Reschedule */\\
\indent \indent \indent \indent \indent O [ i ]. L $\leftarrow$ \textbf{LocalMin} ( i );\\
\indent \indent \indent \indent \indent \textbf{Heapify} ( i );\\
\indent \indent \indent \indent \textbf{else return} ( E i ( j ) );\\
\indent \indent  \textbf{EndNextEvent}\\
\paragraph{Deletion}
Pseudocode for deletion of events: \\ \\
\indent \indent  \textbf{Delete (i)}\\
\indent \indent \indent O [ i ]. L $\leftarrow$ NULL;\\
\indent \indent \indent O [ i ].c $\leftarrow$ O[ i ].c + 1;\\
\indent \indent  \textbf{EndDelete}\\
\\

\subsection{Complexity Analysis}\label{Complexity Analysis}
\paragraph{Insertion}
Insertion requires adding an event to its specific linked list followed by updating the heap upwards from the linked list towards the root. This operation consumes $\theta(log(n))$ number of operations, where $\mathbf{n}$ is the number of particles in the system.

\paragraph{Next Event}
The operation to extract the next event in the queue is a constant time process if the dequeued event is valid. If the dequeued is invalid, the \textbf{LocalMinima} for the ball element is evaluated and the \textbf{Heap} is then heapified to maintain heap properties. This consumes $\theta(c)$ (constant) operations for the valid case, and $\theta(n * log(n))$ operations for the invalid case. In practise, the valid case occurs far greater times than the invalid case.\\
 
\paragraph{Deletion or Invalidation}
Deletion is a contant time operation in the LM Priority Queue, as it only involves clearing the linked list pertaining to a ball.

\section{Results}\label{results}
Thus, we observe that the Priority Queue being used, the \textbf{LocalMinima} is one of the most efficient Priority Queues for the given application.

\section{Plots}\label{plots}
Plot for Ball 1:\\
% GNUPLOT: LaTeX picture
\setlength{\unitlength}{0.240900pt}
\ifx\plotpoint\undefined\newsavebox{\plotpoint}\fi
\sbox{\plotpoint}{\rule[-0.200pt]{0.400pt}{0.400pt}}%
\begin{picture}(1500,900)(0,0)
\sbox{\plotpoint}{\rule[-0.200pt]{0.400pt}{0.400pt}}%
\put(130.0,90.0){\rule[-0.200pt]{4.818pt}{0.400pt}}
\put(110,90){\makebox(0,0)[r]{ 0}}
\put(1419.0,90.0){\rule[-0.200pt]{4.818pt}{0.400pt}}
\put(130.0,242.0){\rule[-0.200pt]{4.818pt}{0.400pt}}
\put(110,242){\makebox(0,0)[r]{ 0.2}}
\put(1419.0,242.0){\rule[-0.200pt]{4.818pt}{0.400pt}}
\put(130.0,394.0){\rule[-0.200pt]{4.818pt}{0.400pt}}
\put(110,394){\makebox(0,0)[r]{ 0.4}}
\put(1419.0,394.0){\rule[-0.200pt]{4.818pt}{0.400pt}}
\put(130.0,547.0){\rule[-0.200pt]{4.818pt}{0.400pt}}
\put(110,547){\makebox(0,0)[r]{ 0.6}}
\put(1419.0,547.0){\rule[-0.200pt]{4.818pt}{0.400pt}}
\put(130.0,699.0){\rule[-0.200pt]{4.818pt}{0.400pt}}
\put(110,699){\makebox(0,0)[r]{ 0.8}}
\put(1419.0,699.0){\rule[-0.200pt]{4.818pt}{0.400pt}}
\put(130.0,851.0){\rule[-0.200pt]{4.818pt}{0.400pt}}
\put(110,851){\makebox(0,0)[r]{ 1}}
\put(1419.0,851.0){\rule[-0.200pt]{4.818pt}{0.400pt}}
\put(130.0,82.0){\rule[-0.200pt]{0.400pt}{4.818pt}}
\put(130,41){\makebox(0,0){ 0}}
\put(130.0,839.0){\rule[-0.200pt]{0.400pt}{4.818pt}}
\put(392.0,82.0){\rule[-0.200pt]{0.400pt}{4.818pt}}
\put(392,41){\makebox(0,0){ 0.2}}
\put(392.0,839.0){\rule[-0.200pt]{0.400pt}{4.818pt}}
\put(654.0,82.0){\rule[-0.200pt]{0.400pt}{4.818pt}}
\put(654,41){\makebox(0,0){ 0.4}}
\put(654.0,839.0){\rule[-0.200pt]{0.400pt}{4.818pt}}
\put(915.0,82.0){\rule[-0.200pt]{0.400pt}{4.818pt}}
\put(915,41){\makebox(0,0){ 0.6}}
\put(915.0,839.0){\rule[-0.200pt]{0.400pt}{4.818pt}}
\put(1177.0,82.0){\rule[-0.200pt]{0.400pt}{4.818pt}}
\put(1177,41){\makebox(0,0){ 0.8}}
\put(1177.0,839.0){\rule[-0.200pt]{0.400pt}{4.818pt}}
\put(1439.0,82.0){\rule[-0.200pt]{0.400pt}{4.818pt}}
\put(1439,41){\makebox(0,0){ 1}}
\put(1439.0,839.0){\rule[-0.200pt]{0.400pt}{4.818pt}}
\put(130.0,82.0){\rule[-0.200pt]{0.400pt}{187.179pt}}
\put(130.0,82.0){\rule[-0.200pt]{315.338pt}{0.400pt}}
\put(1439.0,82.0){\rule[-0.200pt]{0.400pt}{187.179pt}}
\put(130.0,859.0){\rule[-0.200pt]{315.338pt}{0.400pt}}
\put(1279,819){\makebox(0,0)[r]{'-'}}
\put(1299.0,819.0){\rule[-0.200pt]{24.090pt}{0.400pt}}
\put(856,602){\usebox{\plotpoint}}
\put(856,600.67){\rule{0.723pt}{0.400pt}}
\multiput(856.00,601.17)(1.500,-1.000){2}{\rule{0.361pt}{0.400pt}}
\put(859,599.67){\rule{0.723pt}{0.400pt}}
\multiput(859.00,600.17)(1.500,-1.000){2}{\rule{0.361pt}{0.400pt}}
\put(862,598.17){\rule{0.700pt}{0.400pt}}
\multiput(862.00,599.17)(1.547,-2.000){2}{\rule{0.350pt}{0.400pt}}
\put(865,596.67){\rule{0.723pt}{0.400pt}}
\multiput(865.00,597.17)(1.500,-1.000){2}{\rule{0.361pt}{0.400pt}}
\put(868,595.67){\rule{0.723pt}{0.400pt}}
\multiput(868.00,596.17)(1.500,-1.000){2}{\rule{0.361pt}{0.400pt}}
\put(871,594.67){\rule{0.723pt}{0.400pt}}
\multiput(871.00,595.17)(1.500,-1.000){2}{\rule{0.361pt}{0.400pt}}
\put(874,593.67){\rule{0.723pt}{0.400pt}}
\multiput(874.00,594.17)(1.500,-1.000){2}{\rule{0.361pt}{0.400pt}}
\put(877,592.67){\rule{0.723pt}{0.400pt}}
\multiput(877.00,593.17)(1.500,-1.000){2}{\rule{0.361pt}{0.400pt}}
\put(880,591.17){\rule{0.700pt}{0.400pt}}
\multiput(880.00,592.17)(1.547,-2.000){2}{\rule{0.350pt}{0.400pt}}
\put(883,589.67){\rule{0.964pt}{0.400pt}}
\multiput(883.00,590.17)(2.000,-1.000){2}{\rule{0.482pt}{0.400pt}}
\put(887,588.67){\rule{0.723pt}{0.400pt}}
\multiput(887.00,589.17)(1.500,-1.000){2}{\rule{0.361pt}{0.400pt}}
\put(890,587.67){\rule{0.723pt}{0.400pt}}
\multiput(890.00,588.17)(1.500,-1.000){2}{\rule{0.361pt}{0.400pt}}
\put(893,586.67){\rule{0.723pt}{0.400pt}}
\multiput(893.00,587.17)(1.500,-1.000){2}{\rule{0.361pt}{0.400pt}}
\put(896,585.67){\rule{0.723pt}{0.400pt}}
\multiput(896.00,586.17)(1.500,-1.000){2}{\rule{0.361pt}{0.400pt}}
\put(899,584.17){\rule{0.700pt}{0.400pt}}
\multiput(899.00,585.17)(1.547,-2.000){2}{\rule{0.350pt}{0.400pt}}
\put(902,582.67){\rule{0.723pt}{0.400pt}}
\multiput(902.00,583.17)(1.500,-1.000){2}{\rule{0.361pt}{0.400pt}}
\put(905,581.67){\rule{0.723pt}{0.400pt}}
\multiput(905.00,582.17)(1.500,-1.000){2}{\rule{0.361pt}{0.400pt}}
\put(908,580.67){\rule{0.723pt}{0.400pt}}
\multiput(908.00,581.17)(1.500,-1.000){2}{\rule{0.361pt}{0.400pt}}
\put(911,579.67){\rule{0.723pt}{0.400pt}}
\multiput(911.00,580.17)(1.500,-1.000){2}{\rule{0.361pt}{0.400pt}}
\put(914,578.67){\rule{0.723pt}{0.400pt}}
\multiput(914.00,579.17)(1.500,-1.000){2}{\rule{0.361pt}{0.400pt}}
\put(917,577.17){\rule{0.700pt}{0.400pt}}
\multiput(917.00,578.17)(1.547,-2.000){2}{\rule{0.350pt}{0.400pt}}
\put(920,575.67){\rule{0.723pt}{0.400pt}}
\multiput(920.00,576.17)(1.500,-1.000){2}{\rule{0.361pt}{0.400pt}}
\put(923,574.67){\rule{0.723pt}{0.400pt}}
\multiput(923.00,575.17)(1.500,-1.000){2}{\rule{0.361pt}{0.400pt}}
\put(926,573.67){\rule{0.723pt}{0.400pt}}
\multiput(926.00,574.17)(1.500,-1.000){2}{\rule{0.361pt}{0.400pt}}
\put(929,572.67){\rule{0.723pt}{0.400pt}}
\multiput(929.00,573.17)(1.500,-1.000){2}{\rule{0.361pt}{0.400pt}}
\put(932,571.67){\rule{0.723pt}{0.400pt}}
\multiput(932.00,572.17)(1.500,-1.000){2}{\rule{0.361pt}{0.400pt}}
\put(935,570.17){\rule{0.700pt}{0.400pt}}
\multiput(935.00,571.17)(1.547,-2.000){2}{\rule{0.350pt}{0.400pt}}
\put(938,568.67){\rule{0.723pt}{0.400pt}}
\multiput(938.00,569.17)(1.500,-1.000){2}{\rule{0.361pt}{0.400pt}}
\put(941,567.67){\rule{0.723pt}{0.400pt}}
\multiput(941.00,568.17)(1.500,-1.000){2}{\rule{0.361pt}{0.400pt}}
\put(944,566.67){\rule{0.964pt}{0.400pt}}
\multiput(944.00,567.17)(2.000,-1.000){2}{\rule{0.482pt}{0.400pt}}
\put(948,565.67){\rule{0.723pt}{0.400pt}}
\multiput(948.00,566.17)(1.500,-1.000){2}{\rule{0.361pt}{0.400pt}}
\put(951,564.67){\rule{0.723pt}{0.400pt}}
\multiput(951.00,565.17)(1.500,-1.000){2}{\rule{0.361pt}{0.400pt}}
\put(954,563.17){\rule{0.700pt}{0.400pt}}
\multiput(954.00,564.17)(1.547,-2.000){2}{\rule{0.350pt}{0.400pt}}
\put(957,561.67){\rule{0.723pt}{0.400pt}}
\multiput(957.00,562.17)(1.500,-1.000){2}{\rule{0.361pt}{0.400pt}}
\put(960,560.67){\rule{0.723pt}{0.400pt}}
\multiput(960.00,561.17)(1.500,-1.000){2}{\rule{0.361pt}{0.400pt}}
\put(963,559.67){\rule{0.723pt}{0.400pt}}
\multiput(963.00,560.17)(1.500,-1.000){2}{\rule{0.361pt}{0.400pt}}
\put(966,558.67){\rule{0.723pt}{0.400pt}}
\multiput(966.00,559.17)(1.500,-1.000){2}{\rule{0.361pt}{0.400pt}}
\put(969,557.67){\rule{0.723pt}{0.400pt}}
\multiput(969.00,558.17)(1.500,-1.000){2}{\rule{0.361pt}{0.400pt}}
\put(972,556.17){\rule{0.700pt}{0.400pt}}
\multiput(972.00,557.17)(1.547,-2.000){2}{\rule{0.350pt}{0.400pt}}
\put(975,554.67){\rule{0.723pt}{0.400pt}}
\multiput(975.00,555.17)(1.500,-1.000){2}{\rule{0.361pt}{0.400pt}}
\put(978,553.67){\rule{0.723pt}{0.400pt}}
\multiput(978.00,554.17)(1.500,-1.000){2}{\rule{0.361pt}{0.400pt}}
\put(981,552.67){\rule{0.723pt}{0.400pt}}
\multiput(981.00,553.17)(1.500,-1.000){2}{\rule{0.361pt}{0.400pt}}
\put(984,551.67){\rule{0.723pt}{0.400pt}}
\multiput(984.00,552.17)(1.500,-1.000){2}{\rule{0.361pt}{0.400pt}}
\put(987,550.67){\rule{0.723pt}{0.400pt}}
\multiput(987.00,551.17)(1.500,-1.000){2}{\rule{0.361pt}{0.400pt}}
\put(990,549.17){\rule{0.700pt}{0.400pt}}
\multiput(990.00,550.17)(1.547,-2.000){2}{\rule{0.350pt}{0.400pt}}
\put(993,547.67){\rule{0.723pt}{0.400pt}}
\multiput(993.00,548.17)(1.500,-1.000){2}{\rule{0.361pt}{0.400pt}}
\put(996,546.67){\rule{0.723pt}{0.400pt}}
\multiput(996.00,547.17)(1.500,-1.000){2}{\rule{0.361pt}{0.400pt}}
\put(999,545.67){\rule{0.723pt}{0.400pt}}
\multiput(999.00,546.17)(1.500,-1.000){2}{\rule{0.361pt}{0.400pt}}
\put(1002,544.67){\rule{0.964pt}{0.400pt}}
\multiput(1002.00,545.17)(2.000,-1.000){2}{\rule{0.482pt}{0.400pt}}
\put(1006,543.67){\rule{0.723pt}{0.400pt}}
\multiput(1006.00,544.17)(1.500,-1.000){2}{\rule{0.361pt}{0.400pt}}
\put(1009,542.17){\rule{0.700pt}{0.400pt}}
\multiput(1009.00,543.17)(1.547,-2.000){2}{\rule{0.350pt}{0.400pt}}
\put(1012,540.67){\rule{0.723pt}{0.400pt}}
\multiput(1012.00,541.17)(1.500,-1.000){2}{\rule{0.361pt}{0.400pt}}
\put(1015,539.67){\rule{0.723pt}{0.400pt}}
\multiput(1015.00,540.17)(1.500,-1.000){2}{\rule{0.361pt}{0.400pt}}
\put(1018,538.67){\rule{0.723pt}{0.400pt}}
\multiput(1018.00,539.17)(1.500,-1.000){2}{\rule{0.361pt}{0.400pt}}
\put(1021,537.67){\rule{0.723pt}{0.400pt}}
\multiput(1021.00,538.17)(1.500,-1.000){2}{\rule{0.361pt}{0.400pt}}
\put(1024,536.67){\rule{0.723pt}{0.400pt}}
\multiput(1024.00,537.17)(1.500,-1.000){2}{\rule{0.361pt}{0.400pt}}
\put(1027,535.17){\rule{0.700pt}{0.400pt}}
\multiput(1027.00,536.17)(1.547,-2.000){2}{\rule{0.350pt}{0.400pt}}
\put(1030,533.67){\rule{0.723pt}{0.400pt}}
\multiput(1030.00,534.17)(1.500,-1.000){2}{\rule{0.361pt}{0.400pt}}
\put(1033,532.67){\rule{0.723pt}{0.400pt}}
\multiput(1033.00,533.17)(1.500,-1.000){2}{\rule{0.361pt}{0.400pt}}
\put(1036,531.67){\rule{0.723pt}{0.400pt}}
\multiput(1036.00,532.17)(1.500,-1.000){2}{\rule{0.361pt}{0.400pt}}
\put(1039,530.67){\rule{0.723pt}{0.400pt}}
\multiput(1039.00,531.17)(1.500,-1.000){2}{\rule{0.361pt}{0.400pt}}
\put(1042,529.67){\rule{0.723pt}{0.400pt}}
\multiput(1042.00,530.17)(1.500,-1.000){2}{\rule{0.361pt}{0.400pt}}
\put(1045,528.17){\rule{0.700pt}{0.400pt}}
\multiput(1045.00,529.17)(1.547,-2.000){2}{\rule{0.350pt}{0.400pt}}
\put(1048,526.67){\rule{0.723pt}{0.400pt}}
\multiput(1048.00,527.17)(1.500,-1.000){2}{\rule{0.361pt}{0.400pt}}
\put(1051,525.67){\rule{0.723pt}{0.400pt}}
\multiput(1051.00,526.17)(1.500,-1.000){2}{\rule{0.361pt}{0.400pt}}
\put(1054,524.67){\rule{0.723pt}{0.400pt}}
\multiput(1054.00,525.17)(1.500,-1.000){2}{\rule{0.361pt}{0.400pt}}
\put(1057,523.67){\rule{0.723pt}{0.400pt}}
\multiput(1057.00,524.17)(1.500,-1.000){2}{\rule{0.361pt}{0.400pt}}
\put(1060,522.67){\rule{0.964pt}{0.400pt}}
\multiput(1060.00,523.17)(2.000,-1.000){2}{\rule{0.482pt}{0.400pt}}
\put(1064,521.17){\rule{0.700pt}{0.400pt}}
\multiput(1064.00,522.17)(1.547,-2.000){2}{\rule{0.350pt}{0.400pt}}
\put(1067,519.67){\rule{0.723pt}{0.400pt}}
\multiput(1067.00,520.17)(1.500,-1.000){2}{\rule{0.361pt}{0.400pt}}
\put(1070,518.67){\rule{0.723pt}{0.400pt}}
\multiput(1070.00,519.17)(1.500,-1.000){2}{\rule{0.361pt}{0.400pt}}
\put(1073,517.67){\rule{0.723pt}{0.400pt}}
\multiput(1073.00,518.17)(1.500,-1.000){2}{\rule{0.361pt}{0.400pt}}
\put(1076,516.67){\rule{0.723pt}{0.400pt}}
\multiput(1076.00,517.17)(1.500,-1.000){2}{\rule{0.361pt}{0.400pt}}
\put(1079,515.67){\rule{0.723pt}{0.400pt}}
\multiput(1079.00,516.17)(1.500,-1.000){2}{\rule{0.361pt}{0.400pt}}
\put(1082,514.17){\rule{0.700pt}{0.400pt}}
\multiput(1082.00,515.17)(1.547,-2.000){2}{\rule{0.350pt}{0.400pt}}
\put(1085,512.67){\rule{0.723pt}{0.400pt}}
\multiput(1085.00,513.17)(1.500,-1.000){2}{\rule{0.361pt}{0.400pt}}
\put(1088,511.67){\rule{0.723pt}{0.400pt}}
\multiput(1088.00,512.17)(1.500,-1.000){2}{\rule{0.361pt}{0.400pt}}
\put(1091,510.67){\rule{0.723pt}{0.400pt}}
\multiput(1091.00,511.17)(1.500,-1.000){2}{\rule{0.361pt}{0.400pt}}
\put(1094,509.67){\rule{0.723pt}{0.400pt}}
\multiput(1094.00,510.17)(1.500,-1.000){2}{\rule{0.361pt}{0.400pt}}
\put(1097,508.67){\rule{0.723pt}{0.400pt}}
\multiput(1097.00,509.17)(1.500,-1.000){2}{\rule{0.361pt}{0.400pt}}
\put(1100,507.67){\rule{0.723pt}{0.400pt}}
\multiput(1100.00,508.17)(1.500,-1.000){2}{\rule{0.361pt}{0.400pt}}
\put(1103,506.17){\rule{0.700pt}{0.400pt}}
\multiput(1103.00,507.17)(1.547,-2.000){2}{\rule{0.350pt}{0.400pt}}
\put(1106,504.67){\rule{0.723pt}{0.400pt}}
\multiput(1106.00,505.17)(1.500,-1.000){2}{\rule{0.361pt}{0.400pt}}
\put(1109,503.67){\rule{0.723pt}{0.400pt}}
\multiput(1109.00,504.17)(1.500,-1.000){2}{\rule{0.361pt}{0.400pt}}
\put(1112,502.67){\rule{0.723pt}{0.400pt}}
\multiput(1112.00,503.17)(1.500,-1.000){2}{\rule{0.361pt}{0.400pt}}
\put(1115,501.67){\rule{0.723pt}{0.400pt}}
\multiput(1115.00,502.17)(1.500,-1.000){2}{\rule{0.361pt}{0.400pt}}
\put(1118,500.67){\rule{0.964pt}{0.400pt}}
\multiput(1118.00,501.17)(2.000,-1.000){2}{\rule{0.482pt}{0.400pt}}
\put(1122,499.17){\rule{0.700pt}{0.400pt}}
\multiput(1122.00,500.17)(1.547,-2.000){2}{\rule{0.350pt}{0.400pt}}
\put(1125,497.67){\rule{0.723pt}{0.400pt}}
\multiput(1125.00,498.17)(1.500,-1.000){2}{\rule{0.361pt}{0.400pt}}
\put(1128,496.67){\rule{0.723pt}{0.400pt}}
\multiput(1128.00,497.17)(1.500,-1.000){2}{\rule{0.361pt}{0.400pt}}
\put(1131,495.67){\rule{0.723pt}{0.400pt}}
\multiput(1131.00,496.17)(1.500,-1.000){2}{\rule{0.361pt}{0.400pt}}
\put(1134,494.67){\rule{0.723pt}{0.400pt}}
\multiput(1134.00,495.17)(1.500,-1.000){2}{\rule{0.361pt}{0.400pt}}
\put(1137,493.67){\rule{0.723pt}{0.400pt}}
\multiput(1137.00,494.17)(1.500,-1.000){2}{\rule{0.361pt}{0.400pt}}
\put(1140,492.17){\rule{0.700pt}{0.400pt}}
\multiput(1140.00,493.17)(1.547,-2.000){2}{\rule{0.350pt}{0.400pt}}
\put(1143,490.67){\rule{0.723pt}{0.400pt}}
\multiput(1143.00,491.17)(1.500,-1.000){2}{\rule{0.361pt}{0.400pt}}
\put(1146,489.67){\rule{0.723pt}{0.400pt}}
\multiput(1146.00,490.17)(1.500,-1.000){2}{\rule{0.361pt}{0.400pt}}
\put(1149,488.67){\rule{0.723pt}{0.400pt}}
\multiput(1149.00,489.17)(1.500,-1.000){2}{\rule{0.361pt}{0.400pt}}
\put(1152,487.67){\rule{0.723pt}{0.400pt}}
\multiput(1152.00,488.17)(1.500,-1.000){2}{\rule{0.361pt}{0.400pt}}
\put(1155,486.67){\rule{0.723pt}{0.400pt}}
\multiput(1155.00,487.17)(1.500,-1.000){2}{\rule{0.361pt}{0.400pt}}
\put(1158,485.17){\rule{0.700pt}{0.400pt}}
\multiput(1158.00,486.17)(1.547,-2.000){2}{\rule{0.350pt}{0.400pt}}
\put(1161,483.67){\rule{0.723pt}{0.400pt}}
\multiput(1161.00,484.17)(1.500,-1.000){2}{\rule{0.361pt}{0.400pt}}
\put(1164,482.67){\rule{0.723pt}{0.400pt}}
\multiput(1164.00,483.17)(1.500,-1.000){2}{\rule{0.361pt}{0.400pt}}
\put(1167,481.67){\rule{0.723pt}{0.400pt}}
\multiput(1167.00,482.17)(1.500,-1.000){2}{\rule{0.361pt}{0.400pt}}
\put(1170,480.67){\rule{0.723pt}{0.400pt}}
\multiput(1170.00,481.17)(1.500,-1.000){2}{\rule{0.361pt}{0.400pt}}
\put(1173,479.67){\rule{0.723pt}{0.400pt}}
\multiput(1173.00,480.17)(1.500,-1.000){2}{\rule{0.361pt}{0.400pt}}
\put(1176,478.17){\rule{0.900pt}{0.400pt}}
\multiput(1176.00,479.17)(2.132,-2.000){2}{\rule{0.450pt}{0.400pt}}
\put(1180,476.67){\rule{0.723pt}{0.400pt}}
\multiput(1180.00,477.17)(1.500,-1.000){2}{\rule{0.361pt}{0.400pt}}
\put(1183,475.67){\rule{0.723pt}{0.400pt}}
\multiput(1183.00,476.17)(1.500,-1.000){2}{\rule{0.361pt}{0.400pt}}
\put(1186,474.67){\rule{0.723pt}{0.400pt}}
\multiput(1186.00,475.17)(1.500,-1.000){2}{\rule{0.361pt}{0.400pt}}
\put(1189,473.67){\rule{0.723pt}{0.400pt}}
\multiput(1189.00,474.17)(1.500,-1.000){2}{\rule{0.361pt}{0.400pt}}
\put(1192,472.67){\rule{0.723pt}{0.400pt}}
\multiput(1192.00,473.17)(1.500,-1.000){2}{\rule{0.361pt}{0.400pt}}
\put(1195,471.17){\rule{0.700pt}{0.400pt}}
\multiput(1195.00,472.17)(1.547,-2.000){2}{\rule{0.350pt}{0.400pt}}
\put(1198,469.67){\rule{0.723pt}{0.400pt}}
\multiput(1198.00,470.17)(1.500,-1.000){2}{\rule{0.361pt}{0.400pt}}
\put(1201,468.67){\rule{0.723pt}{0.400pt}}
\multiput(1201.00,469.17)(1.500,-1.000){2}{\rule{0.361pt}{0.400pt}}
\put(1204,467.67){\rule{0.723pt}{0.400pt}}
\multiput(1204.00,468.17)(1.500,-1.000){2}{\rule{0.361pt}{0.400pt}}
\put(1207,466.67){\rule{0.723pt}{0.400pt}}
\multiput(1207.00,467.17)(1.500,-1.000){2}{\rule{0.361pt}{0.400pt}}
\put(1210,465.67){\rule{0.723pt}{0.400pt}}
\multiput(1210.00,466.17)(1.500,-1.000){2}{\rule{0.361pt}{0.400pt}}
\put(1213,464.17){\rule{0.700pt}{0.400pt}}
\multiput(1213.00,465.17)(1.547,-2.000){2}{\rule{0.350pt}{0.400pt}}
\put(1216,462.67){\rule{0.723pt}{0.400pt}}
\multiput(1216.00,463.17)(1.500,-1.000){2}{\rule{0.361pt}{0.400pt}}
\put(1219,461.67){\rule{0.723pt}{0.400pt}}
\multiput(1219.00,462.17)(1.500,-1.000){2}{\rule{0.361pt}{0.400pt}}
\put(1222,460.67){\rule{0.723pt}{0.400pt}}
\multiput(1222.00,461.17)(1.500,-1.000){2}{\rule{0.361pt}{0.400pt}}
\put(1225,459.67){\rule{0.723pt}{0.400pt}}
\multiput(1225.00,460.17)(1.500,-1.000){2}{\rule{0.361pt}{0.400pt}}
\put(1228,458.67){\rule{0.723pt}{0.400pt}}
\multiput(1228.00,459.17)(1.500,-1.000){2}{\rule{0.361pt}{0.400pt}}
\put(1231,457.17){\rule{0.700pt}{0.400pt}}
\multiput(1231.00,458.17)(1.547,-2.000){2}{\rule{0.350pt}{0.400pt}}
\put(1234,455.67){\rule{0.964pt}{0.400pt}}
\multiput(1234.00,456.17)(2.000,-1.000){2}{\rule{0.482pt}{0.400pt}}
\put(1238,454.67){\rule{0.723pt}{0.400pt}}
\multiput(1238.00,455.17)(1.500,-1.000){2}{\rule{0.361pt}{0.400pt}}
\put(1241,453.67){\rule{0.723pt}{0.400pt}}
\multiput(1241.00,454.17)(1.500,-1.000){2}{\rule{0.361pt}{0.400pt}}
\put(1244,452.67){\rule{0.723pt}{0.400pt}}
\multiput(1244.00,453.17)(1.500,-1.000){2}{\rule{0.361pt}{0.400pt}}
\put(1247,451.67){\rule{0.723pt}{0.400pt}}
\multiput(1247.00,452.17)(1.500,-1.000){2}{\rule{0.361pt}{0.400pt}}
\put(1250,450.17){\rule{0.700pt}{0.400pt}}
\multiput(1250.00,451.17)(1.547,-2.000){2}{\rule{0.350pt}{0.400pt}}
\put(1253,448.67){\rule{0.723pt}{0.400pt}}
\multiput(1253.00,449.17)(1.500,-1.000){2}{\rule{0.361pt}{0.400pt}}
\put(1256,447.67){\rule{0.723pt}{0.400pt}}
\multiput(1256.00,448.17)(1.500,-1.000){2}{\rule{0.361pt}{0.400pt}}
\put(1259,446.67){\rule{0.723pt}{0.400pt}}
\multiput(1259.00,447.17)(1.500,-1.000){2}{\rule{0.361pt}{0.400pt}}
\put(1262,445.67){\rule{0.723pt}{0.400pt}}
\multiput(1262.00,446.17)(1.500,-1.000){2}{\rule{0.361pt}{0.400pt}}
\put(1265,444.67){\rule{0.723pt}{0.400pt}}
\multiput(1265.00,445.17)(1.500,-1.000){2}{\rule{0.361pt}{0.400pt}}
\put(1268,443.17){\rule{0.700pt}{0.400pt}}
\multiput(1268.00,444.17)(1.547,-2.000){2}{\rule{0.350pt}{0.400pt}}
\put(1271,441.67){\rule{0.723pt}{0.400pt}}
\multiput(1271.00,442.17)(1.500,-1.000){2}{\rule{0.361pt}{0.400pt}}
\put(1274,440.67){\rule{0.723pt}{0.400pt}}
\multiput(1274.00,441.17)(1.500,-1.000){2}{\rule{0.361pt}{0.400pt}}
\put(1277,439.67){\rule{0.723pt}{0.400pt}}
\multiput(1277.00,440.17)(1.500,-1.000){2}{\rule{0.361pt}{0.400pt}}
\put(1280,438.67){\rule{0.723pt}{0.400pt}}
\multiput(1280.00,439.17)(1.500,-1.000){2}{\rule{0.361pt}{0.400pt}}
\put(1283,437.67){\rule{0.723pt}{0.400pt}}
\multiput(1283.00,438.17)(1.500,-1.000){2}{\rule{0.361pt}{0.400pt}}
\put(1286,436.17){\rule{0.700pt}{0.400pt}}
\multiput(1286.00,437.17)(1.547,-2.000){2}{\rule{0.350pt}{0.400pt}}
\put(1289,434.67){\rule{0.723pt}{0.400pt}}
\multiput(1289.00,435.17)(1.500,-1.000){2}{\rule{0.361pt}{0.400pt}}
\put(1292,433.67){\rule{0.723pt}{0.400pt}}
\multiput(1292.00,434.17)(1.500,-1.000){2}{\rule{0.361pt}{0.400pt}}
\put(1295,432.67){\rule{0.964pt}{0.400pt}}
\multiput(1295.00,433.17)(2.000,-1.000){2}{\rule{0.482pt}{0.400pt}}
\put(1299,431.67){\rule{0.723pt}{0.400pt}}
\multiput(1299.00,432.17)(1.500,-1.000){2}{\rule{0.361pt}{0.400pt}}
\put(1302,430.67){\rule{0.723pt}{0.400pt}}
\multiput(1302.00,431.17)(1.500,-1.000){2}{\rule{0.361pt}{0.400pt}}
\put(1305,429.17){\rule{0.700pt}{0.400pt}}
\multiput(1305.00,430.17)(1.547,-2.000){2}{\rule{0.350pt}{0.400pt}}
\put(1308,427.67){\rule{0.723pt}{0.400pt}}
\multiput(1308.00,428.17)(1.500,-1.000){2}{\rule{0.361pt}{0.400pt}}
\put(1311,426.67){\rule{0.723pt}{0.400pt}}
\multiput(1311.00,427.17)(1.500,-1.000){2}{\rule{0.361pt}{0.400pt}}
\put(1314,425.67){\rule{0.723pt}{0.400pt}}
\multiput(1314.00,426.17)(1.500,-1.000){2}{\rule{0.361pt}{0.400pt}}
\put(1317,424.67){\rule{0.723pt}{0.400pt}}
\multiput(1317.00,425.17)(1.500,-1.000){2}{\rule{0.361pt}{0.400pt}}
\put(1320,423.67){\rule{0.723pt}{0.400pt}}
\multiput(1320.00,424.17)(1.500,-1.000){2}{\rule{0.361pt}{0.400pt}}
\put(1323,422.17){\rule{0.700pt}{0.400pt}}
\multiput(1323.00,423.17)(1.547,-2.000){2}{\rule{0.350pt}{0.400pt}}
\put(1326,420.67){\rule{0.723pt}{0.400pt}}
\multiput(1326.00,421.17)(1.500,-1.000){2}{\rule{0.361pt}{0.400pt}}
\put(1329,419.67){\rule{0.723pt}{0.400pt}}
\multiput(1329.00,420.17)(1.500,-1.000){2}{\rule{0.361pt}{0.400pt}}
\put(1332,418.67){\rule{0.723pt}{0.400pt}}
\multiput(1332.00,419.17)(1.500,-1.000){2}{\rule{0.361pt}{0.400pt}}
\put(1335,417.67){\rule{0.723pt}{0.400pt}}
\multiput(1335.00,418.17)(1.500,-1.000){2}{\rule{0.361pt}{0.400pt}}
\put(1338,416.67){\rule{0.723pt}{0.400pt}}
\multiput(1338.00,417.17)(1.500,-1.000){2}{\rule{0.361pt}{0.400pt}}
\put(1341,415.17){\rule{0.700pt}{0.400pt}}
\multiput(1341.00,416.17)(1.547,-2.000){2}{\rule{0.350pt}{0.400pt}}
\put(1344,413.67){\rule{0.723pt}{0.400pt}}
\multiput(1344.00,414.17)(1.500,-1.000){2}{\rule{0.361pt}{0.400pt}}
\put(1347,412.67){\rule{0.723pt}{0.400pt}}
\multiput(1347.00,413.17)(1.500,-1.000){2}{\rule{0.361pt}{0.400pt}}
\put(1350,411.67){\rule{0.723pt}{0.400pt}}
\multiput(1350.00,412.17)(1.500,-1.000){2}{\rule{0.361pt}{0.400pt}}
\put(1353,410.67){\rule{0.964pt}{0.400pt}}
\multiput(1353.00,411.17)(2.000,-1.000){2}{\rule{0.482pt}{0.400pt}}
\put(1357,409.67){\rule{0.723pt}{0.400pt}}
\multiput(1357.00,410.17)(1.500,-1.000){2}{\rule{0.361pt}{0.400pt}}
\put(1360,408.17){\rule{0.700pt}{0.400pt}}
\multiput(1360.00,409.17)(1.547,-2.000){2}{\rule{0.350pt}{0.400pt}}
\put(1363,406.67){\rule{0.723pt}{0.400pt}}
\multiput(1363.00,407.17)(1.500,-1.000){2}{\rule{0.361pt}{0.400pt}}
\put(1366,405.67){\rule{0.723pt}{0.400pt}}
\multiput(1366.00,406.17)(1.500,-1.000){2}{\rule{0.361pt}{0.400pt}}
\put(1369,404.67){\rule{0.723pt}{0.400pt}}
\multiput(1369.00,405.17)(1.500,-1.000){2}{\rule{0.361pt}{0.400pt}}
\put(1372,403.67){\rule{0.723pt}{0.400pt}}
\multiput(1372.00,404.17)(1.500,-1.000){2}{\rule{0.361pt}{0.400pt}}
\put(1375,402.67){\rule{0.723pt}{0.400pt}}
\multiput(1375.00,403.17)(1.500,-1.000){2}{\rule{0.361pt}{0.400pt}}
\put(1378,401.17){\rule{0.700pt}{0.400pt}}
\multiput(1378.00,402.17)(1.547,-2.000){2}{\rule{0.350pt}{0.400pt}}
\put(1381,399.67){\rule{0.723pt}{0.400pt}}
\multiput(1381.00,400.17)(1.500,-1.000){2}{\rule{0.361pt}{0.400pt}}
\put(1384,398.67){\rule{0.723pt}{0.400pt}}
\multiput(1384.00,399.17)(1.500,-1.000){2}{\rule{0.361pt}{0.400pt}}
\put(1387,397.67){\rule{0.723pt}{0.400pt}}
\multiput(1387.00,398.17)(1.500,-1.000){2}{\rule{0.361pt}{0.400pt}}
\put(1390,396.67){\rule{0.723pt}{0.400pt}}
\multiput(1390.00,397.17)(1.500,-1.000){2}{\rule{0.361pt}{0.400pt}}
\put(1393,395.67){\rule{0.723pt}{0.400pt}}
\multiput(1393.00,396.17)(1.500,-1.000){2}{\rule{0.361pt}{0.400pt}}
\put(1396,394.17){\rule{0.700pt}{0.400pt}}
\multiput(1396.00,395.17)(1.547,-2.000){2}{\rule{0.350pt}{0.400pt}}
\put(1399,392.67){\rule{0.723pt}{0.400pt}}
\multiput(1399.00,393.17)(1.500,-1.000){2}{\rule{0.361pt}{0.400pt}}
\put(1402,391.67){\rule{0.723pt}{0.400pt}}
\multiput(1402.00,392.17)(1.500,-1.000){2}{\rule{0.361pt}{0.400pt}}
\put(1405,390.67){\rule{0.723pt}{0.400pt}}
\multiput(1405.00,391.17)(1.500,-1.000){2}{\rule{0.361pt}{0.400pt}}
\put(1408,389.67){\rule{0.723pt}{0.400pt}}
\multiput(1408.00,390.17)(1.500,-1.000){2}{\rule{0.361pt}{0.400pt}}
\put(1411,388.67){\rule{0.964pt}{0.400pt}}
\multiput(1411.00,389.17)(2.000,-1.000){2}{\rule{0.482pt}{0.400pt}}
\put(1411,387.17){\rule{0.900pt}{0.400pt}}
\multiput(1413.13,388.17)(-2.132,-2.000){2}{\rule{0.450pt}{0.400pt}}
\put(1408,385.67){\rule{0.723pt}{0.400pt}}
\multiput(1409.50,386.17)(-1.500,-1.000){2}{\rule{0.361pt}{0.400pt}}
\put(1405,384.67){\rule{0.723pt}{0.400pt}}
\multiput(1406.50,385.17)(-1.500,-1.000){2}{\rule{0.361pt}{0.400pt}}
\put(1402,383.67){\rule{0.723pt}{0.400pt}}
\multiput(1403.50,384.17)(-1.500,-1.000){2}{\rule{0.361pt}{0.400pt}}
\put(1399,382.67){\rule{0.723pt}{0.400pt}}
\multiput(1400.50,383.17)(-1.500,-1.000){2}{\rule{0.361pt}{0.400pt}}
\put(1396,381.67){\rule{0.723pt}{0.400pt}}
\multiput(1397.50,382.17)(-1.500,-1.000){2}{\rule{0.361pt}{0.400pt}}
\put(1393,380.17){\rule{0.700pt}{0.400pt}}
\multiput(1394.55,381.17)(-1.547,-2.000){2}{\rule{0.350pt}{0.400pt}}
\put(1390,378.67){\rule{0.723pt}{0.400pt}}
\multiput(1391.50,379.17)(-1.500,-1.000){2}{\rule{0.361pt}{0.400pt}}
\put(1387,377.67){\rule{0.723pt}{0.400pt}}
\multiput(1388.50,378.17)(-1.500,-1.000){2}{\rule{0.361pt}{0.400pt}}
\put(1384,376.67){\rule{0.723pt}{0.400pt}}
\multiput(1385.50,377.17)(-1.500,-1.000){2}{\rule{0.361pt}{0.400pt}}
\put(1381,375.67){\rule{0.723pt}{0.400pt}}
\multiput(1382.50,376.17)(-1.500,-1.000){2}{\rule{0.361pt}{0.400pt}}
\put(1378,374.67){\rule{0.723pt}{0.400pt}}
\multiput(1379.50,375.17)(-1.500,-1.000){2}{\rule{0.361pt}{0.400pt}}
\put(1375,373.17){\rule{0.700pt}{0.400pt}}
\multiput(1376.55,374.17)(-1.547,-2.000){2}{\rule{0.350pt}{0.400pt}}
\put(1372,371.67){\rule{0.723pt}{0.400pt}}
\multiput(1373.50,372.17)(-1.500,-1.000){2}{\rule{0.361pt}{0.400pt}}
\put(1369,370.67){\rule{0.723pt}{0.400pt}}
\multiput(1370.50,371.17)(-1.500,-1.000){2}{\rule{0.361pt}{0.400pt}}
\put(1366,369.67){\rule{0.723pt}{0.400pt}}
\multiput(1367.50,370.17)(-1.500,-1.000){2}{\rule{0.361pt}{0.400pt}}
\put(1363,368.67){\rule{0.723pt}{0.400pt}}
\multiput(1364.50,369.17)(-1.500,-1.000){2}{\rule{0.361pt}{0.400pt}}
\put(1360,367.67){\rule{0.723pt}{0.400pt}}
\multiput(1361.50,368.17)(-1.500,-1.000){2}{\rule{0.361pt}{0.400pt}}
\put(1357,366.17){\rule{0.700pt}{0.400pt}}
\multiput(1358.55,367.17)(-1.547,-2.000){2}{\rule{0.350pt}{0.400pt}}
\put(1353,364.67){\rule{0.964pt}{0.400pt}}
\multiput(1355.00,365.17)(-2.000,-1.000){2}{\rule{0.482pt}{0.400pt}}
\put(1350,363.67){\rule{0.723pt}{0.400pt}}
\multiput(1351.50,364.17)(-1.500,-1.000){2}{\rule{0.361pt}{0.400pt}}
\put(1347,362.67){\rule{0.723pt}{0.400pt}}
\multiput(1348.50,363.17)(-1.500,-1.000){2}{\rule{0.361pt}{0.400pt}}
\put(1344,361.67){\rule{0.723pt}{0.400pt}}
\multiput(1345.50,362.17)(-1.500,-1.000){2}{\rule{0.361pt}{0.400pt}}
\put(1341,360.67){\rule{0.723pt}{0.400pt}}
\multiput(1342.50,361.17)(-1.500,-1.000){2}{\rule{0.361pt}{0.400pt}}
\put(1338,359.17){\rule{0.700pt}{0.400pt}}
\multiput(1339.55,360.17)(-1.547,-2.000){2}{\rule{0.350pt}{0.400pt}}
\put(1335,357.67){\rule{0.723pt}{0.400pt}}
\multiput(1336.50,358.17)(-1.500,-1.000){2}{\rule{0.361pt}{0.400pt}}
\put(1332,356.67){\rule{0.723pt}{0.400pt}}
\multiput(1333.50,357.17)(-1.500,-1.000){2}{\rule{0.361pt}{0.400pt}}
\put(1329,355.67){\rule{0.723pt}{0.400pt}}
\multiput(1330.50,356.17)(-1.500,-1.000){2}{\rule{0.361pt}{0.400pt}}
\put(1326,354.67){\rule{0.723pt}{0.400pt}}
\multiput(1327.50,355.17)(-1.500,-1.000){2}{\rule{0.361pt}{0.400pt}}
\put(1323,353.67){\rule{0.723pt}{0.400pt}}
\multiput(1324.50,354.17)(-1.500,-1.000){2}{\rule{0.361pt}{0.400pt}}
\put(1320,352.17){\rule{0.700pt}{0.400pt}}
\multiput(1321.55,353.17)(-1.547,-2.000){2}{\rule{0.350pt}{0.400pt}}
\put(1317,350.67){\rule{0.723pt}{0.400pt}}
\multiput(1318.50,351.17)(-1.500,-1.000){2}{\rule{0.361pt}{0.400pt}}
\put(1314,349.67){\rule{0.723pt}{0.400pt}}
\multiput(1315.50,350.17)(-1.500,-1.000){2}{\rule{0.361pt}{0.400pt}}
\put(1311,348.67){\rule{0.723pt}{0.400pt}}
\multiput(1312.50,349.17)(-1.500,-1.000){2}{\rule{0.361pt}{0.400pt}}
\put(1308,347.67){\rule{0.723pt}{0.400pt}}
\multiput(1309.50,348.17)(-1.500,-1.000){2}{\rule{0.361pt}{0.400pt}}
\put(1305,346.67){\rule{0.723pt}{0.400pt}}
\multiput(1306.50,347.17)(-1.500,-1.000){2}{\rule{0.361pt}{0.400pt}}
\put(1302,345.17){\rule{0.700pt}{0.400pt}}
\multiput(1303.55,346.17)(-1.547,-2.000){2}{\rule{0.350pt}{0.400pt}}
\put(1299,343.67){\rule{0.723pt}{0.400pt}}
\multiput(1300.50,344.17)(-1.500,-1.000){2}{\rule{0.361pt}{0.400pt}}
\put(1295,342.67){\rule{0.964pt}{0.400pt}}
\multiput(1297.00,343.17)(-2.000,-1.000){2}{\rule{0.482pt}{0.400pt}}
\put(1292,341.67){\rule{0.723pt}{0.400pt}}
\multiput(1293.50,342.17)(-1.500,-1.000){2}{\rule{0.361pt}{0.400pt}}
\put(1289,340.67){\rule{0.723pt}{0.400pt}}
\multiput(1290.50,341.17)(-1.500,-1.000){2}{\rule{0.361pt}{0.400pt}}
\put(1286,339.67){\rule{0.723pt}{0.400pt}}
\multiput(1287.50,340.17)(-1.500,-1.000){2}{\rule{0.361pt}{0.400pt}}
\put(1283,338.17){\rule{0.700pt}{0.400pt}}
\multiput(1284.55,339.17)(-1.547,-2.000){2}{\rule{0.350pt}{0.400pt}}
\put(1280,336.67){\rule{0.723pt}{0.400pt}}
\multiput(1281.50,337.17)(-1.500,-1.000){2}{\rule{0.361pt}{0.400pt}}
\put(1277,335.67){\rule{0.723pt}{0.400pt}}
\multiput(1278.50,336.17)(-1.500,-1.000){2}{\rule{0.361pt}{0.400pt}}
\put(1274,334.67){\rule{0.723pt}{0.400pt}}
\multiput(1275.50,335.17)(-1.500,-1.000){2}{\rule{0.361pt}{0.400pt}}
\put(1271,333.67){\rule{0.723pt}{0.400pt}}
\multiput(1272.50,334.17)(-1.500,-1.000){2}{\rule{0.361pt}{0.400pt}}
\put(1268,332.67){\rule{0.723pt}{0.400pt}}
\multiput(1269.50,333.17)(-1.500,-1.000){2}{\rule{0.361pt}{0.400pt}}
\put(1265,331.17){\rule{0.700pt}{0.400pt}}
\multiput(1266.55,332.17)(-1.547,-2.000){2}{\rule{0.350pt}{0.400pt}}
\put(1262,329.67){\rule{0.723pt}{0.400pt}}
\multiput(1263.50,330.17)(-1.500,-1.000){2}{\rule{0.361pt}{0.400pt}}
\put(1259,328.67){\rule{0.723pt}{0.400pt}}
\multiput(1260.50,329.17)(-1.500,-1.000){2}{\rule{0.361pt}{0.400pt}}
\put(1256,327.67){\rule{0.723pt}{0.400pt}}
\multiput(1257.50,328.17)(-1.500,-1.000){2}{\rule{0.361pt}{0.400pt}}
\put(1253,326.67){\rule{0.723pt}{0.400pt}}
\multiput(1254.50,327.17)(-1.500,-1.000){2}{\rule{0.361pt}{0.400pt}}
\put(1250,325.67){\rule{0.723pt}{0.400pt}}
\multiput(1251.50,326.17)(-1.500,-1.000){2}{\rule{0.361pt}{0.400pt}}
\put(1247,324.17){\rule{0.700pt}{0.400pt}}
\multiput(1248.55,325.17)(-1.547,-2.000){2}{\rule{0.350pt}{0.400pt}}
\put(1244,322.67){\rule{0.723pt}{0.400pt}}
\multiput(1245.50,323.17)(-1.500,-1.000){2}{\rule{0.361pt}{0.400pt}}
\put(1241,321.67){\rule{0.723pt}{0.400pt}}
\multiput(1242.50,322.17)(-1.500,-1.000){2}{\rule{0.361pt}{0.400pt}}
\put(1238,320.67){\rule{0.723pt}{0.400pt}}
\multiput(1239.50,321.17)(-1.500,-1.000){2}{\rule{0.361pt}{0.400pt}}
\put(1234,319.67){\rule{0.964pt}{0.400pt}}
\multiput(1236.00,320.17)(-2.000,-1.000){2}{\rule{0.482pt}{0.400pt}}
\put(1231,318.67){\rule{0.723pt}{0.400pt}}
\multiput(1232.50,319.17)(-1.500,-1.000){2}{\rule{0.361pt}{0.400pt}}
\put(1228,317.17){\rule{0.700pt}{0.400pt}}
\multiput(1229.55,318.17)(-1.547,-2.000){2}{\rule{0.350pt}{0.400pt}}
\put(1225,315.67){\rule{0.723pt}{0.400pt}}
\multiput(1226.50,316.17)(-1.500,-1.000){2}{\rule{0.361pt}{0.400pt}}
\put(1222,314.67){\rule{0.723pt}{0.400pt}}
\multiput(1223.50,315.17)(-1.500,-1.000){2}{\rule{0.361pt}{0.400pt}}
\put(1219,313.67){\rule{0.723pt}{0.400pt}}
\multiput(1220.50,314.17)(-1.500,-1.000){2}{\rule{0.361pt}{0.400pt}}
\put(1216,312.67){\rule{0.723pt}{0.400pt}}
\multiput(1217.50,313.17)(-1.500,-1.000){2}{\rule{0.361pt}{0.400pt}}
\put(1213,311.67){\rule{0.723pt}{0.400pt}}
\multiput(1214.50,312.17)(-1.500,-1.000){2}{\rule{0.361pt}{0.400pt}}
\put(1210,310.67){\rule{0.723pt}{0.400pt}}
\multiput(1211.50,311.17)(-1.500,-1.000){2}{\rule{0.361pt}{0.400pt}}
\put(1207,309.17){\rule{0.700pt}{0.400pt}}
\multiput(1208.55,310.17)(-1.547,-2.000){2}{\rule{0.350pt}{0.400pt}}
\put(1204,307.67){\rule{0.723pt}{0.400pt}}
\multiput(1205.50,308.17)(-1.500,-1.000){2}{\rule{0.361pt}{0.400pt}}
\put(1201,306.67){\rule{0.723pt}{0.400pt}}
\multiput(1202.50,307.17)(-1.500,-1.000){2}{\rule{0.361pt}{0.400pt}}
\put(1198,305.67){\rule{0.723pt}{0.400pt}}
\multiput(1199.50,306.17)(-1.500,-1.000){2}{\rule{0.361pt}{0.400pt}}
\put(1195,304.67){\rule{0.723pt}{0.400pt}}
\multiput(1196.50,305.17)(-1.500,-1.000){2}{\rule{0.361pt}{0.400pt}}
\put(1192,303.67){\rule{0.723pt}{0.400pt}}
\multiput(1193.50,304.17)(-1.500,-1.000){2}{\rule{0.361pt}{0.400pt}}
\put(1189,302.17){\rule{0.700pt}{0.400pt}}
\multiput(1190.55,303.17)(-1.547,-2.000){2}{\rule{0.350pt}{0.400pt}}
\put(1186,300.67){\rule{0.723pt}{0.400pt}}
\multiput(1187.50,301.17)(-1.500,-1.000){2}{\rule{0.361pt}{0.400pt}}
\put(1183,299.67){\rule{0.723pt}{0.400pt}}
\multiput(1184.50,300.17)(-1.500,-1.000){2}{\rule{0.361pt}{0.400pt}}
\put(1180,298.67){\rule{0.723pt}{0.400pt}}
\multiput(1181.50,299.17)(-1.500,-1.000){2}{\rule{0.361pt}{0.400pt}}
\put(1176,297.67){\rule{0.964pt}{0.400pt}}
\multiput(1178.00,298.17)(-2.000,-1.000){2}{\rule{0.482pt}{0.400pt}}
\put(1173,296.67){\rule{0.723pt}{0.400pt}}
\multiput(1174.50,297.17)(-1.500,-1.000){2}{\rule{0.361pt}{0.400pt}}
\put(1170,295.17){\rule{0.700pt}{0.400pt}}
\multiput(1171.55,296.17)(-1.547,-2.000){2}{\rule{0.350pt}{0.400pt}}
\put(1167,293.67){\rule{0.723pt}{0.400pt}}
\multiput(1168.50,294.17)(-1.500,-1.000){2}{\rule{0.361pt}{0.400pt}}
\put(1164,292.67){\rule{0.723pt}{0.400pt}}
\multiput(1165.50,293.17)(-1.500,-1.000){2}{\rule{0.361pt}{0.400pt}}
\put(1161,291.67){\rule{0.723pt}{0.400pt}}
\multiput(1162.50,292.17)(-1.500,-1.000){2}{\rule{0.361pt}{0.400pt}}
\put(1158,290.67){\rule{0.723pt}{0.400pt}}
\multiput(1159.50,291.17)(-1.500,-1.000){2}{\rule{0.361pt}{0.400pt}}
\put(1155,289.67){\rule{0.723pt}{0.400pt}}
\multiput(1156.50,290.17)(-1.500,-1.000){2}{\rule{0.361pt}{0.400pt}}
\put(1152,288.17){\rule{0.700pt}{0.400pt}}
\multiput(1153.55,289.17)(-1.547,-2.000){2}{\rule{0.350pt}{0.400pt}}
\put(1149,286.67){\rule{0.723pt}{0.400pt}}
\multiput(1150.50,287.17)(-1.500,-1.000){2}{\rule{0.361pt}{0.400pt}}
\put(1146,285.67){\rule{0.723pt}{0.400pt}}
\multiput(1147.50,286.17)(-1.500,-1.000){2}{\rule{0.361pt}{0.400pt}}
\put(1143,284.67){\rule{0.723pt}{0.400pt}}
\multiput(1144.50,285.17)(-1.500,-1.000){2}{\rule{0.361pt}{0.400pt}}
\put(1140,283.67){\rule{0.723pt}{0.400pt}}
\multiput(1141.50,284.17)(-1.500,-1.000){2}{\rule{0.361pt}{0.400pt}}
\put(1137,282.67){\rule{0.723pt}{0.400pt}}
\multiput(1138.50,283.17)(-1.500,-1.000){2}{\rule{0.361pt}{0.400pt}}
\put(1134,281.17){\rule{0.700pt}{0.400pt}}
\multiput(1135.55,282.17)(-1.547,-2.000){2}{\rule{0.350pt}{0.400pt}}
\put(1131,279.67){\rule{0.723pt}{0.400pt}}
\multiput(1132.50,280.17)(-1.500,-1.000){2}{\rule{0.361pt}{0.400pt}}
\put(1128,278.67){\rule{0.723pt}{0.400pt}}
\multiput(1129.50,279.17)(-1.500,-1.000){2}{\rule{0.361pt}{0.400pt}}
\put(1125,277.67){\rule{0.723pt}{0.400pt}}
\multiput(1126.50,278.17)(-1.500,-1.000){2}{\rule{0.361pt}{0.400pt}}
\put(1122,276.67){\rule{0.723pt}{0.400pt}}
\multiput(1123.50,277.17)(-1.500,-1.000){2}{\rule{0.361pt}{0.400pt}}
\put(1118,275.67){\rule{0.964pt}{0.400pt}}
\multiput(1120.00,276.17)(-2.000,-1.000){2}{\rule{0.482pt}{0.400pt}}
\put(1115,274.17){\rule{0.700pt}{0.400pt}}
\multiput(1116.55,275.17)(-1.547,-2.000){2}{\rule{0.350pt}{0.400pt}}
\put(1112,272.67){\rule{0.723pt}{0.400pt}}
\multiput(1113.50,273.17)(-1.500,-1.000){2}{\rule{0.361pt}{0.400pt}}
\put(1109,271.67){\rule{0.723pt}{0.400pt}}
\multiput(1110.50,272.17)(-1.500,-1.000){2}{\rule{0.361pt}{0.400pt}}
\put(1106,270.67){\rule{0.723pt}{0.400pt}}
\multiput(1107.50,271.17)(-1.500,-1.000){2}{\rule{0.361pt}{0.400pt}}
\put(1103,269.67){\rule{0.723pt}{0.400pt}}
\multiput(1104.50,270.17)(-1.500,-1.000){2}{\rule{0.361pt}{0.400pt}}
\put(1100,268.67){\rule{0.723pt}{0.400pt}}
\multiput(1101.50,269.17)(-1.500,-1.000){2}{\rule{0.361pt}{0.400pt}}
\put(1097,267.17){\rule{0.700pt}{0.400pt}}
\multiput(1098.55,268.17)(-1.547,-2.000){2}{\rule{0.350pt}{0.400pt}}
\put(1094,265.67){\rule{0.723pt}{0.400pt}}
\multiput(1095.50,266.17)(-1.500,-1.000){2}{\rule{0.361pt}{0.400pt}}
\put(1091,264.67){\rule{0.723pt}{0.400pt}}
\multiput(1092.50,265.17)(-1.500,-1.000){2}{\rule{0.361pt}{0.400pt}}
\put(1088,263.67){\rule{0.723pt}{0.400pt}}
\multiput(1089.50,264.17)(-1.500,-1.000){2}{\rule{0.361pt}{0.400pt}}
\put(1085,262.67){\rule{0.723pt}{0.400pt}}
\multiput(1086.50,263.17)(-1.500,-1.000){2}{\rule{0.361pt}{0.400pt}}
\put(1082,261.67){\rule{0.723pt}{0.400pt}}
\multiput(1083.50,262.17)(-1.500,-1.000){2}{\rule{0.361pt}{0.400pt}}
\put(1079,260.17){\rule{0.700pt}{0.400pt}}
\multiput(1080.55,261.17)(-1.547,-2.000){2}{\rule{0.350pt}{0.400pt}}
\put(1076,258.67){\rule{0.723pt}{0.400pt}}
\multiput(1077.50,259.17)(-1.500,-1.000){2}{\rule{0.361pt}{0.400pt}}
\put(1073,257.67){\rule{0.723pt}{0.400pt}}
\multiput(1074.50,258.17)(-1.500,-1.000){2}{\rule{0.361pt}{0.400pt}}
\put(1070,256.67){\rule{0.723pt}{0.400pt}}
\multiput(1071.50,257.17)(-1.500,-1.000){2}{\rule{0.361pt}{0.400pt}}
\put(1067,255.67){\rule{0.723pt}{0.400pt}}
\multiput(1068.50,256.17)(-1.500,-1.000){2}{\rule{0.361pt}{0.400pt}}
\put(1064,254.67){\rule{0.723pt}{0.400pt}}
\multiput(1065.50,255.17)(-1.500,-1.000){2}{\rule{0.361pt}{0.400pt}}
\put(1060,253.17){\rule{0.900pt}{0.400pt}}
\multiput(1062.13,254.17)(-2.132,-2.000){2}{\rule{0.450pt}{0.400pt}}
\put(1057,251.67){\rule{0.723pt}{0.400pt}}
\multiput(1058.50,252.17)(-1.500,-1.000){2}{\rule{0.361pt}{0.400pt}}
\put(1054,250.67){\rule{0.723pt}{0.400pt}}
\multiput(1055.50,251.17)(-1.500,-1.000){2}{\rule{0.361pt}{0.400pt}}
\put(1051,249.67){\rule{0.723pt}{0.400pt}}
\multiput(1052.50,250.17)(-1.500,-1.000){2}{\rule{0.361pt}{0.400pt}}
\put(1048,248.67){\rule{0.723pt}{0.400pt}}
\multiput(1049.50,249.17)(-1.500,-1.000){2}{\rule{0.361pt}{0.400pt}}
\put(1045,247.67){\rule{0.723pt}{0.400pt}}
\multiput(1046.50,248.17)(-1.500,-1.000){2}{\rule{0.361pt}{0.400pt}}
\put(1042,246.17){\rule{0.700pt}{0.400pt}}
\multiput(1043.55,247.17)(-1.547,-2.000){2}{\rule{0.350pt}{0.400pt}}
\put(1039,244.67){\rule{0.723pt}{0.400pt}}
\multiput(1040.50,245.17)(-1.500,-1.000){2}{\rule{0.361pt}{0.400pt}}
\put(1036,243.67){\rule{0.723pt}{0.400pt}}
\multiput(1037.50,244.17)(-1.500,-1.000){2}{\rule{0.361pt}{0.400pt}}
\put(1033,242.67){\rule{0.723pt}{0.400pt}}
\multiput(1034.50,243.17)(-1.500,-1.000){2}{\rule{0.361pt}{0.400pt}}
\put(1030,241.67){\rule{0.723pt}{0.400pt}}
\multiput(1031.50,242.17)(-1.500,-1.000){2}{\rule{0.361pt}{0.400pt}}
\put(1027,240.67){\rule{0.723pt}{0.400pt}}
\multiput(1028.50,241.17)(-1.500,-1.000){2}{\rule{0.361pt}{0.400pt}}
\put(1024,239.17){\rule{0.700pt}{0.400pt}}
\multiput(1025.55,240.17)(-1.547,-2.000){2}{\rule{0.350pt}{0.400pt}}
\put(1021,237.67){\rule{0.723pt}{0.400pt}}
\multiput(1022.50,238.17)(-1.500,-1.000){2}{\rule{0.361pt}{0.400pt}}
\put(1018,236.67){\rule{0.723pt}{0.400pt}}
\multiput(1019.50,237.17)(-1.500,-1.000){2}{\rule{0.361pt}{0.400pt}}
\put(1015,235.67){\rule{0.723pt}{0.400pt}}
\multiput(1016.50,236.17)(-1.500,-1.000){2}{\rule{0.361pt}{0.400pt}}
\put(1012,234.67){\rule{0.723pt}{0.400pt}}
\multiput(1013.50,235.17)(-1.500,-1.000){2}{\rule{0.361pt}{0.400pt}}
\put(1009,233.67){\rule{0.723pt}{0.400pt}}
\multiput(1010.50,234.17)(-1.500,-1.000){2}{\rule{0.361pt}{0.400pt}}
\put(1006,232.17){\rule{0.700pt}{0.400pt}}
\multiput(1007.55,233.17)(-1.547,-2.000){2}{\rule{0.350pt}{0.400pt}}
\put(1002,230.67){\rule{0.964pt}{0.400pt}}
\multiput(1004.00,231.17)(-2.000,-1.000){2}{\rule{0.482pt}{0.400pt}}
\put(999,229.67){\rule{0.723pt}{0.400pt}}
\multiput(1000.50,230.17)(-1.500,-1.000){2}{\rule{0.361pt}{0.400pt}}
\put(996,228.67){\rule{0.723pt}{0.400pt}}
\multiput(997.50,229.17)(-1.500,-1.000){2}{\rule{0.361pt}{0.400pt}}
\put(993,227.67){\rule{0.723pt}{0.400pt}}
\multiput(994.50,228.17)(-1.500,-1.000){2}{\rule{0.361pt}{0.400pt}}
\put(990,226.67){\rule{0.723pt}{0.400pt}}
\multiput(991.50,227.17)(-1.500,-1.000){2}{\rule{0.361pt}{0.400pt}}
\put(987,225.17){\rule{0.700pt}{0.400pt}}
\multiput(988.55,226.17)(-1.547,-2.000){2}{\rule{0.350pt}{0.400pt}}
\put(984,223.67){\rule{0.723pt}{0.400pt}}
\multiput(985.50,224.17)(-1.500,-1.000){2}{\rule{0.361pt}{0.400pt}}
\put(981,222.67){\rule{0.723pt}{0.400pt}}
\multiput(982.50,223.17)(-1.500,-1.000){2}{\rule{0.361pt}{0.400pt}}
\put(978,221.67){\rule{0.723pt}{0.400pt}}
\multiput(979.50,222.17)(-1.500,-1.000){2}{\rule{0.361pt}{0.400pt}}
\put(975,220.67){\rule{0.723pt}{0.400pt}}
\multiput(976.50,221.17)(-1.500,-1.000){2}{\rule{0.361pt}{0.400pt}}
\put(972,219.67){\rule{0.723pt}{0.400pt}}
\multiput(973.50,220.17)(-1.500,-1.000){2}{\rule{0.361pt}{0.400pt}}
\put(969,218.17){\rule{0.700pt}{0.400pt}}
\multiput(970.55,219.17)(-1.547,-2.000){2}{\rule{0.350pt}{0.400pt}}
\put(966,216.67){\rule{0.723pt}{0.400pt}}
\multiput(967.50,217.17)(-1.500,-1.000){2}{\rule{0.361pt}{0.400pt}}
\put(963,215.67){\rule{0.723pt}{0.400pt}}
\multiput(964.50,216.17)(-1.500,-1.000){2}{\rule{0.361pt}{0.400pt}}
\put(960,214.67){\rule{0.723pt}{0.400pt}}
\multiput(961.50,215.17)(-1.500,-1.000){2}{\rule{0.361pt}{0.400pt}}
\put(957,213.67){\rule{0.723pt}{0.400pt}}
\multiput(958.50,214.17)(-1.500,-1.000){2}{\rule{0.361pt}{0.400pt}}
\put(954,212.67){\rule{0.723pt}{0.400pt}}
\multiput(955.50,213.17)(-1.500,-1.000){2}{\rule{0.361pt}{0.400pt}}
\put(951,211.17){\rule{0.700pt}{0.400pt}}
\multiput(952.55,212.17)(-1.547,-2.000){2}{\rule{0.350pt}{0.400pt}}
\put(948,209.67){\rule{0.723pt}{0.400pt}}
\multiput(949.50,210.17)(-1.500,-1.000){2}{\rule{0.361pt}{0.400pt}}
\put(944,208.67){\rule{0.964pt}{0.400pt}}
\multiput(946.00,209.17)(-2.000,-1.000){2}{\rule{0.482pt}{0.400pt}}
\put(941,207.67){\rule{0.723pt}{0.400pt}}
\multiput(942.50,208.17)(-1.500,-1.000){2}{\rule{0.361pt}{0.400pt}}
\put(938,206.67){\rule{0.723pt}{0.400pt}}
\multiput(939.50,207.17)(-1.500,-1.000){2}{\rule{0.361pt}{0.400pt}}
\put(935,205.67){\rule{0.723pt}{0.400pt}}
\multiput(936.50,206.17)(-1.500,-1.000){2}{\rule{0.361pt}{0.400pt}}
\put(932,204.17){\rule{0.700pt}{0.400pt}}
\multiput(933.55,205.17)(-1.547,-2.000){2}{\rule{0.350pt}{0.400pt}}
\put(929,202.67){\rule{0.723pt}{0.400pt}}
\multiput(930.50,203.17)(-1.500,-1.000){2}{\rule{0.361pt}{0.400pt}}
\put(926,201.67){\rule{0.723pt}{0.400pt}}
\multiput(927.50,202.17)(-1.500,-1.000){2}{\rule{0.361pt}{0.400pt}}
\put(923,200.67){\rule{0.723pt}{0.400pt}}
\multiput(924.50,201.17)(-1.500,-1.000){2}{\rule{0.361pt}{0.400pt}}
\put(920,199.67){\rule{0.723pt}{0.400pt}}
\multiput(921.50,200.17)(-1.500,-1.000){2}{\rule{0.361pt}{0.400pt}}
\put(917,198.67){\rule{0.723pt}{0.400pt}}
\multiput(918.50,199.17)(-1.500,-1.000){2}{\rule{0.361pt}{0.400pt}}
\put(914,197.17){\rule{0.700pt}{0.400pt}}
\multiput(915.55,198.17)(-1.547,-2.000){2}{\rule{0.350pt}{0.400pt}}
\put(911,195.67){\rule{0.723pt}{0.400pt}}
\multiput(912.50,196.17)(-1.500,-1.000){2}{\rule{0.361pt}{0.400pt}}
\put(908,194.67){\rule{0.723pt}{0.400pt}}
\multiput(909.50,195.17)(-1.500,-1.000){2}{\rule{0.361pt}{0.400pt}}
\put(905,193.67){\rule{0.723pt}{0.400pt}}
\multiput(906.50,194.17)(-1.500,-1.000){2}{\rule{0.361pt}{0.400pt}}
\put(902,192.67){\rule{0.723pt}{0.400pt}}
\multiput(903.50,193.17)(-1.500,-1.000){2}{\rule{0.361pt}{0.400pt}}
\put(899,191.67){\rule{0.723pt}{0.400pt}}
\multiput(900.50,192.17)(-1.500,-1.000){2}{\rule{0.361pt}{0.400pt}}
\put(896,190.17){\rule{0.700pt}{0.400pt}}
\multiput(897.55,191.17)(-1.547,-2.000){2}{\rule{0.350pt}{0.400pt}}
\put(893,188.67){\rule{0.723pt}{0.400pt}}
\multiput(894.50,189.17)(-1.500,-1.000){2}{\rule{0.361pt}{0.400pt}}
\put(890,187.67){\rule{0.723pt}{0.400pt}}
\multiput(891.50,188.17)(-1.500,-1.000){2}{\rule{0.361pt}{0.400pt}}
\put(887,186.67){\rule{0.723pt}{0.400pt}}
\multiput(888.50,187.17)(-1.500,-1.000){2}{\rule{0.361pt}{0.400pt}}
\put(883,185.67){\rule{0.964pt}{0.400pt}}
\multiput(885.00,186.17)(-2.000,-1.000){2}{\rule{0.482pt}{0.400pt}}
\put(880,184.67){\rule{0.723pt}{0.400pt}}
\multiput(881.50,185.17)(-1.500,-1.000){2}{\rule{0.361pt}{0.400pt}}
\put(877,183.17){\rule{0.700pt}{0.400pt}}
\multiput(878.55,184.17)(-1.547,-2.000){2}{\rule{0.350pt}{0.400pt}}
\put(874,181.67){\rule{0.723pt}{0.400pt}}
\multiput(875.50,182.17)(-1.500,-1.000){2}{\rule{0.361pt}{0.400pt}}
\put(871,180.67){\rule{0.723pt}{0.400pt}}
\multiput(872.50,181.17)(-1.500,-1.000){2}{\rule{0.361pt}{0.400pt}}
\put(868,179.67){\rule{0.723pt}{0.400pt}}
\multiput(869.50,180.17)(-1.500,-1.000){2}{\rule{0.361pt}{0.400pt}}
\put(865,178.67){\rule{0.723pt}{0.400pt}}
\multiput(866.50,179.17)(-1.500,-1.000){2}{\rule{0.361pt}{0.400pt}}
\put(862,177.67){\rule{0.723pt}{0.400pt}}
\multiput(863.50,178.17)(-1.500,-1.000){2}{\rule{0.361pt}{0.400pt}}
\put(859,176.17){\rule{0.700pt}{0.400pt}}
\multiput(860.55,177.17)(-1.547,-2.000){2}{\rule{0.350pt}{0.400pt}}
\put(856,174.67){\rule{0.723pt}{0.400pt}}
\multiput(857.50,175.17)(-1.500,-1.000){2}{\rule{0.361pt}{0.400pt}}
\put(853,173.67){\rule{0.723pt}{0.400pt}}
\multiput(854.50,174.17)(-1.500,-1.000){2}{\rule{0.361pt}{0.400pt}}
\put(850,172.67){\rule{0.723pt}{0.400pt}}
\multiput(851.50,173.17)(-1.500,-1.000){2}{\rule{0.361pt}{0.400pt}}
\put(847,171.67){\rule{0.723pt}{0.400pt}}
\multiput(848.50,172.17)(-1.500,-1.000){2}{\rule{0.361pt}{0.400pt}}
\put(844,170.67){\rule{0.723pt}{0.400pt}}
\multiput(845.50,171.17)(-1.500,-1.000){2}{\rule{0.361pt}{0.400pt}}
\put(841,169.17){\rule{0.700pt}{0.400pt}}
\multiput(842.55,170.17)(-1.547,-2.000){2}{\rule{0.350pt}{0.400pt}}
\put(838,167.67){\rule{0.723pt}{0.400pt}}
\multiput(839.50,168.17)(-1.500,-1.000){2}{\rule{0.361pt}{0.400pt}}
\put(835,166.67){\rule{0.723pt}{0.400pt}}
\multiput(836.50,167.17)(-1.500,-1.000){2}{\rule{0.361pt}{0.400pt}}
\put(832,165.67){\rule{0.723pt}{0.400pt}}
\multiput(833.50,166.17)(-1.500,-1.000){2}{\rule{0.361pt}{0.400pt}}
\put(829,164.67){\rule{0.723pt}{0.400pt}}
\multiput(830.50,165.17)(-1.500,-1.000){2}{\rule{0.361pt}{0.400pt}}
\put(825,163.67){\rule{0.964pt}{0.400pt}}
\multiput(827.00,164.17)(-2.000,-1.000){2}{\rule{0.482pt}{0.400pt}}
\put(822,162.17){\rule{0.700pt}{0.400pt}}
\multiput(823.55,163.17)(-1.547,-2.000){2}{\rule{0.350pt}{0.400pt}}
\put(819,160.67){\rule{0.723pt}{0.400pt}}
\multiput(820.50,161.17)(-1.500,-1.000){2}{\rule{0.361pt}{0.400pt}}
\put(816,159.67){\rule{0.723pt}{0.400pt}}
\multiput(817.50,160.17)(-1.500,-1.000){2}{\rule{0.361pt}{0.400pt}}
\put(813,158.67){\rule{0.723pt}{0.400pt}}
\multiput(814.50,159.17)(-1.500,-1.000){2}{\rule{0.361pt}{0.400pt}}
\put(810,157.67){\rule{0.723pt}{0.400pt}}
\multiput(811.50,158.17)(-1.500,-1.000){2}{\rule{0.361pt}{0.400pt}}
\put(807,156.67){\rule{0.723pt}{0.400pt}}
\multiput(808.50,157.17)(-1.500,-1.000){2}{\rule{0.361pt}{0.400pt}}
\put(804,155.17){\rule{0.700pt}{0.400pt}}
\multiput(805.55,156.17)(-1.547,-2.000){2}{\rule{0.350pt}{0.400pt}}
\put(801,153.67){\rule{0.723pt}{0.400pt}}
\multiput(802.50,154.17)(-1.500,-1.000){2}{\rule{0.361pt}{0.400pt}}
\put(798,152.67){\rule{0.723pt}{0.400pt}}
\multiput(799.50,153.17)(-1.500,-1.000){2}{\rule{0.361pt}{0.400pt}}
\put(795,151.67){\rule{0.723pt}{0.400pt}}
\multiput(796.50,152.17)(-1.500,-1.000){2}{\rule{0.361pt}{0.400pt}}
\put(792,150.67){\rule{0.723pt}{0.400pt}}
\multiput(793.50,151.17)(-1.500,-1.000){2}{\rule{0.361pt}{0.400pt}}
\put(789,149.67){\rule{0.723pt}{0.400pt}}
\multiput(790.50,150.17)(-1.500,-1.000){2}{\rule{0.361pt}{0.400pt}}
\put(786,148.17){\rule{0.700pt}{0.400pt}}
\multiput(787.55,149.17)(-1.547,-2.000){2}{\rule{0.350pt}{0.400pt}}
\put(783,146.67){\rule{0.723pt}{0.400pt}}
\multiput(784.50,147.17)(-1.500,-1.000){2}{\rule{0.361pt}{0.400pt}}
\put(780,145.67){\rule{0.723pt}{0.400pt}}
\multiput(781.50,146.17)(-1.500,-1.000){2}{\rule{0.361pt}{0.400pt}}
\put(777,144.67){\rule{0.723pt}{0.400pt}}
\multiput(778.50,145.17)(-1.500,-1.000){2}{\rule{0.361pt}{0.400pt}}
\put(774,143.67){\rule{0.723pt}{0.400pt}}
\multiput(775.50,144.17)(-1.500,-1.000){2}{\rule{0.361pt}{0.400pt}}
\put(771,142.67){\rule{0.723pt}{0.400pt}}
\multiput(772.50,143.17)(-1.500,-1.000){2}{\rule{0.361pt}{0.400pt}}
\put(767,141.17){\rule{0.900pt}{0.400pt}}
\multiput(769.13,142.17)(-2.132,-2.000){2}{\rule{0.450pt}{0.400pt}}
\put(764,139.67){\rule{0.723pt}{0.400pt}}
\multiput(765.50,140.17)(-1.500,-1.000){2}{\rule{0.361pt}{0.400pt}}
\put(761,138.67){\rule{0.723pt}{0.400pt}}
\multiput(762.50,139.17)(-1.500,-1.000){2}{\rule{0.361pt}{0.400pt}}
\put(758,137.67){\rule{0.723pt}{0.400pt}}
\multiput(759.50,138.17)(-1.500,-1.000){2}{\rule{0.361pt}{0.400pt}}
\put(755,136.67){\rule{0.723pt}{0.400pt}}
\multiput(756.50,137.17)(-1.500,-1.000){2}{\rule{0.361pt}{0.400pt}}
\put(752,135.67){\rule{0.723pt}{0.400pt}}
\multiput(753.50,136.17)(-1.500,-1.000){2}{\rule{0.361pt}{0.400pt}}
\put(749,134.17){\rule{0.700pt}{0.400pt}}
\multiput(750.55,135.17)(-1.547,-2.000){2}{\rule{0.350pt}{0.400pt}}
\put(746,132.67){\rule{0.723pt}{0.400pt}}
\multiput(747.50,133.17)(-1.500,-1.000){2}{\rule{0.361pt}{0.400pt}}
\put(743,131.67){\rule{0.723pt}{0.400pt}}
\multiput(744.50,132.17)(-1.500,-1.000){2}{\rule{0.361pt}{0.400pt}}
\put(740,130.67){\rule{0.723pt}{0.400pt}}
\multiput(741.50,131.17)(-1.500,-1.000){2}{\rule{0.361pt}{0.400pt}}
\put(737,129.67){\rule{0.723pt}{0.400pt}}
\multiput(738.50,130.17)(-1.500,-1.000){2}{\rule{0.361pt}{0.400pt}}
\put(734,128.67){\rule{0.723pt}{0.400pt}}
\multiput(735.50,129.17)(-1.500,-1.000){2}{\rule{0.361pt}{0.400pt}}
\put(731,127.17){\rule{0.700pt}{0.400pt}}
\multiput(732.55,128.17)(-1.547,-2.000){2}{\rule{0.350pt}{0.400pt}}
\put(728,125.67){\rule{0.723pt}{0.400pt}}
\multiput(729.50,126.17)(-1.500,-1.000){2}{\rule{0.361pt}{0.400pt}}
\put(725,124.67){\rule{0.723pt}{0.400pt}}
\multiput(726.50,125.17)(-1.500,-1.000){2}{\rule{0.361pt}{0.400pt}}
\put(722,123.67){\rule{0.723pt}{0.400pt}}
\multiput(723.50,124.17)(-1.500,-1.000){2}{\rule{0.361pt}{0.400pt}}
\put(719,122.67){\rule{0.723pt}{0.400pt}}
\multiput(720.50,123.17)(-1.500,-1.000){2}{\rule{0.361pt}{0.400pt}}
\put(716,121.67){\rule{0.723pt}{0.400pt}}
\multiput(717.50,122.17)(-1.500,-1.000){2}{\rule{0.361pt}{0.400pt}}
\put(713,120.17){\rule{0.700pt}{0.400pt}}
\multiput(714.55,121.17)(-1.547,-2.000){2}{\rule{0.350pt}{0.400pt}}
\put(709,118.67){\rule{0.964pt}{0.400pt}}
\multiput(711.00,119.17)(-2.000,-1.000){2}{\rule{0.482pt}{0.400pt}}
\put(706,117.67){\rule{0.723pt}{0.400pt}}
\multiput(707.50,118.17)(-1.500,-1.000){2}{\rule{0.361pt}{0.400pt}}
\put(703,116.67){\rule{0.723pt}{0.400pt}}
\multiput(704.50,117.17)(-1.500,-1.000){2}{\rule{0.361pt}{0.400pt}}
\put(700,115.67){\rule{0.723pt}{0.400pt}}
\multiput(701.50,116.17)(-1.500,-1.000){2}{\rule{0.361pt}{0.400pt}}
\put(697,114.67){\rule{0.723pt}{0.400pt}}
\multiput(698.50,115.17)(-1.500,-1.000){2}{\rule{0.361pt}{0.400pt}}
\put(694,113.67){\rule{0.723pt}{0.400pt}}
\multiput(695.50,114.17)(-1.500,-1.000){2}{\rule{0.361pt}{0.400pt}}
\put(691,112.17){\rule{0.700pt}{0.400pt}}
\multiput(692.55,113.17)(-1.547,-2.000){2}{\rule{0.350pt}{0.400pt}}
\put(688,110.67){\rule{0.723pt}{0.400pt}}
\multiput(689.50,111.17)(-1.500,-1.000){2}{\rule{0.361pt}{0.400pt}}
\put(685,109.67){\rule{0.723pt}{0.400pt}}
\multiput(686.50,110.17)(-1.500,-1.000){2}{\rule{0.361pt}{0.400pt}}
\put(682,108.67){\rule{0.723pt}{0.400pt}}
\multiput(683.50,109.17)(-1.500,-1.000){2}{\rule{0.361pt}{0.400pt}}
\put(679,107.67){\rule{0.723pt}{0.400pt}}
\multiput(680.50,108.17)(-1.500,-1.000){2}{\rule{0.361pt}{0.400pt}}
\put(676,106.67){\rule{0.723pt}{0.400pt}}
\multiput(677.50,107.17)(-1.500,-1.000){2}{\rule{0.361pt}{0.400pt}}
\put(673,105.17){\rule{0.700pt}{0.400pt}}
\multiput(674.55,106.17)(-1.547,-2.000){2}{\rule{0.350pt}{0.400pt}}
\put(670,103.67){\rule{0.723pt}{0.400pt}}
\multiput(671.50,104.17)(-1.500,-1.000){2}{\rule{0.361pt}{0.400pt}}
\put(667,103.67){\rule{0.723pt}{0.400pt}}
\multiput(668.50,103.17)(-1.500,1.000){2}{\rule{0.361pt}{0.400pt}}
\put(664,105.17){\rule{0.700pt}{0.400pt}}
\multiput(665.55,104.17)(-1.547,2.000){2}{\rule{0.350pt}{0.400pt}}
\put(661,106.67){\rule{0.723pt}{0.400pt}}
\multiput(662.50,106.17)(-1.500,1.000){2}{\rule{0.361pt}{0.400pt}}
\put(658,107.67){\rule{0.723pt}{0.400pt}}
\multiput(659.50,107.17)(-1.500,1.000){2}{\rule{0.361pt}{0.400pt}}
\put(655,108.67){\rule{0.723pt}{0.400pt}}
\multiput(656.50,108.17)(-1.500,1.000){2}{\rule{0.361pt}{0.400pt}}
\put(651,109.67){\rule{0.964pt}{0.400pt}}
\multiput(653.00,109.17)(-2.000,1.000){2}{\rule{0.482pt}{0.400pt}}
\put(648,110.67){\rule{0.723pt}{0.400pt}}
\multiput(649.50,110.17)(-1.500,1.000){2}{\rule{0.361pt}{0.400pt}}
\put(645,112.17){\rule{0.700pt}{0.400pt}}
\multiput(646.55,111.17)(-1.547,2.000){2}{\rule{0.350pt}{0.400pt}}
\put(642,113.67){\rule{0.723pt}{0.400pt}}
\multiput(643.50,113.17)(-1.500,1.000){2}{\rule{0.361pt}{0.400pt}}
\put(639,114.67){\rule{0.723pt}{0.400pt}}
\multiput(640.50,114.17)(-1.500,1.000){2}{\rule{0.361pt}{0.400pt}}
\put(636,115.67){\rule{0.723pt}{0.400pt}}
\multiput(637.50,115.17)(-1.500,1.000){2}{\rule{0.361pt}{0.400pt}}
\put(633,116.67){\rule{0.723pt}{0.400pt}}
\multiput(634.50,116.17)(-1.500,1.000){2}{\rule{0.361pt}{0.400pt}}
\put(630,117.67){\rule{0.723pt}{0.400pt}}
\multiput(631.50,117.17)(-1.500,1.000){2}{\rule{0.361pt}{0.400pt}}
\put(627,118.67){\rule{0.723pt}{0.400pt}}
\multiput(628.50,118.17)(-1.500,1.000){2}{\rule{0.361pt}{0.400pt}}
\put(624,120.17){\rule{0.700pt}{0.400pt}}
\multiput(625.55,119.17)(-1.547,2.000){2}{\rule{0.350pt}{0.400pt}}
\put(621,121.67){\rule{0.723pt}{0.400pt}}
\multiput(622.50,121.17)(-1.500,1.000){2}{\rule{0.361pt}{0.400pt}}
\put(618,122.67){\rule{0.723pt}{0.400pt}}
\multiput(619.50,122.17)(-1.500,1.000){2}{\rule{0.361pt}{0.400pt}}
\put(615,123.67){\rule{0.723pt}{0.400pt}}
\multiput(616.50,123.17)(-1.500,1.000){2}{\rule{0.361pt}{0.400pt}}
\put(612,124.67){\rule{0.723pt}{0.400pt}}
\multiput(613.50,124.17)(-1.500,1.000){2}{\rule{0.361pt}{0.400pt}}
\put(609,125.67){\rule{0.723pt}{0.400pt}}
\multiput(610.50,125.17)(-1.500,1.000){2}{\rule{0.361pt}{0.400pt}}
\put(606,127.17){\rule{0.700pt}{0.400pt}}
\multiput(607.55,126.17)(-1.547,2.000){2}{\rule{0.350pt}{0.400pt}}
\put(603,128.67){\rule{0.723pt}{0.400pt}}
\multiput(604.50,128.17)(-1.500,1.000){2}{\rule{0.361pt}{0.400pt}}
\put(600,129.67){\rule{0.723pt}{0.400pt}}
\multiput(601.50,129.17)(-1.500,1.000){2}{\rule{0.361pt}{0.400pt}}
\put(597,130.67){\rule{0.723pt}{0.400pt}}
\multiput(598.50,130.17)(-1.500,1.000){2}{\rule{0.361pt}{0.400pt}}
\put(593,131.67){\rule{0.964pt}{0.400pt}}
\multiput(595.00,131.17)(-2.000,1.000){2}{\rule{0.482pt}{0.400pt}}
\put(590,132.67){\rule{0.723pt}{0.400pt}}
\multiput(591.50,132.17)(-1.500,1.000){2}{\rule{0.361pt}{0.400pt}}
\put(587,134.17){\rule{0.700pt}{0.400pt}}
\multiput(588.55,133.17)(-1.547,2.000){2}{\rule{0.350pt}{0.400pt}}
\put(584,135.67){\rule{0.723pt}{0.400pt}}
\multiput(585.50,135.17)(-1.500,1.000){2}{\rule{0.361pt}{0.400pt}}
\put(581,136.67){\rule{0.723pt}{0.400pt}}
\multiput(582.50,136.17)(-1.500,1.000){2}{\rule{0.361pt}{0.400pt}}
\put(578,137.67){\rule{0.723pt}{0.400pt}}
\multiput(579.50,137.17)(-1.500,1.000){2}{\rule{0.361pt}{0.400pt}}
\put(575,138.67){\rule{0.723pt}{0.400pt}}
\multiput(576.50,138.17)(-1.500,1.000){2}{\rule{0.361pt}{0.400pt}}
\put(572,139.67){\rule{0.723pt}{0.400pt}}
\multiput(573.50,139.17)(-1.500,1.000){2}{\rule{0.361pt}{0.400pt}}
\put(569,141.17){\rule{0.700pt}{0.400pt}}
\multiput(570.55,140.17)(-1.547,2.000){2}{\rule{0.350pt}{0.400pt}}
\put(566,142.67){\rule{0.723pt}{0.400pt}}
\multiput(567.50,142.17)(-1.500,1.000){2}{\rule{0.361pt}{0.400pt}}
\put(563,143.67){\rule{0.723pt}{0.400pt}}
\multiput(564.50,143.17)(-1.500,1.000){2}{\rule{0.361pt}{0.400pt}}
\put(560,144.67){\rule{0.723pt}{0.400pt}}
\multiput(561.50,144.17)(-1.500,1.000){2}{\rule{0.361pt}{0.400pt}}
\put(557,145.67){\rule{0.723pt}{0.400pt}}
\multiput(558.50,145.17)(-1.500,1.000){2}{\rule{0.361pt}{0.400pt}}
\put(554,146.67){\rule{0.723pt}{0.400pt}}
\multiput(555.50,146.17)(-1.500,1.000){2}{\rule{0.361pt}{0.400pt}}
\put(551,148.17){\rule{0.700pt}{0.400pt}}
\multiput(552.55,147.17)(-1.547,2.000){2}{\rule{0.350pt}{0.400pt}}
\put(548,149.67){\rule{0.723pt}{0.400pt}}
\multiput(549.50,149.17)(-1.500,1.000){2}{\rule{0.361pt}{0.400pt}}
\put(545,150.67){\rule{0.723pt}{0.400pt}}
\multiput(546.50,150.17)(-1.500,1.000){2}{\rule{0.361pt}{0.400pt}}
\put(542,151.67){\rule{0.723pt}{0.400pt}}
\multiput(543.50,151.17)(-1.500,1.000){2}{\rule{0.361pt}{0.400pt}}
\put(539,152.67){\rule{0.723pt}{0.400pt}}
\multiput(540.50,152.17)(-1.500,1.000){2}{\rule{0.361pt}{0.400pt}}
\put(536,153.67){\rule{0.723pt}{0.400pt}}
\multiput(537.50,153.17)(-1.500,1.000){2}{\rule{0.361pt}{0.400pt}}
\put(532,155.17){\rule{0.900pt}{0.400pt}}
\multiput(534.13,154.17)(-2.132,2.000){2}{\rule{0.450pt}{0.400pt}}
\put(529,156.67){\rule{0.723pt}{0.400pt}}
\multiput(530.50,156.17)(-1.500,1.000){2}{\rule{0.361pt}{0.400pt}}
\put(526,157.67){\rule{0.723pt}{0.400pt}}
\multiput(527.50,157.17)(-1.500,1.000){2}{\rule{0.361pt}{0.400pt}}
\put(523,158.67){\rule{0.723pt}{0.400pt}}
\multiput(524.50,158.17)(-1.500,1.000){2}{\rule{0.361pt}{0.400pt}}
\put(520,159.67){\rule{0.723pt}{0.400pt}}
\multiput(521.50,159.17)(-1.500,1.000){2}{\rule{0.361pt}{0.400pt}}
\put(517,160.67){\rule{0.723pt}{0.400pt}}
\multiput(518.50,160.17)(-1.500,1.000){2}{\rule{0.361pt}{0.400pt}}
\put(514,162.17){\rule{0.700pt}{0.400pt}}
\multiput(515.55,161.17)(-1.547,2.000){2}{\rule{0.350pt}{0.400pt}}
\put(511,163.67){\rule{0.723pt}{0.400pt}}
\multiput(512.50,163.17)(-1.500,1.000){2}{\rule{0.361pt}{0.400pt}}
\put(508,164.67){\rule{0.723pt}{0.400pt}}
\multiput(509.50,164.17)(-1.500,1.000){2}{\rule{0.361pt}{0.400pt}}
\put(505,165.67){\rule{0.723pt}{0.400pt}}
\multiput(506.50,165.17)(-1.500,1.000){2}{\rule{0.361pt}{0.400pt}}
\put(502,166.67){\rule{0.723pt}{0.400pt}}
\multiput(503.50,166.17)(-1.500,1.000){2}{\rule{0.361pt}{0.400pt}}
\put(499,167.67){\rule{0.723pt}{0.400pt}}
\multiput(500.50,167.17)(-1.500,1.000){2}{\rule{0.361pt}{0.400pt}}
\put(496,169.17){\rule{0.700pt}{0.400pt}}
\multiput(497.55,168.17)(-1.547,2.000){2}{\rule{0.350pt}{0.400pt}}
\put(493,170.67){\rule{0.723pt}{0.400pt}}
\multiput(494.50,170.17)(-1.500,1.000){2}{\rule{0.361pt}{0.400pt}}
\put(490,171.67){\rule{0.723pt}{0.400pt}}
\multiput(491.50,171.17)(-1.500,1.000){2}{\rule{0.361pt}{0.400pt}}
\put(487,172.67){\rule{0.723pt}{0.400pt}}
\multiput(488.50,172.17)(-1.500,1.000){2}{\rule{0.361pt}{0.400pt}}
\put(484,173.67){\rule{0.723pt}{0.400pt}}
\multiput(485.50,173.17)(-1.500,1.000){2}{\rule{0.361pt}{0.400pt}}
\put(481,174.67){\rule{0.723pt}{0.400pt}}
\multiput(482.50,174.17)(-1.500,1.000){2}{\rule{0.361pt}{0.400pt}}
\put(478,176.17){\rule{0.700pt}{0.400pt}}
\multiput(479.55,175.17)(-1.547,2.000){2}{\rule{0.350pt}{0.400pt}}
\put(474,177.67){\rule{0.964pt}{0.400pt}}
\multiput(476.00,177.17)(-2.000,1.000){2}{\rule{0.482pt}{0.400pt}}
\put(471,178.67){\rule{0.723pt}{0.400pt}}
\multiput(472.50,178.17)(-1.500,1.000){2}{\rule{0.361pt}{0.400pt}}
\put(468,179.67){\rule{0.723pt}{0.400pt}}
\multiput(469.50,179.17)(-1.500,1.000){2}{\rule{0.361pt}{0.400pt}}
\put(465,180.67){\rule{0.723pt}{0.400pt}}
\multiput(466.50,180.17)(-1.500,1.000){2}{\rule{0.361pt}{0.400pt}}
\put(462,181.67){\rule{0.723pt}{0.400pt}}
\multiput(463.50,181.17)(-1.500,1.000){2}{\rule{0.361pt}{0.400pt}}
\put(459,183.17){\rule{0.700pt}{0.400pt}}
\multiput(460.55,182.17)(-1.547,2.000){2}{\rule{0.350pt}{0.400pt}}
\put(456,184.67){\rule{0.723pt}{0.400pt}}
\multiput(457.50,184.17)(-1.500,1.000){2}{\rule{0.361pt}{0.400pt}}
\put(453,185.67){\rule{0.723pt}{0.400pt}}
\multiput(454.50,185.17)(-1.500,1.000){2}{\rule{0.361pt}{0.400pt}}
\put(450,186.67){\rule{0.723pt}{0.400pt}}
\multiput(451.50,186.17)(-1.500,1.000){2}{\rule{0.361pt}{0.400pt}}
\put(447,187.67){\rule{0.723pt}{0.400pt}}
\multiput(448.50,187.17)(-1.500,1.000){2}{\rule{0.361pt}{0.400pt}}
\put(444,188.67){\rule{0.723pt}{0.400pt}}
\multiput(445.50,188.17)(-1.500,1.000){2}{\rule{0.361pt}{0.400pt}}
\put(441,190.17){\rule{0.700pt}{0.400pt}}
\multiput(442.55,189.17)(-1.547,2.000){2}{\rule{0.350pt}{0.400pt}}
\put(438,191.67){\rule{0.723pt}{0.400pt}}
\multiput(439.50,191.17)(-1.500,1.000){2}{\rule{0.361pt}{0.400pt}}
\put(435,192.67){\rule{0.723pt}{0.400pt}}
\multiput(436.50,192.17)(-1.500,1.000){2}{\rule{0.361pt}{0.400pt}}
\put(432,193.67){\rule{0.723pt}{0.400pt}}
\multiput(433.50,193.17)(-1.500,1.000){2}{\rule{0.361pt}{0.400pt}}
\put(429,194.67){\rule{0.723pt}{0.400pt}}
\multiput(430.50,194.17)(-1.500,1.000){2}{\rule{0.361pt}{0.400pt}}
\put(426,195.67){\rule{0.723pt}{0.400pt}}
\multiput(427.50,195.17)(-1.500,1.000){2}{\rule{0.361pt}{0.400pt}}
\put(423,197.17){\rule{0.700pt}{0.400pt}}
\multiput(424.55,196.17)(-1.547,2.000){2}{\rule{0.350pt}{0.400pt}}
\put(420,198.67){\rule{0.723pt}{0.400pt}}
\multiput(421.50,198.17)(-1.500,1.000){2}{\rule{0.361pt}{0.400pt}}
\put(416,199.67){\rule{0.964pt}{0.400pt}}
\multiput(418.00,199.17)(-2.000,1.000){2}{\rule{0.482pt}{0.400pt}}
\put(413,200.67){\rule{0.723pt}{0.400pt}}
\multiput(414.50,200.17)(-1.500,1.000){2}{\rule{0.361pt}{0.400pt}}
\put(410,201.67){\rule{0.723pt}{0.400pt}}
\multiput(411.50,201.17)(-1.500,1.000){2}{\rule{0.361pt}{0.400pt}}
\put(407,202.67){\rule{0.723pt}{0.400pt}}
\multiput(408.50,202.17)(-1.500,1.000){2}{\rule{0.361pt}{0.400pt}}
\put(404,204.17){\rule{0.700pt}{0.400pt}}
\multiput(405.55,203.17)(-1.547,2.000){2}{\rule{0.350pt}{0.400pt}}
\put(401,205.67){\rule{0.723pt}{0.400pt}}
\multiput(402.50,205.17)(-1.500,1.000){2}{\rule{0.361pt}{0.400pt}}
\put(398,206.67){\rule{0.723pt}{0.400pt}}
\multiput(399.50,206.17)(-1.500,1.000){2}{\rule{0.361pt}{0.400pt}}
\put(395,207.67){\rule{0.723pt}{0.400pt}}
\multiput(396.50,207.17)(-1.500,1.000){2}{\rule{0.361pt}{0.400pt}}
\put(392,208.67){\rule{0.723pt}{0.400pt}}
\multiput(393.50,208.17)(-1.500,1.000){2}{\rule{0.361pt}{0.400pt}}
\put(389,209.67){\rule{0.723pt}{0.400pt}}
\multiput(390.50,209.17)(-1.500,1.000){2}{\rule{0.361pt}{0.400pt}}
\put(386,211.17){\rule{0.700pt}{0.400pt}}
\multiput(387.55,210.17)(-1.547,2.000){2}{\rule{0.350pt}{0.400pt}}
\put(383,212.67){\rule{0.723pt}{0.400pt}}
\multiput(384.50,212.17)(-1.500,1.000){2}{\rule{0.361pt}{0.400pt}}
\put(380,213.67){\rule{0.723pt}{0.400pt}}
\multiput(381.50,213.17)(-1.500,1.000){2}{\rule{0.361pt}{0.400pt}}
\put(377,214.67){\rule{0.723pt}{0.400pt}}
\multiput(378.50,214.17)(-1.500,1.000){2}{\rule{0.361pt}{0.400pt}}
\put(374,215.67){\rule{0.723pt}{0.400pt}}
\multiput(375.50,215.17)(-1.500,1.000){2}{\rule{0.361pt}{0.400pt}}
\put(371,216.67){\rule{0.723pt}{0.400pt}}
\multiput(372.50,216.17)(-1.500,1.000){2}{\rule{0.361pt}{0.400pt}}
\put(368,218.17){\rule{0.700pt}{0.400pt}}
\multiput(369.55,217.17)(-1.547,2.000){2}{\rule{0.350pt}{0.400pt}}
\put(365,219.67){\rule{0.723pt}{0.400pt}}
\multiput(366.50,219.17)(-1.500,1.000){2}{\rule{0.361pt}{0.400pt}}
\put(362,220.67){\rule{0.723pt}{0.400pt}}
\multiput(363.50,220.17)(-1.500,1.000){2}{\rule{0.361pt}{0.400pt}}
\put(358,221.67){\rule{0.964pt}{0.400pt}}
\multiput(360.00,221.17)(-2.000,1.000){2}{\rule{0.482pt}{0.400pt}}
\put(355,222.67){\rule{0.723pt}{0.400pt}}
\multiput(356.50,222.17)(-1.500,1.000){2}{\rule{0.361pt}{0.400pt}}
\put(352,223.67){\rule{0.723pt}{0.400pt}}
\multiput(353.50,223.17)(-1.500,1.000){2}{\rule{0.361pt}{0.400pt}}
\put(349,225.17){\rule{0.700pt}{0.400pt}}
\multiput(350.55,224.17)(-1.547,2.000){2}{\rule{0.350pt}{0.400pt}}
\put(346,226.67){\rule{0.723pt}{0.400pt}}
\multiput(347.50,226.17)(-1.500,1.000){2}{\rule{0.361pt}{0.400pt}}
\put(343,227.67){\rule{0.723pt}{0.400pt}}
\multiput(344.50,227.17)(-1.500,1.000){2}{\rule{0.361pt}{0.400pt}}
\put(340,228.67){\rule{0.723pt}{0.400pt}}
\multiput(341.50,228.17)(-1.500,1.000){2}{\rule{0.361pt}{0.400pt}}
\put(337,229.67){\rule{0.723pt}{0.400pt}}
\multiput(338.50,229.17)(-1.500,1.000){2}{\rule{0.361pt}{0.400pt}}
\put(334,230.67){\rule{0.723pt}{0.400pt}}
\multiput(335.50,230.17)(-1.500,1.000){2}{\rule{0.361pt}{0.400pt}}
\put(331,232.17){\rule{0.700pt}{0.400pt}}
\multiput(332.55,231.17)(-1.547,2.000){2}{\rule{0.350pt}{0.400pt}}
\put(328,233.67){\rule{0.723pt}{0.400pt}}
\multiput(329.50,233.17)(-1.500,1.000){2}{\rule{0.361pt}{0.400pt}}
\put(325,234.67){\rule{0.723pt}{0.400pt}}
\multiput(326.50,234.17)(-1.500,1.000){2}{\rule{0.361pt}{0.400pt}}
\put(322,235.67){\rule{0.723pt}{0.400pt}}
\multiput(323.50,235.17)(-1.500,1.000){2}{\rule{0.361pt}{0.400pt}}
\put(319,236.67){\rule{0.723pt}{0.400pt}}
\multiput(320.50,236.17)(-1.500,1.000){2}{\rule{0.361pt}{0.400pt}}
\put(316,237.67){\rule{0.723pt}{0.400pt}}
\multiput(317.50,237.17)(-1.500,1.000){2}{\rule{0.361pt}{0.400pt}}
\put(313,239.17){\rule{0.700pt}{0.400pt}}
\multiput(314.55,238.17)(-1.547,2.000){2}{\rule{0.350pt}{0.400pt}}
\put(310,240.67){\rule{0.723pt}{0.400pt}}
\multiput(311.50,240.17)(-1.500,1.000){2}{\rule{0.361pt}{0.400pt}}
\put(307,241.67){\rule{0.723pt}{0.400pt}}
\multiput(308.50,241.17)(-1.500,1.000){2}{\rule{0.361pt}{0.400pt}}
\put(304,242.67){\rule{0.723pt}{0.400pt}}
\multiput(305.50,242.17)(-1.500,1.000){2}{\rule{0.361pt}{0.400pt}}
\put(300,243.67){\rule{0.964pt}{0.400pt}}
\multiput(302.00,243.17)(-2.000,1.000){2}{\rule{0.482pt}{0.400pt}}
\put(297,244.67){\rule{0.723pt}{0.400pt}}
\multiput(298.50,244.17)(-1.500,1.000){2}{\rule{0.361pt}{0.400pt}}
\put(294,246.17){\rule{0.700pt}{0.400pt}}
\multiput(295.55,245.17)(-1.547,2.000){2}{\rule{0.350pt}{0.400pt}}
\put(291,247.67){\rule{0.723pt}{0.400pt}}
\multiput(292.50,247.17)(-1.500,1.000){2}{\rule{0.361pt}{0.400pt}}
\put(288,248.67){\rule{0.723pt}{0.400pt}}
\multiput(289.50,248.17)(-1.500,1.000){2}{\rule{0.361pt}{0.400pt}}
\put(285,249.67){\rule{0.723pt}{0.400pt}}
\multiput(286.50,249.17)(-1.500,1.000){2}{\rule{0.361pt}{0.400pt}}
\put(282,250.67){\rule{0.723pt}{0.400pt}}
\multiput(283.50,250.17)(-1.500,1.000){2}{\rule{0.361pt}{0.400pt}}
\put(279,251.67){\rule{0.723pt}{0.400pt}}
\multiput(280.50,251.17)(-1.500,1.000){2}{\rule{0.361pt}{0.400pt}}
\put(276,253.17){\rule{0.700pt}{0.400pt}}
\multiput(277.55,252.17)(-1.547,2.000){2}{\rule{0.350pt}{0.400pt}}
\put(273,254.67){\rule{0.723pt}{0.400pt}}
\multiput(274.50,254.17)(-1.500,1.000){2}{\rule{0.361pt}{0.400pt}}
\put(270,255.67){\rule{0.723pt}{0.400pt}}
\multiput(271.50,255.17)(-1.500,1.000){2}{\rule{0.361pt}{0.400pt}}
\put(267,256.67){\rule{0.723pt}{0.400pt}}
\multiput(268.50,256.17)(-1.500,1.000){2}{\rule{0.361pt}{0.400pt}}
\put(264,257.67){\rule{0.723pt}{0.400pt}}
\multiput(265.50,257.17)(-1.500,1.000){2}{\rule{0.361pt}{0.400pt}}
\put(261,258.67){\rule{0.723pt}{0.400pt}}
\multiput(262.50,258.17)(-1.500,1.000){2}{\rule{0.361pt}{0.400pt}}
\put(258,260.17){\rule{0.700pt}{0.400pt}}
\multiput(259.55,259.17)(-1.547,2.000){2}{\rule{0.350pt}{0.400pt}}
\put(255,261.67){\rule{0.723pt}{0.400pt}}
\multiput(256.50,261.17)(-1.500,1.000){2}{\rule{0.361pt}{0.400pt}}
\put(252,262.67){\rule{0.723pt}{0.400pt}}
\multiput(253.50,262.17)(-1.500,1.000){2}{\rule{0.361pt}{0.400pt}}
\put(249,263.67){\rule{0.723pt}{0.400pt}}
\multiput(250.50,263.17)(-1.500,1.000){2}{\rule{0.361pt}{0.400pt}}
\put(246,264.67){\rule{0.723pt}{0.400pt}}
\multiput(247.50,264.17)(-1.500,1.000){2}{\rule{0.361pt}{0.400pt}}
\put(242,265.67){\rule{0.964pt}{0.400pt}}
\multiput(244.00,265.17)(-2.000,1.000){2}{\rule{0.482pt}{0.400pt}}
\put(239,267.17){\rule{0.700pt}{0.400pt}}
\multiput(240.55,266.17)(-1.547,2.000){2}{\rule{0.350pt}{0.400pt}}
\put(236,268.67){\rule{0.723pt}{0.400pt}}
\multiput(237.50,268.17)(-1.500,1.000){2}{\rule{0.361pt}{0.400pt}}
\put(233,269.67){\rule{0.723pt}{0.400pt}}
\multiput(234.50,269.17)(-1.500,1.000){2}{\rule{0.361pt}{0.400pt}}
\put(230,270.67){\rule{0.723pt}{0.400pt}}
\multiput(231.50,270.17)(-1.500,1.000){2}{\rule{0.361pt}{0.400pt}}
\put(227,271.67){\rule{0.723pt}{0.400pt}}
\multiput(228.50,271.17)(-1.500,1.000){2}{\rule{0.361pt}{0.400pt}}
\put(224,272.67){\rule{0.723pt}{0.400pt}}
\multiput(225.50,272.17)(-1.500,1.000){2}{\rule{0.361pt}{0.400pt}}
\put(221,274.17){\rule{0.700pt}{0.400pt}}
\multiput(222.55,273.17)(-1.547,2.000){2}{\rule{0.350pt}{0.400pt}}
\put(218,275.67){\rule{0.723pt}{0.400pt}}
\multiput(219.50,275.17)(-1.500,1.000){2}{\rule{0.361pt}{0.400pt}}
\put(215,276.67){\rule{0.723pt}{0.400pt}}
\multiput(216.50,276.17)(-1.500,1.000){2}{\rule{0.361pt}{0.400pt}}
\put(212,277.67){\rule{0.723pt}{0.400pt}}
\multiput(213.50,277.17)(-1.500,1.000){2}{\rule{0.361pt}{0.400pt}}
\put(209,278.67){\rule{0.723pt}{0.400pt}}
\multiput(210.50,278.17)(-1.500,1.000){2}{\rule{0.361pt}{0.400pt}}
\put(206,279.67){\rule{0.723pt}{0.400pt}}
\multiput(207.50,279.17)(-1.500,1.000){2}{\rule{0.361pt}{0.400pt}}
\put(203,281.17){\rule{0.700pt}{0.400pt}}
\multiput(204.55,280.17)(-1.547,2.000){2}{\rule{0.350pt}{0.400pt}}
\put(200,282.67){\rule{0.723pt}{0.400pt}}
\multiput(201.50,282.17)(-1.500,1.000){2}{\rule{0.361pt}{0.400pt}}
\put(197,283.67){\rule{0.723pt}{0.400pt}}
\multiput(198.50,283.17)(-1.500,1.000){2}{\rule{0.361pt}{0.400pt}}
\put(194,284.67){\rule{0.723pt}{0.400pt}}
\multiput(195.50,284.17)(-1.500,1.000){2}{\rule{0.361pt}{0.400pt}}
\put(191,285.67){\rule{0.723pt}{0.400pt}}
\multiput(192.50,285.17)(-1.500,1.000){2}{\rule{0.361pt}{0.400pt}}
\put(188,286.67){\rule{0.723pt}{0.400pt}}
\multiput(189.50,286.17)(-1.500,1.000){2}{\rule{0.361pt}{0.400pt}}
\put(185,288.17){\rule{0.700pt}{0.400pt}}
\multiput(186.55,287.17)(-1.547,2.000){2}{\rule{0.350pt}{0.400pt}}
\put(181,289.67){\rule{0.964pt}{0.400pt}}
\multiput(183.00,289.17)(-2.000,1.000){2}{\rule{0.482pt}{0.400pt}}
\put(178,290.67){\rule{0.723pt}{0.400pt}}
\multiput(179.50,290.17)(-1.500,1.000){2}{\rule{0.361pt}{0.400pt}}
\put(175,291.67){\rule{0.723pt}{0.400pt}}
\multiput(176.50,291.17)(-1.500,1.000){2}{\rule{0.361pt}{0.400pt}}
\put(172,292.67){\rule{0.723pt}{0.400pt}}
\multiput(173.50,292.17)(-1.500,1.000){2}{\rule{0.361pt}{0.400pt}}
\put(169,293.67){\rule{0.723pt}{0.400pt}}
\multiput(170.50,293.17)(-1.500,1.000){2}{\rule{0.361pt}{0.400pt}}
\put(166,295.17){\rule{0.700pt}{0.400pt}}
\multiput(167.55,294.17)(-1.547,2.000){2}{\rule{0.350pt}{0.400pt}}
\put(163,296.67){\rule{0.723pt}{0.400pt}}
\multiput(164.50,296.17)(-1.500,1.000){2}{\rule{0.361pt}{0.400pt}}
\put(160,297.67){\rule{0.723pt}{0.400pt}}
\multiput(161.50,297.17)(-1.500,1.000){2}{\rule{0.361pt}{0.400pt}}
\put(157,298.67){\rule{0.723pt}{0.400pt}}
\multiput(158.50,298.17)(-1.500,1.000){2}{\rule{0.361pt}{0.400pt}}
\put(154,299.67){\rule{0.723pt}{0.400pt}}
\multiput(155.50,299.17)(-1.500,1.000){2}{\rule{0.361pt}{0.400pt}}
\put(154,300.67){\rule{0.723pt}{0.400pt}}
\multiput(154.00,300.17)(1.500,1.000){2}{\rule{0.361pt}{0.400pt}}
\put(157,302.17){\rule{0.700pt}{0.400pt}}
\multiput(157.00,301.17)(1.547,2.000){2}{\rule{0.350pt}{0.400pt}}
\put(160,303.67){\rule{0.723pt}{0.400pt}}
\multiput(160.00,303.17)(1.500,1.000){2}{\rule{0.361pt}{0.400pt}}
\put(163,304.67){\rule{0.723pt}{0.400pt}}
\multiput(163.00,304.17)(1.500,1.000){2}{\rule{0.361pt}{0.400pt}}
\put(166,305.67){\rule{0.723pt}{0.400pt}}
\multiput(166.00,305.17)(1.500,1.000){2}{\rule{0.361pt}{0.400pt}}
\put(169,306.67){\rule{0.723pt}{0.400pt}}
\multiput(169.00,306.17)(1.500,1.000){2}{\rule{0.361pt}{0.400pt}}
\put(172,307.67){\rule{0.723pt}{0.400pt}}
\multiput(172.00,307.17)(1.500,1.000){2}{\rule{0.361pt}{0.400pt}}
\put(175,309.17){\rule{0.700pt}{0.400pt}}
\multiput(175.00,308.17)(1.547,2.000){2}{\rule{0.350pt}{0.400pt}}
\put(178,310.67){\rule{0.723pt}{0.400pt}}
\multiput(178.00,310.17)(1.500,1.000){2}{\rule{0.361pt}{0.400pt}}
\put(181,311.67){\rule{0.964pt}{0.400pt}}
\multiput(181.00,311.17)(2.000,1.000){2}{\rule{0.482pt}{0.400pt}}
\put(185,312.67){\rule{0.723pt}{0.400pt}}
\multiput(185.00,312.17)(1.500,1.000){2}{\rule{0.361pt}{0.400pt}}
\put(188,313.67){\rule{0.723pt}{0.400pt}}
\multiput(188.00,313.17)(1.500,1.000){2}{\rule{0.361pt}{0.400pt}}
\put(191,314.67){\rule{0.723pt}{0.400pt}}
\multiput(191.00,314.17)(1.500,1.000){2}{\rule{0.361pt}{0.400pt}}
\put(194,315.67){\rule{0.723pt}{0.400pt}}
\multiput(194.00,315.17)(1.500,1.000){2}{\rule{0.361pt}{0.400pt}}
\put(197,317.17){\rule{0.700pt}{0.400pt}}
\multiput(197.00,316.17)(1.547,2.000){2}{\rule{0.350pt}{0.400pt}}
\put(200,318.67){\rule{0.723pt}{0.400pt}}
\multiput(200.00,318.17)(1.500,1.000){2}{\rule{0.361pt}{0.400pt}}
\put(203,319.67){\rule{0.723pt}{0.400pt}}
\multiput(203.00,319.17)(1.500,1.000){2}{\rule{0.361pt}{0.400pt}}
\put(206,320.67){\rule{0.723pt}{0.400pt}}
\multiput(206.00,320.17)(1.500,1.000){2}{\rule{0.361pt}{0.400pt}}
\put(209,321.67){\rule{0.723pt}{0.400pt}}
\multiput(209.00,321.17)(1.500,1.000){2}{\rule{0.361pt}{0.400pt}}
\put(212,322.67){\rule{0.723pt}{0.400pt}}
\multiput(212.00,322.17)(1.500,1.000){2}{\rule{0.361pt}{0.400pt}}
\put(215,324.17){\rule{0.700pt}{0.400pt}}
\multiput(215.00,323.17)(1.547,2.000){2}{\rule{0.350pt}{0.400pt}}
\put(218,325.67){\rule{0.723pt}{0.400pt}}
\multiput(218.00,325.17)(1.500,1.000){2}{\rule{0.361pt}{0.400pt}}
\put(221,326.67){\rule{0.723pt}{0.400pt}}
\multiput(221.00,326.17)(1.500,1.000){2}{\rule{0.361pt}{0.400pt}}
\put(224,327.67){\rule{0.723pt}{0.400pt}}
\multiput(224.00,327.17)(1.500,1.000){2}{\rule{0.361pt}{0.400pt}}
\put(227,328.67){\rule{0.723pt}{0.400pt}}
\multiput(227.00,328.17)(1.500,1.000){2}{\rule{0.361pt}{0.400pt}}
\put(230,329.67){\rule{0.723pt}{0.400pt}}
\multiput(230.00,329.17)(1.500,1.000){2}{\rule{0.361pt}{0.400pt}}
\put(233,331.17){\rule{0.700pt}{0.400pt}}
\multiput(233.00,330.17)(1.547,2.000){2}{\rule{0.350pt}{0.400pt}}
\put(236,332.67){\rule{0.723pt}{0.400pt}}
\multiput(236.00,332.17)(1.500,1.000){2}{\rule{0.361pt}{0.400pt}}
\put(239,333.67){\rule{0.723pt}{0.400pt}}
\multiput(239.00,333.17)(1.500,1.000){2}{\rule{0.361pt}{0.400pt}}
\put(242,334.67){\rule{0.964pt}{0.400pt}}
\multiput(242.00,334.17)(2.000,1.000){2}{\rule{0.482pt}{0.400pt}}
\put(246,335.67){\rule{0.723pt}{0.400pt}}
\multiput(246.00,335.17)(1.500,1.000){2}{\rule{0.361pt}{0.400pt}}
\put(249,336.67){\rule{0.723pt}{0.400pt}}
\multiput(249.00,336.17)(1.500,1.000){2}{\rule{0.361pt}{0.400pt}}
\put(252,338.17){\rule{0.700pt}{0.400pt}}
\multiput(252.00,337.17)(1.547,2.000){2}{\rule{0.350pt}{0.400pt}}
\put(255,339.67){\rule{0.723pt}{0.400pt}}
\multiput(255.00,339.17)(1.500,1.000){2}{\rule{0.361pt}{0.400pt}}
\put(258,340.67){\rule{0.723pt}{0.400pt}}
\multiput(258.00,340.17)(1.500,1.000){2}{\rule{0.361pt}{0.400pt}}
\put(261,341.67){\rule{0.723pt}{0.400pt}}
\multiput(261.00,341.17)(1.500,1.000){2}{\rule{0.361pt}{0.400pt}}
\put(264,342.67){\rule{0.723pt}{0.400pt}}
\multiput(264.00,342.17)(1.500,1.000){2}{\rule{0.361pt}{0.400pt}}
\put(267,343.67){\rule{0.723pt}{0.400pt}}
\multiput(267.00,343.17)(1.500,1.000){2}{\rule{0.361pt}{0.400pt}}
\put(270,345.17){\rule{0.700pt}{0.400pt}}
\multiput(270.00,344.17)(1.547,2.000){2}{\rule{0.350pt}{0.400pt}}
\put(273,346.67){\rule{0.723pt}{0.400pt}}
\multiput(273.00,346.17)(1.500,1.000){2}{\rule{0.361pt}{0.400pt}}
\put(276,347.67){\rule{0.723pt}{0.400pt}}
\multiput(276.00,347.17)(1.500,1.000){2}{\rule{0.361pt}{0.400pt}}
\put(279,348.67){\rule{0.723pt}{0.400pt}}
\multiput(279.00,348.17)(1.500,1.000){2}{\rule{0.361pt}{0.400pt}}
\put(282,349.67){\rule{0.723pt}{0.400pt}}
\multiput(282.00,349.17)(1.500,1.000){2}{\rule{0.361pt}{0.400pt}}
\put(285,350.67){\rule{0.723pt}{0.400pt}}
\multiput(285.00,350.17)(1.500,1.000){2}{\rule{0.361pt}{0.400pt}}
\put(288,352.17){\rule{0.700pt}{0.400pt}}
\multiput(288.00,351.17)(1.547,2.000){2}{\rule{0.350pt}{0.400pt}}
\put(291,353.67){\rule{0.723pt}{0.400pt}}
\multiput(291.00,353.17)(1.500,1.000){2}{\rule{0.361pt}{0.400pt}}
\put(294,354.67){\rule{0.723pt}{0.400pt}}
\multiput(294.00,354.17)(1.500,1.000){2}{\rule{0.361pt}{0.400pt}}
\put(297,355.67){\rule{0.723pt}{0.400pt}}
\multiput(297.00,355.17)(1.500,1.000){2}{\rule{0.361pt}{0.400pt}}
\put(300,356.67){\rule{0.964pt}{0.400pt}}
\multiput(300.00,356.17)(2.000,1.000){2}{\rule{0.482pt}{0.400pt}}
\put(304,357.67){\rule{0.723pt}{0.400pt}}
\multiput(304.00,357.17)(1.500,1.000){2}{\rule{0.361pt}{0.400pt}}
\put(307,359.17){\rule{0.700pt}{0.400pt}}
\multiput(307.00,358.17)(1.547,2.000){2}{\rule{0.350pt}{0.400pt}}
\put(310,360.67){\rule{0.723pt}{0.400pt}}
\multiput(310.00,360.17)(1.500,1.000){2}{\rule{0.361pt}{0.400pt}}
\put(313,361.67){\rule{0.723pt}{0.400pt}}
\multiput(313.00,361.17)(1.500,1.000){2}{\rule{0.361pt}{0.400pt}}
\put(316,362.67){\rule{0.723pt}{0.400pt}}
\multiput(316.00,362.17)(1.500,1.000){2}{\rule{0.361pt}{0.400pt}}
\put(319,363.67){\rule{0.723pt}{0.400pt}}
\multiput(319.00,363.17)(1.500,1.000){2}{\rule{0.361pt}{0.400pt}}
\put(322,364.67){\rule{0.723pt}{0.400pt}}
\multiput(322.00,364.17)(1.500,1.000){2}{\rule{0.361pt}{0.400pt}}
\put(325,366.17){\rule{0.700pt}{0.400pt}}
\multiput(325.00,365.17)(1.547,2.000){2}{\rule{0.350pt}{0.400pt}}
\put(328,367.67){\rule{0.723pt}{0.400pt}}
\multiput(328.00,367.17)(1.500,1.000){2}{\rule{0.361pt}{0.400pt}}
\put(331,368.67){\rule{0.723pt}{0.400pt}}
\multiput(331.00,368.17)(1.500,1.000){2}{\rule{0.361pt}{0.400pt}}
\put(334,369.67){\rule{0.723pt}{0.400pt}}
\multiput(334.00,369.17)(1.500,1.000){2}{\rule{0.361pt}{0.400pt}}
\put(337,370.67){\rule{0.723pt}{0.400pt}}
\multiput(337.00,370.17)(1.500,1.000){2}{\rule{0.361pt}{0.400pt}}
\put(340,371.67){\rule{0.723pt}{0.400pt}}
\multiput(340.00,371.17)(1.500,1.000){2}{\rule{0.361pt}{0.400pt}}
\put(343,373.17){\rule{0.700pt}{0.400pt}}
\multiput(343.00,372.17)(1.547,2.000){2}{\rule{0.350pt}{0.400pt}}
\put(346,374.67){\rule{0.723pt}{0.400pt}}
\multiput(346.00,374.17)(1.500,1.000){2}{\rule{0.361pt}{0.400pt}}
\put(349,375.67){\rule{0.723pt}{0.400pt}}
\multiput(349.00,375.17)(1.500,1.000){2}{\rule{0.361pt}{0.400pt}}
\put(352,376.67){\rule{0.723pt}{0.400pt}}
\multiput(352.00,376.17)(1.500,1.000){2}{\rule{0.361pt}{0.400pt}}
\put(355,377.67){\rule{0.723pt}{0.400pt}}
\multiput(355.00,377.17)(1.500,1.000){2}{\rule{0.361pt}{0.400pt}}
\put(358,378.67){\rule{0.964pt}{0.400pt}}
\multiput(358.00,378.17)(2.000,1.000){2}{\rule{0.482pt}{0.400pt}}
\put(362,380.17){\rule{0.700pt}{0.400pt}}
\multiput(362.00,379.17)(1.547,2.000){2}{\rule{0.350pt}{0.400pt}}
\put(365,381.67){\rule{0.723pt}{0.400pt}}
\multiput(365.00,381.17)(1.500,1.000){2}{\rule{0.361pt}{0.400pt}}
\put(368,382.67){\rule{0.723pt}{0.400pt}}
\multiput(368.00,382.17)(1.500,1.000){2}{\rule{0.361pt}{0.400pt}}
\put(371,383.67){\rule{0.723pt}{0.400pt}}
\multiput(371.00,383.17)(1.500,1.000){2}{\rule{0.361pt}{0.400pt}}
\put(374,384.67){\rule{0.723pt}{0.400pt}}
\multiput(374.00,384.17)(1.500,1.000){2}{\rule{0.361pt}{0.400pt}}
\put(377,385.67){\rule{0.723pt}{0.400pt}}
\multiput(377.00,385.17)(1.500,1.000){2}{\rule{0.361pt}{0.400pt}}
\put(380,387.17){\rule{0.700pt}{0.400pt}}
\multiput(380.00,386.17)(1.547,2.000){2}{\rule{0.350pt}{0.400pt}}
\put(383,388.67){\rule{0.723pt}{0.400pt}}
\multiput(383.00,388.17)(1.500,1.000){2}{\rule{0.361pt}{0.400pt}}
\put(386,389.67){\rule{0.723pt}{0.400pt}}
\multiput(386.00,389.17)(1.500,1.000){2}{\rule{0.361pt}{0.400pt}}
\put(389,390.67){\rule{0.723pt}{0.400pt}}
\multiput(389.00,390.17)(1.500,1.000){2}{\rule{0.361pt}{0.400pt}}
\put(392,391.67){\rule{0.723pt}{0.400pt}}
\multiput(392.00,391.17)(1.500,1.000){2}{\rule{0.361pt}{0.400pt}}
\put(395,392.67){\rule{0.723pt}{0.400pt}}
\multiput(395.00,392.17)(1.500,1.000){2}{\rule{0.361pt}{0.400pt}}
\put(398,394.17){\rule{0.700pt}{0.400pt}}
\multiput(398.00,393.17)(1.547,2.000){2}{\rule{0.350pt}{0.400pt}}
\put(401,395.67){\rule{0.723pt}{0.400pt}}
\multiput(401.00,395.17)(1.500,1.000){2}{\rule{0.361pt}{0.400pt}}
\put(404,396.67){\rule{0.723pt}{0.400pt}}
\multiput(404.00,396.17)(1.500,1.000){2}{\rule{0.361pt}{0.400pt}}
\put(407,397.67){\rule{0.723pt}{0.400pt}}
\multiput(407.00,397.17)(1.500,1.000){2}{\rule{0.361pt}{0.400pt}}
\put(410,398.67){\rule{0.723pt}{0.400pt}}
\multiput(410.00,398.17)(1.500,1.000){2}{\rule{0.361pt}{0.400pt}}
\put(413,399.67){\rule{0.723pt}{0.400pt}}
\multiput(413.00,399.17)(1.500,1.000){2}{\rule{0.361pt}{0.400pt}}
\put(416,401.17){\rule{0.900pt}{0.400pt}}
\multiput(416.00,400.17)(2.132,2.000){2}{\rule{0.450pt}{0.400pt}}
\put(420,402.67){\rule{0.723pt}{0.400pt}}
\multiput(420.00,402.17)(1.500,1.000){2}{\rule{0.361pt}{0.400pt}}
\put(423,403.67){\rule{0.723pt}{0.400pt}}
\multiput(423.00,403.17)(1.500,1.000){2}{\rule{0.361pt}{0.400pt}}
\put(426,404.67){\rule{0.723pt}{0.400pt}}
\multiput(426.00,404.17)(1.500,1.000){2}{\rule{0.361pt}{0.400pt}}
\put(429,405.67){\rule{0.723pt}{0.400pt}}
\multiput(429.00,405.17)(1.500,1.000){2}{\rule{0.361pt}{0.400pt}}
\put(432,406.67){\rule{0.723pt}{0.400pt}}
\multiput(432.00,406.17)(1.500,1.000){2}{\rule{0.361pt}{0.400pt}}
\put(435,408.17){\rule{0.700pt}{0.400pt}}
\multiput(435.00,407.17)(1.547,2.000){2}{\rule{0.350pt}{0.400pt}}
\put(438,409.67){\rule{0.723pt}{0.400pt}}
\multiput(438.00,409.17)(1.500,1.000){2}{\rule{0.361pt}{0.400pt}}
\put(441,410.67){\rule{0.723pt}{0.400pt}}
\multiput(441.00,410.17)(1.500,1.000){2}{\rule{0.361pt}{0.400pt}}
\put(444,411.67){\rule{0.723pt}{0.400pt}}
\multiput(444.00,411.17)(1.500,1.000){2}{\rule{0.361pt}{0.400pt}}
\put(447,412.67){\rule{0.723pt}{0.400pt}}
\multiput(447.00,412.17)(1.500,1.000){2}{\rule{0.361pt}{0.400pt}}
\put(450,413.67){\rule{0.723pt}{0.400pt}}
\multiput(450.00,413.17)(1.500,1.000){2}{\rule{0.361pt}{0.400pt}}
\put(453,415.17){\rule{0.700pt}{0.400pt}}
\multiput(453.00,414.17)(1.547,2.000){2}{\rule{0.350pt}{0.400pt}}
\put(456,416.67){\rule{0.723pt}{0.400pt}}
\multiput(456.00,416.17)(1.500,1.000){2}{\rule{0.361pt}{0.400pt}}
\put(459,417.67){\rule{0.723pt}{0.400pt}}
\multiput(459.00,417.17)(1.500,1.000){2}{\rule{0.361pt}{0.400pt}}
\put(462,418.67){\rule{0.723pt}{0.400pt}}
\multiput(462.00,418.17)(1.500,1.000){2}{\rule{0.361pt}{0.400pt}}
\put(465,419.67){\rule{0.723pt}{0.400pt}}
\multiput(465.00,419.17)(1.500,1.000){2}{\rule{0.361pt}{0.400pt}}
\put(468,420.67){\rule{0.723pt}{0.400pt}}
\multiput(468.00,420.17)(1.500,1.000){2}{\rule{0.361pt}{0.400pt}}
\put(471,422.17){\rule{0.700pt}{0.400pt}}
\multiput(471.00,421.17)(1.547,2.000){2}{\rule{0.350pt}{0.400pt}}
\put(474,423.67){\rule{0.964pt}{0.400pt}}
\multiput(474.00,423.17)(2.000,1.000){2}{\rule{0.482pt}{0.400pt}}
\put(478,424.67){\rule{0.723pt}{0.400pt}}
\multiput(478.00,424.17)(1.500,1.000){2}{\rule{0.361pt}{0.400pt}}
\put(481,425.67){\rule{0.723pt}{0.400pt}}
\multiput(481.00,425.17)(1.500,1.000){2}{\rule{0.361pt}{0.400pt}}
\put(484,426.67){\rule{0.723pt}{0.400pt}}
\multiput(484.00,426.17)(1.500,1.000){2}{\rule{0.361pt}{0.400pt}}
\put(487,427.67){\rule{0.723pt}{0.400pt}}
\multiput(487.00,427.17)(1.500,1.000){2}{\rule{0.361pt}{0.400pt}}
\put(490,429.17){\rule{0.700pt}{0.400pt}}
\multiput(490.00,428.17)(1.547,2.000){2}{\rule{0.350pt}{0.400pt}}
\put(493,430.67){\rule{0.723pt}{0.400pt}}
\multiput(493.00,430.17)(1.500,1.000){2}{\rule{0.361pt}{0.400pt}}
\put(496,431.67){\rule{0.723pt}{0.400pt}}
\multiput(496.00,431.17)(1.500,1.000){2}{\rule{0.361pt}{0.400pt}}
\put(499,432.67){\rule{0.723pt}{0.400pt}}
\multiput(499.00,432.17)(1.500,1.000){2}{\rule{0.361pt}{0.400pt}}
\put(502,433.67){\rule{0.723pt}{0.400pt}}
\multiput(502.00,433.17)(1.500,1.000){2}{\rule{0.361pt}{0.400pt}}
\put(505,434.67){\rule{0.723pt}{0.400pt}}
\multiput(505.00,434.17)(1.500,1.000){2}{\rule{0.361pt}{0.400pt}}
\put(508,436.17){\rule{0.700pt}{0.400pt}}
\multiput(508.00,435.17)(1.547,2.000){2}{\rule{0.350pt}{0.400pt}}
\put(511,437.67){\rule{0.723pt}{0.400pt}}
\multiput(511.00,437.17)(1.500,1.000){2}{\rule{0.361pt}{0.400pt}}
\put(514,438.67){\rule{0.723pt}{0.400pt}}
\multiput(514.00,438.17)(1.500,1.000){2}{\rule{0.361pt}{0.400pt}}
\put(517,439.67){\rule{0.723pt}{0.400pt}}
\multiput(517.00,439.17)(1.500,1.000){2}{\rule{0.361pt}{0.400pt}}
\put(520,440.67){\rule{0.723pt}{0.400pt}}
\multiput(520.00,440.17)(1.500,1.000){2}{\rule{0.361pt}{0.400pt}}
\put(523,441.67){\rule{0.723pt}{0.400pt}}
\multiput(523.00,441.17)(1.500,1.000){2}{\rule{0.361pt}{0.400pt}}
\put(526,443.17){\rule{0.700pt}{0.400pt}}
\multiput(526.00,442.17)(1.547,2.000){2}{\rule{0.350pt}{0.400pt}}
\put(529,444.67){\rule{0.723pt}{0.400pt}}
\multiput(529.00,444.17)(1.500,1.000){2}{\rule{0.361pt}{0.400pt}}
\put(532,445.67){\rule{0.964pt}{0.400pt}}
\multiput(532.00,445.17)(2.000,1.000){2}{\rule{0.482pt}{0.400pt}}
\put(536,446.67){\rule{0.723pt}{0.400pt}}
\multiput(536.00,446.17)(1.500,1.000){2}{\rule{0.361pt}{0.400pt}}
\put(539,447.67){\rule{0.723pt}{0.400pt}}
\multiput(539.00,447.17)(1.500,1.000){2}{\rule{0.361pt}{0.400pt}}
\put(542,448.67){\rule{0.723pt}{0.400pt}}
\multiput(542.00,448.17)(1.500,1.000){2}{\rule{0.361pt}{0.400pt}}
\put(545,450.17){\rule{0.700pt}{0.400pt}}
\multiput(545.00,449.17)(1.547,2.000){2}{\rule{0.350pt}{0.400pt}}
\put(548,451.67){\rule{0.723pt}{0.400pt}}
\multiput(548.00,451.17)(1.500,1.000){2}{\rule{0.361pt}{0.400pt}}
\put(551,452.67){\rule{0.723pt}{0.400pt}}
\multiput(551.00,452.17)(1.500,1.000){2}{\rule{0.361pt}{0.400pt}}
\put(554,453.67){\rule{0.723pt}{0.400pt}}
\multiput(554.00,453.17)(1.500,1.000){2}{\rule{0.361pt}{0.400pt}}
\put(557,454.67){\rule{0.723pt}{0.400pt}}
\multiput(557.00,454.17)(1.500,1.000){2}{\rule{0.361pt}{0.400pt}}
\put(560,455.67){\rule{0.723pt}{0.400pt}}
\multiput(560.00,455.17)(1.500,1.000){2}{\rule{0.361pt}{0.400pt}}
\put(563,457.17){\rule{0.700pt}{0.400pt}}
\multiput(563.00,456.17)(1.547,2.000){2}{\rule{0.350pt}{0.400pt}}
\put(566,458.67){\rule{0.723pt}{0.400pt}}
\multiput(566.00,458.17)(1.500,1.000){2}{\rule{0.361pt}{0.400pt}}
\put(569,459.67){\rule{0.723pt}{0.400pt}}
\multiput(569.00,459.17)(1.500,1.000){2}{\rule{0.361pt}{0.400pt}}
\put(572,460.67){\rule{0.723pt}{0.400pt}}
\multiput(572.00,460.17)(1.500,1.000){2}{\rule{0.361pt}{0.400pt}}
\put(575,461.67){\rule{0.723pt}{0.400pt}}
\multiput(575.00,461.17)(1.500,1.000){2}{\rule{0.361pt}{0.400pt}}
\put(578,462.67){\rule{0.723pt}{0.400pt}}
\multiput(578.00,462.17)(1.500,1.000){2}{\rule{0.361pt}{0.400pt}}
\put(581,464.17){\rule{0.700pt}{0.400pt}}
\multiput(581.00,463.17)(1.547,2.000){2}{\rule{0.350pt}{0.400pt}}
\put(584,465.67){\rule{0.723pt}{0.400pt}}
\multiput(584.00,465.17)(1.500,1.000){2}{\rule{0.361pt}{0.400pt}}
\put(587,466.67){\rule{0.723pt}{0.400pt}}
\multiput(587.00,466.17)(1.500,1.000){2}{\rule{0.361pt}{0.400pt}}
\put(590,467.67){\rule{0.723pt}{0.400pt}}
\multiput(590.00,467.17)(1.500,1.000){2}{\rule{0.361pt}{0.400pt}}
\put(593,468.67){\rule{0.964pt}{0.400pt}}
\multiput(593.00,468.17)(2.000,1.000){2}{\rule{0.482pt}{0.400pt}}
\put(597,469.67){\rule{0.723pt}{0.400pt}}
\multiput(597.00,469.17)(1.500,1.000){2}{\rule{0.361pt}{0.400pt}}
\put(600,471.17){\rule{0.700pt}{0.400pt}}
\multiput(600.00,470.17)(1.547,2.000){2}{\rule{0.350pt}{0.400pt}}
\put(603,472.67){\rule{0.723pt}{0.400pt}}
\multiput(603.00,472.17)(1.500,1.000){2}{\rule{0.361pt}{0.400pt}}
\put(606,473.67){\rule{0.723pt}{0.400pt}}
\multiput(606.00,473.17)(1.500,1.000){2}{\rule{0.361pt}{0.400pt}}
\put(609,474.67){\rule{0.723pt}{0.400pt}}
\multiput(609.00,474.17)(1.500,1.000){2}{\rule{0.361pt}{0.400pt}}
\put(612,475.67){\rule{0.723pt}{0.400pt}}
\multiput(612.00,475.17)(1.500,1.000){2}{\rule{0.361pt}{0.400pt}}
\put(615,476.67){\rule{0.723pt}{0.400pt}}
\multiput(615.00,476.17)(1.500,1.000){2}{\rule{0.361pt}{0.400pt}}
\put(618,478.17){\rule{0.700pt}{0.400pt}}
\multiput(618.00,477.17)(1.547,2.000){2}{\rule{0.350pt}{0.400pt}}
\put(621,479.67){\rule{0.723pt}{0.400pt}}
\multiput(621.00,479.17)(1.500,1.000){2}{\rule{0.361pt}{0.400pt}}
\put(624,480.67){\rule{0.723pt}{0.400pt}}
\multiput(624.00,480.17)(1.500,1.000){2}{\rule{0.361pt}{0.400pt}}
\put(627,481.67){\rule{0.723pt}{0.400pt}}
\multiput(627.00,481.17)(1.500,1.000){2}{\rule{0.361pt}{0.400pt}}
\put(630,482.67){\rule{0.723pt}{0.400pt}}
\multiput(630.00,482.17)(1.500,1.000){2}{\rule{0.361pt}{0.400pt}}
\put(633,483.67){\rule{0.723pt}{0.400pt}}
\multiput(633.00,483.17)(1.500,1.000){2}{\rule{0.361pt}{0.400pt}}
\put(636,485.17){\rule{0.700pt}{0.400pt}}
\multiput(636.00,484.17)(1.547,2.000){2}{\rule{0.350pt}{0.400pt}}
\put(639,486.67){\rule{0.723pt}{0.400pt}}
\multiput(639.00,486.17)(1.500,1.000){2}{\rule{0.361pt}{0.400pt}}
\put(642,487.67){\rule{0.723pt}{0.400pt}}
\multiput(642.00,487.17)(1.500,1.000){2}{\rule{0.361pt}{0.400pt}}
\put(645,488.67){\rule{0.723pt}{0.400pt}}
\multiput(645.00,488.17)(1.500,1.000){2}{\rule{0.361pt}{0.400pt}}
\put(648,489.67){\rule{0.723pt}{0.400pt}}
\multiput(648.00,489.17)(1.500,1.000){2}{\rule{0.361pt}{0.400pt}}
\put(651,490.67){\rule{0.964pt}{0.400pt}}
\multiput(651.00,490.17)(2.000,1.000){2}{\rule{0.482pt}{0.400pt}}
\put(655,492.17){\rule{0.700pt}{0.400pt}}
\multiput(655.00,491.17)(1.547,2.000){2}{\rule{0.350pt}{0.400pt}}
\put(658,493.67){\rule{0.723pt}{0.400pt}}
\multiput(658.00,493.17)(1.500,1.000){2}{\rule{0.361pt}{0.400pt}}
\put(661,494.67){\rule{0.723pt}{0.400pt}}
\multiput(661.00,494.17)(1.500,1.000){2}{\rule{0.361pt}{0.400pt}}
\put(664,495.67){\rule{0.723pt}{0.400pt}}
\multiput(664.00,495.17)(1.500,1.000){2}{\rule{0.361pt}{0.400pt}}
\put(667,496.67){\rule{0.723pt}{0.400pt}}
\multiput(667.00,496.17)(1.500,1.000){2}{\rule{0.361pt}{0.400pt}}
\put(670,497.67){\rule{0.723pt}{0.400pt}}
\multiput(670.00,497.17)(1.500,1.000){2}{\rule{0.361pt}{0.400pt}}
\put(673,499.17){\rule{0.700pt}{0.400pt}}
\multiput(673.00,498.17)(1.547,2.000){2}{\rule{0.350pt}{0.400pt}}
\put(676,500.67){\rule{0.723pt}{0.400pt}}
\multiput(676.00,500.17)(1.500,1.000){2}{\rule{0.361pt}{0.400pt}}
\put(679,501.67){\rule{0.723pt}{0.400pt}}
\multiput(679.00,501.17)(1.500,1.000){2}{\rule{0.361pt}{0.400pt}}
\put(682,502.67){\rule{0.723pt}{0.400pt}}
\multiput(682.00,502.17)(1.500,1.000){2}{\rule{0.361pt}{0.400pt}}
\put(685,503.67){\rule{0.723pt}{0.400pt}}
\multiput(685.00,503.17)(1.500,1.000){2}{\rule{0.361pt}{0.400pt}}
\put(688,504.67){\rule{0.723pt}{0.400pt}}
\multiput(688.00,504.17)(1.500,1.000){2}{\rule{0.361pt}{0.400pt}}
\put(691,506.17){\rule{0.700pt}{0.400pt}}
\multiput(691.00,505.17)(1.547,2.000){2}{\rule{0.350pt}{0.400pt}}
\put(694,507.67){\rule{0.723pt}{0.400pt}}
\multiput(694.00,507.17)(1.500,1.000){2}{\rule{0.361pt}{0.400pt}}
\put(697,508.67){\rule{0.723pt}{0.400pt}}
\multiput(697.00,508.17)(1.500,1.000){2}{\rule{0.361pt}{0.400pt}}
\put(700,509.67){\rule{0.723pt}{0.400pt}}
\multiput(700.00,509.17)(1.500,1.000){2}{\rule{0.361pt}{0.400pt}}
\put(703,510.67){\rule{0.723pt}{0.400pt}}
\multiput(703.00,510.17)(1.500,1.000){2}{\rule{0.361pt}{0.400pt}}
\put(706,511.67){\rule{0.723pt}{0.400pt}}
\multiput(706.00,511.17)(1.500,1.000){2}{\rule{0.361pt}{0.400pt}}
\put(709,512.67){\rule{0.964pt}{0.400pt}}
\multiput(709.00,512.17)(2.000,1.000){2}{\rule{0.482pt}{0.400pt}}
\put(713,514.17){\rule{0.700pt}{0.400pt}}
\multiput(713.00,513.17)(1.547,2.000){2}{\rule{0.350pt}{0.400pt}}
\put(716,515.67){\rule{0.723pt}{0.400pt}}
\multiput(716.00,515.17)(1.500,1.000){2}{\rule{0.361pt}{0.400pt}}
\put(719,516.67){\rule{0.723pt}{0.400pt}}
\multiput(719.00,516.17)(1.500,1.000){2}{\rule{0.361pt}{0.400pt}}
\put(722,517.67){\rule{0.723pt}{0.400pt}}
\multiput(722.00,517.17)(1.500,1.000){2}{\rule{0.361pt}{0.400pt}}
\put(725,518.67){\rule{0.723pt}{0.400pt}}
\multiput(725.00,518.17)(1.500,1.000){2}{\rule{0.361pt}{0.400pt}}
\put(728,519.67){\rule{0.723pt}{0.400pt}}
\multiput(728.00,519.17)(1.500,1.000){2}{\rule{0.361pt}{0.400pt}}
\put(731,521.17){\rule{0.700pt}{0.400pt}}
\multiput(731.00,520.17)(1.547,2.000){2}{\rule{0.350pt}{0.400pt}}
\put(734,522.67){\rule{0.723pt}{0.400pt}}
\multiput(734.00,522.17)(1.500,1.000){2}{\rule{0.361pt}{0.400pt}}
\put(737,523.67){\rule{0.723pt}{0.400pt}}
\multiput(737.00,523.17)(1.500,1.000){2}{\rule{0.361pt}{0.400pt}}
\put(740,524.67){\rule{0.723pt}{0.400pt}}
\multiput(740.00,524.17)(1.500,1.000){2}{\rule{0.361pt}{0.400pt}}
\put(743,525.67){\rule{0.723pt}{0.400pt}}
\multiput(743.00,525.17)(1.500,1.000){2}{\rule{0.361pt}{0.400pt}}
\put(746,526.67){\rule{0.723pt}{0.400pt}}
\multiput(746.00,526.17)(1.500,1.000){2}{\rule{0.361pt}{0.400pt}}
\put(749,528.17){\rule{0.700pt}{0.400pt}}
\multiput(749.00,527.17)(1.547,2.000){2}{\rule{0.350pt}{0.400pt}}
\put(752,529.67){\rule{0.723pt}{0.400pt}}
\multiput(752.00,529.17)(1.500,1.000){2}{\rule{0.361pt}{0.400pt}}
\put(755,530.67){\rule{0.723pt}{0.400pt}}
\multiput(755.00,530.17)(1.500,1.000){2}{\rule{0.361pt}{0.400pt}}
\put(758,531.67){\rule{0.723pt}{0.400pt}}
\multiput(758.00,531.17)(1.500,1.000){2}{\rule{0.361pt}{0.400pt}}
\put(761,532.67){\rule{0.723pt}{0.400pt}}
\multiput(761.00,532.17)(1.500,1.000){2}{\rule{0.361pt}{0.400pt}}
\put(764,533.67){\rule{0.723pt}{0.400pt}}
\multiput(764.00,533.17)(1.500,1.000){2}{\rule{0.361pt}{0.400pt}}
\put(767,535.17){\rule{0.900pt}{0.400pt}}
\multiput(767.00,534.17)(2.132,2.000){2}{\rule{0.450pt}{0.400pt}}
\put(771,536.67){\rule{0.723pt}{0.400pt}}
\multiput(771.00,536.17)(1.500,1.000){2}{\rule{0.361pt}{0.400pt}}
\put(774,537.67){\rule{0.723pt}{0.400pt}}
\multiput(774.00,537.17)(1.500,1.000){2}{\rule{0.361pt}{0.400pt}}
\put(777,538.67){\rule{0.723pt}{0.400pt}}
\multiput(777.00,538.17)(1.500,1.000){2}{\rule{0.361pt}{0.400pt}}
\put(780,539.67){\rule{0.723pt}{0.400pt}}
\multiput(780.00,539.17)(1.500,1.000){2}{\rule{0.361pt}{0.400pt}}
\put(783,540.67){\rule{0.723pt}{0.400pt}}
\multiput(783.00,540.17)(1.500,1.000){2}{\rule{0.361pt}{0.400pt}}
\put(786,542.17){\rule{0.700pt}{0.400pt}}
\multiput(786.00,541.17)(1.547,2.000){2}{\rule{0.350pt}{0.400pt}}
\put(789,543.67){\rule{0.723pt}{0.400pt}}
\multiput(789.00,543.17)(1.500,1.000){2}{\rule{0.361pt}{0.400pt}}
\put(792,544.67){\rule{0.723pt}{0.400pt}}
\multiput(792.00,544.17)(1.500,1.000){2}{\rule{0.361pt}{0.400pt}}
\put(795,545.67){\rule{0.723pt}{0.400pt}}
\multiput(795.00,545.17)(1.500,1.000){2}{\rule{0.361pt}{0.400pt}}
\put(798,546.67){\rule{0.723pt}{0.400pt}}
\multiput(798.00,546.17)(1.500,1.000){2}{\rule{0.361pt}{0.400pt}}
\put(801,547.67){\rule{0.723pt}{0.400pt}}
\multiput(801.00,547.17)(1.500,1.000){2}{\rule{0.361pt}{0.400pt}}
\put(804,549.17){\rule{0.700pt}{0.400pt}}
\multiput(804.00,548.17)(1.547,2.000){2}{\rule{0.350pt}{0.400pt}}
\put(807,550.67){\rule{0.723pt}{0.400pt}}
\multiput(807.00,550.17)(1.500,1.000){2}{\rule{0.361pt}{0.400pt}}
\put(810,551.67){\rule{0.723pt}{0.400pt}}
\multiput(810.00,551.17)(1.500,1.000){2}{\rule{0.361pt}{0.400pt}}
\put(813,552.67){\rule{0.723pt}{0.400pt}}
\multiput(813.00,552.17)(1.500,1.000){2}{\rule{0.361pt}{0.400pt}}
\put(816,553.67){\rule{0.723pt}{0.400pt}}
\multiput(816.00,553.17)(1.500,1.000){2}{\rule{0.361pt}{0.400pt}}
\put(819,554.67){\rule{0.723pt}{0.400pt}}
\multiput(819.00,554.17)(1.500,1.000){2}{\rule{0.361pt}{0.400pt}}
\put(822,556.17){\rule{0.700pt}{0.400pt}}
\multiput(822.00,555.17)(1.547,2.000){2}{\rule{0.350pt}{0.400pt}}
\put(825,557.67){\rule{0.964pt}{0.400pt}}
\multiput(825.00,557.17)(2.000,1.000){2}{\rule{0.482pt}{0.400pt}}
\put(829,558.67){\rule{0.723pt}{0.400pt}}
\multiput(829.00,558.17)(1.500,1.000){2}{\rule{0.361pt}{0.400pt}}
\put(832,559.67){\rule{0.723pt}{0.400pt}}
\multiput(832.00,559.17)(1.500,1.000){2}{\rule{0.361pt}{0.400pt}}
\put(835,560.67){\rule{0.723pt}{0.400pt}}
\multiput(835.00,560.17)(1.500,1.000){2}{\rule{0.361pt}{0.400pt}}
\put(838,561.67){\rule{0.723pt}{0.400pt}}
\multiput(838.00,561.17)(1.500,1.000){2}{\rule{0.361pt}{0.400pt}}
\put(841,563.17){\rule{0.700pt}{0.400pt}}
\multiput(841.00,562.17)(1.547,2.000){2}{\rule{0.350pt}{0.400pt}}
\put(844,564.67){\rule{0.723pt}{0.400pt}}
\multiput(844.00,564.17)(1.500,1.000){2}{\rule{0.361pt}{0.400pt}}
\put(847,565.67){\rule{0.723pt}{0.400pt}}
\multiput(847.00,565.17)(1.500,1.000){2}{\rule{0.361pt}{0.400pt}}
\put(850,566.67){\rule{0.723pt}{0.400pt}}
\multiput(850.00,566.17)(1.500,1.000){2}{\rule{0.361pt}{0.400pt}}
\put(853,567.67){\rule{0.723pt}{0.400pt}}
\multiput(853.00,567.17)(1.500,1.000){2}{\rule{0.361pt}{0.400pt}}
\put(856,568.67){\rule{0.723pt}{0.400pt}}
\multiput(856.00,568.17)(1.500,1.000){2}{\rule{0.361pt}{0.400pt}}
\put(859,570.17){\rule{0.700pt}{0.400pt}}
\multiput(859.00,569.17)(1.547,2.000){2}{\rule{0.350pt}{0.400pt}}
\put(862,571.67){\rule{0.723pt}{0.400pt}}
\multiput(862.00,571.17)(1.500,1.000){2}{\rule{0.361pt}{0.400pt}}
\put(865,572.67){\rule{0.723pt}{0.400pt}}
\multiput(865.00,572.17)(1.500,1.000){2}{\rule{0.361pt}{0.400pt}}
\put(868,573.67){\rule{0.723pt}{0.400pt}}
\multiput(868.00,573.17)(1.500,1.000){2}{\rule{0.361pt}{0.400pt}}
\put(871,574.67){\rule{0.723pt}{0.400pt}}
\multiput(871.00,574.17)(1.500,1.000){2}{\rule{0.361pt}{0.400pt}}
\put(874,575.67){\rule{0.723pt}{0.400pt}}
\multiput(874.00,575.17)(1.500,1.000){2}{\rule{0.361pt}{0.400pt}}
\put(877,577.17){\rule{0.700pt}{0.400pt}}
\multiput(877.00,576.17)(1.547,2.000){2}{\rule{0.350pt}{0.400pt}}
\put(880,578.67){\rule{0.723pt}{0.400pt}}
\multiput(880.00,578.17)(1.500,1.000){2}{\rule{0.361pt}{0.400pt}}
\put(883,579.67){\rule{0.964pt}{0.400pt}}
\multiput(883.00,579.17)(2.000,1.000){2}{\rule{0.482pt}{0.400pt}}
\put(887,580.67){\rule{0.723pt}{0.400pt}}
\multiput(887.00,580.17)(1.500,1.000){2}{\rule{0.361pt}{0.400pt}}
\put(890,581.67){\rule{0.723pt}{0.400pt}}
\multiput(890.00,581.17)(1.500,1.000){2}{\rule{0.361pt}{0.400pt}}
\put(893,582.67){\rule{0.723pt}{0.400pt}}
\multiput(893.00,582.17)(1.500,1.000){2}{\rule{0.361pt}{0.400pt}}
\put(896,584.17){\rule{0.700pt}{0.400pt}}
\multiput(896.00,583.17)(1.547,2.000){2}{\rule{0.350pt}{0.400pt}}
\put(899,585.67){\rule{0.723pt}{0.400pt}}
\multiput(899.00,585.17)(1.500,1.000){2}{\rule{0.361pt}{0.400pt}}
\put(902,586.67){\rule{0.723pt}{0.400pt}}
\multiput(902.00,586.17)(1.500,1.000){2}{\rule{0.361pt}{0.400pt}}
\put(905,587.67){\rule{0.723pt}{0.400pt}}
\multiput(905.00,587.17)(1.500,1.000){2}{\rule{0.361pt}{0.400pt}}
\put(908,588.67){\rule{0.723pt}{0.400pt}}
\multiput(908.00,588.17)(1.500,1.000){2}{\rule{0.361pt}{0.400pt}}
\put(911,589.67){\rule{0.723pt}{0.400pt}}
\multiput(911.00,589.17)(1.500,1.000){2}{\rule{0.361pt}{0.400pt}}
\put(914,591.17){\rule{0.700pt}{0.400pt}}
\multiput(914.00,590.17)(1.547,2.000){2}{\rule{0.350pt}{0.400pt}}
\put(917,592.67){\rule{0.723pt}{0.400pt}}
\multiput(917.00,592.17)(1.500,1.000){2}{\rule{0.361pt}{0.400pt}}
\put(920,593.67){\rule{0.723pt}{0.400pt}}
\multiput(920.00,593.17)(1.500,1.000){2}{\rule{0.361pt}{0.400pt}}
\put(923,594.67){\rule{0.723pt}{0.400pt}}
\multiput(923.00,594.17)(1.500,1.000){2}{\rule{0.361pt}{0.400pt}}
\put(926,595.67){\rule{0.723pt}{0.400pt}}
\multiput(926.00,595.17)(1.500,1.000){2}{\rule{0.361pt}{0.400pt}}
\put(929,596.67){\rule{0.723pt}{0.400pt}}
\multiput(929.00,596.17)(1.500,1.000){2}{\rule{0.361pt}{0.400pt}}
\put(932,598.17){\rule{0.700pt}{0.400pt}}
\multiput(932.00,597.17)(1.547,2.000){2}{\rule{0.350pt}{0.400pt}}
\put(935,599.67){\rule{0.723pt}{0.400pt}}
\multiput(935.00,599.17)(1.500,1.000){2}{\rule{0.361pt}{0.400pt}}
\put(938,600.67){\rule{0.723pt}{0.400pt}}
\multiput(938.00,600.17)(1.500,1.000){2}{\rule{0.361pt}{0.400pt}}
\put(941,601.67){\rule{0.723pt}{0.400pt}}
\multiput(941.00,601.17)(1.500,1.000){2}{\rule{0.361pt}{0.400pt}}
\put(944,602.67){\rule{0.964pt}{0.400pt}}
\multiput(944.00,602.17)(2.000,1.000){2}{\rule{0.482pt}{0.400pt}}
\put(948,603.67){\rule{0.723pt}{0.400pt}}
\multiput(948.00,603.17)(1.500,1.000){2}{\rule{0.361pt}{0.400pt}}
\put(951,605.17){\rule{0.700pt}{0.400pt}}
\multiput(951.00,604.17)(1.547,2.000){2}{\rule{0.350pt}{0.400pt}}
\put(954,606.67){\rule{0.723pt}{0.400pt}}
\multiput(954.00,606.17)(1.500,1.000){2}{\rule{0.361pt}{0.400pt}}
\put(957,607.67){\rule{0.723pt}{0.400pt}}
\multiput(957.00,607.17)(1.500,1.000){2}{\rule{0.361pt}{0.400pt}}
\put(960,608.67){\rule{0.723pt}{0.400pt}}
\multiput(960.00,608.17)(1.500,1.000){2}{\rule{0.361pt}{0.400pt}}
\put(963,609.67){\rule{0.723pt}{0.400pt}}
\multiput(963.00,609.17)(1.500,1.000){2}{\rule{0.361pt}{0.400pt}}
\put(966,610.67){\rule{0.723pt}{0.400pt}}
\multiput(966.00,610.17)(1.500,1.000){2}{\rule{0.361pt}{0.400pt}}
\put(969,612.17){\rule{0.700pt}{0.400pt}}
\multiput(969.00,611.17)(1.547,2.000){2}{\rule{0.350pt}{0.400pt}}
\put(972,613.67){\rule{0.723pt}{0.400pt}}
\multiput(972.00,613.17)(1.500,1.000){2}{\rule{0.361pt}{0.400pt}}
\put(975,614.67){\rule{0.723pt}{0.400pt}}
\multiput(975.00,614.17)(1.500,1.000){2}{\rule{0.361pt}{0.400pt}}
\put(978,615.67){\rule{0.723pt}{0.400pt}}
\multiput(978.00,615.17)(1.500,1.000){2}{\rule{0.361pt}{0.400pt}}
\put(981,616.67){\rule{0.723pt}{0.400pt}}
\multiput(981.00,616.17)(1.500,1.000){2}{\rule{0.361pt}{0.400pt}}
\put(984,617.67){\rule{0.723pt}{0.400pt}}
\multiput(984.00,617.17)(1.500,1.000){2}{\rule{0.361pt}{0.400pt}}
\put(987,619.17){\rule{0.700pt}{0.400pt}}
\multiput(987.00,618.17)(1.547,2.000){2}{\rule{0.350pt}{0.400pt}}
\put(990,620.67){\rule{0.723pt}{0.400pt}}
\multiput(990.00,620.17)(1.500,1.000){2}{\rule{0.361pt}{0.400pt}}
\put(993,621.67){\rule{0.723pt}{0.400pt}}
\multiput(993.00,621.17)(1.500,1.000){2}{\rule{0.361pt}{0.400pt}}
\put(996,622.67){\rule{0.723pt}{0.400pt}}
\multiput(996.00,622.17)(1.500,1.000){2}{\rule{0.361pt}{0.400pt}}
\put(999,623.67){\rule{0.723pt}{0.400pt}}
\multiput(999.00,623.17)(1.500,1.000){2}{\rule{0.361pt}{0.400pt}}
\put(1002,624.67){\rule{0.964pt}{0.400pt}}
\multiput(1002.00,624.17)(2.000,1.000){2}{\rule{0.482pt}{0.400pt}}
\put(1006,626.17){\rule{0.700pt}{0.400pt}}
\multiput(1006.00,625.17)(1.547,2.000){2}{\rule{0.350pt}{0.400pt}}
\put(1009,627.67){\rule{0.723pt}{0.400pt}}
\multiput(1009.00,627.17)(1.500,1.000){2}{\rule{0.361pt}{0.400pt}}
\put(1012,628.67){\rule{0.723pt}{0.400pt}}
\multiput(1012.00,628.17)(1.500,1.000){2}{\rule{0.361pt}{0.400pt}}
\put(1015,629.67){\rule{0.723pt}{0.400pt}}
\multiput(1015.00,629.17)(1.500,1.000){2}{\rule{0.361pt}{0.400pt}}
\put(1018,630.67){\rule{0.723pt}{0.400pt}}
\multiput(1018.00,630.17)(1.500,1.000){2}{\rule{0.361pt}{0.400pt}}
\put(1021,631.67){\rule{0.723pt}{0.400pt}}
\multiput(1021.00,631.17)(1.500,1.000){2}{\rule{0.361pt}{0.400pt}}
\put(1024,633.17){\rule{0.700pt}{0.400pt}}
\multiput(1024.00,632.17)(1.547,2.000){2}{\rule{0.350pt}{0.400pt}}
\put(1027,634.67){\rule{0.723pt}{0.400pt}}
\multiput(1027.00,634.17)(1.500,1.000){2}{\rule{0.361pt}{0.400pt}}
\put(1030,635.67){\rule{0.723pt}{0.400pt}}
\multiput(1030.00,635.17)(1.500,1.000){2}{\rule{0.361pt}{0.400pt}}
\put(1033,636.67){\rule{0.723pt}{0.400pt}}
\multiput(1033.00,636.17)(1.500,1.000){2}{\rule{0.361pt}{0.400pt}}
\put(1036,637.67){\rule{0.723pt}{0.400pt}}
\multiput(1036.00,637.17)(1.500,1.000){2}{\rule{0.361pt}{0.400pt}}
\put(1039,638.67){\rule{0.723pt}{0.400pt}}
\multiput(1039.00,638.17)(1.500,1.000){2}{\rule{0.361pt}{0.400pt}}
\put(1042,640.17){\rule{0.700pt}{0.400pt}}
\multiput(1042.00,639.17)(1.547,2.000){2}{\rule{0.350pt}{0.400pt}}
\put(1045,641.67){\rule{0.723pt}{0.400pt}}
\multiput(1045.00,641.17)(1.500,1.000){2}{\rule{0.361pt}{0.400pt}}
\put(1048,642.67){\rule{0.723pt}{0.400pt}}
\multiput(1048.00,642.17)(1.500,1.000){2}{\rule{0.361pt}{0.400pt}}
\put(1051,643.67){\rule{0.723pt}{0.400pt}}
\multiput(1051.00,643.17)(1.500,1.000){2}{\rule{0.361pt}{0.400pt}}
\put(1054,644.67){\rule{0.723pt}{0.400pt}}
\multiput(1054.00,644.17)(1.500,1.000){2}{\rule{0.361pt}{0.400pt}}
\put(1057,645.67){\rule{0.723pt}{0.400pt}}
\multiput(1057.00,645.17)(1.500,1.000){2}{\rule{0.361pt}{0.400pt}}
\put(1060,647.17){\rule{0.900pt}{0.400pt}}
\multiput(1060.00,646.17)(2.132,2.000){2}{\rule{0.450pt}{0.400pt}}
\put(1064,648.67){\rule{0.723pt}{0.400pt}}
\multiput(1064.00,648.17)(1.500,1.000){2}{\rule{0.361pt}{0.400pt}}
\put(1067,649.67){\rule{0.723pt}{0.400pt}}
\multiput(1067.00,649.17)(1.500,1.000){2}{\rule{0.361pt}{0.400pt}}
\put(1070,650.67){\rule{0.723pt}{0.400pt}}
\multiput(1070.00,650.17)(1.500,1.000){2}{\rule{0.361pt}{0.400pt}}
\put(1073,651.67){\rule{0.723pt}{0.400pt}}
\multiput(1073.00,651.17)(1.500,1.000){2}{\rule{0.361pt}{0.400pt}}
\put(1076,652.67){\rule{0.723pt}{0.400pt}}
\multiput(1076.00,652.17)(1.500,1.000){2}{\rule{0.361pt}{0.400pt}}
\put(1079,654.17){\rule{0.700pt}{0.400pt}}
\multiput(1079.00,653.17)(1.547,2.000){2}{\rule{0.350pt}{0.400pt}}
\put(1082,655.67){\rule{0.723pt}{0.400pt}}
\multiput(1082.00,655.17)(1.500,1.000){2}{\rule{0.361pt}{0.400pt}}
\put(1085,656.67){\rule{0.723pt}{0.400pt}}
\multiput(1085.00,656.17)(1.500,1.000){2}{\rule{0.361pt}{0.400pt}}
\put(1088,657.67){\rule{0.723pt}{0.400pt}}
\multiput(1088.00,657.17)(1.500,1.000){2}{\rule{0.361pt}{0.400pt}}
\put(1091,658.67){\rule{0.723pt}{0.400pt}}
\multiput(1091.00,658.17)(1.500,1.000){2}{\rule{0.361pt}{0.400pt}}
\put(1094,659.67){\rule{0.723pt}{0.400pt}}
\multiput(1094.00,659.17)(1.500,1.000){2}{\rule{0.361pt}{0.400pt}}
\put(1097,661.17){\rule{0.700pt}{0.400pt}}
\multiput(1097.00,660.17)(1.547,2.000){2}{\rule{0.350pt}{0.400pt}}
\put(1100,662.67){\rule{0.723pt}{0.400pt}}
\multiput(1100.00,662.17)(1.500,1.000){2}{\rule{0.361pt}{0.400pt}}
\put(1103,663.67){\rule{0.723pt}{0.400pt}}
\multiput(1103.00,663.17)(1.500,1.000){2}{\rule{0.361pt}{0.400pt}}
\put(1106,664.67){\rule{0.723pt}{0.400pt}}
\multiput(1106.00,664.17)(1.500,1.000){2}{\rule{0.361pt}{0.400pt}}
\put(1109,665.67){\rule{0.723pt}{0.400pt}}
\multiput(1109.00,665.17)(1.500,1.000){2}{\rule{0.361pt}{0.400pt}}
\put(1112,666.67){\rule{0.723pt}{0.400pt}}
\multiput(1112.00,666.17)(1.500,1.000){2}{\rule{0.361pt}{0.400pt}}
\put(1115,668.17){\rule{0.700pt}{0.400pt}}
\multiput(1115.00,667.17)(1.547,2.000){2}{\rule{0.350pt}{0.400pt}}
\put(1118,669.67){\rule{0.964pt}{0.400pt}}
\multiput(1118.00,669.17)(2.000,1.000){2}{\rule{0.482pt}{0.400pt}}
\put(1122,670.67){\rule{0.723pt}{0.400pt}}
\multiput(1122.00,670.17)(1.500,1.000){2}{\rule{0.361pt}{0.400pt}}
\put(1125,671.67){\rule{0.723pt}{0.400pt}}
\multiput(1125.00,671.17)(1.500,1.000){2}{\rule{0.361pt}{0.400pt}}
\put(1128,672.67){\rule{0.723pt}{0.400pt}}
\multiput(1128.00,672.17)(1.500,1.000){2}{\rule{0.361pt}{0.400pt}}
\put(1131,673.67){\rule{0.723pt}{0.400pt}}
\multiput(1131.00,673.17)(1.500,1.000){2}{\rule{0.361pt}{0.400pt}}
\put(1134,675.17){\rule{0.700pt}{0.400pt}}
\multiput(1134.00,674.17)(1.547,2.000){2}{\rule{0.350pt}{0.400pt}}
\put(1137,676.67){\rule{0.723pt}{0.400pt}}
\multiput(1137.00,676.17)(1.500,1.000){2}{\rule{0.361pt}{0.400pt}}
\put(1140,677.67){\rule{0.723pt}{0.400pt}}
\multiput(1140.00,677.17)(1.500,1.000){2}{\rule{0.361pt}{0.400pt}}
\put(1143,678.67){\rule{0.723pt}{0.400pt}}
\multiput(1143.00,678.17)(1.500,1.000){2}{\rule{0.361pt}{0.400pt}}
\put(1146,679.67){\rule{0.723pt}{0.400pt}}
\multiput(1146.00,679.17)(1.500,1.000){2}{\rule{0.361pt}{0.400pt}}
\put(1149,680.67){\rule{0.723pt}{0.400pt}}
\multiput(1149.00,680.17)(1.500,1.000){2}{\rule{0.361pt}{0.400pt}}
\put(1152,682.17){\rule{0.700pt}{0.400pt}}
\multiput(1152.00,681.17)(1.547,2.000){2}{\rule{0.350pt}{0.400pt}}
\put(1155,683.67){\rule{0.723pt}{0.400pt}}
\multiput(1155.00,683.17)(1.500,1.000){2}{\rule{0.361pt}{0.400pt}}
\put(1158,684.67){\rule{0.723pt}{0.400pt}}
\multiput(1158.00,684.17)(1.500,1.000){2}{\rule{0.361pt}{0.400pt}}
\put(1161,685.67){\rule{0.723pt}{0.400pt}}
\multiput(1161.00,685.17)(1.500,1.000){2}{\rule{0.361pt}{0.400pt}}
\put(1164,686.67){\rule{0.723pt}{0.400pt}}
\multiput(1164.00,686.17)(1.500,1.000){2}{\rule{0.361pt}{0.400pt}}
\put(1167,687.67){\rule{0.723pt}{0.400pt}}
\multiput(1167.00,687.17)(1.500,1.000){2}{\rule{0.361pt}{0.400pt}}
\put(1170,689.17){\rule{0.700pt}{0.400pt}}
\multiput(1170.00,688.17)(1.547,2.000){2}{\rule{0.350pt}{0.400pt}}
\put(1173,690.67){\rule{0.723pt}{0.400pt}}
\multiput(1173.00,690.17)(1.500,1.000){2}{\rule{0.361pt}{0.400pt}}
\put(1176,691.67){\rule{0.964pt}{0.400pt}}
\multiput(1176.00,691.17)(2.000,1.000){2}{\rule{0.482pt}{0.400pt}}
\put(1180,692.67){\rule{0.723pt}{0.400pt}}
\multiput(1180.00,692.17)(1.500,1.000){2}{\rule{0.361pt}{0.400pt}}
\put(1183,693.67){\rule{0.723pt}{0.400pt}}
\multiput(1183.00,693.17)(1.500,1.000){2}{\rule{0.361pt}{0.400pt}}
\put(1186,694.67){\rule{0.723pt}{0.400pt}}
\multiput(1186.00,694.17)(1.500,1.000){2}{\rule{0.361pt}{0.400pt}}
\put(1189,696.17){\rule{0.700pt}{0.400pt}}
\multiput(1189.00,695.17)(1.547,2.000){2}{\rule{0.350pt}{0.400pt}}
\put(1192,697.67){\rule{0.723pt}{0.400pt}}
\multiput(1192.00,697.17)(1.500,1.000){2}{\rule{0.361pt}{0.400pt}}
\put(1195,698.67){\rule{0.723pt}{0.400pt}}
\multiput(1195.00,698.17)(1.500,1.000){2}{\rule{0.361pt}{0.400pt}}
\put(1198,699.67){\rule{0.723pt}{0.400pt}}
\multiput(1198.00,699.17)(1.500,1.000){2}{\rule{0.361pt}{0.400pt}}
\put(1201,700.67){\rule{0.723pt}{0.400pt}}
\multiput(1201.00,700.17)(1.500,1.000){2}{\rule{0.361pt}{0.400pt}}
\put(1204,701.67){\rule{0.723pt}{0.400pt}}
\multiput(1204.00,701.17)(1.500,1.000){2}{\rule{0.361pt}{0.400pt}}
\put(1207,703.17){\rule{0.700pt}{0.400pt}}
\multiput(1207.00,702.17)(1.547,2.000){2}{\rule{0.350pt}{0.400pt}}
\put(1210,704.67){\rule{0.723pt}{0.400pt}}
\multiput(1210.00,704.17)(1.500,1.000){2}{\rule{0.361pt}{0.400pt}}
\put(1213,705.67){\rule{0.723pt}{0.400pt}}
\multiput(1213.00,705.17)(1.500,1.000){2}{\rule{0.361pt}{0.400pt}}
\put(1216,706.67){\rule{0.723pt}{0.400pt}}
\multiput(1216.00,706.17)(1.500,1.000){2}{\rule{0.361pt}{0.400pt}}
\put(1219,707.67){\rule{0.723pt}{0.400pt}}
\multiput(1219.00,707.17)(1.500,1.000){2}{\rule{0.361pt}{0.400pt}}
\put(1222,708.67){\rule{0.723pt}{0.400pt}}
\multiput(1222.00,708.17)(1.500,1.000){2}{\rule{0.361pt}{0.400pt}}
\put(1225,709.67){\rule{0.723pt}{0.400pt}}
\multiput(1225.00,709.17)(1.500,1.000){2}{\rule{0.361pt}{0.400pt}}
\put(1228,711.17){\rule{0.700pt}{0.400pt}}
\multiput(1228.00,710.17)(1.547,2.000){2}{\rule{0.350pt}{0.400pt}}
\put(1231,712.67){\rule{0.723pt}{0.400pt}}
\multiput(1231.00,712.17)(1.500,1.000){2}{\rule{0.361pt}{0.400pt}}
\put(1234,713.67){\rule{0.964pt}{0.400pt}}
\multiput(1234.00,713.17)(2.000,1.000){2}{\rule{0.482pt}{0.400pt}}
\put(1238,714.67){\rule{0.723pt}{0.400pt}}
\multiput(1238.00,714.17)(1.500,1.000){2}{\rule{0.361pt}{0.400pt}}
\put(1241,715.67){\rule{0.723pt}{0.400pt}}
\multiput(1241.00,715.17)(1.500,1.000){2}{\rule{0.361pt}{0.400pt}}
\put(1244,716.67){\rule{0.723pt}{0.400pt}}
\multiput(1244.00,716.17)(1.500,1.000){2}{\rule{0.361pt}{0.400pt}}
\put(1247,718.17){\rule{0.700pt}{0.400pt}}
\multiput(1247.00,717.17)(1.547,2.000){2}{\rule{0.350pt}{0.400pt}}
\put(1250,719.67){\rule{0.723pt}{0.400pt}}
\multiput(1250.00,719.17)(1.500,1.000){2}{\rule{0.361pt}{0.400pt}}
\put(1253,720.67){\rule{0.723pt}{0.400pt}}
\multiput(1253.00,720.17)(1.500,1.000){2}{\rule{0.361pt}{0.400pt}}
\put(1256,721.67){\rule{0.723pt}{0.400pt}}
\multiput(1256.00,721.17)(1.500,1.000){2}{\rule{0.361pt}{0.400pt}}
\put(1259,722.67){\rule{0.723pt}{0.400pt}}
\multiput(1259.00,722.17)(1.500,1.000){2}{\rule{0.361pt}{0.400pt}}
\put(1262,723.67){\rule{0.723pt}{0.400pt}}
\multiput(1262.00,723.17)(1.500,1.000){2}{\rule{0.361pt}{0.400pt}}
\put(1265,725.17){\rule{0.700pt}{0.400pt}}
\multiput(1265.00,724.17)(1.547,2.000){2}{\rule{0.350pt}{0.400pt}}
\put(1268,726.67){\rule{0.723pt}{0.400pt}}
\multiput(1268.00,726.17)(1.500,1.000){2}{\rule{0.361pt}{0.400pt}}
\put(1271,727.67){\rule{0.723pt}{0.400pt}}
\multiput(1271.00,727.17)(1.500,1.000){2}{\rule{0.361pt}{0.400pt}}
\put(1274,728.67){\rule{0.723pt}{0.400pt}}
\multiput(1274.00,728.17)(1.500,1.000){2}{\rule{0.361pt}{0.400pt}}
\put(1277,729.67){\rule{0.723pt}{0.400pt}}
\multiput(1277.00,729.17)(1.500,1.000){2}{\rule{0.361pt}{0.400pt}}
\put(1280,730.67){\rule{0.723pt}{0.400pt}}
\multiput(1280.00,730.17)(1.500,1.000){2}{\rule{0.361pt}{0.400pt}}
\put(1283,732.17){\rule{0.700pt}{0.400pt}}
\multiput(1283.00,731.17)(1.547,2.000){2}{\rule{0.350pt}{0.400pt}}
\put(1286,733.67){\rule{0.723pt}{0.400pt}}
\multiput(1286.00,733.17)(1.500,1.000){2}{\rule{0.361pt}{0.400pt}}
\put(1289,734.67){\rule{0.723pt}{0.400pt}}
\multiput(1289.00,734.17)(1.500,1.000){2}{\rule{0.361pt}{0.400pt}}
\put(1292,735.67){\rule{0.723pt}{0.400pt}}
\multiput(1292.00,735.17)(1.500,1.000){2}{\rule{0.361pt}{0.400pt}}
\put(1295,736.67){\rule{0.964pt}{0.400pt}}
\multiput(1295.00,736.17)(2.000,1.000){2}{\rule{0.482pt}{0.400pt}}
\put(1299,737.67){\rule{0.723pt}{0.400pt}}
\multiput(1299.00,737.17)(1.500,1.000){2}{\rule{0.361pt}{0.400pt}}
\put(1302,739.17){\rule{0.700pt}{0.400pt}}
\multiput(1302.00,738.17)(1.547,2.000){2}{\rule{0.350pt}{0.400pt}}
\put(1305,740.67){\rule{0.723pt}{0.400pt}}
\multiput(1305.00,740.17)(1.500,1.000){2}{\rule{0.361pt}{0.400pt}}
\put(1308,741.67){\rule{0.723pt}{0.400pt}}
\multiput(1308.00,741.17)(1.500,1.000){2}{\rule{0.361pt}{0.400pt}}
\put(1311,742.67){\rule{0.723pt}{0.400pt}}
\multiput(1311.00,742.17)(1.500,1.000){2}{\rule{0.361pt}{0.400pt}}
\put(1314,743.67){\rule{0.723pt}{0.400pt}}
\multiput(1314.00,743.17)(1.500,1.000){2}{\rule{0.361pt}{0.400pt}}
\put(1317,744.67){\rule{0.723pt}{0.400pt}}
\multiput(1317.00,744.17)(1.500,1.000){2}{\rule{0.361pt}{0.400pt}}
\put(1320,746.17){\rule{0.700pt}{0.400pt}}
\multiput(1320.00,745.17)(1.547,2.000){2}{\rule{0.350pt}{0.400pt}}
\put(1323,747.67){\rule{0.723pt}{0.400pt}}
\multiput(1323.00,747.17)(1.500,1.000){2}{\rule{0.361pt}{0.400pt}}
\put(1326,748.67){\rule{0.723pt}{0.400pt}}
\multiput(1326.00,748.17)(1.500,1.000){2}{\rule{0.361pt}{0.400pt}}
\put(1329,749.67){\rule{0.723pt}{0.400pt}}
\multiput(1329.00,749.17)(1.500,1.000){2}{\rule{0.361pt}{0.400pt}}
\put(1332,750.67){\rule{0.723pt}{0.400pt}}
\multiput(1332.00,750.17)(1.500,1.000){2}{\rule{0.361pt}{0.400pt}}
\put(1335,751.67){\rule{0.723pt}{0.400pt}}
\multiput(1335.00,751.17)(1.500,1.000){2}{\rule{0.361pt}{0.400pt}}
\put(1338,753.17){\rule{0.700pt}{0.400pt}}
\multiput(1338.00,752.17)(1.547,2.000){2}{\rule{0.350pt}{0.400pt}}
\put(1341,754.67){\rule{0.723pt}{0.400pt}}
\multiput(1341.00,754.17)(1.500,1.000){2}{\rule{0.361pt}{0.400pt}}
\put(1344,755.67){\rule{0.723pt}{0.400pt}}
\multiput(1344.00,755.17)(1.500,1.000){2}{\rule{0.361pt}{0.400pt}}
\put(1347,756.67){\rule{0.723pt}{0.400pt}}
\multiput(1347.00,756.17)(1.500,1.000){2}{\rule{0.361pt}{0.400pt}}
\put(1350,757.67){\rule{0.723pt}{0.400pt}}
\multiput(1350.00,757.17)(1.500,1.000){2}{\rule{0.361pt}{0.400pt}}
\put(1353,758.67){\rule{0.964pt}{0.400pt}}
\multiput(1353.00,758.17)(2.000,1.000){2}{\rule{0.482pt}{0.400pt}}
\put(1357,760.17){\rule{0.700pt}{0.400pt}}
\multiput(1357.00,759.17)(1.547,2.000){2}{\rule{0.350pt}{0.400pt}}
\put(1360,761.67){\rule{0.723pt}{0.400pt}}
\multiput(1360.00,761.17)(1.500,1.000){2}{\rule{0.361pt}{0.400pt}}
\put(1363,762.67){\rule{0.723pt}{0.400pt}}
\multiput(1363.00,762.17)(1.500,1.000){2}{\rule{0.361pt}{0.400pt}}
\put(1366,763.67){\rule{0.723pt}{0.400pt}}
\multiput(1366.00,763.17)(1.500,1.000){2}{\rule{0.361pt}{0.400pt}}
\put(1369,764.67){\rule{0.723pt}{0.400pt}}
\multiput(1369.00,764.17)(1.500,1.000){2}{\rule{0.361pt}{0.400pt}}
\put(1372,765.67){\rule{0.723pt}{0.400pt}}
\multiput(1372.00,765.17)(1.500,1.000){2}{\rule{0.361pt}{0.400pt}}
\put(1375,767.17){\rule{0.700pt}{0.400pt}}
\multiput(1375.00,766.17)(1.547,2.000){2}{\rule{0.350pt}{0.400pt}}
\put(1378,768.67){\rule{0.723pt}{0.400pt}}
\multiput(1378.00,768.17)(1.500,1.000){2}{\rule{0.361pt}{0.400pt}}
\put(1381,769.67){\rule{0.723pt}{0.400pt}}
\multiput(1381.00,769.17)(1.500,1.000){2}{\rule{0.361pt}{0.400pt}}
\put(1384,770.67){\rule{0.723pt}{0.400pt}}
\multiput(1384.00,770.17)(1.500,1.000){2}{\rule{0.361pt}{0.400pt}}
\put(1387,771.67){\rule{0.723pt}{0.400pt}}
\multiput(1387.00,771.17)(1.500,1.000){2}{\rule{0.361pt}{0.400pt}}
\put(1390,772.67){\rule{0.723pt}{0.400pt}}
\multiput(1390.00,772.17)(1.500,1.000){2}{\rule{0.361pt}{0.400pt}}
\put(1393,774.17){\rule{0.700pt}{0.400pt}}
\multiput(1393.00,773.17)(1.547,2.000){2}{\rule{0.350pt}{0.400pt}}
\put(1396,775.67){\rule{0.723pt}{0.400pt}}
\multiput(1396.00,775.17)(1.500,1.000){2}{\rule{0.361pt}{0.400pt}}
\put(130.0,82.0){\rule[-0.200pt]{0.400pt}{187.179pt}}
\put(130.0,82.0){\rule[-0.200pt]{315.338pt}{0.400pt}}
\put(1439.0,82.0){\rule[-0.200pt]{0.400pt}{187.179pt}}
\put(130.0,859.0){\rule[-0.200pt]{315.338pt}{0.400pt}}
\end{picture}

Plot for Ball 2:\\
% GNUPLOT: LaTeX picture
\setlength{\unitlength}{0.240900pt}
\ifx\plotpoint\undefined\newsavebox{\plotpoint}\fi
\sbox{\plotpoint}{\rule[-0.200pt]{0.400pt}{0.400pt}}%
\begin{picture}(1500,900)(0,0)
\sbox{\plotpoint}{\rule[-0.200pt]{0.400pt}{0.400pt}}%
\put(130.0,90.0){\rule[-0.200pt]{4.818pt}{0.400pt}}
\put(110,90){\makebox(0,0)[r]{ 0}}
\put(1419.0,90.0){\rule[-0.200pt]{4.818pt}{0.400pt}}
\put(130.0,242.0){\rule[-0.200pt]{4.818pt}{0.400pt}}
\put(110,242){\makebox(0,0)[r]{ 0.2}}
\put(1419.0,242.0){\rule[-0.200pt]{4.818pt}{0.400pt}}
\put(130.0,394.0){\rule[-0.200pt]{4.818pt}{0.400pt}}
\put(110,394){\makebox(0,0)[r]{ 0.4}}
\put(1419.0,394.0){\rule[-0.200pt]{4.818pt}{0.400pt}}
\put(130.0,547.0){\rule[-0.200pt]{4.818pt}{0.400pt}}
\put(110,547){\makebox(0,0)[r]{ 0.6}}
\put(1419.0,547.0){\rule[-0.200pt]{4.818pt}{0.400pt}}
\put(130.0,699.0){\rule[-0.200pt]{4.818pt}{0.400pt}}
\put(110,699){\makebox(0,0)[r]{ 0.8}}
\put(1419.0,699.0){\rule[-0.200pt]{4.818pt}{0.400pt}}
\put(130.0,851.0){\rule[-0.200pt]{4.818pt}{0.400pt}}
\put(110,851){\makebox(0,0)[r]{ 1}}
\put(1419.0,851.0){\rule[-0.200pt]{4.818pt}{0.400pt}}
\put(130.0,82.0){\rule[-0.200pt]{0.400pt}{4.818pt}}
\put(130,41){\makebox(0,0){ 0}}
\put(130.0,839.0){\rule[-0.200pt]{0.400pt}{4.818pt}}
\put(392.0,82.0){\rule[-0.200pt]{0.400pt}{4.818pt}}
\put(392,41){\makebox(0,0){ 0.2}}
\put(392.0,839.0){\rule[-0.200pt]{0.400pt}{4.818pt}}
\put(654.0,82.0){\rule[-0.200pt]{0.400pt}{4.818pt}}
\put(654,41){\makebox(0,0){ 0.4}}
\put(654.0,839.0){\rule[-0.200pt]{0.400pt}{4.818pt}}
\put(915.0,82.0){\rule[-0.200pt]{0.400pt}{4.818pt}}
\put(915,41){\makebox(0,0){ 0.6}}
\put(915.0,839.0){\rule[-0.200pt]{0.400pt}{4.818pt}}
\put(1177.0,82.0){\rule[-0.200pt]{0.400pt}{4.818pt}}
\put(1177,41){\makebox(0,0){ 0.8}}
\put(1177.0,839.0){\rule[-0.200pt]{0.400pt}{4.818pt}}
\put(1439.0,82.0){\rule[-0.200pt]{0.400pt}{4.818pt}}
\put(1439,41){\makebox(0,0){ 1}}
\put(1439.0,839.0){\rule[-0.200pt]{0.400pt}{4.818pt}}
\put(130.0,82.0){\rule[-0.200pt]{0.400pt}{187.179pt}}
\put(130.0,82.0){\rule[-0.200pt]{315.338pt}{0.400pt}}
\put(1439.0,82.0){\rule[-0.200pt]{0.400pt}{187.179pt}}
\put(130.0,859.0){\rule[-0.200pt]{315.338pt}{0.400pt}}
\put(1279,819){\makebox(0,0)[r]{'-'}}
\put(1299.0,819.0){\rule[-0.200pt]{24.090pt}{0.400pt}}
\put(1309,509){\usebox{\plotpoint}}
\put(1309,509.17){\rule{0.700pt}{0.400pt}}
\multiput(1309.00,508.17)(1.547,2.000){2}{\rule{0.350pt}{0.400pt}}
\put(1312,511.17){\rule{0.700pt}{0.400pt}}
\multiput(1312.00,510.17)(1.547,2.000){2}{\rule{0.350pt}{0.400pt}}
\put(1315,513.17){\rule{0.900pt}{0.400pt}}
\multiput(1315.00,512.17)(2.132,2.000){2}{\rule{0.450pt}{0.400pt}}
\put(1319,515.17){\rule{0.700pt}{0.400pt}}
\multiput(1319.00,514.17)(1.547,2.000){2}{\rule{0.350pt}{0.400pt}}
\put(1322,517.17){\rule{0.900pt}{0.400pt}}
\multiput(1322.00,516.17)(2.132,2.000){2}{\rule{0.450pt}{0.400pt}}
\put(1326,519.17){\rule{0.700pt}{0.400pt}}
\multiput(1326.00,518.17)(1.547,2.000){2}{\rule{0.350pt}{0.400pt}}
\put(1329,521.17){\rule{0.900pt}{0.400pt}}
\multiput(1329.00,520.17)(2.132,2.000){2}{\rule{0.450pt}{0.400pt}}
\put(1333,523.17){\rule{0.700pt}{0.400pt}}
\multiput(1333.00,522.17)(1.547,2.000){2}{\rule{0.350pt}{0.400pt}}
\put(1336,525.17){\rule{0.700pt}{0.400pt}}
\multiput(1336.00,524.17)(1.547,2.000){2}{\rule{0.350pt}{0.400pt}}
\put(1339,527.17){\rule{0.900pt}{0.400pt}}
\multiput(1339.00,526.17)(2.132,2.000){2}{\rule{0.450pt}{0.400pt}}
\put(1343,529.17){\rule{0.700pt}{0.400pt}}
\multiput(1343.00,528.17)(1.547,2.000){2}{\rule{0.350pt}{0.400pt}}
\put(1346,531.17){\rule{0.900pt}{0.400pt}}
\multiput(1346.00,530.17)(2.132,2.000){2}{\rule{0.450pt}{0.400pt}}
\put(1350,533.17){\rule{0.700pt}{0.400pt}}
\multiput(1350.00,532.17)(1.547,2.000){2}{\rule{0.350pt}{0.400pt}}
\put(1353,535.17){\rule{0.700pt}{0.400pt}}
\multiput(1353.00,534.17)(1.547,2.000){2}{\rule{0.350pt}{0.400pt}}
\put(1356,537.17){\rule{0.900pt}{0.400pt}}
\multiput(1356.00,536.17)(2.132,2.000){2}{\rule{0.450pt}{0.400pt}}
\put(1360,539.17){\rule{0.700pt}{0.400pt}}
\multiput(1360.00,538.17)(1.547,2.000){2}{\rule{0.350pt}{0.400pt}}
\put(1363,541.17){\rule{0.900pt}{0.400pt}}
\multiput(1363.00,540.17)(2.132,2.000){2}{\rule{0.450pt}{0.400pt}}
\put(1367,543.17){\rule{0.700pt}{0.400pt}}
\multiput(1367.00,542.17)(1.547,2.000){2}{\rule{0.350pt}{0.400pt}}
\put(1370,545.17){\rule{0.700pt}{0.400pt}}
\multiput(1370.00,544.17)(1.547,2.000){2}{\rule{0.350pt}{0.400pt}}
\put(1373,547.17){\rule{0.900pt}{0.400pt}}
\multiput(1373.00,546.17)(2.132,2.000){2}{\rule{0.450pt}{0.400pt}}
\put(1377,549.17){\rule{0.700pt}{0.400pt}}
\multiput(1377.00,548.17)(1.547,2.000){2}{\rule{0.350pt}{0.400pt}}
\put(1380,551.17){\rule{0.900pt}{0.400pt}}
\multiput(1380.00,550.17)(2.132,2.000){2}{\rule{0.450pt}{0.400pt}}
\put(1384,553.17){\rule{0.700pt}{0.400pt}}
\multiput(1384.00,552.17)(1.547,2.000){2}{\rule{0.350pt}{0.400pt}}
\put(1387,555.17){\rule{0.900pt}{0.400pt}}
\multiput(1387.00,554.17)(2.132,2.000){2}{\rule{0.450pt}{0.400pt}}
\put(1391,557.17){\rule{0.700pt}{0.400pt}}
\multiput(1391.00,556.17)(1.547,2.000){2}{\rule{0.350pt}{0.400pt}}
\put(1394,559.17){\rule{0.700pt}{0.400pt}}
\multiput(1394.00,558.17)(1.547,2.000){2}{\rule{0.350pt}{0.400pt}}
\put(1397,561.17){\rule{0.900pt}{0.400pt}}
\multiput(1397.00,560.17)(2.132,2.000){2}{\rule{0.450pt}{0.400pt}}
\put(1401,563.17){\rule{0.700pt}{0.400pt}}
\multiput(1401.00,562.17)(1.547,2.000){2}{\rule{0.350pt}{0.400pt}}
\put(1404,565.17){\rule{0.900pt}{0.400pt}}
\multiput(1404.00,564.17)(2.132,2.000){2}{\rule{0.450pt}{0.400pt}}
\put(1408,567.17){\rule{0.700pt}{0.400pt}}
\multiput(1408.00,566.17)(1.547,2.000){2}{\rule{0.350pt}{0.400pt}}
\put(1411,569.17){\rule{0.700pt}{0.400pt}}
\multiput(1411.00,568.17)(1.547,2.000){2}{\rule{0.350pt}{0.400pt}}
\put(1411,571.17){\rule{0.700pt}{0.400pt}}
\multiput(1412.55,570.17)(-1.547,2.000){2}{\rule{0.350pt}{0.400pt}}
\put(1408,573.17){\rule{0.700pt}{0.400pt}}
\multiput(1409.55,572.17)(-1.547,2.000){2}{\rule{0.350pt}{0.400pt}}
\put(1404,575.17){\rule{0.900pt}{0.400pt}}
\multiput(1406.13,574.17)(-2.132,2.000){2}{\rule{0.450pt}{0.400pt}}
\put(1401,577.17){\rule{0.700pt}{0.400pt}}
\multiput(1402.55,576.17)(-1.547,2.000){2}{\rule{0.350pt}{0.400pt}}
\put(1397,579.17){\rule{0.900pt}{0.400pt}}
\multiput(1399.13,578.17)(-2.132,2.000){2}{\rule{0.450pt}{0.400pt}}
\put(1394,581.17){\rule{0.700pt}{0.400pt}}
\multiput(1395.55,580.17)(-1.547,2.000){2}{\rule{0.350pt}{0.400pt}}
\put(1391,583.17){\rule{0.700pt}{0.400pt}}
\multiput(1392.55,582.17)(-1.547,2.000){2}{\rule{0.350pt}{0.400pt}}
\put(1387,585.17){\rule{0.900pt}{0.400pt}}
\multiput(1389.13,584.17)(-2.132,2.000){2}{\rule{0.450pt}{0.400pt}}
\put(1384,587.17){\rule{0.700pt}{0.400pt}}
\multiput(1385.55,586.17)(-1.547,2.000){2}{\rule{0.350pt}{0.400pt}}
\put(1380,589.17){\rule{0.900pt}{0.400pt}}
\multiput(1382.13,588.17)(-2.132,2.000){2}{\rule{0.450pt}{0.400pt}}
\put(1377,591.17){\rule{0.700pt}{0.400pt}}
\multiput(1378.55,590.17)(-1.547,2.000){2}{\rule{0.350pt}{0.400pt}}
\put(1373,593.17){\rule{0.900pt}{0.400pt}}
\multiput(1375.13,592.17)(-2.132,2.000){2}{\rule{0.450pt}{0.400pt}}
\put(1370,595.17){\rule{0.700pt}{0.400pt}}
\multiput(1371.55,594.17)(-1.547,2.000){2}{\rule{0.350pt}{0.400pt}}
\put(1367,597.17){\rule{0.700pt}{0.400pt}}
\multiput(1368.55,596.17)(-1.547,2.000){2}{\rule{0.350pt}{0.400pt}}
\put(1363,599.17){\rule{0.900pt}{0.400pt}}
\multiput(1365.13,598.17)(-2.132,2.000){2}{\rule{0.450pt}{0.400pt}}
\put(1360,601.17){\rule{0.700pt}{0.400pt}}
\multiput(1361.55,600.17)(-1.547,2.000){2}{\rule{0.350pt}{0.400pt}}
\put(1356,603.17){\rule{0.900pt}{0.400pt}}
\multiput(1358.13,602.17)(-2.132,2.000){2}{\rule{0.450pt}{0.400pt}}
\put(1353,605.17){\rule{0.700pt}{0.400pt}}
\multiput(1354.55,604.17)(-1.547,2.000){2}{\rule{0.350pt}{0.400pt}}
\put(1350,607.17){\rule{0.700pt}{0.400pt}}
\multiput(1351.55,606.17)(-1.547,2.000){2}{\rule{0.350pt}{0.400pt}}
\put(1346,609.17){\rule{0.900pt}{0.400pt}}
\multiput(1348.13,608.17)(-2.132,2.000){2}{\rule{0.450pt}{0.400pt}}
\put(1343,611.17){\rule{0.700pt}{0.400pt}}
\multiput(1344.55,610.17)(-1.547,2.000){2}{\rule{0.350pt}{0.400pt}}
\put(1339,613.17){\rule{0.900pt}{0.400pt}}
\multiput(1341.13,612.17)(-2.132,2.000){2}{\rule{0.450pt}{0.400pt}}
\put(1336,615.17){\rule{0.700pt}{0.400pt}}
\multiput(1337.55,614.17)(-1.547,2.000){2}{\rule{0.350pt}{0.400pt}}
\put(1333,617.17){\rule{0.700pt}{0.400pt}}
\multiput(1334.55,616.17)(-1.547,2.000){2}{\rule{0.350pt}{0.400pt}}
\put(1329,619.17){\rule{0.900pt}{0.400pt}}
\multiput(1331.13,618.17)(-2.132,2.000){2}{\rule{0.450pt}{0.400pt}}
\put(1326,621.17){\rule{0.700pt}{0.400pt}}
\multiput(1327.55,620.17)(-1.547,2.000){2}{\rule{0.350pt}{0.400pt}}
\put(1322,623.17){\rule{0.900pt}{0.400pt}}
\multiput(1324.13,622.17)(-2.132,2.000){2}{\rule{0.450pt}{0.400pt}}
\put(1319,625.17){\rule{0.700pt}{0.400pt}}
\multiput(1320.55,624.17)(-1.547,2.000){2}{\rule{0.350pt}{0.400pt}}
\put(1315,627.17){\rule{0.900pt}{0.400pt}}
\multiput(1317.13,626.17)(-2.132,2.000){2}{\rule{0.450pt}{0.400pt}}
\put(1312,628.67){\rule{0.723pt}{0.400pt}}
\multiput(1313.50,628.17)(-1.500,1.000){2}{\rule{0.361pt}{0.400pt}}
\put(1309,630.17){\rule{0.700pt}{0.400pt}}
\multiput(1310.55,629.17)(-1.547,2.000){2}{\rule{0.350pt}{0.400pt}}
\put(1305,632.17){\rule{0.900pt}{0.400pt}}
\multiput(1307.13,631.17)(-2.132,2.000){2}{\rule{0.450pt}{0.400pt}}
\put(1302,634.17){\rule{0.700pt}{0.400pt}}
\multiput(1303.55,633.17)(-1.547,2.000){2}{\rule{0.350pt}{0.400pt}}
\put(1298,636.17){\rule{0.900pt}{0.400pt}}
\multiput(1300.13,635.17)(-2.132,2.000){2}{\rule{0.450pt}{0.400pt}}
\put(1295,638.17){\rule{0.700pt}{0.400pt}}
\multiput(1296.55,637.17)(-1.547,2.000){2}{\rule{0.350pt}{0.400pt}}
\put(1292,640.17){\rule{0.700pt}{0.400pt}}
\multiput(1293.55,639.17)(-1.547,2.000){2}{\rule{0.350pt}{0.400pt}}
\put(1288,642.17){\rule{0.900pt}{0.400pt}}
\multiput(1290.13,641.17)(-2.132,2.000){2}{\rule{0.450pt}{0.400pt}}
\put(1285,644.17){\rule{0.700pt}{0.400pt}}
\multiput(1286.55,643.17)(-1.547,2.000){2}{\rule{0.350pt}{0.400pt}}
\put(1281,646.17){\rule{0.900pt}{0.400pt}}
\multiput(1283.13,645.17)(-2.132,2.000){2}{\rule{0.450pt}{0.400pt}}
\put(1278,648.17){\rule{0.700pt}{0.400pt}}
\multiput(1279.55,647.17)(-1.547,2.000){2}{\rule{0.350pt}{0.400pt}}
\put(1275,650.17){\rule{0.700pt}{0.400pt}}
\multiput(1276.55,649.17)(-1.547,2.000){2}{\rule{0.350pt}{0.400pt}}
\put(1271,652.17){\rule{0.900pt}{0.400pt}}
\multiput(1273.13,651.17)(-2.132,2.000){2}{\rule{0.450pt}{0.400pt}}
\put(1268,654.17){\rule{0.700pt}{0.400pt}}
\multiput(1269.55,653.17)(-1.547,2.000){2}{\rule{0.350pt}{0.400pt}}
\put(1264,656.17){\rule{0.900pt}{0.400pt}}
\multiput(1266.13,655.17)(-2.132,2.000){2}{\rule{0.450pt}{0.400pt}}
\put(1261,658.17){\rule{0.700pt}{0.400pt}}
\multiput(1262.55,657.17)(-1.547,2.000){2}{\rule{0.350pt}{0.400pt}}
\put(1258,660.17){\rule{0.700pt}{0.400pt}}
\multiput(1259.55,659.17)(-1.547,2.000){2}{\rule{0.350pt}{0.400pt}}
\put(1254,662.17){\rule{0.900pt}{0.400pt}}
\multiput(1256.13,661.17)(-2.132,2.000){2}{\rule{0.450pt}{0.400pt}}
\put(1251,664.17){\rule{0.700pt}{0.400pt}}
\multiput(1252.55,663.17)(-1.547,2.000){2}{\rule{0.350pt}{0.400pt}}
\put(1247,666.17){\rule{0.900pt}{0.400pt}}
\multiput(1249.13,665.17)(-2.132,2.000){2}{\rule{0.450pt}{0.400pt}}
\put(1244,668.17){\rule{0.700pt}{0.400pt}}
\multiput(1245.55,667.17)(-1.547,2.000){2}{\rule{0.350pt}{0.400pt}}
\put(1240,670.17){\rule{0.900pt}{0.400pt}}
\multiput(1242.13,669.17)(-2.132,2.000){2}{\rule{0.450pt}{0.400pt}}
\put(1237,672.17){\rule{0.700pt}{0.400pt}}
\multiput(1238.55,671.17)(-1.547,2.000){2}{\rule{0.350pt}{0.400pt}}
\put(1234,674.17){\rule{0.700pt}{0.400pt}}
\multiput(1235.55,673.17)(-1.547,2.000){2}{\rule{0.350pt}{0.400pt}}
\put(1230,676.17){\rule{0.900pt}{0.400pt}}
\multiput(1232.13,675.17)(-2.132,2.000){2}{\rule{0.450pt}{0.400pt}}
\put(1227,678.17){\rule{0.700pt}{0.400pt}}
\multiput(1228.55,677.17)(-1.547,2.000){2}{\rule{0.350pt}{0.400pt}}
\put(1223,680.17){\rule{0.900pt}{0.400pt}}
\multiput(1225.13,679.17)(-2.132,2.000){2}{\rule{0.450pt}{0.400pt}}
\put(1220,682.17){\rule{0.700pt}{0.400pt}}
\multiput(1221.55,681.17)(-1.547,2.000){2}{\rule{0.350pt}{0.400pt}}
\put(1217,684.17){\rule{0.700pt}{0.400pt}}
\multiput(1218.55,683.17)(-1.547,2.000){2}{\rule{0.350pt}{0.400pt}}
\put(1213,686.17){\rule{0.900pt}{0.400pt}}
\multiput(1215.13,685.17)(-2.132,2.000){2}{\rule{0.450pt}{0.400pt}}
\put(1210,688.17){\rule{0.700pt}{0.400pt}}
\multiput(1211.55,687.17)(-1.547,2.000){2}{\rule{0.350pt}{0.400pt}}
\put(1206,690.17){\rule{0.900pt}{0.400pt}}
\multiput(1208.13,689.17)(-2.132,2.000){2}{\rule{0.450pt}{0.400pt}}
\put(1203,692.17){\rule{0.700pt}{0.400pt}}
\multiput(1204.55,691.17)(-1.547,2.000){2}{\rule{0.350pt}{0.400pt}}
\put(1200,694.17){\rule{0.700pt}{0.400pt}}
\multiput(1201.55,693.17)(-1.547,2.000){2}{\rule{0.350pt}{0.400pt}}
\put(1196,696.17){\rule{0.900pt}{0.400pt}}
\multiput(1198.13,695.17)(-2.132,2.000){2}{\rule{0.450pt}{0.400pt}}
\put(1193,698.17){\rule{0.700pt}{0.400pt}}
\multiput(1194.55,697.17)(-1.547,2.000){2}{\rule{0.350pt}{0.400pt}}
\put(1189,700.17){\rule{0.900pt}{0.400pt}}
\multiput(1191.13,699.17)(-2.132,2.000){2}{\rule{0.450pt}{0.400pt}}
\put(1186,702.17){\rule{0.700pt}{0.400pt}}
\multiput(1187.55,701.17)(-1.547,2.000){2}{\rule{0.350pt}{0.400pt}}
\put(1183,704.17){\rule{0.700pt}{0.400pt}}
\multiput(1184.55,703.17)(-1.547,2.000){2}{\rule{0.350pt}{0.400pt}}
\put(1179,706.17){\rule{0.900pt}{0.400pt}}
\multiput(1181.13,705.17)(-2.132,2.000){2}{\rule{0.450pt}{0.400pt}}
\put(1176,708.17){\rule{0.700pt}{0.400pt}}
\multiput(1177.55,707.17)(-1.547,2.000){2}{\rule{0.350pt}{0.400pt}}
\put(1172,710.17){\rule{0.900pt}{0.400pt}}
\multiput(1174.13,709.17)(-2.132,2.000){2}{\rule{0.450pt}{0.400pt}}
\put(1169,712.17){\rule{0.700pt}{0.400pt}}
\multiput(1170.55,711.17)(-1.547,2.000){2}{\rule{0.350pt}{0.400pt}}
\put(1165,714.17){\rule{0.900pt}{0.400pt}}
\multiput(1167.13,713.17)(-2.132,2.000){2}{\rule{0.450pt}{0.400pt}}
\put(1162,716.17){\rule{0.700pt}{0.400pt}}
\multiput(1163.55,715.17)(-1.547,2.000){2}{\rule{0.350pt}{0.400pt}}
\put(1159,718.17){\rule{0.700pt}{0.400pt}}
\multiput(1160.55,717.17)(-1.547,2.000){2}{\rule{0.350pt}{0.400pt}}
\put(1155,720.17){\rule{0.900pt}{0.400pt}}
\multiput(1157.13,719.17)(-2.132,2.000){2}{\rule{0.450pt}{0.400pt}}
\put(1152,722.17){\rule{0.700pt}{0.400pt}}
\multiput(1153.55,721.17)(-1.547,2.000){2}{\rule{0.350pt}{0.400pt}}
\put(1148,724.17){\rule{0.900pt}{0.400pt}}
\multiput(1150.13,723.17)(-2.132,2.000){2}{\rule{0.450pt}{0.400pt}}
\put(1145,726.17){\rule{0.700pt}{0.400pt}}
\multiput(1146.55,725.17)(-1.547,2.000){2}{\rule{0.350pt}{0.400pt}}
\put(1142,728.17){\rule{0.700pt}{0.400pt}}
\multiput(1143.55,727.17)(-1.547,2.000){2}{\rule{0.350pt}{0.400pt}}
\put(1138,730.17){\rule{0.900pt}{0.400pt}}
\multiput(1140.13,729.17)(-2.132,2.000){2}{\rule{0.450pt}{0.400pt}}
\put(1135,732.17){\rule{0.700pt}{0.400pt}}
\multiput(1136.55,731.17)(-1.547,2.000){2}{\rule{0.350pt}{0.400pt}}
\put(1131,734.17){\rule{0.900pt}{0.400pt}}
\multiput(1133.13,733.17)(-2.132,2.000){2}{\rule{0.450pt}{0.400pt}}
\put(1128,736.17){\rule{0.700pt}{0.400pt}}
\multiput(1129.55,735.17)(-1.547,2.000){2}{\rule{0.350pt}{0.400pt}}
\put(1125,738.17){\rule{0.700pt}{0.400pt}}
\multiput(1126.55,737.17)(-1.547,2.000){2}{\rule{0.350pt}{0.400pt}}
\put(1121,740.17){\rule{0.900pt}{0.400pt}}
\multiput(1123.13,739.17)(-2.132,2.000){2}{\rule{0.450pt}{0.400pt}}
\put(1118,742.17){\rule{0.700pt}{0.400pt}}
\multiput(1119.55,741.17)(-1.547,2.000){2}{\rule{0.350pt}{0.400pt}}
\put(1114,744.17){\rule{0.900pt}{0.400pt}}
\multiput(1116.13,743.17)(-2.132,2.000){2}{\rule{0.450pt}{0.400pt}}
\put(1111,746.17){\rule{0.700pt}{0.400pt}}
\multiput(1112.55,745.17)(-1.547,2.000){2}{\rule{0.350pt}{0.400pt}}
\put(1108,748.17){\rule{0.700pt}{0.400pt}}
\multiput(1109.55,747.17)(-1.547,2.000){2}{\rule{0.350pt}{0.400pt}}
\put(1104,750.17){\rule{0.900pt}{0.400pt}}
\multiput(1106.13,749.17)(-2.132,2.000){2}{\rule{0.450pt}{0.400pt}}
\put(1101,752.17){\rule{0.700pt}{0.400pt}}
\multiput(1102.55,751.17)(-1.547,2.000){2}{\rule{0.350pt}{0.400pt}}
\put(1097,754.17){\rule{0.900pt}{0.400pt}}
\multiput(1099.13,753.17)(-2.132,2.000){2}{\rule{0.450pt}{0.400pt}}
\put(1094,756.17){\rule{0.700pt}{0.400pt}}
\multiput(1095.55,755.17)(-1.547,2.000){2}{\rule{0.350pt}{0.400pt}}
\put(1090,758.17){\rule{0.900pt}{0.400pt}}
\multiput(1092.13,757.17)(-2.132,2.000){2}{\rule{0.450pt}{0.400pt}}
\put(1087,760.17){\rule{0.700pt}{0.400pt}}
\multiput(1088.55,759.17)(-1.547,2.000){2}{\rule{0.350pt}{0.400pt}}
\put(1084,762.17){\rule{0.700pt}{0.400pt}}
\multiput(1085.55,761.17)(-1.547,2.000){2}{\rule{0.350pt}{0.400pt}}
\put(1080,764.17){\rule{0.900pt}{0.400pt}}
\multiput(1082.13,763.17)(-2.132,2.000){2}{\rule{0.450pt}{0.400pt}}
\put(1077,766.17){\rule{0.700pt}{0.400pt}}
\multiput(1078.55,765.17)(-1.547,2.000){2}{\rule{0.350pt}{0.400pt}}
\put(1073,768.17){\rule{0.900pt}{0.400pt}}
\multiput(1075.13,767.17)(-2.132,2.000){2}{\rule{0.450pt}{0.400pt}}
\put(1070,770.17){\rule{0.700pt}{0.400pt}}
\multiput(1071.55,769.17)(-1.547,2.000){2}{\rule{0.350pt}{0.400pt}}
\put(1067,772.17){\rule{0.700pt}{0.400pt}}
\multiput(1068.55,771.17)(-1.547,2.000){2}{\rule{0.350pt}{0.400pt}}
\put(1063,774.17){\rule{0.900pt}{0.400pt}}
\multiput(1065.13,773.17)(-2.132,2.000){2}{\rule{0.450pt}{0.400pt}}
\put(1060,776.17){\rule{0.700pt}{0.400pt}}
\multiput(1061.55,775.17)(-1.547,2.000){2}{\rule{0.350pt}{0.400pt}}
\put(1056,778.17){\rule{0.900pt}{0.400pt}}
\multiput(1058.13,777.17)(-2.132,2.000){2}{\rule{0.450pt}{0.400pt}}
\put(1053,780.17){\rule{0.700pt}{0.400pt}}
\multiput(1054.55,779.17)(-1.547,2.000){2}{\rule{0.350pt}{0.400pt}}
\put(1050,782.17){\rule{0.700pt}{0.400pt}}
\multiput(1051.55,781.17)(-1.547,2.000){2}{\rule{0.350pt}{0.400pt}}
\put(1046,784.17){\rule{0.900pt}{0.400pt}}
\multiput(1048.13,783.17)(-2.132,2.000){2}{\rule{0.450pt}{0.400pt}}
\put(1043,786.17){\rule{0.700pt}{0.400pt}}
\multiput(1044.55,785.17)(-1.547,2.000){2}{\rule{0.350pt}{0.400pt}}
\put(1039,787.67){\rule{0.964pt}{0.400pt}}
\multiput(1041.00,787.17)(-2.000,1.000){2}{\rule{0.482pt}{0.400pt}}
\put(1036,789.17){\rule{0.700pt}{0.400pt}}
\multiput(1037.55,788.17)(-1.547,2.000){2}{\rule{0.350pt}{0.400pt}}
\put(1032,791.17){\rule{0.900pt}{0.400pt}}
\multiput(1034.13,790.17)(-2.132,2.000){2}{\rule{0.450pt}{0.400pt}}
\put(1029,793.17){\rule{0.700pt}{0.400pt}}
\multiput(1030.55,792.17)(-1.547,2.000){2}{\rule{0.350pt}{0.400pt}}
\put(1026,795.17){\rule{0.700pt}{0.400pt}}
\multiput(1027.55,794.17)(-1.547,2.000){2}{\rule{0.350pt}{0.400pt}}
\put(1022,797.17){\rule{0.900pt}{0.400pt}}
\multiput(1024.13,796.17)(-2.132,2.000){2}{\rule{0.450pt}{0.400pt}}
\put(1019,799.17){\rule{0.700pt}{0.400pt}}
\multiput(1020.55,798.17)(-1.547,2.000){2}{\rule{0.350pt}{0.400pt}}
\put(1015,801.17){\rule{0.900pt}{0.400pt}}
\multiput(1017.13,800.17)(-2.132,2.000){2}{\rule{0.450pt}{0.400pt}}
\put(1012,803.17){\rule{0.700pt}{0.400pt}}
\multiput(1013.55,802.17)(-1.547,2.000){2}{\rule{0.350pt}{0.400pt}}
\put(1009,805.17){\rule{0.700pt}{0.400pt}}
\multiput(1010.55,804.17)(-1.547,2.000){2}{\rule{0.350pt}{0.400pt}}
\put(1005,807.17){\rule{0.900pt}{0.400pt}}
\multiput(1007.13,806.17)(-2.132,2.000){2}{\rule{0.450pt}{0.400pt}}
\put(1002,809.17){\rule{0.700pt}{0.400pt}}
\multiput(1003.55,808.17)(-1.547,2.000){2}{\rule{0.350pt}{0.400pt}}
\put(998,811.17){\rule{0.900pt}{0.400pt}}
\multiput(1000.13,810.17)(-2.132,2.000){2}{\rule{0.450pt}{0.400pt}}
\put(995,813.17){\rule{0.700pt}{0.400pt}}
\multiput(996.55,812.17)(-1.547,2.000){2}{\rule{0.350pt}{0.400pt}}
\put(992,815.17){\rule{0.700pt}{0.400pt}}
\multiput(993.55,814.17)(-1.547,2.000){2}{\rule{0.350pt}{0.400pt}}
\put(988,817.17){\rule{0.900pt}{0.400pt}}
\multiput(990.13,816.17)(-2.132,2.000){2}{\rule{0.450pt}{0.400pt}}
\put(985,819.17){\rule{0.700pt}{0.400pt}}
\multiput(986.55,818.17)(-1.547,2.000){2}{\rule{0.350pt}{0.400pt}}
\put(981,821.17){\rule{0.900pt}{0.400pt}}
\multiput(983.13,820.17)(-2.132,2.000){2}{\rule{0.450pt}{0.400pt}}
\put(978,823.17){\rule{0.700pt}{0.400pt}}
\multiput(979.55,822.17)(-1.547,2.000){2}{\rule{0.350pt}{0.400pt}}
\put(975,825.17){\rule{0.700pt}{0.400pt}}
\multiput(976.55,824.17)(-1.547,2.000){2}{\rule{0.350pt}{0.400pt}}
\put(971,827.17){\rule{0.900pt}{0.400pt}}
\multiput(973.13,826.17)(-2.132,2.000){2}{\rule{0.450pt}{0.400pt}}
\put(968,829.17){\rule{0.700pt}{0.400pt}}
\multiput(969.55,828.17)(-1.547,2.000){2}{\rule{0.350pt}{0.400pt}}
\put(964,831.17){\rule{0.900pt}{0.400pt}}
\multiput(966.13,830.17)(-2.132,2.000){2}{\rule{0.450pt}{0.400pt}}
\put(961,833.17){\rule{0.700pt}{0.400pt}}
\multiput(962.55,832.17)(-1.547,2.000){2}{\rule{0.350pt}{0.400pt}}
\put(957,835.17){\rule{0.900pt}{0.400pt}}
\multiput(959.13,834.17)(-2.132,2.000){2}{\rule{0.450pt}{0.400pt}}
\put(954,835.17){\rule{0.700pt}{0.400pt}}
\multiput(955.55,836.17)(-1.547,-2.000){2}{\rule{0.350pt}{0.400pt}}
\put(951,833.17){\rule{0.700pt}{0.400pt}}
\multiput(952.55,834.17)(-1.547,-2.000){2}{\rule{0.350pt}{0.400pt}}
\put(947,831.17){\rule{0.900pt}{0.400pt}}
\multiput(949.13,832.17)(-2.132,-2.000){2}{\rule{0.450pt}{0.400pt}}
\put(944,829.17){\rule{0.700pt}{0.400pt}}
\multiput(945.55,830.17)(-1.547,-2.000){2}{\rule{0.350pt}{0.400pt}}
\put(940,827.17){\rule{0.900pt}{0.400pt}}
\multiput(942.13,828.17)(-2.132,-2.000){2}{\rule{0.450pt}{0.400pt}}
\put(937,825.17){\rule{0.700pt}{0.400pt}}
\multiput(938.55,826.17)(-1.547,-2.000){2}{\rule{0.350pt}{0.400pt}}
\put(934,823.17){\rule{0.700pt}{0.400pt}}
\multiput(935.55,824.17)(-1.547,-2.000){2}{\rule{0.350pt}{0.400pt}}
\put(930,821.17){\rule{0.900pt}{0.400pt}}
\multiput(932.13,822.17)(-2.132,-2.000){2}{\rule{0.450pt}{0.400pt}}
\put(927,819.17){\rule{0.700pt}{0.400pt}}
\multiput(928.55,820.17)(-1.547,-2.000){2}{\rule{0.350pt}{0.400pt}}
\put(923,817.17){\rule{0.900pt}{0.400pt}}
\multiput(925.13,818.17)(-2.132,-2.000){2}{\rule{0.450pt}{0.400pt}}
\put(920,815.17){\rule{0.700pt}{0.400pt}}
\multiput(921.55,816.17)(-1.547,-2.000){2}{\rule{0.350pt}{0.400pt}}
\put(917,813.17){\rule{0.700pt}{0.400pt}}
\multiput(918.55,814.17)(-1.547,-2.000){2}{\rule{0.350pt}{0.400pt}}
\put(913,811.17){\rule{0.900pt}{0.400pt}}
\multiput(915.13,812.17)(-2.132,-2.000){2}{\rule{0.450pt}{0.400pt}}
\put(910,809.17){\rule{0.700pt}{0.400pt}}
\multiput(911.55,810.17)(-1.547,-2.000){2}{\rule{0.350pt}{0.400pt}}
\put(906,807.17){\rule{0.900pt}{0.400pt}}
\multiput(908.13,808.17)(-2.132,-2.000){2}{\rule{0.450pt}{0.400pt}}
\put(903,805.17){\rule{0.700pt}{0.400pt}}
\multiput(904.55,806.17)(-1.547,-2.000){2}{\rule{0.350pt}{0.400pt}}
\put(900,803.17){\rule{0.700pt}{0.400pt}}
\multiput(901.55,804.17)(-1.547,-2.000){2}{\rule{0.350pt}{0.400pt}}
\put(896,801.17){\rule{0.900pt}{0.400pt}}
\multiput(898.13,802.17)(-2.132,-2.000){2}{\rule{0.450pt}{0.400pt}}
\put(893,799.17){\rule{0.700pt}{0.400pt}}
\multiput(894.55,800.17)(-1.547,-2.000){2}{\rule{0.350pt}{0.400pt}}
\put(889,797.17){\rule{0.900pt}{0.400pt}}
\multiput(891.13,798.17)(-2.132,-2.000){2}{\rule{0.450pt}{0.400pt}}
\put(886,795.17){\rule{0.700pt}{0.400pt}}
\multiput(887.55,796.17)(-1.547,-2.000){2}{\rule{0.350pt}{0.400pt}}
\put(882,793.17){\rule{0.900pt}{0.400pt}}
\multiput(884.13,794.17)(-2.132,-2.000){2}{\rule{0.450pt}{0.400pt}}
\put(879,791.17){\rule{0.700pt}{0.400pt}}
\multiput(880.55,792.17)(-1.547,-2.000){2}{\rule{0.350pt}{0.400pt}}
\put(876,789.17){\rule{0.700pt}{0.400pt}}
\multiput(877.55,790.17)(-1.547,-2.000){2}{\rule{0.350pt}{0.400pt}}
\put(872,787.67){\rule{0.964pt}{0.400pt}}
\multiput(874.00,788.17)(-2.000,-1.000){2}{\rule{0.482pt}{0.400pt}}
\put(869,786.17){\rule{0.700pt}{0.400pt}}
\multiput(870.55,787.17)(-1.547,-2.000){2}{\rule{0.350pt}{0.400pt}}
\put(865,784.17){\rule{0.900pt}{0.400pt}}
\multiput(867.13,785.17)(-2.132,-2.000){2}{\rule{0.450pt}{0.400pt}}
\put(862,782.17){\rule{0.700pt}{0.400pt}}
\multiput(863.55,783.17)(-1.547,-2.000){2}{\rule{0.350pt}{0.400pt}}
\put(859,780.17){\rule{0.700pt}{0.400pt}}
\multiput(860.55,781.17)(-1.547,-2.000){2}{\rule{0.350pt}{0.400pt}}
\put(855,778.17){\rule{0.900pt}{0.400pt}}
\multiput(857.13,779.17)(-2.132,-2.000){2}{\rule{0.450pt}{0.400pt}}
\put(852,776.17){\rule{0.700pt}{0.400pt}}
\multiput(853.55,777.17)(-1.547,-2.000){2}{\rule{0.350pt}{0.400pt}}
\put(848,774.17){\rule{0.900pt}{0.400pt}}
\multiput(850.13,775.17)(-2.132,-2.000){2}{\rule{0.450pt}{0.400pt}}
\put(845,772.17){\rule{0.700pt}{0.400pt}}
\multiput(846.55,773.17)(-1.547,-2.000){2}{\rule{0.350pt}{0.400pt}}
\put(842,770.17){\rule{0.700pt}{0.400pt}}
\multiput(843.55,771.17)(-1.547,-2.000){2}{\rule{0.350pt}{0.400pt}}
\put(838,768.17){\rule{0.900pt}{0.400pt}}
\multiput(840.13,769.17)(-2.132,-2.000){2}{\rule{0.450pt}{0.400pt}}
\put(835,766.17){\rule{0.700pt}{0.400pt}}
\multiput(836.55,767.17)(-1.547,-2.000){2}{\rule{0.350pt}{0.400pt}}
\put(831,764.17){\rule{0.900pt}{0.400pt}}
\multiput(833.13,765.17)(-2.132,-2.000){2}{\rule{0.450pt}{0.400pt}}
\put(828,762.17){\rule{0.700pt}{0.400pt}}
\multiput(829.55,763.17)(-1.547,-2.000){2}{\rule{0.350pt}{0.400pt}}
\put(825,760.17){\rule{0.700pt}{0.400pt}}
\multiput(826.55,761.17)(-1.547,-2.000){2}{\rule{0.350pt}{0.400pt}}
\put(821,758.17){\rule{0.900pt}{0.400pt}}
\multiput(823.13,759.17)(-2.132,-2.000){2}{\rule{0.450pt}{0.400pt}}
\put(818,756.17){\rule{0.700pt}{0.400pt}}
\multiput(819.55,757.17)(-1.547,-2.000){2}{\rule{0.350pt}{0.400pt}}
\put(814,754.17){\rule{0.900pt}{0.400pt}}
\multiput(816.13,755.17)(-2.132,-2.000){2}{\rule{0.450pt}{0.400pt}}
\put(811,752.17){\rule{0.700pt}{0.400pt}}
\multiput(812.55,753.17)(-1.547,-2.000){2}{\rule{0.350pt}{0.400pt}}
\put(807,750.17){\rule{0.900pt}{0.400pt}}
\multiput(809.13,751.17)(-2.132,-2.000){2}{\rule{0.450pt}{0.400pt}}
\put(804,748.17){\rule{0.700pt}{0.400pt}}
\multiput(805.55,749.17)(-1.547,-2.000){2}{\rule{0.350pt}{0.400pt}}
\put(801,746.17){\rule{0.700pt}{0.400pt}}
\multiput(802.55,747.17)(-1.547,-2.000){2}{\rule{0.350pt}{0.400pt}}
\put(797,744.17){\rule{0.900pt}{0.400pt}}
\multiput(799.13,745.17)(-2.132,-2.000){2}{\rule{0.450pt}{0.400pt}}
\put(794,742.17){\rule{0.700pt}{0.400pt}}
\multiput(795.55,743.17)(-1.547,-2.000){2}{\rule{0.350pt}{0.400pt}}
\put(790,740.17){\rule{0.900pt}{0.400pt}}
\multiput(792.13,741.17)(-2.132,-2.000){2}{\rule{0.450pt}{0.400pt}}
\put(787,738.17){\rule{0.700pt}{0.400pt}}
\multiput(788.55,739.17)(-1.547,-2.000){2}{\rule{0.350pt}{0.400pt}}
\put(784,736.17){\rule{0.700pt}{0.400pt}}
\multiput(785.55,737.17)(-1.547,-2.000){2}{\rule{0.350pt}{0.400pt}}
\put(780,734.17){\rule{0.900pt}{0.400pt}}
\multiput(782.13,735.17)(-2.132,-2.000){2}{\rule{0.450pt}{0.400pt}}
\put(777,732.17){\rule{0.700pt}{0.400pt}}
\multiput(778.55,733.17)(-1.547,-2.000){2}{\rule{0.350pt}{0.400pt}}
\put(773,730.17){\rule{0.900pt}{0.400pt}}
\multiput(775.13,731.17)(-2.132,-2.000){2}{\rule{0.450pt}{0.400pt}}
\put(770,728.17){\rule{0.700pt}{0.400pt}}
\multiput(771.55,729.17)(-1.547,-2.000){2}{\rule{0.350pt}{0.400pt}}
\put(767,726.17){\rule{0.700pt}{0.400pt}}
\multiput(768.55,727.17)(-1.547,-2.000){2}{\rule{0.350pt}{0.400pt}}
\put(763,724.17){\rule{0.900pt}{0.400pt}}
\multiput(765.13,725.17)(-2.132,-2.000){2}{\rule{0.450pt}{0.400pt}}
\put(760,722.17){\rule{0.700pt}{0.400pt}}
\multiput(761.55,723.17)(-1.547,-2.000){2}{\rule{0.350pt}{0.400pt}}
\put(756,720.17){\rule{0.900pt}{0.400pt}}
\multiput(758.13,721.17)(-2.132,-2.000){2}{\rule{0.450pt}{0.400pt}}
\put(753,718.17){\rule{0.700pt}{0.400pt}}
\multiput(754.55,719.17)(-1.547,-2.000){2}{\rule{0.350pt}{0.400pt}}
\put(749,716.17){\rule{0.900pt}{0.400pt}}
\multiput(751.13,717.17)(-2.132,-2.000){2}{\rule{0.450pt}{0.400pt}}
\put(746,714.17){\rule{0.700pt}{0.400pt}}
\multiput(747.55,715.17)(-1.547,-2.000){2}{\rule{0.350pt}{0.400pt}}
\put(743,712.17){\rule{0.700pt}{0.400pt}}
\multiput(744.55,713.17)(-1.547,-2.000){2}{\rule{0.350pt}{0.400pt}}
\put(739,710.17){\rule{0.900pt}{0.400pt}}
\multiput(741.13,711.17)(-2.132,-2.000){2}{\rule{0.450pt}{0.400pt}}
\put(736,708.17){\rule{0.700pt}{0.400pt}}
\multiput(737.55,709.17)(-1.547,-2.000){2}{\rule{0.350pt}{0.400pt}}
\put(732,706.17){\rule{0.900pt}{0.400pt}}
\multiput(734.13,707.17)(-2.132,-2.000){2}{\rule{0.450pt}{0.400pt}}
\put(729,704.17){\rule{0.700pt}{0.400pt}}
\multiput(730.55,705.17)(-1.547,-2.000){2}{\rule{0.350pt}{0.400pt}}
\put(726,702.17){\rule{0.700pt}{0.400pt}}
\multiput(727.55,703.17)(-1.547,-2.000){2}{\rule{0.350pt}{0.400pt}}
\put(722,700.17){\rule{0.900pt}{0.400pt}}
\multiput(724.13,701.17)(-2.132,-2.000){2}{\rule{0.450pt}{0.400pt}}
\put(719,698.17){\rule{0.700pt}{0.400pt}}
\multiput(720.55,699.17)(-1.547,-2.000){2}{\rule{0.350pt}{0.400pt}}
\put(715,696.17){\rule{0.900pt}{0.400pt}}
\multiput(717.13,697.17)(-2.132,-2.000){2}{\rule{0.450pt}{0.400pt}}
\put(712,694.17){\rule{0.700pt}{0.400pt}}
\multiput(713.55,695.17)(-1.547,-2.000){2}{\rule{0.350pt}{0.400pt}}
\put(709,692.17){\rule{0.700pt}{0.400pt}}
\multiput(710.55,693.17)(-1.547,-2.000){2}{\rule{0.350pt}{0.400pt}}
\put(705,690.17){\rule{0.900pt}{0.400pt}}
\multiput(707.13,691.17)(-2.132,-2.000){2}{\rule{0.450pt}{0.400pt}}
\put(702,688.17){\rule{0.700pt}{0.400pt}}
\multiput(703.55,689.17)(-1.547,-2.000){2}{\rule{0.350pt}{0.400pt}}
\put(698,686.17){\rule{0.900pt}{0.400pt}}
\multiput(700.13,687.17)(-2.132,-2.000){2}{\rule{0.450pt}{0.400pt}}
\put(695,684.17){\rule{0.700pt}{0.400pt}}
\multiput(696.55,685.17)(-1.547,-2.000){2}{\rule{0.350pt}{0.400pt}}
\put(692,682.17){\rule{0.700pt}{0.400pt}}
\multiput(693.55,683.17)(-1.547,-2.000){2}{\rule{0.350pt}{0.400pt}}
\put(688,680.17){\rule{0.900pt}{0.400pt}}
\multiput(690.13,681.17)(-2.132,-2.000){2}{\rule{0.450pt}{0.400pt}}
\put(685,678.17){\rule{0.700pt}{0.400pt}}
\multiput(686.55,679.17)(-1.547,-2.000){2}{\rule{0.350pt}{0.400pt}}
\put(681,676.17){\rule{0.900pt}{0.400pt}}
\multiput(683.13,677.17)(-2.132,-2.000){2}{\rule{0.450pt}{0.400pt}}
\put(678,674.17){\rule{0.700pt}{0.400pt}}
\multiput(679.55,675.17)(-1.547,-2.000){2}{\rule{0.350pt}{0.400pt}}
\put(674,672.17){\rule{0.900pt}{0.400pt}}
\multiput(676.13,673.17)(-2.132,-2.000){2}{\rule{0.450pt}{0.400pt}}
\put(671,670.17){\rule{0.700pt}{0.400pt}}
\multiput(672.55,671.17)(-1.547,-2.000){2}{\rule{0.350pt}{0.400pt}}
\put(668,668.17){\rule{0.700pt}{0.400pt}}
\multiput(669.55,669.17)(-1.547,-2.000){2}{\rule{0.350pt}{0.400pt}}
\put(664,666.17){\rule{0.900pt}{0.400pt}}
\multiput(666.13,667.17)(-2.132,-2.000){2}{\rule{0.450pt}{0.400pt}}
\put(661,664.17){\rule{0.700pt}{0.400pt}}
\multiput(662.55,665.17)(-1.547,-2.000){2}{\rule{0.350pt}{0.400pt}}
\put(657,662.17){\rule{0.900pt}{0.400pt}}
\multiput(659.13,663.17)(-2.132,-2.000){2}{\rule{0.450pt}{0.400pt}}
\put(654,660.17){\rule{0.700pt}{0.400pt}}
\multiput(655.55,661.17)(-1.547,-2.000){2}{\rule{0.350pt}{0.400pt}}
\put(651,658.17){\rule{0.700pt}{0.400pt}}
\multiput(652.55,659.17)(-1.547,-2.000){2}{\rule{0.350pt}{0.400pt}}
\put(647,656.17){\rule{0.900pt}{0.400pt}}
\multiput(649.13,657.17)(-2.132,-2.000){2}{\rule{0.450pt}{0.400pt}}
\put(644,654.17){\rule{0.700pt}{0.400pt}}
\multiput(645.55,655.17)(-1.547,-2.000){2}{\rule{0.350pt}{0.400pt}}
\put(640,652.17){\rule{0.900pt}{0.400pt}}
\multiput(642.13,653.17)(-2.132,-2.000){2}{\rule{0.450pt}{0.400pt}}
\put(637,650.17){\rule{0.700pt}{0.400pt}}
\multiput(638.55,651.17)(-1.547,-2.000){2}{\rule{0.350pt}{0.400pt}}
\put(634,648.17){\rule{0.700pt}{0.400pt}}
\multiput(635.55,649.17)(-1.547,-2.000){2}{\rule{0.350pt}{0.400pt}}
\put(630,646.17){\rule{0.900pt}{0.400pt}}
\multiput(632.13,647.17)(-2.132,-2.000){2}{\rule{0.450pt}{0.400pt}}
\put(627,644.17){\rule{0.700pt}{0.400pt}}
\multiput(628.55,645.17)(-1.547,-2.000){2}{\rule{0.350pt}{0.400pt}}
\put(623,642.17){\rule{0.900pt}{0.400pt}}
\multiput(625.13,643.17)(-2.132,-2.000){2}{\rule{0.450pt}{0.400pt}}
\put(620,640.17){\rule{0.700pt}{0.400pt}}
\multiput(621.55,641.17)(-1.547,-2.000){2}{\rule{0.350pt}{0.400pt}}
\put(617,638.17){\rule{0.700pt}{0.400pt}}
\multiput(618.55,639.17)(-1.547,-2.000){2}{\rule{0.350pt}{0.400pt}}
\put(613,636.17){\rule{0.900pt}{0.400pt}}
\multiput(615.13,637.17)(-2.132,-2.000){2}{\rule{0.450pt}{0.400pt}}
\put(610,634.17){\rule{0.700pt}{0.400pt}}
\multiput(611.55,635.17)(-1.547,-2.000){2}{\rule{0.350pt}{0.400pt}}
\put(606,632.17){\rule{0.900pt}{0.400pt}}
\multiput(608.13,633.17)(-2.132,-2.000){2}{\rule{0.450pt}{0.400pt}}
\put(603,630.17){\rule{0.700pt}{0.400pt}}
\multiput(604.55,631.17)(-1.547,-2.000){2}{\rule{0.350pt}{0.400pt}}
\put(599,628.67){\rule{0.964pt}{0.400pt}}
\multiput(601.00,629.17)(-2.000,-1.000){2}{\rule{0.482pt}{0.400pt}}
\put(596,627.17){\rule{0.700pt}{0.400pt}}
\multiput(597.55,628.17)(-1.547,-2.000){2}{\rule{0.350pt}{0.400pt}}
\put(593,625.17){\rule{0.700pt}{0.400pt}}
\multiput(594.55,626.17)(-1.547,-2.000){2}{\rule{0.350pt}{0.400pt}}
\put(589,623.17){\rule{0.900pt}{0.400pt}}
\multiput(591.13,624.17)(-2.132,-2.000){2}{\rule{0.450pt}{0.400pt}}
\put(586,621.17){\rule{0.700pt}{0.400pt}}
\multiput(587.55,622.17)(-1.547,-2.000){2}{\rule{0.350pt}{0.400pt}}
\put(582,619.17){\rule{0.900pt}{0.400pt}}
\multiput(584.13,620.17)(-2.132,-2.000){2}{\rule{0.450pt}{0.400pt}}
\put(579,617.17){\rule{0.700pt}{0.400pt}}
\multiput(580.55,618.17)(-1.547,-2.000){2}{\rule{0.350pt}{0.400pt}}
\put(576,615.17){\rule{0.700pt}{0.400pt}}
\multiput(577.55,616.17)(-1.547,-2.000){2}{\rule{0.350pt}{0.400pt}}
\put(572,613.17){\rule{0.900pt}{0.400pt}}
\multiput(574.13,614.17)(-2.132,-2.000){2}{\rule{0.450pt}{0.400pt}}
\put(569,611.17){\rule{0.700pt}{0.400pt}}
\multiput(570.55,612.17)(-1.547,-2.000){2}{\rule{0.350pt}{0.400pt}}
\put(565,609.17){\rule{0.900pt}{0.400pt}}
\multiput(567.13,610.17)(-2.132,-2.000){2}{\rule{0.450pt}{0.400pt}}
\put(562,607.17){\rule{0.700pt}{0.400pt}}
\multiput(563.55,608.17)(-1.547,-2.000){2}{\rule{0.350pt}{0.400pt}}
\put(559,605.17){\rule{0.700pt}{0.400pt}}
\multiput(560.55,606.17)(-1.547,-2.000){2}{\rule{0.350pt}{0.400pt}}
\put(555,603.17){\rule{0.900pt}{0.400pt}}
\multiput(557.13,604.17)(-2.132,-2.000){2}{\rule{0.450pt}{0.400pt}}
\put(552,601.17){\rule{0.700pt}{0.400pt}}
\multiput(553.55,602.17)(-1.547,-2.000){2}{\rule{0.350pt}{0.400pt}}
\put(548,599.17){\rule{0.900pt}{0.400pt}}
\multiput(550.13,600.17)(-2.132,-2.000){2}{\rule{0.450pt}{0.400pt}}
\put(545,597.17){\rule{0.700pt}{0.400pt}}
\multiput(546.55,598.17)(-1.547,-2.000){2}{\rule{0.350pt}{0.400pt}}
\put(542,595.17){\rule{0.700pt}{0.400pt}}
\multiput(543.55,596.17)(-1.547,-2.000){2}{\rule{0.350pt}{0.400pt}}
\put(538,593.17){\rule{0.900pt}{0.400pt}}
\multiput(540.13,594.17)(-2.132,-2.000){2}{\rule{0.450pt}{0.400pt}}
\put(535,591.17){\rule{0.700pt}{0.400pt}}
\multiput(536.55,592.17)(-1.547,-2.000){2}{\rule{0.350pt}{0.400pt}}
\put(531,589.17){\rule{0.900pt}{0.400pt}}
\multiput(533.13,590.17)(-2.132,-2.000){2}{\rule{0.450pt}{0.400pt}}
\put(528,587.17){\rule{0.700pt}{0.400pt}}
\multiput(529.55,588.17)(-1.547,-2.000){2}{\rule{0.350pt}{0.400pt}}
\put(524,585.17){\rule{0.900pt}{0.400pt}}
\multiput(526.13,586.17)(-2.132,-2.000){2}{\rule{0.450pt}{0.400pt}}
\put(521,583.17){\rule{0.700pt}{0.400pt}}
\multiput(522.55,584.17)(-1.547,-2.000){2}{\rule{0.350pt}{0.400pt}}
\put(518,581.17){\rule{0.700pt}{0.400pt}}
\multiput(519.55,582.17)(-1.547,-2.000){2}{\rule{0.350pt}{0.400pt}}
\put(514,579.17){\rule{0.900pt}{0.400pt}}
\multiput(516.13,580.17)(-2.132,-2.000){2}{\rule{0.450pt}{0.400pt}}
\put(511,577.17){\rule{0.700pt}{0.400pt}}
\multiput(512.55,578.17)(-1.547,-2.000){2}{\rule{0.350pt}{0.400pt}}
\put(507,575.17){\rule{0.900pt}{0.400pt}}
\multiput(509.13,576.17)(-2.132,-2.000){2}{\rule{0.450pt}{0.400pt}}
\put(504,573.17){\rule{0.700pt}{0.400pt}}
\multiput(505.55,574.17)(-1.547,-2.000){2}{\rule{0.350pt}{0.400pt}}
\put(501,571.17){\rule{0.700pt}{0.400pt}}
\multiput(502.55,572.17)(-1.547,-2.000){2}{\rule{0.350pt}{0.400pt}}
\put(497,569.17){\rule{0.900pt}{0.400pt}}
\multiput(499.13,570.17)(-2.132,-2.000){2}{\rule{0.450pt}{0.400pt}}
\put(494,567.17){\rule{0.700pt}{0.400pt}}
\multiput(495.55,568.17)(-1.547,-2.000){2}{\rule{0.350pt}{0.400pt}}
\put(490,565.17){\rule{0.900pt}{0.400pt}}
\multiput(492.13,566.17)(-2.132,-2.000){2}{\rule{0.450pt}{0.400pt}}
\put(487,563.17){\rule{0.700pt}{0.400pt}}
\multiput(488.55,564.17)(-1.547,-2.000){2}{\rule{0.350pt}{0.400pt}}
\put(484,561.17){\rule{0.700pt}{0.400pt}}
\multiput(485.55,562.17)(-1.547,-2.000){2}{\rule{0.350pt}{0.400pt}}
\put(480,559.17){\rule{0.900pt}{0.400pt}}
\multiput(482.13,560.17)(-2.132,-2.000){2}{\rule{0.450pt}{0.400pt}}
\put(477,557.17){\rule{0.700pt}{0.400pt}}
\multiput(478.55,558.17)(-1.547,-2.000){2}{\rule{0.350pt}{0.400pt}}
\put(473,555.17){\rule{0.900pt}{0.400pt}}
\multiput(475.13,556.17)(-2.132,-2.000){2}{\rule{0.450pt}{0.400pt}}
\put(470,553.17){\rule{0.700pt}{0.400pt}}
\multiput(471.55,554.17)(-1.547,-2.000){2}{\rule{0.350pt}{0.400pt}}
\put(466,551.17){\rule{0.900pt}{0.400pt}}
\multiput(468.13,552.17)(-2.132,-2.000){2}{\rule{0.450pt}{0.400pt}}
\put(463,549.17){\rule{0.700pt}{0.400pt}}
\multiput(464.55,550.17)(-1.547,-2.000){2}{\rule{0.350pt}{0.400pt}}
\put(460,547.17){\rule{0.700pt}{0.400pt}}
\multiput(461.55,548.17)(-1.547,-2.000){2}{\rule{0.350pt}{0.400pt}}
\put(456,545.17){\rule{0.900pt}{0.400pt}}
\multiput(458.13,546.17)(-2.132,-2.000){2}{\rule{0.450pt}{0.400pt}}
\put(453,543.17){\rule{0.700pt}{0.400pt}}
\multiput(454.55,544.17)(-1.547,-2.000){2}{\rule{0.350pt}{0.400pt}}
\put(449,541.17){\rule{0.900pt}{0.400pt}}
\multiput(451.13,542.17)(-2.132,-2.000){2}{\rule{0.450pt}{0.400pt}}
\put(446,539.17){\rule{0.700pt}{0.400pt}}
\multiput(447.55,540.17)(-1.547,-2.000){2}{\rule{0.350pt}{0.400pt}}
\put(443,537.17){\rule{0.700pt}{0.400pt}}
\multiput(444.55,538.17)(-1.547,-2.000){2}{\rule{0.350pt}{0.400pt}}
\put(439,535.17){\rule{0.900pt}{0.400pt}}
\multiput(441.13,536.17)(-2.132,-2.000){2}{\rule{0.450pt}{0.400pt}}
\put(436,533.17){\rule{0.700pt}{0.400pt}}
\multiput(437.55,534.17)(-1.547,-2.000){2}{\rule{0.350pt}{0.400pt}}
\put(432,531.17){\rule{0.900pt}{0.400pt}}
\multiput(434.13,532.17)(-2.132,-2.000){2}{\rule{0.450pt}{0.400pt}}
\put(429,529.17){\rule{0.700pt}{0.400pt}}
\multiput(430.55,530.17)(-1.547,-2.000){2}{\rule{0.350pt}{0.400pt}}
\put(426,527.17){\rule{0.700pt}{0.400pt}}
\multiput(427.55,528.17)(-1.547,-2.000){2}{\rule{0.350pt}{0.400pt}}
\put(422,525.17){\rule{0.900pt}{0.400pt}}
\multiput(424.13,526.17)(-2.132,-2.000){2}{\rule{0.450pt}{0.400pt}}
\put(419,523.17){\rule{0.700pt}{0.400pt}}
\multiput(420.55,524.17)(-1.547,-2.000){2}{\rule{0.350pt}{0.400pt}}
\put(415,521.17){\rule{0.900pt}{0.400pt}}
\multiput(417.13,522.17)(-2.132,-2.000){2}{\rule{0.450pt}{0.400pt}}
\put(412,519.17){\rule{0.700pt}{0.400pt}}
\multiput(413.55,520.17)(-1.547,-2.000){2}{\rule{0.350pt}{0.400pt}}
\put(409,517.17){\rule{0.700pt}{0.400pt}}
\multiput(410.55,518.17)(-1.547,-2.000){2}{\rule{0.350pt}{0.400pt}}
\put(405,515.17){\rule{0.900pt}{0.400pt}}
\multiput(407.13,516.17)(-2.132,-2.000){2}{\rule{0.450pt}{0.400pt}}
\put(402,513.17){\rule{0.700pt}{0.400pt}}
\multiput(403.55,514.17)(-1.547,-2.000){2}{\rule{0.350pt}{0.400pt}}
\put(398,511.17){\rule{0.900pt}{0.400pt}}
\multiput(400.13,512.17)(-2.132,-2.000){2}{\rule{0.450pt}{0.400pt}}
\put(395,509.17){\rule{0.700pt}{0.400pt}}
\multiput(396.55,510.17)(-1.547,-2.000){2}{\rule{0.350pt}{0.400pt}}
\put(391,507.17){\rule{0.900pt}{0.400pt}}
\multiput(393.13,508.17)(-2.132,-2.000){2}{\rule{0.450pt}{0.400pt}}
\put(388,505.17){\rule{0.700pt}{0.400pt}}
\multiput(389.55,506.17)(-1.547,-2.000){2}{\rule{0.350pt}{0.400pt}}
\put(385,503.17){\rule{0.700pt}{0.400pt}}
\multiput(386.55,504.17)(-1.547,-2.000){2}{\rule{0.350pt}{0.400pt}}
\put(381,501.17){\rule{0.900pt}{0.400pt}}
\multiput(383.13,502.17)(-2.132,-2.000){2}{\rule{0.450pt}{0.400pt}}
\put(378,499.17){\rule{0.700pt}{0.400pt}}
\multiput(379.55,500.17)(-1.547,-2.000){2}{\rule{0.350pt}{0.400pt}}
\put(374,497.17){\rule{0.900pt}{0.400pt}}
\multiput(376.13,498.17)(-2.132,-2.000){2}{\rule{0.450pt}{0.400pt}}
\put(371,495.17){\rule{0.700pt}{0.400pt}}
\multiput(372.55,496.17)(-1.547,-2.000){2}{\rule{0.350pt}{0.400pt}}
\put(368,493.17){\rule{0.700pt}{0.400pt}}
\multiput(369.55,494.17)(-1.547,-2.000){2}{\rule{0.350pt}{0.400pt}}
\put(364,491.17){\rule{0.900pt}{0.400pt}}
\multiput(366.13,492.17)(-2.132,-2.000){2}{\rule{0.450pt}{0.400pt}}
\put(361,489.17){\rule{0.700pt}{0.400pt}}
\multiput(362.55,490.17)(-1.547,-2.000){2}{\rule{0.350pt}{0.400pt}}
\put(357,487.17){\rule{0.900pt}{0.400pt}}
\multiput(359.13,488.17)(-2.132,-2.000){2}{\rule{0.450pt}{0.400pt}}
\put(354,485.17){\rule{0.700pt}{0.400pt}}
\multiput(355.55,486.17)(-1.547,-2.000){2}{\rule{0.350pt}{0.400pt}}
\put(351,483.17){\rule{0.700pt}{0.400pt}}
\multiput(352.55,484.17)(-1.547,-2.000){2}{\rule{0.350pt}{0.400pt}}
\put(347,481.17){\rule{0.900pt}{0.400pt}}
\multiput(349.13,482.17)(-2.132,-2.000){2}{\rule{0.450pt}{0.400pt}}
\put(344,479.17){\rule{0.700pt}{0.400pt}}
\multiput(345.55,480.17)(-1.547,-2.000){2}{\rule{0.350pt}{0.400pt}}
\put(340,477.17){\rule{0.900pt}{0.400pt}}
\multiput(342.13,478.17)(-2.132,-2.000){2}{\rule{0.450pt}{0.400pt}}
\put(337,475.17){\rule{0.700pt}{0.400pt}}
\multiput(338.55,476.17)(-1.547,-2.000){2}{\rule{0.350pt}{0.400pt}}
\put(334,473.17){\rule{0.700pt}{0.400pt}}
\multiput(335.55,474.17)(-1.547,-2.000){2}{\rule{0.350pt}{0.400pt}}
\put(330,471.17){\rule{0.900pt}{0.400pt}}
\multiput(332.13,472.17)(-2.132,-2.000){2}{\rule{0.450pt}{0.400pt}}
\put(327,469.67){\rule{0.723pt}{0.400pt}}
\multiput(328.50,470.17)(-1.500,-1.000){2}{\rule{0.361pt}{0.400pt}}
\put(323,468.17){\rule{0.900pt}{0.400pt}}
\multiput(325.13,469.17)(-2.132,-2.000){2}{\rule{0.450pt}{0.400pt}}
\put(320,466.17){\rule{0.700pt}{0.400pt}}
\multiput(321.55,467.17)(-1.547,-2.000){2}{\rule{0.350pt}{0.400pt}}
\put(316,464.17){\rule{0.900pt}{0.400pt}}
\multiput(318.13,465.17)(-2.132,-2.000){2}{\rule{0.450pt}{0.400pt}}
\put(313,462.17){\rule{0.700pt}{0.400pt}}
\multiput(314.55,463.17)(-1.547,-2.000){2}{\rule{0.350pt}{0.400pt}}
\put(310,460.17){\rule{0.700pt}{0.400pt}}
\multiput(311.55,461.17)(-1.547,-2.000){2}{\rule{0.350pt}{0.400pt}}
\put(306,458.17){\rule{0.900pt}{0.400pt}}
\multiput(308.13,459.17)(-2.132,-2.000){2}{\rule{0.450pt}{0.400pt}}
\put(303,456.17){\rule{0.700pt}{0.400pt}}
\multiput(304.55,457.17)(-1.547,-2.000){2}{\rule{0.350pt}{0.400pt}}
\put(299,454.17){\rule{0.900pt}{0.400pt}}
\multiput(301.13,455.17)(-2.132,-2.000){2}{\rule{0.450pt}{0.400pt}}
\put(296,452.17){\rule{0.700pt}{0.400pt}}
\multiput(297.55,453.17)(-1.547,-2.000){2}{\rule{0.350pt}{0.400pt}}
\put(293,450.17){\rule{0.700pt}{0.400pt}}
\multiput(294.55,451.17)(-1.547,-2.000){2}{\rule{0.350pt}{0.400pt}}
\put(289,448.17){\rule{0.900pt}{0.400pt}}
\multiput(291.13,449.17)(-2.132,-2.000){2}{\rule{0.450pt}{0.400pt}}
\put(286,446.17){\rule{0.700pt}{0.400pt}}
\multiput(287.55,447.17)(-1.547,-2.000){2}{\rule{0.350pt}{0.400pt}}
\put(282,444.17){\rule{0.900pt}{0.400pt}}
\multiput(284.13,445.17)(-2.132,-2.000){2}{\rule{0.450pt}{0.400pt}}
\put(279,442.17){\rule{0.700pt}{0.400pt}}
\multiput(280.55,443.17)(-1.547,-2.000){2}{\rule{0.350pt}{0.400pt}}
\put(276,440.17){\rule{0.700pt}{0.400pt}}
\multiput(277.55,441.17)(-1.547,-2.000){2}{\rule{0.350pt}{0.400pt}}
\put(272,438.17){\rule{0.900pt}{0.400pt}}
\multiput(274.13,439.17)(-2.132,-2.000){2}{\rule{0.450pt}{0.400pt}}
\put(269,436.17){\rule{0.700pt}{0.400pt}}
\multiput(270.55,437.17)(-1.547,-2.000){2}{\rule{0.350pt}{0.400pt}}
\put(265,434.17){\rule{0.900pt}{0.400pt}}
\multiput(267.13,435.17)(-2.132,-2.000){2}{\rule{0.450pt}{0.400pt}}
\put(262,432.17){\rule{0.700pt}{0.400pt}}
\multiput(263.55,433.17)(-1.547,-2.000){2}{\rule{0.350pt}{0.400pt}}
\put(259,430.17){\rule{0.700pt}{0.400pt}}
\multiput(260.55,431.17)(-1.547,-2.000){2}{\rule{0.350pt}{0.400pt}}
\put(255,428.17){\rule{0.900pt}{0.400pt}}
\multiput(257.13,429.17)(-2.132,-2.000){2}{\rule{0.450pt}{0.400pt}}
\put(252,426.17){\rule{0.700pt}{0.400pt}}
\multiput(253.55,427.17)(-1.547,-2.000){2}{\rule{0.350pt}{0.400pt}}
\put(248,424.17){\rule{0.900pt}{0.400pt}}
\multiput(250.13,425.17)(-2.132,-2.000){2}{\rule{0.450pt}{0.400pt}}
\put(245,422.17){\rule{0.700pt}{0.400pt}}
\multiput(246.55,423.17)(-1.547,-2.000){2}{\rule{0.350pt}{0.400pt}}
\put(241,420.17){\rule{0.900pt}{0.400pt}}
\multiput(243.13,421.17)(-2.132,-2.000){2}{\rule{0.450pt}{0.400pt}}
\put(238,418.17){\rule{0.700pt}{0.400pt}}
\multiput(239.55,419.17)(-1.547,-2.000){2}{\rule{0.350pt}{0.400pt}}
\put(235,416.17){\rule{0.700pt}{0.400pt}}
\multiput(236.55,417.17)(-1.547,-2.000){2}{\rule{0.350pt}{0.400pt}}
\put(231,414.17){\rule{0.900pt}{0.400pt}}
\multiput(233.13,415.17)(-2.132,-2.000){2}{\rule{0.450pt}{0.400pt}}
\put(228,412.17){\rule{0.700pt}{0.400pt}}
\multiput(229.55,413.17)(-1.547,-2.000){2}{\rule{0.350pt}{0.400pt}}
\put(224,410.17){\rule{0.900pt}{0.400pt}}
\multiput(226.13,411.17)(-2.132,-2.000){2}{\rule{0.450pt}{0.400pt}}
\put(221,408.17){\rule{0.700pt}{0.400pt}}
\multiput(222.55,409.17)(-1.547,-2.000){2}{\rule{0.350pt}{0.400pt}}
\put(218,406.17){\rule{0.700pt}{0.400pt}}
\multiput(219.55,407.17)(-1.547,-2.000){2}{\rule{0.350pt}{0.400pt}}
\put(214,404.17){\rule{0.900pt}{0.400pt}}
\multiput(216.13,405.17)(-2.132,-2.000){2}{\rule{0.450pt}{0.400pt}}
\put(211,402.17){\rule{0.700pt}{0.400pt}}
\multiput(212.55,403.17)(-1.547,-2.000){2}{\rule{0.350pt}{0.400pt}}
\put(207,400.17){\rule{0.900pt}{0.400pt}}
\multiput(209.13,401.17)(-2.132,-2.000){2}{\rule{0.450pt}{0.400pt}}
\put(204,398.17){\rule{0.700pt}{0.400pt}}
\multiput(205.55,399.17)(-1.547,-2.000){2}{\rule{0.350pt}{0.400pt}}
\put(201,396.17){\rule{0.700pt}{0.400pt}}
\multiput(202.55,397.17)(-1.547,-2.000){2}{\rule{0.350pt}{0.400pt}}
\put(197,394.17){\rule{0.900pt}{0.400pt}}
\multiput(199.13,395.17)(-2.132,-2.000){2}{\rule{0.450pt}{0.400pt}}
\put(194,392.17){\rule{0.700pt}{0.400pt}}
\multiput(195.55,393.17)(-1.547,-2.000){2}{\rule{0.350pt}{0.400pt}}
\put(190,390.17){\rule{0.900pt}{0.400pt}}
\multiput(192.13,391.17)(-2.132,-2.000){2}{\rule{0.450pt}{0.400pt}}
\put(187,388.17){\rule{0.700pt}{0.400pt}}
\multiput(188.55,389.17)(-1.547,-2.000){2}{\rule{0.350pt}{0.400pt}}
\put(183,386.17){\rule{0.900pt}{0.400pt}}
\multiput(185.13,387.17)(-2.132,-2.000){2}{\rule{0.450pt}{0.400pt}}
\put(180,384.17){\rule{0.700pt}{0.400pt}}
\multiput(181.55,385.17)(-1.547,-2.000){2}{\rule{0.350pt}{0.400pt}}
\put(177,382.17){\rule{0.700pt}{0.400pt}}
\multiput(178.55,383.17)(-1.547,-2.000){2}{\rule{0.350pt}{0.400pt}}
\put(173,380.17){\rule{0.900pt}{0.400pt}}
\multiput(175.13,381.17)(-2.132,-2.000){2}{\rule{0.450pt}{0.400pt}}
\put(170,378.17){\rule{0.700pt}{0.400pt}}
\multiput(171.55,379.17)(-1.547,-2.000){2}{\rule{0.350pt}{0.400pt}}
\put(166,376.17){\rule{0.900pt}{0.400pt}}
\multiput(168.13,377.17)(-2.132,-2.000){2}{\rule{0.450pt}{0.400pt}}
\put(163,374.17){\rule{0.700pt}{0.400pt}}
\multiput(164.55,375.17)(-1.547,-2.000){2}{\rule{0.350pt}{0.400pt}}
\put(160,372.17){\rule{0.700pt}{0.400pt}}
\multiput(161.55,373.17)(-1.547,-2.000){2}{\rule{0.350pt}{0.400pt}}
\put(156,370.17){\rule{0.900pt}{0.400pt}}
\multiput(158.13,371.17)(-2.132,-2.000){2}{\rule{0.450pt}{0.400pt}}
\put(153,368.17){\rule{0.700pt}{0.400pt}}
\multiput(154.55,369.17)(-1.547,-2.000){2}{\rule{0.350pt}{0.400pt}}
\put(153,366.17){\rule{0.700pt}{0.400pt}}
\multiput(153.00,367.17)(1.547,-2.000){2}{\rule{0.350pt}{0.400pt}}
\put(156,364.17){\rule{0.900pt}{0.400pt}}
\multiput(156.00,365.17)(2.132,-2.000){2}{\rule{0.450pt}{0.400pt}}
\put(160,362.17){\rule{0.700pt}{0.400pt}}
\multiput(160.00,363.17)(1.547,-2.000){2}{\rule{0.350pt}{0.400pt}}
\put(163,360.17){\rule{0.700pt}{0.400pt}}
\multiput(163.00,361.17)(1.547,-2.000){2}{\rule{0.350pt}{0.400pt}}
\put(166,358.17){\rule{0.900pt}{0.400pt}}
\multiput(166.00,359.17)(2.132,-2.000){2}{\rule{0.450pt}{0.400pt}}
\put(170,356.17){\rule{0.700pt}{0.400pt}}
\multiput(170.00,357.17)(1.547,-2.000){2}{\rule{0.350pt}{0.400pt}}
\put(173,354.17){\rule{0.900pt}{0.400pt}}
\multiput(173.00,355.17)(2.132,-2.000){2}{\rule{0.450pt}{0.400pt}}
\put(177,352.17){\rule{0.700pt}{0.400pt}}
\multiput(177.00,353.17)(1.547,-2.000){2}{\rule{0.350pt}{0.400pt}}
\put(180,350.17){\rule{0.700pt}{0.400pt}}
\multiput(180.00,351.17)(1.547,-2.000){2}{\rule{0.350pt}{0.400pt}}
\put(183,348.17){\rule{0.900pt}{0.400pt}}
\multiput(183.00,349.17)(2.132,-2.000){2}{\rule{0.450pt}{0.400pt}}
\put(187,346.17){\rule{0.700pt}{0.400pt}}
\multiput(187.00,347.17)(1.547,-2.000){2}{\rule{0.350pt}{0.400pt}}
\put(190,344.17){\rule{0.900pt}{0.400pt}}
\multiput(190.00,345.17)(2.132,-2.000){2}{\rule{0.450pt}{0.400pt}}
\put(194,342.17){\rule{0.700pt}{0.400pt}}
\multiput(194.00,343.17)(1.547,-2.000){2}{\rule{0.350pt}{0.400pt}}
\put(197,340.17){\rule{0.900pt}{0.400pt}}
\multiput(197.00,341.17)(2.132,-2.000){2}{\rule{0.450pt}{0.400pt}}
\put(201,338.17){\rule{0.700pt}{0.400pt}}
\multiput(201.00,339.17)(1.547,-2.000){2}{\rule{0.350pt}{0.400pt}}
\put(204,336.17){\rule{0.700pt}{0.400pt}}
\multiput(204.00,337.17)(1.547,-2.000){2}{\rule{0.350pt}{0.400pt}}
\put(207,334.17){\rule{0.900pt}{0.400pt}}
\multiput(207.00,335.17)(2.132,-2.000){2}{\rule{0.450pt}{0.400pt}}
\put(211,332.17){\rule{0.700pt}{0.400pt}}
\multiput(211.00,333.17)(1.547,-2.000){2}{\rule{0.350pt}{0.400pt}}
\put(214,330.17){\rule{0.900pt}{0.400pt}}
\multiput(214.00,331.17)(2.132,-2.000){2}{\rule{0.450pt}{0.400pt}}
\put(218,328.17){\rule{0.700pt}{0.400pt}}
\multiput(218.00,329.17)(1.547,-2.000){2}{\rule{0.350pt}{0.400pt}}
\put(221,326.17){\rule{0.700pt}{0.400pt}}
\multiput(221.00,327.17)(1.547,-2.000){2}{\rule{0.350pt}{0.400pt}}
\put(224,324.17){\rule{0.900pt}{0.400pt}}
\multiput(224.00,325.17)(2.132,-2.000){2}{\rule{0.450pt}{0.400pt}}
\put(228,322.17){\rule{0.700pt}{0.400pt}}
\multiput(228.00,323.17)(1.547,-2.000){2}{\rule{0.350pt}{0.400pt}}
\put(231,320.17){\rule{0.900pt}{0.400pt}}
\multiput(231.00,321.17)(2.132,-2.000){2}{\rule{0.450pt}{0.400pt}}
\put(235,318.17){\rule{0.700pt}{0.400pt}}
\multiput(235.00,319.17)(1.547,-2.000){2}{\rule{0.350pt}{0.400pt}}
\put(238,316.17){\rule{0.700pt}{0.400pt}}
\multiput(238.00,317.17)(1.547,-2.000){2}{\rule{0.350pt}{0.400pt}}
\put(241,314.17){\rule{0.900pt}{0.400pt}}
\multiput(241.00,315.17)(2.132,-2.000){2}{\rule{0.450pt}{0.400pt}}
\put(245,312.17){\rule{0.700pt}{0.400pt}}
\multiput(245.00,313.17)(1.547,-2.000){2}{\rule{0.350pt}{0.400pt}}
\put(248,310.17){\rule{0.900pt}{0.400pt}}
\multiput(248.00,311.17)(2.132,-2.000){2}{\rule{0.450pt}{0.400pt}}
\put(252,308.67){\rule{0.723pt}{0.400pt}}
\multiput(252.00,309.17)(1.500,-1.000){2}{\rule{0.361pt}{0.400pt}}
\put(255,307.17){\rule{0.900pt}{0.400pt}}
\multiput(255.00,308.17)(2.132,-2.000){2}{\rule{0.450pt}{0.400pt}}
\put(259,305.17){\rule{0.700pt}{0.400pt}}
\multiput(259.00,306.17)(1.547,-2.000){2}{\rule{0.350pt}{0.400pt}}
\put(262,303.17){\rule{0.700pt}{0.400pt}}
\multiput(262.00,304.17)(1.547,-2.000){2}{\rule{0.350pt}{0.400pt}}
\put(265,301.17){\rule{0.900pt}{0.400pt}}
\multiput(265.00,302.17)(2.132,-2.000){2}{\rule{0.450pt}{0.400pt}}
\put(269,299.17){\rule{0.700pt}{0.400pt}}
\multiput(269.00,300.17)(1.547,-2.000){2}{\rule{0.350pt}{0.400pt}}
\put(272,297.17){\rule{0.900pt}{0.400pt}}
\multiput(272.00,298.17)(2.132,-2.000){2}{\rule{0.450pt}{0.400pt}}
\put(276,295.17){\rule{0.700pt}{0.400pt}}
\multiput(276.00,296.17)(1.547,-2.000){2}{\rule{0.350pt}{0.400pt}}
\put(279,293.17){\rule{0.700pt}{0.400pt}}
\multiput(279.00,294.17)(1.547,-2.000){2}{\rule{0.350pt}{0.400pt}}
\put(282,291.17){\rule{0.900pt}{0.400pt}}
\multiput(282.00,292.17)(2.132,-2.000){2}{\rule{0.450pt}{0.400pt}}
\put(286,289.17){\rule{0.700pt}{0.400pt}}
\multiput(286.00,290.17)(1.547,-2.000){2}{\rule{0.350pt}{0.400pt}}
\put(289,287.17){\rule{0.900pt}{0.400pt}}
\multiput(289.00,288.17)(2.132,-2.000){2}{\rule{0.450pt}{0.400pt}}
\put(293,285.17){\rule{0.700pt}{0.400pt}}
\multiput(293.00,286.17)(1.547,-2.000){2}{\rule{0.350pt}{0.400pt}}
\put(296,283.17){\rule{0.700pt}{0.400pt}}
\multiput(296.00,284.17)(1.547,-2.000){2}{\rule{0.350pt}{0.400pt}}
\put(299,281.17){\rule{0.900pt}{0.400pt}}
\multiput(299.00,282.17)(2.132,-2.000){2}{\rule{0.450pt}{0.400pt}}
\put(303,279.17){\rule{0.700pt}{0.400pt}}
\multiput(303.00,280.17)(1.547,-2.000){2}{\rule{0.350pt}{0.400pt}}
\put(306,277.17){\rule{0.900pt}{0.400pt}}
\multiput(306.00,278.17)(2.132,-2.000){2}{\rule{0.450pt}{0.400pt}}
\put(310,275.17){\rule{0.700pt}{0.400pt}}
\multiput(310.00,276.17)(1.547,-2.000){2}{\rule{0.350pt}{0.400pt}}
\put(313,273.17){\rule{0.700pt}{0.400pt}}
\multiput(313.00,274.17)(1.547,-2.000){2}{\rule{0.350pt}{0.400pt}}
\put(316,271.17){\rule{0.900pt}{0.400pt}}
\multiput(316.00,272.17)(2.132,-2.000){2}{\rule{0.450pt}{0.400pt}}
\put(320,269.17){\rule{0.700pt}{0.400pt}}
\multiput(320.00,270.17)(1.547,-2.000){2}{\rule{0.350pt}{0.400pt}}
\put(323,267.17){\rule{0.900pt}{0.400pt}}
\multiput(323.00,268.17)(2.132,-2.000){2}{\rule{0.450pt}{0.400pt}}
\put(327,265.17){\rule{0.700pt}{0.400pt}}
\multiput(327.00,266.17)(1.547,-2.000){2}{\rule{0.350pt}{0.400pt}}
\put(330,263.17){\rule{0.900pt}{0.400pt}}
\multiput(330.00,264.17)(2.132,-2.000){2}{\rule{0.450pt}{0.400pt}}
\put(334,261.17){\rule{0.700pt}{0.400pt}}
\multiput(334.00,262.17)(1.547,-2.000){2}{\rule{0.350pt}{0.400pt}}
\put(337,259.17){\rule{0.700pt}{0.400pt}}
\multiput(337.00,260.17)(1.547,-2.000){2}{\rule{0.350pt}{0.400pt}}
\put(340,257.17){\rule{0.900pt}{0.400pt}}
\multiput(340.00,258.17)(2.132,-2.000){2}{\rule{0.450pt}{0.400pt}}
\put(344,255.17){\rule{0.700pt}{0.400pt}}
\multiput(344.00,256.17)(1.547,-2.000){2}{\rule{0.350pt}{0.400pt}}
\put(347,253.17){\rule{0.900pt}{0.400pt}}
\multiput(347.00,254.17)(2.132,-2.000){2}{\rule{0.450pt}{0.400pt}}
\put(351,251.17){\rule{0.700pt}{0.400pt}}
\multiput(351.00,252.17)(1.547,-2.000){2}{\rule{0.350pt}{0.400pt}}
\put(354,249.17){\rule{0.700pt}{0.400pt}}
\multiput(354.00,250.17)(1.547,-2.000){2}{\rule{0.350pt}{0.400pt}}
\put(357,247.17){\rule{0.900pt}{0.400pt}}
\multiput(357.00,248.17)(2.132,-2.000){2}{\rule{0.450pt}{0.400pt}}
\put(361,245.17){\rule{0.700pt}{0.400pt}}
\multiput(361.00,246.17)(1.547,-2.000){2}{\rule{0.350pt}{0.400pt}}
\put(364,243.17){\rule{0.900pt}{0.400pt}}
\multiput(364.00,244.17)(2.132,-2.000){2}{\rule{0.450pt}{0.400pt}}
\put(368,241.17){\rule{0.700pt}{0.400pt}}
\multiput(368.00,242.17)(1.547,-2.000){2}{\rule{0.350pt}{0.400pt}}
\put(371,239.17){\rule{0.700pt}{0.400pt}}
\multiput(371.00,240.17)(1.547,-2.000){2}{\rule{0.350pt}{0.400pt}}
\put(374,237.17){\rule{0.900pt}{0.400pt}}
\multiput(374.00,238.17)(2.132,-2.000){2}{\rule{0.450pt}{0.400pt}}
\put(378,235.17){\rule{0.700pt}{0.400pt}}
\multiput(378.00,236.17)(1.547,-2.000){2}{\rule{0.350pt}{0.400pt}}
\put(381,233.17){\rule{0.900pt}{0.400pt}}
\multiput(381.00,234.17)(2.132,-2.000){2}{\rule{0.450pt}{0.400pt}}
\put(385,231.17){\rule{0.700pt}{0.400pt}}
\multiput(385.00,232.17)(1.547,-2.000){2}{\rule{0.350pt}{0.400pt}}
\put(388,229.17){\rule{0.700pt}{0.400pt}}
\multiput(388.00,230.17)(1.547,-2.000){2}{\rule{0.350pt}{0.400pt}}
\put(391,227.17){\rule{0.900pt}{0.400pt}}
\multiput(391.00,228.17)(2.132,-2.000){2}{\rule{0.450pt}{0.400pt}}
\put(395,225.17){\rule{0.700pt}{0.400pt}}
\multiput(395.00,226.17)(1.547,-2.000){2}{\rule{0.350pt}{0.400pt}}
\put(398,223.17){\rule{0.900pt}{0.400pt}}
\multiput(398.00,224.17)(2.132,-2.000){2}{\rule{0.450pt}{0.400pt}}
\put(402,221.17){\rule{0.700pt}{0.400pt}}
\multiput(402.00,222.17)(1.547,-2.000){2}{\rule{0.350pt}{0.400pt}}
\put(405,219.17){\rule{0.900pt}{0.400pt}}
\multiput(405.00,220.17)(2.132,-2.000){2}{\rule{0.450pt}{0.400pt}}
\put(409,217.17){\rule{0.700pt}{0.400pt}}
\multiput(409.00,218.17)(1.547,-2.000){2}{\rule{0.350pt}{0.400pt}}
\put(412,215.17){\rule{0.700pt}{0.400pt}}
\multiput(412.00,216.17)(1.547,-2.000){2}{\rule{0.350pt}{0.400pt}}
\put(415,213.17){\rule{0.900pt}{0.400pt}}
\multiput(415.00,214.17)(2.132,-2.000){2}{\rule{0.450pt}{0.400pt}}
\put(419,211.17){\rule{0.700pt}{0.400pt}}
\multiput(419.00,212.17)(1.547,-2.000){2}{\rule{0.350pt}{0.400pt}}
\put(422,209.17){\rule{0.900pt}{0.400pt}}
\multiput(422.00,210.17)(2.132,-2.000){2}{\rule{0.450pt}{0.400pt}}
\put(426,207.17){\rule{0.700pt}{0.400pt}}
\multiput(426.00,208.17)(1.547,-2.000){2}{\rule{0.350pt}{0.400pt}}
\put(429,205.17){\rule{0.700pt}{0.400pt}}
\multiput(429.00,206.17)(1.547,-2.000){2}{\rule{0.350pt}{0.400pt}}
\put(432,203.17){\rule{0.900pt}{0.400pt}}
\multiput(432.00,204.17)(2.132,-2.000){2}{\rule{0.450pt}{0.400pt}}
\put(436,201.17){\rule{0.700pt}{0.400pt}}
\multiput(436.00,202.17)(1.547,-2.000){2}{\rule{0.350pt}{0.400pt}}
\put(439,199.17){\rule{0.900pt}{0.400pt}}
\multiput(439.00,200.17)(2.132,-2.000){2}{\rule{0.450pt}{0.400pt}}
\put(443,197.17){\rule{0.700pt}{0.400pt}}
\multiput(443.00,198.17)(1.547,-2.000){2}{\rule{0.350pt}{0.400pt}}
\put(446,195.17){\rule{0.700pt}{0.400pt}}
\multiput(446.00,196.17)(1.547,-2.000){2}{\rule{0.350pt}{0.400pt}}
\put(449,193.17){\rule{0.900pt}{0.400pt}}
\multiput(449.00,194.17)(2.132,-2.000){2}{\rule{0.450pt}{0.400pt}}
\put(453,191.17){\rule{0.700pt}{0.400pt}}
\multiput(453.00,192.17)(1.547,-2.000){2}{\rule{0.350pt}{0.400pt}}
\put(456,189.17){\rule{0.900pt}{0.400pt}}
\multiput(456.00,190.17)(2.132,-2.000){2}{\rule{0.450pt}{0.400pt}}
\put(460,187.17){\rule{0.700pt}{0.400pt}}
\multiput(460.00,188.17)(1.547,-2.000){2}{\rule{0.350pt}{0.400pt}}
\put(463,185.17){\rule{0.700pt}{0.400pt}}
\multiput(463.00,186.17)(1.547,-2.000){2}{\rule{0.350pt}{0.400pt}}
\put(466,183.17){\rule{0.900pt}{0.400pt}}
\multiput(466.00,184.17)(2.132,-2.000){2}{\rule{0.450pt}{0.400pt}}
\put(470,181.17){\rule{0.700pt}{0.400pt}}
\multiput(470.00,182.17)(1.547,-2.000){2}{\rule{0.350pt}{0.400pt}}
\put(473,179.17){\rule{0.900pt}{0.400pt}}
\multiput(473.00,180.17)(2.132,-2.000){2}{\rule{0.450pt}{0.400pt}}
\put(477,177.17){\rule{0.700pt}{0.400pt}}
\multiput(477.00,178.17)(1.547,-2.000){2}{\rule{0.350pt}{0.400pt}}
\put(480,175.17){\rule{0.900pt}{0.400pt}}
\multiput(480.00,176.17)(2.132,-2.000){2}{\rule{0.450pt}{0.400pt}}
\put(484,173.17){\rule{0.700pt}{0.400pt}}
\multiput(484.00,174.17)(1.547,-2.000){2}{\rule{0.350pt}{0.400pt}}
\put(487,171.17){\rule{0.700pt}{0.400pt}}
\multiput(487.00,172.17)(1.547,-2.000){2}{\rule{0.350pt}{0.400pt}}
\put(490,169.17){\rule{0.900pt}{0.400pt}}
\multiput(490.00,170.17)(2.132,-2.000){2}{\rule{0.450pt}{0.400pt}}
\put(494,167.17){\rule{0.700pt}{0.400pt}}
\multiput(494.00,168.17)(1.547,-2.000){2}{\rule{0.350pt}{0.400pt}}
\put(497,165.17){\rule{0.900pt}{0.400pt}}
\multiput(497.00,166.17)(2.132,-2.000){2}{\rule{0.450pt}{0.400pt}}
\put(501,163.17){\rule{0.700pt}{0.400pt}}
\multiput(501.00,164.17)(1.547,-2.000){2}{\rule{0.350pt}{0.400pt}}
\put(504,161.17){\rule{0.700pt}{0.400pt}}
\multiput(504.00,162.17)(1.547,-2.000){2}{\rule{0.350pt}{0.400pt}}
\put(507,159.17){\rule{0.900pt}{0.400pt}}
\multiput(507.00,160.17)(2.132,-2.000){2}{\rule{0.450pt}{0.400pt}}
\put(511,157.17){\rule{0.700pt}{0.400pt}}
\multiput(511.00,158.17)(1.547,-2.000){2}{\rule{0.350pt}{0.400pt}}
\put(514,155.17){\rule{0.900pt}{0.400pt}}
\multiput(514.00,156.17)(2.132,-2.000){2}{\rule{0.450pt}{0.400pt}}
\put(518,153.17){\rule{0.700pt}{0.400pt}}
\multiput(518.00,154.17)(1.547,-2.000){2}{\rule{0.350pt}{0.400pt}}
\put(521,151.17){\rule{0.700pt}{0.400pt}}
\multiput(521.00,152.17)(1.547,-2.000){2}{\rule{0.350pt}{0.400pt}}
\put(524,149.67){\rule{0.964pt}{0.400pt}}
\multiput(524.00,150.17)(2.000,-1.000){2}{\rule{0.482pt}{0.400pt}}
\put(528,148.17){\rule{0.700pt}{0.400pt}}
\multiput(528.00,149.17)(1.547,-2.000){2}{\rule{0.350pt}{0.400pt}}
\put(531,146.17){\rule{0.900pt}{0.400pt}}
\multiput(531.00,147.17)(2.132,-2.000){2}{\rule{0.450pt}{0.400pt}}
\put(535,144.17){\rule{0.700pt}{0.400pt}}
\multiput(535.00,145.17)(1.547,-2.000){2}{\rule{0.350pt}{0.400pt}}
\put(538,142.17){\rule{0.900pt}{0.400pt}}
\multiput(538.00,143.17)(2.132,-2.000){2}{\rule{0.450pt}{0.400pt}}
\put(542,140.17){\rule{0.700pt}{0.400pt}}
\multiput(542.00,141.17)(1.547,-2.000){2}{\rule{0.350pt}{0.400pt}}
\put(545,138.17){\rule{0.700pt}{0.400pt}}
\multiput(545.00,139.17)(1.547,-2.000){2}{\rule{0.350pt}{0.400pt}}
\put(548,136.17){\rule{0.900pt}{0.400pt}}
\multiput(548.00,137.17)(2.132,-2.000){2}{\rule{0.450pt}{0.400pt}}
\put(552,134.17){\rule{0.700pt}{0.400pt}}
\multiput(552.00,135.17)(1.547,-2.000){2}{\rule{0.350pt}{0.400pt}}
\put(555,132.17){\rule{0.900pt}{0.400pt}}
\multiput(555.00,133.17)(2.132,-2.000){2}{\rule{0.450pt}{0.400pt}}
\put(559,130.17){\rule{0.700pt}{0.400pt}}
\multiput(559.00,131.17)(1.547,-2.000){2}{\rule{0.350pt}{0.400pt}}
\put(562,128.17){\rule{0.700pt}{0.400pt}}
\multiput(562.00,129.17)(1.547,-2.000){2}{\rule{0.350pt}{0.400pt}}
\put(565,126.17){\rule{0.900pt}{0.400pt}}
\multiput(565.00,127.17)(2.132,-2.000){2}{\rule{0.450pt}{0.400pt}}
\put(569,124.17){\rule{0.700pt}{0.400pt}}
\multiput(569.00,125.17)(1.547,-2.000){2}{\rule{0.350pt}{0.400pt}}
\put(572,122.17){\rule{0.900pt}{0.400pt}}
\multiput(572.00,123.17)(2.132,-2.000){2}{\rule{0.450pt}{0.400pt}}
\put(576,120.17){\rule{0.700pt}{0.400pt}}
\multiput(576.00,121.17)(1.547,-2.000){2}{\rule{0.350pt}{0.400pt}}
\put(579,118.17){\rule{0.700pt}{0.400pt}}
\multiput(579.00,119.17)(1.547,-2.000){2}{\rule{0.350pt}{0.400pt}}
\put(582,116.17){\rule{0.900pt}{0.400pt}}
\multiput(582.00,117.17)(2.132,-2.000){2}{\rule{0.450pt}{0.400pt}}
\put(586,114.17){\rule{0.700pt}{0.400pt}}
\multiput(586.00,115.17)(1.547,-2.000){2}{\rule{0.350pt}{0.400pt}}
\put(589,112.17){\rule{0.900pt}{0.400pt}}
\multiput(589.00,113.17)(2.132,-2.000){2}{\rule{0.450pt}{0.400pt}}
\put(593,110.17){\rule{0.700pt}{0.400pt}}
\multiput(593.00,111.17)(1.547,-2.000){2}{\rule{0.350pt}{0.400pt}}
\put(596,108.17){\rule{0.700pt}{0.400pt}}
\multiput(596.00,109.17)(1.547,-2.000){2}{\rule{0.350pt}{0.400pt}}
\put(599,106.17){\rule{0.900pt}{0.400pt}}
\multiput(599.00,107.17)(2.132,-2.000){2}{\rule{0.450pt}{0.400pt}}
\put(603,104.17){\rule{0.700pt}{0.400pt}}
\multiput(603.00,105.17)(1.547,-2.000){2}{\rule{0.350pt}{0.400pt}}
\put(606,104.17){\rule{0.900pt}{0.400pt}}
\multiput(606.00,103.17)(2.132,2.000){2}{\rule{0.450pt}{0.400pt}}
\put(610,106.17){\rule{0.700pt}{0.400pt}}
\multiput(610.00,105.17)(1.547,2.000){2}{\rule{0.350pt}{0.400pt}}
\put(613,108.17){\rule{0.900pt}{0.400pt}}
\multiput(613.00,107.17)(2.132,2.000){2}{\rule{0.450pt}{0.400pt}}
\put(617,110.17){\rule{0.700pt}{0.400pt}}
\multiput(617.00,109.17)(1.547,2.000){2}{\rule{0.350pt}{0.400pt}}
\put(620,112.17){\rule{0.700pt}{0.400pt}}
\multiput(620.00,111.17)(1.547,2.000){2}{\rule{0.350pt}{0.400pt}}
\put(623,114.17){\rule{0.900pt}{0.400pt}}
\multiput(623.00,113.17)(2.132,2.000){2}{\rule{0.450pt}{0.400pt}}
\put(627,116.17){\rule{0.700pt}{0.400pt}}
\multiput(627.00,115.17)(1.547,2.000){2}{\rule{0.350pt}{0.400pt}}
\put(630,118.17){\rule{0.900pt}{0.400pt}}
\multiput(630.00,117.17)(2.132,2.000){2}{\rule{0.450pt}{0.400pt}}
\put(634,120.17){\rule{0.700pt}{0.400pt}}
\multiput(634.00,119.17)(1.547,2.000){2}{\rule{0.350pt}{0.400pt}}
\put(637,122.17){\rule{0.700pt}{0.400pt}}
\multiput(637.00,121.17)(1.547,2.000){2}{\rule{0.350pt}{0.400pt}}
\put(640,124.17){\rule{0.900pt}{0.400pt}}
\multiput(640.00,123.17)(2.132,2.000){2}{\rule{0.450pt}{0.400pt}}
\put(644,126.17){\rule{0.700pt}{0.400pt}}
\multiput(644.00,125.17)(1.547,2.000){2}{\rule{0.350pt}{0.400pt}}
\put(647,128.17){\rule{0.900pt}{0.400pt}}
\multiput(647.00,127.17)(2.132,2.000){2}{\rule{0.450pt}{0.400pt}}
\put(651,130.17){\rule{0.700pt}{0.400pt}}
\multiput(651.00,129.17)(1.547,2.000){2}{\rule{0.350pt}{0.400pt}}
\put(654,132.17){\rule{0.700pt}{0.400pt}}
\multiput(654.00,131.17)(1.547,2.000){2}{\rule{0.350pt}{0.400pt}}
\put(657,134.17){\rule{0.900pt}{0.400pt}}
\multiput(657.00,133.17)(2.132,2.000){2}{\rule{0.450pt}{0.400pt}}
\put(661,136.17){\rule{0.700pt}{0.400pt}}
\multiput(661.00,135.17)(1.547,2.000){2}{\rule{0.350pt}{0.400pt}}
\put(664,138.17){\rule{0.900pt}{0.400pt}}
\multiput(664.00,137.17)(2.132,2.000){2}{\rule{0.450pt}{0.400pt}}
\put(668,140.17){\rule{0.700pt}{0.400pt}}
\multiput(668.00,139.17)(1.547,2.000){2}{\rule{0.350pt}{0.400pt}}
\put(671,142.17){\rule{0.700pt}{0.400pt}}
\multiput(671.00,141.17)(1.547,2.000){2}{\rule{0.350pt}{0.400pt}}
\put(674,144.17){\rule{0.900pt}{0.400pt}}
\multiput(674.00,143.17)(2.132,2.000){2}{\rule{0.450pt}{0.400pt}}
\put(678,146.17){\rule{0.700pt}{0.400pt}}
\multiput(678.00,145.17)(1.547,2.000){2}{\rule{0.350pt}{0.400pt}}
\put(681,148.17){\rule{0.900pt}{0.400pt}}
\multiput(681.00,147.17)(2.132,2.000){2}{\rule{0.450pt}{0.400pt}}
\put(685,149.67){\rule{0.723pt}{0.400pt}}
\multiput(685.00,149.17)(1.500,1.000){2}{\rule{0.361pt}{0.400pt}}
\put(688,151.17){\rule{0.900pt}{0.400pt}}
\multiput(688.00,150.17)(2.132,2.000){2}{\rule{0.450pt}{0.400pt}}
\put(692,153.17){\rule{0.700pt}{0.400pt}}
\multiput(692.00,152.17)(1.547,2.000){2}{\rule{0.350pt}{0.400pt}}
\put(695,155.17){\rule{0.700pt}{0.400pt}}
\multiput(695.00,154.17)(1.547,2.000){2}{\rule{0.350pt}{0.400pt}}
\put(698,157.17){\rule{0.900pt}{0.400pt}}
\multiput(698.00,156.17)(2.132,2.000){2}{\rule{0.450pt}{0.400pt}}
\put(702,159.17){\rule{0.700pt}{0.400pt}}
\multiput(702.00,158.17)(1.547,2.000){2}{\rule{0.350pt}{0.400pt}}
\put(705,161.17){\rule{0.900pt}{0.400pt}}
\multiput(705.00,160.17)(2.132,2.000){2}{\rule{0.450pt}{0.400pt}}
\put(709,163.17){\rule{0.700pt}{0.400pt}}
\multiput(709.00,162.17)(1.547,2.000){2}{\rule{0.350pt}{0.400pt}}
\put(712,165.17){\rule{0.700pt}{0.400pt}}
\multiput(712.00,164.17)(1.547,2.000){2}{\rule{0.350pt}{0.400pt}}
\put(715,167.17){\rule{0.900pt}{0.400pt}}
\multiput(715.00,166.17)(2.132,2.000){2}{\rule{0.450pt}{0.400pt}}
\put(719,169.17){\rule{0.700pt}{0.400pt}}
\multiput(719.00,168.17)(1.547,2.000){2}{\rule{0.350pt}{0.400pt}}
\put(722,171.17){\rule{0.900pt}{0.400pt}}
\multiput(722.00,170.17)(2.132,2.000){2}{\rule{0.450pt}{0.400pt}}
\put(726,173.17){\rule{0.700pt}{0.400pt}}
\multiput(726.00,172.17)(1.547,2.000){2}{\rule{0.350pt}{0.400pt}}
\put(729,175.17){\rule{0.700pt}{0.400pt}}
\multiput(729.00,174.17)(1.547,2.000){2}{\rule{0.350pt}{0.400pt}}
\put(732,177.17){\rule{0.900pt}{0.400pt}}
\multiput(732.00,176.17)(2.132,2.000){2}{\rule{0.450pt}{0.400pt}}
\put(736,179.17){\rule{0.700pt}{0.400pt}}
\multiput(736.00,178.17)(1.547,2.000){2}{\rule{0.350pt}{0.400pt}}
\put(739,181.17){\rule{0.900pt}{0.400pt}}
\multiput(739.00,180.17)(2.132,2.000){2}{\rule{0.450pt}{0.400pt}}
\put(743,183.17){\rule{0.700pt}{0.400pt}}
\multiput(743.00,182.17)(1.547,2.000){2}{\rule{0.350pt}{0.400pt}}
\put(746,185.17){\rule{0.700pt}{0.400pt}}
\multiput(746.00,184.17)(1.547,2.000){2}{\rule{0.350pt}{0.400pt}}
\put(749,187.17){\rule{0.900pt}{0.400pt}}
\multiput(749.00,186.17)(2.132,2.000){2}{\rule{0.450pt}{0.400pt}}
\put(753,189.17){\rule{0.700pt}{0.400pt}}
\multiput(753.00,188.17)(1.547,2.000){2}{\rule{0.350pt}{0.400pt}}
\put(756,191.17){\rule{0.900pt}{0.400pt}}
\multiput(756.00,190.17)(2.132,2.000){2}{\rule{0.450pt}{0.400pt}}
\put(760,193.17){\rule{0.700pt}{0.400pt}}
\multiput(760.00,192.17)(1.547,2.000){2}{\rule{0.350pt}{0.400pt}}
\put(763,195.17){\rule{0.900pt}{0.400pt}}
\multiput(763.00,194.17)(2.132,2.000){2}{\rule{0.450pt}{0.400pt}}
\put(767,197.17){\rule{0.700pt}{0.400pt}}
\multiput(767.00,196.17)(1.547,2.000){2}{\rule{0.350pt}{0.400pt}}
\put(770,199.17){\rule{0.700pt}{0.400pt}}
\multiput(770.00,198.17)(1.547,2.000){2}{\rule{0.350pt}{0.400pt}}
\put(773,201.17){\rule{0.900pt}{0.400pt}}
\multiput(773.00,200.17)(2.132,2.000){2}{\rule{0.450pt}{0.400pt}}
\put(777,203.17){\rule{0.700pt}{0.400pt}}
\multiput(777.00,202.17)(1.547,2.000){2}{\rule{0.350pt}{0.400pt}}
\put(780,205.17){\rule{0.900pt}{0.400pt}}
\multiput(780.00,204.17)(2.132,2.000){2}{\rule{0.450pt}{0.400pt}}
\put(784,207.17){\rule{0.700pt}{0.400pt}}
\multiput(784.00,206.17)(1.547,2.000){2}{\rule{0.350pt}{0.400pt}}
\put(787,209.17){\rule{0.700pt}{0.400pt}}
\multiput(787.00,208.17)(1.547,2.000){2}{\rule{0.350pt}{0.400pt}}
\put(790,211.17){\rule{0.900pt}{0.400pt}}
\multiput(790.00,210.17)(2.132,2.000){2}{\rule{0.450pt}{0.400pt}}
\put(794,213.17){\rule{0.700pt}{0.400pt}}
\multiput(794.00,212.17)(1.547,2.000){2}{\rule{0.350pt}{0.400pt}}
\put(797,215.17){\rule{0.900pt}{0.400pt}}
\multiput(797.00,214.17)(2.132,2.000){2}{\rule{0.450pt}{0.400pt}}
\put(801,217.17){\rule{0.700pt}{0.400pt}}
\multiput(801.00,216.17)(1.547,2.000){2}{\rule{0.350pt}{0.400pt}}
\put(804,219.17){\rule{0.700pt}{0.400pt}}
\multiput(804.00,218.17)(1.547,2.000){2}{\rule{0.350pt}{0.400pt}}
\put(807,221.17){\rule{0.900pt}{0.400pt}}
\multiput(807.00,220.17)(2.132,2.000){2}{\rule{0.450pt}{0.400pt}}
\put(811,223.17){\rule{0.700pt}{0.400pt}}
\multiput(811.00,222.17)(1.547,2.000){2}{\rule{0.350pt}{0.400pt}}
\put(814,225.17){\rule{0.900pt}{0.400pt}}
\multiput(814.00,224.17)(2.132,2.000){2}{\rule{0.450pt}{0.400pt}}
\put(818,227.17){\rule{0.700pt}{0.400pt}}
\multiput(818.00,226.17)(1.547,2.000){2}{\rule{0.350pt}{0.400pt}}
\put(821,229.17){\rule{0.900pt}{0.400pt}}
\multiput(821.00,228.17)(2.132,2.000){2}{\rule{0.450pt}{0.400pt}}
\put(825,231.17){\rule{0.700pt}{0.400pt}}
\multiput(825.00,230.17)(1.547,2.000){2}{\rule{0.350pt}{0.400pt}}
\put(828,233.17){\rule{0.700pt}{0.400pt}}
\multiput(828.00,232.17)(1.547,2.000){2}{\rule{0.350pt}{0.400pt}}
\put(831,235.17){\rule{0.900pt}{0.400pt}}
\multiput(831.00,234.17)(2.132,2.000){2}{\rule{0.450pt}{0.400pt}}
\put(835,237.17){\rule{0.700pt}{0.400pt}}
\multiput(835.00,236.17)(1.547,2.000){2}{\rule{0.350pt}{0.400pt}}
\put(838,239.17){\rule{0.900pt}{0.400pt}}
\multiput(838.00,238.17)(2.132,2.000){2}{\rule{0.450pt}{0.400pt}}
\put(842,241.17){\rule{0.700pt}{0.400pt}}
\multiput(842.00,240.17)(1.547,2.000){2}{\rule{0.350pt}{0.400pt}}
\put(845,243.17){\rule{0.700pt}{0.400pt}}
\multiput(845.00,242.17)(1.547,2.000){2}{\rule{0.350pt}{0.400pt}}
\put(848,245.17){\rule{0.900pt}{0.400pt}}
\multiput(848.00,244.17)(2.132,2.000){2}{\rule{0.450pt}{0.400pt}}
\put(852,247.17){\rule{0.700pt}{0.400pt}}
\multiput(852.00,246.17)(1.547,2.000){2}{\rule{0.350pt}{0.400pt}}
\put(855,249.17){\rule{0.900pt}{0.400pt}}
\multiput(855.00,248.17)(2.132,2.000){2}{\rule{0.450pt}{0.400pt}}
\put(859,251.17){\rule{0.700pt}{0.400pt}}
\multiput(859.00,250.17)(1.547,2.000){2}{\rule{0.350pt}{0.400pt}}
\put(862,253.17){\rule{0.700pt}{0.400pt}}
\multiput(862.00,252.17)(1.547,2.000){2}{\rule{0.350pt}{0.400pt}}
\put(865,255.17){\rule{0.900pt}{0.400pt}}
\multiput(865.00,254.17)(2.132,2.000){2}{\rule{0.450pt}{0.400pt}}
\put(869,257.17){\rule{0.700pt}{0.400pt}}
\multiput(869.00,256.17)(1.547,2.000){2}{\rule{0.350pt}{0.400pt}}
\put(872,259.17){\rule{0.900pt}{0.400pt}}
\multiput(872.00,258.17)(2.132,2.000){2}{\rule{0.450pt}{0.400pt}}
\put(876,261.17){\rule{0.700pt}{0.400pt}}
\multiput(876.00,260.17)(1.547,2.000){2}{\rule{0.350pt}{0.400pt}}
\put(879,263.17){\rule{0.700pt}{0.400pt}}
\multiput(879.00,262.17)(1.547,2.000){2}{\rule{0.350pt}{0.400pt}}
\put(882,265.17){\rule{0.900pt}{0.400pt}}
\multiput(882.00,264.17)(2.132,2.000){2}{\rule{0.450pt}{0.400pt}}
\put(886,267.17){\rule{0.700pt}{0.400pt}}
\multiput(886.00,266.17)(1.547,2.000){2}{\rule{0.350pt}{0.400pt}}
\put(889,269.17){\rule{0.900pt}{0.400pt}}
\multiput(889.00,268.17)(2.132,2.000){2}{\rule{0.450pt}{0.400pt}}
\put(893,271.17){\rule{0.700pt}{0.400pt}}
\multiput(893.00,270.17)(1.547,2.000){2}{\rule{0.350pt}{0.400pt}}
\put(896,273.17){\rule{0.900pt}{0.400pt}}
\multiput(896.00,272.17)(2.132,2.000){2}{\rule{0.450pt}{0.400pt}}
\put(900,275.17){\rule{0.700pt}{0.400pt}}
\multiput(900.00,274.17)(1.547,2.000){2}{\rule{0.350pt}{0.400pt}}
\put(903,277.17){\rule{0.700pt}{0.400pt}}
\multiput(903.00,276.17)(1.547,2.000){2}{\rule{0.350pt}{0.400pt}}
\put(906,279.17){\rule{0.900pt}{0.400pt}}
\multiput(906.00,278.17)(2.132,2.000){2}{\rule{0.450pt}{0.400pt}}
\put(910,281.17){\rule{0.700pt}{0.400pt}}
\multiput(910.00,280.17)(1.547,2.000){2}{\rule{0.350pt}{0.400pt}}
\put(913,283.17){\rule{0.900pt}{0.400pt}}
\multiput(913.00,282.17)(2.132,2.000){2}{\rule{0.450pt}{0.400pt}}
\put(917,285.17){\rule{0.700pt}{0.400pt}}
\multiput(917.00,284.17)(1.547,2.000){2}{\rule{0.350pt}{0.400pt}}
\put(920,287.17){\rule{0.700pt}{0.400pt}}
\multiput(920.00,286.17)(1.547,2.000){2}{\rule{0.350pt}{0.400pt}}
\put(923,289.17){\rule{0.900pt}{0.400pt}}
\multiput(923.00,288.17)(2.132,2.000){2}{\rule{0.450pt}{0.400pt}}
\put(927,291.17){\rule{0.700pt}{0.400pt}}
\multiput(927.00,290.17)(1.547,2.000){2}{\rule{0.350pt}{0.400pt}}
\put(930,293.17){\rule{0.900pt}{0.400pt}}
\multiput(930.00,292.17)(2.132,2.000){2}{\rule{0.450pt}{0.400pt}}
\put(934,295.17){\rule{0.700pt}{0.400pt}}
\multiput(934.00,294.17)(1.547,2.000){2}{\rule{0.350pt}{0.400pt}}
\put(937,297.17){\rule{0.700pt}{0.400pt}}
\multiput(937.00,296.17)(1.547,2.000){2}{\rule{0.350pt}{0.400pt}}
\put(940,299.17){\rule{0.900pt}{0.400pt}}
\multiput(940.00,298.17)(2.132,2.000){2}{\rule{0.450pt}{0.400pt}}
\put(944,301.17){\rule{0.700pt}{0.400pt}}
\multiput(944.00,300.17)(1.547,2.000){2}{\rule{0.350pt}{0.400pt}}
\put(947,303.17){\rule{0.900pt}{0.400pt}}
\multiput(947.00,302.17)(2.132,2.000){2}{\rule{0.450pt}{0.400pt}}
\put(951,305.17){\rule{0.700pt}{0.400pt}}
\multiput(951.00,304.17)(1.547,2.000){2}{\rule{0.350pt}{0.400pt}}
\put(954,307.17){\rule{0.700pt}{0.400pt}}
\multiput(954.00,306.17)(1.547,2.000){2}{\rule{0.350pt}{0.400pt}}
\put(957,308.67){\rule{0.964pt}{0.400pt}}
\multiput(957.00,308.17)(2.000,1.000){2}{\rule{0.482pt}{0.400pt}}
\put(961,310.17){\rule{0.700pt}{0.400pt}}
\multiput(961.00,309.17)(1.547,2.000){2}{\rule{0.350pt}{0.400pt}}
\put(964,312.17){\rule{0.900pt}{0.400pt}}
\multiput(964.00,311.17)(2.132,2.000){2}{\rule{0.450pt}{0.400pt}}
\put(968,314.17){\rule{0.700pt}{0.400pt}}
\multiput(968.00,313.17)(1.547,2.000){2}{\rule{0.350pt}{0.400pt}}
\put(971,316.17){\rule{0.900pt}{0.400pt}}
\multiput(971.00,315.17)(2.132,2.000){2}{\rule{0.450pt}{0.400pt}}
\put(975,318.17){\rule{0.700pt}{0.400pt}}
\multiput(975.00,317.17)(1.547,2.000){2}{\rule{0.350pt}{0.400pt}}
\put(978,320.17){\rule{0.700pt}{0.400pt}}
\multiput(978.00,319.17)(1.547,2.000){2}{\rule{0.350pt}{0.400pt}}
\put(981,322.17){\rule{0.900pt}{0.400pt}}
\multiput(981.00,321.17)(2.132,2.000){2}{\rule{0.450pt}{0.400pt}}
\put(985,324.17){\rule{0.700pt}{0.400pt}}
\multiput(985.00,323.17)(1.547,2.000){2}{\rule{0.350pt}{0.400pt}}
\put(988,326.17){\rule{0.900pt}{0.400pt}}
\multiput(988.00,325.17)(2.132,2.000){2}{\rule{0.450pt}{0.400pt}}
\put(992,328.17){\rule{0.700pt}{0.400pt}}
\multiput(992.00,327.17)(1.547,2.000){2}{\rule{0.350pt}{0.400pt}}
\put(995,330.17){\rule{0.700pt}{0.400pt}}
\multiput(995.00,329.17)(1.547,2.000){2}{\rule{0.350pt}{0.400pt}}
\put(998,332.17){\rule{0.900pt}{0.400pt}}
\multiput(998.00,331.17)(2.132,2.000){2}{\rule{0.450pt}{0.400pt}}
\put(1002,334.17){\rule{0.700pt}{0.400pt}}
\multiput(1002.00,333.17)(1.547,2.000){2}{\rule{0.350pt}{0.400pt}}
\put(1005,336.17){\rule{0.900pt}{0.400pt}}
\multiput(1005.00,335.17)(2.132,2.000){2}{\rule{0.450pt}{0.400pt}}
\put(1009,338.17){\rule{0.700pt}{0.400pt}}
\multiput(1009.00,337.17)(1.547,2.000){2}{\rule{0.350pt}{0.400pt}}
\put(1012,340.17){\rule{0.700pt}{0.400pt}}
\multiput(1012.00,339.17)(1.547,2.000){2}{\rule{0.350pt}{0.400pt}}
\put(1015,342.17){\rule{0.900pt}{0.400pt}}
\multiput(1015.00,341.17)(2.132,2.000){2}{\rule{0.450pt}{0.400pt}}
\put(1019,344.17){\rule{0.700pt}{0.400pt}}
\multiput(1019.00,343.17)(1.547,2.000){2}{\rule{0.350pt}{0.400pt}}
\put(1022,346.17){\rule{0.900pt}{0.400pt}}
\multiput(1022.00,345.17)(2.132,2.000){2}{\rule{0.450pt}{0.400pt}}
\put(1026,348.17){\rule{0.700pt}{0.400pt}}
\multiput(1026.00,347.17)(1.547,2.000){2}{\rule{0.350pt}{0.400pt}}
\put(1029,350.17){\rule{0.700pt}{0.400pt}}
\multiput(1029.00,349.17)(1.547,2.000){2}{\rule{0.350pt}{0.400pt}}
\put(1032,352.17){\rule{0.900pt}{0.400pt}}
\multiput(1032.00,351.17)(2.132,2.000){2}{\rule{0.450pt}{0.400pt}}
\put(1036,354.17){\rule{0.700pt}{0.400pt}}
\multiput(1036.00,353.17)(1.547,2.000){2}{\rule{0.350pt}{0.400pt}}
\put(1039,356.17){\rule{0.900pt}{0.400pt}}
\multiput(1039.00,355.17)(2.132,2.000){2}{\rule{0.450pt}{0.400pt}}
\put(1043,358.17){\rule{0.700pt}{0.400pt}}
\multiput(1043.00,357.17)(1.547,2.000){2}{\rule{0.350pt}{0.400pt}}
\put(1046,360.17){\rule{0.900pt}{0.400pt}}
\multiput(1046.00,359.17)(2.132,2.000){2}{\rule{0.450pt}{0.400pt}}
\put(1050,362.17){\rule{0.700pt}{0.400pt}}
\multiput(1050.00,361.17)(1.547,2.000){2}{\rule{0.350pt}{0.400pt}}
\put(1053,364.17){\rule{0.700pt}{0.400pt}}
\multiput(1053.00,363.17)(1.547,2.000){2}{\rule{0.350pt}{0.400pt}}
\put(1056,366.17){\rule{0.900pt}{0.400pt}}
\multiput(1056.00,365.17)(2.132,2.000){2}{\rule{0.450pt}{0.400pt}}
\put(1060,368.17){\rule{0.700pt}{0.400pt}}
\multiput(1060.00,367.17)(1.547,2.000){2}{\rule{0.350pt}{0.400pt}}
\put(1063,370.17){\rule{0.900pt}{0.400pt}}
\multiput(1063.00,369.17)(2.132,2.000){2}{\rule{0.450pt}{0.400pt}}
\put(1067,372.17){\rule{0.700pt}{0.400pt}}
\multiput(1067.00,371.17)(1.547,2.000){2}{\rule{0.350pt}{0.400pt}}
\put(1070,374.17){\rule{0.700pt}{0.400pt}}
\multiput(1070.00,373.17)(1.547,2.000){2}{\rule{0.350pt}{0.400pt}}
\put(1073,376.17){\rule{0.900pt}{0.400pt}}
\multiput(1073.00,375.17)(2.132,2.000){2}{\rule{0.450pt}{0.400pt}}
\put(1077,378.17){\rule{0.700pt}{0.400pt}}
\multiput(1077.00,377.17)(1.547,2.000){2}{\rule{0.350pt}{0.400pt}}
\put(1080,380.17){\rule{0.900pt}{0.400pt}}
\multiput(1080.00,379.17)(2.132,2.000){2}{\rule{0.450pt}{0.400pt}}
\put(1084,382.17){\rule{0.700pt}{0.400pt}}
\multiput(1084.00,381.17)(1.547,2.000){2}{\rule{0.350pt}{0.400pt}}
\put(1087,384.17){\rule{0.700pt}{0.400pt}}
\multiput(1087.00,383.17)(1.547,2.000){2}{\rule{0.350pt}{0.400pt}}
\put(1090,386.17){\rule{0.900pt}{0.400pt}}
\multiput(1090.00,385.17)(2.132,2.000){2}{\rule{0.450pt}{0.400pt}}
\put(1094,388.17){\rule{0.700pt}{0.400pt}}
\multiput(1094.00,387.17)(1.547,2.000){2}{\rule{0.350pt}{0.400pt}}
\put(1097,390.17){\rule{0.900pt}{0.400pt}}
\multiput(1097.00,389.17)(2.132,2.000){2}{\rule{0.450pt}{0.400pt}}
\put(1101,392.17){\rule{0.700pt}{0.400pt}}
\multiput(1101.00,391.17)(1.547,2.000){2}{\rule{0.350pt}{0.400pt}}
\put(1104,394.17){\rule{0.900pt}{0.400pt}}
\multiput(1104.00,393.17)(2.132,2.000){2}{\rule{0.450pt}{0.400pt}}
\put(1108,396.17){\rule{0.700pt}{0.400pt}}
\multiput(1108.00,395.17)(1.547,2.000){2}{\rule{0.350pt}{0.400pt}}
\put(1111,398.17){\rule{0.700pt}{0.400pt}}
\multiput(1111.00,397.17)(1.547,2.000){2}{\rule{0.350pt}{0.400pt}}
\put(1114,400.17){\rule{0.900pt}{0.400pt}}
\multiput(1114.00,399.17)(2.132,2.000){2}{\rule{0.450pt}{0.400pt}}
\put(1118,402.17){\rule{0.700pt}{0.400pt}}
\multiput(1118.00,401.17)(1.547,2.000){2}{\rule{0.350pt}{0.400pt}}
\put(1121,404.17){\rule{0.900pt}{0.400pt}}
\multiput(1121.00,403.17)(2.132,2.000){2}{\rule{0.450pt}{0.400pt}}
\put(1125,406.17){\rule{0.700pt}{0.400pt}}
\multiput(1125.00,405.17)(1.547,2.000){2}{\rule{0.350pt}{0.400pt}}
\put(1128,408.17){\rule{0.700pt}{0.400pt}}
\multiput(1128.00,407.17)(1.547,2.000){2}{\rule{0.350pt}{0.400pt}}
\put(1131,410.17){\rule{0.900pt}{0.400pt}}
\multiput(1131.00,409.17)(2.132,2.000){2}{\rule{0.450pt}{0.400pt}}
\put(1135,412.17){\rule{0.700pt}{0.400pt}}
\multiput(1135.00,411.17)(1.547,2.000){2}{\rule{0.350pt}{0.400pt}}
\put(1138,414.17){\rule{0.900pt}{0.400pt}}
\multiput(1138.00,413.17)(2.132,2.000){2}{\rule{0.450pt}{0.400pt}}
\put(1142,416.17){\rule{0.700pt}{0.400pt}}
\multiput(1142.00,415.17)(1.547,2.000){2}{\rule{0.350pt}{0.400pt}}
\put(1145,418.17){\rule{0.700pt}{0.400pt}}
\multiput(1145.00,417.17)(1.547,2.000){2}{\rule{0.350pt}{0.400pt}}
\put(1148,420.17){\rule{0.900pt}{0.400pt}}
\multiput(1148.00,419.17)(2.132,2.000){2}{\rule{0.450pt}{0.400pt}}
\put(1152,422.17){\rule{0.700pt}{0.400pt}}
\multiput(1152.00,421.17)(1.547,2.000){2}{\rule{0.350pt}{0.400pt}}
\put(1155,424.17){\rule{0.900pt}{0.400pt}}
\multiput(1155.00,423.17)(2.132,2.000){2}{\rule{0.450pt}{0.400pt}}
\put(1159,426.17){\rule{0.700pt}{0.400pt}}
\multiput(1159.00,425.17)(1.547,2.000){2}{\rule{0.350pt}{0.400pt}}
\put(1162,428.17){\rule{0.700pt}{0.400pt}}
\multiput(1162.00,427.17)(1.547,2.000){2}{\rule{0.350pt}{0.400pt}}
\put(1165,430.17){\rule{0.900pt}{0.400pt}}
\multiput(1165.00,429.17)(2.132,2.000){2}{\rule{0.450pt}{0.400pt}}
\put(1169,432.17){\rule{0.700pt}{0.400pt}}
\multiput(1169.00,431.17)(1.547,2.000){2}{\rule{0.350pt}{0.400pt}}
\put(1172,434.17){\rule{0.900pt}{0.400pt}}
\multiput(1172.00,433.17)(2.132,2.000){2}{\rule{0.450pt}{0.400pt}}
\put(1176,436.17){\rule{0.700pt}{0.400pt}}
\multiput(1176.00,435.17)(1.547,2.000){2}{\rule{0.350pt}{0.400pt}}
\put(1179,438.17){\rule{0.900pt}{0.400pt}}
\multiput(1179.00,437.17)(2.132,2.000){2}{\rule{0.450pt}{0.400pt}}
\put(1183,440.17){\rule{0.700pt}{0.400pt}}
\multiput(1183.00,439.17)(1.547,2.000){2}{\rule{0.350pt}{0.400pt}}
\put(1186,442.17){\rule{0.700pt}{0.400pt}}
\multiput(1186.00,441.17)(1.547,2.000){2}{\rule{0.350pt}{0.400pt}}
\put(1189,444.17){\rule{0.900pt}{0.400pt}}
\multiput(1189.00,443.17)(2.132,2.000){2}{\rule{0.450pt}{0.400pt}}
\put(1193,446.17){\rule{0.700pt}{0.400pt}}
\multiput(1193.00,445.17)(1.547,2.000){2}{\rule{0.350pt}{0.400pt}}
\put(1196,448.17){\rule{0.900pt}{0.400pt}}
\multiput(1196.00,447.17)(2.132,2.000){2}{\rule{0.450pt}{0.400pt}}
\put(1200,450.17){\rule{0.700pt}{0.400pt}}
\multiput(1200.00,449.17)(1.547,2.000){2}{\rule{0.350pt}{0.400pt}}
\put(1203,452.17){\rule{0.700pt}{0.400pt}}
\multiput(1203.00,451.17)(1.547,2.000){2}{\rule{0.350pt}{0.400pt}}
\put(1206,454.17){\rule{0.900pt}{0.400pt}}
\multiput(1206.00,453.17)(2.132,2.000){2}{\rule{0.450pt}{0.400pt}}
\put(1210,456.17){\rule{0.700pt}{0.400pt}}
\multiput(1210.00,455.17)(1.547,2.000){2}{\rule{0.350pt}{0.400pt}}
\put(1213,458.17){\rule{0.900pt}{0.400pt}}
\multiput(1213.00,457.17)(2.132,2.000){2}{\rule{0.450pt}{0.400pt}}
\put(1217,460.17){\rule{0.700pt}{0.400pt}}
\multiput(1217.00,459.17)(1.547,2.000){2}{\rule{0.350pt}{0.400pt}}
\put(1220,462.17){\rule{0.700pt}{0.400pt}}
\multiput(1220.00,461.17)(1.547,2.000){2}{\rule{0.350pt}{0.400pt}}
\put(1223,464.17){\rule{0.900pt}{0.400pt}}
\multiput(1223.00,463.17)(2.132,2.000){2}{\rule{0.450pt}{0.400pt}}
\put(1227,466.17){\rule{0.700pt}{0.400pt}}
\multiput(1227.00,465.17)(1.547,2.000){2}{\rule{0.350pt}{0.400pt}}
\put(1230,468.17){\rule{0.900pt}{0.400pt}}
\multiput(1230.00,467.17)(2.132,2.000){2}{\rule{0.450pt}{0.400pt}}
\put(1234,469.67){\rule{0.723pt}{0.400pt}}
\multiput(1234.00,469.17)(1.500,1.000){2}{\rule{0.361pt}{0.400pt}}
\put(1237,471.17){\rule{0.700pt}{0.400pt}}
\multiput(1237.00,470.17)(1.547,2.000){2}{\rule{0.350pt}{0.400pt}}
\put(1240,473.17){\rule{0.900pt}{0.400pt}}
\multiput(1240.00,472.17)(2.132,2.000){2}{\rule{0.450pt}{0.400pt}}
\put(1244,475.17){\rule{0.700pt}{0.400pt}}
\multiput(1244.00,474.17)(1.547,2.000){2}{\rule{0.350pt}{0.400pt}}
\put(1247,477.17){\rule{0.900pt}{0.400pt}}
\multiput(1247.00,476.17)(2.132,2.000){2}{\rule{0.450pt}{0.400pt}}
\put(1251,479.17){\rule{0.700pt}{0.400pt}}
\multiput(1251.00,478.17)(1.547,2.000){2}{\rule{0.350pt}{0.400pt}}
\put(1254,481.17){\rule{0.900pt}{0.400pt}}
\multiput(1254.00,480.17)(2.132,2.000){2}{\rule{0.450pt}{0.400pt}}
\put(1258,483.17){\rule{0.700pt}{0.400pt}}
\multiput(1258.00,482.17)(1.547,2.000){2}{\rule{0.350pt}{0.400pt}}
\put(1261,485.17){\rule{0.700pt}{0.400pt}}
\multiput(1261.00,484.17)(1.547,2.000){2}{\rule{0.350pt}{0.400pt}}
\put(1264,487.17){\rule{0.900pt}{0.400pt}}
\multiput(1264.00,486.17)(2.132,2.000){2}{\rule{0.450pt}{0.400pt}}
\put(1268,489.17){\rule{0.700pt}{0.400pt}}
\multiput(1268.00,488.17)(1.547,2.000){2}{\rule{0.350pt}{0.400pt}}
\put(1271,491.17){\rule{0.900pt}{0.400pt}}
\multiput(1271.00,490.17)(2.132,2.000){2}{\rule{0.450pt}{0.400pt}}
\put(1275,493.17){\rule{0.700pt}{0.400pt}}
\multiput(1275.00,492.17)(1.547,2.000){2}{\rule{0.350pt}{0.400pt}}
\put(1278,495.17){\rule{0.700pt}{0.400pt}}
\multiput(1278.00,494.17)(1.547,2.000){2}{\rule{0.350pt}{0.400pt}}
\put(1281,497.17){\rule{0.900pt}{0.400pt}}
\multiput(1281.00,496.17)(2.132,2.000){2}{\rule{0.450pt}{0.400pt}}
\put(1285,499.17){\rule{0.700pt}{0.400pt}}
\multiput(1285.00,498.17)(1.547,2.000){2}{\rule{0.350pt}{0.400pt}}
\put(1288,501.17){\rule{0.900pt}{0.400pt}}
\multiput(1288.00,500.17)(2.132,2.000){2}{\rule{0.450pt}{0.400pt}}
\put(1292,503.17){\rule{0.700pt}{0.400pt}}
\multiput(1292.00,502.17)(1.547,2.000){2}{\rule{0.350pt}{0.400pt}}
\put(1295,505.17){\rule{0.700pt}{0.400pt}}
\multiput(1295.00,504.17)(1.547,2.000){2}{\rule{0.350pt}{0.400pt}}
\put(1298,507.17){\rule{0.900pt}{0.400pt}}
\multiput(1298.00,506.17)(2.132,2.000){2}{\rule{0.450pt}{0.400pt}}
\put(1302,509.17){\rule{0.700pt}{0.400pt}}
\multiput(1302.00,508.17)(1.547,2.000){2}{\rule{0.350pt}{0.400pt}}
\put(1305,511.17){\rule{0.900pt}{0.400pt}}
\multiput(1305.00,510.17)(2.132,2.000){2}{\rule{0.450pt}{0.400pt}}
\put(1309,513.17){\rule{0.700pt}{0.400pt}}
\multiput(1309.00,512.17)(1.547,2.000){2}{\rule{0.350pt}{0.400pt}}
\put(1312,515.17){\rule{0.700pt}{0.400pt}}
\multiput(1312.00,514.17)(1.547,2.000){2}{\rule{0.350pt}{0.400pt}}
\put(1315,517.17){\rule{0.900pt}{0.400pt}}
\multiput(1315.00,516.17)(2.132,2.000){2}{\rule{0.450pt}{0.400pt}}
\put(1319,519.17){\rule{0.700pt}{0.400pt}}
\multiput(1319.00,518.17)(1.547,2.000){2}{\rule{0.350pt}{0.400pt}}
\put(1322,521.17){\rule{0.900pt}{0.400pt}}
\multiput(1322.00,520.17)(2.132,2.000){2}{\rule{0.450pt}{0.400pt}}
\put(1326,523.17){\rule{0.700pt}{0.400pt}}
\multiput(1326.00,522.17)(1.547,2.000){2}{\rule{0.350pt}{0.400pt}}
\put(1329,525.17){\rule{0.900pt}{0.400pt}}
\multiput(1329.00,524.17)(2.132,2.000){2}{\rule{0.450pt}{0.400pt}}
\put(1333,527.17){\rule{0.700pt}{0.400pt}}
\multiput(1333.00,526.17)(1.547,2.000){2}{\rule{0.350pt}{0.400pt}}
\put(1336,529.17){\rule{0.700pt}{0.400pt}}
\multiput(1336.00,528.17)(1.547,2.000){2}{\rule{0.350pt}{0.400pt}}
\put(1339,531.17){\rule{0.900pt}{0.400pt}}
\multiput(1339.00,530.17)(2.132,2.000){2}{\rule{0.450pt}{0.400pt}}
\put(1343,533.17){\rule{0.700pt}{0.400pt}}
\multiput(1343.00,532.17)(1.547,2.000){2}{\rule{0.350pt}{0.400pt}}
\put(1346,535.17){\rule{0.900pt}{0.400pt}}
\multiput(1346.00,534.17)(2.132,2.000){2}{\rule{0.450pt}{0.400pt}}
\put(1350,537.17){\rule{0.700pt}{0.400pt}}
\multiput(1350.00,536.17)(1.547,2.000){2}{\rule{0.350pt}{0.400pt}}
\put(1353,539.17){\rule{0.700pt}{0.400pt}}
\multiput(1353.00,538.17)(1.547,2.000){2}{\rule{0.350pt}{0.400pt}}
\put(1356,541.17){\rule{0.900pt}{0.400pt}}
\multiput(1356.00,540.17)(2.132,2.000){2}{\rule{0.450pt}{0.400pt}}
\put(1360,543.17){\rule{0.700pt}{0.400pt}}
\multiput(1360.00,542.17)(1.547,2.000){2}{\rule{0.350pt}{0.400pt}}
\put(1363,545.17){\rule{0.900pt}{0.400pt}}
\multiput(1363.00,544.17)(2.132,2.000){2}{\rule{0.450pt}{0.400pt}}
\put(1367,547.17){\rule{0.700pt}{0.400pt}}
\multiput(1367.00,546.17)(1.547,2.000){2}{\rule{0.350pt}{0.400pt}}
\put(1370,549.17){\rule{0.700pt}{0.400pt}}
\multiput(1370.00,548.17)(1.547,2.000){2}{\rule{0.350pt}{0.400pt}}
\put(1373,551.17){\rule{0.900pt}{0.400pt}}
\multiput(1373.00,550.17)(2.132,2.000){2}{\rule{0.450pt}{0.400pt}}
\put(1377,553.17){\rule{0.700pt}{0.400pt}}
\multiput(1377.00,552.17)(1.547,2.000){2}{\rule{0.350pt}{0.400pt}}
\put(1380,555.17){\rule{0.900pt}{0.400pt}}
\multiput(1380.00,554.17)(2.132,2.000){2}{\rule{0.450pt}{0.400pt}}
\put(1384,557.17){\rule{0.700pt}{0.400pt}}
\multiput(1384.00,556.17)(1.547,2.000){2}{\rule{0.350pt}{0.400pt}}
\put(1387,559.17){\rule{0.900pt}{0.400pt}}
\multiput(1387.00,558.17)(2.132,2.000){2}{\rule{0.450pt}{0.400pt}}
\put(1391,561.17){\rule{0.700pt}{0.400pt}}
\multiput(1391.00,560.17)(1.547,2.000){2}{\rule{0.350pt}{0.400pt}}
\put(1394,563.17){\rule{0.700pt}{0.400pt}}
\multiput(1394.00,562.17)(1.547,2.000){2}{\rule{0.350pt}{0.400pt}}
\put(1397,565.17){\rule{0.900pt}{0.400pt}}
\multiput(1397.00,564.17)(2.132,2.000){2}{\rule{0.450pt}{0.400pt}}
\put(1401,567.17){\rule{0.700pt}{0.400pt}}
\multiput(1401.00,566.17)(1.547,2.000){2}{\rule{0.350pt}{0.400pt}}
\put(1404,569.17){\rule{0.900pt}{0.400pt}}
\multiput(1404.00,568.17)(2.132,2.000){2}{\rule{0.450pt}{0.400pt}}
\put(1408,571.17){\rule{0.700pt}{0.400pt}}
\multiput(1408.00,570.17)(1.547,2.000){2}{\rule{0.350pt}{0.400pt}}
\put(1411,573.17){\rule{0.700pt}{0.400pt}}
\multiput(1411.00,572.17)(1.547,2.000){2}{\rule{0.350pt}{0.400pt}}
\put(1411,575.17){\rule{0.700pt}{0.400pt}}
\multiput(1412.55,574.17)(-1.547,2.000){2}{\rule{0.350pt}{0.400pt}}
\put(1408,577.17){\rule{0.700pt}{0.400pt}}
\multiput(1409.55,576.17)(-1.547,2.000){2}{\rule{0.350pt}{0.400pt}}
\put(1404,579.17){\rule{0.900pt}{0.400pt}}
\multiput(1406.13,578.17)(-2.132,2.000){2}{\rule{0.450pt}{0.400pt}}
\put(1401,581.17){\rule{0.700pt}{0.400pt}}
\multiput(1402.55,580.17)(-1.547,2.000){2}{\rule{0.350pt}{0.400pt}}
\put(1397,583.17){\rule{0.900pt}{0.400pt}}
\multiput(1399.13,582.17)(-2.132,2.000){2}{\rule{0.450pt}{0.400pt}}
\put(1394,585.17){\rule{0.700pt}{0.400pt}}
\multiput(1395.55,584.17)(-1.547,2.000){2}{\rule{0.350pt}{0.400pt}}
\put(1391,587.17){\rule{0.700pt}{0.400pt}}
\multiput(1392.55,586.17)(-1.547,2.000){2}{\rule{0.350pt}{0.400pt}}
\put(1387,589.17){\rule{0.900pt}{0.400pt}}
\multiput(1389.13,588.17)(-2.132,2.000){2}{\rule{0.450pt}{0.400pt}}
\put(1384,591.17){\rule{0.700pt}{0.400pt}}
\multiput(1385.55,590.17)(-1.547,2.000){2}{\rule{0.350pt}{0.400pt}}
\put(1380,593.17){\rule{0.900pt}{0.400pt}}
\multiput(1382.13,592.17)(-2.132,2.000){2}{\rule{0.450pt}{0.400pt}}
\put(1377,595.17){\rule{0.700pt}{0.400pt}}
\multiput(1378.55,594.17)(-1.547,2.000){2}{\rule{0.350pt}{0.400pt}}
\put(1373,597.17){\rule{0.900pt}{0.400pt}}
\multiput(1375.13,596.17)(-2.132,2.000){2}{\rule{0.450pt}{0.400pt}}
\put(1370,599.17){\rule{0.700pt}{0.400pt}}
\multiput(1371.55,598.17)(-1.547,2.000){2}{\rule{0.350pt}{0.400pt}}
\put(1367,601.17){\rule{0.700pt}{0.400pt}}
\multiput(1368.55,600.17)(-1.547,2.000){2}{\rule{0.350pt}{0.400pt}}
\put(1363,603.17){\rule{0.900pt}{0.400pt}}
\multiput(1365.13,602.17)(-2.132,2.000){2}{\rule{0.450pt}{0.400pt}}
\put(1360,605.17){\rule{0.700pt}{0.400pt}}
\multiput(1361.55,604.17)(-1.547,2.000){2}{\rule{0.350pt}{0.400pt}}
\put(1356,607.17){\rule{0.900pt}{0.400pt}}
\multiput(1358.13,606.17)(-2.132,2.000){2}{\rule{0.450pt}{0.400pt}}
\put(1353,609.17){\rule{0.700pt}{0.400pt}}
\multiput(1354.55,608.17)(-1.547,2.000){2}{\rule{0.350pt}{0.400pt}}
\put(1350,611.17){\rule{0.700pt}{0.400pt}}
\multiput(1351.55,610.17)(-1.547,2.000){2}{\rule{0.350pt}{0.400pt}}
\put(1346,613.17){\rule{0.900pt}{0.400pt}}
\multiput(1348.13,612.17)(-2.132,2.000){2}{\rule{0.450pt}{0.400pt}}
\put(1343,615.17){\rule{0.700pt}{0.400pt}}
\multiput(1344.55,614.17)(-1.547,2.000){2}{\rule{0.350pt}{0.400pt}}
\put(1339,617.17){\rule{0.900pt}{0.400pt}}
\multiput(1341.13,616.17)(-2.132,2.000){2}{\rule{0.450pt}{0.400pt}}
\put(1336,619.17){\rule{0.700pt}{0.400pt}}
\multiput(1337.55,618.17)(-1.547,2.000){2}{\rule{0.350pt}{0.400pt}}
\put(1333,621.17){\rule{0.700pt}{0.400pt}}
\multiput(1334.55,620.17)(-1.547,2.000){2}{\rule{0.350pt}{0.400pt}}
\put(1329,623.17){\rule{0.900pt}{0.400pt}}
\multiput(1331.13,622.17)(-2.132,2.000){2}{\rule{0.450pt}{0.400pt}}
\put(1326,625.17){\rule{0.700pt}{0.400pt}}
\multiput(1327.55,624.17)(-1.547,2.000){2}{\rule{0.350pt}{0.400pt}}
\put(1322,627.17){\rule{0.900pt}{0.400pt}}
\multiput(1324.13,626.17)(-2.132,2.000){2}{\rule{0.450pt}{0.400pt}}
\put(1319,628.67){\rule{0.723pt}{0.400pt}}
\multiput(1320.50,628.17)(-1.500,1.000){2}{\rule{0.361pt}{0.400pt}}
\put(1315,630.17){\rule{0.900pt}{0.400pt}}
\multiput(1317.13,629.17)(-2.132,2.000){2}{\rule{0.450pt}{0.400pt}}
\put(1312,632.17){\rule{0.700pt}{0.400pt}}
\multiput(1313.55,631.17)(-1.547,2.000){2}{\rule{0.350pt}{0.400pt}}
\put(1309,634.17){\rule{0.700pt}{0.400pt}}
\multiput(1310.55,633.17)(-1.547,2.000){2}{\rule{0.350pt}{0.400pt}}
\put(1305,636.17){\rule{0.900pt}{0.400pt}}
\multiput(1307.13,635.17)(-2.132,2.000){2}{\rule{0.450pt}{0.400pt}}
\put(1302,638.17){\rule{0.700pt}{0.400pt}}
\multiput(1303.55,637.17)(-1.547,2.000){2}{\rule{0.350pt}{0.400pt}}
\put(1298,640.17){\rule{0.900pt}{0.400pt}}
\multiput(1300.13,639.17)(-2.132,2.000){2}{\rule{0.450pt}{0.400pt}}
\put(1295,642.17){\rule{0.700pt}{0.400pt}}
\multiput(1296.55,641.17)(-1.547,2.000){2}{\rule{0.350pt}{0.400pt}}
\put(1292,644.17){\rule{0.700pt}{0.400pt}}
\multiput(1293.55,643.17)(-1.547,2.000){2}{\rule{0.350pt}{0.400pt}}
\put(1288,646.17){\rule{0.900pt}{0.400pt}}
\multiput(1290.13,645.17)(-2.132,2.000){2}{\rule{0.450pt}{0.400pt}}
\put(1285,648.17){\rule{0.700pt}{0.400pt}}
\multiput(1286.55,647.17)(-1.547,2.000){2}{\rule{0.350pt}{0.400pt}}
\put(1281,650.17){\rule{0.900pt}{0.400pt}}
\multiput(1283.13,649.17)(-2.132,2.000){2}{\rule{0.450pt}{0.400pt}}
\put(1278,652.17){\rule{0.700pt}{0.400pt}}
\multiput(1279.55,651.17)(-1.547,2.000){2}{\rule{0.350pt}{0.400pt}}
\put(1275,654.17){\rule{0.700pt}{0.400pt}}
\multiput(1276.55,653.17)(-1.547,2.000){2}{\rule{0.350pt}{0.400pt}}
\put(1271,656.17){\rule{0.900pt}{0.400pt}}
\multiput(1273.13,655.17)(-2.132,2.000){2}{\rule{0.450pt}{0.400pt}}
\put(1268,658.17){\rule{0.700pt}{0.400pt}}
\multiput(1269.55,657.17)(-1.547,2.000){2}{\rule{0.350pt}{0.400pt}}
\put(1264,660.17){\rule{0.900pt}{0.400pt}}
\multiput(1266.13,659.17)(-2.132,2.000){2}{\rule{0.450pt}{0.400pt}}
\put(1261,662.17){\rule{0.700pt}{0.400pt}}
\multiput(1262.55,661.17)(-1.547,2.000){2}{\rule{0.350pt}{0.400pt}}
\put(1258,664.17){\rule{0.700pt}{0.400pt}}
\multiput(1259.55,663.17)(-1.547,2.000){2}{\rule{0.350pt}{0.400pt}}
\put(1254,666.17){\rule{0.900pt}{0.400pt}}
\multiput(1256.13,665.17)(-2.132,2.000){2}{\rule{0.450pt}{0.400pt}}
\put(1251,668.17){\rule{0.700pt}{0.400pt}}
\multiput(1252.55,667.17)(-1.547,2.000){2}{\rule{0.350pt}{0.400pt}}
\put(1247,670.17){\rule{0.900pt}{0.400pt}}
\multiput(1249.13,669.17)(-2.132,2.000){2}{\rule{0.450pt}{0.400pt}}
\put(1244,672.17){\rule{0.700pt}{0.400pt}}
\multiput(1245.55,671.17)(-1.547,2.000){2}{\rule{0.350pt}{0.400pt}}
\put(1240,674.17){\rule{0.900pt}{0.400pt}}
\multiput(1242.13,673.17)(-2.132,2.000){2}{\rule{0.450pt}{0.400pt}}
\put(1237,676.17){\rule{0.700pt}{0.400pt}}
\multiput(1238.55,675.17)(-1.547,2.000){2}{\rule{0.350pt}{0.400pt}}
\put(1234,678.17){\rule{0.700pt}{0.400pt}}
\multiput(1235.55,677.17)(-1.547,2.000){2}{\rule{0.350pt}{0.400pt}}
\put(1230,680.17){\rule{0.900pt}{0.400pt}}
\multiput(1232.13,679.17)(-2.132,2.000){2}{\rule{0.450pt}{0.400pt}}
\put(1227,682.17){\rule{0.700pt}{0.400pt}}
\multiput(1228.55,681.17)(-1.547,2.000){2}{\rule{0.350pt}{0.400pt}}
\put(1223,684.17){\rule{0.900pt}{0.400pt}}
\multiput(1225.13,683.17)(-2.132,2.000){2}{\rule{0.450pt}{0.400pt}}
\put(1220,686.17){\rule{0.700pt}{0.400pt}}
\multiput(1221.55,685.17)(-1.547,2.000){2}{\rule{0.350pt}{0.400pt}}
\put(1217,688.17){\rule{0.700pt}{0.400pt}}
\multiput(1218.55,687.17)(-1.547,2.000){2}{\rule{0.350pt}{0.400pt}}
\put(1213,690.17){\rule{0.900pt}{0.400pt}}
\multiput(1215.13,689.17)(-2.132,2.000){2}{\rule{0.450pt}{0.400pt}}
\put(1210,692.17){\rule{0.700pt}{0.400pt}}
\multiput(1211.55,691.17)(-1.547,2.000){2}{\rule{0.350pt}{0.400pt}}
\put(1206,694.17){\rule{0.900pt}{0.400pt}}
\multiput(1208.13,693.17)(-2.132,2.000){2}{\rule{0.450pt}{0.400pt}}
\put(1203,696.17){\rule{0.700pt}{0.400pt}}
\multiput(1204.55,695.17)(-1.547,2.000){2}{\rule{0.350pt}{0.400pt}}
\put(1200,698.17){\rule{0.700pt}{0.400pt}}
\multiput(1201.55,697.17)(-1.547,2.000){2}{\rule{0.350pt}{0.400pt}}
\put(1196,700.17){\rule{0.900pt}{0.400pt}}
\multiput(1198.13,699.17)(-2.132,2.000){2}{\rule{0.450pt}{0.400pt}}
\put(1193,702.17){\rule{0.700pt}{0.400pt}}
\multiput(1194.55,701.17)(-1.547,2.000){2}{\rule{0.350pt}{0.400pt}}
\put(1189,704.17){\rule{0.900pt}{0.400pt}}
\multiput(1191.13,703.17)(-2.132,2.000){2}{\rule{0.450pt}{0.400pt}}
\put(1186,706.17){\rule{0.700pt}{0.400pt}}
\multiput(1187.55,705.17)(-1.547,2.000){2}{\rule{0.350pt}{0.400pt}}
\put(1183,708.17){\rule{0.700pt}{0.400pt}}
\multiput(1184.55,707.17)(-1.547,2.000){2}{\rule{0.350pt}{0.400pt}}
\put(1179,710.17){\rule{0.900pt}{0.400pt}}
\multiput(1181.13,709.17)(-2.132,2.000){2}{\rule{0.450pt}{0.400pt}}
\put(1176,712.17){\rule{0.700pt}{0.400pt}}
\multiput(1177.55,711.17)(-1.547,2.000){2}{\rule{0.350pt}{0.400pt}}
\put(1172,714.17){\rule{0.900pt}{0.400pt}}
\multiput(1174.13,713.17)(-2.132,2.000){2}{\rule{0.450pt}{0.400pt}}
\put(1169,716.17){\rule{0.700pt}{0.400pt}}
\multiput(1170.55,715.17)(-1.547,2.000){2}{\rule{0.350pt}{0.400pt}}
\put(1165,718.17){\rule{0.900pt}{0.400pt}}
\multiput(1167.13,717.17)(-2.132,2.000){2}{\rule{0.450pt}{0.400pt}}
\put(1162,720.17){\rule{0.700pt}{0.400pt}}
\multiput(1163.55,719.17)(-1.547,2.000){2}{\rule{0.350pt}{0.400pt}}
\put(1159,722.17){\rule{0.700pt}{0.400pt}}
\multiput(1160.55,721.17)(-1.547,2.000){2}{\rule{0.350pt}{0.400pt}}
\put(1155,724.17){\rule{0.900pt}{0.400pt}}
\multiput(1157.13,723.17)(-2.132,2.000){2}{\rule{0.450pt}{0.400pt}}
\put(1152,726.17){\rule{0.700pt}{0.400pt}}
\multiput(1153.55,725.17)(-1.547,2.000){2}{\rule{0.350pt}{0.400pt}}
\put(1148,728.17){\rule{0.900pt}{0.400pt}}
\multiput(1150.13,727.17)(-2.132,2.000){2}{\rule{0.450pt}{0.400pt}}
\put(1145,730.17){\rule{0.700pt}{0.400pt}}
\multiput(1146.55,729.17)(-1.547,2.000){2}{\rule{0.350pt}{0.400pt}}
\put(1142,732.17){\rule{0.700pt}{0.400pt}}
\multiput(1143.55,731.17)(-1.547,2.000){2}{\rule{0.350pt}{0.400pt}}
\put(1138,734.17){\rule{0.900pt}{0.400pt}}
\multiput(1140.13,733.17)(-2.132,2.000){2}{\rule{0.450pt}{0.400pt}}
\put(1135,736.17){\rule{0.700pt}{0.400pt}}
\multiput(1136.55,735.17)(-1.547,2.000){2}{\rule{0.350pt}{0.400pt}}
\put(1131,738.17){\rule{0.900pt}{0.400pt}}
\multiput(1133.13,737.17)(-2.132,2.000){2}{\rule{0.450pt}{0.400pt}}
\put(1128,740.17){\rule{0.700pt}{0.400pt}}
\multiput(1129.55,739.17)(-1.547,2.000){2}{\rule{0.350pt}{0.400pt}}
\put(1125,742.17){\rule{0.700pt}{0.400pt}}
\multiput(1126.55,741.17)(-1.547,2.000){2}{\rule{0.350pt}{0.400pt}}
\put(1121,744.17){\rule{0.900pt}{0.400pt}}
\multiput(1123.13,743.17)(-2.132,2.000){2}{\rule{0.450pt}{0.400pt}}
\put(1118,746.17){\rule{0.700pt}{0.400pt}}
\multiput(1119.55,745.17)(-1.547,2.000){2}{\rule{0.350pt}{0.400pt}}
\put(1114,748.17){\rule{0.900pt}{0.400pt}}
\multiput(1116.13,747.17)(-2.132,2.000){2}{\rule{0.450pt}{0.400pt}}
\put(1111,750.17){\rule{0.700pt}{0.400pt}}
\multiput(1112.55,749.17)(-1.547,2.000){2}{\rule{0.350pt}{0.400pt}}
\put(1108,752.17){\rule{0.700pt}{0.400pt}}
\multiput(1109.55,751.17)(-1.547,2.000){2}{\rule{0.350pt}{0.400pt}}
\put(1104,754.17){\rule{0.900pt}{0.400pt}}
\multiput(1106.13,753.17)(-2.132,2.000){2}{\rule{0.450pt}{0.400pt}}
\put(1101,756.17){\rule{0.700pt}{0.400pt}}
\multiput(1102.55,755.17)(-1.547,2.000){2}{\rule{0.350pt}{0.400pt}}
\put(1097,758.17){\rule{0.900pt}{0.400pt}}
\multiput(1099.13,757.17)(-2.132,2.000){2}{\rule{0.450pt}{0.400pt}}
\put(1094,760.17){\rule{0.700pt}{0.400pt}}
\multiput(1095.55,759.17)(-1.547,2.000){2}{\rule{0.350pt}{0.400pt}}
\put(1090,762.17){\rule{0.900pt}{0.400pt}}
\multiput(1092.13,761.17)(-2.132,2.000){2}{\rule{0.450pt}{0.400pt}}
\put(1087,764.17){\rule{0.700pt}{0.400pt}}
\multiput(1088.55,763.17)(-1.547,2.000){2}{\rule{0.350pt}{0.400pt}}
\put(1084,766.17){\rule{0.700pt}{0.400pt}}
\multiput(1085.55,765.17)(-1.547,2.000){2}{\rule{0.350pt}{0.400pt}}
\put(1080,768.17){\rule{0.900pt}{0.400pt}}
\multiput(1082.13,767.17)(-2.132,2.000){2}{\rule{0.450pt}{0.400pt}}
\put(1077,770.17){\rule{0.700pt}{0.400pt}}
\multiput(1078.55,769.17)(-1.547,2.000){2}{\rule{0.350pt}{0.400pt}}
\put(1073,772.17){\rule{0.900pt}{0.400pt}}
\multiput(1075.13,771.17)(-2.132,2.000){2}{\rule{0.450pt}{0.400pt}}
\put(1070,774.17){\rule{0.700pt}{0.400pt}}
\multiput(1071.55,773.17)(-1.547,2.000){2}{\rule{0.350pt}{0.400pt}}
\put(1067,776.17){\rule{0.700pt}{0.400pt}}
\multiput(1068.55,775.17)(-1.547,2.000){2}{\rule{0.350pt}{0.400pt}}
\put(1063,778.17){\rule{0.900pt}{0.400pt}}
\multiput(1065.13,777.17)(-2.132,2.000){2}{\rule{0.450pt}{0.400pt}}
\put(1060,780.17){\rule{0.700pt}{0.400pt}}
\multiput(1061.55,779.17)(-1.547,2.000){2}{\rule{0.350pt}{0.400pt}}
\put(1056,782.17){\rule{0.900pt}{0.400pt}}
\multiput(1058.13,781.17)(-2.132,2.000){2}{\rule{0.450pt}{0.400pt}}
\put(1053,784.17){\rule{0.700pt}{0.400pt}}
\multiput(1054.55,783.17)(-1.547,2.000){2}{\rule{0.350pt}{0.400pt}}
\put(1050,786.17){\rule{0.700pt}{0.400pt}}
\multiput(1051.55,785.17)(-1.547,2.000){2}{\rule{0.350pt}{0.400pt}}
\put(1046,787.67){\rule{0.964pt}{0.400pt}}
\multiput(1048.00,787.17)(-2.000,1.000){2}{\rule{0.482pt}{0.400pt}}
\put(1043,789.17){\rule{0.700pt}{0.400pt}}
\multiput(1044.55,788.17)(-1.547,2.000){2}{\rule{0.350pt}{0.400pt}}
\put(1039,791.17){\rule{0.900pt}{0.400pt}}
\multiput(1041.13,790.17)(-2.132,2.000){2}{\rule{0.450pt}{0.400pt}}
\put(1036,793.17){\rule{0.700pt}{0.400pt}}
\multiput(1037.55,792.17)(-1.547,2.000){2}{\rule{0.350pt}{0.400pt}}
\put(1032,795.17){\rule{0.900pt}{0.400pt}}
\multiput(1034.13,794.17)(-2.132,2.000){2}{\rule{0.450pt}{0.400pt}}
\put(1029,797.17){\rule{0.700pt}{0.400pt}}
\multiput(1030.55,796.17)(-1.547,2.000){2}{\rule{0.350pt}{0.400pt}}
\put(1026,799.17){\rule{0.700pt}{0.400pt}}
\multiput(1027.55,798.17)(-1.547,2.000){2}{\rule{0.350pt}{0.400pt}}
\put(1022,801.17){\rule{0.900pt}{0.400pt}}
\multiput(1024.13,800.17)(-2.132,2.000){2}{\rule{0.450pt}{0.400pt}}
\put(1019,803.17){\rule{0.700pt}{0.400pt}}
\multiput(1020.55,802.17)(-1.547,2.000){2}{\rule{0.350pt}{0.400pt}}
\put(1015,805.17){\rule{0.900pt}{0.400pt}}
\multiput(1017.13,804.17)(-2.132,2.000){2}{\rule{0.450pt}{0.400pt}}
\put(1012,807.17){\rule{0.700pt}{0.400pt}}
\multiput(1013.55,806.17)(-1.547,2.000){2}{\rule{0.350pt}{0.400pt}}
\put(1009,809.17){\rule{0.700pt}{0.400pt}}
\multiput(1010.55,808.17)(-1.547,2.000){2}{\rule{0.350pt}{0.400pt}}
\put(1005,811.17){\rule{0.900pt}{0.400pt}}
\multiput(1007.13,810.17)(-2.132,2.000){2}{\rule{0.450pt}{0.400pt}}
\put(1002,813.17){\rule{0.700pt}{0.400pt}}
\multiput(1003.55,812.17)(-1.547,2.000){2}{\rule{0.350pt}{0.400pt}}
\put(998,815.17){\rule{0.900pt}{0.400pt}}
\multiput(1000.13,814.17)(-2.132,2.000){2}{\rule{0.450pt}{0.400pt}}
\put(995,817.17){\rule{0.700pt}{0.400pt}}
\multiput(996.55,816.17)(-1.547,2.000){2}{\rule{0.350pt}{0.400pt}}
\put(992,819.17){\rule{0.700pt}{0.400pt}}
\multiput(993.55,818.17)(-1.547,2.000){2}{\rule{0.350pt}{0.400pt}}
\put(988,821.17){\rule{0.900pt}{0.400pt}}
\multiput(990.13,820.17)(-2.132,2.000){2}{\rule{0.450pt}{0.400pt}}
\put(985,823.17){\rule{0.700pt}{0.400pt}}
\multiput(986.55,822.17)(-1.547,2.000){2}{\rule{0.350pt}{0.400pt}}
\put(981,825.17){\rule{0.900pt}{0.400pt}}
\multiput(983.13,824.17)(-2.132,2.000){2}{\rule{0.450pt}{0.400pt}}
\put(978,827.17){\rule{0.700pt}{0.400pt}}
\multiput(979.55,826.17)(-1.547,2.000){2}{\rule{0.350pt}{0.400pt}}
\put(975,829.17){\rule{0.700pt}{0.400pt}}
\multiput(976.55,828.17)(-1.547,2.000){2}{\rule{0.350pt}{0.400pt}}
\put(971,831.17){\rule{0.900pt}{0.400pt}}
\multiput(973.13,830.17)(-2.132,2.000){2}{\rule{0.450pt}{0.400pt}}
\put(968,833.17){\rule{0.700pt}{0.400pt}}
\multiput(969.55,832.17)(-1.547,2.000){2}{\rule{0.350pt}{0.400pt}}
\put(964,835.17){\rule{0.900pt}{0.400pt}}
\multiput(966.13,834.17)(-2.132,2.000){2}{\rule{0.450pt}{0.400pt}}
\put(961,835.17){\rule{0.700pt}{0.400pt}}
\multiput(962.55,836.17)(-1.547,-2.000){2}{\rule{0.350pt}{0.400pt}}
\put(957,833.17){\rule{0.900pt}{0.400pt}}
\multiput(959.13,834.17)(-2.132,-2.000){2}{\rule{0.450pt}{0.400pt}}
\put(954,831.17){\rule{0.700pt}{0.400pt}}
\multiput(955.55,832.17)(-1.547,-2.000){2}{\rule{0.350pt}{0.400pt}}
\put(951,829.17){\rule{0.700pt}{0.400pt}}
\multiput(952.55,830.17)(-1.547,-2.000){2}{\rule{0.350pt}{0.400pt}}
\put(947,827.17){\rule{0.900pt}{0.400pt}}
\multiput(949.13,828.17)(-2.132,-2.000){2}{\rule{0.450pt}{0.400pt}}
\put(944,825.17){\rule{0.700pt}{0.400pt}}
\multiput(945.55,826.17)(-1.547,-2.000){2}{\rule{0.350pt}{0.400pt}}
\put(940,823.17){\rule{0.900pt}{0.400pt}}
\multiput(942.13,824.17)(-2.132,-2.000){2}{\rule{0.450pt}{0.400pt}}
\put(937,821.17){\rule{0.700pt}{0.400pt}}
\multiput(938.55,822.17)(-1.547,-2.000){2}{\rule{0.350pt}{0.400pt}}
\put(934,819.17){\rule{0.700pt}{0.400pt}}
\multiput(935.55,820.17)(-1.547,-2.000){2}{\rule{0.350pt}{0.400pt}}
\put(930,817.17){\rule{0.900pt}{0.400pt}}
\multiput(932.13,818.17)(-2.132,-2.000){2}{\rule{0.450pt}{0.400pt}}
\put(927,815.17){\rule{0.700pt}{0.400pt}}
\multiput(928.55,816.17)(-1.547,-2.000){2}{\rule{0.350pt}{0.400pt}}
\put(923,813.17){\rule{0.900pt}{0.400pt}}
\multiput(925.13,814.17)(-2.132,-2.000){2}{\rule{0.450pt}{0.400pt}}
\put(920,811.17){\rule{0.700pt}{0.400pt}}
\multiput(921.55,812.17)(-1.547,-2.000){2}{\rule{0.350pt}{0.400pt}}
\put(917,809.17){\rule{0.700pt}{0.400pt}}
\multiput(918.55,810.17)(-1.547,-2.000){2}{\rule{0.350pt}{0.400pt}}
\put(913,807.17){\rule{0.900pt}{0.400pt}}
\multiput(915.13,808.17)(-2.132,-2.000){2}{\rule{0.450pt}{0.400pt}}
\put(910,805.17){\rule{0.700pt}{0.400pt}}
\multiput(911.55,806.17)(-1.547,-2.000){2}{\rule{0.350pt}{0.400pt}}
\put(906,803.17){\rule{0.900pt}{0.400pt}}
\multiput(908.13,804.17)(-2.132,-2.000){2}{\rule{0.450pt}{0.400pt}}
\put(903,801.17){\rule{0.700pt}{0.400pt}}
\multiput(904.55,802.17)(-1.547,-2.000){2}{\rule{0.350pt}{0.400pt}}
\put(900,799.17){\rule{0.700pt}{0.400pt}}
\multiput(901.55,800.17)(-1.547,-2.000){2}{\rule{0.350pt}{0.400pt}}
\put(896,797.17){\rule{0.900pt}{0.400pt}}
\multiput(898.13,798.17)(-2.132,-2.000){2}{\rule{0.450pt}{0.400pt}}
\put(893,795.17){\rule{0.700pt}{0.400pt}}
\multiput(894.55,796.17)(-1.547,-2.000){2}{\rule{0.350pt}{0.400pt}}
\put(889,793.17){\rule{0.900pt}{0.400pt}}
\multiput(891.13,794.17)(-2.132,-2.000){2}{\rule{0.450pt}{0.400pt}}
\put(886,791.17){\rule{0.700pt}{0.400pt}}
\multiput(887.55,792.17)(-1.547,-2.000){2}{\rule{0.350pt}{0.400pt}}
\put(882,789.17){\rule{0.900pt}{0.400pt}}
\multiput(884.13,790.17)(-2.132,-2.000){2}{\rule{0.450pt}{0.400pt}}
\put(879,787.67){\rule{0.723pt}{0.400pt}}
\multiput(880.50,788.17)(-1.500,-1.000){2}{\rule{0.361pt}{0.400pt}}
\put(876,786.17){\rule{0.700pt}{0.400pt}}
\multiput(877.55,787.17)(-1.547,-2.000){2}{\rule{0.350pt}{0.400pt}}
\put(872,784.17){\rule{0.900pt}{0.400pt}}
\multiput(874.13,785.17)(-2.132,-2.000){2}{\rule{0.450pt}{0.400pt}}
\put(869,782.17){\rule{0.700pt}{0.400pt}}
\multiput(870.55,783.17)(-1.547,-2.000){2}{\rule{0.350pt}{0.400pt}}
\put(865,780.17){\rule{0.900pt}{0.400pt}}
\multiput(867.13,781.17)(-2.132,-2.000){2}{\rule{0.450pt}{0.400pt}}
\put(862,778.17){\rule{0.700pt}{0.400pt}}
\multiput(863.55,779.17)(-1.547,-2.000){2}{\rule{0.350pt}{0.400pt}}
\put(859,776.17){\rule{0.700pt}{0.400pt}}
\multiput(860.55,777.17)(-1.547,-2.000){2}{\rule{0.350pt}{0.400pt}}
\put(855,774.17){\rule{0.900pt}{0.400pt}}
\multiput(857.13,775.17)(-2.132,-2.000){2}{\rule{0.450pt}{0.400pt}}
\put(852,772.17){\rule{0.700pt}{0.400pt}}
\multiput(853.55,773.17)(-1.547,-2.000){2}{\rule{0.350pt}{0.400pt}}
\put(848,770.17){\rule{0.900pt}{0.400pt}}
\multiput(850.13,771.17)(-2.132,-2.000){2}{\rule{0.450pt}{0.400pt}}
\put(845,768.17){\rule{0.700pt}{0.400pt}}
\multiput(846.55,769.17)(-1.547,-2.000){2}{\rule{0.350pt}{0.400pt}}
\put(842,766.17){\rule{0.700pt}{0.400pt}}
\multiput(843.55,767.17)(-1.547,-2.000){2}{\rule{0.350pt}{0.400pt}}
\put(838,764.17){\rule{0.900pt}{0.400pt}}
\multiput(840.13,765.17)(-2.132,-2.000){2}{\rule{0.450pt}{0.400pt}}
\put(835,762.17){\rule{0.700pt}{0.400pt}}
\multiput(836.55,763.17)(-1.547,-2.000){2}{\rule{0.350pt}{0.400pt}}
\put(831,760.17){\rule{0.900pt}{0.400pt}}
\multiput(833.13,761.17)(-2.132,-2.000){2}{\rule{0.450pt}{0.400pt}}
\put(828,758.17){\rule{0.700pt}{0.400pt}}
\multiput(829.55,759.17)(-1.547,-2.000){2}{\rule{0.350pt}{0.400pt}}
\put(825,756.17){\rule{0.700pt}{0.400pt}}
\multiput(826.55,757.17)(-1.547,-2.000){2}{\rule{0.350pt}{0.400pt}}
\put(821,754.17){\rule{0.900pt}{0.400pt}}
\multiput(823.13,755.17)(-2.132,-2.000){2}{\rule{0.450pt}{0.400pt}}
\put(818,752.17){\rule{0.700pt}{0.400pt}}
\multiput(819.55,753.17)(-1.547,-2.000){2}{\rule{0.350pt}{0.400pt}}
\put(814,750.17){\rule{0.900pt}{0.400pt}}
\multiput(816.13,751.17)(-2.132,-2.000){2}{\rule{0.450pt}{0.400pt}}
\put(811,748.17){\rule{0.700pt}{0.400pt}}
\multiput(812.55,749.17)(-1.547,-2.000){2}{\rule{0.350pt}{0.400pt}}
\put(807,746.17){\rule{0.900pt}{0.400pt}}
\multiput(809.13,747.17)(-2.132,-2.000){2}{\rule{0.450pt}{0.400pt}}
\put(804,744.17){\rule{0.700pt}{0.400pt}}
\multiput(805.55,745.17)(-1.547,-2.000){2}{\rule{0.350pt}{0.400pt}}
\put(801,742.17){\rule{0.700pt}{0.400pt}}
\multiput(802.55,743.17)(-1.547,-2.000){2}{\rule{0.350pt}{0.400pt}}
\put(797,740.17){\rule{0.900pt}{0.400pt}}
\multiput(799.13,741.17)(-2.132,-2.000){2}{\rule{0.450pt}{0.400pt}}
\put(794,738.17){\rule{0.700pt}{0.400pt}}
\multiput(795.55,739.17)(-1.547,-2.000){2}{\rule{0.350pt}{0.400pt}}
\put(790,736.17){\rule{0.900pt}{0.400pt}}
\multiput(792.13,737.17)(-2.132,-2.000){2}{\rule{0.450pt}{0.400pt}}
\put(787,734.17){\rule{0.700pt}{0.400pt}}
\multiput(788.55,735.17)(-1.547,-2.000){2}{\rule{0.350pt}{0.400pt}}
\put(784,732.17){\rule{0.700pt}{0.400pt}}
\multiput(785.55,733.17)(-1.547,-2.000){2}{\rule{0.350pt}{0.400pt}}
\put(780,730.17){\rule{0.900pt}{0.400pt}}
\multiput(782.13,731.17)(-2.132,-2.000){2}{\rule{0.450pt}{0.400pt}}
\put(777,728.17){\rule{0.700pt}{0.400pt}}
\multiput(778.55,729.17)(-1.547,-2.000){2}{\rule{0.350pt}{0.400pt}}
\put(773,726.17){\rule{0.900pt}{0.400pt}}
\multiput(775.13,727.17)(-2.132,-2.000){2}{\rule{0.450pt}{0.400pt}}
\put(770,724.17){\rule{0.700pt}{0.400pt}}
\multiput(771.55,725.17)(-1.547,-2.000){2}{\rule{0.350pt}{0.400pt}}
\put(767,722.17){\rule{0.700pt}{0.400pt}}
\multiput(768.55,723.17)(-1.547,-2.000){2}{\rule{0.350pt}{0.400pt}}
\put(763,720.17){\rule{0.900pt}{0.400pt}}
\multiput(765.13,721.17)(-2.132,-2.000){2}{\rule{0.450pt}{0.400pt}}
\put(760,718.17){\rule{0.700pt}{0.400pt}}
\multiput(761.55,719.17)(-1.547,-2.000){2}{\rule{0.350pt}{0.400pt}}
\put(756,716.17){\rule{0.900pt}{0.400pt}}
\multiput(758.13,717.17)(-2.132,-2.000){2}{\rule{0.450pt}{0.400pt}}
\put(753,714.17){\rule{0.700pt}{0.400pt}}
\multiput(754.55,715.17)(-1.547,-2.000){2}{\rule{0.350pt}{0.400pt}}
\put(749,712.17){\rule{0.900pt}{0.400pt}}
\multiput(751.13,713.17)(-2.132,-2.000){2}{\rule{0.450pt}{0.400pt}}
\put(746,710.17){\rule{0.700pt}{0.400pt}}
\multiput(747.55,711.17)(-1.547,-2.000){2}{\rule{0.350pt}{0.400pt}}
\put(743,708.17){\rule{0.700pt}{0.400pt}}
\multiput(744.55,709.17)(-1.547,-2.000){2}{\rule{0.350pt}{0.400pt}}
\put(739,706.17){\rule{0.900pt}{0.400pt}}
\multiput(741.13,707.17)(-2.132,-2.000){2}{\rule{0.450pt}{0.400pt}}
\put(736,704.17){\rule{0.700pt}{0.400pt}}
\multiput(737.55,705.17)(-1.547,-2.000){2}{\rule{0.350pt}{0.400pt}}
\put(732,702.17){\rule{0.900pt}{0.400pt}}
\multiput(734.13,703.17)(-2.132,-2.000){2}{\rule{0.450pt}{0.400pt}}
\put(729,700.17){\rule{0.700pt}{0.400pt}}
\multiput(730.55,701.17)(-1.547,-2.000){2}{\rule{0.350pt}{0.400pt}}
\put(726,698.17){\rule{0.700pt}{0.400pt}}
\multiput(727.55,699.17)(-1.547,-2.000){2}{\rule{0.350pt}{0.400pt}}
\put(722,696.17){\rule{0.900pt}{0.400pt}}
\multiput(724.13,697.17)(-2.132,-2.000){2}{\rule{0.450pt}{0.400pt}}
\put(719,694.17){\rule{0.700pt}{0.400pt}}
\multiput(720.55,695.17)(-1.547,-2.000){2}{\rule{0.350pt}{0.400pt}}
\put(715,692.17){\rule{0.900pt}{0.400pt}}
\multiput(717.13,693.17)(-2.132,-2.000){2}{\rule{0.450pt}{0.400pt}}
\put(712,690.17){\rule{0.700pt}{0.400pt}}
\multiput(713.55,691.17)(-1.547,-2.000){2}{\rule{0.350pt}{0.400pt}}
\put(709,688.17){\rule{0.700pt}{0.400pt}}
\multiput(710.55,689.17)(-1.547,-2.000){2}{\rule{0.350pt}{0.400pt}}
\put(705,686.17){\rule{0.900pt}{0.400pt}}
\multiput(707.13,687.17)(-2.132,-2.000){2}{\rule{0.450pt}{0.400pt}}
\put(702,684.17){\rule{0.700pt}{0.400pt}}
\multiput(703.55,685.17)(-1.547,-2.000){2}{\rule{0.350pt}{0.400pt}}
\put(698,682.17){\rule{0.900pt}{0.400pt}}
\multiput(700.13,683.17)(-2.132,-2.000){2}{\rule{0.450pt}{0.400pt}}
\put(695,680.17){\rule{0.700pt}{0.400pt}}
\multiput(696.55,681.17)(-1.547,-2.000){2}{\rule{0.350pt}{0.400pt}}
\put(692,678.17){\rule{0.700pt}{0.400pt}}
\multiput(693.55,679.17)(-1.547,-2.000){2}{\rule{0.350pt}{0.400pt}}
\put(688,676.17){\rule{0.900pt}{0.400pt}}
\multiput(690.13,677.17)(-2.132,-2.000){2}{\rule{0.450pt}{0.400pt}}
\put(685,674.17){\rule{0.700pt}{0.400pt}}
\multiput(686.55,675.17)(-1.547,-2.000){2}{\rule{0.350pt}{0.400pt}}
\put(681,672.17){\rule{0.900pt}{0.400pt}}
\multiput(683.13,673.17)(-2.132,-2.000){2}{\rule{0.450pt}{0.400pt}}
\put(678,670.17){\rule{0.700pt}{0.400pt}}
\multiput(679.55,671.17)(-1.547,-2.000){2}{\rule{0.350pt}{0.400pt}}
\put(674,668.17){\rule{0.900pt}{0.400pt}}
\multiput(676.13,669.17)(-2.132,-2.000){2}{\rule{0.450pt}{0.400pt}}
\put(671,666.17){\rule{0.700pt}{0.400pt}}
\multiput(672.55,667.17)(-1.547,-2.000){2}{\rule{0.350pt}{0.400pt}}
\put(668,664.17){\rule{0.700pt}{0.400pt}}
\multiput(669.55,665.17)(-1.547,-2.000){2}{\rule{0.350pt}{0.400pt}}
\put(664,662.17){\rule{0.900pt}{0.400pt}}
\multiput(666.13,663.17)(-2.132,-2.000){2}{\rule{0.450pt}{0.400pt}}
\put(661,660.17){\rule{0.700pt}{0.400pt}}
\multiput(662.55,661.17)(-1.547,-2.000){2}{\rule{0.350pt}{0.400pt}}
\put(657,658.17){\rule{0.900pt}{0.400pt}}
\multiput(659.13,659.17)(-2.132,-2.000){2}{\rule{0.450pt}{0.400pt}}
\put(654,656.17){\rule{0.700pt}{0.400pt}}
\multiput(655.55,657.17)(-1.547,-2.000){2}{\rule{0.350pt}{0.400pt}}
\put(651,654.17){\rule{0.700pt}{0.400pt}}
\multiput(652.55,655.17)(-1.547,-2.000){2}{\rule{0.350pt}{0.400pt}}
\put(647,652.17){\rule{0.900pt}{0.400pt}}
\multiput(649.13,653.17)(-2.132,-2.000){2}{\rule{0.450pt}{0.400pt}}
\put(644,650.17){\rule{0.700pt}{0.400pt}}
\multiput(645.55,651.17)(-1.547,-2.000){2}{\rule{0.350pt}{0.400pt}}
\put(640,648.17){\rule{0.900pt}{0.400pt}}
\multiput(642.13,649.17)(-2.132,-2.000){2}{\rule{0.450pt}{0.400pt}}
\put(637,646.17){\rule{0.700pt}{0.400pt}}
\multiput(638.55,647.17)(-1.547,-2.000){2}{\rule{0.350pt}{0.400pt}}
\put(634,644.17){\rule{0.700pt}{0.400pt}}
\multiput(635.55,645.17)(-1.547,-2.000){2}{\rule{0.350pt}{0.400pt}}
\put(630,642.17){\rule{0.900pt}{0.400pt}}
\multiput(632.13,643.17)(-2.132,-2.000){2}{\rule{0.450pt}{0.400pt}}
\put(627,640.17){\rule{0.700pt}{0.400pt}}
\multiput(628.55,641.17)(-1.547,-2.000){2}{\rule{0.350pt}{0.400pt}}
\put(623,638.17){\rule{0.900pt}{0.400pt}}
\multiput(625.13,639.17)(-2.132,-2.000){2}{\rule{0.450pt}{0.400pt}}
\put(620,636.17){\rule{0.700pt}{0.400pt}}
\multiput(621.55,637.17)(-1.547,-2.000){2}{\rule{0.350pt}{0.400pt}}
\put(617,634.17){\rule{0.700pt}{0.400pt}}
\multiput(618.55,635.17)(-1.547,-2.000){2}{\rule{0.350pt}{0.400pt}}
\put(613,632.17){\rule{0.900pt}{0.400pt}}
\multiput(615.13,633.17)(-2.132,-2.000){2}{\rule{0.450pt}{0.400pt}}
\put(610,630.17){\rule{0.700pt}{0.400pt}}
\multiput(611.55,631.17)(-1.547,-2.000){2}{\rule{0.350pt}{0.400pt}}
\put(606,628.67){\rule{0.964pt}{0.400pt}}
\multiput(608.00,629.17)(-2.000,-1.000){2}{\rule{0.482pt}{0.400pt}}
\put(603,627.17){\rule{0.700pt}{0.400pt}}
\multiput(604.55,628.17)(-1.547,-2.000){2}{\rule{0.350pt}{0.400pt}}
\put(599,625.17){\rule{0.900pt}{0.400pt}}
\multiput(601.13,626.17)(-2.132,-2.000){2}{\rule{0.450pt}{0.400pt}}
\put(596,623.17){\rule{0.700pt}{0.400pt}}
\multiput(597.55,624.17)(-1.547,-2.000){2}{\rule{0.350pt}{0.400pt}}
\put(593,621.17){\rule{0.700pt}{0.400pt}}
\multiput(594.55,622.17)(-1.547,-2.000){2}{\rule{0.350pt}{0.400pt}}
\put(589,619.17){\rule{0.900pt}{0.400pt}}
\multiput(591.13,620.17)(-2.132,-2.000){2}{\rule{0.450pt}{0.400pt}}
\put(586,617.17){\rule{0.700pt}{0.400pt}}
\multiput(587.55,618.17)(-1.547,-2.000){2}{\rule{0.350pt}{0.400pt}}
\put(582,615.17){\rule{0.900pt}{0.400pt}}
\multiput(584.13,616.17)(-2.132,-2.000){2}{\rule{0.450pt}{0.400pt}}
\put(579,613.17){\rule{0.700pt}{0.400pt}}
\multiput(580.55,614.17)(-1.547,-2.000){2}{\rule{0.350pt}{0.400pt}}
\put(576,611.17){\rule{0.700pt}{0.400pt}}
\multiput(577.55,612.17)(-1.547,-2.000){2}{\rule{0.350pt}{0.400pt}}
\put(572,609.17){\rule{0.900pt}{0.400pt}}
\multiput(574.13,610.17)(-2.132,-2.000){2}{\rule{0.450pt}{0.400pt}}
\put(569,607.17){\rule{0.700pt}{0.400pt}}
\multiput(570.55,608.17)(-1.547,-2.000){2}{\rule{0.350pt}{0.400pt}}
\put(565,605.17){\rule{0.900pt}{0.400pt}}
\multiput(567.13,606.17)(-2.132,-2.000){2}{\rule{0.450pt}{0.400pt}}
\put(562,603.17){\rule{0.700pt}{0.400pt}}
\multiput(563.55,604.17)(-1.547,-2.000){2}{\rule{0.350pt}{0.400pt}}
\put(559,601.17){\rule{0.700pt}{0.400pt}}
\multiput(560.55,602.17)(-1.547,-2.000){2}{\rule{0.350pt}{0.400pt}}
\put(555,599.17){\rule{0.900pt}{0.400pt}}
\multiput(557.13,600.17)(-2.132,-2.000){2}{\rule{0.450pt}{0.400pt}}
\put(552,597.17){\rule{0.700pt}{0.400pt}}
\multiput(553.55,598.17)(-1.547,-2.000){2}{\rule{0.350pt}{0.400pt}}
\put(548,595.17){\rule{0.900pt}{0.400pt}}
\multiput(550.13,596.17)(-2.132,-2.000){2}{\rule{0.450pt}{0.400pt}}
\put(545,593.17){\rule{0.700pt}{0.400pt}}
\multiput(546.55,594.17)(-1.547,-2.000){2}{\rule{0.350pt}{0.400pt}}
\put(542,591.17){\rule{0.700pt}{0.400pt}}
\multiput(543.55,592.17)(-1.547,-2.000){2}{\rule{0.350pt}{0.400pt}}
\put(538,589.17){\rule{0.900pt}{0.400pt}}
\multiput(540.13,590.17)(-2.132,-2.000){2}{\rule{0.450pt}{0.400pt}}
\put(535,587.17){\rule{0.700pt}{0.400pt}}
\multiput(536.55,588.17)(-1.547,-2.000){2}{\rule{0.350pt}{0.400pt}}
\put(531,585.17){\rule{0.900pt}{0.400pt}}
\multiput(533.13,586.17)(-2.132,-2.000){2}{\rule{0.450pt}{0.400pt}}
\put(528,583.17){\rule{0.700pt}{0.400pt}}
\multiput(529.55,584.17)(-1.547,-2.000){2}{\rule{0.350pt}{0.400pt}}
\put(524,581.17){\rule{0.900pt}{0.400pt}}
\multiput(526.13,582.17)(-2.132,-2.000){2}{\rule{0.450pt}{0.400pt}}
\put(521,579.17){\rule{0.700pt}{0.400pt}}
\multiput(522.55,580.17)(-1.547,-2.000){2}{\rule{0.350pt}{0.400pt}}
\put(518,577.17){\rule{0.700pt}{0.400pt}}
\multiput(519.55,578.17)(-1.547,-2.000){2}{\rule{0.350pt}{0.400pt}}
\put(514,575.17){\rule{0.900pt}{0.400pt}}
\multiput(516.13,576.17)(-2.132,-2.000){2}{\rule{0.450pt}{0.400pt}}
\put(511,573.17){\rule{0.700pt}{0.400pt}}
\multiput(512.55,574.17)(-1.547,-2.000){2}{\rule{0.350pt}{0.400pt}}
\put(507,571.17){\rule{0.900pt}{0.400pt}}
\multiput(509.13,572.17)(-2.132,-2.000){2}{\rule{0.450pt}{0.400pt}}
\put(504,569.17){\rule{0.700pt}{0.400pt}}
\multiput(505.55,570.17)(-1.547,-2.000){2}{\rule{0.350pt}{0.400pt}}
\put(501,567.17){\rule{0.700pt}{0.400pt}}
\multiput(502.55,568.17)(-1.547,-2.000){2}{\rule{0.350pt}{0.400pt}}
\put(497,565.17){\rule{0.900pt}{0.400pt}}
\multiput(499.13,566.17)(-2.132,-2.000){2}{\rule{0.450pt}{0.400pt}}
\put(494,563.17){\rule{0.700pt}{0.400pt}}
\multiput(495.55,564.17)(-1.547,-2.000){2}{\rule{0.350pt}{0.400pt}}
\put(490,561.17){\rule{0.900pt}{0.400pt}}
\multiput(492.13,562.17)(-2.132,-2.000){2}{\rule{0.450pt}{0.400pt}}
\put(487,559.17){\rule{0.700pt}{0.400pt}}
\multiput(488.55,560.17)(-1.547,-2.000){2}{\rule{0.350pt}{0.400pt}}
\put(484,557.17){\rule{0.700pt}{0.400pt}}
\multiput(485.55,558.17)(-1.547,-2.000){2}{\rule{0.350pt}{0.400pt}}
\put(480,555.17){\rule{0.900pt}{0.400pt}}
\multiput(482.13,556.17)(-2.132,-2.000){2}{\rule{0.450pt}{0.400pt}}
\put(477,553.17){\rule{0.700pt}{0.400pt}}
\multiput(478.55,554.17)(-1.547,-2.000){2}{\rule{0.350pt}{0.400pt}}
\put(473,551.17){\rule{0.900pt}{0.400pt}}
\multiput(475.13,552.17)(-2.132,-2.000){2}{\rule{0.450pt}{0.400pt}}
\put(130.0,82.0){\rule[-0.200pt]{0.400pt}{187.179pt}}
\put(130.0,82.0){\rule[-0.200pt]{315.338pt}{0.400pt}}
\put(1439.0,82.0){\rule[-0.200pt]{0.400pt}{187.179pt}}
\put(130.0,859.0){\rule[-0.200pt]{315.338pt}{0.400pt}}
\end{picture}

Plot for Ball 3:\\
% GNUPLOT: LaTeX picture
\setlength{\unitlength}{0.240900pt}
\ifx\plotpoint\undefined\newsavebox{\plotpoint}\fi
\sbox{\plotpoint}{\rule[-0.200pt]{0.400pt}{0.400pt}}%
\begin{picture}(1500,900)(0,0)
\sbox{\plotpoint}{\rule[-0.200pt]{0.400pt}{0.400pt}}%
\put(130.0,90.0){\rule[-0.200pt]{4.818pt}{0.400pt}}
\put(110,90){\makebox(0,0)[r]{ 0}}
\put(1419.0,90.0){\rule[-0.200pt]{4.818pt}{0.400pt}}
\put(130.0,242.0){\rule[-0.200pt]{4.818pt}{0.400pt}}
\put(110,242){\makebox(0,0)[r]{ 0.2}}
\put(1419.0,242.0){\rule[-0.200pt]{4.818pt}{0.400pt}}
\put(130.0,394.0){\rule[-0.200pt]{4.818pt}{0.400pt}}
\put(110,394){\makebox(0,0)[r]{ 0.4}}
\put(1419.0,394.0){\rule[-0.200pt]{4.818pt}{0.400pt}}
\put(130.0,547.0){\rule[-0.200pt]{4.818pt}{0.400pt}}
\put(110,547){\makebox(0,0)[r]{ 0.6}}
\put(1419.0,547.0){\rule[-0.200pt]{4.818pt}{0.400pt}}
\put(130.0,699.0){\rule[-0.200pt]{4.818pt}{0.400pt}}
\put(110,699){\makebox(0,0)[r]{ 0.8}}
\put(1419.0,699.0){\rule[-0.200pt]{4.818pt}{0.400pt}}
\put(130.0,851.0){\rule[-0.200pt]{4.818pt}{0.400pt}}
\put(110,851){\makebox(0,0)[r]{ 1}}
\put(1419.0,851.0){\rule[-0.200pt]{4.818pt}{0.400pt}}
\put(130.0,82.0){\rule[-0.200pt]{0.400pt}{4.818pt}}
\put(130,41){\makebox(0,0){ 0}}
\put(130.0,839.0){\rule[-0.200pt]{0.400pt}{4.818pt}}
\put(392.0,82.0){\rule[-0.200pt]{0.400pt}{4.818pt}}
\put(392,41){\makebox(0,0){ 0.2}}
\put(392.0,839.0){\rule[-0.200pt]{0.400pt}{4.818pt}}
\put(654.0,82.0){\rule[-0.200pt]{0.400pt}{4.818pt}}
\put(654,41){\makebox(0,0){ 0.4}}
\put(654.0,839.0){\rule[-0.200pt]{0.400pt}{4.818pt}}
\put(915.0,82.0){\rule[-0.200pt]{0.400pt}{4.818pt}}
\put(915,41){\makebox(0,0){ 0.6}}
\put(915.0,839.0){\rule[-0.200pt]{0.400pt}{4.818pt}}
\put(1177.0,82.0){\rule[-0.200pt]{0.400pt}{4.818pt}}
\put(1177,41){\makebox(0,0){ 0.8}}
\put(1177.0,839.0){\rule[-0.200pt]{0.400pt}{4.818pt}}
\put(1439.0,82.0){\rule[-0.200pt]{0.400pt}{4.818pt}}
\put(1439,41){\makebox(0,0){ 1}}
\put(1439.0,839.0){\rule[-0.200pt]{0.400pt}{4.818pt}}
\put(130.0,82.0){\rule[-0.200pt]{0.400pt}{187.179pt}}
\put(130.0,82.0){\rule[-0.200pt]{315.338pt}{0.400pt}}
\put(1439.0,82.0){\rule[-0.200pt]{0.400pt}{187.179pt}}
\put(130.0,859.0){\rule[-0.200pt]{315.338pt}{0.400pt}}
\put(1279,819){\makebox(0,0)[r]{'-'}}
\put(1299.0,819.0){\rule[-0.200pt]{24.090pt}{0.400pt}}
\put(1116,241){\usebox{\plotpoint}}
\multiput(1113.92,241.61)(-0.462,0.447){3}{\rule{0.500pt}{0.108pt}}
\multiput(1114.96,240.17)(-1.962,3.000){2}{\rule{0.250pt}{0.400pt}}
\put(1111.17,244){\rule{0.400pt}{0.700pt}}
\multiput(1112.17,244.00)(-2.000,1.547){2}{\rule{0.400pt}{0.350pt}}
\multiput(1108.92,247.61)(-0.462,0.447){3}{\rule{0.500pt}{0.108pt}}
\multiput(1109.96,246.17)(-1.962,3.000){2}{\rule{0.250pt}{0.400pt}}
\put(1106.17,250){\rule{0.400pt}{0.900pt}}
\multiput(1107.17,250.00)(-2.000,2.132){2}{\rule{0.400pt}{0.450pt}}
\multiput(1103.92,254.61)(-0.462,0.447){3}{\rule{0.500pt}{0.108pt}}
\multiput(1104.96,253.17)(-1.962,3.000){2}{\rule{0.250pt}{0.400pt}}
\multiput(1100.92,257.61)(-0.462,0.447){3}{\rule{0.500pt}{0.108pt}}
\multiput(1101.96,256.17)(-1.962,3.000){2}{\rule{0.250pt}{0.400pt}}
\put(1098.17,260){\rule{0.400pt}{0.700pt}}
\multiput(1099.17,260.00)(-2.000,1.547){2}{\rule{0.400pt}{0.350pt}}
\multiput(1096.95,263.00)(-0.447,0.685){3}{\rule{0.108pt}{0.633pt}}
\multiput(1097.17,263.00)(-3.000,2.685){2}{\rule{0.400pt}{0.317pt}}
\put(1093.17,267){\rule{0.400pt}{0.700pt}}
\multiput(1094.17,267.00)(-2.000,1.547){2}{\rule{0.400pt}{0.350pt}}
\multiput(1090.92,270.61)(-0.462,0.447){3}{\rule{0.500pt}{0.108pt}}
\multiput(1091.96,269.17)(-1.962,3.000){2}{\rule{0.250pt}{0.400pt}}
\multiput(1087.92,273.61)(-0.462,0.447){3}{\rule{0.500pt}{0.108pt}}
\multiput(1088.96,272.17)(-1.962,3.000){2}{\rule{0.250pt}{0.400pt}}
\put(1085.17,276){\rule{0.400pt}{0.900pt}}
\multiput(1086.17,276.00)(-2.000,2.132){2}{\rule{0.400pt}{0.450pt}}
\multiput(1082.92,280.61)(-0.462,0.447){3}{\rule{0.500pt}{0.108pt}}
\multiput(1083.96,279.17)(-1.962,3.000){2}{\rule{0.250pt}{0.400pt}}
\multiput(1079.92,283.61)(-0.462,0.447){3}{\rule{0.500pt}{0.108pt}}
\multiput(1080.96,282.17)(-1.962,3.000){2}{\rule{0.250pt}{0.400pt}}
\put(1077.17,286){\rule{0.400pt}{0.700pt}}
\multiput(1078.17,286.00)(-2.000,1.547){2}{\rule{0.400pt}{0.350pt}}
\multiput(1074.92,289.61)(-0.462,0.447){3}{\rule{0.500pt}{0.108pt}}
\multiput(1075.96,288.17)(-1.962,3.000){2}{\rule{0.250pt}{0.400pt}}
\put(1072.17,292){\rule{0.400pt}{0.900pt}}
\multiput(1073.17,292.00)(-2.000,2.132){2}{\rule{0.400pt}{0.450pt}}
\multiput(1069.92,296.61)(-0.462,0.447){3}{\rule{0.500pt}{0.108pt}}
\multiput(1070.96,295.17)(-1.962,3.000){2}{\rule{0.250pt}{0.400pt}}
\multiput(1066.92,299.61)(-0.462,0.447){3}{\rule{0.500pt}{0.108pt}}
\multiput(1067.96,298.17)(-1.962,3.000){2}{\rule{0.250pt}{0.400pt}}
\put(1064.17,302){\rule{0.400pt}{0.700pt}}
\multiput(1065.17,302.00)(-2.000,1.547){2}{\rule{0.400pt}{0.350pt}}
\multiput(1062.95,305.00)(-0.447,0.685){3}{\rule{0.108pt}{0.633pt}}
\multiput(1063.17,305.00)(-3.000,2.685){2}{\rule{0.400pt}{0.317pt}}
\multiput(1058.92,309.61)(-0.462,0.447){3}{\rule{0.500pt}{0.108pt}}
\multiput(1059.96,308.17)(-1.962,3.000){2}{\rule{0.250pt}{0.400pt}}
\put(1056.17,312){\rule{0.400pt}{0.700pt}}
\multiput(1057.17,312.00)(-2.000,1.547){2}{\rule{0.400pt}{0.350pt}}
\multiput(1053.92,315.61)(-0.462,0.447){3}{\rule{0.500pt}{0.108pt}}
\multiput(1054.96,314.17)(-1.962,3.000){2}{\rule{0.250pt}{0.400pt}}
\put(1051.17,318){\rule{0.400pt}{0.900pt}}
\multiput(1052.17,318.00)(-2.000,2.132){2}{\rule{0.400pt}{0.450pt}}
\multiput(1048.92,322.61)(-0.462,0.447){3}{\rule{0.500pt}{0.108pt}}
\multiput(1049.96,321.17)(-1.962,3.000){2}{\rule{0.250pt}{0.400pt}}
\multiput(1045.92,325.61)(-0.462,0.447){3}{\rule{0.500pt}{0.108pt}}
\multiput(1046.96,324.17)(-1.962,3.000){2}{\rule{0.250pt}{0.400pt}}
\put(1043.17,328){\rule{0.400pt}{0.700pt}}
\multiput(1044.17,328.00)(-2.000,1.547){2}{\rule{0.400pt}{0.350pt}}
\multiput(1041.95,331.00)(-0.447,0.685){3}{\rule{0.108pt}{0.633pt}}
\multiput(1042.17,331.00)(-3.000,2.685){2}{\rule{0.400pt}{0.317pt}}
\put(1038.17,335){\rule{0.400pt}{0.700pt}}
\multiput(1039.17,335.00)(-2.000,1.547){2}{\rule{0.400pt}{0.350pt}}
\multiput(1035.92,338.61)(-0.462,0.447){3}{\rule{0.500pt}{0.108pt}}
\multiput(1036.96,337.17)(-1.962,3.000){2}{\rule{0.250pt}{0.400pt}}
\multiput(1032.92,341.61)(-0.462,0.447){3}{\rule{0.500pt}{0.108pt}}
\multiput(1033.96,340.17)(-1.962,3.000){2}{\rule{0.250pt}{0.400pt}}
\put(1030.17,344){\rule{0.400pt}{0.900pt}}
\multiput(1031.17,344.00)(-2.000,2.132){2}{\rule{0.400pt}{0.450pt}}
\multiput(1027.92,348.61)(-0.462,0.447){3}{\rule{0.500pt}{0.108pt}}
\multiput(1028.96,347.17)(-1.962,3.000){2}{\rule{0.250pt}{0.400pt}}
\multiput(1024.92,351.61)(-0.462,0.447){3}{\rule{0.500pt}{0.108pt}}
\multiput(1025.96,350.17)(-1.962,3.000){2}{\rule{0.250pt}{0.400pt}}
\put(1022.17,354){\rule{0.400pt}{0.700pt}}
\multiput(1023.17,354.00)(-2.000,1.547){2}{\rule{0.400pt}{0.350pt}}
\multiput(1020.95,357.00)(-0.447,0.685){3}{\rule{0.108pt}{0.633pt}}
\multiput(1021.17,357.00)(-3.000,2.685){2}{\rule{0.400pt}{0.317pt}}
\put(1017.17,361){\rule{0.400pt}{0.700pt}}
\multiput(1018.17,361.00)(-2.000,1.547){2}{\rule{0.400pt}{0.350pt}}
\multiput(1014.92,364.61)(-0.462,0.447){3}{\rule{0.500pt}{0.108pt}}
\multiput(1015.96,363.17)(-1.962,3.000){2}{\rule{0.250pt}{0.400pt}}
\multiput(1011.92,367.61)(-0.462,0.447){3}{\rule{0.500pt}{0.108pt}}
\multiput(1012.96,366.17)(-1.962,3.000){2}{\rule{0.250pt}{0.400pt}}
\put(1009.17,370){\rule{0.400pt}{0.900pt}}
\multiput(1010.17,370.00)(-2.000,2.132){2}{\rule{0.400pt}{0.450pt}}
\multiput(1006.92,374.61)(-0.462,0.447){3}{\rule{0.500pt}{0.108pt}}
\multiput(1007.96,373.17)(-1.962,3.000){2}{\rule{0.250pt}{0.400pt}}
\multiput(1003.92,377.61)(-0.462,0.447){3}{\rule{0.500pt}{0.108pt}}
\multiput(1004.96,376.17)(-1.962,3.000){2}{\rule{0.250pt}{0.400pt}}
\put(1001.17,380){\rule{0.400pt}{0.700pt}}
\multiput(1002.17,380.00)(-2.000,1.547){2}{\rule{0.400pt}{0.350pt}}
\multiput(999.95,383.00)(-0.447,0.685){3}{\rule{0.108pt}{0.633pt}}
\multiput(1000.17,383.00)(-3.000,2.685){2}{\rule{0.400pt}{0.317pt}}
\put(996.17,387){\rule{0.400pt}{0.700pt}}
\multiput(997.17,387.00)(-2.000,1.547){2}{\rule{0.400pt}{0.350pt}}
\multiput(993.92,390.61)(-0.462,0.447){3}{\rule{0.500pt}{0.108pt}}
\multiput(994.96,389.17)(-1.962,3.000){2}{\rule{0.250pt}{0.400pt}}
\multiput(990.92,393.61)(-0.462,0.447){3}{\rule{0.500pt}{0.108pt}}
\multiput(991.96,392.17)(-1.962,3.000){2}{\rule{0.250pt}{0.400pt}}
\put(988.17,396){\rule{0.400pt}{0.900pt}}
\multiput(989.17,396.00)(-2.000,2.132){2}{\rule{0.400pt}{0.450pt}}
\multiput(985.92,400.61)(-0.462,0.447){3}{\rule{0.500pt}{0.108pt}}
\multiput(986.96,399.17)(-1.962,3.000){2}{\rule{0.250pt}{0.400pt}}
\put(983.17,403){\rule{0.400pt}{0.700pt}}
\multiput(984.17,403.00)(-2.000,1.547){2}{\rule{0.400pt}{0.350pt}}
\multiput(980.92,406.61)(-0.462,0.447){3}{\rule{0.500pt}{0.108pt}}
\multiput(981.96,405.17)(-1.962,3.000){2}{\rule{0.250pt}{0.400pt}}
\multiput(978.95,409.00)(-0.447,0.685){3}{\rule{0.108pt}{0.633pt}}
\multiput(979.17,409.00)(-3.000,2.685){2}{\rule{0.400pt}{0.317pt}}
\put(975.17,413){\rule{0.400pt}{0.700pt}}
\multiput(976.17,413.00)(-2.000,1.547){2}{\rule{0.400pt}{0.350pt}}
\multiput(972.92,416.61)(-0.462,0.447){3}{\rule{0.500pt}{0.108pt}}
\multiput(973.96,415.17)(-1.962,3.000){2}{\rule{0.250pt}{0.400pt}}
\multiput(969.92,419.61)(-0.462,0.447){3}{\rule{0.500pt}{0.108pt}}
\multiput(970.96,418.17)(-1.962,3.000){2}{\rule{0.250pt}{0.400pt}}
\put(967.17,422){\rule{0.400pt}{0.900pt}}
\multiput(968.17,422.00)(-2.000,2.132){2}{\rule{0.400pt}{0.450pt}}
\multiput(964.92,426.61)(-0.462,0.447){3}{\rule{0.500pt}{0.108pt}}
\multiput(965.96,425.17)(-1.962,3.000){2}{\rule{0.250pt}{0.400pt}}
\put(962.17,429){\rule{0.400pt}{0.700pt}}
\multiput(963.17,429.00)(-2.000,1.547){2}{\rule{0.400pt}{0.350pt}}
\multiput(959.92,432.61)(-0.462,0.447){3}{\rule{0.500pt}{0.108pt}}
\multiput(960.96,431.17)(-1.962,3.000){2}{\rule{0.250pt}{0.400pt}}
\multiput(956.92,435.61)(-0.462,0.447){3}{\rule{0.500pt}{0.108pt}}
\multiput(957.96,434.17)(-1.962,3.000){2}{\rule{0.250pt}{0.400pt}}
\put(954.17,438){\rule{0.400pt}{0.900pt}}
\multiput(955.17,438.00)(-2.000,2.132){2}{\rule{0.400pt}{0.450pt}}
\multiput(951.92,442.61)(-0.462,0.447){3}{\rule{0.500pt}{0.108pt}}
\multiput(952.96,441.17)(-1.962,3.000){2}{\rule{0.250pt}{0.400pt}}
\multiput(948.92,445.61)(-0.462,0.447){3}{\rule{0.500pt}{0.108pt}}
\multiput(949.96,444.17)(-1.962,3.000){2}{\rule{0.250pt}{0.400pt}}
\put(946.17,448){\rule{0.400pt}{0.700pt}}
\multiput(947.17,448.00)(-2.000,1.547){2}{\rule{0.400pt}{0.350pt}}
\multiput(944.95,451.00)(-0.447,0.685){3}{\rule{0.108pt}{0.633pt}}
\multiput(945.17,451.00)(-3.000,2.685){2}{\rule{0.400pt}{0.317pt}}
\put(941.17,455){\rule{0.400pt}{0.700pt}}
\multiput(942.17,455.00)(-2.000,1.547){2}{\rule{0.400pt}{0.350pt}}
\multiput(938.92,458.61)(-0.462,0.447){3}{\rule{0.500pt}{0.108pt}}
\multiput(939.96,457.17)(-1.962,3.000){2}{\rule{0.250pt}{0.400pt}}
\multiput(935.92,461.61)(-0.462,0.447){3}{\rule{0.500pt}{0.108pt}}
\multiput(936.96,460.17)(-1.962,3.000){2}{\rule{0.250pt}{0.400pt}}
\put(933.17,464){\rule{0.400pt}{0.900pt}}
\multiput(934.17,464.00)(-2.000,2.132){2}{\rule{0.400pt}{0.450pt}}
\multiput(930.92,468.61)(-0.462,0.447){3}{\rule{0.500pt}{0.108pt}}
\multiput(931.96,467.17)(-1.962,3.000){2}{\rule{0.250pt}{0.400pt}}
\put(928.17,471){\rule{0.400pt}{0.700pt}}
\multiput(929.17,471.00)(-2.000,1.547){2}{\rule{0.400pt}{0.350pt}}
\multiput(925.92,474.61)(-0.462,0.447){3}{\rule{0.500pt}{0.108pt}}
\multiput(926.96,473.17)(-1.962,3.000){2}{\rule{0.250pt}{0.400pt}}
\multiput(923.95,477.00)(-0.447,0.685){3}{\rule{0.108pt}{0.633pt}}
\multiput(924.17,477.00)(-3.000,2.685){2}{\rule{0.400pt}{0.317pt}}
\put(920.17,481){\rule{0.400pt}{0.700pt}}
\multiput(921.17,481.00)(-2.000,1.547){2}{\rule{0.400pt}{0.350pt}}
\multiput(917.92,484.61)(-0.462,0.447){3}{\rule{0.500pt}{0.108pt}}
\multiput(918.96,483.17)(-1.962,3.000){2}{\rule{0.250pt}{0.400pt}}
\multiput(914.92,487.61)(-0.462,0.447){3}{\rule{0.500pt}{0.108pt}}
\multiput(915.96,486.17)(-1.962,3.000){2}{\rule{0.250pt}{0.400pt}}
\put(912.17,490){\rule{0.400pt}{0.900pt}}
\multiput(913.17,490.00)(-2.000,2.132){2}{\rule{0.400pt}{0.450pt}}
\multiput(909.92,494.61)(-0.462,0.447){3}{\rule{0.500pt}{0.108pt}}
\multiput(910.96,493.17)(-1.962,3.000){2}{\rule{0.250pt}{0.400pt}}
\put(907.17,497){\rule{0.400pt}{0.700pt}}
\multiput(908.17,497.00)(-2.000,1.547){2}{\rule{0.400pt}{0.350pt}}
\multiput(904.92,500.61)(-0.462,0.447){3}{\rule{0.500pt}{0.108pt}}
\multiput(905.96,499.17)(-1.962,3.000){2}{\rule{0.250pt}{0.400pt}}
\multiput(902.95,503.00)(-0.447,0.685){3}{\rule{0.108pt}{0.633pt}}
\multiput(903.17,503.00)(-3.000,2.685){2}{\rule{0.400pt}{0.317pt}}
\put(899.17,507){\rule{0.400pt}{0.700pt}}
\multiput(900.17,507.00)(-2.000,1.547){2}{\rule{0.400pt}{0.350pt}}
\multiput(896.92,510.61)(-0.462,0.447){3}{\rule{0.500pt}{0.108pt}}
\multiput(897.96,509.17)(-1.962,3.000){2}{\rule{0.250pt}{0.400pt}}
\put(894.17,513){\rule{0.400pt}{0.700pt}}
\multiput(895.17,513.00)(-2.000,1.547){2}{\rule{0.400pt}{0.350pt}}
\multiput(892.95,516.00)(-0.447,0.685){3}{\rule{0.108pt}{0.633pt}}
\multiput(893.17,516.00)(-3.000,2.685){2}{\rule{0.400pt}{0.317pt}}
\multiput(888.92,520.61)(-0.462,0.447){3}{\rule{0.500pt}{0.108pt}}
\multiput(889.96,519.17)(-1.962,3.000){2}{\rule{0.250pt}{0.400pt}}
\put(886.17,523){\rule{0.400pt}{0.700pt}}
\multiput(887.17,523.00)(-2.000,1.547){2}{\rule{0.400pt}{0.350pt}}
\multiput(883.92,526.61)(-0.462,0.447){3}{\rule{0.500pt}{0.108pt}}
\multiput(884.96,525.17)(-1.962,3.000){2}{\rule{0.250pt}{0.400pt}}
\multiput(881.95,529.00)(-0.447,0.685){3}{\rule{0.108pt}{0.633pt}}
\multiput(882.17,529.00)(-3.000,2.685){2}{\rule{0.400pt}{0.317pt}}
\put(878.17,533){\rule{0.400pt}{0.700pt}}
\multiput(879.17,533.00)(-2.000,1.547){2}{\rule{0.400pt}{0.350pt}}
\multiput(875.92,536.61)(-0.462,0.447){3}{\rule{0.500pt}{0.108pt}}
\multiput(876.96,535.17)(-1.962,3.000){2}{\rule{0.250pt}{0.400pt}}
\put(873.17,539){\rule{0.400pt}{0.700pt}}
\multiput(874.17,539.00)(-2.000,1.547){2}{\rule{0.400pt}{0.350pt}}
\multiput(871.95,542.00)(-0.447,0.685){3}{\rule{0.108pt}{0.633pt}}
\multiput(872.17,542.00)(-3.000,2.685){2}{\rule{0.400pt}{0.317pt}}
\multiput(867.92,546.61)(-0.462,0.447){3}{\rule{0.500pt}{0.108pt}}
\multiput(868.96,545.17)(-1.962,3.000){2}{\rule{0.250pt}{0.400pt}}
\put(865.17,549){\rule{0.400pt}{0.700pt}}
\multiput(866.17,549.00)(-2.000,1.547){2}{\rule{0.400pt}{0.350pt}}
\multiput(862.92,552.61)(-0.462,0.447){3}{\rule{0.500pt}{0.108pt}}
\multiput(863.96,551.17)(-1.962,3.000){2}{\rule{0.250pt}{0.400pt}}
\multiput(860.95,555.00)(-0.447,0.685){3}{\rule{0.108pt}{0.633pt}}
\multiput(861.17,555.00)(-3.000,2.685){2}{\rule{0.400pt}{0.317pt}}
\put(857.17,559){\rule{0.400pt}{0.700pt}}
\multiput(858.17,559.00)(-2.000,1.547){2}{\rule{0.400pt}{0.350pt}}
\multiput(854.92,562.61)(-0.462,0.447){3}{\rule{0.500pt}{0.108pt}}
\multiput(855.96,561.17)(-1.962,3.000){2}{\rule{0.250pt}{0.400pt}}
\put(852.17,565){\rule{0.400pt}{0.700pt}}
\multiput(853.17,565.00)(-2.000,1.547){2}{\rule{0.400pt}{0.350pt}}
\multiput(850.95,568.00)(-0.447,0.685){3}{\rule{0.108pt}{0.633pt}}
\multiput(851.17,568.00)(-3.000,2.685){2}{\rule{0.400pt}{0.317pt}}
\multiput(846.92,572.61)(-0.462,0.447){3}{\rule{0.500pt}{0.108pt}}
\multiput(847.96,571.17)(-1.962,3.000){2}{\rule{0.250pt}{0.400pt}}
\put(844.17,575){\rule{0.400pt}{0.700pt}}
\multiput(845.17,575.00)(-2.000,1.547){2}{\rule{0.400pt}{0.350pt}}
\multiput(841.92,578.61)(-0.462,0.447){3}{\rule{0.500pt}{0.108pt}}
\multiput(842.96,577.17)(-1.962,3.000){2}{\rule{0.250pt}{0.400pt}}
\put(839.17,581){\rule{0.400pt}{0.900pt}}
\multiput(840.17,581.00)(-2.000,2.132){2}{\rule{0.400pt}{0.450pt}}
\multiput(836.92,585.61)(-0.462,0.447){3}{\rule{0.500pt}{0.108pt}}
\multiput(837.96,584.17)(-1.962,3.000){2}{\rule{0.250pt}{0.400pt}}
\multiput(833.92,588.61)(-0.462,0.447){3}{\rule{0.500pt}{0.108pt}}
\multiput(834.96,587.17)(-1.962,3.000){2}{\rule{0.250pt}{0.400pt}}
\put(831.17,591){\rule{0.400pt}{0.700pt}}
\multiput(832.17,591.00)(-2.000,1.547){2}{\rule{0.400pt}{0.350pt}}
\multiput(828.92,594.61)(-0.462,0.447){3}{\rule{0.500pt}{0.108pt}}
\multiput(829.96,593.17)(-1.962,3.000){2}{\rule{0.250pt}{0.400pt}}
\multiput(826.95,597.00)(-0.447,0.685){3}{\rule{0.108pt}{0.633pt}}
\multiput(827.17,597.00)(-3.000,2.685){2}{\rule{0.400pt}{0.317pt}}
\put(823.17,601){\rule{0.400pt}{0.700pt}}
\multiput(824.17,601.00)(-2.000,1.547){2}{\rule{0.400pt}{0.350pt}}
\multiput(820.92,604.61)(-0.462,0.447){3}{\rule{0.500pt}{0.108pt}}
\multiput(821.96,603.17)(-1.962,3.000){2}{\rule{0.250pt}{0.400pt}}
\put(818.17,607){\rule{0.400pt}{0.700pt}}
\multiput(819.17,607.00)(-2.000,1.547){2}{\rule{0.400pt}{0.350pt}}
\multiput(816.95,610.00)(-0.447,0.685){3}{\rule{0.108pt}{0.633pt}}
\multiput(817.17,610.00)(-3.000,2.685){2}{\rule{0.400pt}{0.317pt}}
\multiput(812.92,614.61)(-0.462,0.447){3}{\rule{0.500pt}{0.108pt}}
\multiput(813.96,613.17)(-1.962,3.000){2}{\rule{0.250pt}{0.400pt}}
\put(810.17,617){\rule{0.400pt}{0.700pt}}
\multiput(811.17,617.00)(-2.000,1.547){2}{\rule{0.400pt}{0.350pt}}
\multiput(807.92,620.61)(-0.462,0.447){3}{\rule{0.500pt}{0.108pt}}
\multiput(808.96,619.17)(-1.962,3.000){2}{\rule{0.250pt}{0.400pt}}
\multiput(805.95,623.00)(-0.447,0.685){3}{\rule{0.108pt}{0.633pt}}
\multiput(806.17,623.00)(-3.000,2.685){2}{\rule{0.400pt}{0.317pt}}
\put(802.17,627){\rule{0.400pt}{0.700pt}}
\multiput(803.17,627.00)(-2.000,1.547){2}{\rule{0.400pt}{0.350pt}}
\multiput(799.92,630.61)(-0.462,0.447){3}{\rule{0.500pt}{0.108pt}}
\multiput(800.96,629.17)(-1.962,3.000){2}{\rule{0.250pt}{0.400pt}}
\put(797.17,633){\rule{0.400pt}{0.700pt}}
\multiput(798.17,633.00)(-2.000,1.547){2}{\rule{0.400pt}{0.350pt}}
\multiput(795.95,636.00)(-0.447,0.685){3}{\rule{0.108pt}{0.633pt}}
\multiput(796.17,636.00)(-3.000,2.685){2}{\rule{0.400pt}{0.317pt}}
\multiput(791.92,640.61)(-0.462,0.447){3}{\rule{0.500pt}{0.108pt}}
\multiput(792.96,639.17)(-1.962,3.000){2}{\rule{0.250pt}{0.400pt}}
\put(789.17,643){\rule{0.400pt}{0.700pt}}
\multiput(790.17,643.00)(-2.000,1.547){2}{\rule{0.400pt}{0.350pt}}
\multiput(786.92,646.61)(-0.462,0.447){3}{\rule{0.500pt}{0.108pt}}
\multiput(787.96,645.17)(-1.962,3.000){2}{\rule{0.250pt}{0.400pt}}
\put(784.17,649){\rule{0.400pt}{0.900pt}}
\multiput(785.17,649.00)(-2.000,2.132){2}{\rule{0.400pt}{0.450pt}}
\multiput(781.92,653.61)(-0.462,0.447){3}{\rule{0.500pt}{0.108pt}}
\multiput(782.96,652.17)(-1.962,3.000){2}{\rule{0.250pt}{0.400pt}}
\multiput(778.92,656.61)(-0.462,0.447){3}{\rule{0.500pt}{0.108pt}}
\multiput(779.96,655.17)(-1.962,3.000){2}{\rule{0.250pt}{0.400pt}}
\put(776.17,659){\rule{0.400pt}{0.700pt}}
\multiput(777.17,659.00)(-2.000,1.547){2}{\rule{0.400pt}{0.350pt}}
\multiput(774.95,662.00)(-0.447,0.685){3}{\rule{0.108pt}{0.633pt}}
\multiput(775.17,662.00)(-3.000,2.685){2}{\rule{0.400pt}{0.317pt}}
\multiput(770.92,666.61)(-0.462,0.447){3}{\rule{0.500pt}{0.108pt}}
\multiput(771.96,665.17)(-1.962,3.000){2}{\rule{0.250pt}{0.400pt}}
\put(768.17,669){\rule{0.400pt}{0.700pt}}
\multiput(769.17,669.00)(-2.000,1.547){2}{\rule{0.400pt}{0.350pt}}
\multiput(765.92,672.61)(-0.462,0.447){3}{\rule{0.500pt}{0.108pt}}
\multiput(766.96,671.17)(-1.962,3.000){2}{\rule{0.250pt}{0.400pt}}
\put(763.17,675){\rule{0.400pt}{0.900pt}}
\multiput(764.17,675.00)(-2.000,2.132){2}{\rule{0.400pt}{0.450pt}}
\multiput(760.92,679.61)(-0.462,0.447){3}{\rule{0.500pt}{0.108pt}}
\multiput(761.96,678.17)(-1.962,3.000){2}{\rule{0.250pt}{0.400pt}}
\multiput(757.92,682.61)(-0.462,0.447){3}{\rule{0.500pt}{0.108pt}}
\multiput(758.96,681.17)(-1.962,3.000){2}{\rule{0.250pt}{0.400pt}}
\put(755.17,685){\rule{0.400pt}{0.700pt}}
\multiput(756.17,685.00)(-2.000,1.547){2}{\rule{0.400pt}{0.350pt}}
\multiput(753.95,688.00)(-0.447,0.685){3}{\rule{0.108pt}{0.633pt}}
\multiput(754.17,688.00)(-3.000,2.685){2}{\rule{0.400pt}{0.317pt}}
\put(750.17,692){\rule{0.400pt}{0.700pt}}
\multiput(751.17,692.00)(-2.000,1.547){2}{\rule{0.400pt}{0.350pt}}
\multiput(747.92,695.61)(-0.462,0.447){3}{\rule{0.500pt}{0.108pt}}
\multiput(748.96,694.17)(-1.962,3.000){2}{\rule{0.250pt}{0.400pt}}
\multiput(744.92,698.61)(-0.462,0.447){3}{\rule{0.500pt}{0.108pt}}
\multiput(745.96,697.17)(-1.962,3.000){2}{\rule{0.250pt}{0.400pt}}
\put(742.17,701){\rule{0.400pt}{0.900pt}}
\multiput(743.17,701.00)(-2.000,2.132){2}{\rule{0.400pt}{0.450pt}}
\multiput(739.92,705.61)(-0.462,0.447){3}{\rule{0.500pt}{0.108pt}}
\multiput(740.96,704.17)(-1.962,3.000){2}{\rule{0.250pt}{0.400pt}}
\multiput(736.92,708.61)(-0.462,0.447){3}{\rule{0.500pt}{0.108pt}}
\multiput(737.96,707.17)(-1.962,3.000){2}{\rule{0.250pt}{0.400pt}}
\put(734.17,711){\rule{0.400pt}{0.700pt}}
\multiput(735.17,711.00)(-2.000,1.547){2}{\rule{0.400pt}{0.350pt}}
\multiput(732.95,714.00)(-0.447,0.685){3}{\rule{0.108pt}{0.633pt}}
\multiput(733.17,714.00)(-3.000,2.685){2}{\rule{0.400pt}{0.317pt}}
\put(729.17,718){\rule{0.400pt}{0.700pt}}
\multiput(730.17,718.00)(-2.000,1.547){2}{\rule{0.400pt}{0.350pt}}
\multiput(726.92,721.61)(-0.462,0.447){3}{\rule{0.500pt}{0.108pt}}
\multiput(727.96,720.17)(-1.962,3.000){2}{\rule{0.250pt}{0.400pt}}
\multiput(723.92,724.61)(-0.462,0.447){3}{\rule{0.500pt}{0.108pt}}
\multiput(724.96,723.17)(-1.962,3.000){2}{\rule{0.250pt}{0.400pt}}
\put(721.17,727){\rule{0.400pt}{0.900pt}}
\multiput(722.17,727.00)(-2.000,2.132){2}{\rule{0.400pt}{0.450pt}}
\multiput(718.92,731.61)(-0.462,0.447){3}{\rule{0.500pt}{0.108pt}}
\multiput(719.96,730.17)(-1.962,3.000){2}{\rule{0.250pt}{0.400pt}}
\multiput(715.92,734.61)(-0.462,0.447){3}{\rule{0.500pt}{0.108pt}}
\multiput(716.96,733.17)(-1.962,3.000){2}{\rule{0.250pt}{0.400pt}}
\put(713.17,737){\rule{0.400pt}{0.700pt}}
\multiput(714.17,737.00)(-2.000,1.547){2}{\rule{0.400pt}{0.350pt}}
\multiput(711.95,740.00)(-0.447,0.685){3}{\rule{0.108pt}{0.633pt}}
\multiput(712.17,740.00)(-3.000,2.685){2}{\rule{0.400pt}{0.317pt}}
\put(708.17,744){\rule{0.400pt}{0.700pt}}
\multiput(709.17,744.00)(-2.000,1.547){2}{\rule{0.400pt}{0.350pt}}
\multiput(705.92,747.61)(-0.462,0.447){3}{\rule{0.500pt}{0.108pt}}
\multiput(706.96,746.17)(-1.962,3.000){2}{\rule{0.250pt}{0.400pt}}
\multiput(702.92,750.61)(-0.462,0.447){3}{\rule{0.500pt}{0.108pt}}
\multiput(703.96,749.17)(-1.962,3.000){2}{\rule{0.250pt}{0.400pt}}
\put(700.17,753){\rule{0.400pt}{0.700pt}}
\multiput(701.17,753.00)(-2.000,1.547){2}{\rule{0.400pt}{0.350pt}}
\multiput(698.95,756.00)(-0.447,0.685){3}{\rule{0.108pt}{0.633pt}}
\multiput(699.17,756.00)(-3.000,2.685){2}{\rule{0.400pt}{0.317pt}}
\put(695.17,760){\rule{0.400pt}{0.700pt}}
\multiput(696.17,760.00)(-2.000,1.547){2}{\rule{0.400pt}{0.350pt}}
\multiput(692.92,763.61)(-0.462,0.447){3}{\rule{0.500pt}{0.108pt}}
\multiput(693.96,762.17)(-1.962,3.000){2}{\rule{0.250pt}{0.400pt}}
\multiput(689.92,766.61)(-0.462,0.447){3}{\rule{0.500pt}{0.108pt}}
\multiput(690.96,765.17)(-1.962,3.000){2}{\rule{0.250pt}{0.400pt}}
\put(687.17,769){\rule{0.400pt}{0.900pt}}
\multiput(688.17,769.00)(-2.000,2.132){2}{\rule{0.400pt}{0.450pt}}
\multiput(684.92,773.61)(-0.462,0.447){3}{\rule{0.500pt}{0.108pt}}
\multiput(685.96,772.17)(-1.962,3.000){2}{\rule{0.250pt}{0.400pt}}
\multiput(681.92,776.61)(-0.462,0.447){3}{\rule{0.500pt}{0.108pt}}
\multiput(682.96,775.17)(-1.962,3.000){2}{\rule{0.250pt}{0.400pt}}
\put(679.17,779){\rule{0.400pt}{0.700pt}}
\multiput(680.17,779.00)(-2.000,1.547){2}{\rule{0.400pt}{0.350pt}}
\multiput(677.95,782.00)(-0.447,0.685){3}{\rule{0.108pt}{0.633pt}}
\multiput(678.17,782.00)(-3.000,2.685){2}{\rule{0.400pt}{0.317pt}}
\put(674.17,786){\rule{0.400pt}{0.700pt}}
\multiput(675.17,786.00)(-2.000,1.547){2}{\rule{0.400pt}{0.350pt}}
\multiput(671.92,789.61)(-0.462,0.447){3}{\rule{0.500pt}{0.108pt}}
\multiput(672.96,788.17)(-1.962,3.000){2}{\rule{0.250pt}{0.400pt}}
\multiput(668.92,792.61)(-0.462,0.447){3}{\rule{0.500pt}{0.108pt}}
\multiput(669.96,791.17)(-1.962,3.000){2}{\rule{0.250pt}{0.400pt}}
\put(666.17,795){\rule{0.400pt}{0.900pt}}
\multiput(667.17,795.00)(-2.000,2.132){2}{\rule{0.400pt}{0.450pt}}
\multiput(663.92,799.61)(-0.462,0.447){3}{\rule{0.500pt}{0.108pt}}
\multiput(664.96,798.17)(-1.962,3.000){2}{\rule{0.250pt}{0.400pt}}
\multiput(660.92,802.61)(-0.462,0.447){3}{\rule{0.500pt}{0.108pt}}
\multiput(661.96,801.17)(-1.962,3.000){2}{\rule{0.250pt}{0.400pt}}
\put(658.17,805){\rule{0.400pt}{0.700pt}}
\multiput(659.17,805.00)(-2.000,1.547){2}{\rule{0.400pt}{0.350pt}}
\multiput(656.95,808.00)(-0.447,0.685){3}{\rule{0.108pt}{0.633pt}}
\multiput(657.17,808.00)(-3.000,2.685){2}{\rule{0.400pt}{0.317pt}}
\put(653.17,812){\rule{0.400pt}{0.700pt}}
\multiput(654.17,812.00)(-2.000,1.547){2}{\rule{0.400pt}{0.350pt}}
\multiput(650.92,815.61)(-0.462,0.447){3}{\rule{0.500pt}{0.108pt}}
\multiput(651.96,814.17)(-1.962,3.000){2}{\rule{0.250pt}{0.400pt}}
\multiput(647.92,818.61)(-0.462,0.447){3}{\rule{0.500pt}{0.108pt}}
\multiput(648.96,817.17)(-1.962,3.000){2}{\rule{0.250pt}{0.400pt}}
\put(645.17,821){\rule{0.400pt}{0.900pt}}
\multiput(646.17,821.00)(-2.000,2.132){2}{\rule{0.400pt}{0.450pt}}
\multiput(642.92,825.61)(-0.462,0.447){3}{\rule{0.500pt}{0.108pt}}
\multiput(643.96,824.17)(-1.962,3.000){2}{\rule{0.250pt}{0.400pt}}
\put(640.17,828){\rule{0.400pt}{0.700pt}}
\multiput(641.17,828.00)(-2.000,1.547){2}{\rule{0.400pt}{0.350pt}}
\multiput(637.92,831.61)(-0.462,0.447){3}{\rule{0.500pt}{0.108pt}}
\multiput(638.96,830.17)(-1.962,3.000){2}{\rule{0.250pt}{0.400pt}}
\multiput(635.95,834.00)(-0.447,0.685){3}{\rule{0.108pt}{0.633pt}}
\multiput(636.17,834.00)(-3.000,2.685){2}{\rule{0.400pt}{0.317pt}}
\put(632.17,834){\rule{0.400pt}{0.900pt}}
\multiput(633.17,836.13)(-2.000,-2.132){2}{\rule{0.400pt}{0.450pt}}
\multiput(629.92,832.95)(-0.462,-0.447){3}{\rule{0.500pt}{0.108pt}}
\multiput(630.96,833.17)(-1.962,-3.000){2}{\rule{0.250pt}{0.400pt}}
\multiput(626.92,829.95)(-0.462,-0.447){3}{\rule{0.500pt}{0.108pt}}
\multiput(627.96,830.17)(-1.962,-3.000){2}{\rule{0.250pt}{0.400pt}}
\put(624.17,825){\rule{0.400pt}{0.700pt}}
\multiput(625.17,826.55)(-2.000,-1.547){2}{\rule{0.400pt}{0.350pt}}
\multiput(622.95,822.37)(-0.447,-0.685){3}{\rule{0.108pt}{0.633pt}}
\multiput(623.17,823.69)(-3.000,-2.685){2}{\rule{0.400pt}{0.317pt}}
\put(619.17,818){\rule{0.400pt}{0.700pt}}
\multiput(620.17,819.55)(-2.000,-1.547){2}{\rule{0.400pt}{0.350pt}}
\multiput(616.92,816.95)(-0.462,-0.447){3}{\rule{0.500pt}{0.108pt}}
\multiput(617.96,817.17)(-1.962,-3.000){2}{\rule{0.250pt}{0.400pt}}
\multiput(613.92,813.95)(-0.462,-0.447){3}{\rule{0.500pt}{0.108pt}}
\multiput(614.96,814.17)(-1.962,-3.000){2}{\rule{0.250pt}{0.400pt}}
\put(611.17,808){\rule{0.400pt}{0.900pt}}
\multiput(612.17,810.13)(-2.000,-2.132){2}{\rule{0.400pt}{0.450pt}}
\multiput(608.92,806.95)(-0.462,-0.447){3}{\rule{0.500pt}{0.108pt}}
\multiput(609.96,807.17)(-1.962,-3.000){2}{\rule{0.250pt}{0.400pt}}
\put(606.17,802){\rule{0.400pt}{0.700pt}}
\multiput(607.17,803.55)(-2.000,-1.547){2}{\rule{0.400pt}{0.350pt}}
\multiput(603.92,800.95)(-0.462,-0.447){3}{\rule{0.500pt}{0.108pt}}
\multiput(604.96,801.17)(-1.962,-3.000){2}{\rule{0.250pt}{0.400pt}}
\multiput(601.95,796.37)(-0.447,-0.685){3}{\rule{0.108pt}{0.633pt}}
\multiput(602.17,797.69)(-3.000,-2.685){2}{\rule{0.400pt}{0.317pt}}
\put(598.17,792){\rule{0.400pt}{0.700pt}}
\multiput(599.17,793.55)(-2.000,-1.547){2}{\rule{0.400pt}{0.350pt}}
\multiput(595.92,790.95)(-0.462,-0.447){3}{\rule{0.500pt}{0.108pt}}
\multiput(596.96,791.17)(-1.962,-3.000){2}{\rule{0.250pt}{0.400pt}}
\multiput(592.92,787.95)(-0.462,-0.447){3}{\rule{0.500pt}{0.108pt}}
\multiput(593.96,788.17)(-1.962,-3.000){2}{\rule{0.250pt}{0.400pt}}
\put(590.17,782){\rule{0.400pt}{0.900pt}}
\multiput(591.17,784.13)(-2.000,-2.132){2}{\rule{0.400pt}{0.450pt}}
\multiput(587.92,780.95)(-0.462,-0.447){3}{\rule{0.500pt}{0.108pt}}
\multiput(588.96,781.17)(-1.962,-3.000){2}{\rule{0.250pt}{0.400pt}}
\put(585.17,776){\rule{0.400pt}{0.700pt}}
\multiput(586.17,777.55)(-2.000,-1.547){2}{\rule{0.400pt}{0.350pt}}
\multiput(582.92,774.95)(-0.462,-0.447){3}{\rule{0.500pt}{0.108pt}}
\multiput(583.96,775.17)(-1.962,-3.000){2}{\rule{0.250pt}{0.400pt}}
\multiput(580.95,770.37)(-0.447,-0.685){3}{\rule{0.108pt}{0.633pt}}
\multiput(581.17,771.69)(-3.000,-2.685){2}{\rule{0.400pt}{0.317pt}}
\put(577.17,766){\rule{0.400pt}{0.700pt}}
\multiput(578.17,767.55)(-2.000,-1.547){2}{\rule{0.400pt}{0.350pt}}
\multiput(574.92,764.95)(-0.462,-0.447){3}{\rule{0.500pt}{0.108pt}}
\multiput(575.96,765.17)(-1.962,-3.000){2}{\rule{0.250pt}{0.400pt}}
\multiput(571.92,761.95)(-0.462,-0.447){3}{\rule{0.500pt}{0.108pt}}
\multiput(572.96,762.17)(-1.962,-3.000){2}{\rule{0.250pt}{0.400pt}}
\put(569.17,756){\rule{0.400pt}{0.900pt}}
\multiput(570.17,758.13)(-2.000,-2.132){2}{\rule{0.400pt}{0.450pt}}
\multiput(566.92,754.95)(-0.462,-0.447){3}{\rule{0.500pt}{0.108pt}}
\multiput(567.96,755.17)(-1.962,-3.000){2}{\rule{0.250pt}{0.400pt}}
\put(564.17,750){\rule{0.400pt}{0.700pt}}
\multiput(565.17,751.55)(-2.000,-1.547){2}{\rule{0.400pt}{0.350pt}}
\multiput(561.92,748.95)(-0.462,-0.447){3}{\rule{0.500pt}{0.108pt}}
\multiput(562.96,749.17)(-1.962,-3.000){2}{\rule{0.250pt}{0.400pt}}
\multiput(558.92,745.95)(-0.462,-0.447){3}{\rule{0.500pt}{0.108pt}}
\multiput(559.96,746.17)(-1.962,-3.000){2}{\rule{0.250pt}{0.400pt}}
\put(556.17,740){\rule{0.400pt}{0.900pt}}
\multiput(557.17,742.13)(-2.000,-2.132){2}{\rule{0.400pt}{0.450pt}}
\multiput(553.92,738.95)(-0.462,-0.447){3}{\rule{0.500pt}{0.108pt}}
\multiput(554.96,739.17)(-1.962,-3.000){2}{\rule{0.250pt}{0.400pt}}
\put(551.17,734){\rule{0.400pt}{0.700pt}}
\multiput(552.17,735.55)(-2.000,-1.547){2}{\rule{0.400pt}{0.350pt}}
\multiput(548.92,732.95)(-0.462,-0.447){3}{\rule{0.500pt}{0.108pt}}
\multiput(549.96,733.17)(-1.962,-3.000){2}{\rule{0.250pt}{0.400pt}}
\multiput(546.95,728.37)(-0.447,-0.685){3}{\rule{0.108pt}{0.633pt}}
\multiput(547.17,729.69)(-3.000,-2.685){2}{\rule{0.400pt}{0.317pt}}
\put(543.17,724){\rule{0.400pt}{0.700pt}}
\multiput(544.17,725.55)(-2.000,-1.547){2}{\rule{0.400pt}{0.350pt}}
\multiput(540.92,722.95)(-0.462,-0.447){3}{\rule{0.500pt}{0.108pt}}
\multiput(541.96,723.17)(-1.962,-3.000){2}{\rule{0.250pt}{0.400pt}}
\multiput(537.92,719.95)(-0.462,-0.447){3}{\rule{0.500pt}{0.108pt}}
\multiput(538.96,720.17)(-1.962,-3.000){2}{\rule{0.250pt}{0.400pt}}
\put(535.17,714){\rule{0.400pt}{0.900pt}}
\multiput(536.17,716.13)(-2.000,-2.132){2}{\rule{0.400pt}{0.450pt}}
\multiput(532.92,712.95)(-0.462,-0.447){3}{\rule{0.500pt}{0.108pt}}
\multiput(533.96,713.17)(-1.962,-3.000){2}{\rule{0.250pt}{0.400pt}}
\put(530.17,708){\rule{0.400pt}{0.700pt}}
\multiput(531.17,709.55)(-2.000,-1.547){2}{\rule{0.400pt}{0.350pt}}
\multiput(527.92,706.95)(-0.462,-0.447){3}{\rule{0.500pt}{0.108pt}}
\multiput(528.96,707.17)(-1.962,-3.000){2}{\rule{0.250pt}{0.400pt}}
\multiput(525.95,702.37)(-0.447,-0.685){3}{\rule{0.108pt}{0.633pt}}
\multiput(526.17,703.69)(-3.000,-2.685){2}{\rule{0.400pt}{0.317pt}}
\put(522.17,698){\rule{0.400pt}{0.700pt}}
\multiput(523.17,699.55)(-2.000,-1.547){2}{\rule{0.400pt}{0.350pt}}
\multiput(519.92,696.95)(-0.462,-0.447){3}{\rule{0.500pt}{0.108pt}}
\multiput(520.96,697.17)(-1.962,-3.000){2}{\rule{0.250pt}{0.400pt}}
\multiput(516.92,693.95)(-0.462,-0.447){3}{\rule{0.500pt}{0.108pt}}
\multiput(517.96,694.17)(-1.962,-3.000){2}{\rule{0.250pt}{0.400pt}}
\put(514.17,688){\rule{0.400pt}{0.900pt}}
\multiput(515.17,690.13)(-2.000,-2.132){2}{\rule{0.400pt}{0.450pt}}
\multiput(511.92,686.95)(-0.462,-0.447){3}{\rule{0.500pt}{0.108pt}}
\multiput(512.96,687.17)(-1.962,-3.000){2}{\rule{0.250pt}{0.400pt}}
\put(509.17,682){\rule{0.400pt}{0.700pt}}
\multiput(510.17,683.55)(-2.000,-1.547){2}{\rule{0.400pt}{0.350pt}}
\multiput(506.92,680.95)(-0.462,-0.447){3}{\rule{0.500pt}{0.108pt}}
\multiput(507.96,681.17)(-1.962,-3.000){2}{\rule{0.250pt}{0.400pt}}
\multiput(504.95,676.37)(-0.447,-0.685){3}{\rule{0.108pt}{0.633pt}}
\multiput(505.17,677.69)(-3.000,-2.685){2}{\rule{0.400pt}{0.317pt}}
\put(501.17,672){\rule{0.400pt}{0.700pt}}
\multiput(502.17,673.55)(-2.000,-1.547){2}{\rule{0.400pt}{0.350pt}}
\multiput(498.92,670.95)(-0.462,-0.447){3}{\rule{0.500pt}{0.108pt}}
\multiput(499.96,671.17)(-1.962,-3.000){2}{\rule{0.250pt}{0.400pt}}
\put(496.17,666){\rule{0.400pt}{0.700pt}}
\multiput(497.17,667.55)(-2.000,-1.547){2}{\rule{0.400pt}{0.350pt}}
\multiput(494.95,663.37)(-0.447,-0.685){3}{\rule{0.108pt}{0.633pt}}
\multiput(495.17,664.69)(-3.000,-2.685){2}{\rule{0.400pt}{0.317pt}}
\multiput(490.92,660.95)(-0.462,-0.447){3}{\rule{0.500pt}{0.108pt}}
\multiput(491.96,661.17)(-1.962,-3.000){2}{\rule{0.250pt}{0.400pt}}
\put(488.17,656){\rule{0.400pt}{0.700pt}}
\multiput(489.17,657.55)(-2.000,-1.547){2}{\rule{0.400pt}{0.350pt}}
\multiput(485.92,654.95)(-0.462,-0.447){3}{\rule{0.500pt}{0.108pt}}
\multiput(486.96,655.17)(-1.962,-3.000){2}{\rule{0.250pt}{0.400pt}}
\multiput(483.95,650.37)(-0.447,-0.685){3}{\rule{0.108pt}{0.633pt}}
\multiput(484.17,651.69)(-3.000,-2.685){2}{\rule{0.400pt}{0.317pt}}
\put(480.17,646){\rule{0.400pt}{0.700pt}}
\multiput(481.17,647.55)(-2.000,-1.547){2}{\rule{0.400pt}{0.350pt}}
\multiput(477.92,644.95)(-0.462,-0.447){3}{\rule{0.500pt}{0.108pt}}
\multiput(478.96,645.17)(-1.962,-3.000){2}{\rule{0.250pt}{0.400pt}}
\put(475.17,640){\rule{0.400pt}{0.700pt}}
\multiput(476.17,641.55)(-2.000,-1.547){2}{\rule{0.400pt}{0.350pt}}
\multiput(473.95,637.37)(-0.447,-0.685){3}{\rule{0.108pt}{0.633pt}}
\multiput(474.17,638.69)(-3.000,-2.685){2}{\rule{0.400pt}{0.317pt}}
\multiput(469.92,634.95)(-0.462,-0.447){3}{\rule{0.500pt}{0.108pt}}
\multiput(470.96,635.17)(-1.962,-3.000){2}{\rule{0.250pt}{0.400pt}}
\put(467.17,630){\rule{0.400pt}{0.700pt}}
\multiput(468.17,631.55)(-2.000,-1.547){2}{\rule{0.400pt}{0.350pt}}
\multiput(464.92,628.95)(-0.462,-0.447){3}{\rule{0.500pt}{0.108pt}}
\multiput(465.96,629.17)(-1.962,-3.000){2}{\rule{0.250pt}{0.400pt}}
\put(462.17,623){\rule{0.400pt}{0.900pt}}
\multiput(463.17,625.13)(-2.000,-2.132){2}{\rule{0.400pt}{0.450pt}}
\multiput(459.92,621.95)(-0.462,-0.447){3}{\rule{0.500pt}{0.108pt}}
\multiput(460.96,622.17)(-1.962,-3.000){2}{\rule{0.250pt}{0.400pt}}
\multiput(456.92,618.95)(-0.462,-0.447){3}{\rule{0.500pt}{0.108pt}}
\multiput(457.96,619.17)(-1.962,-3.000){2}{\rule{0.250pt}{0.400pt}}
\put(454.17,614){\rule{0.400pt}{0.700pt}}
\multiput(455.17,615.55)(-2.000,-1.547){2}{\rule{0.400pt}{0.350pt}}
\multiput(452.95,611.37)(-0.447,-0.685){3}{\rule{0.108pt}{0.633pt}}
\multiput(453.17,612.69)(-3.000,-2.685){2}{\rule{0.400pt}{0.317pt}}
\multiput(448.92,608.95)(-0.462,-0.447){3}{\rule{0.500pt}{0.108pt}}
\multiput(449.96,609.17)(-1.962,-3.000){2}{\rule{0.250pt}{0.400pt}}
\put(446.17,604){\rule{0.400pt}{0.700pt}}
\multiput(447.17,605.55)(-2.000,-1.547){2}{\rule{0.400pt}{0.350pt}}
\multiput(443.92,602.95)(-0.462,-0.447){3}{\rule{0.500pt}{0.108pt}}
\multiput(444.96,603.17)(-1.962,-3.000){2}{\rule{0.250pt}{0.400pt}}
\put(441.17,597){\rule{0.400pt}{0.900pt}}
\multiput(442.17,599.13)(-2.000,-2.132){2}{\rule{0.400pt}{0.450pt}}
\multiput(438.92,595.95)(-0.462,-0.447){3}{\rule{0.500pt}{0.108pt}}
\multiput(439.96,596.17)(-1.962,-3.000){2}{\rule{0.250pt}{0.400pt}}
\multiput(435.92,592.95)(-0.462,-0.447){3}{\rule{0.500pt}{0.108pt}}
\multiput(436.96,593.17)(-1.962,-3.000){2}{\rule{0.250pt}{0.400pt}}
\put(433.17,588){\rule{0.400pt}{0.700pt}}
\multiput(434.17,589.55)(-2.000,-1.547){2}{\rule{0.400pt}{0.350pt}}
\multiput(430.92,586.95)(-0.462,-0.447){3}{\rule{0.500pt}{0.108pt}}
\multiput(431.96,587.17)(-1.962,-3.000){2}{\rule{0.250pt}{0.400pt}}
\multiput(428.95,582.37)(-0.447,-0.685){3}{\rule{0.108pt}{0.633pt}}
\multiput(429.17,583.69)(-3.000,-2.685){2}{\rule{0.400pt}{0.317pt}}
\put(425.17,578){\rule{0.400pt}{0.700pt}}
\multiput(426.17,579.55)(-2.000,-1.547){2}{\rule{0.400pt}{0.350pt}}
\multiput(422.92,576.95)(-0.462,-0.447){3}{\rule{0.500pt}{0.108pt}}
\multiput(423.96,577.17)(-1.962,-3.000){2}{\rule{0.250pt}{0.400pt}}
\put(420.17,572){\rule{0.400pt}{0.700pt}}
\multiput(421.17,573.55)(-2.000,-1.547){2}{\rule{0.400pt}{0.350pt}}
\multiput(418.95,569.37)(-0.447,-0.685){3}{\rule{0.108pt}{0.633pt}}
\multiput(419.17,570.69)(-3.000,-2.685){2}{\rule{0.400pt}{0.317pt}}
\multiput(414.92,566.95)(-0.462,-0.447){3}{\rule{0.500pt}{0.108pt}}
\multiput(415.96,567.17)(-1.962,-3.000){2}{\rule{0.250pt}{0.400pt}}
\put(412.17,562){\rule{0.400pt}{0.700pt}}
\multiput(413.17,563.55)(-2.000,-1.547){2}{\rule{0.400pt}{0.350pt}}
\multiput(409.92,560.95)(-0.462,-0.447){3}{\rule{0.500pt}{0.108pt}}
\multiput(410.96,561.17)(-1.962,-3.000){2}{\rule{0.250pt}{0.400pt}}
\put(407.17,555){\rule{0.400pt}{0.900pt}}
\multiput(408.17,557.13)(-2.000,-2.132){2}{\rule{0.400pt}{0.450pt}}
\multiput(404.92,553.95)(-0.462,-0.447){3}{\rule{0.500pt}{0.108pt}}
\multiput(405.96,554.17)(-1.962,-3.000){2}{\rule{0.250pt}{0.400pt}}
\multiput(401.92,550.95)(-0.462,-0.447){3}{\rule{0.500pt}{0.108pt}}
\multiput(402.96,551.17)(-1.962,-3.000){2}{\rule{0.250pt}{0.400pt}}
\put(399.17,546){\rule{0.400pt}{0.700pt}}
\multiput(400.17,547.55)(-2.000,-1.547){2}{\rule{0.400pt}{0.350pt}}
\multiput(397.95,543.37)(-0.447,-0.685){3}{\rule{0.108pt}{0.633pt}}
\multiput(398.17,544.69)(-3.000,-2.685){2}{\rule{0.400pt}{0.317pt}}
\multiput(393.92,540.95)(-0.462,-0.447){3}{\rule{0.500pt}{0.108pt}}
\multiput(394.96,541.17)(-1.962,-3.000){2}{\rule{0.250pt}{0.400pt}}
\put(391.17,536){\rule{0.400pt}{0.700pt}}
\multiput(392.17,537.55)(-2.000,-1.547){2}{\rule{0.400pt}{0.350pt}}
\multiput(388.92,534.95)(-0.462,-0.447){3}{\rule{0.500pt}{0.108pt}}
\multiput(389.96,535.17)(-1.962,-3.000){2}{\rule{0.250pt}{0.400pt}}
\put(386.17,529){\rule{0.400pt}{0.900pt}}
\multiput(387.17,531.13)(-2.000,-2.132){2}{\rule{0.400pt}{0.450pt}}
\multiput(383.92,527.95)(-0.462,-0.447){3}{\rule{0.500pt}{0.108pt}}
\multiput(384.96,528.17)(-1.962,-3.000){2}{\rule{0.250pt}{0.400pt}}
\multiput(380.92,524.95)(-0.462,-0.447){3}{\rule{0.500pt}{0.108pt}}
\multiput(381.96,525.17)(-1.962,-3.000){2}{\rule{0.250pt}{0.400pt}}
\put(378.17,520){\rule{0.400pt}{0.700pt}}
\multiput(379.17,521.55)(-2.000,-1.547){2}{\rule{0.400pt}{0.350pt}}
\multiput(376.95,517.37)(-0.447,-0.685){3}{\rule{0.108pt}{0.633pt}}
\multiput(377.17,518.69)(-3.000,-2.685){2}{\rule{0.400pt}{0.317pt}}
\multiput(372.92,514.95)(-0.462,-0.447){3}{\rule{0.500pt}{0.108pt}}
\multiput(373.96,515.17)(-1.962,-3.000){2}{\rule{0.250pt}{0.400pt}}
\put(370.17,510){\rule{0.400pt}{0.700pt}}
\multiput(371.17,511.55)(-2.000,-1.547){2}{\rule{0.400pt}{0.350pt}}
\multiput(367.92,508.95)(-0.462,-0.447){3}{\rule{0.500pt}{0.108pt}}
\multiput(368.96,509.17)(-1.962,-3.000){2}{\rule{0.250pt}{0.400pt}}
\put(365.17,503){\rule{0.400pt}{0.900pt}}
\multiput(366.17,505.13)(-2.000,-2.132){2}{\rule{0.400pt}{0.450pt}}
\multiput(362.92,501.95)(-0.462,-0.447){3}{\rule{0.500pt}{0.108pt}}
\multiput(363.96,502.17)(-1.962,-3.000){2}{\rule{0.250pt}{0.400pt}}
\multiput(359.92,498.95)(-0.462,-0.447){3}{\rule{0.500pt}{0.108pt}}
\multiput(360.96,499.17)(-1.962,-3.000){2}{\rule{0.250pt}{0.400pt}}
\put(357.17,494){\rule{0.400pt}{0.700pt}}
\multiput(358.17,495.55)(-2.000,-1.547){2}{\rule{0.400pt}{0.350pt}}
\multiput(355.95,491.37)(-0.447,-0.685){3}{\rule{0.108pt}{0.633pt}}
\multiput(356.17,492.69)(-3.000,-2.685){2}{\rule{0.400pt}{0.317pt}}
\put(352.17,487){\rule{0.400pt}{0.700pt}}
\multiput(353.17,488.55)(-2.000,-1.547){2}{\rule{0.400pt}{0.350pt}}
\multiput(349.92,485.95)(-0.462,-0.447){3}{\rule{0.500pt}{0.108pt}}
\multiput(350.96,486.17)(-1.962,-3.000){2}{\rule{0.250pt}{0.400pt}}
\multiput(346.92,482.95)(-0.462,-0.447){3}{\rule{0.500pt}{0.108pt}}
\multiput(347.96,483.17)(-1.962,-3.000){2}{\rule{0.250pt}{0.400pt}}
\put(344.17,477){\rule{0.400pt}{0.900pt}}
\multiput(345.17,479.13)(-2.000,-2.132){2}{\rule{0.400pt}{0.450pt}}
\multiput(341.92,475.95)(-0.462,-0.447){3}{\rule{0.500pt}{0.108pt}}
\multiput(342.96,476.17)(-1.962,-3.000){2}{\rule{0.250pt}{0.400pt}}
\multiput(338.92,472.95)(-0.462,-0.447){3}{\rule{0.500pt}{0.108pt}}
\multiput(339.96,473.17)(-1.962,-3.000){2}{\rule{0.250pt}{0.400pt}}
\put(336.17,468){\rule{0.400pt}{0.700pt}}
\multiput(337.17,469.55)(-2.000,-1.547){2}{\rule{0.400pt}{0.350pt}}
\multiput(334.95,465.37)(-0.447,-0.685){3}{\rule{0.108pt}{0.633pt}}
\multiput(335.17,466.69)(-3.000,-2.685){2}{\rule{0.400pt}{0.317pt}}
\put(331.17,461){\rule{0.400pt}{0.700pt}}
\multiput(332.17,462.55)(-2.000,-1.547){2}{\rule{0.400pt}{0.350pt}}
\multiput(328.92,459.95)(-0.462,-0.447){3}{\rule{0.500pt}{0.108pt}}
\multiput(329.96,460.17)(-1.962,-3.000){2}{\rule{0.250pt}{0.400pt}}
\multiput(325.92,456.95)(-0.462,-0.447){3}{\rule{0.500pt}{0.108pt}}
\multiput(326.96,457.17)(-1.962,-3.000){2}{\rule{0.250pt}{0.400pt}}
\put(323.17,451){\rule{0.400pt}{0.900pt}}
\multiput(324.17,453.13)(-2.000,-2.132){2}{\rule{0.400pt}{0.450pt}}
\multiput(320.92,449.95)(-0.462,-0.447){3}{\rule{0.500pt}{0.108pt}}
\multiput(321.96,450.17)(-1.962,-3.000){2}{\rule{0.250pt}{0.400pt}}
\multiput(317.92,446.95)(-0.462,-0.447){3}{\rule{0.500pt}{0.108pt}}
\multiput(318.96,447.17)(-1.962,-3.000){2}{\rule{0.250pt}{0.400pt}}
\put(315.17,442){\rule{0.400pt}{0.700pt}}
\multiput(316.17,443.55)(-2.000,-1.547){2}{\rule{0.400pt}{0.350pt}}
\multiput(313.95,439.37)(-0.447,-0.685){3}{\rule{0.108pt}{0.633pt}}
\multiput(314.17,440.69)(-3.000,-2.685){2}{\rule{0.400pt}{0.317pt}}
\put(310.17,435){\rule{0.400pt}{0.700pt}}
\multiput(311.17,436.55)(-2.000,-1.547){2}{\rule{0.400pt}{0.350pt}}
\multiput(307.92,433.95)(-0.462,-0.447){3}{\rule{0.500pt}{0.108pt}}
\multiput(308.96,434.17)(-1.962,-3.000){2}{\rule{0.250pt}{0.400pt}}
\multiput(304.92,430.95)(-0.462,-0.447){3}{\rule{0.500pt}{0.108pt}}
\multiput(305.96,431.17)(-1.962,-3.000){2}{\rule{0.250pt}{0.400pt}}
\put(302.17,426){\rule{0.400pt}{0.700pt}}
\multiput(303.17,427.55)(-2.000,-1.547){2}{\rule{0.400pt}{0.350pt}}
\multiput(300.95,423.37)(-0.447,-0.685){3}{\rule{0.108pt}{0.633pt}}
\multiput(301.17,424.69)(-3.000,-2.685){2}{\rule{0.400pt}{0.317pt}}
\put(297.17,419){\rule{0.400pt}{0.700pt}}
\multiput(298.17,420.55)(-2.000,-1.547){2}{\rule{0.400pt}{0.350pt}}
\multiput(294.92,417.95)(-0.462,-0.447){3}{\rule{0.500pt}{0.108pt}}
\multiput(295.96,418.17)(-1.962,-3.000){2}{\rule{0.250pt}{0.400pt}}
\multiput(291.92,414.95)(-0.462,-0.447){3}{\rule{0.500pt}{0.108pt}}
\multiput(292.96,415.17)(-1.962,-3.000){2}{\rule{0.250pt}{0.400pt}}
\put(289.17,409){\rule{0.400pt}{0.900pt}}
\multiput(290.17,411.13)(-2.000,-2.132){2}{\rule{0.400pt}{0.450pt}}
\multiput(286.92,407.95)(-0.462,-0.447){3}{\rule{0.500pt}{0.108pt}}
\multiput(287.96,408.17)(-1.962,-3.000){2}{\rule{0.250pt}{0.400pt}}
\multiput(283.92,404.95)(-0.462,-0.447){3}{\rule{0.500pt}{0.108pt}}
\multiput(284.96,405.17)(-1.962,-3.000){2}{\rule{0.250pt}{0.400pt}}
\put(281.17,400){\rule{0.400pt}{0.700pt}}
\multiput(282.17,401.55)(-2.000,-1.547){2}{\rule{0.400pt}{0.350pt}}
\multiput(279.95,397.37)(-0.447,-0.685){3}{\rule{0.108pt}{0.633pt}}
\multiput(280.17,398.69)(-3.000,-2.685){2}{\rule{0.400pt}{0.317pt}}
\put(276.17,393){\rule{0.400pt}{0.700pt}}
\multiput(277.17,394.55)(-2.000,-1.547){2}{\rule{0.400pt}{0.350pt}}
\multiput(273.92,391.95)(-0.462,-0.447){3}{\rule{0.500pt}{0.108pt}}
\multiput(274.96,392.17)(-1.962,-3.000){2}{\rule{0.250pt}{0.400pt}}
\multiput(270.92,388.95)(-0.462,-0.447){3}{\rule{0.500pt}{0.108pt}}
\multiput(271.96,389.17)(-1.962,-3.000){2}{\rule{0.250pt}{0.400pt}}
\put(268.17,383){\rule{0.400pt}{0.900pt}}
\multiput(269.17,385.13)(-2.000,-2.132){2}{\rule{0.400pt}{0.450pt}}
\multiput(265.92,381.95)(-0.462,-0.447){3}{\rule{0.500pt}{0.108pt}}
\multiput(266.96,382.17)(-1.962,-3.000){2}{\rule{0.250pt}{0.400pt}}
\put(263.17,377){\rule{0.400pt}{0.700pt}}
\multiput(264.17,378.55)(-2.000,-1.547){2}{\rule{0.400pt}{0.350pt}}
\multiput(260.92,375.95)(-0.462,-0.447){3}{\rule{0.500pt}{0.108pt}}
\multiput(261.96,376.17)(-1.962,-3.000){2}{\rule{0.250pt}{0.400pt}}
\multiput(258.95,371.37)(-0.447,-0.685){3}{\rule{0.108pt}{0.633pt}}
\multiput(259.17,372.69)(-3.000,-2.685){2}{\rule{0.400pt}{0.317pt}}
\put(255.17,367){\rule{0.400pt}{0.700pt}}
\multiput(256.17,368.55)(-2.000,-1.547){2}{\rule{0.400pt}{0.350pt}}
\multiput(252.92,365.95)(-0.462,-0.447){3}{\rule{0.500pt}{0.108pt}}
\multiput(253.96,366.17)(-1.962,-3.000){2}{\rule{0.250pt}{0.400pt}}
\multiput(249.92,362.95)(-0.462,-0.447){3}{\rule{0.500pt}{0.108pt}}
\multiput(250.96,363.17)(-1.962,-3.000){2}{\rule{0.250pt}{0.400pt}}
\put(247.17,357){\rule{0.400pt}{0.900pt}}
\multiput(248.17,359.13)(-2.000,-2.132){2}{\rule{0.400pt}{0.450pt}}
\multiput(244.92,355.95)(-0.462,-0.447){3}{\rule{0.500pt}{0.108pt}}
\multiput(245.96,356.17)(-1.962,-3.000){2}{\rule{0.250pt}{0.400pt}}
\put(242.17,351){\rule{0.400pt}{0.700pt}}
\multiput(243.17,352.55)(-2.000,-1.547){2}{\rule{0.400pt}{0.350pt}}
\multiput(239.92,349.95)(-0.462,-0.447){3}{\rule{0.500pt}{0.108pt}}
\multiput(240.96,350.17)(-1.962,-3.000){2}{\rule{0.250pt}{0.400pt}}
\multiput(237.95,345.37)(-0.447,-0.685){3}{\rule{0.108pt}{0.633pt}}
\multiput(238.17,346.69)(-3.000,-2.685){2}{\rule{0.400pt}{0.317pt}}
\put(234.17,341){\rule{0.400pt}{0.700pt}}
\multiput(235.17,342.55)(-2.000,-1.547){2}{\rule{0.400pt}{0.350pt}}
\multiput(231.92,339.95)(-0.462,-0.447){3}{\rule{0.500pt}{0.108pt}}
\multiput(232.96,340.17)(-1.962,-3.000){2}{\rule{0.250pt}{0.400pt}}
\multiput(228.92,336.95)(-0.462,-0.447){3}{\rule{0.500pt}{0.108pt}}
\multiput(229.96,337.17)(-1.962,-3.000){2}{\rule{0.250pt}{0.400pt}}
\put(226.17,331){\rule{0.400pt}{0.900pt}}
\multiput(227.17,333.13)(-2.000,-2.132){2}{\rule{0.400pt}{0.450pt}}
\multiput(223.92,329.95)(-0.462,-0.447){3}{\rule{0.500pt}{0.108pt}}
\multiput(224.96,330.17)(-1.962,-3.000){2}{\rule{0.250pt}{0.400pt}}
\put(221.17,325){\rule{0.400pt}{0.700pt}}
\multiput(222.17,326.55)(-2.000,-1.547){2}{\rule{0.400pt}{0.350pt}}
\multiput(218.92,323.95)(-0.462,-0.447){3}{\rule{0.500pt}{0.108pt}}
\multiput(219.96,324.17)(-1.962,-3.000){2}{\rule{0.250pt}{0.400pt}}
\multiput(216.95,319.37)(-0.447,-0.685){3}{\rule{0.108pt}{0.633pt}}
\multiput(217.17,320.69)(-3.000,-2.685){2}{\rule{0.400pt}{0.317pt}}
\put(213.17,315){\rule{0.400pt}{0.700pt}}
\multiput(214.17,316.55)(-2.000,-1.547){2}{\rule{0.400pt}{0.350pt}}
\multiput(210.92,313.95)(-0.462,-0.447){3}{\rule{0.500pt}{0.108pt}}
\multiput(211.96,314.17)(-1.962,-3.000){2}{\rule{0.250pt}{0.400pt}}
\put(208.17,309){\rule{0.400pt}{0.700pt}}
\multiput(209.17,310.55)(-2.000,-1.547){2}{\rule{0.400pt}{0.350pt}}
\multiput(206.95,306.37)(-0.447,-0.685){3}{\rule{0.108pt}{0.633pt}}
\multiput(207.17,307.69)(-3.000,-2.685){2}{\rule{0.400pt}{0.317pt}}
\multiput(202.92,303.95)(-0.462,-0.447){3}{\rule{0.500pt}{0.108pt}}
\multiput(203.96,304.17)(-1.962,-3.000){2}{\rule{0.250pt}{0.400pt}}
\put(200.17,299){\rule{0.400pt}{0.700pt}}
\multiput(201.17,300.55)(-2.000,-1.547){2}{\rule{0.400pt}{0.350pt}}
\multiput(197.92,297.95)(-0.462,-0.447){3}{\rule{0.500pt}{0.108pt}}
\multiput(198.96,298.17)(-1.962,-3.000){2}{\rule{0.250pt}{0.400pt}}
\multiput(195.95,293.37)(-0.447,-0.685){3}{\rule{0.108pt}{0.633pt}}
\multiput(196.17,294.69)(-3.000,-2.685){2}{\rule{0.400pt}{0.317pt}}
\put(192.17,289){\rule{0.400pt}{0.700pt}}
\multiput(193.17,290.55)(-2.000,-1.547){2}{\rule{0.400pt}{0.350pt}}
\multiput(189.92,287.95)(-0.462,-0.447){3}{\rule{0.500pt}{0.108pt}}
\multiput(190.96,288.17)(-1.962,-3.000){2}{\rule{0.250pt}{0.400pt}}
\put(187.17,283){\rule{0.400pt}{0.700pt}}
\multiput(188.17,284.55)(-2.000,-1.547){2}{\rule{0.400pt}{0.350pt}}
\multiput(184.92,281.95)(-0.462,-0.447){3}{\rule{0.500pt}{0.108pt}}
\multiput(185.96,282.17)(-1.962,-3.000){2}{\rule{0.250pt}{0.400pt}}
\multiput(182.95,277.37)(-0.447,-0.685){3}{\rule{0.108pt}{0.633pt}}
\multiput(183.17,278.69)(-3.000,-2.685){2}{\rule{0.400pt}{0.317pt}}
\put(179.17,273){\rule{0.400pt}{0.700pt}}
\multiput(180.17,274.55)(-2.000,-1.547){2}{\rule{0.400pt}{0.350pt}}
\multiput(176.92,271.95)(-0.462,-0.447){3}{\rule{0.500pt}{0.108pt}}
\multiput(177.96,272.17)(-1.962,-3.000){2}{\rule{0.250pt}{0.400pt}}
\multiput(173.92,268.95)(-0.462,-0.447){3}{\rule{0.500pt}{0.108pt}}
\multiput(174.96,269.17)(-1.962,-3.000){2}{\rule{0.250pt}{0.400pt}}
\put(171.17,263){\rule{0.400pt}{0.900pt}}
\multiput(172.17,265.13)(-2.000,-2.132){2}{\rule{0.400pt}{0.450pt}}
\multiput(168.92,261.95)(-0.462,-0.447){3}{\rule{0.500pt}{0.108pt}}
\multiput(169.96,262.17)(-1.962,-3.000){2}{\rule{0.250pt}{0.400pt}}
\put(166.17,257){\rule{0.400pt}{0.700pt}}
\multiput(167.17,258.55)(-2.000,-1.547){2}{\rule{0.400pt}{0.350pt}}
\multiput(163.92,255.95)(-0.462,-0.447){3}{\rule{0.500pt}{0.108pt}}
\multiput(164.96,256.17)(-1.962,-3.000){2}{\rule{0.250pt}{0.400pt}}
\multiput(161.95,251.37)(-0.447,-0.685){3}{\rule{0.108pt}{0.633pt}}
\multiput(162.17,252.69)(-3.000,-2.685){2}{\rule{0.400pt}{0.317pt}}
\put(158.17,247){\rule{0.400pt}{0.700pt}}
\multiput(159.17,248.55)(-2.000,-1.547){2}{\rule{0.400pt}{0.350pt}}
\multiput(155.92,245.95)(-0.462,-0.447){3}{\rule{0.500pt}{0.108pt}}
\multiput(156.96,246.17)(-1.962,-3.000){2}{\rule{0.250pt}{0.400pt}}
\multiput(155.00,242.95)(0.462,-0.447){3}{\rule{0.500pt}{0.108pt}}
\multiput(155.00,243.17)(1.962,-3.000){2}{\rule{0.250pt}{0.400pt}}
\put(158.17,237){\rule{0.400pt}{0.900pt}}
\multiput(157.17,239.13)(2.000,-2.132){2}{\rule{0.400pt}{0.450pt}}
\multiput(160.00,235.95)(0.462,-0.447){3}{\rule{0.500pt}{0.108pt}}
\multiput(160.00,236.17)(1.962,-3.000){2}{\rule{0.250pt}{0.400pt}}
\multiput(163.00,232.95)(0.462,-0.447){3}{\rule{0.500pt}{0.108pt}}
\multiput(163.00,233.17)(1.962,-3.000){2}{\rule{0.250pt}{0.400pt}}
\put(166.17,228){\rule{0.400pt}{0.700pt}}
\multiput(165.17,229.55)(2.000,-1.547){2}{\rule{0.400pt}{0.350pt}}
\multiput(168.61,225.37)(0.447,-0.685){3}{\rule{0.108pt}{0.633pt}}
\multiput(167.17,226.69)(3.000,-2.685){2}{\rule{0.400pt}{0.317pt}}
\put(171.17,221){\rule{0.400pt}{0.700pt}}
\multiput(170.17,222.55)(2.000,-1.547){2}{\rule{0.400pt}{0.350pt}}
\multiput(173.00,219.95)(0.462,-0.447){3}{\rule{0.500pt}{0.108pt}}
\multiput(173.00,220.17)(1.962,-3.000){2}{\rule{0.250pt}{0.400pt}}
\multiput(176.00,216.95)(0.462,-0.447){3}{\rule{0.500pt}{0.108pt}}
\multiput(176.00,217.17)(1.962,-3.000){2}{\rule{0.250pt}{0.400pt}}
\put(179.17,211){\rule{0.400pt}{0.900pt}}
\multiput(178.17,213.13)(2.000,-2.132){2}{\rule{0.400pt}{0.450pt}}
\multiput(181.00,209.95)(0.462,-0.447){3}{\rule{0.500pt}{0.108pt}}
\multiput(181.00,210.17)(1.962,-3.000){2}{\rule{0.250pt}{0.400pt}}
\multiput(184.00,206.95)(0.462,-0.447){3}{\rule{0.500pt}{0.108pt}}
\multiput(184.00,207.17)(1.962,-3.000){2}{\rule{0.250pt}{0.400pt}}
\put(187.17,202){\rule{0.400pt}{0.700pt}}
\multiput(186.17,203.55)(2.000,-1.547){2}{\rule{0.400pt}{0.350pt}}
\multiput(189.61,199.37)(0.447,-0.685){3}{\rule{0.108pt}{0.633pt}}
\multiput(188.17,200.69)(3.000,-2.685){2}{\rule{0.400pt}{0.317pt}}
\put(192.17,195){\rule{0.400pt}{0.700pt}}
\multiput(191.17,196.55)(2.000,-1.547){2}{\rule{0.400pt}{0.350pt}}
\multiput(194.00,193.95)(0.462,-0.447){3}{\rule{0.500pt}{0.108pt}}
\multiput(194.00,194.17)(1.962,-3.000){2}{\rule{0.250pt}{0.400pt}}
\multiput(197.00,190.95)(0.462,-0.447){3}{\rule{0.500pt}{0.108pt}}
\multiput(197.00,191.17)(1.962,-3.000){2}{\rule{0.250pt}{0.400pt}}
\put(200.17,185){\rule{0.400pt}{0.900pt}}
\multiput(199.17,187.13)(2.000,-2.132){2}{\rule{0.400pt}{0.450pt}}
\multiput(202.00,183.95)(0.462,-0.447){3}{\rule{0.500pt}{0.108pt}}
\multiput(202.00,184.17)(1.962,-3.000){2}{\rule{0.250pt}{0.400pt}}
\multiput(205.00,180.95)(0.462,-0.447){3}{\rule{0.500pt}{0.108pt}}
\multiput(205.00,181.17)(1.962,-3.000){2}{\rule{0.250pt}{0.400pt}}
\put(208.17,176){\rule{0.400pt}{0.700pt}}
\multiput(207.17,177.55)(2.000,-1.547){2}{\rule{0.400pt}{0.350pt}}
\multiput(210.61,173.37)(0.447,-0.685){3}{\rule{0.108pt}{0.633pt}}
\multiput(209.17,174.69)(3.000,-2.685){2}{\rule{0.400pt}{0.317pt}}
\put(213.17,169){\rule{0.400pt}{0.700pt}}
\multiput(212.17,170.55)(2.000,-1.547){2}{\rule{0.400pt}{0.350pt}}
\multiput(215.00,167.95)(0.462,-0.447){3}{\rule{0.500pt}{0.108pt}}
\multiput(215.00,168.17)(1.962,-3.000){2}{\rule{0.250pt}{0.400pt}}
\multiput(218.00,164.95)(0.462,-0.447){3}{\rule{0.500pt}{0.108pt}}
\multiput(218.00,165.17)(1.962,-3.000){2}{\rule{0.250pt}{0.400pt}}
\put(221.17,159){\rule{0.400pt}{0.900pt}}
\multiput(220.17,161.13)(2.000,-2.132){2}{\rule{0.400pt}{0.450pt}}
\multiput(223.00,157.95)(0.462,-0.447){3}{\rule{0.500pt}{0.108pt}}
\multiput(223.00,158.17)(1.962,-3.000){2}{\rule{0.250pt}{0.400pt}}
\put(226.17,153){\rule{0.400pt}{0.700pt}}
\multiput(225.17,154.55)(2.000,-1.547){2}{\rule{0.400pt}{0.350pt}}
\multiput(228.00,151.95)(0.462,-0.447){3}{\rule{0.500pt}{0.108pt}}
\multiput(228.00,152.17)(1.962,-3.000){2}{\rule{0.250pt}{0.400pt}}
\multiput(231.61,147.37)(0.447,-0.685){3}{\rule{0.108pt}{0.633pt}}
\multiput(230.17,148.69)(3.000,-2.685){2}{\rule{0.400pt}{0.317pt}}
\put(234.17,143){\rule{0.400pt}{0.700pt}}
\multiput(233.17,144.55)(2.000,-1.547){2}{\rule{0.400pt}{0.350pt}}
\multiput(236.00,141.95)(0.462,-0.447){3}{\rule{0.500pt}{0.108pt}}
\multiput(236.00,142.17)(1.962,-3.000){2}{\rule{0.250pt}{0.400pt}}
\multiput(239.00,138.95)(0.462,-0.447){3}{\rule{0.500pt}{0.108pt}}
\multiput(239.00,139.17)(1.962,-3.000){2}{\rule{0.250pt}{0.400pt}}
\put(242.17,133){\rule{0.400pt}{0.900pt}}
\multiput(241.17,135.13)(2.000,-2.132){2}{\rule{0.400pt}{0.450pt}}
\multiput(244.00,131.95)(0.462,-0.447){3}{\rule{0.500pt}{0.108pt}}
\multiput(244.00,132.17)(1.962,-3.000){2}{\rule{0.250pt}{0.400pt}}
\put(247.17,127){\rule{0.400pt}{0.700pt}}
\multiput(246.17,128.55)(2.000,-1.547){2}{\rule{0.400pt}{0.350pt}}
\multiput(249.00,125.95)(0.462,-0.447){3}{\rule{0.500pt}{0.108pt}}
\multiput(249.00,126.17)(1.962,-3.000){2}{\rule{0.250pt}{0.400pt}}
\multiput(252.00,122.95)(0.462,-0.447){3}{\rule{0.500pt}{0.108pt}}
\multiput(252.00,123.17)(1.962,-3.000){2}{\rule{0.250pt}{0.400pt}}
\put(255.17,117){\rule{0.400pt}{0.900pt}}
\multiput(254.17,119.13)(2.000,-2.132){2}{\rule{0.400pt}{0.450pt}}
\multiput(257.00,115.95)(0.462,-0.447){3}{\rule{0.500pt}{0.108pt}}
\multiput(257.00,116.17)(1.962,-3.000){2}{\rule{0.250pt}{0.400pt}}
\multiput(260.00,112.95)(0.462,-0.447){3}{\rule{0.500pt}{0.108pt}}
\multiput(260.00,113.17)(1.962,-3.000){2}{\rule{0.250pt}{0.400pt}}
\put(263.17,108){\rule{0.400pt}{0.700pt}}
\multiput(262.17,109.55)(2.000,-1.547){2}{\rule{0.400pt}{0.350pt}}
\multiput(265.61,105.37)(0.447,-0.685){3}{\rule{0.108pt}{0.633pt}}
\multiput(264.17,106.69)(3.000,-2.685){2}{\rule{0.400pt}{0.317pt}}
\put(268.17,104){\rule{0.400pt}{0.900pt}}
\multiput(267.17,104.00)(2.000,2.132){2}{\rule{0.400pt}{0.450pt}}
\multiput(270.00,108.61)(0.462,0.447){3}{\rule{0.500pt}{0.108pt}}
\multiput(270.00,107.17)(1.962,3.000){2}{\rule{0.250pt}{0.400pt}}
\multiput(273.00,111.61)(0.462,0.447){3}{\rule{0.500pt}{0.108pt}}
\multiput(273.00,110.17)(1.962,3.000){2}{\rule{0.250pt}{0.400pt}}
\put(276.17,114){\rule{0.400pt}{0.700pt}}
\multiput(275.17,114.00)(2.000,1.547){2}{\rule{0.400pt}{0.350pt}}
\multiput(278.61,117.00)(0.447,0.685){3}{\rule{0.108pt}{0.633pt}}
\multiput(277.17,117.00)(3.000,2.685){2}{\rule{0.400pt}{0.317pt}}
\put(281.17,121){\rule{0.400pt}{0.700pt}}
\multiput(280.17,121.00)(2.000,1.547){2}{\rule{0.400pt}{0.350pt}}
\multiput(283.00,124.61)(0.462,0.447){3}{\rule{0.500pt}{0.108pt}}
\multiput(283.00,123.17)(1.962,3.000){2}{\rule{0.250pt}{0.400pt}}
\multiput(286.00,127.61)(0.462,0.447){3}{\rule{0.500pt}{0.108pt}}
\multiput(286.00,126.17)(1.962,3.000){2}{\rule{0.250pt}{0.400pt}}
\put(289.17,130){\rule{0.400pt}{0.700pt}}
\multiput(288.17,130.00)(2.000,1.547){2}{\rule{0.400pt}{0.350pt}}
\multiput(291.61,133.00)(0.447,0.685){3}{\rule{0.108pt}{0.633pt}}
\multiput(290.17,133.00)(3.000,2.685){2}{\rule{0.400pt}{0.317pt}}
\multiput(294.00,137.61)(0.462,0.447){3}{\rule{0.500pt}{0.108pt}}
\multiput(294.00,136.17)(1.962,3.000){2}{\rule{0.250pt}{0.400pt}}
\put(297.17,140){\rule{0.400pt}{0.700pt}}
\multiput(296.17,140.00)(2.000,1.547){2}{\rule{0.400pt}{0.350pt}}
\multiput(299.00,143.61)(0.462,0.447){3}{\rule{0.500pt}{0.108pt}}
\multiput(299.00,142.17)(1.962,3.000){2}{\rule{0.250pt}{0.400pt}}
\put(302.17,146){\rule{0.400pt}{0.900pt}}
\multiput(301.17,146.00)(2.000,2.132){2}{\rule{0.400pt}{0.450pt}}
\multiput(304.00,150.61)(0.462,0.447){3}{\rule{0.500pt}{0.108pt}}
\multiput(304.00,149.17)(1.962,3.000){2}{\rule{0.250pt}{0.400pt}}
\multiput(307.00,153.61)(0.462,0.447){3}{\rule{0.500pt}{0.108pt}}
\multiput(307.00,152.17)(1.962,3.000){2}{\rule{0.250pt}{0.400pt}}
\put(310.17,156){\rule{0.400pt}{0.700pt}}
\multiput(309.17,156.00)(2.000,1.547){2}{\rule{0.400pt}{0.350pt}}
\multiput(312.61,159.00)(0.447,0.685){3}{\rule{0.108pt}{0.633pt}}
\multiput(311.17,159.00)(3.000,2.685){2}{\rule{0.400pt}{0.317pt}}
\put(315.17,163){\rule{0.400pt}{0.700pt}}
\multiput(314.17,163.00)(2.000,1.547){2}{\rule{0.400pt}{0.350pt}}
\multiput(317.00,166.61)(0.462,0.447){3}{\rule{0.500pt}{0.108pt}}
\multiput(317.00,165.17)(1.962,3.000){2}{\rule{0.250pt}{0.400pt}}
\multiput(320.00,169.61)(0.462,0.447){3}{\rule{0.500pt}{0.108pt}}
\multiput(320.00,168.17)(1.962,3.000){2}{\rule{0.250pt}{0.400pt}}
\put(323.17,172){\rule{0.400pt}{0.900pt}}
\multiput(322.17,172.00)(2.000,2.132){2}{\rule{0.400pt}{0.450pt}}
\multiput(325.00,176.61)(0.462,0.447){3}{\rule{0.500pt}{0.108pt}}
\multiput(325.00,175.17)(1.962,3.000){2}{\rule{0.250pt}{0.400pt}}
\multiput(328.00,179.61)(0.462,0.447){3}{\rule{0.500pt}{0.108pt}}
\multiput(328.00,178.17)(1.962,3.000){2}{\rule{0.250pt}{0.400pt}}
\put(331.17,182){\rule{0.400pt}{0.700pt}}
\multiput(330.17,182.00)(2.000,1.547){2}{\rule{0.400pt}{0.350pt}}
\multiput(333.61,185.00)(0.447,0.685){3}{\rule{0.108pt}{0.633pt}}
\multiput(332.17,185.00)(3.000,2.685){2}{\rule{0.400pt}{0.317pt}}
\put(336.17,189){\rule{0.400pt}{0.700pt}}
\multiput(335.17,189.00)(2.000,1.547){2}{\rule{0.400pt}{0.350pt}}
\multiput(338.00,192.61)(0.462,0.447){3}{\rule{0.500pt}{0.108pt}}
\multiput(338.00,191.17)(1.962,3.000){2}{\rule{0.250pt}{0.400pt}}
\multiput(341.00,195.61)(0.462,0.447){3}{\rule{0.500pt}{0.108pt}}
\multiput(341.00,194.17)(1.962,3.000){2}{\rule{0.250pt}{0.400pt}}
\put(344.17,198){\rule{0.400pt}{0.900pt}}
\multiput(343.17,198.00)(2.000,2.132){2}{\rule{0.400pt}{0.450pt}}
\multiput(346.00,202.61)(0.462,0.447){3}{\rule{0.500pt}{0.108pt}}
\multiput(346.00,201.17)(1.962,3.000){2}{\rule{0.250pt}{0.400pt}}
\multiput(349.00,205.61)(0.462,0.447){3}{\rule{0.500pt}{0.108pt}}
\multiput(349.00,204.17)(1.962,3.000){2}{\rule{0.250pt}{0.400pt}}
\put(352.17,208){\rule{0.400pt}{0.700pt}}
\multiput(351.17,208.00)(2.000,1.547){2}{\rule{0.400pt}{0.350pt}}
\multiput(354.61,211.00)(0.447,0.685){3}{\rule{0.108pt}{0.633pt}}
\multiput(353.17,211.00)(3.000,2.685){2}{\rule{0.400pt}{0.317pt}}
\put(357.17,215){\rule{0.400pt}{0.700pt}}
\multiput(356.17,215.00)(2.000,1.547){2}{\rule{0.400pt}{0.350pt}}
\multiput(359.00,218.61)(0.462,0.447){3}{\rule{0.500pt}{0.108pt}}
\multiput(359.00,217.17)(1.962,3.000){2}{\rule{0.250pt}{0.400pt}}
\multiput(362.00,221.61)(0.462,0.447){3}{\rule{0.500pt}{0.108pt}}
\multiput(362.00,220.17)(1.962,3.000){2}{\rule{0.250pt}{0.400pt}}
\put(365.17,224){\rule{0.400pt}{0.900pt}}
\multiput(364.17,224.00)(2.000,2.132){2}{\rule{0.400pt}{0.450pt}}
\multiput(367.00,228.61)(0.462,0.447){3}{\rule{0.500pt}{0.108pt}}
\multiput(367.00,227.17)(1.962,3.000){2}{\rule{0.250pt}{0.400pt}}
\put(370.17,231){\rule{0.400pt}{0.700pt}}
\multiput(369.17,231.00)(2.000,1.547){2}{\rule{0.400pt}{0.350pt}}
\multiput(372.00,234.61)(0.462,0.447){3}{\rule{0.500pt}{0.108pt}}
\multiput(372.00,233.17)(1.962,3.000){2}{\rule{0.250pt}{0.400pt}}
\multiput(375.61,237.00)(0.447,0.685){3}{\rule{0.108pt}{0.633pt}}
\multiput(374.17,237.00)(3.000,2.685){2}{\rule{0.400pt}{0.317pt}}
\put(378.17,241){\rule{0.400pt}{0.700pt}}
\multiput(377.17,241.00)(2.000,1.547){2}{\rule{0.400pt}{0.350pt}}
\multiput(380.00,244.61)(0.462,0.447){3}{\rule{0.500pt}{0.108pt}}
\multiput(380.00,243.17)(1.962,3.000){2}{\rule{0.250pt}{0.400pt}}
\multiput(383.00,247.61)(0.462,0.447){3}{\rule{0.500pt}{0.108pt}}
\multiput(383.00,246.17)(1.962,3.000){2}{\rule{0.250pt}{0.400pt}}
\put(386.17,250){\rule{0.400pt}{0.900pt}}
\multiput(385.17,250.00)(2.000,2.132){2}{\rule{0.400pt}{0.450pt}}
\multiput(388.00,254.61)(0.462,0.447){3}{\rule{0.500pt}{0.108pt}}
\multiput(388.00,253.17)(1.962,3.000){2}{\rule{0.250pt}{0.400pt}}
\put(391.17,257){\rule{0.400pt}{0.700pt}}
\multiput(390.17,257.00)(2.000,1.547){2}{\rule{0.400pt}{0.350pt}}
\multiput(393.00,260.61)(0.462,0.447){3}{\rule{0.500pt}{0.108pt}}
\multiput(393.00,259.17)(1.962,3.000){2}{\rule{0.250pt}{0.400pt}}
\multiput(396.61,263.00)(0.447,0.685){3}{\rule{0.108pt}{0.633pt}}
\multiput(395.17,263.00)(3.000,2.685){2}{\rule{0.400pt}{0.317pt}}
\put(399.17,267){\rule{0.400pt}{0.700pt}}
\multiput(398.17,267.00)(2.000,1.547){2}{\rule{0.400pt}{0.350pt}}
\multiput(401.00,270.61)(0.462,0.447){3}{\rule{0.500pt}{0.108pt}}
\multiput(401.00,269.17)(1.962,3.000){2}{\rule{0.250pt}{0.400pt}}
\multiput(404.00,273.61)(0.462,0.447){3}{\rule{0.500pt}{0.108pt}}
\multiput(404.00,272.17)(1.962,3.000){2}{\rule{0.250pt}{0.400pt}}
\put(407.17,276){\rule{0.400pt}{0.900pt}}
\multiput(406.17,276.00)(2.000,2.132){2}{\rule{0.400pt}{0.450pt}}
\multiput(409.00,280.61)(0.462,0.447){3}{\rule{0.500pt}{0.108pt}}
\multiput(409.00,279.17)(1.962,3.000){2}{\rule{0.250pt}{0.400pt}}
\put(412.17,283){\rule{0.400pt}{0.700pt}}
\multiput(411.17,283.00)(2.000,1.547){2}{\rule{0.400pt}{0.350pt}}
\multiput(414.00,286.61)(0.462,0.447){3}{\rule{0.500pt}{0.108pt}}
\multiput(414.00,285.17)(1.962,3.000){2}{\rule{0.250pt}{0.400pt}}
\multiput(417.00,289.61)(0.462,0.447){3}{\rule{0.500pt}{0.108pt}}
\multiput(417.00,288.17)(1.962,3.000){2}{\rule{0.250pt}{0.400pt}}
\put(420.17,292){\rule{0.400pt}{0.900pt}}
\multiput(419.17,292.00)(2.000,2.132){2}{\rule{0.400pt}{0.450pt}}
\multiput(422.00,296.61)(0.462,0.447){3}{\rule{0.500pt}{0.108pt}}
\multiput(422.00,295.17)(1.962,3.000){2}{\rule{0.250pt}{0.400pt}}
\put(425.17,299){\rule{0.400pt}{0.700pt}}
\multiput(424.17,299.00)(2.000,1.547){2}{\rule{0.400pt}{0.350pt}}
\multiput(427.00,302.61)(0.462,0.447){3}{\rule{0.500pt}{0.108pt}}
\multiput(427.00,301.17)(1.962,3.000){2}{\rule{0.250pt}{0.400pt}}
\multiput(430.61,305.00)(0.447,0.685){3}{\rule{0.108pt}{0.633pt}}
\multiput(429.17,305.00)(3.000,2.685){2}{\rule{0.400pt}{0.317pt}}
\put(433.17,309){\rule{0.400pt}{0.700pt}}
\multiput(432.17,309.00)(2.000,1.547){2}{\rule{0.400pt}{0.350pt}}
\multiput(435.00,312.61)(0.462,0.447){3}{\rule{0.500pt}{0.108pt}}
\multiput(435.00,311.17)(1.962,3.000){2}{\rule{0.250pt}{0.400pt}}
\multiput(438.00,315.61)(0.462,0.447){3}{\rule{0.500pt}{0.108pt}}
\multiput(438.00,314.17)(1.962,3.000){2}{\rule{0.250pt}{0.400pt}}
\put(441.17,318){\rule{0.400pt}{0.900pt}}
\multiput(440.17,318.00)(2.000,2.132){2}{\rule{0.400pt}{0.450pt}}
\multiput(443.00,322.61)(0.462,0.447){3}{\rule{0.500pt}{0.108pt}}
\multiput(443.00,321.17)(1.962,3.000){2}{\rule{0.250pt}{0.400pt}}
\put(446.17,325){\rule{0.400pt}{0.700pt}}
\multiput(445.17,325.00)(2.000,1.547){2}{\rule{0.400pt}{0.350pt}}
\multiput(448.00,328.61)(0.462,0.447){3}{\rule{0.500pt}{0.108pt}}
\multiput(448.00,327.17)(1.962,3.000){2}{\rule{0.250pt}{0.400pt}}
\multiput(451.61,331.00)(0.447,0.685){3}{\rule{0.108pt}{0.633pt}}
\multiput(450.17,331.00)(3.000,2.685){2}{\rule{0.400pt}{0.317pt}}
\put(454.17,335){\rule{0.400pt}{0.700pt}}
\multiput(453.17,335.00)(2.000,1.547){2}{\rule{0.400pt}{0.350pt}}
\multiput(456.00,338.61)(0.462,0.447){3}{\rule{0.500pt}{0.108pt}}
\multiput(456.00,337.17)(1.962,3.000){2}{\rule{0.250pt}{0.400pt}}
\multiput(459.00,341.61)(0.462,0.447){3}{\rule{0.500pt}{0.108pt}}
\multiput(459.00,340.17)(1.962,3.000){2}{\rule{0.250pt}{0.400pt}}
\put(462.17,344){\rule{0.400pt}{0.900pt}}
\multiput(461.17,344.00)(2.000,2.132){2}{\rule{0.400pt}{0.450pt}}
\multiput(464.00,348.61)(0.462,0.447){3}{\rule{0.500pt}{0.108pt}}
\multiput(464.00,347.17)(1.962,3.000){2}{\rule{0.250pt}{0.400pt}}
\put(467.17,351){\rule{0.400pt}{0.700pt}}
\multiput(466.17,351.00)(2.000,1.547){2}{\rule{0.400pt}{0.350pt}}
\multiput(469.00,354.61)(0.462,0.447){3}{\rule{0.500pt}{0.108pt}}
\multiput(469.00,353.17)(1.962,3.000){2}{\rule{0.250pt}{0.400pt}}
\multiput(472.61,357.00)(0.447,0.685){3}{\rule{0.108pt}{0.633pt}}
\multiput(471.17,357.00)(3.000,2.685){2}{\rule{0.400pt}{0.317pt}}
\put(475.17,361){\rule{0.400pt}{0.700pt}}
\multiput(474.17,361.00)(2.000,1.547){2}{\rule{0.400pt}{0.350pt}}
\multiput(477.00,364.61)(0.462,0.447){3}{\rule{0.500pt}{0.108pt}}
\multiput(477.00,363.17)(1.962,3.000){2}{\rule{0.250pt}{0.400pt}}
\put(480.17,367){\rule{0.400pt}{0.700pt}}
\multiput(479.17,367.00)(2.000,1.547){2}{\rule{0.400pt}{0.350pt}}
\multiput(482.61,370.00)(0.447,0.685){3}{\rule{0.108pt}{0.633pt}}
\multiput(481.17,370.00)(3.000,2.685){2}{\rule{0.400pt}{0.317pt}}
\multiput(485.00,374.61)(0.462,0.447){3}{\rule{0.500pt}{0.108pt}}
\multiput(485.00,373.17)(1.962,3.000){2}{\rule{0.250pt}{0.400pt}}
\put(488.17,377){\rule{0.400pt}{0.700pt}}
\multiput(487.17,377.00)(2.000,1.547){2}{\rule{0.400pt}{0.350pt}}
\multiput(490.00,380.61)(0.462,0.447){3}{\rule{0.500pt}{0.108pt}}
\multiput(490.00,379.17)(1.962,3.000){2}{\rule{0.250pt}{0.400pt}}
\multiput(493.61,383.00)(0.447,0.685){3}{\rule{0.108pt}{0.633pt}}
\multiput(492.17,383.00)(3.000,2.685){2}{\rule{0.400pt}{0.317pt}}
\put(496.17,387){\rule{0.400pt}{0.700pt}}
\multiput(495.17,387.00)(2.000,1.547){2}{\rule{0.400pt}{0.350pt}}
\multiput(498.00,390.61)(0.462,0.447){3}{\rule{0.500pt}{0.108pt}}
\multiput(498.00,389.17)(1.962,3.000){2}{\rule{0.250pt}{0.400pt}}
\put(501.17,393){\rule{0.400pt}{0.700pt}}
\multiput(500.17,393.00)(2.000,1.547){2}{\rule{0.400pt}{0.350pt}}
\multiput(503.61,396.00)(0.447,0.685){3}{\rule{0.108pt}{0.633pt}}
\multiput(502.17,396.00)(3.000,2.685){2}{\rule{0.400pt}{0.317pt}}
\multiput(506.00,400.61)(0.462,0.447){3}{\rule{0.500pt}{0.108pt}}
\multiput(506.00,399.17)(1.962,3.000){2}{\rule{0.250pt}{0.400pt}}
\put(509.17,403){\rule{0.400pt}{0.700pt}}
\multiput(508.17,403.00)(2.000,1.547){2}{\rule{0.400pt}{0.350pt}}
\multiput(511.00,406.61)(0.462,0.447){3}{\rule{0.500pt}{0.108pt}}
\multiput(511.00,405.17)(1.962,3.000){2}{\rule{0.250pt}{0.400pt}}
\put(514.17,409){\rule{0.400pt}{0.900pt}}
\multiput(513.17,409.00)(2.000,2.132){2}{\rule{0.400pt}{0.450pt}}
\multiput(516.00,413.61)(0.462,0.447){3}{\rule{0.500pt}{0.108pt}}
\multiput(516.00,412.17)(1.962,3.000){2}{\rule{0.250pt}{0.400pt}}
\multiput(519.00,416.61)(0.462,0.447){3}{\rule{0.500pt}{0.108pt}}
\multiput(519.00,415.17)(1.962,3.000){2}{\rule{0.250pt}{0.400pt}}
\put(522.17,419){\rule{0.400pt}{0.700pt}}
\multiput(521.17,419.00)(2.000,1.547){2}{\rule{0.400pt}{0.350pt}}
\multiput(524.61,422.00)(0.447,0.685){3}{\rule{0.108pt}{0.633pt}}
\multiput(523.17,422.00)(3.000,2.685){2}{\rule{0.400pt}{0.317pt}}
\multiput(527.00,426.61)(0.462,0.447){3}{\rule{0.500pt}{0.108pt}}
\multiput(527.00,425.17)(1.962,3.000){2}{\rule{0.250pt}{0.400pt}}
\put(530.17,429){\rule{0.400pt}{0.700pt}}
\multiput(529.17,429.00)(2.000,1.547){2}{\rule{0.400pt}{0.350pt}}
\multiput(532.00,432.61)(0.462,0.447){3}{\rule{0.500pt}{0.108pt}}
\multiput(532.00,431.17)(1.962,3.000){2}{\rule{0.250pt}{0.400pt}}
\put(535.17,435){\rule{0.400pt}{0.700pt}}
\multiput(534.17,435.00)(2.000,1.547){2}{\rule{0.400pt}{0.350pt}}
\multiput(537.61,438.00)(0.447,0.685){3}{\rule{0.108pt}{0.633pt}}
\multiput(536.17,438.00)(3.000,2.685){2}{\rule{0.400pt}{0.317pt}}
\multiput(540.00,442.61)(0.462,0.447){3}{\rule{0.500pt}{0.108pt}}
\multiput(540.00,441.17)(1.962,3.000){2}{\rule{0.250pt}{0.400pt}}
\put(543.17,445){\rule{0.400pt}{0.700pt}}
\multiput(542.17,445.00)(2.000,1.547){2}{\rule{0.400pt}{0.350pt}}
\multiput(545.00,448.61)(0.462,0.447){3}{\rule{0.500pt}{0.108pt}}
\multiput(545.00,447.17)(1.962,3.000){2}{\rule{0.250pt}{0.400pt}}
\multiput(548.61,451.00)(0.447,0.685){3}{\rule{0.108pt}{0.633pt}}
\multiput(547.17,451.00)(3.000,2.685){2}{\rule{0.400pt}{0.317pt}}
\put(551.17,455){\rule{0.400pt}{0.700pt}}
\multiput(550.17,455.00)(2.000,1.547){2}{\rule{0.400pt}{0.350pt}}
\multiput(553.00,458.61)(0.462,0.447){3}{\rule{0.500pt}{0.108pt}}
\multiput(553.00,457.17)(1.962,3.000){2}{\rule{0.250pt}{0.400pt}}
\put(556.17,461){\rule{0.400pt}{0.700pt}}
\multiput(555.17,461.00)(2.000,1.547){2}{\rule{0.400pt}{0.350pt}}
\multiput(558.61,464.00)(0.447,0.685){3}{\rule{0.108pt}{0.633pt}}
\multiput(557.17,464.00)(3.000,2.685){2}{\rule{0.400pt}{0.317pt}}
\multiput(561.00,468.61)(0.462,0.447){3}{\rule{0.500pt}{0.108pt}}
\multiput(561.00,467.17)(1.962,3.000){2}{\rule{0.250pt}{0.400pt}}
\put(564.17,471){\rule{0.400pt}{0.700pt}}
\multiput(563.17,471.00)(2.000,1.547){2}{\rule{0.400pt}{0.350pt}}
\multiput(566.00,474.61)(0.462,0.447){3}{\rule{0.500pt}{0.108pt}}
\multiput(566.00,473.17)(1.962,3.000){2}{\rule{0.250pt}{0.400pt}}
\put(569.17,477){\rule{0.400pt}{0.900pt}}
\multiput(568.17,477.00)(2.000,2.132){2}{\rule{0.400pt}{0.450pt}}
\multiput(571.00,481.61)(0.462,0.447){3}{\rule{0.500pt}{0.108pt}}
\multiput(571.00,480.17)(1.962,3.000){2}{\rule{0.250pt}{0.400pt}}
\multiput(574.00,484.61)(0.462,0.447){3}{\rule{0.500pt}{0.108pt}}
\multiput(574.00,483.17)(1.962,3.000){2}{\rule{0.250pt}{0.400pt}}
\put(577.17,487){\rule{0.400pt}{0.700pt}}
\multiput(576.17,487.00)(2.000,1.547){2}{\rule{0.400pt}{0.350pt}}
\multiput(579.61,490.00)(0.447,0.685){3}{\rule{0.108pt}{0.633pt}}
\multiput(578.17,490.00)(3.000,2.685){2}{\rule{0.400pt}{0.317pt}}
\multiput(582.00,494.61)(0.462,0.447){3}{\rule{0.500pt}{0.108pt}}
\multiput(582.00,493.17)(1.962,3.000){2}{\rule{0.250pt}{0.400pt}}
\put(585.17,497){\rule{0.400pt}{0.700pt}}
\multiput(584.17,497.00)(2.000,1.547){2}{\rule{0.400pt}{0.350pt}}
\multiput(587.00,500.61)(0.462,0.447){3}{\rule{0.500pt}{0.108pt}}
\multiput(587.00,499.17)(1.962,3.000){2}{\rule{0.250pt}{0.400pt}}
\put(590.17,503){\rule{0.400pt}{0.900pt}}
\multiput(589.17,503.00)(2.000,2.132){2}{\rule{0.400pt}{0.450pt}}
\multiput(592.00,507.61)(0.462,0.447){3}{\rule{0.500pt}{0.108pt}}
\multiput(592.00,506.17)(1.962,3.000){2}{\rule{0.250pt}{0.400pt}}
\multiput(595.00,510.61)(0.462,0.447){3}{\rule{0.500pt}{0.108pt}}
\multiput(595.00,509.17)(1.962,3.000){2}{\rule{0.250pt}{0.400pt}}
\put(598.17,513){\rule{0.400pt}{0.700pt}}
\multiput(597.17,513.00)(2.000,1.547){2}{\rule{0.400pt}{0.350pt}}
\multiput(600.61,516.00)(0.447,0.685){3}{\rule{0.108pt}{0.633pt}}
\multiput(599.17,516.00)(3.000,2.685){2}{\rule{0.400pt}{0.317pt}}
\multiput(603.00,520.61)(0.462,0.447){3}{\rule{0.500pt}{0.108pt}}
\multiput(603.00,519.17)(1.962,3.000){2}{\rule{0.250pt}{0.400pt}}
\put(606.17,523){\rule{0.400pt}{0.700pt}}
\multiput(605.17,523.00)(2.000,1.547){2}{\rule{0.400pt}{0.350pt}}
\multiput(608.00,526.61)(0.462,0.447){3}{\rule{0.500pt}{0.108pt}}
\multiput(608.00,525.17)(1.962,3.000){2}{\rule{0.250pt}{0.400pt}}
\put(611.17,529){\rule{0.400pt}{0.900pt}}
\multiput(610.17,529.00)(2.000,2.132){2}{\rule{0.400pt}{0.450pt}}
\multiput(613.00,533.61)(0.462,0.447){3}{\rule{0.500pt}{0.108pt}}
\multiput(613.00,532.17)(1.962,3.000){2}{\rule{0.250pt}{0.400pt}}
\multiput(616.00,536.61)(0.462,0.447){3}{\rule{0.500pt}{0.108pt}}
\multiput(616.00,535.17)(1.962,3.000){2}{\rule{0.250pt}{0.400pt}}
\put(619.17,539){\rule{0.400pt}{0.700pt}}
\multiput(618.17,539.00)(2.000,1.547){2}{\rule{0.400pt}{0.350pt}}
\multiput(621.61,542.00)(0.447,0.685){3}{\rule{0.108pt}{0.633pt}}
\multiput(620.17,542.00)(3.000,2.685){2}{\rule{0.400pt}{0.317pt}}
\put(624.17,546){\rule{0.400pt}{0.700pt}}
\multiput(623.17,546.00)(2.000,1.547){2}{\rule{0.400pt}{0.350pt}}
\multiput(626.00,549.61)(0.462,0.447){3}{\rule{0.500pt}{0.108pt}}
\multiput(626.00,548.17)(1.962,3.000){2}{\rule{0.250pt}{0.400pt}}
\multiput(629.00,552.61)(0.462,0.447){3}{\rule{0.500pt}{0.108pt}}
\multiput(629.00,551.17)(1.962,3.000){2}{\rule{0.250pt}{0.400pt}}
\put(632.17,555){\rule{0.400pt}{0.900pt}}
\multiput(631.17,555.00)(2.000,2.132){2}{\rule{0.400pt}{0.450pt}}
\multiput(634.00,559.61)(0.462,0.447){3}{\rule{0.500pt}{0.108pt}}
\multiput(634.00,558.17)(1.962,3.000){2}{\rule{0.250pt}{0.400pt}}
\multiput(637.00,562.61)(0.462,0.447){3}{\rule{0.500pt}{0.108pt}}
\multiput(637.00,561.17)(1.962,3.000){2}{\rule{0.250pt}{0.400pt}}
\put(640.17,565){\rule{0.400pt}{0.700pt}}
\multiput(639.17,565.00)(2.000,1.547){2}{\rule{0.400pt}{0.350pt}}
\multiput(642.61,568.00)(0.447,0.685){3}{\rule{0.108pt}{0.633pt}}
\multiput(641.17,568.00)(3.000,2.685){2}{\rule{0.400pt}{0.317pt}}
\put(645.17,572){\rule{0.400pt}{0.700pt}}
\multiput(644.17,572.00)(2.000,1.547){2}{\rule{0.400pt}{0.350pt}}
\multiput(647.00,575.61)(0.462,0.447){3}{\rule{0.500pt}{0.108pt}}
\multiput(647.00,574.17)(1.962,3.000){2}{\rule{0.250pt}{0.400pt}}
\multiput(650.00,578.61)(0.462,0.447){3}{\rule{0.500pt}{0.108pt}}
\multiput(650.00,577.17)(1.962,3.000){2}{\rule{0.250pt}{0.400pt}}
\put(653.17,581){\rule{0.400pt}{0.900pt}}
\multiput(652.17,581.00)(2.000,2.132){2}{\rule{0.400pt}{0.450pt}}
\multiput(655.00,585.61)(0.462,0.447){3}{\rule{0.500pt}{0.108pt}}
\multiput(655.00,584.17)(1.962,3.000){2}{\rule{0.250pt}{0.400pt}}
\put(658.17,588){\rule{0.400pt}{0.700pt}}
\multiput(657.17,588.00)(2.000,1.547){2}{\rule{0.400pt}{0.350pt}}
\multiput(660.00,591.61)(0.462,0.447){3}{\rule{0.500pt}{0.108pt}}
\multiput(660.00,590.17)(1.962,3.000){2}{\rule{0.250pt}{0.400pt}}
\multiput(663.00,594.61)(0.462,0.447){3}{\rule{0.500pt}{0.108pt}}
\multiput(663.00,593.17)(1.962,3.000){2}{\rule{0.250pt}{0.400pt}}
\put(666.17,597){\rule{0.400pt}{0.900pt}}
\multiput(665.17,597.00)(2.000,2.132){2}{\rule{0.400pt}{0.450pt}}
\multiput(668.00,601.61)(0.462,0.447){3}{\rule{0.500pt}{0.108pt}}
\multiput(668.00,600.17)(1.962,3.000){2}{\rule{0.250pt}{0.400pt}}
\multiput(671.00,604.61)(0.462,0.447){3}{\rule{0.500pt}{0.108pt}}
\multiput(671.00,603.17)(1.962,3.000){2}{\rule{0.250pt}{0.400pt}}
\put(674.17,607){\rule{0.400pt}{0.700pt}}
\multiput(673.17,607.00)(2.000,1.547){2}{\rule{0.400pt}{0.350pt}}
\multiput(676.61,610.00)(0.447,0.685){3}{\rule{0.108pt}{0.633pt}}
\multiput(675.17,610.00)(3.000,2.685){2}{\rule{0.400pt}{0.317pt}}
\put(679.17,614){\rule{0.400pt}{0.700pt}}
\multiput(678.17,614.00)(2.000,1.547){2}{\rule{0.400pt}{0.350pt}}
\multiput(681.00,617.61)(0.462,0.447){3}{\rule{0.500pt}{0.108pt}}
\multiput(681.00,616.17)(1.962,3.000){2}{\rule{0.250pt}{0.400pt}}
\multiput(684.00,620.61)(0.462,0.447){3}{\rule{0.500pt}{0.108pt}}
\multiput(684.00,619.17)(1.962,3.000){2}{\rule{0.250pt}{0.400pt}}
\put(687.17,623){\rule{0.400pt}{0.900pt}}
\multiput(686.17,623.00)(2.000,2.132){2}{\rule{0.400pt}{0.450pt}}
\multiput(689.00,627.61)(0.462,0.447){3}{\rule{0.500pt}{0.108pt}}
\multiput(689.00,626.17)(1.962,3.000){2}{\rule{0.250pt}{0.400pt}}
\multiput(692.00,630.61)(0.462,0.447){3}{\rule{0.500pt}{0.108pt}}
\multiput(692.00,629.17)(1.962,3.000){2}{\rule{0.250pt}{0.400pt}}
\put(695.17,633){\rule{0.400pt}{0.700pt}}
\multiput(694.17,633.00)(2.000,1.547){2}{\rule{0.400pt}{0.350pt}}
\multiput(697.61,636.00)(0.447,0.685){3}{\rule{0.108pt}{0.633pt}}
\multiput(696.17,636.00)(3.000,2.685){2}{\rule{0.400pt}{0.317pt}}
\put(700.17,640){\rule{0.400pt}{0.700pt}}
\multiput(699.17,640.00)(2.000,1.547){2}{\rule{0.400pt}{0.350pt}}
\multiput(702.00,643.61)(0.462,0.447){3}{\rule{0.500pt}{0.108pt}}
\multiput(702.00,642.17)(1.962,3.000){2}{\rule{0.250pt}{0.400pt}}
\multiput(705.00,646.61)(0.462,0.447){3}{\rule{0.500pt}{0.108pt}}
\multiput(705.00,645.17)(1.962,3.000){2}{\rule{0.250pt}{0.400pt}}
\put(708.17,649){\rule{0.400pt}{0.900pt}}
\multiput(707.17,649.00)(2.000,2.132){2}{\rule{0.400pt}{0.450pt}}
\multiput(710.00,653.61)(0.462,0.447){3}{\rule{0.500pt}{0.108pt}}
\multiput(710.00,652.17)(1.962,3.000){2}{\rule{0.250pt}{0.400pt}}
\put(713.17,656){\rule{0.400pt}{0.700pt}}
\multiput(712.17,656.00)(2.000,1.547){2}{\rule{0.400pt}{0.350pt}}
\multiput(715.00,659.61)(0.462,0.447){3}{\rule{0.500pt}{0.108pt}}
\multiput(715.00,658.17)(1.962,3.000){2}{\rule{0.250pt}{0.400pt}}
\multiput(718.61,662.00)(0.447,0.685){3}{\rule{0.108pt}{0.633pt}}
\multiput(717.17,662.00)(3.000,2.685){2}{\rule{0.400pt}{0.317pt}}
\put(721.17,666){\rule{0.400pt}{0.700pt}}
\multiput(720.17,666.00)(2.000,1.547){2}{\rule{0.400pt}{0.350pt}}
\multiput(723.00,669.61)(0.462,0.447){3}{\rule{0.500pt}{0.108pt}}
\multiput(723.00,668.17)(1.962,3.000){2}{\rule{0.250pt}{0.400pt}}
\multiput(726.00,672.61)(0.462,0.447){3}{\rule{0.500pt}{0.108pt}}
\multiput(726.00,671.17)(1.962,3.000){2}{\rule{0.250pt}{0.400pt}}
\put(729.17,675){\rule{0.400pt}{0.900pt}}
\multiput(728.17,675.00)(2.000,2.132){2}{\rule{0.400pt}{0.450pt}}
\multiput(731.00,679.61)(0.462,0.447){3}{\rule{0.500pt}{0.108pt}}
\multiput(731.00,678.17)(1.962,3.000){2}{\rule{0.250pt}{0.400pt}}
\put(734.17,682){\rule{0.400pt}{0.700pt}}
\multiput(733.17,682.00)(2.000,1.547){2}{\rule{0.400pt}{0.350pt}}
\multiput(736.00,685.61)(0.462,0.447){3}{\rule{0.500pt}{0.108pt}}
\multiput(736.00,684.17)(1.962,3.000){2}{\rule{0.250pt}{0.400pt}}
\multiput(739.61,688.00)(0.447,0.685){3}{\rule{0.108pt}{0.633pt}}
\multiput(738.17,688.00)(3.000,2.685){2}{\rule{0.400pt}{0.317pt}}
\put(742.17,692){\rule{0.400pt}{0.700pt}}
\multiput(741.17,692.00)(2.000,1.547){2}{\rule{0.400pt}{0.350pt}}
\multiput(744.00,695.61)(0.462,0.447){3}{\rule{0.500pt}{0.108pt}}
\multiput(744.00,694.17)(1.962,3.000){2}{\rule{0.250pt}{0.400pt}}
\multiput(747.00,698.61)(0.462,0.447){3}{\rule{0.500pt}{0.108pt}}
\multiput(747.00,697.17)(1.962,3.000){2}{\rule{0.250pt}{0.400pt}}
\put(750.17,701){\rule{0.400pt}{0.900pt}}
\multiput(749.17,701.00)(2.000,2.132){2}{\rule{0.400pt}{0.450pt}}
\multiput(752.00,705.61)(0.462,0.447){3}{\rule{0.500pt}{0.108pt}}
\multiput(752.00,704.17)(1.962,3.000){2}{\rule{0.250pt}{0.400pt}}
\put(755.17,708){\rule{0.400pt}{0.700pt}}
\multiput(754.17,708.00)(2.000,1.547){2}{\rule{0.400pt}{0.350pt}}
\multiput(757.00,711.61)(0.462,0.447){3}{\rule{0.500pt}{0.108pt}}
\multiput(757.00,710.17)(1.962,3.000){2}{\rule{0.250pt}{0.400pt}}
\multiput(760.61,714.00)(0.447,0.685){3}{\rule{0.108pt}{0.633pt}}
\multiput(759.17,714.00)(3.000,2.685){2}{\rule{0.400pt}{0.317pt}}
\put(763.17,718){\rule{0.400pt}{0.700pt}}
\multiput(762.17,718.00)(2.000,1.547){2}{\rule{0.400pt}{0.350pt}}
\multiput(765.00,721.61)(0.462,0.447){3}{\rule{0.500pt}{0.108pt}}
\multiput(765.00,720.17)(1.962,3.000){2}{\rule{0.250pt}{0.400pt}}
\put(768.17,724){\rule{0.400pt}{0.700pt}}
\multiput(767.17,724.00)(2.000,1.547){2}{\rule{0.400pt}{0.350pt}}
\multiput(770.61,727.00)(0.447,0.685){3}{\rule{0.108pt}{0.633pt}}
\multiput(769.17,727.00)(3.000,2.685){2}{\rule{0.400pt}{0.317pt}}
\multiput(773.00,731.61)(0.462,0.447){3}{\rule{0.500pt}{0.108pt}}
\multiput(773.00,730.17)(1.962,3.000){2}{\rule{0.250pt}{0.400pt}}
\put(776.17,734){\rule{0.400pt}{0.700pt}}
\multiput(775.17,734.00)(2.000,1.547){2}{\rule{0.400pt}{0.350pt}}
\multiput(778.00,737.61)(0.462,0.447){3}{\rule{0.500pt}{0.108pt}}
\multiput(778.00,736.17)(1.962,3.000){2}{\rule{0.250pt}{0.400pt}}
\multiput(781.61,740.00)(0.447,0.685){3}{\rule{0.108pt}{0.633pt}}
\multiput(780.17,740.00)(3.000,2.685){2}{\rule{0.400pt}{0.317pt}}
\put(784.17,744){\rule{0.400pt}{0.700pt}}
\multiput(783.17,744.00)(2.000,1.547){2}{\rule{0.400pt}{0.350pt}}
\multiput(786.00,747.61)(0.462,0.447){3}{\rule{0.500pt}{0.108pt}}
\multiput(786.00,746.17)(1.962,3.000){2}{\rule{0.250pt}{0.400pt}}
\put(789.17,750){\rule{0.400pt}{0.700pt}}
\multiput(788.17,750.00)(2.000,1.547){2}{\rule{0.400pt}{0.350pt}}
\multiput(791.00,753.61)(0.462,0.447){3}{\rule{0.500pt}{0.108pt}}
\multiput(791.00,752.17)(1.962,3.000){2}{\rule{0.250pt}{0.400pt}}
\multiput(794.61,756.00)(0.447,0.685){3}{\rule{0.108pt}{0.633pt}}
\multiput(793.17,756.00)(3.000,2.685){2}{\rule{0.400pt}{0.317pt}}
\put(797.17,760){\rule{0.400pt}{0.700pt}}
\multiput(796.17,760.00)(2.000,1.547){2}{\rule{0.400pt}{0.350pt}}
\multiput(799.00,763.61)(0.462,0.447){3}{\rule{0.500pt}{0.108pt}}
\multiput(799.00,762.17)(1.962,3.000){2}{\rule{0.250pt}{0.400pt}}
\put(802.17,766){\rule{0.400pt}{0.700pt}}
\multiput(801.17,766.00)(2.000,1.547){2}{\rule{0.400pt}{0.350pt}}
\multiput(804.61,769.00)(0.447,0.685){3}{\rule{0.108pt}{0.633pt}}
\multiput(803.17,769.00)(3.000,2.685){2}{\rule{0.400pt}{0.317pt}}
\multiput(807.00,773.61)(0.462,0.447){3}{\rule{0.500pt}{0.108pt}}
\multiput(807.00,772.17)(1.962,3.000){2}{\rule{0.250pt}{0.400pt}}
\put(810.17,776){\rule{0.400pt}{0.700pt}}
\multiput(809.17,776.00)(2.000,1.547){2}{\rule{0.400pt}{0.350pt}}
\multiput(812.00,779.61)(0.462,0.447){3}{\rule{0.500pt}{0.108pt}}
\multiput(812.00,778.17)(1.962,3.000){2}{\rule{0.250pt}{0.400pt}}
\multiput(815.61,782.00)(0.447,0.685){3}{\rule{0.108pt}{0.633pt}}
\multiput(814.17,782.00)(3.000,2.685){2}{\rule{0.400pt}{0.317pt}}
\put(818.17,786){\rule{0.400pt}{0.700pt}}
\multiput(817.17,786.00)(2.000,1.547){2}{\rule{0.400pt}{0.350pt}}
\multiput(820.00,789.61)(0.462,0.447){3}{\rule{0.500pt}{0.108pt}}
\multiput(820.00,788.17)(1.962,3.000){2}{\rule{0.250pt}{0.400pt}}
\put(823.17,792){\rule{0.400pt}{0.700pt}}
\multiput(822.17,792.00)(2.000,1.547){2}{\rule{0.400pt}{0.350pt}}
\multiput(825.61,795.00)(0.447,0.685){3}{\rule{0.108pt}{0.633pt}}
\multiput(824.17,795.00)(3.000,2.685){2}{\rule{0.400pt}{0.317pt}}
\multiput(828.00,799.61)(0.462,0.447){3}{\rule{0.500pt}{0.108pt}}
\multiput(828.00,798.17)(1.962,3.000){2}{\rule{0.250pt}{0.400pt}}
\put(831.17,802){\rule{0.400pt}{0.700pt}}
\multiput(830.17,802.00)(2.000,1.547){2}{\rule{0.400pt}{0.350pt}}
\multiput(833.00,805.61)(0.462,0.447){3}{\rule{0.500pt}{0.108pt}}
\multiput(833.00,804.17)(1.962,3.000){2}{\rule{0.250pt}{0.400pt}}
\multiput(836.61,808.00)(0.447,0.685){3}{\rule{0.108pt}{0.633pt}}
\multiput(835.17,808.00)(3.000,2.685){2}{\rule{0.400pt}{0.317pt}}
\put(839.17,812){\rule{0.400pt}{0.700pt}}
\multiput(838.17,812.00)(2.000,1.547){2}{\rule{0.400pt}{0.350pt}}
\multiput(841.00,815.61)(0.462,0.447){3}{\rule{0.500pt}{0.108pt}}
\multiput(841.00,814.17)(1.962,3.000){2}{\rule{0.250pt}{0.400pt}}
\put(844.17,818){\rule{0.400pt}{0.700pt}}
\multiput(843.17,818.00)(2.000,1.547){2}{\rule{0.400pt}{0.350pt}}
\multiput(846.61,821.00)(0.447,0.685){3}{\rule{0.108pt}{0.633pt}}
\multiput(845.17,821.00)(3.000,2.685){2}{\rule{0.400pt}{0.317pt}}
\multiput(849.00,825.61)(0.462,0.447){3}{\rule{0.500pt}{0.108pt}}
\multiput(849.00,824.17)(1.962,3.000){2}{\rule{0.250pt}{0.400pt}}
\put(852.17,828){\rule{0.400pt}{0.700pt}}
\multiput(851.17,828.00)(2.000,1.547){2}{\rule{0.400pt}{0.350pt}}
\multiput(854.00,831.61)(0.462,0.447){3}{\rule{0.500pt}{0.108pt}}
\multiput(854.00,830.17)(1.962,3.000){2}{\rule{0.250pt}{0.400pt}}
\put(857.17,834){\rule{0.400pt}{0.900pt}}
\multiput(856.17,834.00)(2.000,2.132){2}{\rule{0.400pt}{0.450pt}}
\multiput(859.61,835.37)(0.447,-0.685){3}{\rule{0.108pt}{0.633pt}}
\multiput(858.17,836.69)(3.000,-2.685){2}{\rule{0.400pt}{0.317pt}}
\multiput(862.00,832.95)(0.462,-0.447){3}{\rule{0.500pt}{0.108pt}}
\multiput(862.00,833.17)(1.962,-3.000){2}{\rule{0.250pt}{0.400pt}}
\put(865.17,828){\rule{0.400pt}{0.700pt}}
\multiput(864.17,829.55)(2.000,-1.547){2}{\rule{0.400pt}{0.350pt}}
\multiput(867.00,826.95)(0.462,-0.447){3}{\rule{0.500pt}{0.108pt}}
\multiput(867.00,827.17)(1.962,-3.000){2}{\rule{0.250pt}{0.400pt}}
\multiput(870.61,822.37)(0.447,-0.685){3}{\rule{0.108pt}{0.633pt}}
\multiput(869.17,823.69)(3.000,-2.685){2}{\rule{0.400pt}{0.317pt}}
\put(873.17,818){\rule{0.400pt}{0.700pt}}
\multiput(872.17,819.55)(2.000,-1.547){2}{\rule{0.400pt}{0.350pt}}
\multiput(875.00,816.95)(0.462,-0.447){3}{\rule{0.500pt}{0.108pt}}
\multiput(875.00,817.17)(1.962,-3.000){2}{\rule{0.250pt}{0.400pt}}
\put(878.17,812){\rule{0.400pt}{0.700pt}}
\multiput(877.17,813.55)(2.000,-1.547){2}{\rule{0.400pt}{0.350pt}}
\multiput(880.61,809.37)(0.447,-0.685){3}{\rule{0.108pt}{0.633pt}}
\multiput(879.17,810.69)(3.000,-2.685){2}{\rule{0.400pt}{0.317pt}}
\multiput(883.00,806.95)(0.462,-0.447){3}{\rule{0.500pt}{0.108pt}}
\multiput(883.00,807.17)(1.962,-3.000){2}{\rule{0.250pt}{0.400pt}}
\put(886.17,802){\rule{0.400pt}{0.700pt}}
\multiput(885.17,803.55)(2.000,-1.547){2}{\rule{0.400pt}{0.350pt}}
\multiput(888.00,800.95)(0.462,-0.447){3}{\rule{0.500pt}{0.108pt}}
\multiput(888.00,801.17)(1.962,-3.000){2}{\rule{0.250pt}{0.400pt}}
\multiput(891.61,796.37)(0.447,-0.685){3}{\rule{0.108pt}{0.633pt}}
\multiput(890.17,797.69)(3.000,-2.685){2}{\rule{0.400pt}{0.317pt}}
\put(894.17,792){\rule{0.400pt}{0.700pt}}
\multiput(893.17,793.55)(2.000,-1.547){2}{\rule{0.400pt}{0.350pt}}
\multiput(896.00,790.95)(0.462,-0.447){3}{\rule{0.500pt}{0.108pt}}
\multiput(896.00,791.17)(1.962,-3.000){2}{\rule{0.250pt}{0.400pt}}
\put(899.17,786){\rule{0.400pt}{0.700pt}}
\multiput(898.17,787.55)(2.000,-1.547){2}{\rule{0.400pt}{0.350pt}}
\multiput(901.61,783.37)(0.447,-0.685){3}{\rule{0.108pt}{0.633pt}}
\multiput(900.17,784.69)(3.000,-2.685){2}{\rule{0.400pt}{0.317pt}}
\multiput(904.00,780.95)(0.462,-0.447){3}{\rule{0.500pt}{0.108pt}}
\multiput(904.00,781.17)(1.962,-3.000){2}{\rule{0.250pt}{0.400pt}}
\put(907.17,776){\rule{0.400pt}{0.700pt}}
\multiput(906.17,777.55)(2.000,-1.547){2}{\rule{0.400pt}{0.350pt}}
\multiput(909.00,774.95)(0.462,-0.447){3}{\rule{0.500pt}{0.108pt}}
\multiput(909.00,775.17)(1.962,-3.000){2}{\rule{0.250pt}{0.400pt}}
\put(912.17,769){\rule{0.400pt}{0.900pt}}
\multiput(911.17,771.13)(2.000,-2.132){2}{\rule{0.400pt}{0.450pt}}
\multiput(914.00,767.95)(0.462,-0.447){3}{\rule{0.500pt}{0.108pt}}
\multiput(914.00,768.17)(1.962,-3.000){2}{\rule{0.250pt}{0.400pt}}
\multiput(917.00,764.95)(0.462,-0.447){3}{\rule{0.500pt}{0.108pt}}
\multiput(917.00,765.17)(1.962,-3.000){2}{\rule{0.250pt}{0.400pt}}
\put(920.17,760){\rule{0.400pt}{0.700pt}}
\multiput(919.17,761.55)(2.000,-1.547){2}{\rule{0.400pt}{0.350pt}}
\multiput(922.61,757.37)(0.447,-0.685){3}{\rule{0.108pt}{0.633pt}}
\multiput(921.17,758.69)(3.000,-2.685){2}{\rule{0.400pt}{0.317pt}}
\multiput(925.00,754.95)(0.462,-0.447){3}{\rule{0.500pt}{0.108pt}}
\multiput(925.00,755.17)(1.962,-3.000){2}{\rule{0.250pt}{0.400pt}}
\put(928.17,750){\rule{0.400pt}{0.700pt}}
\multiput(927.17,751.55)(2.000,-1.547){2}{\rule{0.400pt}{0.350pt}}
\multiput(930.00,748.95)(0.462,-0.447){3}{\rule{0.500pt}{0.108pt}}
\multiput(930.00,749.17)(1.962,-3.000){2}{\rule{0.250pt}{0.400pt}}
\put(933.17,744){\rule{0.400pt}{0.700pt}}
\multiput(932.17,745.55)(2.000,-1.547){2}{\rule{0.400pt}{0.350pt}}
\multiput(935.61,741.37)(0.447,-0.685){3}{\rule{0.108pt}{0.633pt}}
\multiput(934.17,742.69)(3.000,-2.685){2}{\rule{0.400pt}{0.317pt}}
\multiput(938.00,738.95)(0.462,-0.447){3}{\rule{0.500pt}{0.108pt}}
\multiput(938.00,739.17)(1.962,-3.000){2}{\rule{0.250pt}{0.400pt}}
\put(941.17,734){\rule{0.400pt}{0.700pt}}
\multiput(940.17,735.55)(2.000,-1.547){2}{\rule{0.400pt}{0.350pt}}
\multiput(943.00,732.95)(0.462,-0.447){3}{\rule{0.500pt}{0.108pt}}
\multiput(943.00,733.17)(1.962,-3.000){2}{\rule{0.250pt}{0.400pt}}
\put(946.17,727){\rule{0.400pt}{0.900pt}}
\multiput(945.17,729.13)(2.000,-2.132){2}{\rule{0.400pt}{0.450pt}}
\multiput(948.00,725.95)(0.462,-0.447){3}{\rule{0.500pt}{0.108pt}}
\multiput(948.00,726.17)(1.962,-3.000){2}{\rule{0.250pt}{0.400pt}}
\multiput(951.00,722.95)(0.462,-0.447){3}{\rule{0.500pt}{0.108pt}}
\multiput(951.00,723.17)(1.962,-3.000){2}{\rule{0.250pt}{0.400pt}}
\put(954.17,718){\rule{0.400pt}{0.700pt}}
\multiput(953.17,719.55)(2.000,-1.547){2}{\rule{0.400pt}{0.350pt}}
\multiput(956.61,715.37)(0.447,-0.685){3}{\rule{0.108pt}{0.633pt}}
\multiput(955.17,716.69)(3.000,-2.685){2}{\rule{0.400pt}{0.317pt}}
\multiput(959.00,712.95)(0.462,-0.447){3}{\rule{0.500pt}{0.108pt}}
\multiput(959.00,713.17)(1.962,-3.000){2}{\rule{0.250pt}{0.400pt}}
\put(962.17,708){\rule{0.400pt}{0.700pt}}
\multiput(961.17,709.55)(2.000,-1.547){2}{\rule{0.400pt}{0.350pt}}
\multiput(964.00,706.95)(0.462,-0.447){3}{\rule{0.500pt}{0.108pt}}
\multiput(964.00,707.17)(1.962,-3.000){2}{\rule{0.250pt}{0.400pt}}
\put(967.17,701){\rule{0.400pt}{0.900pt}}
\multiput(966.17,703.13)(2.000,-2.132){2}{\rule{0.400pt}{0.450pt}}
\multiput(969.00,699.95)(0.462,-0.447){3}{\rule{0.500pt}{0.108pt}}
\multiput(969.00,700.17)(1.962,-3.000){2}{\rule{0.250pt}{0.400pt}}
\multiput(972.00,696.95)(0.462,-0.447){3}{\rule{0.500pt}{0.108pt}}
\multiput(972.00,697.17)(1.962,-3.000){2}{\rule{0.250pt}{0.400pt}}
\put(975.17,692){\rule{0.400pt}{0.700pt}}
\multiput(974.17,693.55)(2.000,-1.547){2}{\rule{0.400pt}{0.350pt}}
\multiput(977.61,689.37)(0.447,-0.685){3}{\rule{0.108pt}{0.633pt}}
\multiput(976.17,690.69)(3.000,-2.685){2}{\rule{0.400pt}{0.317pt}}
\multiput(980.00,686.95)(0.462,-0.447){3}{\rule{0.500pt}{0.108pt}}
\multiput(980.00,687.17)(1.962,-3.000){2}{\rule{0.250pt}{0.400pt}}
\put(983.17,682){\rule{0.400pt}{0.700pt}}
\multiput(982.17,683.55)(2.000,-1.547){2}{\rule{0.400pt}{0.350pt}}
\multiput(985.00,680.95)(0.462,-0.447){3}{\rule{0.500pt}{0.108pt}}
\multiput(985.00,681.17)(1.962,-3.000){2}{\rule{0.250pt}{0.400pt}}
\put(988.17,675){\rule{0.400pt}{0.900pt}}
\multiput(987.17,677.13)(2.000,-2.132){2}{\rule{0.400pt}{0.450pt}}
\multiput(990.00,673.95)(0.462,-0.447){3}{\rule{0.500pt}{0.108pt}}
\multiput(990.00,674.17)(1.962,-3.000){2}{\rule{0.250pt}{0.400pt}}
\multiput(993.00,670.95)(0.462,-0.447){3}{\rule{0.500pt}{0.108pt}}
\multiput(993.00,671.17)(1.962,-3.000){2}{\rule{0.250pt}{0.400pt}}
\put(996.17,666){\rule{0.400pt}{0.700pt}}
\multiput(995.17,667.55)(2.000,-1.547){2}{\rule{0.400pt}{0.350pt}}
\multiput(998.61,663.37)(0.447,-0.685){3}{\rule{0.108pt}{0.633pt}}
\multiput(997.17,664.69)(3.000,-2.685){2}{\rule{0.400pt}{0.317pt}}
\put(1001.17,659){\rule{0.400pt}{0.700pt}}
\multiput(1000.17,660.55)(2.000,-1.547){2}{\rule{0.400pt}{0.350pt}}
\multiput(1003.00,657.95)(0.462,-0.447){3}{\rule{0.500pt}{0.108pt}}
\multiput(1003.00,658.17)(1.962,-3.000){2}{\rule{0.250pt}{0.400pt}}
\multiput(1006.00,654.95)(0.462,-0.447){3}{\rule{0.500pt}{0.108pt}}
\multiput(1006.00,655.17)(1.962,-3.000){2}{\rule{0.250pt}{0.400pt}}
\put(1009.17,649){\rule{0.400pt}{0.900pt}}
\multiput(1008.17,651.13)(2.000,-2.132){2}{\rule{0.400pt}{0.450pt}}
\multiput(1011.00,647.95)(0.462,-0.447){3}{\rule{0.500pt}{0.108pt}}
\multiput(1011.00,648.17)(1.962,-3.000){2}{\rule{0.250pt}{0.400pt}}
\multiput(1014.00,644.95)(0.462,-0.447){3}{\rule{0.500pt}{0.108pt}}
\multiput(1014.00,645.17)(1.962,-3.000){2}{\rule{0.250pt}{0.400pt}}
\put(1017.17,640){\rule{0.400pt}{0.700pt}}
\multiput(1016.17,641.55)(2.000,-1.547){2}{\rule{0.400pt}{0.350pt}}
\multiput(1019.61,637.37)(0.447,-0.685){3}{\rule{0.108pt}{0.633pt}}
\multiput(1018.17,638.69)(3.000,-2.685){2}{\rule{0.400pt}{0.317pt}}
\put(1022.17,633){\rule{0.400pt}{0.700pt}}
\multiput(1021.17,634.55)(2.000,-1.547){2}{\rule{0.400pt}{0.350pt}}
\multiput(1024.00,631.95)(0.462,-0.447){3}{\rule{0.500pt}{0.108pt}}
\multiput(1024.00,632.17)(1.962,-3.000){2}{\rule{0.250pt}{0.400pt}}
\multiput(1027.00,628.95)(0.462,-0.447){3}{\rule{0.500pt}{0.108pt}}
\multiput(1027.00,629.17)(1.962,-3.000){2}{\rule{0.250pt}{0.400pt}}
\put(1030.17,623){\rule{0.400pt}{0.900pt}}
\multiput(1029.17,625.13)(2.000,-2.132){2}{\rule{0.400pt}{0.450pt}}
\multiput(1032.00,621.95)(0.462,-0.447){3}{\rule{0.500pt}{0.108pt}}
\multiput(1032.00,622.17)(1.962,-3.000){2}{\rule{0.250pt}{0.400pt}}
\multiput(1035.00,618.95)(0.462,-0.447){3}{\rule{0.500pt}{0.108pt}}
\multiput(1035.00,619.17)(1.962,-3.000){2}{\rule{0.250pt}{0.400pt}}
\put(1038.17,614){\rule{0.400pt}{0.700pt}}
\multiput(1037.17,615.55)(2.000,-1.547){2}{\rule{0.400pt}{0.350pt}}
\multiput(1040.61,611.37)(0.447,-0.685){3}{\rule{0.108pt}{0.633pt}}
\multiput(1039.17,612.69)(3.000,-2.685){2}{\rule{0.400pt}{0.317pt}}
\put(1043.17,607){\rule{0.400pt}{0.700pt}}
\multiput(1042.17,608.55)(2.000,-1.547){2}{\rule{0.400pt}{0.350pt}}
\multiput(1045.00,605.95)(0.462,-0.447){3}{\rule{0.500pt}{0.108pt}}
\multiput(1045.00,606.17)(1.962,-3.000){2}{\rule{0.250pt}{0.400pt}}
\multiput(1048.00,602.95)(0.462,-0.447){3}{\rule{0.500pt}{0.108pt}}
\multiput(1048.00,603.17)(1.962,-3.000){2}{\rule{0.250pt}{0.400pt}}
\put(1051.17,597){\rule{0.400pt}{0.900pt}}
\multiput(1050.17,599.13)(2.000,-2.132){2}{\rule{0.400pt}{0.450pt}}
\multiput(1053.00,595.95)(0.462,-0.447){3}{\rule{0.500pt}{0.108pt}}
\multiput(1053.00,596.17)(1.962,-3.000){2}{\rule{0.250pt}{0.400pt}}
\put(1056.17,591){\rule{0.400pt}{0.700pt}}
\multiput(1055.17,592.55)(2.000,-1.547){2}{\rule{0.400pt}{0.350pt}}
\multiput(1058.00,589.95)(0.462,-0.447){3}{\rule{0.500pt}{0.108pt}}
\multiput(1058.00,590.17)(1.962,-3.000){2}{\rule{0.250pt}{0.400pt}}
\multiput(1061.00,586.95)(0.462,-0.447){3}{\rule{0.500pt}{0.108pt}}
\multiput(1061.00,587.17)(1.962,-3.000){2}{\rule{0.250pt}{0.400pt}}
\put(1064.17,581){\rule{0.400pt}{0.900pt}}
\multiput(1063.17,583.13)(2.000,-2.132){2}{\rule{0.400pt}{0.450pt}}
\multiput(1066.00,579.95)(0.462,-0.447){3}{\rule{0.500pt}{0.108pt}}
\multiput(1066.00,580.17)(1.962,-3.000){2}{\rule{0.250pt}{0.400pt}}
\multiput(1069.00,576.95)(0.462,-0.447){3}{\rule{0.500pt}{0.108pt}}
\multiput(1069.00,577.17)(1.962,-3.000){2}{\rule{0.250pt}{0.400pt}}
\put(1072.17,572){\rule{0.400pt}{0.700pt}}
\multiput(1071.17,573.55)(2.000,-1.547){2}{\rule{0.400pt}{0.350pt}}
\multiput(1074.61,569.37)(0.447,-0.685){3}{\rule{0.108pt}{0.633pt}}
\multiput(1073.17,570.69)(3.000,-2.685){2}{\rule{0.400pt}{0.317pt}}
\put(1077.17,565){\rule{0.400pt}{0.700pt}}
\multiput(1076.17,566.55)(2.000,-1.547){2}{\rule{0.400pt}{0.350pt}}
\multiput(1079.00,563.95)(0.462,-0.447){3}{\rule{0.500pt}{0.108pt}}
\multiput(1079.00,564.17)(1.962,-3.000){2}{\rule{0.250pt}{0.400pt}}
\multiput(1082.00,560.95)(0.462,-0.447){3}{\rule{0.500pt}{0.108pt}}
\multiput(1082.00,561.17)(1.962,-3.000){2}{\rule{0.250pt}{0.400pt}}
\put(1085.17,555){\rule{0.400pt}{0.900pt}}
\multiput(1084.17,557.13)(2.000,-2.132){2}{\rule{0.400pt}{0.450pt}}
\multiput(1087.00,553.95)(0.462,-0.447){3}{\rule{0.500pt}{0.108pt}}
\multiput(1087.00,554.17)(1.962,-3.000){2}{\rule{0.250pt}{0.400pt}}
\multiput(1090.00,550.95)(0.462,-0.447){3}{\rule{0.500pt}{0.108pt}}
\multiput(1090.00,551.17)(1.962,-3.000){2}{\rule{0.250pt}{0.400pt}}
\put(1093.17,546){\rule{0.400pt}{0.700pt}}
\multiput(1092.17,547.55)(2.000,-1.547){2}{\rule{0.400pt}{0.350pt}}
\multiput(1095.61,543.37)(0.447,-0.685){3}{\rule{0.108pt}{0.633pt}}
\multiput(1094.17,544.69)(3.000,-2.685){2}{\rule{0.400pt}{0.317pt}}
\put(1098.17,539){\rule{0.400pt}{0.700pt}}
\multiput(1097.17,540.55)(2.000,-1.547){2}{\rule{0.400pt}{0.350pt}}
\multiput(1100.00,537.95)(0.462,-0.447){3}{\rule{0.500pt}{0.108pt}}
\multiput(1100.00,538.17)(1.962,-3.000){2}{\rule{0.250pt}{0.400pt}}
\multiput(1103.00,534.95)(0.462,-0.447){3}{\rule{0.500pt}{0.108pt}}
\multiput(1103.00,535.17)(1.962,-3.000){2}{\rule{0.250pt}{0.400pt}}
\put(1106.17,529){\rule{0.400pt}{0.900pt}}
\multiput(1105.17,531.13)(2.000,-2.132){2}{\rule{0.400pt}{0.450pt}}
\multiput(1108.00,527.95)(0.462,-0.447){3}{\rule{0.500pt}{0.108pt}}
\multiput(1108.00,528.17)(1.962,-3.000){2}{\rule{0.250pt}{0.400pt}}
\put(1111.17,523){\rule{0.400pt}{0.700pt}}
\multiput(1110.17,524.55)(2.000,-1.547){2}{\rule{0.400pt}{0.350pt}}
\multiput(1113.00,521.95)(0.462,-0.447){3}{\rule{0.500pt}{0.108pt}}
\multiput(1113.00,522.17)(1.962,-3.000){2}{\rule{0.250pt}{0.400pt}}
\multiput(1116.61,517.37)(0.447,-0.685){3}{\rule{0.108pt}{0.633pt}}
\multiput(1115.17,518.69)(3.000,-2.685){2}{\rule{0.400pt}{0.317pt}}
\put(1119.17,513){\rule{0.400pt}{0.700pt}}
\multiput(1118.17,514.55)(2.000,-1.547){2}{\rule{0.400pt}{0.350pt}}
\multiput(1121.00,511.95)(0.462,-0.447){3}{\rule{0.500pt}{0.108pt}}
\multiput(1121.00,512.17)(1.962,-3.000){2}{\rule{0.250pt}{0.400pt}}
\multiput(1124.00,508.95)(0.462,-0.447){3}{\rule{0.500pt}{0.108pt}}
\multiput(1124.00,509.17)(1.962,-3.000){2}{\rule{0.250pt}{0.400pt}}
\put(1127.17,503){\rule{0.400pt}{0.900pt}}
\multiput(1126.17,505.13)(2.000,-2.132){2}{\rule{0.400pt}{0.450pt}}
\multiput(1129.00,501.95)(0.462,-0.447){3}{\rule{0.500pt}{0.108pt}}
\multiput(1129.00,502.17)(1.962,-3.000){2}{\rule{0.250pt}{0.400pt}}
\put(1132.17,497){\rule{0.400pt}{0.700pt}}
\multiput(1131.17,498.55)(2.000,-1.547){2}{\rule{0.400pt}{0.350pt}}
\multiput(1134.00,495.95)(0.462,-0.447){3}{\rule{0.500pt}{0.108pt}}
\multiput(1134.00,496.17)(1.962,-3.000){2}{\rule{0.250pt}{0.400pt}}
\multiput(1137.61,491.37)(0.447,-0.685){3}{\rule{0.108pt}{0.633pt}}
\multiput(1136.17,492.69)(3.000,-2.685){2}{\rule{0.400pt}{0.317pt}}
\put(1140.17,487){\rule{0.400pt}{0.700pt}}
\multiput(1139.17,488.55)(2.000,-1.547){2}{\rule{0.400pt}{0.350pt}}
\multiput(1142.00,485.95)(0.462,-0.447){3}{\rule{0.500pt}{0.108pt}}
\multiput(1142.00,486.17)(1.962,-3.000){2}{\rule{0.250pt}{0.400pt}}
\put(1145.17,481){\rule{0.400pt}{0.700pt}}
\multiput(1144.17,482.55)(2.000,-1.547){2}{\rule{0.400pt}{0.350pt}}
\multiput(1147.61,478.37)(0.447,-0.685){3}{\rule{0.108pt}{0.633pt}}
\multiput(1146.17,479.69)(3.000,-2.685){2}{\rule{0.400pt}{0.317pt}}
\multiput(1150.00,475.95)(0.462,-0.447){3}{\rule{0.500pt}{0.108pt}}
\multiput(1150.00,476.17)(1.962,-3.000){2}{\rule{0.250pt}{0.400pt}}
\put(1153.17,471){\rule{0.400pt}{0.700pt}}
\multiput(1152.17,472.55)(2.000,-1.547){2}{\rule{0.400pt}{0.350pt}}
\multiput(1155.00,469.95)(0.462,-0.447){3}{\rule{0.500pt}{0.108pt}}
\multiput(1155.00,470.17)(1.962,-3.000){2}{\rule{0.250pt}{0.400pt}}
\multiput(1158.61,465.37)(0.447,-0.685){3}{\rule{0.108pt}{0.633pt}}
\multiput(1157.17,466.69)(3.000,-2.685){2}{\rule{0.400pt}{0.317pt}}
\put(1161.17,461){\rule{0.400pt}{0.700pt}}
\multiput(1160.17,462.55)(2.000,-1.547){2}{\rule{0.400pt}{0.350pt}}
\multiput(1163.00,459.95)(0.462,-0.447){3}{\rule{0.500pt}{0.108pt}}
\multiput(1163.00,460.17)(1.962,-3.000){2}{\rule{0.250pt}{0.400pt}}
\put(1166.17,455){\rule{0.400pt}{0.700pt}}
\multiput(1165.17,456.55)(2.000,-1.547){2}{\rule{0.400pt}{0.350pt}}
\multiput(1168.61,452.37)(0.447,-0.685){3}{\rule{0.108pt}{0.633pt}}
\multiput(1167.17,453.69)(3.000,-2.685){2}{\rule{0.400pt}{0.317pt}}
\multiput(1171.00,449.95)(0.462,-0.447){3}{\rule{0.500pt}{0.108pt}}
\multiput(1171.00,450.17)(1.962,-3.000){2}{\rule{0.250pt}{0.400pt}}
\put(1174.17,445){\rule{0.400pt}{0.700pt}}
\multiput(1173.17,446.55)(2.000,-1.547){2}{\rule{0.400pt}{0.350pt}}
\multiput(1176.00,443.95)(0.462,-0.447){3}{\rule{0.500pt}{0.108pt}}
\multiput(1176.00,444.17)(1.962,-3.000){2}{\rule{0.250pt}{0.400pt}}
\multiput(1179.61,439.37)(0.447,-0.685){3}{\rule{0.108pt}{0.633pt}}
\multiput(1178.17,440.69)(3.000,-2.685){2}{\rule{0.400pt}{0.317pt}}
\put(1182.17,435){\rule{0.400pt}{0.700pt}}
\multiput(1181.17,436.55)(2.000,-1.547){2}{\rule{0.400pt}{0.350pt}}
\multiput(1184.00,433.95)(0.462,-0.447){3}{\rule{0.500pt}{0.108pt}}
\multiput(1184.00,434.17)(1.962,-3.000){2}{\rule{0.250pt}{0.400pt}}
\put(1187.17,429){\rule{0.400pt}{0.700pt}}
\multiput(1186.17,430.55)(2.000,-1.547){2}{\rule{0.400pt}{0.350pt}}
\multiput(1189.00,427.95)(0.462,-0.447){3}{\rule{0.500pt}{0.108pt}}
\multiput(1189.00,428.17)(1.962,-3.000){2}{\rule{0.250pt}{0.400pt}}
\multiput(1192.61,423.37)(0.447,-0.685){3}{\rule{0.108pt}{0.633pt}}
\multiput(1191.17,424.69)(3.000,-2.685){2}{\rule{0.400pt}{0.317pt}}
\put(1195.17,419){\rule{0.400pt}{0.700pt}}
\multiput(1194.17,420.55)(2.000,-1.547){2}{\rule{0.400pt}{0.350pt}}
\multiput(1197.00,417.95)(0.462,-0.447){3}{\rule{0.500pt}{0.108pt}}
\multiput(1197.00,418.17)(1.962,-3.000){2}{\rule{0.250pt}{0.400pt}}
\put(1200.17,413){\rule{0.400pt}{0.700pt}}
\multiput(1199.17,414.55)(2.000,-1.547){2}{\rule{0.400pt}{0.350pt}}
\multiput(1202.61,410.37)(0.447,-0.685){3}{\rule{0.108pt}{0.633pt}}
\multiput(1201.17,411.69)(3.000,-2.685){2}{\rule{0.400pt}{0.317pt}}
\multiput(1205.00,407.95)(0.462,-0.447){3}{\rule{0.500pt}{0.108pt}}
\multiput(1205.00,408.17)(1.962,-3.000){2}{\rule{0.250pt}{0.400pt}}
\put(1208.17,403){\rule{0.400pt}{0.700pt}}
\multiput(1207.17,404.55)(2.000,-1.547){2}{\rule{0.400pt}{0.350pt}}
\multiput(1210.00,401.95)(0.462,-0.447){3}{\rule{0.500pt}{0.108pt}}
\multiput(1210.00,402.17)(1.962,-3.000){2}{\rule{0.250pt}{0.400pt}}
\multiput(1213.61,397.37)(0.447,-0.685){3}{\rule{0.108pt}{0.633pt}}
\multiput(1212.17,398.69)(3.000,-2.685){2}{\rule{0.400pt}{0.317pt}}
\put(1216.17,393){\rule{0.400pt}{0.700pt}}
\multiput(1215.17,394.55)(2.000,-1.547){2}{\rule{0.400pt}{0.350pt}}
\multiput(1218.00,391.95)(0.462,-0.447){3}{\rule{0.500pt}{0.108pt}}
\multiput(1218.00,392.17)(1.962,-3.000){2}{\rule{0.250pt}{0.400pt}}
\put(1221.17,387){\rule{0.400pt}{0.700pt}}
\multiput(1220.17,388.55)(2.000,-1.547){2}{\rule{0.400pt}{0.350pt}}
\multiput(1223.61,384.37)(0.447,-0.685){3}{\rule{0.108pt}{0.633pt}}
\multiput(1222.17,385.69)(3.000,-2.685){2}{\rule{0.400pt}{0.317pt}}
\multiput(1226.00,381.95)(0.462,-0.447){3}{\rule{0.500pt}{0.108pt}}
\multiput(1226.00,382.17)(1.962,-3.000){2}{\rule{0.250pt}{0.400pt}}
\put(1229.17,377){\rule{0.400pt}{0.700pt}}
\multiput(1228.17,378.55)(2.000,-1.547){2}{\rule{0.400pt}{0.350pt}}
\multiput(1231.00,375.95)(0.462,-0.447){3}{\rule{0.500pt}{0.108pt}}
\multiput(1231.00,376.17)(1.962,-3.000){2}{\rule{0.250pt}{0.400pt}}
\multiput(1234.61,371.37)(0.447,-0.685){3}{\rule{0.108pt}{0.633pt}}
\multiput(1233.17,372.69)(3.000,-2.685){2}{\rule{0.400pt}{0.317pt}}
\put(1237.17,367){\rule{0.400pt}{0.700pt}}
\multiput(1236.17,368.55)(2.000,-1.547){2}{\rule{0.400pt}{0.350pt}}
\multiput(1239.00,365.95)(0.462,-0.447){3}{\rule{0.500pt}{0.108pt}}
\multiput(1239.00,366.17)(1.962,-3.000){2}{\rule{0.250pt}{0.400pt}}
\put(1242.17,361){\rule{0.400pt}{0.700pt}}
\multiput(1241.17,362.55)(2.000,-1.547){2}{\rule{0.400pt}{0.350pt}}
\multiput(1244.61,358.37)(0.447,-0.685){3}{\rule{0.108pt}{0.633pt}}
\multiput(1243.17,359.69)(3.000,-2.685){2}{\rule{0.400pt}{0.317pt}}
\multiput(1247.00,355.95)(0.462,-0.447){3}{\rule{0.500pt}{0.108pt}}
\multiput(1247.00,356.17)(1.962,-3.000){2}{\rule{0.250pt}{0.400pt}}
\put(1250.17,351){\rule{0.400pt}{0.700pt}}
\multiput(1249.17,352.55)(2.000,-1.547){2}{\rule{0.400pt}{0.350pt}}
\multiput(1252.00,349.95)(0.462,-0.447){3}{\rule{0.500pt}{0.108pt}}
\multiput(1252.00,350.17)(1.962,-3.000){2}{\rule{0.250pt}{0.400pt}}
\put(1255.17,344){\rule{0.400pt}{0.900pt}}
\multiput(1254.17,346.13)(2.000,-2.132){2}{\rule{0.400pt}{0.450pt}}
\multiput(1257.00,342.95)(0.462,-0.447){3}{\rule{0.500pt}{0.108pt}}
\multiput(1257.00,343.17)(1.962,-3.000){2}{\rule{0.250pt}{0.400pt}}
\multiput(1260.00,339.95)(0.462,-0.447){3}{\rule{0.500pt}{0.108pt}}
\multiput(1260.00,340.17)(1.962,-3.000){2}{\rule{0.250pt}{0.400pt}}
\put(1263.17,335){\rule{0.400pt}{0.700pt}}
\multiput(1262.17,336.55)(2.000,-1.547){2}{\rule{0.400pt}{0.350pt}}
\multiput(1265.61,332.37)(0.447,-0.685){3}{\rule{0.108pt}{0.633pt}}
\multiput(1264.17,333.69)(3.000,-2.685){2}{\rule{0.400pt}{0.317pt}}
\multiput(1268.00,329.95)(0.462,-0.447){3}{\rule{0.500pt}{0.108pt}}
\multiput(1268.00,330.17)(1.962,-3.000){2}{\rule{0.250pt}{0.400pt}}
\put(1271.17,325){\rule{0.400pt}{0.700pt}}
\multiput(1270.17,326.55)(2.000,-1.547){2}{\rule{0.400pt}{0.350pt}}
\multiput(1273.00,323.95)(0.462,-0.447){3}{\rule{0.500pt}{0.108pt}}
\multiput(1273.00,324.17)(1.962,-3.000){2}{\rule{0.250pt}{0.400pt}}
\put(1276.17,318){\rule{0.400pt}{0.900pt}}
\multiput(1275.17,320.13)(2.000,-2.132){2}{\rule{0.400pt}{0.450pt}}
\multiput(1278.00,316.95)(0.462,-0.447){3}{\rule{0.500pt}{0.108pt}}
\multiput(1278.00,317.17)(1.962,-3.000){2}{\rule{0.250pt}{0.400pt}}
\multiput(1281.00,313.95)(0.462,-0.447){3}{\rule{0.500pt}{0.108pt}}
\multiput(1281.00,314.17)(1.962,-3.000){2}{\rule{0.250pt}{0.400pt}}
\put(1284.17,309){\rule{0.400pt}{0.700pt}}
\multiput(1283.17,310.55)(2.000,-1.547){2}{\rule{0.400pt}{0.350pt}}
\multiput(1286.61,306.37)(0.447,-0.685){3}{\rule{0.108pt}{0.633pt}}
\multiput(1285.17,307.69)(3.000,-2.685){2}{\rule{0.400pt}{0.317pt}}
\put(1289.17,302){\rule{0.400pt}{0.700pt}}
\multiput(1288.17,303.55)(2.000,-1.547){2}{\rule{0.400pt}{0.350pt}}
\multiput(1291.00,300.95)(0.462,-0.447){3}{\rule{0.500pt}{0.108pt}}
\multiput(1291.00,301.17)(1.962,-3.000){2}{\rule{0.250pt}{0.400pt}}
\multiput(1294.00,297.95)(0.462,-0.447){3}{\rule{0.500pt}{0.108pt}}
\multiput(1294.00,298.17)(1.962,-3.000){2}{\rule{0.250pt}{0.400pt}}
\put(1297.17,292){\rule{0.400pt}{0.900pt}}
\multiput(1296.17,294.13)(2.000,-2.132){2}{\rule{0.400pt}{0.450pt}}
\multiput(1299.00,290.95)(0.462,-0.447){3}{\rule{0.500pt}{0.108pt}}
\multiput(1299.00,291.17)(1.962,-3.000){2}{\rule{0.250pt}{0.400pt}}
\multiput(1302.00,287.95)(0.462,-0.447){3}{\rule{0.500pt}{0.108pt}}
\multiput(1302.00,288.17)(1.962,-3.000){2}{\rule{0.250pt}{0.400pt}}
\put(1305.17,283){\rule{0.400pt}{0.700pt}}
\multiput(1304.17,284.55)(2.000,-1.547){2}{\rule{0.400pt}{0.350pt}}
\multiput(1307.00,281.95)(0.462,-0.447){3}{\rule{0.500pt}{0.108pt}}
\multiput(1307.00,282.17)(1.962,-3.000){2}{\rule{0.250pt}{0.400pt}}
\put(1310.17,276){\rule{0.400pt}{0.900pt}}
\multiput(1309.17,278.13)(2.000,-2.132){2}{\rule{0.400pt}{0.450pt}}
\multiput(1312.00,274.95)(0.462,-0.447){3}{\rule{0.500pt}{0.108pt}}
\multiput(1312.00,275.17)(1.962,-3.000){2}{\rule{0.250pt}{0.400pt}}
\multiput(1315.00,271.95)(0.462,-0.447){3}{\rule{0.500pt}{0.108pt}}
\multiput(1315.00,272.17)(1.962,-3.000){2}{\rule{0.250pt}{0.400pt}}
\put(1318.17,267){\rule{0.400pt}{0.700pt}}
\multiput(1317.17,268.55)(2.000,-1.547){2}{\rule{0.400pt}{0.350pt}}
\multiput(1320.61,264.37)(0.447,-0.685){3}{\rule{0.108pt}{0.633pt}}
\multiput(1319.17,265.69)(3.000,-2.685){2}{\rule{0.400pt}{0.317pt}}
\multiput(1323.00,261.95)(0.462,-0.447){3}{\rule{0.500pt}{0.108pt}}
\multiput(1323.00,262.17)(1.962,-3.000){2}{\rule{0.250pt}{0.400pt}}
\put(1326.17,257){\rule{0.400pt}{0.700pt}}
\multiput(1325.17,258.55)(2.000,-1.547){2}{\rule{0.400pt}{0.350pt}}
\multiput(1328.00,255.95)(0.462,-0.447){3}{\rule{0.500pt}{0.108pt}}
\multiput(1328.00,256.17)(1.962,-3.000){2}{\rule{0.250pt}{0.400pt}}
\put(1331.17,250){\rule{0.400pt}{0.900pt}}
\multiput(1330.17,252.13)(2.000,-2.132){2}{\rule{0.400pt}{0.450pt}}
\multiput(1333.00,248.95)(0.462,-0.447){3}{\rule{0.500pt}{0.108pt}}
\multiput(1333.00,249.17)(1.962,-3.000){2}{\rule{0.250pt}{0.400pt}}
\multiput(1336.00,245.95)(0.462,-0.447){3}{\rule{0.500pt}{0.108pt}}
\multiput(1336.00,246.17)(1.962,-3.000){2}{\rule{0.250pt}{0.400pt}}
\put(1339.17,241){\rule{0.400pt}{0.700pt}}
\multiput(1338.17,242.55)(2.000,-1.547){2}{\rule{0.400pt}{0.350pt}}
\multiput(1341.61,238.37)(0.447,-0.685){3}{\rule{0.108pt}{0.633pt}}
\multiput(1340.17,239.69)(3.000,-2.685){2}{\rule{0.400pt}{0.317pt}}
\put(1344.17,234){\rule{0.400pt}{0.700pt}}
\multiput(1343.17,235.55)(2.000,-1.547){2}{\rule{0.400pt}{0.350pt}}
\multiput(1346.00,232.95)(0.462,-0.447){3}{\rule{0.500pt}{0.108pt}}
\multiput(1346.00,233.17)(1.962,-3.000){2}{\rule{0.250pt}{0.400pt}}
\multiput(1349.00,229.95)(0.462,-0.447){3}{\rule{0.500pt}{0.108pt}}
\multiput(1349.00,230.17)(1.962,-3.000){2}{\rule{0.250pt}{0.400pt}}
\put(1352.17,224){\rule{0.400pt}{0.900pt}}
\multiput(1351.17,226.13)(2.000,-2.132){2}{\rule{0.400pt}{0.450pt}}
\multiput(1354.00,222.95)(0.462,-0.447){3}{\rule{0.500pt}{0.108pt}}
\multiput(1354.00,223.17)(1.962,-3.000){2}{\rule{0.250pt}{0.400pt}}
\multiput(1357.00,219.95)(0.462,-0.447){3}{\rule{0.500pt}{0.108pt}}
\multiput(1357.00,220.17)(1.962,-3.000){2}{\rule{0.250pt}{0.400pt}}
\put(1360.17,215){\rule{0.400pt}{0.700pt}}
\multiput(1359.17,216.55)(2.000,-1.547){2}{\rule{0.400pt}{0.350pt}}
\multiput(1362.61,212.37)(0.447,-0.685){3}{\rule{0.108pt}{0.633pt}}
\multiput(1361.17,213.69)(3.000,-2.685){2}{\rule{0.400pt}{0.317pt}}
\put(1365.17,208){\rule{0.400pt}{0.700pt}}
\multiput(1364.17,209.55)(2.000,-1.547){2}{\rule{0.400pt}{0.350pt}}
\multiput(1367.00,206.95)(0.462,-0.447){3}{\rule{0.500pt}{0.108pt}}
\multiput(1367.00,207.17)(1.962,-3.000){2}{\rule{0.250pt}{0.400pt}}
\multiput(1370.00,203.95)(0.462,-0.447){3}{\rule{0.500pt}{0.108pt}}
\multiput(1370.00,204.17)(1.962,-3.000){2}{\rule{0.250pt}{0.400pt}}
\put(1373.17,198){\rule{0.400pt}{0.900pt}}
\multiput(1372.17,200.13)(2.000,-2.132){2}{\rule{0.400pt}{0.450pt}}
\multiput(1375.00,196.95)(0.462,-0.447){3}{\rule{0.500pt}{0.108pt}}
\multiput(1375.00,197.17)(1.962,-3.000){2}{\rule{0.250pt}{0.400pt}}
\multiput(1378.00,193.95)(0.462,-0.447){3}{\rule{0.500pt}{0.108pt}}
\multiput(1378.00,194.17)(1.962,-3.000){2}{\rule{0.250pt}{0.400pt}}
\put(1381.17,189){\rule{0.400pt}{0.700pt}}
\multiput(1380.17,190.55)(2.000,-1.547){2}{\rule{0.400pt}{0.350pt}}
\multiput(1383.61,186.37)(0.447,-0.685){3}{\rule{0.108pt}{0.633pt}}
\multiput(1382.17,187.69)(3.000,-2.685){2}{\rule{0.400pt}{0.317pt}}
\put(1386.17,182){\rule{0.400pt}{0.700pt}}
\multiput(1385.17,183.55)(2.000,-1.547){2}{\rule{0.400pt}{0.350pt}}
\multiput(1388.00,180.95)(0.462,-0.447){3}{\rule{0.500pt}{0.108pt}}
\multiput(1388.00,181.17)(1.962,-3.000){2}{\rule{0.250pt}{0.400pt}}
\multiput(1391.00,177.95)(0.462,-0.447){3}{\rule{0.500pt}{0.108pt}}
\multiput(1391.00,178.17)(1.962,-3.000){2}{\rule{0.250pt}{0.400pt}}
\put(1394.17,172){\rule{0.400pt}{0.900pt}}
\multiput(1393.17,174.13)(2.000,-2.132){2}{\rule{0.400pt}{0.450pt}}
\multiput(1396.00,170.95)(0.462,-0.447){3}{\rule{0.500pt}{0.108pt}}
\multiput(1396.00,171.17)(1.962,-3.000){2}{\rule{0.250pt}{0.400pt}}
\put(1399.17,166){\rule{0.400pt}{0.700pt}}
\multiput(1398.17,167.55)(2.000,-1.547){2}{\rule{0.400pt}{0.350pt}}
\multiput(1401.00,164.95)(0.462,-0.447){3}{\rule{0.500pt}{0.108pt}}
\multiput(1401.00,165.17)(1.962,-3.000){2}{\rule{0.250pt}{0.400pt}}
\multiput(1404.61,160.37)(0.447,-0.685){3}{\rule{0.108pt}{0.633pt}}
\multiput(1403.17,161.69)(3.000,-2.685){2}{\rule{0.400pt}{0.317pt}}
\put(1407.17,156){\rule{0.400pt}{0.700pt}}
\multiput(1406.17,157.55)(2.000,-1.547){2}{\rule{0.400pt}{0.350pt}}
\multiput(1409.00,154.95)(0.462,-0.447){3}{\rule{0.500pt}{0.108pt}}
\multiput(1409.00,155.17)(1.962,-3.000){2}{\rule{0.250pt}{0.400pt}}
\multiput(1412.00,151.95)(0.462,-0.447){3}{\rule{0.500pt}{0.108pt}}
\multiput(1412.00,152.17)(1.962,-3.000){2}{\rule{0.250pt}{0.400pt}}
\multiput(1413.95,147.37)(-0.447,-0.685){3}{\rule{0.108pt}{0.633pt}}
\multiput(1414.17,148.69)(-3.000,-2.685){2}{\rule{0.400pt}{0.317pt}}
\multiput(1409.92,144.95)(-0.462,-0.447){3}{\rule{0.500pt}{0.108pt}}
\multiput(1410.96,145.17)(-1.962,-3.000){2}{\rule{0.250pt}{0.400pt}}
\put(1407.17,140){\rule{0.400pt}{0.700pt}}
\multiput(1408.17,141.55)(-2.000,-1.547){2}{\rule{0.400pt}{0.350pt}}
\multiput(1404.92,138.95)(-0.462,-0.447){3}{\rule{0.500pt}{0.108pt}}
\multiput(1405.96,139.17)(-1.962,-3.000){2}{\rule{0.250pt}{0.400pt}}
\multiput(1402.95,134.37)(-0.447,-0.685){3}{\rule{0.108pt}{0.633pt}}
\multiput(1403.17,135.69)(-3.000,-2.685){2}{\rule{0.400pt}{0.317pt}}
\put(1399.17,130){\rule{0.400pt}{0.700pt}}
\multiput(1400.17,131.55)(-2.000,-1.547){2}{\rule{0.400pt}{0.350pt}}
\multiput(1396.92,128.95)(-0.462,-0.447){3}{\rule{0.500pt}{0.108pt}}
\multiput(1397.96,129.17)(-1.962,-3.000){2}{\rule{0.250pt}{0.400pt}}
\put(1394.17,124){\rule{0.400pt}{0.700pt}}
\multiput(1395.17,125.55)(-2.000,-1.547){2}{\rule{0.400pt}{0.350pt}}
\multiput(1391.92,122.95)(-0.462,-0.447){3}{\rule{0.500pt}{0.108pt}}
\multiput(1392.96,123.17)(-1.962,-3.000){2}{\rule{0.250pt}{0.400pt}}
\multiput(1389.95,118.37)(-0.447,-0.685){3}{\rule{0.108pt}{0.633pt}}
\multiput(1390.17,119.69)(-3.000,-2.685){2}{\rule{0.400pt}{0.317pt}}
\put(1386.17,114){\rule{0.400pt}{0.700pt}}
\multiput(1387.17,115.55)(-2.000,-1.547){2}{\rule{0.400pt}{0.350pt}}
\multiput(1383.92,112.95)(-0.462,-0.447){3}{\rule{0.500pt}{0.108pt}}
\multiput(1384.96,113.17)(-1.962,-3.000){2}{\rule{0.250pt}{0.400pt}}
\put(1381.17,108){\rule{0.400pt}{0.700pt}}
\multiput(1382.17,109.55)(-2.000,-1.547){2}{\rule{0.400pt}{0.350pt}}
\multiput(1379.95,105.37)(-0.447,-0.685){3}{\rule{0.108pt}{0.633pt}}
\multiput(1380.17,106.69)(-3.000,-2.685){2}{\rule{0.400pt}{0.317pt}}
\multiput(1376.95,104.00)(-0.447,0.685){3}{\rule{0.108pt}{0.633pt}}
\multiput(1377.17,104.00)(-3.000,2.685){2}{\rule{0.400pt}{0.317pt}}
\put(1373.17,108){\rule{0.400pt}{0.700pt}}
\multiput(1374.17,108.00)(-2.000,1.547){2}{\rule{0.400pt}{0.350pt}}
\multiput(1370.92,111.61)(-0.462,0.447){3}{\rule{0.500pt}{0.108pt}}
\multiput(1371.96,110.17)(-1.962,3.000){2}{\rule{0.250pt}{0.400pt}}
\multiput(1367.92,114.61)(-0.462,0.447){3}{\rule{0.500pt}{0.108pt}}
\multiput(1368.96,113.17)(-1.962,3.000){2}{\rule{0.250pt}{0.400pt}}
\put(1365.17,117){\rule{0.400pt}{0.900pt}}
\multiput(1366.17,117.00)(-2.000,2.132){2}{\rule{0.400pt}{0.450pt}}
\multiput(1362.92,121.61)(-0.462,0.447){3}{\rule{0.500pt}{0.108pt}}
\multiput(1363.96,120.17)(-1.962,3.000){2}{\rule{0.250pt}{0.400pt}}
\put(1360.17,124){\rule{0.400pt}{0.700pt}}
\multiput(1361.17,124.00)(-2.000,1.547){2}{\rule{0.400pt}{0.350pt}}
\multiput(1357.92,127.61)(-0.462,0.447){3}{\rule{0.500pt}{0.108pt}}
\multiput(1358.96,126.17)(-1.962,3.000){2}{\rule{0.250pt}{0.400pt}}
\multiput(1354.92,130.61)(-0.462,0.447){3}{\rule{0.500pt}{0.108pt}}
\multiput(1355.96,129.17)(-1.962,3.000){2}{\rule{0.250pt}{0.400pt}}
\put(1352.17,133){\rule{0.400pt}{0.900pt}}
\multiput(1353.17,133.00)(-2.000,2.132){2}{\rule{0.400pt}{0.450pt}}
\multiput(1349.92,137.61)(-0.462,0.447){3}{\rule{0.500pt}{0.108pt}}
\multiput(1350.96,136.17)(-1.962,3.000){2}{\rule{0.250pt}{0.400pt}}
\multiput(1346.92,140.61)(-0.462,0.447){3}{\rule{0.500pt}{0.108pt}}
\multiput(1347.96,139.17)(-1.962,3.000){2}{\rule{0.250pt}{0.400pt}}
\put(1344.17,143){\rule{0.400pt}{0.700pt}}
\multiput(1345.17,143.00)(-2.000,1.547){2}{\rule{0.400pt}{0.350pt}}
\multiput(1342.95,146.00)(-0.447,0.685){3}{\rule{0.108pt}{0.633pt}}
\multiput(1343.17,146.00)(-3.000,2.685){2}{\rule{0.400pt}{0.317pt}}
\put(1339.17,150){\rule{0.400pt}{0.700pt}}
\multiput(1340.17,150.00)(-2.000,1.547){2}{\rule{0.400pt}{0.350pt}}
\multiput(1336.92,153.61)(-0.462,0.447){3}{\rule{0.500pt}{0.108pt}}
\multiput(1337.96,152.17)(-1.962,3.000){2}{\rule{0.250pt}{0.400pt}}
\multiput(1333.92,156.61)(-0.462,0.447){3}{\rule{0.500pt}{0.108pt}}
\multiput(1334.96,155.17)(-1.962,3.000){2}{\rule{0.250pt}{0.400pt}}
\put(1331.17,159){\rule{0.400pt}{0.900pt}}
\multiput(1332.17,159.00)(-2.000,2.132){2}{\rule{0.400pt}{0.450pt}}
\multiput(1328.92,163.61)(-0.462,0.447){3}{\rule{0.500pt}{0.108pt}}
\multiput(1329.96,162.17)(-1.962,3.000){2}{\rule{0.250pt}{0.400pt}}
\put(1326.17,166){\rule{0.400pt}{0.700pt}}
\multiput(1327.17,166.00)(-2.000,1.547){2}{\rule{0.400pt}{0.350pt}}
\multiput(1323.92,169.61)(-0.462,0.447){3}{\rule{0.500pt}{0.108pt}}
\multiput(1324.96,168.17)(-1.962,3.000){2}{\rule{0.250pt}{0.400pt}}
\multiput(1321.95,172.00)(-0.447,0.685){3}{\rule{0.108pt}{0.633pt}}
\multiput(1322.17,172.00)(-3.000,2.685){2}{\rule{0.400pt}{0.317pt}}
\put(1318.17,176){\rule{0.400pt}{0.700pt}}
\multiput(1319.17,176.00)(-2.000,1.547){2}{\rule{0.400pt}{0.350pt}}
\multiput(1315.92,179.61)(-0.462,0.447){3}{\rule{0.500pt}{0.108pt}}
\multiput(1316.96,178.17)(-1.962,3.000){2}{\rule{0.250pt}{0.400pt}}
\multiput(1312.92,182.61)(-0.462,0.447){3}{\rule{0.500pt}{0.108pt}}
\multiput(1313.96,181.17)(-1.962,3.000){2}{\rule{0.250pt}{0.400pt}}
\put(1310.17,185){\rule{0.400pt}{0.900pt}}
\multiput(1311.17,185.00)(-2.000,2.132){2}{\rule{0.400pt}{0.450pt}}
\multiput(1307.92,189.61)(-0.462,0.447){3}{\rule{0.500pt}{0.108pt}}
\multiput(1308.96,188.17)(-1.962,3.000){2}{\rule{0.250pt}{0.400pt}}
\put(1305.17,192){\rule{0.400pt}{0.700pt}}
\multiput(1306.17,192.00)(-2.000,1.547){2}{\rule{0.400pt}{0.350pt}}
\multiput(1302.92,195.61)(-0.462,0.447){3}{\rule{0.500pt}{0.108pt}}
\multiput(1303.96,194.17)(-1.962,3.000){2}{\rule{0.250pt}{0.400pt}}
\multiput(1300.95,198.00)(-0.447,0.685){3}{\rule{0.108pt}{0.633pt}}
\multiput(1301.17,198.00)(-3.000,2.685){2}{\rule{0.400pt}{0.317pt}}
\put(1297.17,202){\rule{0.400pt}{0.700pt}}
\multiput(1298.17,202.00)(-2.000,1.547){2}{\rule{0.400pt}{0.350pt}}
\multiput(1294.92,205.61)(-0.462,0.447){3}{\rule{0.500pt}{0.108pt}}
\multiput(1295.96,204.17)(-1.962,3.000){2}{\rule{0.250pt}{0.400pt}}
\multiput(1291.92,208.61)(-0.462,0.447){3}{\rule{0.500pt}{0.108pt}}
\multiput(1292.96,207.17)(-1.962,3.000){2}{\rule{0.250pt}{0.400pt}}
\put(1289.17,211){\rule{0.400pt}{0.900pt}}
\multiput(1290.17,211.00)(-2.000,2.132){2}{\rule{0.400pt}{0.450pt}}
\multiput(1286.92,215.61)(-0.462,0.447){3}{\rule{0.500pt}{0.108pt}}
\multiput(1287.96,214.17)(-1.962,3.000){2}{\rule{0.250pt}{0.400pt}}
\put(1284.17,218){\rule{0.400pt}{0.700pt}}
\multiput(1285.17,218.00)(-2.000,1.547){2}{\rule{0.400pt}{0.350pt}}
\multiput(1281.92,221.61)(-0.462,0.447){3}{\rule{0.500pt}{0.108pt}}
\multiput(1282.96,220.17)(-1.962,3.000){2}{\rule{0.250pt}{0.400pt}}
\multiput(1279.95,224.00)(-0.447,0.685){3}{\rule{0.108pt}{0.633pt}}
\multiput(1280.17,224.00)(-3.000,2.685){2}{\rule{0.400pt}{0.317pt}}
\put(1276.17,228){\rule{0.400pt}{0.700pt}}
\multiput(1277.17,228.00)(-2.000,1.547){2}{\rule{0.400pt}{0.350pt}}
\multiput(1273.92,231.61)(-0.462,0.447){3}{\rule{0.500pt}{0.108pt}}
\multiput(1274.96,230.17)(-1.962,3.000){2}{\rule{0.250pt}{0.400pt}}
\put(1271.17,234){\rule{0.400pt}{0.700pt}}
\multiput(1272.17,234.00)(-2.000,1.547){2}{\rule{0.400pt}{0.350pt}}
\multiput(1269.95,237.00)(-0.447,0.685){3}{\rule{0.108pt}{0.633pt}}
\multiput(1270.17,237.00)(-3.000,2.685){2}{\rule{0.400pt}{0.317pt}}
\multiput(1265.92,241.61)(-0.462,0.447){3}{\rule{0.500pt}{0.108pt}}
\multiput(1266.96,240.17)(-1.962,3.000){2}{\rule{0.250pt}{0.400pt}}
\put(1263.17,244){\rule{0.400pt}{0.700pt}}
\multiput(1264.17,244.00)(-2.000,1.547){2}{\rule{0.400pt}{0.350pt}}
\multiput(1260.92,247.61)(-0.462,0.447){3}{\rule{0.500pt}{0.108pt}}
\multiput(1261.96,246.17)(-1.962,3.000){2}{\rule{0.250pt}{0.400pt}}
\multiput(1258.95,250.00)(-0.447,0.685){3}{\rule{0.108pt}{0.633pt}}
\multiput(1259.17,250.00)(-3.000,2.685){2}{\rule{0.400pt}{0.317pt}}
\put(1255.17,254){\rule{0.400pt}{0.700pt}}
\multiput(1256.17,254.00)(-2.000,1.547){2}{\rule{0.400pt}{0.350pt}}
\multiput(1252.92,257.61)(-0.462,0.447){3}{\rule{0.500pt}{0.108pt}}
\multiput(1253.96,256.17)(-1.962,3.000){2}{\rule{0.250pt}{0.400pt}}
\put(1250.17,260){\rule{0.400pt}{0.700pt}}
\multiput(1251.17,260.00)(-2.000,1.547){2}{\rule{0.400pt}{0.350pt}}
\multiput(1248.95,263.00)(-0.447,0.685){3}{\rule{0.108pt}{0.633pt}}
\multiput(1249.17,263.00)(-3.000,2.685){2}{\rule{0.400pt}{0.317pt}}
\multiput(1244.92,267.61)(-0.462,0.447){3}{\rule{0.500pt}{0.108pt}}
\multiput(1245.96,266.17)(-1.962,3.000){2}{\rule{0.250pt}{0.400pt}}
\put(1242.17,270){\rule{0.400pt}{0.700pt}}
\multiput(1243.17,270.00)(-2.000,1.547){2}{\rule{0.400pt}{0.350pt}}
\multiput(1239.92,273.61)(-0.462,0.447){3}{\rule{0.500pt}{0.108pt}}
\multiput(1240.96,272.17)(-1.962,3.000){2}{\rule{0.250pt}{0.400pt}}
\put(1237.17,276){\rule{0.400pt}{0.900pt}}
\multiput(1238.17,276.00)(-2.000,2.132){2}{\rule{0.400pt}{0.450pt}}
\multiput(1234.92,280.61)(-0.462,0.447){3}{\rule{0.500pt}{0.108pt}}
\multiput(1235.96,279.17)(-1.962,3.000){2}{\rule{0.250pt}{0.400pt}}
\multiput(1231.92,283.61)(-0.462,0.447){3}{\rule{0.500pt}{0.108pt}}
\multiput(1232.96,282.17)(-1.962,3.000){2}{\rule{0.250pt}{0.400pt}}
\put(1229.17,286){\rule{0.400pt}{0.700pt}}
\multiput(1230.17,286.00)(-2.000,1.547){2}{\rule{0.400pt}{0.350pt}}
\multiput(1226.92,289.61)(-0.462,0.447){3}{\rule{0.500pt}{0.108pt}}
\multiput(1227.96,288.17)(-1.962,3.000){2}{\rule{0.250pt}{0.400pt}}
\multiput(1224.95,292.00)(-0.447,0.685){3}{\rule{0.108pt}{0.633pt}}
\multiput(1225.17,292.00)(-3.000,2.685){2}{\rule{0.400pt}{0.317pt}}
\put(1221.17,296){\rule{0.400pt}{0.700pt}}
\multiput(1222.17,296.00)(-2.000,1.547){2}{\rule{0.400pt}{0.350pt}}
\multiput(1218.92,299.61)(-0.462,0.447){3}{\rule{0.500pt}{0.108pt}}
\multiput(1219.96,298.17)(-1.962,3.000){2}{\rule{0.250pt}{0.400pt}}
\put(1216.17,302){\rule{0.400pt}{0.700pt}}
\multiput(1217.17,302.00)(-2.000,1.547){2}{\rule{0.400pt}{0.350pt}}
\multiput(1214.95,305.00)(-0.447,0.685){3}{\rule{0.108pt}{0.633pt}}
\multiput(1215.17,305.00)(-3.000,2.685){2}{\rule{0.400pt}{0.317pt}}
\multiput(1210.92,309.61)(-0.462,0.447){3}{\rule{0.500pt}{0.108pt}}
\multiput(1211.96,308.17)(-1.962,3.000){2}{\rule{0.250pt}{0.400pt}}
\put(1208.17,312){\rule{0.400pt}{0.700pt}}
\multiput(1209.17,312.00)(-2.000,1.547){2}{\rule{0.400pt}{0.350pt}}
\multiput(1205.92,315.61)(-0.462,0.447){3}{\rule{0.500pt}{0.108pt}}
\multiput(1206.96,314.17)(-1.962,3.000){2}{\rule{0.250pt}{0.400pt}}
\multiput(1203.95,318.00)(-0.447,0.685){3}{\rule{0.108pt}{0.633pt}}
\multiput(1204.17,318.00)(-3.000,2.685){2}{\rule{0.400pt}{0.317pt}}
\put(1200.17,322){\rule{0.400pt}{0.700pt}}
\multiput(1201.17,322.00)(-2.000,1.547){2}{\rule{0.400pt}{0.350pt}}
\multiput(1197.92,325.61)(-0.462,0.447){3}{\rule{0.500pt}{0.108pt}}
\multiput(1198.96,324.17)(-1.962,3.000){2}{\rule{0.250pt}{0.400pt}}
\put(1195.17,328){\rule{0.400pt}{0.700pt}}
\multiput(1196.17,328.00)(-2.000,1.547){2}{\rule{0.400pt}{0.350pt}}
\multiput(1193.95,331.00)(-0.447,0.685){3}{\rule{0.108pt}{0.633pt}}
\multiput(1194.17,331.00)(-3.000,2.685){2}{\rule{0.400pt}{0.317pt}}
\multiput(1189.92,335.61)(-0.462,0.447){3}{\rule{0.500pt}{0.108pt}}
\multiput(1190.96,334.17)(-1.962,3.000){2}{\rule{0.250pt}{0.400pt}}
\put(1187.17,338){\rule{0.400pt}{0.700pt}}
\multiput(1188.17,338.00)(-2.000,1.547){2}{\rule{0.400pt}{0.350pt}}
\multiput(1184.92,341.61)(-0.462,0.447){3}{\rule{0.500pt}{0.108pt}}
\multiput(1185.96,340.17)(-1.962,3.000){2}{\rule{0.250pt}{0.400pt}}
\put(1182.17,344){\rule{0.400pt}{0.900pt}}
\multiput(1183.17,344.00)(-2.000,2.132){2}{\rule{0.400pt}{0.450pt}}
\multiput(1179.92,348.61)(-0.462,0.447){3}{\rule{0.500pt}{0.108pt}}
\multiput(1180.96,347.17)(-1.962,3.000){2}{\rule{0.250pt}{0.400pt}}
\multiput(1176.92,351.61)(-0.462,0.447){3}{\rule{0.500pt}{0.108pt}}
\multiput(1177.96,350.17)(-1.962,3.000){2}{\rule{0.250pt}{0.400pt}}
\put(1174.17,354){\rule{0.400pt}{0.700pt}}
\multiput(1175.17,354.00)(-2.000,1.547){2}{\rule{0.400pt}{0.350pt}}
\multiput(1172.95,357.00)(-0.447,0.685){3}{\rule{0.108pt}{0.633pt}}
\multiput(1173.17,357.00)(-3.000,2.685){2}{\rule{0.400pt}{0.317pt}}
\multiput(1168.92,361.61)(-0.462,0.447){3}{\rule{0.500pt}{0.108pt}}
\multiput(1169.96,360.17)(-1.962,3.000){2}{\rule{0.250pt}{0.400pt}}
\put(1166.17,364){\rule{0.400pt}{0.700pt}}
\multiput(1167.17,364.00)(-2.000,1.547){2}{\rule{0.400pt}{0.350pt}}
\multiput(1163.92,367.61)(-0.462,0.447){3}{\rule{0.500pt}{0.108pt}}
\multiput(1164.96,366.17)(-1.962,3.000){2}{\rule{0.250pt}{0.400pt}}
\put(1161.17,370){\rule{0.400pt}{0.900pt}}
\multiput(1162.17,370.00)(-2.000,2.132){2}{\rule{0.400pt}{0.450pt}}
\multiput(1158.92,374.61)(-0.462,0.447){3}{\rule{0.500pt}{0.108pt}}
\multiput(1159.96,373.17)(-1.962,3.000){2}{\rule{0.250pt}{0.400pt}}
\multiput(1155.92,377.61)(-0.462,0.447){3}{\rule{0.500pt}{0.108pt}}
\multiput(1156.96,376.17)(-1.962,3.000){2}{\rule{0.250pt}{0.400pt}}
\put(1153.17,380){\rule{0.400pt}{0.700pt}}
\multiput(1154.17,380.00)(-2.000,1.547){2}{\rule{0.400pt}{0.350pt}}
\multiput(1151.95,383.00)(-0.447,0.685){3}{\rule{0.108pt}{0.633pt}}
\multiput(1152.17,383.00)(-3.000,2.685){2}{\rule{0.400pt}{0.317pt}}
\multiput(1147.92,387.61)(-0.462,0.447){3}{\rule{0.500pt}{0.108pt}}
\multiput(1148.96,386.17)(-1.962,3.000){2}{\rule{0.250pt}{0.400pt}}
\put(1145.17,390){\rule{0.400pt}{0.700pt}}
\multiput(1146.17,390.00)(-2.000,1.547){2}{\rule{0.400pt}{0.350pt}}
\multiput(1142.92,393.61)(-0.462,0.447){3}{\rule{0.500pt}{0.108pt}}
\multiput(1143.96,392.17)(-1.962,3.000){2}{\rule{0.250pt}{0.400pt}}
\put(1140.17,396){\rule{0.400pt}{0.900pt}}
\multiput(1141.17,396.00)(-2.000,2.132){2}{\rule{0.400pt}{0.450pt}}
\multiput(1137.92,400.61)(-0.462,0.447){3}{\rule{0.500pt}{0.108pt}}
\multiput(1138.96,399.17)(-1.962,3.000){2}{\rule{0.250pt}{0.400pt}}
\multiput(1134.92,403.61)(-0.462,0.447){3}{\rule{0.500pt}{0.108pt}}
\multiput(1135.96,402.17)(-1.962,3.000){2}{\rule{0.250pt}{0.400pt}}
\put(1132.17,406){\rule{0.400pt}{0.700pt}}
\multiput(1133.17,406.00)(-2.000,1.547){2}{\rule{0.400pt}{0.350pt}}
\multiput(1130.95,409.00)(-0.447,0.685){3}{\rule{0.108pt}{0.633pt}}
\multiput(1131.17,409.00)(-3.000,2.685){2}{\rule{0.400pt}{0.317pt}}
\put(1127.17,413){\rule{0.400pt}{0.700pt}}
\multiput(1128.17,413.00)(-2.000,1.547){2}{\rule{0.400pt}{0.350pt}}
\multiput(1124.92,416.61)(-0.462,0.447){3}{\rule{0.500pt}{0.108pt}}
\multiput(1125.96,415.17)(-1.962,3.000){2}{\rule{0.250pt}{0.400pt}}
\multiput(1121.92,419.61)(-0.462,0.447){3}{\rule{0.500pt}{0.108pt}}
\multiput(1122.96,418.17)(-1.962,3.000){2}{\rule{0.250pt}{0.400pt}}
\put(1119.17,422){\rule{0.400pt}{0.900pt}}
\multiput(1120.17,422.00)(-2.000,2.132){2}{\rule{0.400pt}{0.450pt}}
\multiput(1116.92,426.61)(-0.462,0.447){3}{\rule{0.500pt}{0.108pt}}
\multiput(1117.96,425.17)(-1.962,3.000){2}{\rule{0.250pt}{0.400pt}}
\multiput(1113.92,429.61)(-0.462,0.447){3}{\rule{0.500pt}{0.108pt}}
\multiput(1114.96,428.17)(-1.962,3.000){2}{\rule{0.250pt}{0.400pt}}
\put(1111.17,432){\rule{0.400pt}{0.700pt}}
\multiput(1112.17,432.00)(-2.000,1.547){2}{\rule{0.400pt}{0.350pt}}
\multiput(1108.92,435.61)(-0.462,0.447){3}{\rule{0.500pt}{0.108pt}}
\multiput(1109.96,434.17)(-1.962,3.000){2}{\rule{0.250pt}{0.400pt}}
\put(1106.17,438){\rule{0.400pt}{0.900pt}}
\multiput(1107.17,438.00)(-2.000,2.132){2}{\rule{0.400pt}{0.450pt}}
\multiput(1103.92,442.61)(-0.462,0.447){3}{\rule{0.500pt}{0.108pt}}
\multiput(1104.96,441.17)(-1.962,3.000){2}{\rule{0.250pt}{0.400pt}}
\multiput(1100.92,445.61)(-0.462,0.447){3}{\rule{0.500pt}{0.108pt}}
\multiput(1101.96,444.17)(-1.962,3.000){2}{\rule{0.250pt}{0.400pt}}
\put(1098.17,448){\rule{0.400pt}{0.700pt}}
\multiput(1099.17,448.00)(-2.000,1.547){2}{\rule{0.400pt}{0.350pt}}
\multiput(1096.95,451.00)(-0.447,0.685){3}{\rule{0.108pt}{0.633pt}}
\multiput(1097.17,451.00)(-3.000,2.685){2}{\rule{0.400pt}{0.317pt}}
\put(1093.17,455){\rule{0.400pt}{0.700pt}}
\multiput(1094.17,455.00)(-2.000,1.547){2}{\rule{0.400pt}{0.350pt}}
\multiput(1090.92,458.61)(-0.462,0.447){3}{\rule{0.500pt}{0.108pt}}
\multiput(1091.96,457.17)(-1.962,3.000){2}{\rule{0.250pt}{0.400pt}}
\multiput(1087.92,461.61)(-0.462,0.447){3}{\rule{0.500pt}{0.108pt}}
\multiput(1088.96,460.17)(-1.962,3.000){2}{\rule{0.250pt}{0.400pt}}
\put(1085.17,464){\rule{0.400pt}{0.900pt}}
\multiput(1086.17,464.00)(-2.000,2.132){2}{\rule{0.400pt}{0.450pt}}
\multiput(1082.92,468.61)(-0.462,0.447){3}{\rule{0.500pt}{0.108pt}}
\multiput(1083.96,467.17)(-1.962,3.000){2}{\rule{0.250pt}{0.400pt}}
\multiput(1079.92,471.61)(-0.462,0.447){3}{\rule{0.500pt}{0.108pt}}
\multiput(1080.96,470.17)(-1.962,3.000){2}{\rule{0.250pt}{0.400pt}}
\put(1077.17,474){\rule{0.400pt}{0.700pt}}
\multiput(1078.17,474.00)(-2.000,1.547){2}{\rule{0.400pt}{0.350pt}}
\multiput(1075.95,477.00)(-0.447,0.685){3}{\rule{0.108pt}{0.633pt}}
\multiput(1076.17,477.00)(-3.000,2.685){2}{\rule{0.400pt}{0.317pt}}
\put(1072.17,481){\rule{0.400pt}{0.700pt}}
\multiput(1073.17,481.00)(-2.000,1.547){2}{\rule{0.400pt}{0.350pt}}
\multiput(1069.92,484.61)(-0.462,0.447){3}{\rule{0.500pt}{0.108pt}}
\multiput(1070.96,483.17)(-1.962,3.000){2}{\rule{0.250pt}{0.400pt}}
\multiput(1066.92,487.61)(-0.462,0.447){3}{\rule{0.500pt}{0.108pt}}
\multiput(1067.96,486.17)(-1.962,3.000){2}{\rule{0.250pt}{0.400pt}}
\put(1064.17,490){\rule{0.400pt}{0.900pt}}
\multiput(1065.17,490.00)(-2.000,2.132){2}{\rule{0.400pt}{0.450pt}}
\multiput(1061.92,494.61)(-0.462,0.447){3}{\rule{0.500pt}{0.108pt}}
\multiput(1062.96,493.17)(-1.962,3.000){2}{\rule{0.250pt}{0.400pt}}
\multiput(1058.92,497.61)(-0.462,0.447){3}{\rule{0.500pt}{0.108pt}}
\multiput(1059.96,496.17)(-1.962,3.000){2}{\rule{0.250pt}{0.400pt}}
\put(1056.17,500){\rule{0.400pt}{0.700pt}}
\multiput(1057.17,500.00)(-2.000,1.547){2}{\rule{0.400pt}{0.350pt}}
\multiput(1054.95,503.00)(-0.447,0.685){3}{\rule{0.108pt}{0.633pt}}
\multiput(1055.17,503.00)(-3.000,2.685){2}{\rule{0.400pt}{0.317pt}}
\put(1051.17,507){\rule{0.400pt}{0.700pt}}
\multiput(1052.17,507.00)(-2.000,1.547){2}{\rule{0.400pt}{0.350pt}}
\multiput(1048.92,510.61)(-0.462,0.447){3}{\rule{0.500pt}{0.108pt}}
\multiput(1049.96,509.17)(-1.962,3.000){2}{\rule{0.250pt}{0.400pt}}
\multiput(1045.92,513.61)(-0.462,0.447){3}{\rule{0.500pt}{0.108pt}}
\multiput(1046.96,512.17)(-1.962,3.000){2}{\rule{0.250pt}{0.400pt}}
\put(1043.17,516){\rule{0.400pt}{0.900pt}}
\multiput(1044.17,516.00)(-2.000,2.132){2}{\rule{0.400pt}{0.450pt}}
\multiput(1040.92,520.61)(-0.462,0.447){3}{\rule{0.500pt}{0.108pt}}
\multiput(1041.96,519.17)(-1.962,3.000){2}{\rule{0.250pt}{0.400pt}}
\put(1038.17,523){\rule{0.400pt}{0.700pt}}
\multiput(1039.17,523.00)(-2.000,1.547){2}{\rule{0.400pt}{0.350pt}}
\multiput(1035.92,526.61)(-0.462,0.447){3}{\rule{0.500pt}{0.108pt}}
\multiput(1036.96,525.17)(-1.962,3.000){2}{\rule{0.250pt}{0.400pt}}
\multiput(1033.95,529.00)(-0.447,0.685){3}{\rule{0.108pt}{0.633pt}}
\multiput(1034.17,529.00)(-3.000,2.685){2}{\rule{0.400pt}{0.317pt}}
\put(1030.17,533){\rule{0.400pt}{0.700pt}}
\multiput(1031.17,533.00)(-2.000,1.547){2}{\rule{0.400pt}{0.350pt}}
\multiput(1027.92,536.61)(-0.462,0.447){3}{\rule{0.500pt}{0.108pt}}
\multiput(1028.96,535.17)(-1.962,3.000){2}{\rule{0.250pt}{0.400pt}}
\multiput(1024.92,539.61)(-0.462,0.447){3}{\rule{0.500pt}{0.108pt}}
\multiput(1025.96,538.17)(-1.962,3.000){2}{\rule{0.250pt}{0.400pt}}
\put(1022.17,542){\rule{0.400pt}{0.900pt}}
\multiput(1023.17,542.00)(-2.000,2.132){2}{\rule{0.400pt}{0.450pt}}
\multiput(1019.92,546.61)(-0.462,0.447){3}{\rule{0.500pt}{0.108pt}}
\multiput(1020.96,545.17)(-1.962,3.000){2}{\rule{0.250pt}{0.400pt}}
\put(1017.17,549){\rule{0.400pt}{0.700pt}}
\multiput(1018.17,549.00)(-2.000,1.547){2}{\rule{0.400pt}{0.350pt}}
\multiput(1014.92,552.61)(-0.462,0.447){3}{\rule{0.500pt}{0.108pt}}
\multiput(1015.96,551.17)(-1.962,3.000){2}{\rule{0.250pt}{0.400pt}}
\multiput(1012.95,555.00)(-0.447,0.685){3}{\rule{0.108pt}{0.633pt}}
\multiput(1013.17,555.00)(-3.000,2.685){2}{\rule{0.400pt}{0.317pt}}
\put(1009.17,559){\rule{0.400pt}{0.700pt}}
\multiput(1010.17,559.00)(-2.000,1.547){2}{\rule{0.400pt}{0.350pt}}
\multiput(1006.92,562.61)(-0.462,0.447){3}{\rule{0.500pt}{0.108pt}}
\multiput(1007.96,561.17)(-1.962,3.000){2}{\rule{0.250pt}{0.400pt}}
\put(130.0,82.0){\rule[-0.200pt]{0.400pt}{187.179pt}}
\put(130.0,82.0){\rule[-0.200pt]{315.338pt}{0.400pt}}
\put(1439.0,82.0){\rule[-0.200pt]{0.400pt}{187.179pt}}
\put(130.0,859.0){\rule[-0.200pt]{315.338pt}{0.400pt}}
\end{picture}

Plot for Ball 4:\\
% GNUPLOT: LaTeX picture
\setlength{\unitlength}{0.240900pt}
\ifx\plotpoint\undefined\newsavebox{\plotpoint}\fi
\sbox{\plotpoint}{\rule[-0.200pt]{0.400pt}{0.400pt}}%
\begin{picture}(1500,900)(0,0)
\sbox{\plotpoint}{\rule[-0.200pt]{0.400pt}{0.400pt}}%
\put(130.0,90.0){\rule[-0.200pt]{4.818pt}{0.400pt}}
\put(110,90){\makebox(0,0)[r]{ 0}}
\put(1419.0,90.0){\rule[-0.200pt]{4.818pt}{0.400pt}}
\put(130.0,242.0){\rule[-0.200pt]{4.818pt}{0.400pt}}
\put(110,242){\makebox(0,0)[r]{ 0.2}}
\put(1419.0,242.0){\rule[-0.200pt]{4.818pt}{0.400pt}}
\put(130.0,394.0){\rule[-0.200pt]{4.818pt}{0.400pt}}
\put(110,394){\makebox(0,0)[r]{ 0.4}}
\put(1419.0,394.0){\rule[-0.200pt]{4.818pt}{0.400pt}}
\put(130.0,547.0){\rule[-0.200pt]{4.818pt}{0.400pt}}
\put(110,547){\makebox(0,0)[r]{ 0.6}}
\put(1419.0,547.0){\rule[-0.200pt]{4.818pt}{0.400pt}}
\put(130.0,699.0){\rule[-0.200pt]{4.818pt}{0.400pt}}
\put(110,699){\makebox(0,0)[r]{ 0.8}}
\put(1419.0,699.0){\rule[-0.200pt]{4.818pt}{0.400pt}}
\put(130.0,851.0){\rule[-0.200pt]{4.818pt}{0.400pt}}
\put(110,851){\makebox(0,0)[r]{ 1}}
\put(1419.0,851.0){\rule[-0.200pt]{4.818pt}{0.400pt}}
\put(130.0,82.0){\rule[-0.200pt]{0.400pt}{4.818pt}}
\put(130,41){\makebox(0,0){ 0}}
\put(130.0,839.0){\rule[-0.200pt]{0.400pt}{4.818pt}}
\put(392.0,82.0){\rule[-0.200pt]{0.400pt}{4.818pt}}
\put(392,41){\makebox(0,0){ 0.2}}
\put(392.0,839.0){\rule[-0.200pt]{0.400pt}{4.818pt}}
\put(654.0,82.0){\rule[-0.200pt]{0.400pt}{4.818pt}}
\put(654,41){\makebox(0,0){ 0.4}}
\put(654.0,839.0){\rule[-0.200pt]{0.400pt}{4.818pt}}
\put(915.0,82.0){\rule[-0.200pt]{0.400pt}{4.818pt}}
\put(915,41){\makebox(0,0){ 0.6}}
\put(915.0,839.0){\rule[-0.200pt]{0.400pt}{4.818pt}}
\put(1177.0,82.0){\rule[-0.200pt]{0.400pt}{4.818pt}}
\put(1177,41){\makebox(0,0){ 0.8}}
\put(1177.0,839.0){\rule[-0.200pt]{0.400pt}{4.818pt}}
\put(1439.0,82.0){\rule[-0.200pt]{0.400pt}{4.818pt}}
\put(1439,41){\makebox(0,0){ 1}}
\put(1439.0,839.0){\rule[-0.200pt]{0.400pt}{4.818pt}}
\put(130.0,82.0){\rule[-0.200pt]{0.400pt}{187.179pt}}
\put(130.0,82.0){\rule[-0.200pt]{315.338pt}{0.400pt}}
\put(1439.0,82.0){\rule[-0.200pt]{0.400pt}{187.179pt}}
\put(130.0,859.0){\rule[-0.200pt]{315.338pt}{0.400pt}}
\put(1279,819){\makebox(0,0)[r]{'-'}}
\put(1299.0,819.0){\rule[-0.200pt]{24.090pt}{0.400pt}}
\put(957,718){\usebox{\plotpoint}}
\multiput(957.00,716.94)(0.774,-0.468){5}{\rule{0.700pt}{0.113pt}}
\multiput(957.00,717.17)(4.547,-4.000){2}{\rule{0.350pt}{0.400pt}}
\multiput(963.00,712.94)(0.774,-0.468){5}{\rule{0.700pt}{0.113pt}}
\multiput(963.00,713.17)(4.547,-4.000){2}{\rule{0.350pt}{0.400pt}}
\multiput(969.00,708.94)(0.774,-0.468){5}{\rule{0.700pt}{0.113pt}}
\multiput(969.00,709.17)(4.547,-4.000){2}{\rule{0.350pt}{0.400pt}}
\multiput(975.00,704.95)(1.132,-0.447){3}{\rule{0.900pt}{0.108pt}}
\multiput(975.00,705.17)(4.132,-3.000){2}{\rule{0.450pt}{0.400pt}}
\multiput(981.00,701.94)(0.774,-0.468){5}{\rule{0.700pt}{0.113pt}}
\multiput(981.00,702.17)(4.547,-4.000){2}{\rule{0.350pt}{0.400pt}}
\multiput(987.00,697.94)(0.774,-0.468){5}{\rule{0.700pt}{0.113pt}}
\multiput(987.00,698.17)(4.547,-4.000){2}{\rule{0.350pt}{0.400pt}}
\multiput(993.00,693.95)(1.132,-0.447){3}{\rule{0.900pt}{0.108pt}}
\multiput(993.00,694.17)(4.132,-3.000){2}{\rule{0.450pt}{0.400pt}}
\multiput(999.00,690.94)(0.774,-0.468){5}{\rule{0.700pt}{0.113pt}}
\multiput(999.00,691.17)(4.547,-4.000){2}{\rule{0.350pt}{0.400pt}}
\multiput(1005.00,686.94)(0.774,-0.468){5}{\rule{0.700pt}{0.113pt}}
\multiput(1005.00,687.17)(4.547,-4.000){2}{\rule{0.350pt}{0.400pt}}
\multiput(1011.00,682.94)(0.774,-0.468){5}{\rule{0.700pt}{0.113pt}}
\multiput(1011.00,683.17)(4.547,-4.000){2}{\rule{0.350pt}{0.400pt}}
\multiput(1017.00,678.95)(1.132,-0.447){3}{\rule{0.900pt}{0.108pt}}
\multiput(1017.00,679.17)(4.132,-3.000){2}{\rule{0.450pt}{0.400pt}}
\multiput(1023.00,675.94)(0.774,-0.468){5}{\rule{0.700pt}{0.113pt}}
\multiput(1023.00,676.17)(4.547,-4.000){2}{\rule{0.350pt}{0.400pt}}
\multiput(1029.00,671.94)(0.774,-0.468){5}{\rule{0.700pt}{0.113pt}}
\multiput(1029.00,672.17)(4.547,-4.000){2}{\rule{0.350pt}{0.400pt}}
\multiput(1035.00,667.95)(1.355,-0.447){3}{\rule{1.033pt}{0.108pt}}
\multiput(1035.00,668.17)(4.855,-3.000){2}{\rule{0.517pt}{0.400pt}}
\multiput(1042.00,664.94)(0.774,-0.468){5}{\rule{0.700pt}{0.113pt}}
\multiput(1042.00,665.17)(4.547,-4.000){2}{\rule{0.350pt}{0.400pt}}
\multiput(1048.00,660.94)(0.774,-0.468){5}{\rule{0.700pt}{0.113pt}}
\multiput(1048.00,661.17)(4.547,-4.000){2}{\rule{0.350pt}{0.400pt}}
\multiput(1054.00,656.94)(0.774,-0.468){5}{\rule{0.700pt}{0.113pt}}
\multiput(1054.00,657.17)(4.547,-4.000){2}{\rule{0.350pt}{0.400pt}}
\multiput(1060.00,652.95)(1.132,-0.447){3}{\rule{0.900pt}{0.108pt}}
\multiput(1060.00,653.17)(4.132,-3.000){2}{\rule{0.450pt}{0.400pt}}
\multiput(1066.00,649.94)(0.774,-0.468){5}{\rule{0.700pt}{0.113pt}}
\multiput(1066.00,650.17)(4.547,-4.000){2}{\rule{0.350pt}{0.400pt}}
\multiput(1072.00,645.94)(0.774,-0.468){5}{\rule{0.700pt}{0.113pt}}
\multiput(1072.00,646.17)(4.547,-4.000){2}{\rule{0.350pt}{0.400pt}}
\multiput(1078.00,641.95)(1.132,-0.447){3}{\rule{0.900pt}{0.108pt}}
\multiput(1078.00,642.17)(4.132,-3.000){2}{\rule{0.450pt}{0.400pt}}
\multiput(1084.00,638.94)(0.774,-0.468){5}{\rule{0.700pt}{0.113pt}}
\multiput(1084.00,639.17)(4.547,-4.000){2}{\rule{0.350pt}{0.400pt}}
\multiput(1090.00,634.94)(0.774,-0.468){5}{\rule{0.700pt}{0.113pt}}
\multiput(1090.00,635.17)(4.547,-4.000){2}{\rule{0.350pt}{0.400pt}}
\multiput(1096.00,630.94)(0.774,-0.468){5}{\rule{0.700pt}{0.113pt}}
\multiput(1096.00,631.17)(4.547,-4.000){2}{\rule{0.350pt}{0.400pt}}
\multiput(1102.00,626.95)(1.132,-0.447){3}{\rule{0.900pt}{0.108pt}}
\multiput(1102.00,627.17)(4.132,-3.000){2}{\rule{0.450pt}{0.400pt}}
\multiput(1108.00,623.94)(0.774,-0.468){5}{\rule{0.700pt}{0.113pt}}
\multiput(1108.00,624.17)(4.547,-4.000){2}{\rule{0.350pt}{0.400pt}}
\multiput(1114.00,619.94)(0.774,-0.468){5}{\rule{0.700pt}{0.113pt}}
\multiput(1114.00,620.17)(4.547,-4.000){2}{\rule{0.350pt}{0.400pt}}
\multiput(1120.00,615.95)(1.132,-0.447){3}{\rule{0.900pt}{0.108pt}}
\multiput(1120.00,616.17)(4.132,-3.000){2}{\rule{0.450pt}{0.400pt}}
\multiput(1126.00,612.94)(0.774,-0.468){5}{\rule{0.700pt}{0.113pt}}
\multiput(1126.00,613.17)(4.547,-4.000){2}{\rule{0.350pt}{0.400pt}}
\multiput(1132.00,608.94)(0.774,-0.468){5}{\rule{0.700pt}{0.113pt}}
\multiput(1132.00,609.17)(4.547,-4.000){2}{\rule{0.350pt}{0.400pt}}
\multiput(1138.00,604.94)(0.774,-0.468){5}{\rule{0.700pt}{0.113pt}}
\multiput(1138.00,605.17)(4.547,-4.000){2}{\rule{0.350pt}{0.400pt}}
\multiput(1144.00,600.95)(1.132,-0.447){3}{\rule{0.900pt}{0.108pt}}
\multiput(1144.00,601.17)(4.132,-3.000){2}{\rule{0.450pt}{0.400pt}}
\multiput(1150.00,597.94)(0.774,-0.468){5}{\rule{0.700pt}{0.113pt}}
\multiput(1150.00,598.17)(4.547,-4.000){2}{\rule{0.350pt}{0.400pt}}
\multiput(1156.00,593.94)(0.774,-0.468){5}{\rule{0.700pt}{0.113pt}}
\multiput(1156.00,594.17)(4.547,-4.000){2}{\rule{0.350pt}{0.400pt}}
\multiput(1162.00,589.95)(1.132,-0.447){3}{\rule{0.900pt}{0.108pt}}
\multiput(1162.00,590.17)(4.132,-3.000){2}{\rule{0.450pt}{0.400pt}}
\multiput(1168.00,586.94)(0.774,-0.468){5}{\rule{0.700pt}{0.113pt}}
\multiput(1168.00,587.17)(4.547,-4.000){2}{\rule{0.350pt}{0.400pt}}
\multiput(1174.00,582.94)(0.774,-0.468){5}{\rule{0.700pt}{0.113pt}}
\multiput(1174.00,583.17)(4.547,-4.000){2}{\rule{0.350pt}{0.400pt}}
\multiput(1180.00,578.94)(0.774,-0.468){5}{\rule{0.700pt}{0.113pt}}
\multiput(1180.00,579.17)(4.547,-4.000){2}{\rule{0.350pt}{0.400pt}}
\multiput(1186.00,574.95)(1.132,-0.447){3}{\rule{0.900pt}{0.108pt}}
\multiput(1186.00,575.17)(4.132,-3.000){2}{\rule{0.450pt}{0.400pt}}
\multiput(1192.00,571.94)(0.774,-0.468){5}{\rule{0.700pt}{0.113pt}}
\multiput(1192.00,572.17)(4.547,-4.000){2}{\rule{0.350pt}{0.400pt}}
\multiput(1198.00,567.94)(0.774,-0.468){5}{\rule{0.700pt}{0.113pt}}
\multiput(1198.00,568.17)(4.547,-4.000){2}{\rule{0.350pt}{0.400pt}}
\multiput(1204.00,563.95)(1.579,-0.447){3}{\rule{1.167pt}{0.108pt}}
\multiput(1204.00,564.17)(5.579,-3.000){2}{\rule{0.583pt}{0.400pt}}
\multiput(1212.00,560.95)(1.579,-0.447){3}{\rule{1.167pt}{0.108pt}}
\multiput(1212.00,561.17)(5.579,-3.000){2}{\rule{0.583pt}{0.400pt}}
\multiput(1220.00,557.95)(1.579,-0.447){3}{\rule{1.167pt}{0.108pt}}
\multiput(1220.00,558.17)(5.579,-3.000){2}{\rule{0.583pt}{0.400pt}}
\multiput(1228.00,554.95)(1.579,-0.447){3}{\rule{1.167pt}{0.108pt}}
\multiput(1228.00,555.17)(5.579,-3.000){2}{\rule{0.583pt}{0.400pt}}
\multiput(1236.00,551.95)(1.579,-0.447){3}{\rule{1.167pt}{0.108pt}}
\multiput(1236.00,552.17)(5.579,-3.000){2}{\rule{0.583pt}{0.400pt}}
\multiput(1244.00,548.95)(1.579,-0.447){3}{\rule{1.167pt}{0.108pt}}
\multiput(1244.00,549.17)(5.579,-3.000){2}{\rule{0.583pt}{0.400pt}}
\multiput(1252.00,545.95)(1.802,-0.447){3}{\rule{1.300pt}{0.108pt}}
\multiput(1252.00,546.17)(6.302,-3.000){2}{\rule{0.650pt}{0.400pt}}
\multiput(1261.00,542.95)(1.579,-0.447){3}{\rule{1.167pt}{0.108pt}}
\multiput(1261.00,543.17)(5.579,-3.000){2}{\rule{0.583pt}{0.400pt}}
\multiput(1269.00,539.95)(1.579,-0.447){3}{\rule{1.167pt}{0.108pt}}
\multiput(1269.00,540.17)(5.579,-3.000){2}{\rule{0.583pt}{0.400pt}}
\multiput(1277.00,536.95)(1.579,-0.447){3}{\rule{1.167pt}{0.108pt}}
\multiput(1277.00,537.17)(5.579,-3.000){2}{\rule{0.583pt}{0.400pt}}
\multiput(1285.00,533.95)(1.579,-0.447){3}{\rule{1.167pt}{0.108pt}}
\multiput(1285.00,534.17)(5.579,-3.000){2}{\rule{0.583pt}{0.400pt}}
\multiput(1293.00,530.95)(1.579,-0.447){3}{\rule{1.167pt}{0.108pt}}
\multiput(1293.00,531.17)(5.579,-3.000){2}{\rule{0.583pt}{0.400pt}}
\multiput(1301.00,527.94)(1.066,-0.468){5}{\rule{0.900pt}{0.113pt}}
\multiput(1301.00,528.17)(6.132,-4.000){2}{\rule{0.450pt}{0.400pt}}
\multiput(1309.00,523.95)(1.579,-0.447){3}{\rule{1.167pt}{0.108pt}}
\multiput(1309.00,524.17)(5.579,-3.000){2}{\rule{0.583pt}{0.400pt}}
\multiput(1317.00,520.95)(1.579,-0.447){3}{\rule{1.167pt}{0.108pt}}
\multiput(1317.00,521.17)(5.579,-3.000){2}{\rule{0.583pt}{0.400pt}}
\multiput(1325.00,517.95)(1.579,-0.447){3}{\rule{1.167pt}{0.108pt}}
\multiput(1325.00,518.17)(5.579,-3.000){2}{\rule{0.583pt}{0.400pt}}
\multiput(1333.00,514.95)(1.802,-0.447){3}{\rule{1.300pt}{0.108pt}}
\multiput(1333.00,515.17)(6.302,-3.000){2}{\rule{0.650pt}{0.400pt}}
\multiput(1342.00,511.95)(1.579,-0.447){3}{\rule{1.167pt}{0.108pt}}
\multiput(1342.00,512.17)(5.579,-3.000){2}{\rule{0.583pt}{0.400pt}}
\multiput(1350.00,508.95)(1.579,-0.447){3}{\rule{1.167pt}{0.108pt}}
\multiput(1350.00,509.17)(5.579,-3.000){2}{\rule{0.583pt}{0.400pt}}
\multiput(1358.00,505.95)(1.579,-0.447){3}{\rule{1.167pt}{0.108pt}}
\multiput(1358.00,506.17)(5.579,-3.000){2}{\rule{0.583pt}{0.400pt}}
\multiput(1366.00,502.95)(1.579,-0.447){3}{\rule{1.167pt}{0.108pt}}
\multiput(1366.00,503.17)(5.579,-3.000){2}{\rule{0.583pt}{0.400pt}}
\multiput(1374.00,499.95)(1.579,-0.447){3}{\rule{1.167pt}{0.108pt}}
\multiput(1374.00,500.17)(5.579,-3.000){2}{\rule{0.583pt}{0.400pt}}
\multiput(1382.00,496.95)(1.579,-0.447){3}{\rule{1.167pt}{0.108pt}}
\multiput(1382.00,497.17)(5.579,-3.000){2}{\rule{0.583pt}{0.400pt}}
\multiput(1390.00,493.95)(1.579,-0.447){3}{\rule{1.167pt}{0.108pt}}
\multiput(1390.00,494.17)(5.579,-3.000){2}{\rule{0.583pt}{0.400pt}}
\multiput(1398.00,490.95)(1.579,-0.447){3}{\rule{1.167pt}{0.108pt}}
\multiput(1398.00,491.17)(5.579,-3.000){2}{\rule{0.583pt}{0.400pt}}
\multiput(1406.00,487.95)(1.579,-0.447){3}{\rule{1.167pt}{0.108pt}}
\multiput(1406.00,488.17)(5.579,-3.000){2}{\rule{0.583pt}{0.400pt}}
\multiput(1409.16,484.95)(-1.579,-0.447){3}{\rule{1.167pt}{0.108pt}}
\multiput(1411.58,485.17)(-5.579,-3.000){2}{\rule{0.583pt}{0.400pt}}
\multiput(1401.16,481.95)(-1.579,-0.447){3}{\rule{1.167pt}{0.108pt}}
\multiput(1403.58,482.17)(-5.579,-3.000){2}{\rule{0.583pt}{0.400pt}}
\multiput(1393.16,478.95)(-1.579,-0.447){3}{\rule{1.167pt}{0.108pt}}
\multiput(1395.58,479.17)(-5.579,-3.000){2}{\rule{0.583pt}{0.400pt}}
\multiput(1385.16,475.95)(-1.579,-0.447){3}{\rule{1.167pt}{0.108pt}}
\multiput(1387.58,476.17)(-5.579,-3.000){2}{\rule{0.583pt}{0.400pt}}
\multiput(1378.26,472.94)(-1.066,-0.468){5}{\rule{0.900pt}{0.113pt}}
\multiput(1380.13,473.17)(-6.132,-4.000){2}{\rule{0.450pt}{0.400pt}}
\multiput(1369.16,468.95)(-1.579,-0.447){3}{\rule{1.167pt}{0.108pt}}
\multiput(1371.58,469.17)(-5.579,-3.000){2}{\rule{0.583pt}{0.400pt}}
\multiput(1361.16,465.95)(-1.579,-0.447){3}{\rule{1.167pt}{0.108pt}}
\multiput(1363.58,466.17)(-5.579,-3.000){2}{\rule{0.583pt}{0.400pt}}
\multiput(1353.16,462.95)(-1.579,-0.447){3}{\rule{1.167pt}{0.108pt}}
\multiput(1355.58,463.17)(-5.579,-3.000){2}{\rule{0.583pt}{0.400pt}}
\multiput(1345.16,459.95)(-1.579,-0.447){3}{\rule{1.167pt}{0.108pt}}
\multiput(1347.58,460.17)(-5.579,-3.000){2}{\rule{0.583pt}{0.400pt}}
\multiput(1336.60,456.95)(-1.802,-0.447){3}{\rule{1.300pt}{0.108pt}}
\multiput(1339.30,457.17)(-6.302,-3.000){2}{\rule{0.650pt}{0.400pt}}
\multiput(1328.16,453.95)(-1.579,-0.447){3}{\rule{1.167pt}{0.108pt}}
\multiput(1330.58,454.17)(-5.579,-3.000){2}{\rule{0.583pt}{0.400pt}}
\multiput(1320.16,450.95)(-1.579,-0.447){3}{\rule{1.167pt}{0.108pt}}
\multiput(1322.58,451.17)(-5.579,-3.000){2}{\rule{0.583pt}{0.400pt}}
\multiput(1312.16,447.95)(-1.579,-0.447){3}{\rule{1.167pt}{0.108pt}}
\multiput(1314.58,448.17)(-5.579,-3.000){2}{\rule{0.583pt}{0.400pt}}
\multiput(1304.16,444.95)(-1.579,-0.447){3}{\rule{1.167pt}{0.108pt}}
\multiput(1306.58,445.17)(-5.579,-3.000){2}{\rule{0.583pt}{0.400pt}}
\multiput(1296.16,441.95)(-1.579,-0.447){3}{\rule{1.167pt}{0.108pt}}
\multiput(1298.58,442.17)(-5.579,-3.000){2}{\rule{0.583pt}{0.400pt}}
\multiput(1288.16,438.95)(-1.579,-0.447){3}{\rule{1.167pt}{0.108pt}}
\multiput(1290.58,439.17)(-5.579,-3.000){2}{\rule{0.583pt}{0.400pt}}
\multiput(1280.16,435.95)(-1.579,-0.447){3}{\rule{1.167pt}{0.108pt}}
\multiput(1282.58,436.17)(-5.579,-3.000){2}{\rule{0.583pt}{0.400pt}}
\multiput(1272.16,432.95)(-1.579,-0.447){3}{\rule{1.167pt}{0.108pt}}
\multiput(1274.58,433.17)(-5.579,-3.000){2}{\rule{0.583pt}{0.400pt}}
\multiput(1264.16,429.95)(-1.579,-0.447){3}{\rule{1.167pt}{0.108pt}}
\multiput(1266.58,430.17)(-5.579,-3.000){2}{\rule{0.583pt}{0.400pt}}
\multiput(1255.60,426.95)(-1.802,-0.447){3}{\rule{1.300pt}{0.108pt}}
\multiput(1258.30,427.17)(-6.302,-3.000){2}{\rule{0.650pt}{0.400pt}}
\multiput(1247.16,423.95)(-1.579,-0.447){3}{\rule{1.167pt}{0.108pt}}
\multiput(1249.58,424.17)(-5.579,-3.000){2}{\rule{0.583pt}{0.400pt}}
\multiput(1240.26,420.94)(-1.066,-0.468){5}{\rule{0.900pt}{0.113pt}}
\multiput(1242.13,421.17)(-6.132,-4.000){2}{\rule{0.450pt}{0.400pt}}
\multiput(1231.16,416.95)(-1.579,-0.447){3}{\rule{1.167pt}{0.108pt}}
\multiput(1233.58,417.17)(-5.579,-3.000){2}{\rule{0.583pt}{0.400pt}}
\multiput(1223.16,413.95)(-1.579,-0.447){3}{\rule{1.167pt}{0.108pt}}
\multiput(1225.58,414.17)(-5.579,-3.000){2}{\rule{0.583pt}{0.400pt}}
\multiput(1215.16,410.95)(-1.579,-0.447){3}{\rule{1.167pt}{0.108pt}}
\multiput(1217.58,411.17)(-5.579,-3.000){2}{\rule{0.583pt}{0.400pt}}
\multiput(1207.16,407.95)(-1.579,-0.447){3}{\rule{1.167pt}{0.108pt}}
\multiput(1209.58,408.17)(-5.579,-3.000){2}{\rule{0.583pt}{0.400pt}}
\multiput(1199.16,404.95)(-1.579,-0.447){3}{\rule{1.167pt}{0.108pt}}
\multiput(1201.58,405.17)(-5.579,-3.000){2}{\rule{0.583pt}{0.400pt}}
\multiput(1191.16,401.95)(-1.579,-0.447){3}{\rule{1.167pt}{0.108pt}}
\multiput(1193.58,402.17)(-5.579,-3.000){2}{\rule{0.583pt}{0.400pt}}
\multiput(1183.16,398.95)(-1.579,-0.447){3}{\rule{1.167pt}{0.108pt}}
\multiput(1185.58,399.17)(-5.579,-3.000){2}{\rule{0.583pt}{0.400pt}}
\multiput(1174.60,395.95)(-1.802,-0.447){3}{\rule{1.300pt}{0.108pt}}
\multiput(1177.30,396.17)(-6.302,-3.000){2}{\rule{0.650pt}{0.400pt}}
\multiput(1166.16,392.95)(-1.579,-0.447){3}{\rule{1.167pt}{0.108pt}}
\multiput(1168.58,393.17)(-5.579,-3.000){2}{\rule{0.583pt}{0.400pt}}
\multiput(1158.16,389.95)(-1.579,-0.447){3}{\rule{1.167pt}{0.108pt}}
\multiput(1160.58,390.17)(-5.579,-3.000){2}{\rule{0.583pt}{0.400pt}}
\multiput(1150.16,386.95)(-1.579,-0.447){3}{\rule{1.167pt}{0.108pt}}
\multiput(1152.58,387.17)(-5.579,-3.000){2}{\rule{0.583pt}{0.400pt}}
\multiput(1142.16,383.95)(-1.579,-0.447){3}{\rule{1.167pt}{0.108pt}}
\multiput(1144.58,384.17)(-5.579,-3.000){2}{\rule{0.583pt}{0.400pt}}
\multiput(1134.16,380.95)(-1.579,-0.447){3}{\rule{1.167pt}{0.108pt}}
\multiput(1136.58,381.17)(-5.579,-3.000){2}{\rule{0.583pt}{0.400pt}}
\multiput(1126.16,377.95)(-1.579,-0.447){3}{\rule{1.167pt}{0.108pt}}
\multiput(1128.58,378.17)(-5.579,-3.000){2}{\rule{0.583pt}{0.400pt}}
\multiput(1118.16,374.95)(-1.579,-0.447){3}{\rule{1.167pt}{0.108pt}}
\multiput(1120.58,375.17)(-5.579,-3.000){2}{\rule{0.583pt}{0.400pt}}
\multiput(1110.16,371.95)(-1.579,-0.447){3}{\rule{1.167pt}{0.108pt}}
\multiput(1112.58,372.17)(-5.579,-3.000){2}{\rule{0.583pt}{0.400pt}}
\multiput(1101.60,368.95)(-1.802,-0.447){3}{\rule{1.300pt}{0.108pt}}
\multiput(1104.30,369.17)(-6.302,-3.000){2}{\rule{0.650pt}{0.400pt}}
\multiput(1094.26,365.94)(-1.066,-0.468){5}{\rule{0.900pt}{0.113pt}}
\multiput(1096.13,366.17)(-6.132,-4.000){2}{\rule{0.450pt}{0.400pt}}
\multiput(1085.16,361.95)(-1.579,-0.447){3}{\rule{1.167pt}{0.108pt}}
\multiput(1087.58,362.17)(-5.579,-3.000){2}{\rule{0.583pt}{0.400pt}}
\multiput(1077.16,358.95)(-1.579,-0.447){3}{\rule{1.167pt}{0.108pt}}
\multiput(1079.58,359.17)(-5.579,-3.000){2}{\rule{0.583pt}{0.400pt}}
\multiput(1069.16,355.95)(-1.579,-0.447){3}{\rule{1.167pt}{0.108pt}}
\multiput(1071.58,356.17)(-5.579,-3.000){2}{\rule{0.583pt}{0.400pt}}
\multiput(1061.16,352.95)(-1.579,-0.447){3}{\rule{1.167pt}{0.108pt}}
\multiput(1063.58,353.17)(-5.579,-3.000){2}{\rule{0.583pt}{0.400pt}}
\multiput(1053.16,349.95)(-1.579,-0.447){3}{\rule{1.167pt}{0.108pt}}
\multiput(1055.58,350.17)(-5.579,-3.000){2}{\rule{0.583pt}{0.400pt}}
\multiput(1045.16,346.95)(-1.579,-0.447){3}{\rule{1.167pt}{0.108pt}}
\multiput(1047.58,347.17)(-5.579,-3.000){2}{\rule{0.583pt}{0.400pt}}
\multiput(1037.16,343.95)(-1.579,-0.447){3}{\rule{1.167pt}{0.108pt}}
\multiput(1039.58,344.17)(-5.579,-3.000){2}{\rule{0.583pt}{0.400pt}}
\multiput(1029.16,340.95)(-1.579,-0.447){3}{\rule{1.167pt}{0.108pt}}
\multiput(1031.58,341.17)(-5.579,-3.000){2}{\rule{0.583pt}{0.400pt}}
\multiput(1020.60,337.95)(-1.802,-0.447){3}{\rule{1.300pt}{0.108pt}}
\multiput(1023.30,338.17)(-6.302,-3.000){2}{\rule{0.650pt}{0.400pt}}
\multiput(1012.16,334.95)(-1.579,-0.447){3}{\rule{1.167pt}{0.108pt}}
\multiput(1014.58,335.17)(-5.579,-3.000){2}{\rule{0.583pt}{0.400pt}}
\multiput(1004.16,331.95)(-1.579,-0.447){3}{\rule{1.167pt}{0.108pt}}
\multiput(1006.58,332.17)(-5.579,-3.000){2}{\rule{0.583pt}{0.400pt}}
\multiput(996.16,328.95)(-1.579,-0.447){3}{\rule{1.167pt}{0.108pt}}
\multiput(998.58,329.17)(-5.579,-3.000){2}{\rule{0.583pt}{0.400pt}}
\multiput(988.16,325.95)(-1.579,-0.447){3}{\rule{1.167pt}{0.108pt}}
\multiput(990.58,326.17)(-5.579,-3.000){2}{\rule{0.583pt}{0.400pt}}
\multiput(980.16,322.95)(-1.579,-0.447){3}{\rule{1.167pt}{0.108pt}}
\multiput(982.58,323.17)(-5.579,-3.000){2}{\rule{0.583pt}{0.400pt}}
\multiput(972.16,319.95)(-1.579,-0.447){3}{\rule{1.167pt}{0.108pt}}
\multiput(974.58,320.17)(-5.579,-3.000){2}{\rule{0.583pt}{0.400pt}}
\multiput(964.16,316.95)(-1.579,-0.447){3}{\rule{1.167pt}{0.108pt}}
\multiput(966.58,317.17)(-5.579,-3.000){2}{\rule{0.583pt}{0.400pt}}
\multiput(957.26,313.94)(-1.066,-0.468){5}{\rule{0.900pt}{0.113pt}}
\multiput(959.13,314.17)(-6.132,-4.000){2}{\rule{0.450pt}{0.400pt}}
\multiput(948.16,309.95)(-1.579,-0.447){3}{\rule{1.167pt}{0.108pt}}
\multiput(950.58,310.17)(-5.579,-3.000){2}{\rule{0.583pt}{0.400pt}}
\multiput(939.60,306.95)(-1.802,-0.447){3}{\rule{1.300pt}{0.108pt}}
\multiput(942.30,307.17)(-6.302,-3.000){2}{\rule{0.650pt}{0.400pt}}
\multiput(931.16,303.95)(-1.579,-0.447){3}{\rule{1.167pt}{0.108pt}}
\multiput(933.58,304.17)(-5.579,-3.000){2}{\rule{0.583pt}{0.400pt}}
\multiput(923.16,300.95)(-1.579,-0.447){3}{\rule{1.167pt}{0.108pt}}
\multiput(925.58,301.17)(-5.579,-3.000){2}{\rule{0.583pt}{0.400pt}}
\multiput(915.16,297.95)(-1.579,-0.447){3}{\rule{1.167pt}{0.108pt}}
\multiput(917.58,298.17)(-5.579,-3.000){2}{\rule{0.583pt}{0.400pt}}
\multiput(907.16,294.95)(-1.579,-0.447){3}{\rule{1.167pt}{0.108pt}}
\multiput(909.58,295.17)(-5.579,-3.000){2}{\rule{0.583pt}{0.400pt}}
\multiput(899.16,291.95)(-1.579,-0.447){3}{\rule{1.167pt}{0.108pt}}
\multiput(901.58,292.17)(-5.579,-3.000){2}{\rule{0.583pt}{0.400pt}}
\multiput(891.16,288.95)(-1.579,-0.447){3}{\rule{1.167pt}{0.108pt}}
\multiput(893.58,289.17)(-5.579,-3.000){2}{\rule{0.583pt}{0.400pt}}
\multiput(883.16,285.95)(-1.579,-0.447){3}{\rule{1.167pt}{0.108pt}}
\multiput(885.58,286.17)(-5.579,-3.000){2}{\rule{0.583pt}{0.400pt}}
\multiput(875.16,282.95)(-1.579,-0.447){3}{\rule{1.167pt}{0.108pt}}
\multiput(877.58,283.17)(-5.579,-3.000){2}{\rule{0.583pt}{0.400pt}}
\multiput(867.16,279.95)(-1.579,-0.447){3}{\rule{1.167pt}{0.108pt}}
\multiput(869.58,280.17)(-5.579,-3.000){2}{\rule{0.583pt}{0.400pt}}
\multiput(858.60,276.95)(-1.802,-0.447){3}{\rule{1.300pt}{0.108pt}}
\multiput(861.30,277.17)(-6.302,-3.000){2}{\rule{0.650pt}{0.400pt}}
\multiput(850.16,273.95)(-1.579,-0.447){3}{\rule{1.167pt}{0.108pt}}
\multiput(852.58,274.17)(-5.579,-3.000){2}{\rule{0.583pt}{0.400pt}}
\multiput(842.16,270.95)(-1.579,-0.447){3}{\rule{1.167pt}{0.108pt}}
\multiput(844.58,271.17)(-5.579,-3.000){2}{\rule{0.583pt}{0.400pt}}
\multiput(834.16,267.95)(-1.579,-0.447){3}{\rule{1.167pt}{0.108pt}}
\multiput(836.58,268.17)(-5.579,-3.000){2}{\rule{0.583pt}{0.400pt}}
\multiput(826.16,264.95)(-1.579,-0.447){3}{\rule{1.167pt}{0.108pt}}
\multiput(828.58,265.17)(-5.579,-3.000){2}{\rule{0.583pt}{0.400pt}}
\multiput(818.16,261.95)(-1.579,-0.447){3}{\rule{1.167pt}{0.108pt}}
\multiput(820.58,262.17)(-5.579,-3.000){2}{\rule{0.583pt}{0.400pt}}
\multiput(811.26,258.94)(-1.066,-0.468){5}{\rule{0.900pt}{0.113pt}}
\multiput(813.13,259.17)(-6.132,-4.000){2}{\rule{0.450pt}{0.400pt}}
\multiput(802.16,254.95)(-1.579,-0.447){3}{\rule{1.167pt}{0.108pt}}
\multiput(804.58,255.17)(-5.579,-3.000){2}{\rule{0.583pt}{0.400pt}}
\multiput(794.16,251.95)(-1.579,-0.447){3}{\rule{1.167pt}{0.108pt}}
\multiput(796.58,252.17)(-5.579,-3.000){2}{\rule{0.583pt}{0.400pt}}
\multiput(786.16,248.95)(-1.579,-0.447){3}{\rule{1.167pt}{0.108pt}}
\multiput(788.58,249.17)(-5.579,-3.000){2}{\rule{0.583pt}{0.400pt}}
\multiput(777.60,245.95)(-1.802,-0.447){3}{\rule{1.300pt}{0.108pt}}
\multiput(780.30,246.17)(-6.302,-3.000){2}{\rule{0.650pt}{0.400pt}}
\multiput(769.16,242.95)(-1.579,-0.447){3}{\rule{1.167pt}{0.108pt}}
\multiput(771.58,243.17)(-5.579,-3.000){2}{\rule{0.583pt}{0.400pt}}
\multiput(761.16,239.95)(-1.579,-0.447){3}{\rule{1.167pt}{0.108pt}}
\multiput(763.58,240.17)(-5.579,-3.000){2}{\rule{0.583pt}{0.400pt}}
\multiput(753.16,236.95)(-1.579,-0.447){3}{\rule{1.167pt}{0.108pt}}
\multiput(755.58,237.17)(-5.579,-3.000){2}{\rule{0.583pt}{0.400pt}}
\multiput(745.16,233.95)(-1.579,-0.447){3}{\rule{1.167pt}{0.108pt}}
\multiput(747.58,234.17)(-5.579,-3.000){2}{\rule{0.583pt}{0.400pt}}
\multiput(737.16,230.95)(-1.579,-0.447){3}{\rule{1.167pt}{0.108pt}}
\multiput(739.58,231.17)(-5.579,-3.000){2}{\rule{0.583pt}{0.400pt}}
\multiput(729.16,227.95)(-1.579,-0.447){3}{\rule{1.167pt}{0.108pt}}
\multiput(731.58,228.17)(-5.579,-3.000){2}{\rule{0.583pt}{0.400pt}}
\multiput(721.16,224.95)(-1.579,-0.447){3}{\rule{1.167pt}{0.108pt}}
\multiput(723.58,225.17)(-5.579,-3.000){2}{\rule{0.583pt}{0.400pt}}
\multiput(713.16,221.95)(-1.579,-0.447){3}{\rule{1.167pt}{0.108pt}}
\multiput(715.58,222.17)(-5.579,-3.000){2}{\rule{0.583pt}{0.400pt}}
\multiput(704.60,218.95)(-1.802,-0.447){3}{\rule{1.300pt}{0.108pt}}
\multiput(707.30,219.17)(-6.302,-3.000){2}{\rule{0.650pt}{0.400pt}}
\multiput(696.16,215.95)(-1.579,-0.447){3}{\rule{1.167pt}{0.108pt}}
\multiput(698.58,216.17)(-5.579,-3.000){2}{\rule{0.583pt}{0.400pt}}
\multiput(688.16,212.95)(-1.579,-0.447){3}{\rule{1.167pt}{0.108pt}}
\multiput(690.58,213.17)(-5.579,-3.000){2}{\rule{0.583pt}{0.400pt}}
\multiput(680.16,209.95)(-1.579,-0.447){3}{\rule{1.167pt}{0.108pt}}
\multiput(682.58,210.17)(-5.579,-3.000){2}{\rule{0.583pt}{0.400pt}}
\multiput(673.26,206.94)(-1.066,-0.468){5}{\rule{0.900pt}{0.113pt}}
\multiput(675.13,207.17)(-6.132,-4.000){2}{\rule{0.450pt}{0.400pt}}
\multiput(664.16,202.95)(-1.579,-0.447){3}{\rule{1.167pt}{0.108pt}}
\multiput(666.58,203.17)(-5.579,-3.000){2}{\rule{0.583pt}{0.400pt}}
\multiput(656.16,199.95)(-1.579,-0.447){3}{\rule{1.167pt}{0.108pt}}
\multiput(658.58,200.17)(-5.579,-3.000){2}{\rule{0.583pt}{0.400pt}}
\multiput(648.16,196.95)(-1.579,-0.447){3}{\rule{1.167pt}{0.108pt}}
\multiput(650.58,197.17)(-5.579,-3.000){2}{\rule{0.583pt}{0.400pt}}
\multiput(640.16,193.95)(-1.579,-0.447){3}{\rule{1.167pt}{0.108pt}}
\multiput(642.58,194.17)(-5.579,-3.000){2}{\rule{0.583pt}{0.400pt}}
\multiput(632.16,190.95)(-1.579,-0.447){3}{\rule{1.167pt}{0.108pt}}
\multiput(634.58,191.17)(-5.579,-3.000){2}{\rule{0.583pt}{0.400pt}}
\multiput(623.60,187.95)(-1.802,-0.447){3}{\rule{1.300pt}{0.108pt}}
\multiput(626.30,188.17)(-6.302,-3.000){2}{\rule{0.650pt}{0.400pt}}
\multiput(615.16,184.95)(-1.579,-0.447){3}{\rule{1.167pt}{0.108pt}}
\multiput(617.58,185.17)(-5.579,-3.000){2}{\rule{0.583pt}{0.400pt}}
\multiput(607.16,181.95)(-1.579,-0.447){3}{\rule{1.167pt}{0.108pt}}
\multiput(609.58,182.17)(-5.579,-3.000){2}{\rule{0.583pt}{0.400pt}}
\multiput(599.16,178.95)(-1.579,-0.447){3}{\rule{1.167pt}{0.108pt}}
\multiput(601.58,179.17)(-5.579,-3.000){2}{\rule{0.583pt}{0.400pt}}
\multiput(591.16,175.95)(-1.579,-0.447){3}{\rule{1.167pt}{0.108pt}}
\multiput(593.58,176.17)(-5.579,-3.000){2}{\rule{0.583pt}{0.400pt}}
\multiput(583.16,172.95)(-1.579,-0.447){3}{\rule{1.167pt}{0.108pt}}
\multiput(585.58,173.17)(-5.579,-3.000){2}{\rule{0.583pt}{0.400pt}}
\multiput(575.16,169.95)(-1.579,-0.447){3}{\rule{1.167pt}{0.108pt}}
\multiput(577.58,170.17)(-5.579,-3.000){2}{\rule{0.583pt}{0.400pt}}
\multiput(567.16,166.95)(-1.579,-0.447){3}{\rule{1.167pt}{0.108pt}}
\multiput(569.58,167.17)(-5.579,-3.000){2}{\rule{0.583pt}{0.400pt}}
\multiput(559.16,163.95)(-1.579,-0.447){3}{\rule{1.167pt}{0.108pt}}
\multiput(561.58,164.17)(-5.579,-3.000){2}{\rule{0.583pt}{0.400pt}}
\multiput(551.16,160.95)(-1.579,-0.447){3}{\rule{1.167pt}{0.108pt}}
\multiput(553.58,161.17)(-5.579,-3.000){2}{\rule{0.583pt}{0.400pt}}
\multiput(542.60,157.95)(-1.802,-0.447){3}{\rule{1.300pt}{0.108pt}}
\multiput(545.30,158.17)(-6.302,-3.000){2}{\rule{0.650pt}{0.400pt}}
\multiput(535.26,154.94)(-1.066,-0.468){5}{\rule{0.900pt}{0.113pt}}
\multiput(537.13,155.17)(-6.132,-4.000){2}{\rule{0.450pt}{0.400pt}}
\multiput(526.16,150.95)(-1.579,-0.447){3}{\rule{1.167pt}{0.108pt}}
\multiput(528.58,151.17)(-5.579,-3.000){2}{\rule{0.583pt}{0.400pt}}
\multiput(518.16,147.95)(-1.579,-0.447){3}{\rule{1.167pt}{0.108pt}}
\multiput(520.58,148.17)(-5.579,-3.000){2}{\rule{0.583pt}{0.400pt}}
\multiput(510.16,144.95)(-1.579,-0.447){3}{\rule{1.167pt}{0.108pt}}
\multiput(512.58,145.17)(-5.579,-3.000){2}{\rule{0.583pt}{0.400pt}}
\multiput(502.16,141.95)(-1.579,-0.447){3}{\rule{1.167pt}{0.108pt}}
\multiput(504.58,142.17)(-5.579,-3.000){2}{\rule{0.583pt}{0.400pt}}
\multiput(494.16,138.95)(-1.579,-0.447){3}{\rule{1.167pt}{0.108pt}}
\multiput(496.58,139.17)(-5.579,-3.000){2}{\rule{0.583pt}{0.400pt}}
\multiput(486.16,135.95)(-1.579,-0.447){3}{\rule{1.167pt}{0.108pt}}
\multiput(488.58,136.17)(-5.579,-3.000){2}{\rule{0.583pt}{0.400pt}}
\multiput(478.16,132.95)(-1.579,-0.447){3}{\rule{1.167pt}{0.108pt}}
\multiput(480.58,133.17)(-5.579,-3.000){2}{\rule{0.583pt}{0.400pt}}
\multiput(470.16,129.95)(-1.579,-0.447){3}{\rule{1.167pt}{0.108pt}}
\multiput(472.58,130.17)(-5.579,-3.000){2}{\rule{0.583pt}{0.400pt}}
\multiput(461.60,126.95)(-1.802,-0.447){3}{\rule{1.300pt}{0.108pt}}
\multiput(464.30,127.17)(-6.302,-3.000){2}{\rule{0.650pt}{0.400pt}}
\multiput(453.16,123.95)(-1.579,-0.447){3}{\rule{1.167pt}{0.108pt}}
\multiput(455.58,124.17)(-5.579,-3.000){2}{\rule{0.583pt}{0.400pt}}
\multiput(445.16,120.95)(-1.579,-0.447){3}{\rule{1.167pt}{0.108pt}}
\multiput(447.58,121.17)(-5.579,-3.000){2}{\rule{0.583pt}{0.400pt}}
\multiput(437.16,117.95)(-1.579,-0.447){3}{\rule{1.167pt}{0.108pt}}
\multiput(439.58,118.17)(-5.579,-3.000){2}{\rule{0.583pt}{0.400pt}}
\multiput(429.16,114.95)(-1.579,-0.447){3}{\rule{1.167pt}{0.108pt}}
\multiput(431.58,115.17)(-5.579,-3.000){2}{\rule{0.583pt}{0.400pt}}
\multiput(421.16,111.95)(-1.579,-0.447){3}{\rule{1.167pt}{0.108pt}}
\multiput(423.58,112.17)(-5.579,-3.000){2}{\rule{0.583pt}{0.400pt}}
\multiput(413.16,108.95)(-1.579,-0.447){3}{\rule{1.167pt}{0.108pt}}
\multiput(415.58,109.17)(-5.579,-3.000){2}{\rule{0.583pt}{0.400pt}}
\multiput(405.16,105.95)(-1.579,-0.447){3}{\rule{1.167pt}{0.108pt}}
\multiput(407.58,106.17)(-5.579,-3.000){2}{\rule{0.583pt}{0.400pt}}
\multiput(397.16,104.61)(-1.579,0.447){3}{\rule{1.167pt}{0.108pt}}
\multiput(399.58,103.17)(-5.579,3.000){2}{\rule{0.583pt}{0.400pt}}
\multiput(389.16,107.61)(-1.579,0.447){3}{\rule{1.167pt}{0.108pt}}
\multiput(391.58,106.17)(-5.579,3.000){2}{\rule{0.583pt}{0.400pt}}
\multiput(380.60,110.61)(-1.802,0.447){3}{\rule{1.300pt}{0.108pt}}
\multiput(383.30,109.17)(-6.302,3.000){2}{\rule{0.650pt}{0.400pt}}
\multiput(372.16,113.61)(-1.579,0.447){3}{\rule{1.167pt}{0.108pt}}
\multiput(374.58,112.17)(-5.579,3.000){2}{\rule{0.583pt}{0.400pt}}
\multiput(364.16,116.61)(-1.579,0.447){3}{\rule{1.167pt}{0.108pt}}
\multiput(366.58,115.17)(-5.579,3.000){2}{\rule{0.583pt}{0.400pt}}
\multiput(356.16,119.61)(-1.579,0.447){3}{\rule{1.167pt}{0.108pt}}
\multiput(358.58,118.17)(-5.579,3.000){2}{\rule{0.583pt}{0.400pt}}
\multiput(348.16,122.61)(-1.579,0.447){3}{\rule{1.167pt}{0.108pt}}
\multiput(350.58,121.17)(-5.579,3.000){2}{\rule{0.583pt}{0.400pt}}
\multiput(340.16,125.61)(-1.579,0.447){3}{\rule{1.167pt}{0.108pt}}
\multiput(342.58,124.17)(-5.579,3.000){2}{\rule{0.583pt}{0.400pt}}
\multiput(332.16,128.61)(-1.579,0.447){3}{\rule{1.167pt}{0.108pt}}
\multiput(334.58,127.17)(-5.579,3.000){2}{\rule{0.583pt}{0.400pt}}
\multiput(324.16,131.61)(-1.579,0.447){3}{\rule{1.167pt}{0.108pt}}
\multiput(326.58,130.17)(-5.579,3.000){2}{\rule{0.583pt}{0.400pt}}
\multiput(316.16,134.61)(-1.579,0.447){3}{\rule{1.167pt}{0.108pt}}
\multiput(318.58,133.17)(-5.579,3.000){2}{\rule{0.583pt}{0.400pt}}
\multiput(307.60,137.61)(-1.802,0.447){3}{\rule{1.300pt}{0.108pt}}
\multiput(310.30,136.17)(-6.302,3.000){2}{\rule{0.650pt}{0.400pt}}
\multiput(299.16,140.61)(-1.579,0.447){3}{\rule{1.167pt}{0.108pt}}
\multiput(301.58,139.17)(-5.579,3.000){2}{\rule{0.583pt}{0.400pt}}
\multiput(291.16,143.61)(-1.579,0.447){3}{\rule{1.167pt}{0.108pt}}
\multiput(293.58,142.17)(-5.579,3.000){2}{\rule{0.583pt}{0.400pt}}
\multiput(283.16,146.61)(-1.579,0.447){3}{\rule{1.167pt}{0.108pt}}
\multiput(285.58,145.17)(-5.579,3.000){2}{\rule{0.583pt}{0.400pt}}
\multiput(275.16,149.61)(-1.579,0.447){3}{\rule{1.167pt}{0.108pt}}
\multiput(277.58,148.17)(-5.579,3.000){2}{\rule{0.583pt}{0.400pt}}
\multiput(268.26,152.60)(-1.066,0.468){5}{\rule{0.900pt}{0.113pt}}
\multiput(270.13,151.17)(-6.132,4.000){2}{\rule{0.450pt}{0.400pt}}
\multiput(259.16,156.61)(-1.579,0.447){3}{\rule{1.167pt}{0.108pt}}
\multiput(261.58,155.17)(-5.579,3.000){2}{\rule{0.583pt}{0.400pt}}
\multiput(251.16,159.61)(-1.579,0.447){3}{\rule{1.167pt}{0.108pt}}
\multiput(253.58,158.17)(-5.579,3.000){2}{\rule{0.583pt}{0.400pt}}
\multiput(243.16,162.61)(-1.579,0.447){3}{\rule{1.167pt}{0.108pt}}
\multiput(245.58,161.17)(-5.579,3.000){2}{\rule{0.583pt}{0.400pt}}
\multiput(235.16,165.61)(-1.579,0.447){3}{\rule{1.167pt}{0.108pt}}
\multiput(237.58,164.17)(-5.579,3.000){2}{\rule{0.583pt}{0.400pt}}
\multiput(226.60,168.61)(-1.802,0.447){3}{\rule{1.300pt}{0.108pt}}
\multiput(229.30,167.17)(-6.302,3.000){2}{\rule{0.650pt}{0.400pt}}
\multiput(218.16,171.61)(-1.579,0.447){3}{\rule{1.167pt}{0.108pt}}
\multiput(220.58,170.17)(-5.579,3.000){2}{\rule{0.583pt}{0.400pt}}
\multiput(210.16,174.61)(-1.579,0.447){3}{\rule{1.167pt}{0.108pt}}
\multiput(212.58,173.17)(-5.579,3.000){2}{\rule{0.583pt}{0.400pt}}
\multiput(202.16,177.61)(-1.579,0.447){3}{\rule{1.167pt}{0.108pt}}
\multiput(204.58,176.17)(-5.579,3.000){2}{\rule{0.583pt}{0.400pt}}
\multiput(194.16,180.61)(-1.579,0.447){3}{\rule{1.167pt}{0.108pt}}
\multiput(196.58,179.17)(-5.579,3.000){2}{\rule{0.583pt}{0.400pt}}
\multiput(186.16,183.61)(-1.579,0.447){3}{\rule{1.167pt}{0.108pt}}
\multiput(188.58,182.17)(-5.579,3.000){2}{\rule{0.583pt}{0.400pt}}
\multiput(178.16,186.61)(-1.579,0.447){3}{\rule{1.167pt}{0.108pt}}
\multiput(180.58,185.17)(-5.579,3.000){2}{\rule{0.583pt}{0.400pt}}
\multiput(170.16,189.61)(-1.579,0.447){3}{\rule{1.167pt}{0.108pt}}
\multiput(172.58,188.17)(-5.579,3.000){2}{\rule{0.583pt}{0.400pt}}
\multiput(162.16,192.61)(-1.579,0.447){3}{\rule{1.167pt}{0.108pt}}
\multiput(164.58,191.17)(-5.579,3.000){2}{\rule{0.583pt}{0.400pt}}
\multiput(154.16,195.61)(-1.579,0.447){3}{\rule{1.167pt}{0.108pt}}
\multiput(156.58,194.17)(-5.579,3.000){2}{\rule{0.583pt}{0.400pt}}
\multiput(151.00,198.61)(1.579,0.447){3}{\rule{1.167pt}{0.108pt}}
\multiput(151.00,197.17)(5.579,3.000){2}{\rule{0.583pt}{0.400pt}}
\multiput(159.00,201.61)(1.579,0.447){3}{\rule{1.167pt}{0.108pt}}
\multiput(159.00,200.17)(5.579,3.000){2}{\rule{0.583pt}{0.400pt}}
\multiput(167.00,204.60)(1.066,0.468){5}{\rule{0.900pt}{0.113pt}}
\multiput(167.00,203.17)(6.132,4.000){2}{\rule{0.450pt}{0.400pt}}
\multiput(175.00,208.61)(1.579,0.447){3}{\rule{1.167pt}{0.108pt}}
\multiput(175.00,207.17)(5.579,3.000){2}{\rule{0.583pt}{0.400pt}}
\multiput(183.00,211.61)(1.579,0.447){3}{\rule{1.167pt}{0.108pt}}
\multiput(183.00,210.17)(5.579,3.000){2}{\rule{0.583pt}{0.400pt}}
\multiput(191.00,214.61)(1.579,0.447){3}{\rule{1.167pt}{0.108pt}}
\multiput(191.00,213.17)(5.579,3.000){2}{\rule{0.583pt}{0.400pt}}
\multiput(199.00,217.61)(1.579,0.447){3}{\rule{1.167pt}{0.108pt}}
\multiput(199.00,216.17)(5.579,3.000){2}{\rule{0.583pt}{0.400pt}}
\multiput(207.00,220.61)(1.579,0.447){3}{\rule{1.167pt}{0.108pt}}
\multiput(207.00,219.17)(5.579,3.000){2}{\rule{0.583pt}{0.400pt}}
\multiput(215.00,223.61)(1.579,0.447){3}{\rule{1.167pt}{0.108pt}}
\multiput(215.00,222.17)(5.579,3.000){2}{\rule{0.583pt}{0.400pt}}
\multiput(223.00,226.61)(1.802,0.447){3}{\rule{1.300pt}{0.108pt}}
\multiput(223.00,225.17)(6.302,3.000){2}{\rule{0.650pt}{0.400pt}}
\multiput(232.00,229.61)(1.579,0.447){3}{\rule{1.167pt}{0.108pt}}
\multiput(232.00,228.17)(5.579,3.000){2}{\rule{0.583pt}{0.400pt}}
\multiput(240.00,232.61)(1.579,0.447){3}{\rule{1.167pt}{0.108pt}}
\multiput(240.00,231.17)(5.579,3.000){2}{\rule{0.583pt}{0.400pt}}
\multiput(248.00,235.61)(1.579,0.447){3}{\rule{1.167pt}{0.108pt}}
\multiput(248.00,234.17)(5.579,3.000){2}{\rule{0.583pt}{0.400pt}}
\multiput(256.00,238.61)(1.579,0.447){3}{\rule{1.167pt}{0.108pt}}
\multiput(256.00,237.17)(5.579,3.000){2}{\rule{0.583pt}{0.400pt}}
\multiput(264.00,241.61)(1.579,0.447){3}{\rule{1.167pt}{0.108pt}}
\multiput(264.00,240.17)(5.579,3.000){2}{\rule{0.583pt}{0.400pt}}
\multiput(272.00,244.61)(1.579,0.447){3}{\rule{1.167pt}{0.108pt}}
\multiput(272.00,243.17)(5.579,3.000){2}{\rule{0.583pt}{0.400pt}}
\multiput(280.00,247.61)(1.579,0.447){3}{\rule{1.167pt}{0.108pt}}
\multiput(280.00,246.17)(5.579,3.000){2}{\rule{0.583pt}{0.400pt}}
\multiput(288.00,250.61)(1.579,0.447){3}{\rule{1.167pt}{0.108pt}}
\multiput(288.00,249.17)(5.579,3.000){2}{\rule{0.583pt}{0.400pt}}
\multiput(296.00,253.61)(1.579,0.447){3}{\rule{1.167pt}{0.108pt}}
\multiput(296.00,252.17)(5.579,3.000){2}{\rule{0.583pt}{0.400pt}}
\multiput(304.00,256.60)(1.212,0.468){5}{\rule{1.000pt}{0.113pt}}
\multiput(304.00,255.17)(6.924,4.000){2}{\rule{0.500pt}{0.400pt}}
\multiput(313.00,260.61)(1.579,0.447){3}{\rule{1.167pt}{0.108pt}}
\multiput(313.00,259.17)(5.579,3.000){2}{\rule{0.583pt}{0.400pt}}
\multiput(321.00,263.61)(1.579,0.447){3}{\rule{1.167pt}{0.108pt}}
\multiput(321.00,262.17)(5.579,3.000){2}{\rule{0.583pt}{0.400pt}}
\multiput(329.00,266.61)(1.579,0.447){3}{\rule{1.167pt}{0.108pt}}
\multiput(329.00,265.17)(5.579,3.000){2}{\rule{0.583pt}{0.400pt}}
\multiput(337.00,269.61)(1.579,0.447){3}{\rule{1.167pt}{0.108pt}}
\multiput(337.00,268.17)(5.579,3.000){2}{\rule{0.583pt}{0.400pt}}
\multiput(345.00,272.61)(1.579,0.447){3}{\rule{1.167pt}{0.108pt}}
\multiput(345.00,271.17)(5.579,3.000){2}{\rule{0.583pt}{0.400pt}}
\multiput(353.00,275.61)(1.579,0.447){3}{\rule{1.167pt}{0.108pt}}
\multiput(353.00,274.17)(5.579,3.000){2}{\rule{0.583pt}{0.400pt}}
\multiput(361.00,278.61)(1.579,0.447){3}{\rule{1.167pt}{0.108pt}}
\multiput(361.00,277.17)(5.579,3.000){2}{\rule{0.583pt}{0.400pt}}
\multiput(369.00,281.61)(1.579,0.447){3}{\rule{1.167pt}{0.108pt}}
\multiput(369.00,280.17)(5.579,3.000){2}{\rule{0.583pt}{0.400pt}}
\multiput(377.00,284.61)(1.802,0.447){3}{\rule{1.300pt}{0.108pt}}
\multiput(377.00,283.17)(6.302,3.000){2}{\rule{0.650pt}{0.400pt}}
\multiput(386.00,287.61)(1.579,0.447){3}{\rule{1.167pt}{0.108pt}}
\multiput(386.00,286.17)(5.579,3.000){2}{\rule{0.583pt}{0.400pt}}
\multiput(394.00,290.61)(1.579,0.447){3}{\rule{1.167pt}{0.108pt}}
\multiput(394.00,289.17)(5.579,3.000){2}{\rule{0.583pt}{0.400pt}}
\multiput(402.00,293.61)(1.579,0.447){3}{\rule{1.167pt}{0.108pt}}
\multiput(402.00,292.17)(5.579,3.000){2}{\rule{0.583pt}{0.400pt}}
\multiput(410.00,296.61)(1.579,0.447){3}{\rule{1.167pt}{0.108pt}}
\multiput(410.00,295.17)(5.579,3.000){2}{\rule{0.583pt}{0.400pt}}
\multiput(418.00,299.61)(1.579,0.447){3}{\rule{1.167pt}{0.108pt}}
\multiput(418.00,298.17)(5.579,3.000){2}{\rule{0.583pt}{0.400pt}}
\multiput(426.00,302.61)(1.579,0.447){3}{\rule{1.167pt}{0.108pt}}
\multiput(426.00,301.17)(5.579,3.000){2}{\rule{0.583pt}{0.400pt}}
\multiput(434.00,305.61)(1.579,0.447){3}{\rule{1.167pt}{0.108pt}}
\multiput(434.00,304.17)(5.579,3.000){2}{\rule{0.583pt}{0.400pt}}
\multiput(442.00,308.61)(1.579,0.447){3}{\rule{1.167pt}{0.108pt}}
\multiput(442.00,307.17)(5.579,3.000){2}{\rule{0.583pt}{0.400pt}}
\multiput(450.00,311.60)(1.066,0.468){5}{\rule{0.900pt}{0.113pt}}
\multiput(450.00,310.17)(6.132,4.000){2}{\rule{0.450pt}{0.400pt}}
\multiput(458.00,315.61)(1.802,0.447){3}{\rule{1.300pt}{0.108pt}}
\multiput(458.00,314.17)(6.302,3.000){2}{\rule{0.650pt}{0.400pt}}
\multiput(467.00,318.61)(1.579,0.447){3}{\rule{1.167pt}{0.108pt}}
\multiput(467.00,317.17)(5.579,3.000){2}{\rule{0.583pt}{0.400pt}}
\multiput(475.00,321.61)(1.579,0.447){3}{\rule{1.167pt}{0.108pt}}
\multiput(475.00,320.17)(5.579,3.000){2}{\rule{0.583pt}{0.400pt}}
\multiput(483.00,324.61)(1.579,0.447){3}{\rule{1.167pt}{0.108pt}}
\multiput(483.00,323.17)(5.579,3.000){2}{\rule{0.583pt}{0.400pt}}
\multiput(491.00,327.61)(1.579,0.447){3}{\rule{1.167pt}{0.108pt}}
\multiput(491.00,326.17)(5.579,3.000){2}{\rule{0.583pt}{0.400pt}}
\multiput(499.00,330.61)(1.579,0.447){3}{\rule{1.167pt}{0.108pt}}
\multiput(499.00,329.17)(5.579,3.000){2}{\rule{0.583pt}{0.400pt}}
\multiput(507.00,333.61)(1.579,0.447){3}{\rule{1.167pt}{0.108pt}}
\multiput(507.00,332.17)(5.579,3.000){2}{\rule{0.583pt}{0.400pt}}
\multiput(515.00,336.61)(1.579,0.447){3}{\rule{1.167pt}{0.108pt}}
\multiput(515.00,335.17)(5.579,3.000){2}{\rule{0.583pt}{0.400pt}}
\multiput(523.00,339.61)(1.579,0.447){3}{\rule{1.167pt}{0.108pt}}
\multiput(523.00,338.17)(5.579,3.000){2}{\rule{0.583pt}{0.400pt}}
\multiput(531.00,342.61)(1.579,0.447){3}{\rule{1.167pt}{0.108pt}}
\multiput(531.00,341.17)(5.579,3.000){2}{\rule{0.583pt}{0.400pt}}
\multiput(539.00,345.61)(1.802,0.447){3}{\rule{1.300pt}{0.108pt}}
\multiput(539.00,344.17)(6.302,3.000){2}{\rule{0.650pt}{0.400pt}}
\multiput(548.00,348.61)(1.579,0.447){3}{\rule{1.167pt}{0.108pt}}
\multiput(548.00,347.17)(5.579,3.000){2}{\rule{0.583pt}{0.400pt}}
\multiput(556.00,351.61)(1.579,0.447){3}{\rule{1.167pt}{0.108pt}}
\multiput(556.00,350.17)(5.579,3.000){2}{\rule{0.583pt}{0.400pt}}
\multiput(564.00,354.61)(1.579,0.447){3}{\rule{1.167pt}{0.108pt}}
\multiput(564.00,353.17)(5.579,3.000){2}{\rule{0.583pt}{0.400pt}}
\multiput(572.00,357.61)(1.579,0.447){3}{\rule{1.167pt}{0.108pt}}
\multiput(572.00,356.17)(5.579,3.000){2}{\rule{0.583pt}{0.400pt}}
\multiput(580.00,360.61)(1.579,0.447){3}{\rule{1.167pt}{0.108pt}}
\multiput(580.00,359.17)(5.579,3.000){2}{\rule{0.583pt}{0.400pt}}
\multiput(588.00,363.60)(1.066,0.468){5}{\rule{0.900pt}{0.113pt}}
\multiput(588.00,362.17)(6.132,4.000){2}{\rule{0.450pt}{0.400pt}}
\multiput(596.00,367.61)(1.579,0.447){3}{\rule{1.167pt}{0.108pt}}
\multiput(596.00,366.17)(5.579,3.000){2}{\rule{0.583pt}{0.400pt}}
\multiput(604.00,370.61)(1.579,0.447){3}{\rule{1.167pt}{0.108pt}}
\multiput(604.00,369.17)(5.579,3.000){2}{\rule{0.583pt}{0.400pt}}
\multiput(612.00,373.61)(1.579,0.447){3}{\rule{1.167pt}{0.108pt}}
\multiput(612.00,372.17)(5.579,3.000){2}{\rule{0.583pt}{0.400pt}}
\multiput(620.00,376.61)(1.802,0.447){3}{\rule{1.300pt}{0.108pt}}
\multiput(620.00,375.17)(6.302,3.000){2}{\rule{0.650pt}{0.400pt}}
\multiput(629.00,379.61)(1.579,0.447){3}{\rule{1.167pt}{0.108pt}}
\multiput(629.00,378.17)(5.579,3.000){2}{\rule{0.583pt}{0.400pt}}
\multiput(637.00,382.61)(1.579,0.447){3}{\rule{1.167pt}{0.108pt}}
\multiput(637.00,381.17)(5.579,3.000){2}{\rule{0.583pt}{0.400pt}}
\multiput(645.00,385.61)(1.579,0.447){3}{\rule{1.167pt}{0.108pt}}
\multiput(645.00,384.17)(5.579,3.000){2}{\rule{0.583pt}{0.400pt}}
\multiput(653.00,388.61)(1.579,0.447){3}{\rule{1.167pt}{0.108pt}}
\multiput(653.00,387.17)(5.579,3.000){2}{\rule{0.583pt}{0.400pt}}
\multiput(661.00,391.61)(1.579,0.447){3}{\rule{1.167pt}{0.108pt}}
\multiput(661.00,390.17)(5.579,3.000){2}{\rule{0.583pt}{0.400pt}}
\multiput(669.00,394.61)(1.579,0.447){3}{\rule{1.167pt}{0.108pt}}
\multiput(669.00,393.17)(5.579,3.000){2}{\rule{0.583pt}{0.400pt}}
\multiput(677.00,397.61)(1.579,0.447){3}{\rule{1.167pt}{0.108pt}}
\multiput(677.00,396.17)(5.579,3.000){2}{\rule{0.583pt}{0.400pt}}
\multiput(685.00,400.61)(1.579,0.447){3}{\rule{1.167pt}{0.108pt}}
\multiput(685.00,399.17)(5.579,3.000){2}{\rule{0.583pt}{0.400pt}}
\multiput(693.00,403.61)(1.579,0.447){3}{\rule{1.167pt}{0.108pt}}
\multiput(693.00,402.17)(5.579,3.000){2}{\rule{0.583pt}{0.400pt}}
\multiput(701.00,406.61)(1.802,0.447){3}{\rule{1.300pt}{0.108pt}}
\multiput(701.00,405.17)(6.302,3.000){2}{\rule{0.650pt}{0.400pt}}
\multiput(710.00,409.61)(1.579,0.447){3}{\rule{1.167pt}{0.108pt}}
\multiput(710.00,408.17)(5.579,3.000){2}{\rule{0.583pt}{0.400pt}}
\multiput(718.00,412.61)(1.579,0.447){3}{\rule{1.167pt}{0.108pt}}
\multiput(718.00,411.17)(5.579,3.000){2}{\rule{0.583pt}{0.400pt}}
\multiput(726.00,415.61)(1.579,0.447){3}{\rule{1.167pt}{0.108pt}}
\multiput(726.00,414.17)(5.579,3.000){2}{\rule{0.583pt}{0.400pt}}
\multiput(734.00,418.60)(1.066,0.468){5}{\rule{0.900pt}{0.113pt}}
\multiput(734.00,417.17)(6.132,4.000){2}{\rule{0.450pt}{0.400pt}}
\multiput(742.00,422.61)(1.579,0.447){3}{\rule{1.167pt}{0.108pt}}
\multiput(742.00,421.17)(5.579,3.000){2}{\rule{0.583pt}{0.400pt}}
\multiput(750.00,425.61)(1.579,0.447){3}{\rule{1.167pt}{0.108pt}}
\multiput(750.00,424.17)(5.579,3.000){2}{\rule{0.583pt}{0.400pt}}
\multiput(758.00,428.61)(1.579,0.447){3}{\rule{1.167pt}{0.108pt}}
\multiput(758.00,427.17)(5.579,3.000){2}{\rule{0.583pt}{0.400pt}}
\multiput(766.00,431.61)(1.579,0.447){3}{\rule{1.167pt}{0.108pt}}
\multiput(766.00,430.17)(5.579,3.000){2}{\rule{0.583pt}{0.400pt}}
\multiput(774.00,434.61)(1.802,0.447){3}{\rule{1.300pt}{0.108pt}}
\multiput(774.00,433.17)(6.302,3.000){2}{\rule{0.650pt}{0.400pt}}
\multiput(783.00,437.61)(1.579,0.447){3}{\rule{1.167pt}{0.108pt}}
\multiput(783.00,436.17)(5.579,3.000){2}{\rule{0.583pt}{0.400pt}}
\multiput(791.00,440.61)(1.579,0.447){3}{\rule{1.167pt}{0.108pt}}
\multiput(791.00,439.17)(5.579,3.000){2}{\rule{0.583pt}{0.400pt}}
\multiput(799.00,443.61)(1.579,0.447){3}{\rule{1.167pt}{0.108pt}}
\multiput(799.00,442.17)(5.579,3.000){2}{\rule{0.583pt}{0.400pt}}
\multiput(807.00,446.61)(1.579,0.447){3}{\rule{1.167pt}{0.108pt}}
\multiput(807.00,445.17)(5.579,3.000){2}{\rule{0.583pt}{0.400pt}}
\multiput(815.00,449.61)(1.579,0.447){3}{\rule{1.167pt}{0.108pt}}
\multiput(815.00,448.17)(5.579,3.000){2}{\rule{0.583pt}{0.400pt}}
\multiput(823.00,452.61)(1.579,0.447){3}{\rule{1.167pt}{0.108pt}}
\multiput(823.00,451.17)(5.579,3.000){2}{\rule{0.583pt}{0.400pt}}
\multiput(831.00,455.61)(1.579,0.447){3}{\rule{1.167pt}{0.108pt}}
\multiput(831.00,454.17)(5.579,3.000){2}{\rule{0.583pt}{0.400pt}}
\multiput(839.00,458.61)(1.579,0.447){3}{\rule{1.167pt}{0.108pt}}
\multiput(839.00,457.17)(5.579,3.000){2}{\rule{0.583pt}{0.400pt}}
\multiput(847.00,461.61)(1.579,0.447){3}{\rule{1.167pt}{0.108pt}}
\multiput(847.00,460.17)(5.579,3.000){2}{\rule{0.583pt}{0.400pt}}
\multiput(855.00,464.61)(1.802,0.447){3}{\rule{1.300pt}{0.108pt}}
\multiput(855.00,463.17)(6.302,3.000){2}{\rule{0.650pt}{0.400pt}}
\multiput(864.00,467.61)(1.579,0.447){3}{\rule{1.167pt}{0.108pt}}
\multiput(864.00,466.17)(5.579,3.000){2}{\rule{0.583pt}{0.400pt}}
\put(872,468.67){\rule{0.964pt}{0.400pt}}
\multiput(872.00,469.17)(2.000,-1.000){2}{\rule{0.482pt}{0.400pt}}
\put(876,467.67){\rule{1.204pt}{0.400pt}}
\multiput(876.00,468.17)(2.500,-1.000){2}{\rule{0.602pt}{0.400pt}}
\put(881,466.17){\rule{1.100pt}{0.400pt}}
\multiput(881.00,467.17)(2.717,-2.000){2}{\rule{0.550pt}{0.400pt}}
\put(886,464.67){\rule{0.964pt}{0.400pt}}
\multiput(886.00,465.17)(2.000,-1.000){2}{\rule{0.482pt}{0.400pt}}
\put(890,463.17){\rule{1.100pt}{0.400pt}}
\multiput(890.00,464.17)(2.717,-2.000){2}{\rule{0.550pt}{0.400pt}}
\put(895,461.67){\rule{1.204pt}{0.400pt}}
\multiput(895.00,462.17)(2.500,-1.000){2}{\rule{0.602pt}{0.400pt}}
\put(900,460.17){\rule{0.900pt}{0.400pt}}
\multiput(900.00,461.17)(2.132,-2.000){2}{\rule{0.450pt}{0.400pt}}
\put(904,458.67){\rule{1.204pt}{0.400pt}}
\multiput(904.00,459.17)(2.500,-1.000){2}{\rule{0.602pt}{0.400pt}}
\put(909,457.17){\rule{1.100pt}{0.400pt}}
\multiput(909.00,458.17)(2.717,-2.000){2}{\rule{0.550pt}{0.400pt}}
\put(914,455.67){\rule{0.964pt}{0.400pt}}
\multiput(914.00,456.17)(2.000,-1.000){2}{\rule{0.482pt}{0.400pt}}
\put(918,454.17){\rule{1.100pt}{0.400pt}}
\multiput(918.00,455.17)(2.717,-2.000){2}{\rule{0.550pt}{0.400pt}}
\put(923,452.67){\rule{1.204pt}{0.400pt}}
\multiput(923.00,453.17)(2.500,-1.000){2}{\rule{0.602pt}{0.400pt}}
\put(928,451.67){\rule{0.964pt}{0.400pt}}
\multiput(928.00,452.17)(2.000,-1.000){2}{\rule{0.482pt}{0.400pt}}
\put(932,450.17){\rule{1.100pt}{0.400pt}}
\multiput(932.00,451.17)(2.717,-2.000){2}{\rule{0.550pt}{0.400pt}}
\put(937,448.67){\rule{1.204pt}{0.400pt}}
\multiput(937.00,449.17)(2.500,-1.000){2}{\rule{0.602pt}{0.400pt}}
\put(942,447.17){\rule{1.100pt}{0.400pt}}
\multiput(942.00,448.17)(2.717,-2.000){2}{\rule{0.550pt}{0.400pt}}
\put(947,445.67){\rule{0.964pt}{0.400pt}}
\multiput(947.00,446.17)(2.000,-1.000){2}{\rule{0.482pt}{0.400pt}}
\put(951,444.17){\rule{1.100pt}{0.400pt}}
\multiput(951.00,445.17)(2.717,-2.000){2}{\rule{0.550pt}{0.400pt}}
\put(956,442.67){\rule{1.204pt}{0.400pt}}
\multiput(956.00,443.17)(2.500,-1.000){2}{\rule{0.602pt}{0.400pt}}
\put(961,441.17){\rule{0.900pt}{0.400pt}}
\multiput(961.00,442.17)(2.132,-2.000){2}{\rule{0.450pt}{0.400pt}}
\put(965,439.67){\rule{1.204pt}{0.400pt}}
\multiput(965.00,440.17)(2.500,-1.000){2}{\rule{0.602pt}{0.400pt}}
\put(970,438.67){\rule{1.204pt}{0.400pt}}
\multiput(970.00,439.17)(2.500,-1.000){2}{\rule{0.602pt}{0.400pt}}
\put(975,437.17){\rule{0.900pt}{0.400pt}}
\multiput(975.00,438.17)(2.132,-2.000){2}{\rule{0.450pt}{0.400pt}}
\put(979,435.67){\rule{1.204pt}{0.400pt}}
\multiput(979.00,436.17)(2.500,-1.000){2}{\rule{0.602pt}{0.400pt}}
\put(984,434.17){\rule{1.100pt}{0.400pt}}
\multiput(984.00,435.17)(2.717,-2.000){2}{\rule{0.550pt}{0.400pt}}
\put(989,432.67){\rule{0.964pt}{0.400pt}}
\multiput(989.00,433.17)(2.000,-1.000){2}{\rule{0.482pt}{0.400pt}}
\put(993,431.17){\rule{1.100pt}{0.400pt}}
\multiput(993.00,432.17)(2.717,-2.000){2}{\rule{0.550pt}{0.400pt}}
\put(998,429.67){\rule{1.204pt}{0.400pt}}
\multiput(998.00,430.17)(2.500,-1.000){2}{\rule{0.602pt}{0.400pt}}
\put(1003,428.17){\rule{0.900pt}{0.400pt}}
\multiput(1003.00,429.17)(2.132,-2.000){2}{\rule{0.450pt}{0.400pt}}
\put(1007,426.67){\rule{1.204pt}{0.400pt}}
\multiput(1007.00,427.17)(2.500,-1.000){2}{\rule{0.602pt}{0.400pt}}
\put(1012,425.17){\rule{1.100pt}{0.400pt}}
\multiput(1012.00,426.17)(2.717,-2.000){2}{\rule{0.550pt}{0.400pt}}
\put(1017,423.67){\rule{0.964pt}{0.400pt}}
\multiput(1017.00,424.17)(2.000,-1.000){2}{\rule{0.482pt}{0.400pt}}
\put(1021,422.67){\rule{1.204pt}{0.400pt}}
\multiput(1021.00,423.17)(2.500,-1.000){2}{\rule{0.602pt}{0.400pt}}
\put(1026,421.17){\rule{1.100pt}{0.400pt}}
\multiput(1026.00,422.17)(2.717,-2.000){2}{\rule{0.550pt}{0.400pt}}
\put(1031,419.67){\rule{0.964pt}{0.400pt}}
\multiput(1031.00,420.17)(2.000,-1.000){2}{\rule{0.482pt}{0.400pt}}
\put(1035,418.17){\rule{1.100pt}{0.400pt}}
\multiput(1035.00,419.17)(2.717,-2.000){2}{\rule{0.550pt}{0.400pt}}
\put(1040,416.67){\rule{1.204pt}{0.400pt}}
\multiput(1040.00,417.17)(2.500,-1.000){2}{\rule{0.602pt}{0.400pt}}
\put(1045,415.17){\rule{0.900pt}{0.400pt}}
\multiput(1045.00,416.17)(2.132,-2.000){2}{\rule{0.450pt}{0.400pt}}
\put(1049,413.67){\rule{1.204pt}{0.400pt}}
\multiput(1049.00,414.17)(2.500,-1.000){2}{\rule{0.602pt}{0.400pt}}
\put(1054,412.17){\rule{1.100pt}{0.400pt}}
\multiput(1054.00,413.17)(2.717,-2.000){2}{\rule{0.550pt}{0.400pt}}
\put(1059,410.67){\rule{1.204pt}{0.400pt}}
\multiput(1059.00,411.17)(2.500,-1.000){2}{\rule{0.602pt}{0.400pt}}
\put(1064,409.67){\rule{0.964pt}{0.400pt}}
\multiput(1064.00,410.17)(2.000,-1.000){2}{\rule{0.482pt}{0.400pt}}
\put(1068,408.17){\rule{1.100pt}{0.400pt}}
\multiput(1068.00,409.17)(2.717,-2.000){2}{\rule{0.550pt}{0.400pt}}
\put(1073,406.67){\rule{1.204pt}{0.400pt}}
\multiput(1073.00,407.17)(2.500,-1.000){2}{\rule{0.602pt}{0.400pt}}
\put(1078,405.17){\rule{0.900pt}{0.400pt}}
\multiput(1078.00,406.17)(2.132,-2.000){2}{\rule{0.450pt}{0.400pt}}
\put(1082,403.67){\rule{1.204pt}{0.400pt}}
\multiput(1082.00,404.17)(2.500,-1.000){2}{\rule{0.602pt}{0.400pt}}
\put(1087,402.17){\rule{1.100pt}{0.400pt}}
\multiput(1087.00,403.17)(2.717,-2.000){2}{\rule{0.550pt}{0.400pt}}
\put(1092,400.67){\rule{0.964pt}{0.400pt}}
\multiput(1092.00,401.17)(2.000,-1.000){2}{\rule{0.482pt}{0.400pt}}
\put(1096,399.17){\rule{1.100pt}{0.400pt}}
\multiput(1096.00,400.17)(2.717,-2.000){2}{\rule{0.550pt}{0.400pt}}
\put(1101,397.67){\rule{1.204pt}{0.400pt}}
\multiput(1101.00,398.17)(2.500,-1.000){2}{\rule{0.602pt}{0.400pt}}
\put(1106,396.17){\rule{0.900pt}{0.400pt}}
\multiput(1106.00,397.17)(2.132,-2.000){2}{\rule{0.450pt}{0.400pt}}
\put(1110,394.67){\rule{1.204pt}{0.400pt}}
\multiput(1110.00,395.17)(2.500,-1.000){2}{\rule{0.602pt}{0.400pt}}
\put(1115,393.67){\rule{1.204pt}{0.400pt}}
\multiput(1115.00,394.17)(2.500,-1.000){2}{\rule{0.602pt}{0.400pt}}
\put(1120,392.17){\rule{0.900pt}{0.400pt}}
\multiput(1120.00,393.17)(2.132,-2.000){2}{\rule{0.450pt}{0.400pt}}
\put(1124,390.67){\rule{1.204pt}{0.400pt}}
\multiput(1124.00,391.17)(2.500,-1.000){2}{\rule{0.602pt}{0.400pt}}
\put(1129,389.17){\rule{1.100pt}{0.400pt}}
\multiput(1129.00,390.17)(2.717,-2.000){2}{\rule{0.550pt}{0.400pt}}
\put(1134,387.67){\rule{0.964pt}{0.400pt}}
\multiput(1134.00,388.17)(2.000,-1.000){2}{\rule{0.482pt}{0.400pt}}
\put(1138,386.17){\rule{1.100pt}{0.400pt}}
\multiput(1138.00,387.17)(2.717,-2.000){2}{\rule{0.550pt}{0.400pt}}
\put(1143,384.67){\rule{1.204pt}{0.400pt}}
\multiput(1143.00,385.17)(2.500,-1.000){2}{\rule{0.602pt}{0.400pt}}
\put(1148,383.17){\rule{0.900pt}{0.400pt}}
\multiput(1148.00,384.17)(2.132,-2.000){2}{\rule{0.450pt}{0.400pt}}
\put(1152,381.67){\rule{1.204pt}{0.400pt}}
\multiput(1152.00,382.17)(2.500,-1.000){2}{\rule{0.602pt}{0.400pt}}
\put(1157,380.17){\rule{1.100pt}{0.400pt}}
\multiput(1157.00,381.17)(2.717,-2.000){2}{\rule{0.550pt}{0.400pt}}
\put(1162,378.67){\rule{1.204pt}{0.400pt}}
\multiput(1162.00,379.17)(2.500,-1.000){2}{\rule{0.602pt}{0.400pt}}
\put(1167,377.67){\rule{0.964pt}{0.400pt}}
\multiput(1167.00,378.17)(2.000,-1.000){2}{\rule{0.482pt}{0.400pt}}
\put(1171,376.17){\rule{1.100pt}{0.400pt}}
\multiput(1171.00,377.17)(2.717,-2.000){2}{\rule{0.550pt}{0.400pt}}
\put(1176,374.67){\rule{1.204pt}{0.400pt}}
\multiput(1176.00,375.17)(2.500,-1.000){2}{\rule{0.602pt}{0.400pt}}
\put(1181,373.17){\rule{0.900pt}{0.400pt}}
\multiput(1181.00,374.17)(2.132,-2.000){2}{\rule{0.450pt}{0.400pt}}
\put(1185,371.67){\rule{1.204pt}{0.400pt}}
\multiput(1185.00,372.17)(2.500,-1.000){2}{\rule{0.602pt}{0.400pt}}
\put(1190,370.17){\rule{1.100pt}{0.400pt}}
\multiput(1190.00,371.17)(2.717,-2.000){2}{\rule{0.550pt}{0.400pt}}
\put(1195,368.67){\rule{0.964pt}{0.400pt}}
\multiput(1195.00,369.17)(2.000,-1.000){2}{\rule{0.482pt}{0.400pt}}
\put(1199,367.17){\rule{1.100pt}{0.400pt}}
\multiput(1199.00,368.17)(2.717,-2.000){2}{\rule{0.550pt}{0.400pt}}
\put(1204,365.67){\rule{1.204pt}{0.400pt}}
\multiput(1204.00,366.17)(2.500,-1.000){2}{\rule{0.602pt}{0.400pt}}
\put(1209,364.67){\rule{0.964pt}{0.400pt}}
\multiput(1209.00,365.17)(2.000,-1.000){2}{\rule{0.482pt}{0.400pt}}
\put(1213,363.17){\rule{1.100pt}{0.400pt}}
\multiput(1213.00,364.17)(2.717,-2.000){2}{\rule{0.550pt}{0.400pt}}
\put(1218,361.67){\rule{1.204pt}{0.400pt}}
\multiput(1218.00,362.17)(2.500,-1.000){2}{\rule{0.602pt}{0.400pt}}
\put(1223,360.17){\rule{0.900pt}{0.400pt}}
\multiput(1223.00,361.17)(2.132,-2.000){2}{\rule{0.450pt}{0.400pt}}
\put(1227,358.67){\rule{1.204pt}{0.400pt}}
\multiput(1227.00,359.17)(2.500,-1.000){2}{\rule{0.602pt}{0.400pt}}
\put(1232,357.17){\rule{1.100pt}{0.400pt}}
\multiput(1232.00,358.17)(2.717,-2.000){2}{\rule{0.550pt}{0.400pt}}
\put(1237,355.67){\rule{0.964pt}{0.400pt}}
\multiput(1237.00,356.17)(2.000,-1.000){2}{\rule{0.482pt}{0.400pt}}
\put(1241,354.17){\rule{1.100pt}{0.400pt}}
\multiput(1241.00,355.17)(2.717,-2.000){2}{\rule{0.550pt}{0.400pt}}
\put(1246,352.67){\rule{1.204pt}{0.400pt}}
\multiput(1246.00,353.17)(2.500,-1.000){2}{\rule{0.602pt}{0.400pt}}
\put(1251,351.17){\rule{0.900pt}{0.400pt}}
\multiput(1251.00,352.17)(2.132,-2.000){2}{\rule{0.450pt}{0.400pt}}
\put(1255,349.67){\rule{1.204pt}{0.400pt}}
\multiput(1255.00,350.17)(2.500,-1.000){2}{\rule{0.602pt}{0.400pt}}
\put(1260,348.67){\rule{1.204pt}{0.400pt}}
\multiput(1260.00,349.17)(2.500,-1.000){2}{\rule{0.602pt}{0.400pt}}
\put(1265,347.17){\rule{1.100pt}{0.400pt}}
\multiput(1265.00,348.17)(2.717,-2.000){2}{\rule{0.550pt}{0.400pt}}
\put(1270,345.67){\rule{0.964pt}{0.400pt}}
\multiput(1270.00,346.17)(2.000,-1.000){2}{\rule{0.482pt}{0.400pt}}
\put(1274,344.17){\rule{1.100pt}{0.400pt}}
\multiput(1274.00,345.17)(2.717,-2.000){2}{\rule{0.550pt}{0.400pt}}
\put(1279,342.67){\rule{1.204pt}{0.400pt}}
\multiput(1279.00,343.17)(2.500,-1.000){2}{\rule{0.602pt}{0.400pt}}
\put(1284,341.17){\rule{0.900pt}{0.400pt}}
\multiput(1284.00,342.17)(2.132,-2.000){2}{\rule{0.450pt}{0.400pt}}
\put(1288,339.67){\rule{1.204pt}{0.400pt}}
\multiput(1288.00,340.17)(2.500,-1.000){2}{\rule{0.602pt}{0.400pt}}
\put(1293,338.17){\rule{1.100pt}{0.400pt}}
\multiput(1293.00,339.17)(2.717,-2.000){2}{\rule{0.550pt}{0.400pt}}
\put(1298,336.67){\rule{0.964pt}{0.400pt}}
\multiput(1298.00,337.17)(2.000,-1.000){2}{\rule{0.482pt}{0.400pt}}
\put(1302,335.17){\rule{1.100pt}{0.400pt}}
\multiput(1302.00,336.17)(2.717,-2.000){2}{\rule{0.550pt}{0.400pt}}
\put(1307,333.67){\rule{1.204pt}{0.400pt}}
\multiput(1307.00,334.17)(2.500,-1.000){2}{\rule{0.602pt}{0.400pt}}
\put(1312,332.67){\rule{0.964pt}{0.400pt}}
\multiput(1312.00,333.17)(2.000,-1.000){2}{\rule{0.482pt}{0.400pt}}
\put(1316,331.17){\rule{1.100pt}{0.400pt}}
\multiput(1316.00,332.17)(2.717,-2.000){2}{\rule{0.550pt}{0.400pt}}
\put(1321,329.67){\rule{1.204pt}{0.400pt}}
\multiput(1321.00,330.17)(2.500,-1.000){2}{\rule{0.602pt}{0.400pt}}
\put(1326,328.17){\rule{0.900pt}{0.400pt}}
\multiput(1326.00,329.17)(2.132,-2.000){2}{\rule{0.450pt}{0.400pt}}
\put(1330,326.67){\rule{1.204pt}{0.400pt}}
\multiput(1330.00,327.17)(2.500,-1.000){2}{\rule{0.602pt}{0.400pt}}
\put(1335,325.17){\rule{1.100pt}{0.400pt}}
\multiput(1335.00,326.17)(2.717,-2.000){2}{\rule{0.550pt}{0.400pt}}
\put(1340,323.67){\rule{0.964pt}{0.400pt}}
\multiput(1340.00,324.17)(2.000,-1.000){2}{\rule{0.482pt}{0.400pt}}
\put(1344,322.17){\rule{1.100pt}{0.400pt}}
\multiput(1344.00,323.17)(2.717,-2.000){2}{\rule{0.550pt}{0.400pt}}
\put(1349,320.67){\rule{1.204pt}{0.400pt}}
\multiput(1349.00,321.17)(2.500,-1.000){2}{\rule{0.602pt}{0.400pt}}
\put(1354,319.67){\rule{0.964pt}{0.400pt}}
\multiput(1354.00,320.17)(2.000,-1.000){2}{\rule{0.482pt}{0.400pt}}
\put(1358,318.17){\rule{1.100pt}{0.400pt}}
\multiput(1358.00,319.17)(2.717,-2.000){2}{\rule{0.550pt}{0.400pt}}
\put(1363,316.67){\rule{1.204pt}{0.400pt}}
\multiput(1363.00,317.17)(2.500,-1.000){2}{\rule{0.602pt}{0.400pt}}
\put(1368,315.17){\rule{0.900pt}{0.400pt}}
\multiput(1368.00,316.17)(2.132,-2.000){2}{\rule{0.450pt}{0.400pt}}
\put(1372,313.67){\rule{1.204pt}{0.400pt}}
\multiput(1372.00,314.17)(2.500,-1.000){2}{\rule{0.602pt}{0.400pt}}
\put(1377,312.17){\rule{1.100pt}{0.400pt}}
\multiput(1377.00,313.17)(2.717,-2.000){2}{\rule{0.550pt}{0.400pt}}
\put(1382,310.67){\rule{1.204pt}{0.400pt}}
\multiput(1382.00,311.17)(2.500,-1.000){2}{\rule{0.602pt}{0.400pt}}
\put(1387,309.17){\rule{0.900pt}{0.400pt}}
\multiput(1387.00,310.17)(2.132,-2.000){2}{\rule{0.450pt}{0.400pt}}
\put(1391,307.67){\rule{1.204pt}{0.400pt}}
\multiput(1391.00,308.17)(2.500,-1.000){2}{\rule{0.602pt}{0.400pt}}
\put(1396,306.17){\rule{1.100pt}{0.400pt}}
\multiput(1396.00,307.17)(2.717,-2.000){2}{\rule{0.550pt}{0.400pt}}
\put(1401,304.67){\rule{0.964pt}{0.400pt}}
\multiput(1401.00,305.17)(2.000,-1.000){2}{\rule{0.482pt}{0.400pt}}
\put(1405,303.67){\rule{1.204pt}{0.400pt}}
\multiput(1405.00,304.17)(2.500,-1.000){2}{\rule{0.602pt}{0.400pt}}
\put(1410,302.17){\rule{1.100pt}{0.400pt}}
\multiput(1410.00,303.17)(2.717,-2.000){2}{\rule{0.550pt}{0.400pt}}
\put(1410,300.67){\rule{1.204pt}{0.400pt}}
\multiput(1412.50,301.17)(-2.500,-1.000){2}{\rule{0.602pt}{0.400pt}}
\put(1405,299.17){\rule{1.100pt}{0.400pt}}
\multiput(1407.72,300.17)(-2.717,-2.000){2}{\rule{0.550pt}{0.400pt}}
\put(1401,297.67){\rule{0.964pt}{0.400pt}}
\multiput(1403.00,298.17)(-2.000,-1.000){2}{\rule{0.482pt}{0.400pt}}
\put(1396,296.17){\rule{1.100pt}{0.400pt}}
\multiput(1398.72,297.17)(-2.717,-2.000){2}{\rule{0.550pt}{0.400pt}}
\put(1391,294.67){\rule{1.204pt}{0.400pt}}
\multiput(1393.50,295.17)(-2.500,-1.000){2}{\rule{0.602pt}{0.400pt}}
\put(1387,293.17){\rule{0.900pt}{0.400pt}}
\multiput(1389.13,294.17)(-2.132,-2.000){2}{\rule{0.450pt}{0.400pt}}
\put(1382,291.67){\rule{1.204pt}{0.400pt}}
\multiput(1384.50,292.17)(-2.500,-1.000){2}{\rule{0.602pt}{0.400pt}}
\put(1377,290.67){\rule{1.204pt}{0.400pt}}
\multiput(1379.50,291.17)(-2.500,-1.000){2}{\rule{0.602pt}{0.400pt}}
\put(1372,289.17){\rule{1.100pt}{0.400pt}}
\multiput(1374.72,290.17)(-2.717,-2.000){2}{\rule{0.550pt}{0.400pt}}
\put(1368,287.67){\rule{0.964pt}{0.400pt}}
\multiput(1370.00,288.17)(-2.000,-1.000){2}{\rule{0.482pt}{0.400pt}}
\put(1363,286.17){\rule{1.100pt}{0.400pt}}
\multiput(1365.72,287.17)(-2.717,-2.000){2}{\rule{0.550pt}{0.400pt}}
\put(1358,284.67){\rule{1.204pt}{0.400pt}}
\multiput(1360.50,285.17)(-2.500,-1.000){2}{\rule{0.602pt}{0.400pt}}
\put(1354,283.17){\rule{0.900pt}{0.400pt}}
\multiput(1356.13,284.17)(-2.132,-2.000){2}{\rule{0.450pt}{0.400pt}}
\put(1349,281.67){\rule{1.204pt}{0.400pt}}
\multiput(1351.50,282.17)(-2.500,-1.000){2}{\rule{0.602pt}{0.400pt}}
\put(1344,280.17){\rule{1.100pt}{0.400pt}}
\multiput(1346.72,281.17)(-2.717,-2.000){2}{\rule{0.550pt}{0.400pt}}
\put(1340,278.67){\rule{0.964pt}{0.400pt}}
\multiput(1342.00,279.17)(-2.000,-1.000){2}{\rule{0.482pt}{0.400pt}}
\put(1335,277.17){\rule{1.100pt}{0.400pt}}
\multiput(1337.72,278.17)(-2.717,-2.000){2}{\rule{0.550pt}{0.400pt}}
\put(1330,275.67){\rule{1.204pt}{0.400pt}}
\multiput(1332.50,276.17)(-2.500,-1.000){2}{\rule{0.602pt}{0.400pt}}
\put(1326,274.67){\rule{0.964pt}{0.400pt}}
\multiput(1328.00,275.17)(-2.000,-1.000){2}{\rule{0.482pt}{0.400pt}}
\put(1321,273.17){\rule{1.100pt}{0.400pt}}
\multiput(1323.72,274.17)(-2.717,-2.000){2}{\rule{0.550pt}{0.400pt}}
\put(1316,271.67){\rule{1.204pt}{0.400pt}}
\multiput(1318.50,272.17)(-2.500,-1.000){2}{\rule{0.602pt}{0.400pt}}
\put(1312,270.17){\rule{0.900pt}{0.400pt}}
\multiput(1314.13,271.17)(-2.132,-2.000){2}{\rule{0.450pt}{0.400pt}}
\put(1307,268.67){\rule{1.204pt}{0.400pt}}
\multiput(1309.50,269.17)(-2.500,-1.000){2}{\rule{0.602pt}{0.400pt}}
\put(1302,267.17){\rule{1.100pt}{0.400pt}}
\multiput(1304.72,268.17)(-2.717,-2.000){2}{\rule{0.550pt}{0.400pt}}
\put(1298,265.67){\rule{0.964pt}{0.400pt}}
\multiput(1300.00,266.17)(-2.000,-1.000){2}{\rule{0.482pt}{0.400pt}}
\put(1293,264.17){\rule{1.100pt}{0.400pt}}
\multiput(1295.72,265.17)(-2.717,-2.000){2}{\rule{0.550pt}{0.400pt}}
\put(1288,262.67){\rule{1.204pt}{0.400pt}}
\multiput(1290.50,263.17)(-2.500,-1.000){2}{\rule{0.602pt}{0.400pt}}
\put(1284,261.17){\rule{0.900pt}{0.400pt}}
\multiput(1286.13,262.17)(-2.132,-2.000){2}{\rule{0.450pt}{0.400pt}}
\put(1279,259.67){\rule{1.204pt}{0.400pt}}
\multiput(1281.50,260.17)(-2.500,-1.000){2}{\rule{0.602pt}{0.400pt}}
\put(1274,258.67){\rule{1.204pt}{0.400pt}}
\multiput(1276.50,259.17)(-2.500,-1.000){2}{\rule{0.602pt}{0.400pt}}
\put(1270,257.17){\rule{0.900pt}{0.400pt}}
\multiput(1272.13,258.17)(-2.132,-2.000){2}{\rule{0.450pt}{0.400pt}}
\put(1265,255.67){\rule{1.204pt}{0.400pt}}
\multiput(1267.50,256.17)(-2.500,-1.000){2}{\rule{0.602pt}{0.400pt}}
\put(1260,254.17){\rule{1.100pt}{0.400pt}}
\multiput(1262.72,255.17)(-2.717,-2.000){2}{\rule{0.550pt}{0.400pt}}
\put(1255,252.67){\rule{1.204pt}{0.400pt}}
\multiput(1257.50,253.17)(-2.500,-1.000){2}{\rule{0.602pt}{0.400pt}}
\put(1251,251.17){\rule{0.900pt}{0.400pt}}
\multiput(1253.13,252.17)(-2.132,-2.000){2}{\rule{0.450pt}{0.400pt}}
\put(1246,249.67){\rule{1.204pt}{0.400pt}}
\multiput(1248.50,250.17)(-2.500,-1.000){2}{\rule{0.602pt}{0.400pt}}
\put(1241,248.17){\rule{1.100pt}{0.400pt}}
\multiput(1243.72,249.17)(-2.717,-2.000){2}{\rule{0.550pt}{0.400pt}}
\put(1237,246.67){\rule{0.964pt}{0.400pt}}
\multiput(1239.00,247.17)(-2.000,-1.000){2}{\rule{0.482pt}{0.400pt}}
\put(1232,245.67){\rule{1.204pt}{0.400pt}}
\multiput(1234.50,246.17)(-2.500,-1.000){2}{\rule{0.602pt}{0.400pt}}
\put(1227,244.17){\rule{1.100pt}{0.400pt}}
\multiput(1229.72,245.17)(-2.717,-2.000){2}{\rule{0.550pt}{0.400pt}}
\put(1223,242.67){\rule{0.964pt}{0.400pt}}
\multiput(1225.00,243.17)(-2.000,-1.000){2}{\rule{0.482pt}{0.400pt}}
\put(1218,241.17){\rule{1.100pt}{0.400pt}}
\multiput(1220.72,242.17)(-2.717,-2.000){2}{\rule{0.550pt}{0.400pt}}
\put(1213,239.67){\rule{1.204pt}{0.400pt}}
\multiput(1215.50,240.17)(-2.500,-1.000){2}{\rule{0.602pt}{0.400pt}}
\put(1209,238.17){\rule{0.900pt}{0.400pt}}
\multiput(1211.13,239.17)(-2.132,-2.000){2}{\rule{0.450pt}{0.400pt}}
\put(1204,236.67){\rule{1.204pt}{0.400pt}}
\multiput(1206.50,237.17)(-2.500,-1.000){2}{\rule{0.602pt}{0.400pt}}
\put(1199,235.17){\rule{1.100pt}{0.400pt}}
\multiput(1201.72,236.17)(-2.717,-2.000){2}{\rule{0.550pt}{0.400pt}}
\put(1195,233.67){\rule{0.964pt}{0.400pt}}
\multiput(1197.00,234.17)(-2.000,-1.000){2}{\rule{0.482pt}{0.400pt}}
\put(1190,232.17){\rule{1.100pt}{0.400pt}}
\multiput(1192.72,233.17)(-2.717,-2.000){2}{\rule{0.550pt}{0.400pt}}
\put(1185,230.67){\rule{1.204pt}{0.400pt}}
\multiput(1187.50,231.17)(-2.500,-1.000){2}{\rule{0.602pt}{0.400pt}}
\put(1181,229.67){\rule{0.964pt}{0.400pt}}
\multiput(1183.00,230.17)(-2.000,-1.000){2}{\rule{0.482pt}{0.400pt}}
\put(1176,228.17){\rule{1.100pt}{0.400pt}}
\multiput(1178.72,229.17)(-2.717,-2.000){2}{\rule{0.550pt}{0.400pt}}
\put(1171,226.67){\rule{1.204pt}{0.400pt}}
\multiput(1173.50,227.17)(-2.500,-1.000){2}{\rule{0.602pt}{0.400pt}}
\put(1167,225.17){\rule{0.900pt}{0.400pt}}
\multiput(1169.13,226.17)(-2.132,-2.000){2}{\rule{0.450pt}{0.400pt}}
\put(1162,223.67){\rule{1.204pt}{0.400pt}}
\multiput(1164.50,224.17)(-2.500,-1.000){2}{\rule{0.602pt}{0.400pt}}
\put(1157,222.17){\rule{1.100pt}{0.400pt}}
\multiput(1159.72,223.17)(-2.717,-2.000){2}{\rule{0.550pt}{0.400pt}}
\put(1152,220.67){\rule{1.204pt}{0.400pt}}
\multiput(1154.50,221.17)(-2.500,-1.000){2}{\rule{0.602pt}{0.400pt}}
\put(1148,219.17){\rule{0.900pt}{0.400pt}}
\multiput(1150.13,220.17)(-2.132,-2.000){2}{\rule{0.450pt}{0.400pt}}
\put(1143,217.67){\rule{1.204pt}{0.400pt}}
\multiput(1145.50,218.17)(-2.500,-1.000){2}{\rule{0.602pt}{0.400pt}}
\put(1138,216.17){\rule{1.100pt}{0.400pt}}
\multiput(1140.72,217.17)(-2.717,-2.000){2}{\rule{0.550pt}{0.400pt}}
\put(1134,214.67){\rule{0.964pt}{0.400pt}}
\multiput(1136.00,215.17)(-2.000,-1.000){2}{\rule{0.482pt}{0.400pt}}
\put(1129,213.67){\rule{1.204pt}{0.400pt}}
\multiput(1131.50,214.17)(-2.500,-1.000){2}{\rule{0.602pt}{0.400pt}}
\put(1124,212.17){\rule{1.100pt}{0.400pt}}
\multiput(1126.72,213.17)(-2.717,-2.000){2}{\rule{0.550pt}{0.400pt}}
\put(1120,210.67){\rule{0.964pt}{0.400pt}}
\multiput(1122.00,211.17)(-2.000,-1.000){2}{\rule{0.482pt}{0.400pt}}
\put(1115,209.17){\rule{1.100pt}{0.400pt}}
\multiput(1117.72,210.17)(-2.717,-2.000){2}{\rule{0.550pt}{0.400pt}}
\put(1110,207.67){\rule{1.204pt}{0.400pt}}
\multiput(1112.50,208.17)(-2.500,-1.000){2}{\rule{0.602pt}{0.400pt}}
\put(1106,206.17){\rule{0.900pt}{0.400pt}}
\multiput(1108.13,207.17)(-2.132,-2.000){2}{\rule{0.450pt}{0.400pt}}
\put(1101,204.67){\rule{1.204pt}{0.400pt}}
\multiput(1103.50,205.17)(-2.500,-1.000){2}{\rule{0.602pt}{0.400pt}}
\put(1096,203.17){\rule{1.100pt}{0.400pt}}
\multiput(1098.72,204.17)(-2.717,-2.000){2}{\rule{0.550pt}{0.400pt}}
\put(1092,201.67){\rule{0.964pt}{0.400pt}}
\multiput(1094.00,202.17)(-2.000,-1.000){2}{\rule{0.482pt}{0.400pt}}
\put(1087,200.67){\rule{1.204pt}{0.400pt}}
\multiput(1089.50,201.17)(-2.500,-1.000){2}{\rule{0.602pt}{0.400pt}}
\put(1082,199.17){\rule{1.100pt}{0.400pt}}
\multiput(1084.72,200.17)(-2.717,-2.000){2}{\rule{0.550pt}{0.400pt}}
\put(1078,197.67){\rule{0.964pt}{0.400pt}}
\multiput(1080.00,198.17)(-2.000,-1.000){2}{\rule{0.482pt}{0.400pt}}
\put(1073,196.17){\rule{1.100pt}{0.400pt}}
\multiput(1075.72,197.17)(-2.717,-2.000){2}{\rule{0.550pt}{0.400pt}}
\put(1068,194.67){\rule{1.204pt}{0.400pt}}
\multiput(1070.50,195.17)(-2.500,-1.000){2}{\rule{0.602pt}{0.400pt}}
\put(1064,193.17){\rule{0.900pt}{0.400pt}}
\multiput(1066.13,194.17)(-2.132,-2.000){2}{\rule{0.450pt}{0.400pt}}
\put(1059,191.67){\rule{1.204pt}{0.400pt}}
\multiput(1061.50,192.17)(-2.500,-1.000){2}{\rule{0.602pt}{0.400pt}}
\put(1054,190.17){\rule{1.100pt}{0.400pt}}
\multiput(1056.72,191.17)(-2.717,-2.000){2}{\rule{0.550pt}{0.400pt}}
\put(1049,188.67){\rule{1.204pt}{0.400pt}}
\multiput(1051.50,189.17)(-2.500,-1.000){2}{\rule{0.602pt}{0.400pt}}
\put(1045,187.17){\rule{0.900pt}{0.400pt}}
\multiput(1047.13,188.17)(-2.132,-2.000){2}{\rule{0.450pt}{0.400pt}}
\put(1040,185.67){\rule{1.204pt}{0.400pt}}
\multiput(1042.50,186.17)(-2.500,-1.000){2}{\rule{0.602pt}{0.400pt}}
\put(1035,184.67){\rule{1.204pt}{0.400pt}}
\multiput(1037.50,185.17)(-2.500,-1.000){2}{\rule{0.602pt}{0.400pt}}
\put(1031,183.17){\rule{0.900pt}{0.400pt}}
\multiput(1033.13,184.17)(-2.132,-2.000){2}{\rule{0.450pt}{0.400pt}}
\put(1026,181.67){\rule{1.204pt}{0.400pt}}
\multiput(1028.50,182.17)(-2.500,-1.000){2}{\rule{0.602pt}{0.400pt}}
\put(1021,180.17){\rule{1.100pt}{0.400pt}}
\multiput(1023.72,181.17)(-2.717,-2.000){2}{\rule{0.550pt}{0.400pt}}
\put(1017,178.67){\rule{0.964pt}{0.400pt}}
\multiput(1019.00,179.17)(-2.000,-1.000){2}{\rule{0.482pt}{0.400pt}}
\put(1012,177.17){\rule{1.100pt}{0.400pt}}
\multiput(1014.72,178.17)(-2.717,-2.000){2}{\rule{0.550pt}{0.400pt}}
\put(1007,175.67){\rule{1.204pt}{0.400pt}}
\multiput(1009.50,176.17)(-2.500,-1.000){2}{\rule{0.602pt}{0.400pt}}
\put(1003,174.17){\rule{0.900pt}{0.400pt}}
\multiput(1005.13,175.17)(-2.132,-2.000){2}{\rule{0.450pt}{0.400pt}}
\put(998,172.67){\rule{1.204pt}{0.400pt}}
\multiput(1000.50,173.17)(-2.500,-1.000){2}{\rule{0.602pt}{0.400pt}}
\put(993,171.17){\rule{1.100pt}{0.400pt}}
\multiput(995.72,172.17)(-2.717,-2.000){2}{\rule{0.550pt}{0.400pt}}
\put(989,169.67){\rule{0.964pt}{0.400pt}}
\multiput(991.00,170.17)(-2.000,-1.000){2}{\rule{0.482pt}{0.400pt}}
\put(984,168.67){\rule{1.204pt}{0.400pt}}
\multiput(986.50,169.17)(-2.500,-1.000){2}{\rule{0.602pt}{0.400pt}}
\put(979,167.17){\rule{1.100pt}{0.400pt}}
\multiput(981.72,168.17)(-2.717,-2.000){2}{\rule{0.550pt}{0.400pt}}
\put(975,165.67){\rule{0.964pt}{0.400pt}}
\multiput(977.00,166.17)(-2.000,-1.000){2}{\rule{0.482pt}{0.400pt}}
\put(970,164.17){\rule{1.100pt}{0.400pt}}
\multiput(972.72,165.17)(-2.717,-2.000){2}{\rule{0.550pt}{0.400pt}}
\put(965,162.67){\rule{1.204pt}{0.400pt}}
\multiput(967.50,163.17)(-2.500,-1.000){2}{\rule{0.602pt}{0.400pt}}
\put(961,161.17){\rule{0.900pt}{0.400pt}}
\multiput(963.13,162.17)(-2.132,-2.000){2}{\rule{0.450pt}{0.400pt}}
\put(956,159.67){\rule{1.204pt}{0.400pt}}
\multiput(958.50,160.17)(-2.500,-1.000){2}{\rule{0.602pt}{0.400pt}}
\put(951,158.17){\rule{1.100pt}{0.400pt}}
\multiput(953.72,159.17)(-2.717,-2.000){2}{\rule{0.550pt}{0.400pt}}
\put(947,156.67){\rule{0.964pt}{0.400pt}}
\multiput(949.00,157.17)(-2.000,-1.000){2}{\rule{0.482pt}{0.400pt}}
\put(942,155.67){\rule{1.204pt}{0.400pt}}
\multiput(944.50,156.17)(-2.500,-1.000){2}{\rule{0.602pt}{0.400pt}}
\put(937,154.17){\rule{1.100pt}{0.400pt}}
\multiput(939.72,155.17)(-2.717,-2.000){2}{\rule{0.550pt}{0.400pt}}
\put(932,152.67){\rule{1.204pt}{0.400pt}}
\multiput(934.50,153.17)(-2.500,-1.000){2}{\rule{0.602pt}{0.400pt}}
\put(928,151.17){\rule{0.900pt}{0.400pt}}
\multiput(930.13,152.17)(-2.132,-2.000){2}{\rule{0.450pt}{0.400pt}}
\put(923,149.67){\rule{1.204pt}{0.400pt}}
\multiput(925.50,150.17)(-2.500,-1.000){2}{\rule{0.602pt}{0.400pt}}
\put(918,148.17){\rule{1.100pt}{0.400pt}}
\multiput(920.72,149.17)(-2.717,-2.000){2}{\rule{0.550pt}{0.400pt}}
\put(914,146.67){\rule{0.964pt}{0.400pt}}
\multiput(916.00,147.17)(-2.000,-1.000){2}{\rule{0.482pt}{0.400pt}}
\put(909,145.17){\rule{1.100pt}{0.400pt}}
\multiput(911.72,146.17)(-2.717,-2.000){2}{\rule{0.550pt}{0.400pt}}
\put(904,143.67){\rule{1.204pt}{0.400pt}}
\multiput(906.50,144.17)(-2.500,-1.000){2}{\rule{0.602pt}{0.400pt}}
\put(900,142.17){\rule{0.900pt}{0.400pt}}
\multiput(902.13,143.17)(-2.132,-2.000){2}{\rule{0.450pt}{0.400pt}}
\put(895,140.67){\rule{1.204pt}{0.400pt}}
\multiput(897.50,141.17)(-2.500,-1.000){2}{\rule{0.602pt}{0.400pt}}
\put(890,139.67){\rule{1.204pt}{0.400pt}}
\multiput(892.50,140.17)(-2.500,-1.000){2}{\rule{0.602pt}{0.400pt}}
\put(886,138.17){\rule{0.900pt}{0.400pt}}
\multiput(888.13,139.17)(-2.132,-2.000){2}{\rule{0.450pt}{0.400pt}}
\put(881,136.67){\rule{1.204pt}{0.400pt}}
\multiput(883.50,137.17)(-2.500,-1.000){2}{\rule{0.602pt}{0.400pt}}
\put(876,135.17){\rule{1.100pt}{0.400pt}}
\multiput(878.72,136.17)(-2.717,-2.000){2}{\rule{0.550pt}{0.400pt}}
\put(872,133.67){\rule{0.964pt}{0.400pt}}
\multiput(874.00,134.17)(-2.000,-1.000){2}{\rule{0.482pt}{0.400pt}}
\put(867,132.17){\rule{1.100pt}{0.400pt}}
\multiput(869.72,133.17)(-2.717,-2.000){2}{\rule{0.550pt}{0.400pt}}
\put(862,130.67){\rule{1.204pt}{0.400pt}}
\multiput(864.50,131.17)(-2.500,-1.000){2}{\rule{0.602pt}{0.400pt}}
\put(858,129.17){\rule{0.900pt}{0.400pt}}
\multiput(860.13,130.17)(-2.132,-2.000){2}{\rule{0.450pt}{0.400pt}}
\put(853,127.67){\rule{1.204pt}{0.400pt}}
\multiput(855.50,128.17)(-2.500,-1.000){2}{\rule{0.602pt}{0.400pt}}
\put(848,126.67){\rule{1.204pt}{0.400pt}}
\multiput(850.50,127.17)(-2.500,-1.000){2}{\rule{0.602pt}{0.400pt}}
\put(844,125.17){\rule{0.900pt}{0.400pt}}
\multiput(846.13,126.17)(-2.132,-2.000){2}{\rule{0.450pt}{0.400pt}}
\put(839,123.67){\rule{1.204pt}{0.400pt}}
\multiput(841.50,124.17)(-2.500,-1.000){2}{\rule{0.602pt}{0.400pt}}
\put(834,122.17){\rule{1.100pt}{0.400pt}}
\multiput(836.72,123.17)(-2.717,-2.000){2}{\rule{0.550pt}{0.400pt}}
\put(829,120.67){\rule{1.204pt}{0.400pt}}
\multiput(831.50,121.17)(-2.500,-1.000){2}{\rule{0.602pt}{0.400pt}}
\put(825,119.17){\rule{0.900pt}{0.400pt}}
\multiput(827.13,120.17)(-2.132,-2.000){2}{\rule{0.450pt}{0.400pt}}
\put(820,117.67){\rule{1.204pt}{0.400pt}}
\multiput(822.50,118.17)(-2.500,-1.000){2}{\rule{0.602pt}{0.400pt}}
\put(815,116.17){\rule{1.100pt}{0.400pt}}
\multiput(817.72,117.17)(-2.717,-2.000){2}{\rule{0.550pt}{0.400pt}}
\put(811,114.67){\rule{0.964pt}{0.400pt}}
\multiput(813.00,115.17)(-2.000,-1.000){2}{\rule{0.482pt}{0.400pt}}
\put(806,113.17){\rule{1.100pt}{0.400pt}}
\multiput(808.72,114.17)(-2.717,-2.000){2}{\rule{0.550pt}{0.400pt}}
\put(801,111.67){\rule{1.204pt}{0.400pt}}
\multiput(803.50,112.17)(-2.500,-1.000){2}{\rule{0.602pt}{0.400pt}}
\put(797,110.67){\rule{0.964pt}{0.400pt}}
\multiput(799.00,111.17)(-2.000,-1.000){2}{\rule{0.482pt}{0.400pt}}
\put(792,109.17){\rule{1.100pt}{0.400pt}}
\multiput(794.72,110.17)(-2.717,-2.000){2}{\rule{0.550pt}{0.400pt}}
\put(787,107.67){\rule{1.204pt}{0.400pt}}
\multiput(789.50,108.17)(-2.500,-1.000){2}{\rule{0.602pt}{0.400pt}}
\put(783,106.17){\rule{0.900pt}{0.400pt}}
\multiput(785.13,107.17)(-2.132,-2.000){2}{\rule{0.450pt}{0.400pt}}
\put(778,104.67){\rule{1.204pt}{0.400pt}}
\multiput(780.50,105.17)(-2.500,-1.000){2}{\rule{0.602pt}{0.400pt}}
\put(773,104.67){\rule{1.204pt}{0.400pt}}
\multiput(775.50,104.17)(-2.500,1.000){2}{\rule{0.602pt}{0.400pt}}
\put(769,106.17){\rule{0.900pt}{0.400pt}}
\multiput(771.13,105.17)(-2.132,2.000){2}{\rule{0.450pt}{0.400pt}}
\put(764,107.67){\rule{1.204pt}{0.400pt}}
\multiput(766.50,107.17)(-2.500,1.000){2}{\rule{0.602pt}{0.400pt}}
\put(759,109.17){\rule{1.100pt}{0.400pt}}
\multiput(761.72,108.17)(-2.717,2.000){2}{\rule{0.550pt}{0.400pt}}
\put(755,110.67){\rule{0.964pt}{0.400pt}}
\multiput(757.00,110.17)(-2.000,1.000){2}{\rule{0.482pt}{0.400pt}}
\put(750,111.67){\rule{1.204pt}{0.400pt}}
\multiput(752.50,111.17)(-2.500,1.000){2}{\rule{0.602pt}{0.400pt}}
\put(745,113.17){\rule{1.100pt}{0.400pt}}
\multiput(747.72,112.17)(-2.717,2.000){2}{\rule{0.550pt}{0.400pt}}
\put(741,114.67){\rule{0.964pt}{0.400pt}}
\multiput(743.00,114.17)(-2.000,1.000){2}{\rule{0.482pt}{0.400pt}}
\put(736,116.17){\rule{1.100pt}{0.400pt}}
\multiput(738.72,115.17)(-2.717,2.000){2}{\rule{0.550pt}{0.400pt}}
\put(731,117.67){\rule{1.204pt}{0.400pt}}
\multiput(733.50,117.17)(-2.500,1.000){2}{\rule{0.602pt}{0.400pt}}
\put(727,119.17){\rule{0.900pt}{0.400pt}}
\multiput(729.13,118.17)(-2.132,2.000){2}{\rule{0.450pt}{0.400pt}}
\put(722,120.67){\rule{1.204pt}{0.400pt}}
\multiput(724.50,120.17)(-2.500,1.000){2}{\rule{0.602pt}{0.400pt}}
\put(717,122.17){\rule{1.100pt}{0.400pt}}
\multiput(719.72,121.17)(-2.717,2.000){2}{\rule{0.550pt}{0.400pt}}
\put(712,123.67){\rule{1.204pt}{0.400pt}}
\multiput(714.50,123.17)(-2.500,1.000){2}{\rule{0.602pt}{0.400pt}}
\put(708,125.17){\rule{0.900pt}{0.400pt}}
\multiput(710.13,124.17)(-2.132,2.000){2}{\rule{0.450pt}{0.400pt}}
\put(703,126.67){\rule{1.204pt}{0.400pt}}
\multiput(705.50,126.17)(-2.500,1.000){2}{\rule{0.602pt}{0.400pt}}
\put(698,127.67){\rule{1.204pt}{0.400pt}}
\multiput(700.50,127.17)(-2.500,1.000){2}{\rule{0.602pt}{0.400pt}}
\put(694,129.17){\rule{0.900pt}{0.400pt}}
\multiput(696.13,128.17)(-2.132,2.000){2}{\rule{0.450pt}{0.400pt}}
\put(689,130.67){\rule{1.204pt}{0.400pt}}
\multiput(691.50,130.17)(-2.500,1.000){2}{\rule{0.602pt}{0.400pt}}
\put(684,132.17){\rule{1.100pt}{0.400pt}}
\multiput(686.72,131.17)(-2.717,2.000){2}{\rule{0.550pt}{0.400pt}}
\put(680,133.67){\rule{0.964pt}{0.400pt}}
\multiput(682.00,133.17)(-2.000,1.000){2}{\rule{0.482pt}{0.400pt}}
\put(675,135.17){\rule{1.100pt}{0.400pt}}
\multiput(677.72,134.17)(-2.717,2.000){2}{\rule{0.550pt}{0.400pt}}
\put(670,136.67){\rule{1.204pt}{0.400pt}}
\multiput(672.50,136.17)(-2.500,1.000){2}{\rule{0.602pt}{0.400pt}}
\put(666,138.17){\rule{0.900pt}{0.400pt}}
\multiput(668.13,137.17)(-2.132,2.000){2}{\rule{0.450pt}{0.400pt}}
\put(661,139.67){\rule{1.204pt}{0.400pt}}
\multiput(663.50,139.17)(-2.500,1.000){2}{\rule{0.602pt}{0.400pt}}
\put(656,140.67){\rule{1.204pt}{0.400pt}}
\multiput(658.50,140.17)(-2.500,1.000){2}{\rule{0.602pt}{0.400pt}}
\put(652,142.17){\rule{0.900pt}{0.400pt}}
\multiput(654.13,141.17)(-2.132,2.000){2}{\rule{0.450pt}{0.400pt}}
\put(647,143.67){\rule{1.204pt}{0.400pt}}
\multiput(649.50,143.17)(-2.500,1.000){2}{\rule{0.602pt}{0.400pt}}
\put(642,145.17){\rule{1.100pt}{0.400pt}}
\multiput(644.72,144.17)(-2.717,2.000){2}{\rule{0.550pt}{0.400pt}}
\put(638,146.67){\rule{0.964pt}{0.400pt}}
\multiput(640.00,146.17)(-2.000,1.000){2}{\rule{0.482pt}{0.400pt}}
\put(633,148.17){\rule{1.100pt}{0.400pt}}
\multiput(635.72,147.17)(-2.717,2.000){2}{\rule{0.550pt}{0.400pt}}
\put(628,149.67){\rule{1.204pt}{0.400pt}}
\multiput(630.50,149.17)(-2.500,1.000){2}{\rule{0.602pt}{0.400pt}}
\put(624,151.17){\rule{0.900pt}{0.400pt}}
\multiput(626.13,150.17)(-2.132,2.000){2}{\rule{0.450pt}{0.400pt}}
\put(619,152.67){\rule{1.204pt}{0.400pt}}
\multiput(621.50,152.17)(-2.500,1.000){2}{\rule{0.602pt}{0.400pt}}
\put(614,154.17){\rule{1.100pt}{0.400pt}}
\multiput(616.72,153.17)(-2.717,2.000){2}{\rule{0.550pt}{0.400pt}}
\put(609,155.67){\rule{1.204pt}{0.400pt}}
\multiput(611.50,155.17)(-2.500,1.000){2}{\rule{0.602pt}{0.400pt}}
\put(605,156.67){\rule{0.964pt}{0.400pt}}
\multiput(607.00,156.17)(-2.000,1.000){2}{\rule{0.482pt}{0.400pt}}
\put(600,158.17){\rule{1.100pt}{0.400pt}}
\multiput(602.72,157.17)(-2.717,2.000){2}{\rule{0.550pt}{0.400pt}}
\put(595,159.67){\rule{1.204pt}{0.400pt}}
\multiput(597.50,159.17)(-2.500,1.000){2}{\rule{0.602pt}{0.400pt}}
\put(591,161.17){\rule{0.900pt}{0.400pt}}
\multiput(593.13,160.17)(-2.132,2.000){2}{\rule{0.450pt}{0.400pt}}
\put(586,162.67){\rule{1.204pt}{0.400pt}}
\multiput(588.50,162.17)(-2.500,1.000){2}{\rule{0.602pt}{0.400pt}}
\put(581,164.17){\rule{1.100pt}{0.400pt}}
\multiput(583.72,163.17)(-2.717,2.000){2}{\rule{0.550pt}{0.400pt}}
\put(577,165.67){\rule{0.964pt}{0.400pt}}
\multiput(579.00,165.17)(-2.000,1.000){2}{\rule{0.482pt}{0.400pt}}
\put(572,167.17){\rule{1.100pt}{0.400pt}}
\multiput(574.72,166.17)(-2.717,2.000){2}{\rule{0.550pt}{0.400pt}}
\put(567,168.67){\rule{1.204pt}{0.400pt}}
\multiput(569.50,168.17)(-2.500,1.000){2}{\rule{0.602pt}{0.400pt}}
\put(563,169.67){\rule{0.964pt}{0.400pt}}
\multiput(565.00,169.17)(-2.000,1.000){2}{\rule{0.482pt}{0.400pt}}
\put(558,171.17){\rule{1.100pt}{0.400pt}}
\multiput(560.72,170.17)(-2.717,2.000){2}{\rule{0.550pt}{0.400pt}}
\put(553,172.67){\rule{1.204pt}{0.400pt}}
\multiput(555.50,172.17)(-2.500,1.000){2}{\rule{0.602pt}{0.400pt}}
\put(549,174.17){\rule{0.900pt}{0.400pt}}
\multiput(551.13,173.17)(-2.132,2.000){2}{\rule{0.450pt}{0.400pt}}
\put(544,175.67){\rule{1.204pt}{0.400pt}}
\multiput(546.50,175.17)(-2.500,1.000){2}{\rule{0.602pt}{0.400pt}}
\put(539,177.17){\rule{1.100pt}{0.400pt}}
\multiput(541.72,176.17)(-2.717,2.000){2}{\rule{0.550pt}{0.400pt}}
\put(535,178.67){\rule{0.964pt}{0.400pt}}
\multiput(537.00,178.17)(-2.000,1.000){2}{\rule{0.482pt}{0.400pt}}
\put(530,180.17){\rule{1.100pt}{0.400pt}}
\multiput(532.72,179.17)(-2.717,2.000){2}{\rule{0.550pt}{0.400pt}}
\put(525,181.67){\rule{1.204pt}{0.400pt}}
\multiput(527.50,181.17)(-2.500,1.000){2}{\rule{0.602pt}{0.400pt}}
\put(521,183.17){\rule{0.900pt}{0.400pt}}
\multiput(523.13,182.17)(-2.132,2.000){2}{\rule{0.450pt}{0.400pt}}
\put(516,184.67){\rule{1.204pt}{0.400pt}}
\multiput(518.50,184.17)(-2.500,1.000){2}{\rule{0.602pt}{0.400pt}}
\put(511,185.67){\rule{1.204pt}{0.400pt}}
\multiput(513.50,185.17)(-2.500,1.000){2}{\rule{0.602pt}{0.400pt}}
\put(507,187.17){\rule{0.900pt}{0.400pt}}
\multiput(509.13,186.17)(-2.132,2.000){2}{\rule{0.450pt}{0.400pt}}
\put(502,188.67){\rule{1.204pt}{0.400pt}}
\multiput(504.50,188.17)(-2.500,1.000){2}{\rule{0.602pt}{0.400pt}}
\put(497,190.17){\rule{1.100pt}{0.400pt}}
\multiput(499.72,189.17)(-2.717,2.000){2}{\rule{0.550pt}{0.400pt}}
\put(492,191.67){\rule{1.204pt}{0.400pt}}
\multiput(494.50,191.17)(-2.500,1.000){2}{\rule{0.602pt}{0.400pt}}
\put(488,193.17){\rule{0.900pt}{0.400pt}}
\multiput(490.13,192.17)(-2.132,2.000){2}{\rule{0.450pt}{0.400pt}}
\put(483,194.67){\rule{1.204pt}{0.400pt}}
\multiput(485.50,194.17)(-2.500,1.000){2}{\rule{0.602pt}{0.400pt}}
\put(478,196.17){\rule{1.100pt}{0.400pt}}
\multiput(480.72,195.17)(-2.717,2.000){2}{\rule{0.550pt}{0.400pt}}
\put(474,197.67){\rule{0.964pt}{0.400pt}}
\multiput(476.00,197.17)(-2.000,1.000){2}{\rule{0.482pt}{0.400pt}}
\put(469,199.17){\rule{1.100pt}{0.400pt}}
\multiput(471.72,198.17)(-2.717,2.000){2}{\rule{0.550pt}{0.400pt}}
\put(464,200.67){\rule{1.204pt}{0.400pt}}
\multiput(466.50,200.17)(-2.500,1.000){2}{\rule{0.602pt}{0.400pt}}
\put(460,201.67){\rule{0.964pt}{0.400pt}}
\multiput(462.00,201.17)(-2.000,1.000){2}{\rule{0.482pt}{0.400pt}}
\put(455,203.17){\rule{1.100pt}{0.400pt}}
\multiput(457.72,202.17)(-2.717,2.000){2}{\rule{0.550pt}{0.400pt}}
\put(450,204.67){\rule{1.204pt}{0.400pt}}
\multiput(452.50,204.17)(-2.500,1.000){2}{\rule{0.602pt}{0.400pt}}
\put(446,206.17){\rule{0.900pt}{0.400pt}}
\multiput(448.13,205.17)(-2.132,2.000){2}{\rule{0.450pt}{0.400pt}}
\put(441,207.67){\rule{1.204pt}{0.400pt}}
\multiput(443.50,207.17)(-2.500,1.000){2}{\rule{0.602pt}{0.400pt}}
\put(436,209.17){\rule{1.100pt}{0.400pt}}
\multiput(438.72,208.17)(-2.717,2.000){2}{\rule{0.550pt}{0.400pt}}
\put(432,210.67){\rule{0.964pt}{0.400pt}}
\multiput(434.00,210.17)(-2.000,1.000){2}{\rule{0.482pt}{0.400pt}}
\put(427,212.17){\rule{1.100pt}{0.400pt}}
\multiput(429.72,211.17)(-2.717,2.000){2}{\rule{0.550pt}{0.400pt}}
\put(422,213.67){\rule{1.204pt}{0.400pt}}
\multiput(424.50,213.17)(-2.500,1.000){2}{\rule{0.602pt}{0.400pt}}
\put(418,214.67){\rule{0.964pt}{0.400pt}}
\multiput(420.00,214.17)(-2.000,1.000){2}{\rule{0.482pt}{0.400pt}}
\put(413,216.17){\rule{1.100pt}{0.400pt}}
\multiput(415.72,215.17)(-2.717,2.000){2}{\rule{0.550pt}{0.400pt}}
\put(408,217.67){\rule{1.204pt}{0.400pt}}
\multiput(410.50,217.17)(-2.500,1.000){2}{\rule{0.602pt}{0.400pt}}
\put(404,219.17){\rule{0.900pt}{0.400pt}}
\multiput(406.13,218.17)(-2.132,2.000){2}{\rule{0.450pt}{0.400pt}}
\put(399,220.67){\rule{1.204pt}{0.400pt}}
\multiput(401.50,220.17)(-2.500,1.000){2}{\rule{0.602pt}{0.400pt}}
\put(394,222.17){\rule{1.100pt}{0.400pt}}
\multiput(396.72,221.17)(-2.717,2.000){2}{\rule{0.550pt}{0.400pt}}
\put(389,223.67){\rule{1.204pt}{0.400pt}}
\multiput(391.50,223.17)(-2.500,1.000){2}{\rule{0.602pt}{0.400pt}}
\put(385,225.17){\rule{0.900pt}{0.400pt}}
\multiput(387.13,224.17)(-2.132,2.000){2}{\rule{0.450pt}{0.400pt}}
\put(380,226.67){\rule{1.204pt}{0.400pt}}
\multiput(382.50,226.17)(-2.500,1.000){2}{\rule{0.602pt}{0.400pt}}
\put(375,228.17){\rule{1.100pt}{0.400pt}}
\multiput(377.72,227.17)(-2.717,2.000){2}{\rule{0.550pt}{0.400pt}}
\put(371,229.67){\rule{0.964pt}{0.400pt}}
\multiput(373.00,229.17)(-2.000,1.000){2}{\rule{0.482pt}{0.400pt}}
\put(366,230.67){\rule{1.204pt}{0.400pt}}
\multiput(368.50,230.17)(-2.500,1.000){2}{\rule{0.602pt}{0.400pt}}
\put(361,232.17){\rule{1.100pt}{0.400pt}}
\multiput(363.72,231.17)(-2.717,2.000){2}{\rule{0.550pt}{0.400pt}}
\put(357,233.67){\rule{0.964pt}{0.400pt}}
\multiput(359.00,233.17)(-2.000,1.000){2}{\rule{0.482pt}{0.400pt}}
\put(352,235.17){\rule{1.100pt}{0.400pt}}
\multiput(354.72,234.17)(-2.717,2.000){2}{\rule{0.550pt}{0.400pt}}
\put(347,236.67){\rule{1.204pt}{0.400pt}}
\multiput(349.50,236.17)(-2.500,1.000){2}{\rule{0.602pt}{0.400pt}}
\put(343,238.17){\rule{0.900pt}{0.400pt}}
\multiput(345.13,237.17)(-2.132,2.000){2}{\rule{0.450pt}{0.400pt}}
\put(338,239.67){\rule{1.204pt}{0.400pt}}
\multiput(340.50,239.17)(-2.500,1.000){2}{\rule{0.602pt}{0.400pt}}
\put(333,241.17){\rule{1.100pt}{0.400pt}}
\multiput(335.72,240.17)(-2.717,2.000){2}{\rule{0.550pt}{0.400pt}}
\put(329,242.67){\rule{0.964pt}{0.400pt}}
\multiput(331.00,242.17)(-2.000,1.000){2}{\rule{0.482pt}{0.400pt}}
\put(324,244.17){\rule{1.100pt}{0.400pt}}
\multiput(326.72,243.17)(-2.717,2.000){2}{\rule{0.550pt}{0.400pt}}
\put(319,245.67){\rule{1.204pt}{0.400pt}}
\multiput(321.50,245.17)(-2.500,1.000){2}{\rule{0.602pt}{0.400pt}}
\put(315,246.67){\rule{0.964pt}{0.400pt}}
\multiput(317.00,246.17)(-2.000,1.000){2}{\rule{0.482pt}{0.400pt}}
\put(310,248.17){\rule{1.100pt}{0.400pt}}
\multiput(312.72,247.17)(-2.717,2.000){2}{\rule{0.550pt}{0.400pt}}
\put(305,249.67){\rule{1.204pt}{0.400pt}}
\multiput(307.50,249.17)(-2.500,1.000){2}{\rule{0.602pt}{0.400pt}}
\put(301,251.17){\rule{0.900pt}{0.400pt}}
\multiput(303.13,250.17)(-2.132,2.000){2}{\rule{0.450pt}{0.400pt}}
\put(296,252.67){\rule{1.204pt}{0.400pt}}
\multiput(298.50,252.17)(-2.500,1.000){2}{\rule{0.602pt}{0.400pt}}
\put(291,254.17){\rule{1.100pt}{0.400pt}}
\multiput(293.72,253.17)(-2.717,2.000){2}{\rule{0.550pt}{0.400pt}}
\put(287,255.67){\rule{0.964pt}{0.400pt}}
\multiput(289.00,255.17)(-2.000,1.000){2}{\rule{0.482pt}{0.400pt}}
\put(282,257.17){\rule{1.100pt}{0.400pt}}
\multiput(284.72,256.17)(-2.717,2.000){2}{\rule{0.550pt}{0.400pt}}
\put(277,258.67){\rule{1.204pt}{0.400pt}}
\multiput(279.50,258.17)(-2.500,1.000){2}{\rule{0.602pt}{0.400pt}}
\put(272,259.67){\rule{1.204pt}{0.400pt}}
\multiput(274.50,259.17)(-2.500,1.000){2}{\rule{0.602pt}{0.400pt}}
\put(268,261.17){\rule{0.900pt}{0.400pt}}
\multiput(270.13,260.17)(-2.132,2.000){2}{\rule{0.450pt}{0.400pt}}
\put(263,262.67){\rule{1.204pt}{0.400pt}}
\multiput(265.50,262.17)(-2.500,1.000){2}{\rule{0.602pt}{0.400pt}}
\put(258,264.17){\rule{1.100pt}{0.400pt}}
\multiput(260.72,263.17)(-2.717,2.000){2}{\rule{0.550pt}{0.400pt}}
\put(254,265.67){\rule{0.964pt}{0.400pt}}
\multiput(256.00,265.17)(-2.000,1.000){2}{\rule{0.482pt}{0.400pt}}
\put(249,267.17){\rule{1.100pt}{0.400pt}}
\multiput(251.72,266.17)(-2.717,2.000){2}{\rule{0.550pt}{0.400pt}}
\put(244,268.67){\rule{1.204pt}{0.400pt}}
\multiput(246.50,268.17)(-2.500,1.000){2}{\rule{0.602pt}{0.400pt}}
\put(240,270.17){\rule{0.900pt}{0.400pt}}
\multiput(242.13,269.17)(-2.132,2.000){2}{\rule{0.450pt}{0.400pt}}
\put(235,271.67){\rule{1.204pt}{0.400pt}}
\multiput(237.50,271.17)(-2.500,1.000){2}{\rule{0.602pt}{0.400pt}}
\put(230,273.17){\rule{1.100pt}{0.400pt}}
\multiput(232.72,272.17)(-2.717,2.000){2}{\rule{0.550pt}{0.400pt}}
\put(226,274.67){\rule{0.964pt}{0.400pt}}
\multiput(228.00,274.17)(-2.000,1.000){2}{\rule{0.482pt}{0.400pt}}
\put(221,275.67){\rule{1.204pt}{0.400pt}}
\multiput(223.50,275.17)(-2.500,1.000){2}{\rule{0.602pt}{0.400pt}}
\put(216,277.17){\rule{1.100pt}{0.400pt}}
\multiput(218.72,276.17)(-2.717,2.000){2}{\rule{0.550pt}{0.400pt}}
\put(212,278.67){\rule{0.964pt}{0.400pt}}
\multiput(214.00,278.17)(-2.000,1.000){2}{\rule{0.482pt}{0.400pt}}
\put(207,280.17){\rule{1.100pt}{0.400pt}}
\multiput(209.72,279.17)(-2.717,2.000){2}{\rule{0.550pt}{0.400pt}}
\put(202,281.67){\rule{1.204pt}{0.400pt}}
\multiput(204.50,281.17)(-2.500,1.000){2}{\rule{0.602pt}{0.400pt}}
\put(198,283.17){\rule{0.900pt}{0.400pt}}
\multiput(200.13,282.17)(-2.132,2.000){2}{\rule{0.450pt}{0.400pt}}
\put(193,284.67){\rule{1.204pt}{0.400pt}}
\multiput(195.50,284.17)(-2.500,1.000){2}{\rule{0.602pt}{0.400pt}}
\put(188,286.17){\rule{1.100pt}{0.400pt}}
\multiput(190.72,285.17)(-2.717,2.000){2}{\rule{0.550pt}{0.400pt}}
\put(184,287.67){\rule{0.964pt}{0.400pt}}
\multiput(186.00,287.17)(-2.000,1.000){2}{\rule{0.482pt}{0.400pt}}
\put(179,289.17){\rule{1.100pt}{0.400pt}}
\multiput(181.72,288.17)(-2.717,2.000){2}{\rule{0.550pt}{0.400pt}}
\put(174,290.67){\rule{1.204pt}{0.400pt}}
\multiput(176.50,290.17)(-2.500,1.000){2}{\rule{0.602pt}{0.400pt}}
\put(169,291.67){\rule{1.204pt}{0.400pt}}
\multiput(171.50,291.17)(-2.500,1.000){2}{\rule{0.602pt}{0.400pt}}
\put(165,293.17){\rule{0.900pt}{0.400pt}}
\multiput(167.13,292.17)(-2.132,2.000){2}{\rule{0.450pt}{0.400pt}}
\put(160,294.67){\rule{1.204pt}{0.400pt}}
\multiput(162.50,294.17)(-2.500,1.000){2}{\rule{0.602pt}{0.400pt}}
\put(155,296.17){\rule{1.100pt}{0.400pt}}
\multiput(157.72,295.17)(-2.717,2.000){2}{\rule{0.550pt}{0.400pt}}
\put(155,297.67){\rule{1.204pt}{0.400pt}}
\multiput(155.00,297.17)(2.500,1.000){2}{\rule{0.602pt}{0.400pt}}
\put(160,299.17){\rule{1.100pt}{0.400pt}}
\multiput(160.00,298.17)(2.717,2.000){2}{\rule{0.550pt}{0.400pt}}
\put(165,300.67){\rule{0.964pt}{0.400pt}}
\multiput(165.00,300.17)(2.000,1.000){2}{\rule{0.482pt}{0.400pt}}
\put(169,302.17){\rule{1.100pt}{0.400pt}}
\multiput(169.00,301.17)(2.717,2.000){2}{\rule{0.550pt}{0.400pt}}
\put(174,303.67){\rule{1.204pt}{0.400pt}}
\multiput(174.00,303.17)(2.500,1.000){2}{\rule{0.602pt}{0.400pt}}
\put(179,304.67){\rule{1.204pt}{0.400pt}}
\multiput(179.00,304.17)(2.500,1.000){2}{\rule{0.602pt}{0.400pt}}
\put(184,306.17){\rule{0.900pt}{0.400pt}}
\multiput(184.00,305.17)(2.132,2.000){2}{\rule{0.450pt}{0.400pt}}
\put(188,307.67){\rule{1.204pt}{0.400pt}}
\multiput(188.00,307.17)(2.500,1.000){2}{\rule{0.602pt}{0.400pt}}
\put(193,309.17){\rule{1.100pt}{0.400pt}}
\multiput(193.00,308.17)(2.717,2.000){2}{\rule{0.550pt}{0.400pt}}
\put(198,310.67){\rule{0.964pt}{0.400pt}}
\multiput(198.00,310.17)(2.000,1.000){2}{\rule{0.482pt}{0.400pt}}
\put(202,312.17){\rule{1.100pt}{0.400pt}}
\multiput(202.00,311.17)(2.717,2.000){2}{\rule{0.550pt}{0.400pt}}
\put(207,313.67){\rule{1.204pt}{0.400pt}}
\multiput(207.00,313.17)(2.500,1.000){2}{\rule{0.602pt}{0.400pt}}
\put(212,315.17){\rule{0.900pt}{0.400pt}}
\multiput(212.00,314.17)(2.132,2.000){2}{\rule{0.450pt}{0.400pt}}
\put(216,316.67){\rule{1.204pt}{0.400pt}}
\multiput(216.00,316.17)(2.500,1.000){2}{\rule{0.602pt}{0.400pt}}
\put(221,318.17){\rule{1.100pt}{0.400pt}}
\multiput(221.00,317.17)(2.717,2.000){2}{\rule{0.550pt}{0.400pt}}
\put(226,319.67){\rule{0.964pt}{0.400pt}}
\multiput(226.00,319.17)(2.000,1.000){2}{\rule{0.482pt}{0.400pt}}
\put(230,320.67){\rule{1.204pt}{0.400pt}}
\multiput(230.00,320.17)(2.500,1.000){2}{\rule{0.602pt}{0.400pt}}
\put(235,322.17){\rule{1.100pt}{0.400pt}}
\multiput(235.00,321.17)(2.717,2.000){2}{\rule{0.550pt}{0.400pt}}
\put(240,323.67){\rule{0.964pt}{0.400pt}}
\multiput(240.00,323.17)(2.000,1.000){2}{\rule{0.482pt}{0.400pt}}
\put(244,325.17){\rule{1.100pt}{0.400pt}}
\multiput(244.00,324.17)(2.717,2.000){2}{\rule{0.550pt}{0.400pt}}
\put(249,326.67){\rule{1.204pt}{0.400pt}}
\multiput(249.00,326.17)(2.500,1.000){2}{\rule{0.602pt}{0.400pt}}
\put(254,328.17){\rule{0.900pt}{0.400pt}}
\multiput(254.00,327.17)(2.132,2.000){2}{\rule{0.450pt}{0.400pt}}
\put(258,329.67){\rule{1.204pt}{0.400pt}}
\multiput(258.00,329.17)(2.500,1.000){2}{\rule{0.602pt}{0.400pt}}
\put(263,331.17){\rule{1.100pt}{0.400pt}}
\multiput(263.00,330.17)(2.717,2.000){2}{\rule{0.550pt}{0.400pt}}
\put(268,332.67){\rule{0.964pt}{0.400pt}}
\multiput(268.00,332.17)(2.000,1.000){2}{\rule{0.482pt}{0.400pt}}
\put(272,333.67){\rule{1.204pt}{0.400pt}}
\multiput(272.00,333.17)(2.500,1.000){2}{\rule{0.602pt}{0.400pt}}
\put(277,335.17){\rule{1.100pt}{0.400pt}}
\multiput(277.00,334.17)(2.717,2.000){2}{\rule{0.550pt}{0.400pt}}
\put(282,336.67){\rule{1.204pt}{0.400pt}}
\multiput(282.00,336.17)(2.500,1.000){2}{\rule{0.602pt}{0.400pt}}
\put(287,338.17){\rule{0.900pt}{0.400pt}}
\multiput(287.00,337.17)(2.132,2.000){2}{\rule{0.450pt}{0.400pt}}
\put(291,339.67){\rule{1.204pt}{0.400pt}}
\multiput(291.00,339.17)(2.500,1.000){2}{\rule{0.602pt}{0.400pt}}
\put(296,341.17){\rule{1.100pt}{0.400pt}}
\multiput(296.00,340.17)(2.717,2.000){2}{\rule{0.550pt}{0.400pt}}
\put(301,342.67){\rule{0.964pt}{0.400pt}}
\multiput(301.00,342.17)(2.000,1.000){2}{\rule{0.482pt}{0.400pt}}
\put(305,344.17){\rule{1.100pt}{0.400pt}}
\multiput(305.00,343.17)(2.717,2.000){2}{\rule{0.550pt}{0.400pt}}
\put(310,345.67){\rule{1.204pt}{0.400pt}}
\multiput(310.00,345.17)(2.500,1.000){2}{\rule{0.602pt}{0.400pt}}
\put(315,347.17){\rule{0.900pt}{0.400pt}}
\multiput(315.00,346.17)(2.132,2.000){2}{\rule{0.450pt}{0.400pt}}
\put(319,348.67){\rule{1.204pt}{0.400pt}}
\multiput(319.00,348.17)(2.500,1.000){2}{\rule{0.602pt}{0.400pt}}
\put(324,349.67){\rule{1.204pt}{0.400pt}}
\multiput(324.00,349.17)(2.500,1.000){2}{\rule{0.602pt}{0.400pt}}
\put(329,351.17){\rule{0.900pt}{0.400pt}}
\multiput(329.00,350.17)(2.132,2.000){2}{\rule{0.450pt}{0.400pt}}
\put(333,352.67){\rule{1.204pt}{0.400pt}}
\multiput(333.00,352.17)(2.500,1.000){2}{\rule{0.602pt}{0.400pt}}
\put(338,354.17){\rule{1.100pt}{0.400pt}}
\multiput(338.00,353.17)(2.717,2.000){2}{\rule{0.550pt}{0.400pt}}
\put(343,355.67){\rule{0.964pt}{0.400pt}}
\multiput(343.00,355.17)(2.000,1.000){2}{\rule{0.482pt}{0.400pt}}
\put(347,357.17){\rule{1.100pt}{0.400pt}}
\multiput(347.00,356.17)(2.717,2.000){2}{\rule{0.550pt}{0.400pt}}
\put(352,358.67){\rule{1.204pt}{0.400pt}}
\multiput(352.00,358.17)(2.500,1.000){2}{\rule{0.602pt}{0.400pt}}
\put(357,360.17){\rule{0.900pt}{0.400pt}}
\multiput(357.00,359.17)(2.132,2.000){2}{\rule{0.450pt}{0.400pt}}
\put(361,361.67){\rule{1.204pt}{0.400pt}}
\multiput(361.00,361.17)(2.500,1.000){2}{\rule{0.602pt}{0.400pt}}
\put(366,363.17){\rule{1.100pt}{0.400pt}}
\multiput(366.00,362.17)(2.717,2.000){2}{\rule{0.550pt}{0.400pt}}
\put(371,364.67){\rule{0.964pt}{0.400pt}}
\multiput(371.00,364.17)(2.000,1.000){2}{\rule{0.482pt}{0.400pt}}
\put(375,365.67){\rule{1.204pt}{0.400pt}}
\multiput(375.00,365.17)(2.500,1.000){2}{\rule{0.602pt}{0.400pt}}
\put(380,367.17){\rule{1.100pt}{0.400pt}}
\multiput(380.00,366.17)(2.717,2.000){2}{\rule{0.550pt}{0.400pt}}
\put(385,368.67){\rule{0.964pt}{0.400pt}}
\multiput(385.00,368.17)(2.000,1.000){2}{\rule{0.482pt}{0.400pt}}
\put(389,370.17){\rule{1.100pt}{0.400pt}}
\multiput(389.00,369.17)(2.717,2.000){2}{\rule{0.550pt}{0.400pt}}
\put(394,371.67){\rule{1.204pt}{0.400pt}}
\multiput(394.00,371.17)(2.500,1.000){2}{\rule{0.602pt}{0.400pt}}
\put(399,373.17){\rule{1.100pt}{0.400pt}}
\multiput(399.00,372.17)(2.717,2.000){2}{\rule{0.550pt}{0.400pt}}
\put(404,374.67){\rule{0.964pt}{0.400pt}}
\multiput(404.00,374.17)(2.000,1.000){2}{\rule{0.482pt}{0.400pt}}
\put(408,376.17){\rule{1.100pt}{0.400pt}}
\multiput(408.00,375.17)(2.717,2.000){2}{\rule{0.550pt}{0.400pt}}
\put(413,377.67){\rule{1.204pt}{0.400pt}}
\multiput(413.00,377.17)(2.500,1.000){2}{\rule{0.602pt}{0.400pt}}
\put(418,378.67){\rule{0.964pt}{0.400pt}}
\multiput(418.00,378.17)(2.000,1.000){2}{\rule{0.482pt}{0.400pt}}
\put(422,380.17){\rule{1.100pt}{0.400pt}}
\multiput(422.00,379.17)(2.717,2.000){2}{\rule{0.550pt}{0.400pt}}
\put(427,381.67){\rule{1.204pt}{0.400pt}}
\multiput(427.00,381.17)(2.500,1.000){2}{\rule{0.602pt}{0.400pt}}
\put(432,383.17){\rule{0.900pt}{0.400pt}}
\multiput(432.00,382.17)(2.132,2.000){2}{\rule{0.450pt}{0.400pt}}
\put(436,384.67){\rule{1.204pt}{0.400pt}}
\multiput(436.00,384.17)(2.500,1.000){2}{\rule{0.602pt}{0.400pt}}
\put(441,386.17){\rule{1.100pt}{0.400pt}}
\multiput(441.00,385.17)(2.717,2.000){2}{\rule{0.550pt}{0.400pt}}
\put(446,387.67){\rule{0.964pt}{0.400pt}}
\multiput(446.00,387.17)(2.000,1.000){2}{\rule{0.482pt}{0.400pt}}
\put(450,389.17){\rule{1.100pt}{0.400pt}}
\multiput(450.00,388.17)(2.717,2.000){2}{\rule{0.550pt}{0.400pt}}
\put(455,390.67){\rule{1.204pt}{0.400pt}}
\multiput(455.00,390.17)(2.500,1.000){2}{\rule{0.602pt}{0.400pt}}
\put(460,392.17){\rule{0.900pt}{0.400pt}}
\multiput(460.00,391.17)(2.132,2.000){2}{\rule{0.450pt}{0.400pt}}
\put(464,393.67){\rule{1.204pt}{0.400pt}}
\multiput(464.00,393.17)(2.500,1.000){2}{\rule{0.602pt}{0.400pt}}
\put(469,394.67){\rule{1.204pt}{0.400pt}}
\multiput(469.00,394.17)(2.500,1.000){2}{\rule{0.602pt}{0.400pt}}
\put(474,396.17){\rule{0.900pt}{0.400pt}}
\multiput(474.00,395.17)(2.132,2.000){2}{\rule{0.450pt}{0.400pt}}
\put(478,397.67){\rule{1.204pt}{0.400pt}}
\multiput(478.00,397.17)(2.500,1.000){2}{\rule{0.602pt}{0.400pt}}
\put(483,399.17){\rule{1.100pt}{0.400pt}}
\multiput(483.00,398.17)(2.717,2.000){2}{\rule{0.550pt}{0.400pt}}
\put(488,400.67){\rule{0.964pt}{0.400pt}}
\multiput(488.00,400.17)(2.000,1.000){2}{\rule{0.482pt}{0.400pt}}
\put(492,402.17){\rule{1.100pt}{0.400pt}}
\multiput(492.00,401.17)(2.717,2.000){2}{\rule{0.550pt}{0.400pt}}
\put(497,403.67){\rule{1.204pt}{0.400pt}}
\multiput(497.00,403.17)(2.500,1.000){2}{\rule{0.602pt}{0.400pt}}
\put(502,405.17){\rule{1.100pt}{0.400pt}}
\multiput(502.00,404.17)(2.717,2.000){2}{\rule{0.550pt}{0.400pt}}
\put(507,406.67){\rule{0.964pt}{0.400pt}}
\multiput(507.00,406.17)(2.000,1.000){2}{\rule{0.482pt}{0.400pt}}
\put(511,408.17){\rule{1.100pt}{0.400pt}}
\multiput(511.00,407.17)(2.717,2.000){2}{\rule{0.550pt}{0.400pt}}
\put(516,409.67){\rule{1.204pt}{0.400pt}}
\multiput(516.00,409.17)(2.500,1.000){2}{\rule{0.602pt}{0.400pt}}
\put(521,410.67){\rule{0.964pt}{0.400pt}}
\multiput(521.00,410.17)(2.000,1.000){2}{\rule{0.482pt}{0.400pt}}
\put(525,412.17){\rule{1.100pt}{0.400pt}}
\multiput(525.00,411.17)(2.717,2.000){2}{\rule{0.550pt}{0.400pt}}
\put(530,413.67){\rule{1.204pt}{0.400pt}}
\multiput(530.00,413.17)(2.500,1.000){2}{\rule{0.602pt}{0.400pt}}
\put(535,415.17){\rule{0.900pt}{0.400pt}}
\multiput(535.00,414.17)(2.132,2.000){2}{\rule{0.450pt}{0.400pt}}
\put(539,416.67){\rule{1.204pt}{0.400pt}}
\multiput(539.00,416.17)(2.500,1.000){2}{\rule{0.602pt}{0.400pt}}
\put(544,418.17){\rule{1.100pt}{0.400pt}}
\multiput(544.00,417.17)(2.717,2.000){2}{\rule{0.550pt}{0.400pt}}
\put(549,419.67){\rule{0.964pt}{0.400pt}}
\multiput(549.00,419.17)(2.000,1.000){2}{\rule{0.482pt}{0.400pt}}
\put(553,421.17){\rule{1.100pt}{0.400pt}}
\multiput(553.00,420.17)(2.717,2.000){2}{\rule{0.550pt}{0.400pt}}
\put(558,422.67){\rule{1.204pt}{0.400pt}}
\multiput(558.00,422.17)(2.500,1.000){2}{\rule{0.602pt}{0.400pt}}
\put(563,423.67){\rule{0.964pt}{0.400pt}}
\multiput(563.00,423.17)(2.000,1.000){2}{\rule{0.482pt}{0.400pt}}
\put(567,425.17){\rule{1.100pt}{0.400pt}}
\multiput(567.00,424.17)(2.717,2.000){2}{\rule{0.550pt}{0.400pt}}
\put(572,426.67){\rule{1.204pt}{0.400pt}}
\multiput(572.00,426.17)(2.500,1.000){2}{\rule{0.602pt}{0.400pt}}
\put(577,428.17){\rule{0.900pt}{0.400pt}}
\multiput(577.00,427.17)(2.132,2.000){2}{\rule{0.450pt}{0.400pt}}
\put(581,429.67){\rule{1.204pt}{0.400pt}}
\multiput(581.00,429.17)(2.500,1.000){2}{\rule{0.602pt}{0.400pt}}
\put(586,431.17){\rule{1.100pt}{0.400pt}}
\multiput(586.00,430.17)(2.717,2.000){2}{\rule{0.550pt}{0.400pt}}
\put(591,432.67){\rule{0.964pt}{0.400pt}}
\multiput(591.00,432.17)(2.000,1.000){2}{\rule{0.482pt}{0.400pt}}
\put(595,434.17){\rule{1.100pt}{0.400pt}}
\multiput(595.00,433.17)(2.717,2.000){2}{\rule{0.550pt}{0.400pt}}
\put(600,435.67){\rule{1.204pt}{0.400pt}}
\multiput(600.00,435.17)(2.500,1.000){2}{\rule{0.602pt}{0.400pt}}
\put(605,437.17){\rule{0.900pt}{0.400pt}}
\multiput(605.00,436.17)(2.132,2.000){2}{\rule{0.450pt}{0.400pt}}
\put(609,438.67){\rule{1.204pt}{0.400pt}}
\multiput(609.00,438.17)(2.500,1.000){2}{\rule{0.602pt}{0.400pt}}
\put(614,439.67){\rule{1.204pt}{0.400pt}}
\multiput(614.00,439.17)(2.500,1.000){2}{\rule{0.602pt}{0.400pt}}
\put(619,441.17){\rule{1.100pt}{0.400pt}}
\multiput(619.00,440.17)(2.717,2.000){2}{\rule{0.550pt}{0.400pt}}
\put(624,442.67){\rule{0.964pt}{0.400pt}}
\multiput(624.00,442.17)(2.000,1.000){2}{\rule{0.482pt}{0.400pt}}
\put(628,444.17){\rule{1.100pt}{0.400pt}}
\multiput(628.00,443.17)(2.717,2.000){2}{\rule{0.550pt}{0.400pt}}
\put(633,445.67){\rule{1.204pt}{0.400pt}}
\multiput(633.00,445.17)(2.500,1.000){2}{\rule{0.602pt}{0.400pt}}
\put(638,447.17){\rule{0.900pt}{0.400pt}}
\multiput(638.00,446.17)(2.132,2.000){2}{\rule{0.450pt}{0.400pt}}
\put(642,448.67){\rule{1.204pt}{0.400pt}}
\multiput(642.00,448.17)(2.500,1.000){2}{\rule{0.602pt}{0.400pt}}
\put(647,450.17){\rule{1.100pt}{0.400pt}}
\multiput(647.00,449.17)(2.717,2.000){2}{\rule{0.550pt}{0.400pt}}
\put(652,451.67){\rule{0.964pt}{0.400pt}}
\multiput(652.00,451.17)(2.000,1.000){2}{\rule{0.482pt}{0.400pt}}
\put(656,452.67){\rule{1.204pt}{0.400pt}}
\multiput(656.00,452.17)(2.500,1.000){2}{\rule{0.602pt}{0.400pt}}
\put(661,454.17){\rule{1.100pt}{0.400pt}}
\multiput(661.00,453.17)(2.717,2.000){2}{\rule{0.550pt}{0.400pt}}
\put(666,455.67){\rule{0.964pt}{0.400pt}}
\multiput(666.00,455.17)(2.000,1.000){2}{\rule{0.482pt}{0.400pt}}
\put(670,457.17){\rule{1.100pt}{0.400pt}}
\multiput(670.00,456.17)(2.717,2.000){2}{\rule{0.550pt}{0.400pt}}
\put(675,458.67){\rule{1.204pt}{0.400pt}}
\multiput(675.00,458.17)(2.500,1.000){2}{\rule{0.602pt}{0.400pt}}
\put(680,460.17){\rule{0.900pt}{0.400pt}}
\multiput(680.00,459.17)(2.132,2.000){2}{\rule{0.450pt}{0.400pt}}
\put(684,461.67){\rule{1.204pt}{0.400pt}}
\multiput(684.00,461.17)(2.500,1.000){2}{\rule{0.602pt}{0.400pt}}
\put(689,463.17){\rule{1.100pt}{0.400pt}}
\multiput(689.00,462.17)(2.717,2.000){2}{\rule{0.550pt}{0.400pt}}
\put(694,464.67){\rule{0.964pt}{0.400pt}}
\multiput(694.00,464.17)(2.000,1.000){2}{\rule{0.482pt}{0.400pt}}
\put(698,466.17){\rule{1.100pt}{0.400pt}}
\multiput(698.00,465.17)(2.717,2.000){2}{\rule{0.550pt}{0.400pt}}
\put(703,467.67){\rule{1.204pt}{0.400pt}}
\multiput(703.00,467.17)(2.500,1.000){2}{\rule{0.602pt}{0.400pt}}
\put(708,468.67){\rule{0.964pt}{0.400pt}}
\multiput(708.00,468.17)(2.000,1.000){2}{\rule{0.482pt}{0.400pt}}
\put(712,470.17){\rule{1.100pt}{0.400pt}}
\multiput(712.00,469.17)(2.717,2.000){2}{\rule{0.550pt}{0.400pt}}
\put(717,471.67){\rule{1.204pt}{0.400pt}}
\multiput(717.00,471.17)(2.500,1.000){2}{\rule{0.602pt}{0.400pt}}
\put(722,473.17){\rule{1.100pt}{0.400pt}}
\multiput(722.00,472.17)(2.717,2.000){2}{\rule{0.550pt}{0.400pt}}
\put(727,474.67){\rule{0.964pt}{0.400pt}}
\multiput(727.00,474.17)(2.000,1.000){2}{\rule{0.482pt}{0.400pt}}
\put(731,476.17){\rule{1.100pt}{0.400pt}}
\multiput(731.00,475.17)(2.717,2.000){2}{\rule{0.550pt}{0.400pt}}
\put(736,477.67){\rule{1.204pt}{0.400pt}}
\multiput(736.00,477.17)(2.500,1.000){2}{\rule{0.602pt}{0.400pt}}
\put(741,479.17){\rule{0.900pt}{0.400pt}}
\multiput(741.00,478.17)(2.132,2.000){2}{\rule{0.450pt}{0.400pt}}
\put(745,480.67){\rule{1.204pt}{0.400pt}}
\multiput(745.00,480.17)(2.500,1.000){2}{\rule{0.602pt}{0.400pt}}
\put(750,482.17){\rule{1.100pt}{0.400pt}}
\multiput(750.00,481.17)(2.717,2.000){2}{\rule{0.550pt}{0.400pt}}
\put(755,483.67){\rule{0.964pt}{0.400pt}}
\multiput(755.00,483.17)(2.000,1.000){2}{\rule{0.482pt}{0.400pt}}
\put(759,484.67){\rule{1.204pt}{0.400pt}}
\multiput(759.00,484.17)(2.500,1.000){2}{\rule{0.602pt}{0.400pt}}
\put(764,486.17){\rule{1.100pt}{0.400pt}}
\multiput(764.00,485.17)(2.717,2.000){2}{\rule{0.550pt}{0.400pt}}
\put(769,487.67){\rule{0.964pt}{0.400pt}}
\multiput(769.00,487.17)(2.000,1.000){2}{\rule{0.482pt}{0.400pt}}
\put(773,489.17){\rule{1.100pt}{0.400pt}}
\multiput(773.00,488.17)(2.717,2.000){2}{\rule{0.550pt}{0.400pt}}
\put(778,490.67){\rule{1.204pt}{0.400pt}}
\multiput(778.00,490.17)(2.500,1.000){2}{\rule{0.602pt}{0.400pt}}
\put(783,492.17){\rule{0.900pt}{0.400pt}}
\multiput(783.00,491.17)(2.132,2.000){2}{\rule{0.450pt}{0.400pt}}
\put(787,493.67){\rule{1.204pt}{0.400pt}}
\multiput(787.00,493.17)(2.500,1.000){2}{\rule{0.602pt}{0.400pt}}
\put(792,495.17){\rule{1.100pt}{0.400pt}}
\multiput(792.00,494.17)(2.717,2.000){2}{\rule{0.550pt}{0.400pt}}
\put(797,496.67){\rule{0.964pt}{0.400pt}}
\multiput(797.00,496.17)(2.000,1.000){2}{\rule{0.482pt}{0.400pt}}
\put(801,497.67){\rule{1.204pt}{0.400pt}}
\multiput(801.00,497.17)(2.500,1.000){2}{\rule{0.602pt}{0.400pt}}
\put(806,499.17){\rule{1.100pt}{0.400pt}}
\multiput(806.00,498.17)(2.717,2.000){2}{\rule{0.550pt}{0.400pt}}
\put(811,500.67){\rule{0.964pt}{0.400pt}}
\multiput(811.00,500.17)(2.000,1.000){2}{\rule{0.482pt}{0.400pt}}
\put(815,502.17){\rule{1.100pt}{0.400pt}}
\multiput(815.00,501.17)(2.717,2.000){2}{\rule{0.550pt}{0.400pt}}
\put(820,503.67){\rule{1.204pt}{0.400pt}}
\multiput(820.00,503.17)(2.500,1.000){2}{\rule{0.602pt}{0.400pt}}
\put(825,505.17){\rule{0.900pt}{0.400pt}}
\multiput(825.00,504.17)(2.132,2.000){2}{\rule{0.450pt}{0.400pt}}
\put(829,506.67){\rule{1.204pt}{0.400pt}}
\multiput(829.00,506.17)(2.500,1.000){2}{\rule{0.602pt}{0.400pt}}
\put(834,508.17){\rule{1.100pt}{0.400pt}}
\multiput(834.00,507.17)(2.717,2.000){2}{\rule{0.550pt}{0.400pt}}
\put(839,509.67){\rule{1.204pt}{0.400pt}}
\multiput(839.00,509.17)(2.500,1.000){2}{\rule{0.602pt}{0.400pt}}
\put(844,511.17){\rule{0.900pt}{0.400pt}}
\multiput(844.00,510.17)(2.132,2.000){2}{\rule{0.450pt}{0.400pt}}
\put(848,512.67){\rule{1.204pt}{0.400pt}}
\multiput(848.00,512.17)(2.500,1.000){2}{\rule{0.602pt}{0.400pt}}
\put(853,513.67){\rule{1.204pt}{0.400pt}}
\multiput(853.00,513.17)(2.500,1.000){2}{\rule{0.602pt}{0.400pt}}
\put(858,515.17){\rule{0.900pt}{0.400pt}}
\multiput(858.00,514.17)(2.132,2.000){2}{\rule{0.450pt}{0.400pt}}
\put(862,516.67){\rule{1.204pt}{0.400pt}}
\multiput(862.00,516.17)(2.500,1.000){2}{\rule{0.602pt}{0.400pt}}
\put(867,518.17){\rule{1.100pt}{0.400pt}}
\multiput(867.00,517.17)(2.717,2.000){2}{\rule{0.550pt}{0.400pt}}
\put(872,519.67){\rule{0.964pt}{0.400pt}}
\multiput(872.00,519.17)(2.000,1.000){2}{\rule{0.482pt}{0.400pt}}
\put(876,521.17){\rule{1.100pt}{0.400pt}}
\multiput(876.00,520.17)(2.717,2.000){2}{\rule{0.550pt}{0.400pt}}
\put(881,522.67){\rule{1.204pt}{0.400pt}}
\multiput(881.00,522.17)(2.500,1.000){2}{\rule{0.602pt}{0.400pt}}
\put(886,524.17){\rule{0.900pt}{0.400pt}}
\multiput(886.00,523.17)(2.132,2.000){2}{\rule{0.450pt}{0.400pt}}
\put(890,525.67){\rule{1.204pt}{0.400pt}}
\multiput(890.00,525.17)(2.500,1.000){2}{\rule{0.602pt}{0.400pt}}
\put(895,527.17){\rule{1.100pt}{0.400pt}}
\multiput(895.00,526.17)(2.717,2.000){2}{\rule{0.550pt}{0.400pt}}
\put(900,528.67){\rule{0.964pt}{0.400pt}}
\multiput(900.00,528.17)(2.000,1.000){2}{\rule{0.482pt}{0.400pt}}
\put(904,529.67){\rule{1.204pt}{0.400pt}}
\multiput(904.00,529.17)(2.500,1.000){2}{\rule{0.602pt}{0.400pt}}
\put(909,531.17){\rule{1.100pt}{0.400pt}}
\multiput(909.00,530.17)(2.717,2.000){2}{\rule{0.550pt}{0.400pt}}
\put(914,532.67){\rule{0.964pt}{0.400pt}}
\multiput(914.00,532.17)(2.000,1.000){2}{\rule{0.482pt}{0.400pt}}
\put(918,534.17){\rule{1.100pt}{0.400pt}}
\multiput(918.00,533.17)(2.717,2.000){2}{\rule{0.550pt}{0.400pt}}
\put(923,535.67){\rule{1.204pt}{0.400pt}}
\multiput(923.00,535.17)(2.500,1.000){2}{\rule{0.602pt}{0.400pt}}
\put(928,537.17){\rule{0.900pt}{0.400pt}}
\multiput(928.00,536.17)(2.132,2.000){2}{\rule{0.450pt}{0.400pt}}
\put(932,538.67){\rule{1.204pt}{0.400pt}}
\multiput(932.00,538.17)(2.500,1.000){2}{\rule{0.602pt}{0.400pt}}
\put(937,540.17){\rule{1.100pt}{0.400pt}}
\multiput(937.00,539.17)(2.717,2.000){2}{\rule{0.550pt}{0.400pt}}
\put(942,541.67){\rule{1.204pt}{0.400pt}}
\multiput(942.00,541.17)(2.500,1.000){2}{\rule{0.602pt}{0.400pt}}
\put(947,542.67){\rule{0.964pt}{0.400pt}}
\multiput(947.00,542.17)(2.000,1.000){2}{\rule{0.482pt}{0.400pt}}
\put(951,544.17){\rule{1.100pt}{0.400pt}}
\multiput(951.00,543.17)(2.717,2.000){2}{\rule{0.550pt}{0.400pt}}
\put(956,545.67){\rule{1.204pt}{0.400pt}}
\multiput(956.00,545.17)(2.500,1.000){2}{\rule{0.602pt}{0.400pt}}
\put(961,547.17){\rule{0.900pt}{0.400pt}}
\multiput(961.00,546.17)(2.132,2.000){2}{\rule{0.450pt}{0.400pt}}
\put(965,548.67){\rule{1.204pt}{0.400pt}}
\multiput(965.00,548.17)(2.500,1.000){2}{\rule{0.602pt}{0.400pt}}
\put(970,550.17){\rule{1.100pt}{0.400pt}}
\multiput(970.00,549.17)(2.717,2.000){2}{\rule{0.550pt}{0.400pt}}
\put(975,551.67){\rule{0.964pt}{0.400pt}}
\multiput(975.00,551.17)(2.000,1.000){2}{\rule{0.482pt}{0.400pt}}
\put(979,553.17){\rule{1.100pt}{0.400pt}}
\multiput(979.00,552.17)(2.717,2.000){2}{\rule{0.550pt}{0.400pt}}
\put(984,554.67){\rule{1.204pt}{0.400pt}}
\multiput(984.00,554.17)(2.500,1.000){2}{\rule{0.602pt}{0.400pt}}
\put(989,556.17){\rule{0.900pt}{0.400pt}}
\multiput(989.00,555.17)(2.132,2.000){2}{\rule{0.450pt}{0.400pt}}
\put(993,557.67){\rule{1.204pt}{0.400pt}}
\multiput(993.00,557.17)(2.500,1.000){2}{\rule{0.602pt}{0.400pt}}
\put(998,558.67){\rule{1.204pt}{0.400pt}}
\multiput(998.00,558.17)(2.500,1.000){2}{\rule{0.602pt}{0.400pt}}
\put(1003,560.17){\rule{0.900pt}{0.400pt}}
\multiput(1003.00,559.17)(2.132,2.000){2}{\rule{0.450pt}{0.400pt}}
\put(1007,561.67){\rule{1.204pt}{0.400pt}}
\multiput(1007.00,561.17)(2.500,1.000){2}{\rule{0.602pt}{0.400pt}}
\put(1012,563.17){\rule{1.100pt}{0.400pt}}
\multiput(1012.00,562.17)(2.717,2.000){2}{\rule{0.550pt}{0.400pt}}
\put(1017,564.67){\rule{0.964pt}{0.400pt}}
\multiput(1017.00,564.17)(2.000,1.000){2}{\rule{0.482pt}{0.400pt}}
\put(1021,566.17){\rule{1.100pt}{0.400pt}}
\multiput(1021.00,565.17)(2.717,2.000){2}{\rule{0.550pt}{0.400pt}}
\put(1026,567.67){\rule{1.204pt}{0.400pt}}
\multiput(1026.00,567.17)(2.500,1.000){2}{\rule{0.602pt}{0.400pt}}
\put(1031,569.17){\rule{0.900pt}{0.400pt}}
\multiput(1031.00,568.17)(2.132,2.000){2}{\rule{0.450pt}{0.400pt}}
\put(1035,570.67){\rule{1.204pt}{0.400pt}}
\multiput(1035.00,570.17)(2.500,1.000){2}{\rule{0.602pt}{0.400pt}}
\put(1040,571.67){\rule{1.204pt}{0.400pt}}
\multiput(1040.00,571.17)(2.500,1.000){2}{\rule{0.602pt}{0.400pt}}
\put(1045,573.17){\rule{0.900pt}{0.400pt}}
\multiput(1045.00,572.17)(2.132,2.000){2}{\rule{0.450pt}{0.400pt}}
\put(1049,574.67){\rule{1.204pt}{0.400pt}}
\multiput(1049.00,574.17)(2.500,1.000){2}{\rule{0.602pt}{0.400pt}}
\put(1054,576.17){\rule{1.100pt}{0.400pt}}
\multiput(1054.00,575.17)(2.717,2.000){2}{\rule{0.550pt}{0.400pt}}
\put(1059,577.67){\rule{1.204pt}{0.400pt}}
\multiput(1059.00,577.17)(2.500,1.000){2}{\rule{0.602pt}{0.400pt}}
\put(1064,579.17){\rule{0.900pt}{0.400pt}}
\multiput(1064.00,578.17)(2.132,2.000){2}{\rule{0.450pt}{0.400pt}}
\put(1068,580.67){\rule{1.204pt}{0.400pt}}
\multiput(1068.00,580.17)(2.500,1.000){2}{\rule{0.602pt}{0.400pt}}
\put(1073,582.17){\rule{1.100pt}{0.400pt}}
\multiput(1073.00,581.17)(2.717,2.000){2}{\rule{0.550pt}{0.400pt}}
\put(1078,583.67){\rule{0.964pt}{0.400pt}}
\multiput(1078.00,583.17)(2.000,1.000){2}{\rule{0.482pt}{0.400pt}}
\put(1082,585.17){\rule{1.100pt}{0.400pt}}
\multiput(1082.00,584.17)(2.717,2.000){2}{\rule{0.550pt}{0.400pt}}
\put(1087,586.67){\rule{1.204pt}{0.400pt}}
\multiput(1087.00,586.17)(2.500,1.000){2}{\rule{0.602pt}{0.400pt}}
\put(1092,587.67){\rule{0.964pt}{0.400pt}}
\multiput(1092.00,587.17)(2.000,1.000){2}{\rule{0.482pt}{0.400pt}}
\put(1096,589.17){\rule{1.100pt}{0.400pt}}
\multiput(1096.00,588.17)(2.717,2.000){2}{\rule{0.550pt}{0.400pt}}
\put(1101,590.67){\rule{1.204pt}{0.400pt}}
\multiput(1101.00,590.17)(2.500,1.000){2}{\rule{0.602pt}{0.400pt}}
\put(1106,592.17){\rule{0.900pt}{0.400pt}}
\multiput(1106.00,591.17)(2.132,2.000){2}{\rule{0.450pt}{0.400pt}}
\put(1110,593.67){\rule{1.204pt}{0.400pt}}
\multiput(1110.00,593.17)(2.500,1.000){2}{\rule{0.602pt}{0.400pt}}
\put(1115,595.17){\rule{1.100pt}{0.400pt}}
\multiput(1115.00,594.17)(2.717,2.000){2}{\rule{0.550pt}{0.400pt}}
\put(1120,596.67){\rule{0.964pt}{0.400pt}}
\multiput(1120.00,596.17)(2.000,1.000){2}{\rule{0.482pt}{0.400pt}}
\put(1124,598.17){\rule{1.100pt}{0.400pt}}
\multiput(1124.00,597.17)(2.717,2.000){2}{\rule{0.550pt}{0.400pt}}
\put(1129,599.67){\rule{1.204pt}{0.400pt}}
\multiput(1129.00,599.17)(2.500,1.000){2}{\rule{0.602pt}{0.400pt}}
\put(1134,601.17){\rule{0.900pt}{0.400pt}}
\multiput(1134.00,600.17)(2.132,2.000){2}{\rule{0.450pt}{0.400pt}}
\put(1138,602.67){\rule{1.204pt}{0.400pt}}
\multiput(1138.00,602.17)(2.500,1.000){2}{\rule{0.602pt}{0.400pt}}
\put(1143,603.67){\rule{1.204pt}{0.400pt}}
\multiput(1143.00,603.17)(2.500,1.000){2}{\rule{0.602pt}{0.400pt}}
\put(1148,605.17){\rule{0.900pt}{0.400pt}}
\multiput(1148.00,604.17)(2.132,2.000){2}{\rule{0.450pt}{0.400pt}}
\put(1152,606.67){\rule{1.204pt}{0.400pt}}
\multiput(1152.00,606.17)(2.500,1.000){2}{\rule{0.602pt}{0.400pt}}
\put(1157,608.17){\rule{1.100pt}{0.400pt}}
\multiput(1157.00,607.17)(2.717,2.000){2}{\rule{0.550pt}{0.400pt}}
\put(1162,609.67){\rule{1.204pt}{0.400pt}}
\multiput(1162.00,609.17)(2.500,1.000){2}{\rule{0.602pt}{0.400pt}}
\put(1167,611.17){\rule{0.900pt}{0.400pt}}
\multiput(1167.00,610.17)(2.132,2.000){2}{\rule{0.450pt}{0.400pt}}
\put(1171,612.67){\rule{1.204pt}{0.400pt}}
\multiput(1171.00,612.17)(2.500,1.000){2}{\rule{0.602pt}{0.400pt}}
\put(1176,614.17){\rule{1.100pt}{0.400pt}}
\multiput(1176.00,613.17)(2.717,2.000){2}{\rule{0.550pt}{0.400pt}}
\put(1181,615.67){\rule{0.964pt}{0.400pt}}
\multiput(1181.00,615.17)(2.000,1.000){2}{\rule{0.482pt}{0.400pt}}
\put(1185,616.67){\rule{1.204pt}{0.400pt}}
\multiput(1185.00,616.17)(2.500,1.000){2}{\rule{0.602pt}{0.400pt}}
\put(1190,618.17){\rule{1.100pt}{0.400pt}}
\multiput(1190.00,617.17)(2.717,2.000){2}{\rule{0.550pt}{0.400pt}}
\put(1195,619.67){\rule{0.964pt}{0.400pt}}
\multiput(1195.00,619.17)(2.000,1.000){2}{\rule{0.482pt}{0.400pt}}
\put(1199,621.17){\rule{1.100pt}{0.400pt}}
\multiput(1199.00,620.17)(2.717,2.000){2}{\rule{0.550pt}{0.400pt}}
\put(1204,622.67){\rule{1.204pt}{0.400pt}}
\multiput(1204.00,622.17)(2.500,1.000){2}{\rule{0.602pt}{0.400pt}}
\put(1209,624.17){\rule{0.900pt}{0.400pt}}
\multiput(1209.00,623.17)(2.132,2.000){2}{\rule{0.450pt}{0.400pt}}
\put(1213,625.67){\rule{1.204pt}{0.400pt}}
\multiput(1213.00,625.17)(2.500,1.000){2}{\rule{0.602pt}{0.400pt}}
\put(1218,627.17){\rule{1.100pt}{0.400pt}}
\multiput(1218.00,626.17)(2.717,2.000){2}{\rule{0.550pt}{0.400pt}}
\put(1223,628.67){\rule{0.964pt}{0.400pt}}
\multiput(1223.00,628.17)(2.000,1.000){2}{\rule{0.482pt}{0.400pt}}
\put(1227,630.17){\rule{1.100pt}{0.400pt}}
\multiput(1227.00,629.17)(2.717,2.000){2}{\rule{0.550pt}{0.400pt}}
\put(1232,631.67){\rule{1.204pt}{0.400pt}}
\multiput(1232.00,631.17)(2.500,1.000){2}{\rule{0.602pt}{0.400pt}}
\put(1237,632.67){\rule{0.964pt}{0.400pt}}
\multiput(1237.00,632.17)(2.000,1.000){2}{\rule{0.482pt}{0.400pt}}
\put(1241,634.17){\rule{1.100pt}{0.400pt}}
\multiput(1241.00,633.17)(2.717,2.000){2}{\rule{0.550pt}{0.400pt}}
\put(1246,635.67){\rule{1.204pt}{0.400pt}}
\multiput(1246.00,635.17)(2.500,1.000){2}{\rule{0.602pt}{0.400pt}}
\put(1251,637.17){\rule{0.900pt}{0.400pt}}
\multiput(1251.00,636.17)(2.132,2.000){2}{\rule{0.450pt}{0.400pt}}
\put(1255,638.67){\rule{1.204pt}{0.400pt}}
\multiput(1255.00,638.17)(2.500,1.000){2}{\rule{0.602pt}{0.400pt}}
\put(1260,640.17){\rule{1.100pt}{0.400pt}}
\multiput(1260.00,639.17)(2.717,2.000){2}{\rule{0.550pt}{0.400pt}}
\put(1265,641.67){\rule{1.204pt}{0.400pt}}
\multiput(1265.00,641.17)(2.500,1.000){2}{\rule{0.602pt}{0.400pt}}
\put(1270,643.17){\rule{0.900pt}{0.400pt}}
\multiput(1270.00,642.17)(2.132,2.000){2}{\rule{0.450pt}{0.400pt}}
\put(1274,644.67){\rule{1.204pt}{0.400pt}}
\multiput(1274.00,644.17)(2.500,1.000){2}{\rule{0.602pt}{0.400pt}}
\put(1279,646.17){\rule{1.100pt}{0.400pt}}
\multiput(1279.00,645.17)(2.717,2.000){2}{\rule{0.550pt}{0.400pt}}
\put(1284,647.67){\rule{0.964pt}{0.400pt}}
\multiput(1284.00,647.17)(2.000,1.000){2}{\rule{0.482pt}{0.400pt}}
\put(1288,648.67){\rule{1.204pt}{0.400pt}}
\multiput(1288.00,648.17)(2.500,1.000){2}{\rule{0.602pt}{0.400pt}}
\put(1293,650.17){\rule{1.100pt}{0.400pt}}
\multiput(1293.00,649.17)(2.717,2.000){2}{\rule{0.550pt}{0.400pt}}
\put(1298,651.67){\rule{0.964pt}{0.400pt}}
\multiput(1298.00,651.17)(2.000,1.000){2}{\rule{0.482pt}{0.400pt}}
\put(1302,653.17){\rule{1.100pt}{0.400pt}}
\multiput(1302.00,652.17)(2.717,2.000){2}{\rule{0.550pt}{0.400pt}}
\put(1307,654.67){\rule{1.204pt}{0.400pt}}
\multiput(1307.00,654.17)(2.500,1.000){2}{\rule{0.602pt}{0.400pt}}
\put(1312,656.17){\rule{0.900pt}{0.400pt}}
\multiput(1312.00,655.17)(2.132,2.000){2}{\rule{0.450pt}{0.400pt}}
\put(1316,657.67){\rule{1.204pt}{0.400pt}}
\multiput(1316.00,657.17)(2.500,1.000){2}{\rule{0.602pt}{0.400pt}}
\put(1321,659.17){\rule{1.100pt}{0.400pt}}
\multiput(1321.00,658.17)(2.717,2.000){2}{\rule{0.550pt}{0.400pt}}
\put(1326,660.67){\rule{0.964pt}{0.400pt}}
\multiput(1326.00,660.17)(2.000,1.000){2}{\rule{0.482pt}{0.400pt}}
\put(1330,661.67){\rule{1.204pt}{0.400pt}}
\multiput(1330.00,661.17)(2.500,1.000){2}{\rule{0.602pt}{0.400pt}}
\put(1335,663.17){\rule{1.100pt}{0.400pt}}
\multiput(1335.00,662.17)(2.717,2.000){2}{\rule{0.550pt}{0.400pt}}
\put(1340,664.67){\rule{0.964pt}{0.400pt}}
\multiput(1340.00,664.17)(2.000,1.000){2}{\rule{0.482pt}{0.400pt}}
\put(1344,666.17){\rule{1.100pt}{0.400pt}}
\multiput(1344.00,665.17)(2.717,2.000){2}{\rule{0.550pt}{0.400pt}}
\put(1349,667.67){\rule{1.204pt}{0.400pt}}
\multiput(1349.00,667.17)(2.500,1.000){2}{\rule{0.602pt}{0.400pt}}
\put(1354,669.17){\rule{0.900pt}{0.400pt}}
\multiput(1354.00,668.17)(2.132,2.000){2}{\rule{0.450pt}{0.400pt}}
\put(1358,670.67){\rule{1.204pt}{0.400pt}}
\multiput(1358.00,670.17)(2.500,1.000){2}{\rule{0.602pt}{0.400pt}}
\put(1363,672.17){\rule{1.100pt}{0.400pt}}
\multiput(1363.00,671.17)(2.717,2.000){2}{\rule{0.550pt}{0.400pt}}
\put(1368,673.67){\rule{0.964pt}{0.400pt}}
\multiput(1368.00,673.17)(2.000,1.000){2}{\rule{0.482pt}{0.400pt}}
\put(1372,675.17){\rule{1.100pt}{0.400pt}}
\multiput(1372.00,674.17)(2.717,2.000){2}{\rule{0.550pt}{0.400pt}}
\put(1377,676.67){\rule{1.204pt}{0.400pt}}
\multiput(1377.00,676.17)(2.500,1.000){2}{\rule{0.602pt}{0.400pt}}
\put(1382,677.67){\rule{1.204pt}{0.400pt}}
\multiput(1382.00,677.17)(2.500,1.000){2}{\rule{0.602pt}{0.400pt}}
\put(1387,679.17){\rule{0.900pt}{0.400pt}}
\multiput(1387.00,678.17)(2.132,2.000){2}{\rule{0.450pt}{0.400pt}}
\put(1391,680.67){\rule{1.204pt}{0.400pt}}
\multiput(1391.00,680.17)(2.500,1.000){2}{\rule{0.602pt}{0.400pt}}
\put(1396,682.17){\rule{1.100pt}{0.400pt}}
\multiput(1396.00,681.17)(2.717,2.000){2}{\rule{0.550pt}{0.400pt}}
\put(1401,683.67){\rule{0.964pt}{0.400pt}}
\multiput(1401.00,683.17)(2.000,1.000){2}{\rule{0.482pt}{0.400pt}}
\put(1405,685.17){\rule{1.100pt}{0.400pt}}
\multiput(1405.00,684.17)(2.717,2.000){2}{\rule{0.550pt}{0.400pt}}
\put(1410,686.67){\rule{1.204pt}{0.400pt}}
\multiput(1410.00,686.17)(2.500,1.000){2}{\rule{0.602pt}{0.400pt}}
\put(1410,688.17){\rule{1.100pt}{0.400pt}}
\multiput(1412.72,687.17)(-2.717,2.000){2}{\rule{0.550pt}{0.400pt}}
\put(1405,689.67){\rule{1.204pt}{0.400pt}}
\multiput(1407.50,689.17)(-2.500,1.000){2}{\rule{0.602pt}{0.400pt}}
\put(1401,690.67){\rule{0.964pt}{0.400pt}}
\multiput(1403.00,690.17)(-2.000,1.000){2}{\rule{0.482pt}{0.400pt}}
\put(1396,692.17){\rule{1.100pt}{0.400pt}}
\multiput(1398.72,691.17)(-2.717,2.000){2}{\rule{0.550pt}{0.400pt}}
\put(1391,693.67){\rule{1.204pt}{0.400pt}}
\multiput(1393.50,693.17)(-2.500,1.000){2}{\rule{0.602pt}{0.400pt}}
\put(1387,695.17){\rule{0.900pt}{0.400pt}}
\multiput(1389.13,694.17)(-2.132,2.000){2}{\rule{0.450pt}{0.400pt}}
\put(1382,696.67){\rule{1.204pt}{0.400pt}}
\multiput(1384.50,696.17)(-2.500,1.000){2}{\rule{0.602pt}{0.400pt}}
\put(1377,698.17){\rule{1.100pt}{0.400pt}}
\multiput(1379.72,697.17)(-2.717,2.000){2}{\rule{0.550pt}{0.400pt}}
\put(1372,699.67){\rule{1.204pt}{0.400pt}}
\multiput(1374.50,699.17)(-2.500,1.000){2}{\rule{0.602pt}{0.400pt}}
\put(1368,701.17){\rule{0.900pt}{0.400pt}}
\multiput(1370.13,700.17)(-2.132,2.000){2}{\rule{0.450pt}{0.400pt}}
\put(1363,702.67){\rule{1.204pt}{0.400pt}}
\multiput(1365.50,702.17)(-2.500,1.000){2}{\rule{0.602pt}{0.400pt}}
\put(1358,704.17){\rule{1.100pt}{0.400pt}}
\multiput(1360.72,703.17)(-2.717,2.000){2}{\rule{0.550pt}{0.400pt}}
\put(1354,705.67){\rule{0.964pt}{0.400pt}}
\multiput(1356.00,705.17)(-2.000,1.000){2}{\rule{0.482pt}{0.400pt}}
\put(1349,706.67){\rule{1.204pt}{0.400pt}}
\multiput(1351.50,706.17)(-2.500,1.000){2}{\rule{0.602pt}{0.400pt}}
\put(1344,708.17){\rule{1.100pt}{0.400pt}}
\multiput(1346.72,707.17)(-2.717,2.000){2}{\rule{0.550pt}{0.400pt}}
\put(1340,709.67){\rule{0.964pt}{0.400pt}}
\multiput(1342.00,709.17)(-2.000,1.000){2}{\rule{0.482pt}{0.400pt}}
\put(1335,711.17){\rule{1.100pt}{0.400pt}}
\multiput(1337.72,710.17)(-2.717,2.000){2}{\rule{0.550pt}{0.400pt}}
\put(1330,712.67){\rule{1.204pt}{0.400pt}}
\multiput(1332.50,712.17)(-2.500,1.000){2}{\rule{0.602pt}{0.400pt}}
\put(1326,714.17){\rule{0.900pt}{0.400pt}}
\multiput(1328.13,713.17)(-2.132,2.000){2}{\rule{0.450pt}{0.400pt}}
\put(1321,715.67){\rule{1.204pt}{0.400pt}}
\multiput(1323.50,715.17)(-2.500,1.000){2}{\rule{0.602pt}{0.400pt}}
\put(1316,717.17){\rule{1.100pt}{0.400pt}}
\multiput(1318.72,716.17)(-2.717,2.000){2}{\rule{0.550pt}{0.400pt}}
\put(1312,718.67){\rule{0.964pt}{0.400pt}}
\multiput(1314.00,718.17)(-2.000,1.000){2}{\rule{0.482pt}{0.400pt}}
\put(1307,720.17){\rule{1.100pt}{0.400pt}}
\multiput(1309.72,719.17)(-2.717,2.000){2}{\rule{0.550pt}{0.400pt}}
\put(1302,721.67){\rule{1.204pt}{0.400pt}}
\multiput(1304.50,721.17)(-2.500,1.000){2}{\rule{0.602pt}{0.400pt}}
\put(1298,722.67){\rule{0.964pt}{0.400pt}}
\multiput(1300.00,722.17)(-2.000,1.000){2}{\rule{0.482pt}{0.400pt}}
\put(1293,724.17){\rule{1.100pt}{0.400pt}}
\multiput(1295.72,723.17)(-2.717,2.000){2}{\rule{0.550pt}{0.400pt}}
\put(1288,725.67){\rule{1.204pt}{0.400pt}}
\multiput(1290.50,725.17)(-2.500,1.000){2}{\rule{0.602pt}{0.400pt}}
\put(1284,727.17){\rule{0.900pt}{0.400pt}}
\multiput(1286.13,726.17)(-2.132,2.000){2}{\rule{0.450pt}{0.400pt}}
\put(1279,728.67){\rule{1.204pt}{0.400pt}}
\multiput(1281.50,728.17)(-2.500,1.000){2}{\rule{0.602pt}{0.400pt}}
\put(1274,730.17){\rule{1.100pt}{0.400pt}}
\multiput(1276.72,729.17)(-2.717,2.000){2}{\rule{0.550pt}{0.400pt}}
\put(1270,731.67){\rule{0.964pt}{0.400pt}}
\multiput(1272.00,731.17)(-2.000,1.000){2}{\rule{0.482pt}{0.400pt}}
\put(1265,733.17){\rule{1.100pt}{0.400pt}}
\multiput(1267.72,732.17)(-2.717,2.000){2}{\rule{0.550pt}{0.400pt}}
\put(1260,734.67){\rule{1.204pt}{0.400pt}}
\multiput(1262.50,734.17)(-2.500,1.000){2}{\rule{0.602pt}{0.400pt}}
\put(1255,735.67){\rule{1.204pt}{0.400pt}}
\multiput(1257.50,735.17)(-2.500,1.000){2}{\rule{0.602pt}{0.400pt}}
\put(1251,737.17){\rule{0.900pt}{0.400pt}}
\multiput(1253.13,736.17)(-2.132,2.000){2}{\rule{0.450pt}{0.400pt}}
\put(1246,738.67){\rule{1.204pt}{0.400pt}}
\multiput(1248.50,738.17)(-2.500,1.000){2}{\rule{0.602pt}{0.400pt}}
\put(1241,740.17){\rule{1.100pt}{0.400pt}}
\multiput(1243.72,739.17)(-2.717,2.000){2}{\rule{0.550pt}{0.400pt}}
\put(1237,741.67){\rule{0.964pt}{0.400pt}}
\multiput(1239.00,741.17)(-2.000,1.000){2}{\rule{0.482pt}{0.400pt}}
\put(130.0,82.0){\rule[-0.200pt]{0.400pt}{187.179pt}}
\put(130.0,82.0){\rule[-0.200pt]{315.338pt}{0.400pt}}
\put(1439.0,82.0){\rule[-0.200pt]{0.400pt}{187.179pt}}
\put(130.0,859.0){\rule[-0.200pt]{315.338pt}{0.400pt}}
\end{picture}

Plot for Ball 5:\\
% GNUPLOT: LaTeX picture
\setlength{\unitlength}{0.240900pt}
\ifx\plotpoint\undefined\newsavebox{\plotpoint}\fi
\sbox{\plotpoint}{\rule[-0.200pt]{0.400pt}{0.400pt}}%
\begin{picture}(1500,900)(0,0)
\sbox{\plotpoint}{\rule[-0.200pt]{0.400pt}{0.400pt}}%
\put(130.0,90.0){\rule[-0.200pt]{4.818pt}{0.400pt}}
\put(110,90){\makebox(0,0)[r]{ 0}}
\put(1419.0,90.0){\rule[-0.200pt]{4.818pt}{0.400pt}}
\put(130.0,242.0){\rule[-0.200pt]{4.818pt}{0.400pt}}
\put(110,242){\makebox(0,0)[r]{ 0.2}}
\put(1419.0,242.0){\rule[-0.200pt]{4.818pt}{0.400pt}}
\put(130.0,394.0){\rule[-0.200pt]{4.818pt}{0.400pt}}
\put(110,394){\makebox(0,0)[r]{ 0.4}}
\put(1419.0,394.0){\rule[-0.200pt]{4.818pt}{0.400pt}}
\put(130.0,547.0){\rule[-0.200pt]{4.818pt}{0.400pt}}
\put(110,547){\makebox(0,0)[r]{ 0.6}}
\put(1419.0,547.0){\rule[-0.200pt]{4.818pt}{0.400pt}}
\put(130.0,699.0){\rule[-0.200pt]{4.818pt}{0.400pt}}
\put(110,699){\makebox(0,0)[r]{ 0.8}}
\put(1419.0,699.0){\rule[-0.200pt]{4.818pt}{0.400pt}}
\put(130.0,851.0){\rule[-0.200pt]{4.818pt}{0.400pt}}
\put(110,851){\makebox(0,0)[r]{ 1}}
\put(1419.0,851.0){\rule[-0.200pt]{4.818pt}{0.400pt}}
\put(130.0,82.0){\rule[-0.200pt]{0.400pt}{4.818pt}}
\put(130,41){\makebox(0,0){ 0}}
\put(130.0,839.0){\rule[-0.200pt]{0.400pt}{4.818pt}}
\put(392.0,82.0){\rule[-0.200pt]{0.400pt}{4.818pt}}
\put(392,41){\makebox(0,0){ 0.2}}
\put(392.0,839.0){\rule[-0.200pt]{0.400pt}{4.818pt}}
\put(654.0,82.0){\rule[-0.200pt]{0.400pt}{4.818pt}}
\put(654,41){\makebox(0,0){ 0.4}}
\put(654.0,839.0){\rule[-0.200pt]{0.400pt}{4.818pt}}
\put(915.0,82.0){\rule[-0.200pt]{0.400pt}{4.818pt}}
\put(915,41){\makebox(0,0){ 0.6}}
\put(915.0,839.0){\rule[-0.200pt]{0.400pt}{4.818pt}}
\put(1177.0,82.0){\rule[-0.200pt]{0.400pt}{4.818pt}}
\put(1177,41){\makebox(0,0){ 0.8}}
\put(1177.0,839.0){\rule[-0.200pt]{0.400pt}{4.818pt}}
\put(1439.0,82.0){\rule[-0.200pt]{0.400pt}{4.818pt}}
\put(1439,41){\makebox(0,0){ 1}}
\put(1439.0,839.0){\rule[-0.200pt]{0.400pt}{4.818pt}}
\put(130.0,82.0){\rule[-0.200pt]{0.400pt}{187.179pt}}
\put(130.0,82.0){\rule[-0.200pt]{315.338pt}{0.400pt}}
\put(1439.0,82.0){\rule[-0.200pt]{0.400pt}{187.179pt}}
\put(130.0,859.0){\rule[-0.200pt]{315.338pt}{0.400pt}}
\put(1279,819){\makebox(0,0)[r]{'-'}}
\put(1299.0,819.0){\rule[-0.200pt]{24.090pt}{0.400pt}}
\put(644,513){\usebox{\plotpoint}}
\put(607.17,513){\rule{0.400pt}{0.700pt}}
\multiput(608.17,513.00)(-2.000,1.547){2}{\rule{0.400pt}{0.350pt}}
\put(605.67,516){\rule{0.400pt}{0.723pt}}
\multiput(606.17,516.00)(-1.000,1.500){2}{\rule{0.400pt}{0.361pt}}
\put(604.67,519){\rule{0.400pt}{0.723pt}}
\multiput(605.17,519.00)(-1.000,1.500){2}{\rule{0.400pt}{0.361pt}}
\put(603.67,522){\rule{0.400pt}{0.723pt}}
\multiput(604.17,522.00)(-1.000,1.500){2}{\rule{0.400pt}{0.361pt}}
\put(602.67,525){\rule{0.400pt}{0.723pt}}
\multiput(603.17,525.00)(-1.000,1.500){2}{\rule{0.400pt}{0.361pt}}
\put(601.67,528){\rule{0.400pt}{0.723pt}}
\multiput(602.17,528.00)(-1.000,1.500){2}{\rule{0.400pt}{0.361pt}}
\put(600.67,531){\rule{0.400pt}{0.723pt}}
\multiput(601.17,531.00)(-1.000,1.500){2}{\rule{0.400pt}{0.361pt}}
\put(599.17,534){\rule{0.400pt}{0.700pt}}
\multiput(600.17,534.00)(-2.000,1.547){2}{\rule{0.400pt}{0.350pt}}
\put(597.67,537){\rule{0.400pt}{0.723pt}}
\multiput(598.17,537.00)(-1.000,1.500){2}{\rule{0.400pt}{0.361pt}}
\put(596.67,540){\rule{0.400pt}{0.723pt}}
\multiput(597.17,540.00)(-1.000,1.500){2}{\rule{0.400pt}{0.361pt}}
\put(595.67,543){\rule{0.400pt}{0.482pt}}
\multiput(596.17,543.00)(-1.000,1.000){2}{\rule{0.400pt}{0.241pt}}
\put(594.67,545){\rule{0.400pt}{0.723pt}}
\multiput(595.17,545.00)(-1.000,1.500){2}{\rule{0.400pt}{0.361pt}}
\put(593.67,548){\rule{0.400pt}{0.723pt}}
\multiput(594.17,548.00)(-1.000,1.500){2}{\rule{0.400pt}{0.361pt}}
\put(592.67,551){\rule{0.400pt}{0.723pt}}
\multiput(593.17,551.00)(-1.000,1.500){2}{\rule{0.400pt}{0.361pt}}
\put(591.17,554){\rule{0.400pt}{0.700pt}}
\multiput(592.17,554.00)(-2.000,1.547){2}{\rule{0.400pt}{0.350pt}}
\put(589.67,557){\rule{0.400pt}{0.723pt}}
\multiput(590.17,557.00)(-1.000,1.500){2}{\rule{0.400pt}{0.361pt}}
\put(588.67,560){\rule{0.400pt}{0.723pt}}
\multiput(589.17,560.00)(-1.000,1.500){2}{\rule{0.400pt}{0.361pt}}
\put(587.67,563){\rule{0.400pt}{0.723pt}}
\multiput(588.17,563.00)(-1.000,1.500){2}{\rule{0.400pt}{0.361pt}}
\put(586.67,566){\rule{0.400pt}{0.723pt}}
\multiput(587.17,566.00)(-1.000,1.500){2}{\rule{0.400pt}{0.361pt}}
\put(585.67,569){\rule{0.400pt}{0.723pt}}
\multiput(586.17,569.00)(-1.000,1.500){2}{\rule{0.400pt}{0.361pt}}
\put(584.67,572){\rule{0.400pt}{0.723pt}}
\multiput(585.17,572.00)(-1.000,1.500){2}{\rule{0.400pt}{0.361pt}}
\put(583.17,575){\rule{0.400pt}{0.700pt}}
\multiput(584.17,575.00)(-2.000,1.547){2}{\rule{0.400pt}{0.350pt}}
\put(581.67,578){\rule{0.400pt}{0.723pt}}
\multiput(582.17,578.00)(-1.000,1.500){2}{\rule{0.400pt}{0.361pt}}
\put(580.67,581){\rule{0.400pt}{0.723pt}}
\multiput(581.17,581.00)(-1.000,1.500){2}{\rule{0.400pt}{0.361pt}}
\put(579.67,584){\rule{0.400pt}{0.723pt}}
\multiput(580.17,584.00)(-1.000,1.500){2}{\rule{0.400pt}{0.361pt}}
\put(578.67,587){\rule{0.400pt}{0.723pt}}
\multiput(579.17,587.00)(-1.000,1.500){2}{\rule{0.400pt}{0.361pt}}
\put(577.67,590){\rule{0.400pt}{0.723pt}}
\multiput(578.17,590.00)(-1.000,1.500){2}{\rule{0.400pt}{0.361pt}}
\put(576.67,593){\rule{0.400pt}{0.723pt}}
\multiput(577.17,593.00)(-1.000,1.500){2}{\rule{0.400pt}{0.361pt}}
\put(575.17,596){\rule{0.400pt}{0.700pt}}
\multiput(576.17,596.00)(-2.000,1.547){2}{\rule{0.400pt}{0.350pt}}
\put(573.67,599){\rule{0.400pt}{0.723pt}}
\multiput(574.17,599.00)(-1.000,1.500){2}{\rule{0.400pt}{0.361pt}}
\put(572.67,602){\rule{0.400pt}{0.723pt}}
\multiput(573.17,602.00)(-1.000,1.500){2}{\rule{0.400pt}{0.361pt}}
\put(571.67,605){\rule{0.400pt}{0.723pt}}
\multiput(572.17,605.00)(-1.000,1.500){2}{\rule{0.400pt}{0.361pt}}
\put(570.67,608){\rule{0.400pt}{0.482pt}}
\multiput(571.17,608.00)(-1.000,1.000){2}{\rule{0.400pt}{0.241pt}}
\put(569.67,610){\rule{0.400pt}{0.723pt}}
\multiput(570.17,610.00)(-1.000,1.500){2}{\rule{0.400pt}{0.361pt}}
\put(568.17,613){\rule{0.400pt}{0.700pt}}
\multiput(569.17,613.00)(-2.000,1.547){2}{\rule{0.400pt}{0.350pt}}
\put(566.67,616){\rule{0.400pt}{0.723pt}}
\multiput(567.17,616.00)(-1.000,1.500){2}{\rule{0.400pt}{0.361pt}}
\put(565.67,619){\rule{0.400pt}{0.723pt}}
\multiput(566.17,619.00)(-1.000,1.500){2}{\rule{0.400pt}{0.361pt}}
\put(564.67,622){\rule{0.400pt}{0.723pt}}
\multiput(565.17,622.00)(-1.000,1.500){2}{\rule{0.400pt}{0.361pt}}
\put(563.67,625){\rule{0.400pt}{0.723pt}}
\multiput(564.17,625.00)(-1.000,1.500){2}{\rule{0.400pt}{0.361pt}}
\put(562.67,628){\rule{0.400pt}{0.723pt}}
\multiput(563.17,628.00)(-1.000,1.500){2}{\rule{0.400pt}{0.361pt}}
\put(561.67,631){\rule{0.400pt}{0.723pt}}
\multiput(562.17,631.00)(-1.000,1.500){2}{\rule{0.400pt}{0.361pt}}
\put(560.17,634){\rule{0.400pt}{0.700pt}}
\multiput(561.17,634.00)(-2.000,1.547){2}{\rule{0.400pt}{0.350pt}}
\put(558.67,637){\rule{0.400pt}{0.723pt}}
\multiput(559.17,637.00)(-1.000,1.500){2}{\rule{0.400pt}{0.361pt}}
\put(557.67,640){\rule{0.400pt}{0.723pt}}
\multiput(558.17,640.00)(-1.000,1.500){2}{\rule{0.400pt}{0.361pt}}
\put(556.67,643){\rule{0.400pt}{0.723pt}}
\multiput(557.17,643.00)(-1.000,1.500){2}{\rule{0.400pt}{0.361pt}}
\put(555.67,646){\rule{0.400pt}{0.723pt}}
\multiput(556.17,646.00)(-1.000,1.500){2}{\rule{0.400pt}{0.361pt}}
\put(554.67,649){\rule{0.400pt}{0.723pt}}
\multiput(555.17,649.00)(-1.000,1.500){2}{\rule{0.400pt}{0.361pt}}
\put(553.67,652){\rule{0.400pt}{0.723pt}}
\multiput(554.17,652.00)(-1.000,1.500){2}{\rule{0.400pt}{0.361pt}}
\put(552.17,655){\rule{0.400pt}{0.700pt}}
\multiput(553.17,655.00)(-2.000,1.547){2}{\rule{0.400pt}{0.350pt}}
\put(550.67,658){\rule{0.400pt}{0.723pt}}
\multiput(551.17,658.00)(-1.000,1.500){2}{\rule{0.400pt}{0.361pt}}
\put(549.67,661){\rule{0.400pt}{0.723pt}}
\multiput(550.17,661.00)(-1.000,1.500){2}{\rule{0.400pt}{0.361pt}}
\put(548.67,664){\rule{0.400pt}{0.723pt}}
\multiput(549.17,664.00)(-1.000,1.500){2}{\rule{0.400pt}{0.361pt}}
\put(547.67,667){\rule{0.400pt}{0.723pt}}
\multiput(548.17,667.00)(-1.000,1.500){2}{\rule{0.400pt}{0.361pt}}
\put(546.67,670){\rule{0.400pt}{0.482pt}}
\multiput(547.17,670.00)(-1.000,1.000){2}{\rule{0.400pt}{0.241pt}}
\put(545.67,672){\rule{0.400pt}{0.723pt}}
\multiput(546.17,672.00)(-1.000,1.500){2}{\rule{0.400pt}{0.361pt}}
\put(544.17,675){\rule{0.400pt}{0.700pt}}
\multiput(545.17,675.00)(-2.000,1.547){2}{\rule{0.400pt}{0.350pt}}
\put(542.67,678){\rule{0.400pt}{0.723pt}}
\multiput(543.17,678.00)(-1.000,1.500){2}{\rule{0.400pt}{0.361pt}}
\put(541.67,681){\rule{0.400pt}{0.723pt}}
\multiput(542.17,681.00)(-1.000,1.500){2}{\rule{0.400pt}{0.361pt}}
\put(540.67,684){\rule{0.400pt}{0.723pt}}
\multiput(541.17,684.00)(-1.000,1.500){2}{\rule{0.400pt}{0.361pt}}
\put(539.67,687){\rule{0.400pt}{0.723pt}}
\multiput(540.17,687.00)(-1.000,1.500){2}{\rule{0.400pt}{0.361pt}}
\put(538.67,690){\rule{0.400pt}{0.723pt}}
\multiput(539.17,690.00)(-1.000,1.500){2}{\rule{0.400pt}{0.361pt}}
\put(537.67,693){\rule{0.400pt}{0.723pt}}
\multiput(538.17,693.00)(-1.000,1.500){2}{\rule{0.400pt}{0.361pt}}
\put(536.17,696){\rule{0.400pt}{0.700pt}}
\multiput(537.17,696.00)(-2.000,1.547){2}{\rule{0.400pt}{0.350pt}}
\put(534.67,699){\rule{0.400pt}{0.723pt}}
\multiput(535.17,699.00)(-1.000,1.500){2}{\rule{0.400pt}{0.361pt}}
\put(533.67,702){\rule{0.400pt}{0.723pt}}
\multiput(534.17,702.00)(-1.000,1.500){2}{\rule{0.400pt}{0.361pt}}
\put(532.67,705){\rule{0.400pt}{0.723pt}}
\multiput(533.17,705.00)(-1.000,1.500){2}{\rule{0.400pt}{0.361pt}}
\put(531.67,708){\rule{0.400pt}{0.723pt}}
\multiput(532.17,708.00)(-1.000,1.500){2}{\rule{0.400pt}{0.361pt}}
\put(530.67,711){\rule{0.400pt}{0.723pt}}
\multiput(531.17,711.00)(-1.000,1.500){2}{\rule{0.400pt}{0.361pt}}
\put(529.17,714){\rule{0.400pt}{0.700pt}}
\multiput(530.17,714.00)(-2.000,1.547){2}{\rule{0.400pt}{0.350pt}}
\put(527.67,717){\rule{0.400pt}{0.723pt}}
\multiput(528.17,717.00)(-1.000,1.500){2}{\rule{0.400pt}{0.361pt}}
\put(526.67,720){\rule{0.400pt}{0.723pt}}
\multiput(527.17,720.00)(-1.000,1.500){2}{\rule{0.400pt}{0.361pt}}
\put(525.67,723){\rule{0.400pt}{0.723pt}}
\multiput(526.17,723.00)(-1.000,1.500){2}{\rule{0.400pt}{0.361pt}}
\put(524.67,726){\rule{0.400pt}{0.723pt}}
\multiput(525.17,726.00)(-1.000,1.500){2}{\rule{0.400pt}{0.361pt}}
\put(523.67,729){\rule{0.400pt}{0.723pt}}
\multiput(524.17,729.00)(-1.000,1.500){2}{\rule{0.400pt}{0.361pt}}
\put(522.67,732){\rule{0.400pt}{0.723pt}}
\multiput(523.17,732.00)(-1.000,1.500){2}{\rule{0.400pt}{0.361pt}}
\put(521,735.17){\rule{0.482pt}{0.400pt}}
\multiput(522.00,734.17)(-1.000,2.000){2}{\rule{0.241pt}{0.400pt}}
\put(519.67,737){\rule{0.400pt}{0.723pt}}
\multiput(520.17,737.00)(-1.000,1.500){2}{\rule{0.400pt}{0.361pt}}
\put(518.67,740){\rule{0.400pt}{0.723pt}}
\multiput(519.17,740.00)(-1.000,1.500){2}{\rule{0.400pt}{0.361pt}}
\put(517.67,743){\rule{0.400pt}{0.723pt}}
\multiput(518.17,743.00)(-1.000,1.500){2}{\rule{0.400pt}{0.361pt}}
\put(516.67,746){\rule{0.400pt}{0.723pt}}
\multiput(517.17,746.00)(-1.000,1.500){2}{\rule{0.400pt}{0.361pt}}
\put(515.67,749){\rule{0.400pt}{0.723pt}}
\multiput(516.17,749.00)(-1.000,1.500){2}{\rule{0.400pt}{0.361pt}}
\put(514.67,752){\rule{0.400pt}{0.723pt}}
\multiput(515.17,752.00)(-1.000,1.500){2}{\rule{0.400pt}{0.361pt}}
\put(513.17,755){\rule{0.400pt}{0.700pt}}
\multiput(514.17,755.00)(-2.000,1.547){2}{\rule{0.400pt}{0.350pt}}
\put(511.67,758){\rule{0.400pt}{0.723pt}}
\multiput(512.17,758.00)(-1.000,1.500){2}{\rule{0.400pt}{0.361pt}}
\put(510.67,761){\rule{0.400pt}{0.723pt}}
\multiput(511.17,761.00)(-1.000,1.500){2}{\rule{0.400pt}{0.361pt}}
\put(509.67,764){\rule{0.400pt}{0.723pt}}
\multiput(510.17,764.00)(-1.000,1.500){2}{\rule{0.400pt}{0.361pt}}
\put(508.67,767){\rule{0.400pt}{0.723pt}}
\multiput(509.17,767.00)(-1.000,1.500){2}{\rule{0.400pt}{0.361pt}}
\put(507.67,770){\rule{0.400pt}{0.723pt}}
\multiput(508.17,770.00)(-1.000,1.500){2}{\rule{0.400pt}{0.361pt}}
\put(506.67,773){\rule{0.400pt}{0.723pt}}
\multiput(507.17,773.00)(-1.000,1.500){2}{\rule{0.400pt}{0.361pt}}
\put(505.17,776){\rule{0.400pt}{0.700pt}}
\multiput(506.17,776.00)(-2.000,1.547){2}{\rule{0.400pt}{0.350pt}}
\put(503.67,779){\rule{0.400pt}{0.723pt}}
\multiput(504.17,779.00)(-1.000,1.500){2}{\rule{0.400pt}{0.361pt}}
\put(502.67,782){\rule{0.400pt}{0.723pt}}
\multiput(503.17,782.00)(-1.000,1.500){2}{\rule{0.400pt}{0.361pt}}
\put(501.67,785){\rule{0.400pt}{0.723pt}}
\multiput(502.17,785.00)(-1.000,1.500){2}{\rule{0.400pt}{0.361pt}}
\put(500.67,788){\rule{0.400pt}{0.723pt}}
\multiput(501.17,788.00)(-1.000,1.500){2}{\rule{0.400pt}{0.361pt}}
\put(499.67,791){\rule{0.400pt}{0.723pt}}
\multiput(500.17,791.00)(-1.000,1.500){2}{\rule{0.400pt}{0.361pt}}
\put(498.67,794){\rule{0.400pt}{0.723pt}}
\multiput(499.17,794.00)(-1.000,1.500){2}{\rule{0.400pt}{0.361pt}}
\put(497,797.17){\rule{0.482pt}{0.400pt}}
\multiput(498.00,796.17)(-1.000,2.000){2}{\rule{0.241pt}{0.400pt}}
\put(495.67,799){\rule{0.400pt}{0.723pt}}
\multiput(496.17,799.00)(-1.000,1.500){2}{\rule{0.400pt}{0.361pt}}
\put(494.67,802){\rule{0.400pt}{0.723pt}}
\multiput(495.17,802.00)(-1.000,1.500){2}{\rule{0.400pt}{0.361pt}}
\put(493.67,805){\rule{0.400pt}{0.723pt}}
\multiput(494.17,805.00)(-1.000,1.500){2}{\rule{0.400pt}{0.361pt}}
\put(492.67,808){\rule{0.400pt}{0.723pt}}
\multiput(493.17,808.00)(-1.000,1.500){2}{\rule{0.400pt}{0.361pt}}
\put(491.67,811){\rule{0.400pt}{0.723pt}}
\multiput(492.17,811.00)(-1.000,1.500){2}{\rule{0.400pt}{0.361pt}}
\put(490.17,814){\rule{0.400pt}{0.700pt}}
\multiput(491.17,814.00)(-2.000,1.547){2}{\rule{0.400pt}{0.350pt}}
\put(488.67,817){\rule{0.400pt}{0.723pt}}
\multiput(489.17,817.00)(-1.000,1.500){2}{\rule{0.400pt}{0.361pt}}
\put(487.67,820){\rule{0.400pt}{0.723pt}}
\multiput(488.17,820.00)(-1.000,1.500){2}{\rule{0.400pt}{0.361pt}}
\put(486.67,823){\rule{0.400pt}{0.723pt}}
\multiput(487.17,823.00)(-1.000,1.500){2}{\rule{0.400pt}{0.361pt}}
\put(485.67,826){\rule{0.400pt}{0.723pt}}
\multiput(486.17,826.00)(-1.000,1.500){2}{\rule{0.400pt}{0.361pt}}
\put(484.67,829){\rule{0.400pt}{0.723pt}}
\multiput(485.17,829.00)(-1.000,1.500){2}{\rule{0.400pt}{0.361pt}}
\put(483.67,832){\rule{0.400pt}{0.723pt}}
\multiput(484.17,832.00)(-1.000,1.500){2}{\rule{0.400pt}{0.361pt}}
\put(482.17,835){\rule{0.400pt}{0.700pt}}
\multiput(483.17,835.00)(-2.000,1.547){2}{\rule{0.400pt}{0.350pt}}
\put(480.67,835){\rule{0.400pt}{0.723pt}}
\multiput(481.17,836.50)(-1.000,-1.500){2}{\rule{0.400pt}{0.361pt}}
\put(479.67,832){\rule{0.400pt}{0.723pt}}
\multiput(480.17,833.50)(-1.000,-1.500){2}{\rule{0.400pt}{0.361pt}}
\put(478.67,829){\rule{0.400pt}{0.723pt}}
\multiput(479.17,830.50)(-1.000,-1.500){2}{\rule{0.400pt}{0.361pt}}
\put(477.67,826){\rule{0.400pt}{0.723pt}}
\multiput(478.17,827.50)(-1.000,-1.500){2}{\rule{0.400pt}{0.361pt}}
\put(476.67,823){\rule{0.400pt}{0.723pt}}
\multiput(477.17,824.50)(-1.000,-1.500){2}{\rule{0.400pt}{0.361pt}}
\put(475.67,820){\rule{0.400pt}{0.723pt}}
\multiput(476.17,821.50)(-1.000,-1.500){2}{\rule{0.400pt}{0.361pt}}
\put(474.17,817){\rule{0.400pt}{0.700pt}}
\multiput(475.17,818.55)(-2.000,-1.547){2}{\rule{0.400pt}{0.350pt}}
\put(472.67,814){\rule{0.400pt}{0.723pt}}
\multiput(473.17,815.50)(-1.000,-1.500){2}{\rule{0.400pt}{0.361pt}}
\put(471.67,811){\rule{0.400pt}{0.723pt}}
\multiput(472.17,812.50)(-1.000,-1.500){2}{\rule{0.400pt}{0.361pt}}
\put(470.67,808){\rule{0.400pt}{0.723pt}}
\multiput(471.17,809.50)(-1.000,-1.500){2}{\rule{0.400pt}{0.361pt}}
\put(469.67,805){\rule{0.400pt}{0.723pt}}
\multiput(470.17,806.50)(-1.000,-1.500){2}{\rule{0.400pt}{0.361pt}}
\put(468.67,802){\rule{0.400pt}{0.723pt}}
\multiput(469.17,803.50)(-1.000,-1.500){2}{\rule{0.400pt}{0.361pt}}
\put(467.67,799){\rule{0.400pt}{0.723pt}}
\multiput(468.17,800.50)(-1.000,-1.500){2}{\rule{0.400pt}{0.361pt}}
\put(466,797.17){\rule{0.482pt}{0.400pt}}
\multiput(467.00,798.17)(-1.000,-2.000){2}{\rule{0.241pt}{0.400pt}}
\put(464.67,794){\rule{0.400pt}{0.723pt}}
\multiput(465.17,795.50)(-1.000,-1.500){2}{\rule{0.400pt}{0.361pt}}
\put(463.67,791){\rule{0.400pt}{0.723pt}}
\multiput(464.17,792.50)(-1.000,-1.500){2}{\rule{0.400pt}{0.361pt}}
\put(462.67,788){\rule{0.400pt}{0.723pt}}
\multiput(463.17,789.50)(-1.000,-1.500){2}{\rule{0.400pt}{0.361pt}}
\put(461.67,785){\rule{0.400pt}{0.723pt}}
\multiput(462.17,786.50)(-1.000,-1.500){2}{\rule{0.400pt}{0.361pt}}
\put(460.67,782){\rule{0.400pt}{0.723pt}}
\multiput(461.17,783.50)(-1.000,-1.500){2}{\rule{0.400pt}{0.361pt}}
\put(459.17,779){\rule{0.400pt}{0.700pt}}
\multiput(460.17,780.55)(-2.000,-1.547){2}{\rule{0.400pt}{0.350pt}}
\put(457.67,776){\rule{0.400pt}{0.723pt}}
\multiput(458.17,777.50)(-1.000,-1.500){2}{\rule{0.400pt}{0.361pt}}
\put(456.67,773){\rule{0.400pt}{0.723pt}}
\multiput(457.17,774.50)(-1.000,-1.500){2}{\rule{0.400pt}{0.361pt}}
\put(455.67,770){\rule{0.400pt}{0.723pt}}
\multiput(456.17,771.50)(-1.000,-1.500){2}{\rule{0.400pt}{0.361pt}}
\put(454.67,767){\rule{0.400pt}{0.723pt}}
\multiput(455.17,768.50)(-1.000,-1.500){2}{\rule{0.400pt}{0.361pt}}
\put(453.67,764){\rule{0.400pt}{0.723pt}}
\multiput(454.17,765.50)(-1.000,-1.500){2}{\rule{0.400pt}{0.361pt}}
\put(452.67,761){\rule{0.400pt}{0.723pt}}
\multiput(453.17,762.50)(-1.000,-1.500){2}{\rule{0.400pt}{0.361pt}}
\put(451.17,758){\rule{0.400pt}{0.700pt}}
\multiput(452.17,759.55)(-2.000,-1.547){2}{\rule{0.400pt}{0.350pt}}
\put(449.67,755){\rule{0.400pt}{0.723pt}}
\multiput(450.17,756.50)(-1.000,-1.500){2}{\rule{0.400pt}{0.361pt}}
\put(448.67,752){\rule{0.400pt}{0.723pt}}
\multiput(449.17,753.50)(-1.000,-1.500){2}{\rule{0.400pt}{0.361pt}}
\put(447.67,749){\rule{0.400pt}{0.723pt}}
\multiput(448.17,750.50)(-1.000,-1.500){2}{\rule{0.400pt}{0.361pt}}
\put(446.67,746){\rule{0.400pt}{0.723pt}}
\multiput(447.17,747.50)(-1.000,-1.500){2}{\rule{0.400pt}{0.361pt}}
\put(445.67,743){\rule{0.400pt}{0.723pt}}
\multiput(446.17,744.50)(-1.000,-1.500){2}{\rule{0.400pt}{0.361pt}}
\put(444.67,740){\rule{0.400pt}{0.723pt}}
\multiput(445.17,741.50)(-1.000,-1.500){2}{\rule{0.400pt}{0.361pt}}
\put(443.17,737){\rule{0.400pt}{0.700pt}}
\multiput(444.17,738.55)(-2.000,-1.547){2}{\rule{0.400pt}{0.350pt}}
\put(441.67,735){\rule{0.400pt}{0.482pt}}
\multiput(442.17,736.00)(-1.000,-1.000){2}{\rule{0.400pt}{0.241pt}}
\put(440.67,732){\rule{0.400pt}{0.723pt}}
\multiput(441.17,733.50)(-1.000,-1.500){2}{\rule{0.400pt}{0.361pt}}
\put(439.67,729){\rule{0.400pt}{0.723pt}}
\multiput(440.17,730.50)(-1.000,-1.500){2}{\rule{0.400pt}{0.361pt}}
\put(438.67,726){\rule{0.400pt}{0.723pt}}
\multiput(439.17,727.50)(-1.000,-1.500){2}{\rule{0.400pt}{0.361pt}}
\put(437.67,723){\rule{0.400pt}{0.723pt}}
\multiput(438.17,724.50)(-1.000,-1.500){2}{\rule{0.400pt}{0.361pt}}
\put(436.67,720){\rule{0.400pt}{0.723pt}}
\multiput(437.17,721.50)(-1.000,-1.500){2}{\rule{0.400pt}{0.361pt}}
\put(435.17,717){\rule{0.400pt}{0.700pt}}
\multiput(436.17,718.55)(-2.000,-1.547){2}{\rule{0.400pt}{0.350pt}}
\put(433.67,714){\rule{0.400pt}{0.723pt}}
\multiput(434.17,715.50)(-1.000,-1.500){2}{\rule{0.400pt}{0.361pt}}
\put(432.67,711){\rule{0.400pt}{0.723pt}}
\multiput(433.17,712.50)(-1.000,-1.500){2}{\rule{0.400pt}{0.361pt}}
\put(431.67,708){\rule{0.400pt}{0.723pt}}
\multiput(432.17,709.50)(-1.000,-1.500){2}{\rule{0.400pt}{0.361pt}}
\put(430.67,705){\rule{0.400pt}{0.723pt}}
\multiput(431.17,706.50)(-1.000,-1.500){2}{\rule{0.400pt}{0.361pt}}
\put(429.67,702){\rule{0.400pt}{0.723pt}}
\multiput(430.17,703.50)(-1.000,-1.500){2}{\rule{0.400pt}{0.361pt}}
\put(428.67,699){\rule{0.400pt}{0.723pt}}
\multiput(429.17,700.50)(-1.000,-1.500){2}{\rule{0.400pt}{0.361pt}}
\put(427.17,696){\rule{0.400pt}{0.700pt}}
\multiput(428.17,697.55)(-2.000,-1.547){2}{\rule{0.400pt}{0.350pt}}
\put(425.67,693){\rule{0.400pt}{0.723pt}}
\multiput(426.17,694.50)(-1.000,-1.500){2}{\rule{0.400pt}{0.361pt}}
\put(424.67,690){\rule{0.400pt}{0.723pt}}
\multiput(425.17,691.50)(-1.000,-1.500){2}{\rule{0.400pt}{0.361pt}}
\put(423.67,687){\rule{0.400pt}{0.723pt}}
\multiput(424.17,688.50)(-1.000,-1.500){2}{\rule{0.400pt}{0.361pt}}
\put(422.67,684){\rule{0.400pt}{0.723pt}}
\multiput(423.17,685.50)(-1.000,-1.500){2}{\rule{0.400pt}{0.361pt}}
\put(421.67,681){\rule{0.400pt}{0.723pt}}
\multiput(422.17,682.50)(-1.000,-1.500){2}{\rule{0.400pt}{0.361pt}}
\put(420.17,678){\rule{0.400pt}{0.700pt}}
\multiput(421.17,679.55)(-2.000,-1.547){2}{\rule{0.400pt}{0.350pt}}
\put(418.67,675){\rule{0.400pt}{0.723pt}}
\multiput(419.17,676.50)(-1.000,-1.500){2}{\rule{0.400pt}{0.361pt}}
\put(417.67,672){\rule{0.400pt}{0.723pt}}
\multiput(418.17,673.50)(-1.000,-1.500){2}{\rule{0.400pt}{0.361pt}}
\put(416.67,670){\rule{0.400pt}{0.482pt}}
\multiput(417.17,671.00)(-1.000,-1.000){2}{\rule{0.400pt}{0.241pt}}
\put(415.67,667){\rule{0.400pt}{0.723pt}}
\multiput(416.17,668.50)(-1.000,-1.500){2}{\rule{0.400pt}{0.361pt}}
\put(414.67,664){\rule{0.400pt}{0.723pt}}
\multiput(415.17,665.50)(-1.000,-1.500){2}{\rule{0.400pt}{0.361pt}}
\put(413.67,661){\rule{0.400pt}{0.723pt}}
\multiput(414.17,662.50)(-1.000,-1.500){2}{\rule{0.400pt}{0.361pt}}
\put(412.17,658){\rule{0.400pt}{0.700pt}}
\multiput(413.17,659.55)(-2.000,-1.547){2}{\rule{0.400pt}{0.350pt}}
\put(410.67,655){\rule{0.400pt}{0.723pt}}
\multiput(411.17,656.50)(-1.000,-1.500){2}{\rule{0.400pt}{0.361pt}}
\put(409.67,652){\rule{0.400pt}{0.723pt}}
\multiput(410.17,653.50)(-1.000,-1.500){2}{\rule{0.400pt}{0.361pt}}
\put(408.67,649){\rule{0.400pt}{0.723pt}}
\multiput(409.17,650.50)(-1.000,-1.500){2}{\rule{0.400pt}{0.361pt}}
\put(407.67,646){\rule{0.400pt}{0.723pt}}
\multiput(408.17,647.50)(-1.000,-1.500){2}{\rule{0.400pt}{0.361pt}}
\put(406.67,643){\rule{0.400pt}{0.723pt}}
\multiput(407.17,644.50)(-1.000,-1.500){2}{\rule{0.400pt}{0.361pt}}
\put(405.67,640){\rule{0.400pt}{0.723pt}}
\multiput(406.17,641.50)(-1.000,-1.500){2}{\rule{0.400pt}{0.361pt}}
\put(404.17,637){\rule{0.400pt}{0.700pt}}
\multiput(405.17,638.55)(-2.000,-1.547){2}{\rule{0.400pt}{0.350pt}}
\put(402.67,634){\rule{0.400pt}{0.723pt}}
\multiput(403.17,635.50)(-1.000,-1.500){2}{\rule{0.400pt}{0.361pt}}
\put(401.67,631){\rule{0.400pt}{0.723pt}}
\multiput(402.17,632.50)(-1.000,-1.500){2}{\rule{0.400pt}{0.361pt}}
\put(400.67,628){\rule{0.400pt}{0.723pt}}
\multiput(401.17,629.50)(-1.000,-1.500){2}{\rule{0.400pt}{0.361pt}}
\put(399.67,625){\rule{0.400pt}{0.723pt}}
\multiput(400.17,626.50)(-1.000,-1.500){2}{\rule{0.400pt}{0.361pt}}
\put(398.67,622){\rule{0.400pt}{0.723pt}}
\multiput(399.17,623.50)(-1.000,-1.500){2}{\rule{0.400pt}{0.361pt}}
\put(397.67,619){\rule{0.400pt}{0.723pt}}
\multiput(398.17,620.50)(-1.000,-1.500){2}{\rule{0.400pt}{0.361pt}}
\put(396.17,616){\rule{0.400pt}{0.700pt}}
\multiput(397.17,617.55)(-2.000,-1.547){2}{\rule{0.400pt}{0.350pt}}
\put(394.67,613){\rule{0.400pt}{0.723pt}}
\multiput(395.17,614.50)(-1.000,-1.500){2}{\rule{0.400pt}{0.361pt}}
\put(393.67,610){\rule{0.400pt}{0.723pt}}
\multiput(394.17,611.50)(-1.000,-1.500){2}{\rule{0.400pt}{0.361pt}}
\put(392.67,608){\rule{0.400pt}{0.482pt}}
\multiput(393.17,609.00)(-1.000,-1.000){2}{\rule{0.400pt}{0.241pt}}
\put(391.67,605){\rule{0.400pt}{0.723pt}}
\multiput(392.17,606.50)(-1.000,-1.500){2}{\rule{0.400pt}{0.361pt}}
\put(390.67,602){\rule{0.400pt}{0.723pt}}
\multiput(391.17,603.50)(-1.000,-1.500){2}{\rule{0.400pt}{0.361pt}}
\put(389.67,599){\rule{0.400pt}{0.723pt}}
\multiput(390.17,600.50)(-1.000,-1.500){2}{\rule{0.400pt}{0.361pt}}
\put(388.17,596){\rule{0.400pt}{0.700pt}}
\multiput(389.17,597.55)(-2.000,-1.547){2}{\rule{0.400pt}{0.350pt}}
\put(386.67,593){\rule{0.400pt}{0.723pt}}
\multiput(387.17,594.50)(-1.000,-1.500){2}{\rule{0.400pt}{0.361pt}}
\put(385.67,590){\rule{0.400pt}{0.723pt}}
\multiput(386.17,591.50)(-1.000,-1.500){2}{\rule{0.400pt}{0.361pt}}
\put(384.67,587){\rule{0.400pt}{0.723pt}}
\multiput(385.17,588.50)(-1.000,-1.500){2}{\rule{0.400pt}{0.361pt}}
\put(383.67,584){\rule{0.400pt}{0.723pt}}
\multiput(384.17,585.50)(-1.000,-1.500){2}{\rule{0.400pt}{0.361pt}}
\put(382.67,581){\rule{0.400pt}{0.723pt}}
\multiput(383.17,582.50)(-1.000,-1.500){2}{\rule{0.400pt}{0.361pt}}
\put(381.17,578){\rule{0.400pt}{0.700pt}}
\multiput(382.17,579.55)(-2.000,-1.547){2}{\rule{0.400pt}{0.350pt}}
\put(379.67,575){\rule{0.400pt}{0.723pt}}
\multiput(380.17,576.50)(-1.000,-1.500){2}{\rule{0.400pt}{0.361pt}}
\put(378.67,572){\rule{0.400pt}{0.723pt}}
\multiput(379.17,573.50)(-1.000,-1.500){2}{\rule{0.400pt}{0.361pt}}
\put(377.67,569){\rule{0.400pt}{0.723pt}}
\multiput(378.17,570.50)(-1.000,-1.500){2}{\rule{0.400pt}{0.361pt}}
\put(376.67,566){\rule{0.400pt}{0.723pt}}
\multiput(377.17,567.50)(-1.000,-1.500){2}{\rule{0.400pt}{0.361pt}}
\put(375.67,563){\rule{0.400pt}{0.723pt}}
\multiput(376.17,564.50)(-1.000,-1.500){2}{\rule{0.400pt}{0.361pt}}
\put(374.67,560){\rule{0.400pt}{0.723pt}}
\multiput(375.17,561.50)(-1.000,-1.500){2}{\rule{0.400pt}{0.361pt}}
\put(373.17,557){\rule{0.400pt}{0.700pt}}
\multiput(374.17,558.55)(-2.000,-1.547){2}{\rule{0.400pt}{0.350pt}}
\put(371.67,554){\rule{0.400pt}{0.723pt}}
\multiput(372.17,555.50)(-1.000,-1.500){2}{\rule{0.400pt}{0.361pt}}
\put(370.67,551){\rule{0.400pt}{0.723pt}}
\multiput(371.17,552.50)(-1.000,-1.500){2}{\rule{0.400pt}{0.361pt}}
\put(369.67,548){\rule{0.400pt}{0.723pt}}
\multiput(370.17,549.50)(-1.000,-1.500){2}{\rule{0.400pt}{0.361pt}}
\put(368.67,545){\rule{0.400pt}{0.723pt}}
\multiput(369.17,546.50)(-1.000,-1.500){2}{\rule{0.400pt}{0.361pt}}
\put(367.67,543){\rule{0.400pt}{0.482pt}}
\multiput(368.17,544.00)(-1.000,-1.000){2}{\rule{0.400pt}{0.241pt}}
\put(366.67,540){\rule{0.400pt}{0.723pt}}
\multiput(367.17,541.50)(-1.000,-1.500){2}{\rule{0.400pt}{0.361pt}}
\put(365.17,537){\rule{0.400pt}{0.700pt}}
\multiput(366.17,538.55)(-2.000,-1.547){2}{\rule{0.400pt}{0.350pt}}
\put(363.67,534){\rule{0.400pt}{0.723pt}}
\multiput(364.17,535.50)(-1.000,-1.500){2}{\rule{0.400pt}{0.361pt}}
\put(362.67,531){\rule{0.400pt}{0.723pt}}
\multiput(363.17,532.50)(-1.000,-1.500){2}{\rule{0.400pt}{0.361pt}}
\put(361.67,528){\rule{0.400pt}{0.723pt}}
\multiput(362.17,529.50)(-1.000,-1.500){2}{\rule{0.400pt}{0.361pt}}
\put(360.67,525){\rule{0.400pt}{0.723pt}}
\multiput(361.17,526.50)(-1.000,-1.500){2}{\rule{0.400pt}{0.361pt}}
\put(359.67,522){\rule{0.400pt}{0.723pt}}
\multiput(360.17,523.50)(-1.000,-1.500){2}{\rule{0.400pt}{0.361pt}}
\put(358.67,519){\rule{0.400pt}{0.723pt}}
\multiput(359.17,520.50)(-1.000,-1.500){2}{\rule{0.400pt}{0.361pt}}
\put(357.17,516){\rule{0.400pt}{0.700pt}}
\multiput(358.17,517.55)(-2.000,-1.547){2}{\rule{0.400pt}{0.350pt}}
\put(355.67,513){\rule{0.400pt}{0.723pt}}
\multiput(356.17,514.50)(-1.000,-1.500){2}{\rule{0.400pt}{0.361pt}}
\put(354.67,510){\rule{0.400pt}{0.723pt}}
\multiput(355.17,511.50)(-1.000,-1.500){2}{\rule{0.400pt}{0.361pt}}
\put(353.67,507){\rule{0.400pt}{0.723pt}}
\multiput(354.17,508.50)(-1.000,-1.500){2}{\rule{0.400pt}{0.361pt}}
\put(352.67,504){\rule{0.400pt}{0.723pt}}
\multiput(353.17,505.50)(-1.000,-1.500){2}{\rule{0.400pt}{0.361pt}}
\put(351.67,501){\rule{0.400pt}{0.723pt}}
\multiput(352.17,502.50)(-1.000,-1.500){2}{\rule{0.400pt}{0.361pt}}
\put(350.67,498){\rule{0.400pt}{0.723pt}}
\multiput(351.17,499.50)(-1.000,-1.500){2}{\rule{0.400pt}{0.361pt}}
\put(349.17,495){\rule{0.400pt}{0.700pt}}
\multiput(350.17,496.55)(-2.000,-1.547){2}{\rule{0.400pt}{0.350pt}}
\put(347.67,492){\rule{0.400pt}{0.723pt}}
\multiput(348.17,493.50)(-1.000,-1.500){2}{\rule{0.400pt}{0.361pt}}
\put(346.67,489){\rule{0.400pt}{0.723pt}}
\multiput(347.17,490.50)(-1.000,-1.500){2}{\rule{0.400pt}{0.361pt}}
\put(345.67,486){\rule{0.400pt}{0.723pt}}
\multiput(346.17,487.50)(-1.000,-1.500){2}{\rule{0.400pt}{0.361pt}}
\put(344.67,483){\rule{0.400pt}{0.723pt}}
\multiput(345.17,484.50)(-1.000,-1.500){2}{\rule{0.400pt}{0.361pt}}
\put(343.67,481){\rule{0.400pt}{0.482pt}}
\multiput(344.17,482.00)(-1.000,-1.000){2}{\rule{0.400pt}{0.241pt}}
\put(342.17,478){\rule{0.400pt}{0.700pt}}
\multiput(343.17,479.55)(-2.000,-1.547){2}{\rule{0.400pt}{0.350pt}}
\put(340.67,475){\rule{0.400pt}{0.723pt}}
\multiput(341.17,476.50)(-1.000,-1.500){2}{\rule{0.400pt}{0.361pt}}
\put(339.67,472){\rule{0.400pt}{0.723pt}}
\multiput(340.17,473.50)(-1.000,-1.500){2}{\rule{0.400pt}{0.361pt}}
\put(338.67,469){\rule{0.400pt}{0.723pt}}
\multiput(339.17,470.50)(-1.000,-1.500){2}{\rule{0.400pt}{0.361pt}}
\put(337.67,466){\rule{0.400pt}{0.723pt}}
\multiput(338.17,467.50)(-1.000,-1.500){2}{\rule{0.400pt}{0.361pt}}
\put(336.67,463){\rule{0.400pt}{0.723pt}}
\multiput(337.17,464.50)(-1.000,-1.500){2}{\rule{0.400pt}{0.361pt}}
\put(335.67,460){\rule{0.400pt}{0.723pt}}
\multiput(336.17,461.50)(-1.000,-1.500){2}{\rule{0.400pt}{0.361pt}}
\put(334.17,457){\rule{0.400pt}{0.700pt}}
\multiput(335.17,458.55)(-2.000,-1.547){2}{\rule{0.400pt}{0.350pt}}
\put(332.67,454){\rule{0.400pt}{0.723pt}}
\multiput(333.17,455.50)(-1.000,-1.500){2}{\rule{0.400pt}{0.361pt}}
\put(331.67,451){\rule{0.400pt}{0.723pt}}
\multiput(332.17,452.50)(-1.000,-1.500){2}{\rule{0.400pt}{0.361pt}}
\put(330.67,448){\rule{0.400pt}{0.723pt}}
\multiput(331.17,449.50)(-1.000,-1.500){2}{\rule{0.400pt}{0.361pt}}
\put(329.67,445){\rule{0.400pt}{0.723pt}}
\multiput(330.17,446.50)(-1.000,-1.500){2}{\rule{0.400pt}{0.361pt}}
\put(328.67,442){\rule{0.400pt}{0.723pt}}
\multiput(329.17,443.50)(-1.000,-1.500){2}{\rule{0.400pt}{0.361pt}}
\put(327.67,439){\rule{0.400pt}{0.723pt}}
\multiput(328.17,440.50)(-1.000,-1.500){2}{\rule{0.400pt}{0.361pt}}
\put(326.17,436){\rule{0.400pt}{0.700pt}}
\multiput(327.17,437.55)(-2.000,-1.547){2}{\rule{0.400pt}{0.350pt}}
\put(324.67,433){\rule{0.400pt}{0.723pt}}
\multiput(325.17,434.50)(-1.000,-1.500){2}{\rule{0.400pt}{0.361pt}}
\put(323.67,430){\rule{0.400pt}{0.723pt}}
\multiput(324.17,431.50)(-1.000,-1.500){2}{\rule{0.400pt}{0.361pt}}
\put(322.67,427){\rule{0.400pt}{0.723pt}}
\multiput(323.17,428.50)(-1.000,-1.500){2}{\rule{0.400pt}{0.361pt}}
\put(321.67,424){\rule{0.400pt}{0.723pt}}
\multiput(322.17,425.50)(-1.000,-1.500){2}{\rule{0.400pt}{0.361pt}}
\put(320.67,421){\rule{0.400pt}{0.723pt}}
\multiput(321.17,422.50)(-1.000,-1.500){2}{\rule{0.400pt}{0.361pt}}
\put(319.67,418){\rule{0.400pt}{0.723pt}}
\multiput(320.17,419.50)(-1.000,-1.500){2}{\rule{0.400pt}{0.361pt}}
\put(318,416.17){\rule{0.482pt}{0.400pt}}
\multiput(319.00,417.17)(-1.000,-2.000){2}{\rule{0.241pt}{0.400pt}}
\put(316.67,413){\rule{0.400pt}{0.723pt}}
\multiput(317.17,414.50)(-1.000,-1.500){2}{\rule{0.400pt}{0.361pt}}
\put(315.67,410){\rule{0.400pt}{0.723pt}}
\multiput(316.17,411.50)(-1.000,-1.500){2}{\rule{0.400pt}{0.361pt}}
\put(314.67,407){\rule{0.400pt}{0.723pt}}
\multiput(315.17,408.50)(-1.000,-1.500){2}{\rule{0.400pt}{0.361pt}}
\put(313.67,404){\rule{0.400pt}{0.723pt}}
\multiput(314.17,405.50)(-1.000,-1.500){2}{\rule{0.400pt}{0.361pt}}
\put(312.67,401){\rule{0.400pt}{0.723pt}}
\multiput(313.17,402.50)(-1.000,-1.500){2}{\rule{0.400pt}{0.361pt}}
\put(311.67,398){\rule{0.400pt}{0.723pt}}
\multiput(312.17,399.50)(-1.000,-1.500){2}{\rule{0.400pt}{0.361pt}}
\put(310.17,395){\rule{0.400pt}{0.700pt}}
\multiput(311.17,396.55)(-2.000,-1.547){2}{\rule{0.400pt}{0.350pt}}
\put(308.67,392){\rule{0.400pt}{0.723pt}}
\multiput(309.17,393.50)(-1.000,-1.500){2}{\rule{0.400pt}{0.361pt}}
\put(307.67,389){\rule{0.400pt}{0.723pt}}
\multiput(308.17,390.50)(-1.000,-1.500){2}{\rule{0.400pt}{0.361pt}}
\put(306.67,386){\rule{0.400pt}{0.723pt}}
\multiput(307.17,387.50)(-1.000,-1.500){2}{\rule{0.400pt}{0.361pt}}
\put(305.67,383){\rule{0.400pt}{0.723pt}}
\multiput(306.17,384.50)(-1.000,-1.500){2}{\rule{0.400pt}{0.361pt}}
\put(304.67,380){\rule{0.400pt}{0.723pt}}
\multiput(305.17,381.50)(-1.000,-1.500){2}{\rule{0.400pt}{0.361pt}}
\put(303.17,377){\rule{0.400pt}{0.700pt}}
\multiput(304.17,378.55)(-2.000,-1.547){2}{\rule{0.400pt}{0.350pt}}
\put(301.67,374){\rule{0.400pt}{0.723pt}}
\multiput(302.17,375.50)(-1.000,-1.500){2}{\rule{0.400pt}{0.361pt}}
\put(300.67,371){\rule{0.400pt}{0.723pt}}
\multiput(301.17,372.50)(-1.000,-1.500){2}{\rule{0.400pt}{0.361pt}}
\put(299.67,368){\rule{0.400pt}{0.723pt}}
\multiput(300.17,369.50)(-1.000,-1.500){2}{\rule{0.400pt}{0.361pt}}
\put(298.67,365){\rule{0.400pt}{0.723pt}}
\multiput(299.17,366.50)(-1.000,-1.500){2}{\rule{0.400pt}{0.361pt}}
\put(297.67,362){\rule{0.400pt}{0.723pt}}
\multiput(298.17,363.50)(-1.000,-1.500){2}{\rule{0.400pt}{0.361pt}}
\put(294,361.67){\rule{0.964pt}{0.400pt}}
\multiput(296.00,361.17)(-2.000,1.000){2}{\rule{0.482pt}{0.400pt}}
\put(609.0,513.0){\rule[-0.200pt]{8.431pt}{0.400pt}}
\put(285,362.67){\rule{0.964pt}{0.400pt}}
\multiput(287.00,362.17)(-2.000,1.000){2}{\rule{0.482pt}{0.400pt}}
\put(289.0,363.0){\rule[-0.200pt]{1.204pt}{0.400pt}}
\put(277,363.67){\rule{0.964pt}{0.400pt}}
\multiput(279.00,363.17)(-2.000,1.000){2}{\rule{0.482pt}{0.400pt}}
\put(281.0,364.0){\rule[-0.200pt]{0.964pt}{0.400pt}}
\put(264,364.67){\rule{0.964pt}{0.400pt}}
\multiput(266.00,364.17)(-2.000,1.000){2}{\rule{0.482pt}{0.400pt}}
\put(268.0,365.0){\rule[-0.200pt]{2.168pt}{0.400pt}}
\put(256,365.67){\rule{0.964pt}{0.400pt}}
\multiput(258.00,365.17)(-2.000,1.000){2}{\rule{0.482pt}{0.400pt}}
\put(260.0,366.0){\rule[-0.200pt]{0.964pt}{0.400pt}}
\put(247,366.67){\rule{1.204pt}{0.400pt}}
\multiput(249.50,366.17)(-2.500,1.000){2}{\rule{0.602pt}{0.400pt}}
\put(252.0,367.0){\rule[-0.200pt]{0.964pt}{0.400pt}}
\put(239,367.67){\rule{0.964pt}{0.400pt}}
\multiput(241.00,367.17)(-2.000,1.000){2}{\rule{0.482pt}{0.400pt}}
\put(243.0,368.0){\rule[-0.200pt]{0.964pt}{0.400pt}}
\put(226,368.67){\rule{1.204pt}{0.400pt}}
\multiput(228.50,368.17)(-2.500,1.000){2}{\rule{0.602pt}{0.400pt}}
\put(231.0,369.0){\rule[-0.200pt]{1.927pt}{0.400pt}}
\put(218,369.67){\rule{0.964pt}{0.400pt}}
\multiput(220.00,369.17)(-2.000,1.000){2}{\rule{0.482pt}{0.400pt}}
\put(222.0,370.0){\rule[-0.200pt]{0.964pt}{0.400pt}}
\put(210,370.67){\rule{0.964pt}{0.400pt}}
\multiput(212.00,370.17)(-2.000,1.000){2}{\rule{0.482pt}{0.400pt}}
\put(214.0,371.0){\rule[-0.200pt]{0.964pt}{0.400pt}}
\put(197,371.67){\rule{0.964pt}{0.400pt}}
\multiput(199.00,371.17)(-2.000,1.000){2}{\rule{0.482pt}{0.400pt}}
\put(201.0,372.0){\rule[-0.200pt]{2.168pt}{0.400pt}}
\put(189,372.67){\rule{0.964pt}{0.400pt}}
\multiput(191.00,372.17)(-2.000,1.000){2}{\rule{0.482pt}{0.400pt}}
\put(193.0,373.0){\rule[-0.200pt]{0.964pt}{0.400pt}}
\put(180,373.67){\rule{1.204pt}{0.400pt}}
\multiput(182.50,373.17)(-2.500,1.000){2}{\rule{0.602pt}{0.400pt}}
\put(185.0,374.0){\rule[-0.200pt]{0.964pt}{0.400pt}}
\put(172,374.67){\rule{0.964pt}{0.400pt}}
\multiput(174.00,374.17)(-2.000,1.000){2}{\rule{0.482pt}{0.400pt}}
\put(176.0,375.0){\rule[-0.200pt]{0.964pt}{0.400pt}}
\put(159,375.67){\rule{1.204pt}{0.400pt}}
\multiput(161.50,375.17)(-2.500,1.000){2}{\rule{0.602pt}{0.400pt}}
\put(164.0,376.0){\rule[-0.200pt]{1.927pt}{0.400pt}}
\put(155,376.67){\rule{0.964pt}{0.400pt}}
\multiput(155.00,376.17)(2.000,1.000){2}{\rule{0.482pt}{0.400pt}}
\put(155.0,377.0){\rule[-0.200pt]{0.964pt}{0.400pt}}
\put(164,377.67){\rule{0.964pt}{0.400pt}}
\multiput(164.00,377.17)(2.000,1.000){2}{\rule{0.482pt}{0.400pt}}
\put(159.0,378.0){\rule[-0.200pt]{1.204pt}{0.400pt}}
\put(176,378.67){\rule{0.964pt}{0.400pt}}
\multiput(176.00,378.17)(2.000,1.000){2}{\rule{0.482pt}{0.400pt}}
\put(168.0,379.0){\rule[-0.200pt]{1.927pt}{0.400pt}}
\put(185,379.67){\rule{0.964pt}{0.400pt}}
\multiput(185.00,379.17)(2.000,1.000){2}{\rule{0.482pt}{0.400pt}}
\put(180.0,380.0){\rule[-0.200pt]{1.204pt}{0.400pt}}
\put(193,380.67){\rule{0.964pt}{0.400pt}}
\multiput(193.00,380.17)(2.000,1.000){2}{\rule{0.482pt}{0.400pt}}
\put(189.0,381.0){\rule[-0.200pt]{0.964pt}{0.400pt}}
\put(201,381.67){\rule{0.964pt}{0.400pt}}
\multiput(201.00,381.17)(2.000,1.000){2}{\rule{0.482pt}{0.400pt}}
\put(197.0,382.0){\rule[-0.200pt]{0.964pt}{0.400pt}}
\put(214,382.67){\rule{0.964pt}{0.400pt}}
\multiput(214.00,382.17)(2.000,1.000){2}{\rule{0.482pt}{0.400pt}}
\put(205.0,383.0){\rule[-0.200pt]{2.168pt}{0.400pt}}
\put(222,383.67){\rule{0.964pt}{0.400pt}}
\multiput(222.00,383.17)(2.000,1.000){2}{\rule{0.482pt}{0.400pt}}
\put(218.0,384.0){\rule[-0.200pt]{0.964pt}{0.400pt}}
\put(231,384.67){\rule{0.964pt}{0.400pt}}
\multiput(231.00,384.17)(2.000,1.000){2}{\rule{0.482pt}{0.400pt}}
\put(226.0,385.0){\rule[-0.200pt]{1.204pt}{0.400pt}}
\put(243,385.67){\rule{0.964pt}{0.400pt}}
\multiput(243.00,385.17)(2.000,1.000){2}{\rule{0.482pt}{0.400pt}}
\put(235.0,386.0){\rule[-0.200pt]{1.927pt}{0.400pt}}
\put(252,386.67){\rule{0.964pt}{0.400pt}}
\multiput(252.00,386.17)(2.000,1.000){2}{\rule{0.482pt}{0.400pt}}
\put(247.0,387.0){\rule[-0.200pt]{1.204pt}{0.400pt}}
\put(260,387.67){\rule{0.964pt}{0.400pt}}
\multiput(260.00,387.17)(2.000,1.000){2}{\rule{0.482pt}{0.400pt}}
\put(256.0,388.0){\rule[-0.200pt]{0.964pt}{0.400pt}}
\put(268,388.67){\rule{1.204pt}{0.400pt}}
\multiput(268.00,388.17)(2.500,1.000){2}{\rule{0.602pt}{0.400pt}}
\put(264.0,389.0){\rule[-0.200pt]{0.964pt}{0.400pt}}
\put(281,389.67){\rule{0.964pt}{0.400pt}}
\multiput(281.00,389.17)(2.000,1.000){2}{\rule{0.482pt}{0.400pt}}
\put(273.0,390.0){\rule[-0.200pt]{1.927pt}{0.400pt}}
\put(289,390.67){\rule{1.204pt}{0.400pt}}
\multiput(289.00,390.17)(2.500,1.000){2}{\rule{0.602pt}{0.400pt}}
\put(285.0,391.0){\rule[-0.200pt]{0.964pt}{0.400pt}}
\put(298,391.67){\rule{0.964pt}{0.400pt}}
\multiput(298.00,391.17)(2.000,1.000){2}{\rule{0.482pt}{0.400pt}}
\put(294.0,392.0){\rule[-0.200pt]{0.964pt}{0.400pt}}
\put(310,392.67){\rule{1.204pt}{0.400pt}}
\multiput(310.00,392.17)(2.500,1.000){2}{\rule{0.602pt}{0.400pt}}
\put(302.0,393.0){\rule[-0.200pt]{1.927pt}{0.400pt}}
\put(319,393.67){\rule{0.964pt}{0.400pt}}
\multiput(319.00,393.17)(2.000,1.000){2}{\rule{0.482pt}{0.400pt}}
\put(315.0,394.0){\rule[-0.200pt]{0.964pt}{0.400pt}}
\put(327,394.67){\rule{0.964pt}{0.400pt}}
\multiput(327.00,394.17)(2.000,1.000){2}{\rule{0.482pt}{0.400pt}}
\put(323.0,395.0){\rule[-0.200pt]{0.964pt}{0.400pt}}
\put(335,395.67){\rule{1.204pt}{0.400pt}}
\multiput(335.00,395.17)(2.500,1.000){2}{\rule{0.602pt}{0.400pt}}
\put(331.0,396.0){\rule[-0.200pt]{0.964pt}{0.400pt}}
\put(348,396.67){\rule{0.964pt}{0.400pt}}
\multiput(348.00,396.17)(2.000,1.000){2}{\rule{0.482pt}{0.400pt}}
\put(340.0,397.0){\rule[-0.200pt]{1.927pt}{0.400pt}}
\put(356,397.67){\rule{1.204pt}{0.400pt}}
\multiput(356.00,397.17)(2.500,1.000){2}{\rule{0.602pt}{0.400pt}}
\put(352.0,398.0){\rule[-0.200pt]{0.964pt}{0.400pt}}
\put(365,398.67){\rule{0.964pt}{0.400pt}}
\multiput(365.00,398.17)(2.000,1.000){2}{\rule{0.482pt}{0.400pt}}
\put(361.0,399.0){\rule[-0.200pt]{0.964pt}{0.400pt}}
\put(377,399.67){\rule{1.204pt}{0.400pt}}
\multiput(377.00,399.17)(2.500,1.000){2}{\rule{0.602pt}{0.400pt}}
\put(369.0,400.0){\rule[-0.200pt]{1.927pt}{0.400pt}}
\put(386,400.67){\rule{0.964pt}{0.400pt}}
\multiput(386.00,400.17)(2.000,1.000){2}{\rule{0.482pt}{0.400pt}}
\put(382.0,401.0){\rule[-0.200pt]{0.964pt}{0.400pt}}
\put(394,401.67){\rule{0.964pt}{0.400pt}}
\multiput(394.00,401.17)(2.000,1.000){2}{\rule{0.482pt}{0.400pt}}
\put(390.0,402.0){\rule[-0.200pt]{0.964pt}{0.400pt}}
\put(403,402.67){\rule{0.964pt}{0.400pt}}
\multiput(403.00,402.17)(2.000,1.000){2}{\rule{0.482pt}{0.400pt}}
\put(398.0,403.0){\rule[-0.200pt]{1.204pt}{0.400pt}}
\put(415,403.67){\rule{0.964pt}{0.400pt}}
\multiput(415.00,403.17)(2.000,1.000){2}{\rule{0.482pt}{0.400pt}}
\put(407.0,404.0){\rule[-0.200pt]{1.927pt}{0.400pt}}
\put(424,404.67){\rule{0.964pt}{0.400pt}}
\multiput(424.00,404.17)(2.000,1.000){2}{\rule{0.482pt}{0.400pt}}
\put(419.0,405.0){\rule[-0.200pt]{1.204pt}{0.400pt}}
\put(432,405.67){\rule{0.964pt}{0.400pt}}
\multiput(432.00,405.17)(2.000,1.000){2}{\rule{0.482pt}{0.400pt}}
\put(428.0,406.0){\rule[-0.200pt]{0.964pt}{0.400pt}}
\put(445,406.67){\rule{0.964pt}{0.400pt}}
\multiput(445.00,406.17)(2.000,1.000){2}{\rule{0.482pt}{0.400pt}}
\put(436.0,407.0){\rule[-0.200pt]{2.168pt}{0.400pt}}
\put(453,407.67){\rule{0.964pt}{0.400pt}}
\multiput(453.00,407.17)(2.000,1.000){2}{\rule{0.482pt}{0.400pt}}
\put(449.0,408.0){\rule[-0.200pt]{0.964pt}{0.400pt}}
\put(461,408.67){\rule{0.964pt}{0.400pt}}
\multiput(461.00,408.17)(2.000,1.000){2}{\rule{0.482pt}{0.400pt}}
\put(457.0,409.0){\rule[-0.200pt]{0.964pt}{0.400pt}}
\put(470,409.67){\rule{0.964pt}{0.400pt}}
\multiput(470.00,409.17)(2.000,1.000){2}{\rule{0.482pt}{0.400pt}}
\put(465.0,410.0){\rule[-0.200pt]{1.204pt}{0.400pt}}
\put(482,410.67){\rule{0.964pt}{0.400pt}}
\multiput(482.00,410.17)(2.000,1.000){2}{\rule{0.482pt}{0.400pt}}
\put(474.0,411.0){\rule[-0.200pt]{1.927pt}{0.400pt}}
\put(491,411.67){\rule{0.964pt}{0.400pt}}
\multiput(491.00,411.17)(2.000,1.000){2}{\rule{0.482pt}{0.400pt}}
\put(486.0,412.0){\rule[-0.200pt]{1.204pt}{0.400pt}}
\put(499,412.67){\rule{0.964pt}{0.400pt}}
\multiput(499.00,412.17)(2.000,1.000){2}{\rule{0.482pt}{0.400pt}}
\put(495.0,413.0){\rule[-0.200pt]{0.964pt}{0.400pt}}
\put(507,413.67){\rule{1.204pt}{0.400pt}}
\multiput(507.00,413.17)(2.500,1.000){2}{\rule{0.602pt}{0.400pt}}
\put(503.0,414.0){\rule[-0.200pt]{0.964pt}{0.400pt}}
\put(520,414.67){\rule{0.964pt}{0.400pt}}
\multiput(520.00,414.17)(2.000,1.000){2}{\rule{0.482pt}{0.400pt}}
\put(512.0,415.0){\rule[-0.200pt]{1.927pt}{0.400pt}}
\put(528,415.67){\rule{1.204pt}{0.400pt}}
\multiput(528.00,415.17)(2.500,1.000){2}{\rule{0.602pt}{0.400pt}}
\put(524.0,416.0){\rule[-0.200pt]{0.964pt}{0.400pt}}
\put(537,416.67){\rule{0.964pt}{0.400pt}}
\multiput(537.00,416.17)(2.000,1.000){2}{\rule{0.482pt}{0.400pt}}
\put(533.0,417.0){\rule[-0.200pt]{0.964pt}{0.400pt}}
\put(549,417.67){\rule{1.204pt}{0.400pt}}
\multiput(549.00,417.17)(2.500,1.000){2}{\rule{0.602pt}{0.400pt}}
\put(541.0,418.0){\rule[-0.200pt]{1.927pt}{0.400pt}}
\put(558,418.67){\rule{0.964pt}{0.400pt}}
\multiput(558.00,418.17)(2.000,1.000){2}{\rule{0.482pt}{0.400pt}}
\put(554.0,419.0){\rule[-0.200pt]{0.964pt}{0.400pt}}
\put(566,419.67){\rule{0.964pt}{0.400pt}}
\multiput(566.00,419.17)(2.000,1.000){2}{\rule{0.482pt}{0.400pt}}
\put(562.0,420.0){\rule[-0.200pt]{0.964pt}{0.400pt}}
\put(575,420.67){\rule{0.964pt}{0.400pt}}
\multiput(575.00,420.17)(2.000,1.000){2}{\rule{0.482pt}{0.400pt}}
\put(570.0,421.0){\rule[-0.200pt]{1.204pt}{0.400pt}}
\put(587,421.67){\rule{0.964pt}{0.400pt}}
\multiput(587.00,421.17)(2.000,1.000){2}{\rule{0.482pt}{0.400pt}}
\put(579.0,422.0){\rule[-0.200pt]{1.927pt}{0.400pt}}
\put(595,422.67){\rule{1.204pt}{0.400pt}}
\multiput(595.00,422.17)(2.500,1.000){2}{\rule{0.602pt}{0.400pt}}
\put(591.0,423.0){\rule[-0.200pt]{0.964pt}{0.400pt}}
\put(604,423.67){\rule{0.964pt}{0.400pt}}
\multiput(604.00,423.17)(2.000,1.000){2}{\rule{0.482pt}{0.400pt}}
\put(600.0,424.0){\rule[-0.200pt]{0.964pt}{0.400pt}}
\put(616,424.67){\rule{1.204pt}{0.400pt}}
\multiput(616.00,424.17)(2.500,1.000){2}{\rule{0.602pt}{0.400pt}}
\put(608.0,425.0){\rule[-0.200pt]{1.927pt}{0.400pt}}
\put(625,425.67){\rule{0.964pt}{0.400pt}}
\multiput(625.00,425.17)(2.000,1.000){2}{\rule{0.482pt}{0.400pt}}
\put(621.0,426.0){\rule[-0.200pt]{0.964pt}{0.400pt}}
\put(633,426.67){\rule{0.964pt}{0.400pt}}
\multiput(633.00,426.17)(2.000,1.000){2}{\rule{0.482pt}{0.400pt}}
\put(629.0,427.0){\rule[-0.200pt]{0.964pt}{0.400pt}}
\put(642,427.67){\rule{0.964pt}{0.400pt}}
\multiput(642.00,427.17)(2.000,1.000){2}{\rule{0.482pt}{0.400pt}}
\put(637.0,428.0){\rule[-0.200pt]{1.204pt}{0.400pt}}
\put(654,428.67){\rule{0.964pt}{0.400pt}}
\multiput(654.00,428.17)(2.000,1.000){2}{\rule{0.482pt}{0.400pt}}
\put(646.0,429.0){\rule[-0.200pt]{1.927pt}{0.400pt}}
\put(663,429.67){\rule{0.964pt}{0.400pt}}
\multiput(663.00,429.17)(2.000,1.000){2}{\rule{0.482pt}{0.400pt}}
\put(658.0,430.0){\rule[-0.200pt]{1.204pt}{0.400pt}}
\put(671,430.67){\rule{0.964pt}{0.400pt}}
\multiput(671.00,430.17)(2.000,1.000){2}{\rule{0.482pt}{0.400pt}}
\put(667.0,431.0){\rule[-0.200pt]{0.964pt}{0.400pt}}
\put(684,431.67){\rule{0.964pt}{0.400pt}}
\multiput(684.00,431.17)(2.000,1.000){2}{\rule{0.482pt}{0.400pt}}
\put(675.0,432.0){\rule[-0.200pt]{2.168pt}{0.400pt}}
\put(692,432.67){\rule{0.964pt}{0.400pt}}
\multiput(692.00,432.17)(2.000,1.000){2}{\rule{0.482pt}{0.400pt}}
\put(688.0,433.0){\rule[-0.200pt]{0.964pt}{0.400pt}}
\put(700,433.67){\rule{1.204pt}{0.400pt}}
\multiput(700.00,433.17)(2.500,1.000){2}{\rule{0.602pt}{0.400pt}}
\put(696.0,434.0){\rule[-0.200pt]{0.964pt}{0.400pt}}
\put(709,434.67){\rule{0.964pt}{0.400pt}}
\multiput(709.00,434.17)(2.000,1.000){2}{\rule{0.482pt}{0.400pt}}
\put(705.0,435.0){\rule[-0.200pt]{0.964pt}{0.400pt}}
\put(721,435.67){\rule{0.964pt}{0.400pt}}
\multiput(721.00,435.17)(2.000,1.000){2}{\rule{0.482pt}{0.400pt}}
\put(713.0,436.0){\rule[-0.200pt]{1.927pt}{0.400pt}}
\put(730,436.67){\rule{0.964pt}{0.400pt}}
\multiput(730.00,436.17)(2.000,1.000){2}{\rule{0.482pt}{0.400pt}}
\put(725.0,437.0){\rule[-0.200pt]{1.204pt}{0.400pt}}
\put(738,437.67){\rule{0.964pt}{0.400pt}}
\multiput(738.00,437.17)(2.000,1.000){2}{\rule{0.482pt}{0.400pt}}
\put(742,439.17){\rule{1.300pt}{0.400pt}}
\multiput(742.00,438.17)(3.302,2.000){2}{\rule{0.650pt}{0.400pt}}
\put(748,441.17){\rule{1.100pt}{0.400pt}}
\multiput(748.00,440.17)(2.717,2.000){2}{\rule{0.550pt}{0.400pt}}
\put(753,443.17){\rule{1.100pt}{0.400pt}}
\multiput(753.00,442.17)(2.717,2.000){2}{\rule{0.550pt}{0.400pt}}
\put(758,445.17){\rule{1.100pt}{0.400pt}}
\multiput(758.00,444.17)(2.717,2.000){2}{\rule{0.550pt}{0.400pt}}
\put(763,447.17){\rule{1.300pt}{0.400pt}}
\multiput(763.00,446.17)(3.302,2.000){2}{\rule{0.650pt}{0.400pt}}
\put(769,449.17){\rule{1.100pt}{0.400pt}}
\multiput(769.00,448.17)(2.717,2.000){2}{\rule{0.550pt}{0.400pt}}
\put(774,451.17){\rule{1.100pt}{0.400pt}}
\multiput(774.00,450.17)(2.717,2.000){2}{\rule{0.550pt}{0.400pt}}
\multiput(779.00,453.61)(1.132,0.447){3}{\rule{0.900pt}{0.108pt}}
\multiput(779.00,452.17)(4.132,3.000){2}{\rule{0.450pt}{0.400pt}}
\put(785,456.17){\rule{1.100pt}{0.400pt}}
\multiput(785.00,455.17)(2.717,2.000){2}{\rule{0.550pt}{0.400pt}}
\put(790,458.17){\rule{1.100pt}{0.400pt}}
\multiput(790.00,457.17)(2.717,2.000){2}{\rule{0.550pt}{0.400pt}}
\put(795,460.17){\rule{1.100pt}{0.400pt}}
\multiput(795.00,459.17)(2.717,2.000){2}{\rule{0.550pt}{0.400pt}}
\put(800,462.17){\rule{1.300pt}{0.400pt}}
\multiput(800.00,461.17)(3.302,2.000){2}{\rule{0.650pt}{0.400pt}}
\put(806,464.17){\rule{1.100pt}{0.400pt}}
\multiput(806.00,463.17)(2.717,2.000){2}{\rule{0.550pt}{0.400pt}}
\put(811,466.17){\rule{1.100pt}{0.400pt}}
\multiput(811.00,465.17)(2.717,2.000){2}{\rule{0.550pt}{0.400pt}}
\put(816,468.17){\rule{1.300pt}{0.400pt}}
\multiput(816.00,467.17)(3.302,2.000){2}{\rule{0.650pt}{0.400pt}}
\multiput(822.00,470.61)(0.909,0.447){3}{\rule{0.767pt}{0.108pt}}
\multiput(822.00,469.17)(3.409,3.000){2}{\rule{0.383pt}{0.400pt}}
\put(827,473.17){\rule{1.100pt}{0.400pt}}
\multiput(827.00,472.17)(2.717,2.000){2}{\rule{0.550pt}{0.400pt}}
\put(832,475.17){\rule{1.100pt}{0.400pt}}
\multiput(832.00,474.17)(2.717,2.000){2}{\rule{0.550pt}{0.400pt}}
\put(837,477.17){\rule{1.300pt}{0.400pt}}
\multiput(837.00,476.17)(3.302,2.000){2}{\rule{0.650pt}{0.400pt}}
\put(843,479.17){\rule{1.100pt}{0.400pt}}
\multiput(843.00,478.17)(2.717,2.000){2}{\rule{0.550pt}{0.400pt}}
\put(848,481.17){\rule{1.100pt}{0.400pt}}
\multiput(848.00,480.17)(2.717,2.000){2}{\rule{0.550pt}{0.400pt}}
\put(853,483.17){\rule{1.300pt}{0.400pt}}
\multiput(853.00,482.17)(3.302,2.000){2}{\rule{0.650pt}{0.400pt}}
\put(859,485.17){\rule{1.100pt}{0.400pt}}
\multiput(859.00,484.17)(2.717,2.000){2}{\rule{0.550pt}{0.400pt}}
\multiput(864.00,487.61)(0.909,0.447){3}{\rule{0.767pt}{0.108pt}}
\multiput(864.00,486.17)(3.409,3.000){2}{\rule{0.383pt}{0.400pt}}
\put(869,490.17){\rule{1.100pt}{0.400pt}}
\multiput(869.00,489.17)(2.717,2.000){2}{\rule{0.550pt}{0.400pt}}
\put(874,492.17){\rule{1.300pt}{0.400pt}}
\multiput(874.00,491.17)(3.302,2.000){2}{\rule{0.650pt}{0.400pt}}
\put(880,494.17){\rule{1.100pt}{0.400pt}}
\multiput(880.00,493.17)(2.717,2.000){2}{\rule{0.550pt}{0.400pt}}
\put(885,496.17){\rule{1.100pt}{0.400pt}}
\multiput(885.00,495.17)(2.717,2.000){2}{\rule{0.550pt}{0.400pt}}
\put(890,498.17){\rule{1.300pt}{0.400pt}}
\multiput(890.00,497.17)(3.302,2.000){2}{\rule{0.650pt}{0.400pt}}
\put(896,500.17){\rule{1.100pt}{0.400pt}}
\multiput(896.00,499.17)(2.717,2.000){2}{\rule{0.550pt}{0.400pt}}
\multiput(901.00,502.61)(0.909,0.447){3}{\rule{0.767pt}{0.108pt}}
\multiput(901.00,501.17)(3.409,3.000){2}{\rule{0.383pt}{0.400pt}}
\put(906,505.17){\rule{1.100pt}{0.400pt}}
\multiput(906.00,504.17)(2.717,2.000){2}{\rule{0.550pt}{0.400pt}}
\put(911,507.17){\rule{1.300pt}{0.400pt}}
\multiput(911.00,506.17)(3.302,2.000){2}{\rule{0.650pt}{0.400pt}}
\put(917,509.17){\rule{1.100pt}{0.400pt}}
\multiput(917.00,508.17)(2.717,2.000){2}{\rule{0.550pt}{0.400pt}}
\put(922,511.17){\rule{1.100pt}{0.400pt}}
\multiput(922.00,510.17)(2.717,2.000){2}{\rule{0.550pt}{0.400pt}}
\put(927,513.17){\rule{1.300pt}{0.400pt}}
\multiput(927.00,512.17)(3.302,2.000){2}{\rule{0.650pt}{0.400pt}}
\put(933,515.17){\rule{1.100pt}{0.400pt}}
\multiput(933.00,514.17)(2.717,2.000){2}{\rule{0.550pt}{0.400pt}}
\put(938,517.17){\rule{1.100pt}{0.400pt}}
\multiput(938.00,516.17)(2.717,2.000){2}{\rule{0.550pt}{0.400pt}}
\multiput(943.00,519.61)(0.909,0.447){3}{\rule{0.767pt}{0.108pt}}
\multiput(943.00,518.17)(3.409,3.000){2}{\rule{0.383pt}{0.400pt}}
\put(948,522.17){\rule{1.300pt}{0.400pt}}
\multiput(948.00,521.17)(3.302,2.000){2}{\rule{0.650pt}{0.400pt}}
\put(954,524.17){\rule{1.100pt}{0.400pt}}
\multiput(954.00,523.17)(2.717,2.000){2}{\rule{0.550pt}{0.400pt}}
\put(959,526.17){\rule{1.100pt}{0.400pt}}
\multiput(959.00,525.17)(2.717,2.000){2}{\rule{0.550pt}{0.400pt}}
\put(964,528.17){\rule{1.300pt}{0.400pt}}
\multiput(964.00,527.17)(3.302,2.000){2}{\rule{0.650pt}{0.400pt}}
\put(970,530.17){\rule{1.100pt}{0.400pt}}
\multiput(970.00,529.17)(2.717,2.000){2}{\rule{0.550pt}{0.400pt}}
\put(975,532.17){\rule{1.100pt}{0.400pt}}
\multiput(975.00,531.17)(2.717,2.000){2}{\rule{0.550pt}{0.400pt}}
\put(980,534.17){\rule{1.100pt}{0.400pt}}
\multiput(980.00,533.17)(2.717,2.000){2}{\rule{0.550pt}{0.400pt}}
\multiput(985.00,536.61)(1.132,0.447){3}{\rule{0.900pt}{0.108pt}}
\multiput(985.00,535.17)(4.132,3.000){2}{\rule{0.450pt}{0.400pt}}
\put(991,539.17){\rule{1.100pt}{0.400pt}}
\multiput(991.00,538.17)(2.717,2.000){2}{\rule{0.550pt}{0.400pt}}
\put(996,541.17){\rule{1.100pt}{0.400pt}}
\multiput(996.00,540.17)(2.717,2.000){2}{\rule{0.550pt}{0.400pt}}
\put(1001,543.17){\rule{1.300pt}{0.400pt}}
\multiput(1001.00,542.17)(3.302,2.000){2}{\rule{0.650pt}{0.400pt}}
\put(1007,545.17){\rule{1.100pt}{0.400pt}}
\multiput(1007.00,544.17)(2.717,2.000){2}{\rule{0.550pt}{0.400pt}}
\put(1012,547.17){\rule{1.100pt}{0.400pt}}
\multiput(1012.00,546.17)(2.717,2.000){2}{\rule{0.550pt}{0.400pt}}
\put(1017,549.17){\rule{1.100pt}{0.400pt}}
\multiput(1017.00,548.17)(2.717,2.000){2}{\rule{0.550pt}{0.400pt}}
\put(1022,551.17){\rule{1.300pt}{0.400pt}}
\multiput(1022.00,550.17)(3.302,2.000){2}{\rule{0.650pt}{0.400pt}}
\multiput(1028.00,553.61)(0.909,0.447){3}{\rule{0.767pt}{0.108pt}}
\multiput(1028.00,552.17)(3.409,3.000){2}{\rule{0.383pt}{0.400pt}}
\put(1033,556.17){\rule{1.100pt}{0.400pt}}
\multiput(1033.00,555.17)(2.717,2.000){2}{\rule{0.550pt}{0.400pt}}
\put(1038,558.17){\rule{1.300pt}{0.400pt}}
\multiput(1038.00,557.17)(3.302,2.000){2}{\rule{0.650pt}{0.400pt}}
\put(1044,560.17){\rule{1.100pt}{0.400pt}}
\multiput(1044.00,559.17)(2.717,2.000){2}{\rule{0.550pt}{0.400pt}}
\put(1049,562.17){\rule{1.100pt}{0.400pt}}
\multiput(1049.00,561.17)(2.717,2.000){2}{\rule{0.550pt}{0.400pt}}
\put(1054,564.17){\rule{1.100pt}{0.400pt}}
\multiput(1054.00,563.17)(2.717,2.000){2}{\rule{0.550pt}{0.400pt}}
\put(1059,566.17){\rule{1.300pt}{0.400pt}}
\multiput(1059.00,565.17)(3.302,2.000){2}{\rule{0.650pt}{0.400pt}}
\put(1065,568.17){\rule{1.100pt}{0.400pt}}
\multiput(1065.00,567.17)(2.717,2.000){2}{\rule{0.550pt}{0.400pt}}
\multiput(1070.00,570.61)(0.909,0.447){3}{\rule{0.767pt}{0.108pt}}
\multiput(1070.00,569.17)(3.409,3.000){2}{\rule{0.383pt}{0.400pt}}
\put(1075,573.17){\rule{1.300pt}{0.400pt}}
\multiput(1075.00,572.17)(3.302,2.000){2}{\rule{0.650pt}{0.400pt}}
\put(1081,575.17){\rule{1.100pt}{0.400pt}}
\multiput(1081.00,574.17)(2.717,2.000){2}{\rule{0.550pt}{0.400pt}}
\put(1086,577.17){\rule{1.100pt}{0.400pt}}
\multiput(1086.00,576.17)(2.717,2.000){2}{\rule{0.550pt}{0.400pt}}
\put(1091,579.17){\rule{1.100pt}{0.400pt}}
\multiput(1091.00,578.17)(2.717,2.000){2}{\rule{0.550pt}{0.400pt}}
\put(1096,581.17){\rule{1.300pt}{0.400pt}}
\multiput(1096.00,580.17)(3.302,2.000){2}{\rule{0.650pt}{0.400pt}}
\put(1102,583.17){\rule{1.100pt}{0.400pt}}
\multiput(1102.00,582.17)(2.717,2.000){2}{\rule{0.550pt}{0.400pt}}
\put(1107,585.17){\rule{1.100pt}{0.400pt}}
\multiput(1107.00,584.17)(2.717,2.000){2}{\rule{0.550pt}{0.400pt}}
\multiput(1112.00,587.61)(1.132,0.447){3}{\rule{0.900pt}{0.108pt}}
\multiput(1112.00,586.17)(4.132,3.000){2}{\rule{0.450pt}{0.400pt}}
\put(1118,590.17){\rule{1.100pt}{0.400pt}}
\multiput(1118.00,589.17)(2.717,2.000){2}{\rule{0.550pt}{0.400pt}}
\put(1123,592.17){\rule{1.100pt}{0.400pt}}
\multiput(1123.00,591.17)(2.717,2.000){2}{\rule{0.550pt}{0.400pt}}
\put(1128,594.17){\rule{1.100pt}{0.400pt}}
\multiput(1128.00,593.17)(2.717,2.000){2}{\rule{0.550pt}{0.400pt}}
\put(1133,596.17){\rule{1.300pt}{0.400pt}}
\multiput(1133.00,595.17)(3.302,2.000){2}{\rule{0.650pt}{0.400pt}}
\put(1139,598.17){\rule{1.100pt}{0.400pt}}
\multiput(1139.00,597.17)(2.717,2.000){2}{\rule{0.550pt}{0.400pt}}
\put(1144,600.17){\rule{1.100pt}{0.400pt}}
\multiput(1144.00,599.17)(2.717,2.000){2}{\rule{0.550pt}{0.400pt}}
\put(1149,602.17){\rule{1.100pt}{0.400pt}}
\multiput(1149.00,601.17)(2.717,2.000){2}{\rule{0.550pt}{0.400pt}}
\multiput(1154.00,604.61)(1.132,0.447){3}{\rule{0.900pt}{0.108pt}}
\multiput(1154.00,603.17)(4.132,3.000){2}{\rule{0.450pt}{0.400pt}}
\put(1160,607.17){\rule{1.100pt}{0.400pt}}
\multiput(1160.00,606.17)(2.717,2.000){2}{\rule{0.550pt}{0.400pt}}
\put(1165,609.17){\rule{1.100pt}{0.400pt}}
\multiput(1165.00,608.17)(2.717,2.000){2}{\rule{0.550pt}{0.400pt}}
\put(1170,611.17){\rule{1.300pt}{0.400pt}}
\multiput(1170.00,610.17)(3.302,2.000){2}{\rule{0.650pt}{0.400pt}}
\put(1176,613.17){\rule{1.100pt}{0.400pt}}
\multiput(1176.00,612.17)(2.717,2.000){2}{\rule{0.550pt}{0.400pt}}
\put(1181,615.17){\rule{1.100pt}{0.400pt}}
\multiput(1181.00,614.17)(2.717,2.000){2}{\rule{0.550pt}{0.400pt}}
\put(1186,617.17){\rule{1.100pt}{0.400pt}}
\multiput(1186.00,616.17)(2.717,2.000){2}{\rule{0.550pt}{0.400pt}}
\put(1191,619.17){\rule{1.300pt}{0.400pt}}
\multiput(1191.00,618.17)(3.302,2.000){2}{\rule{0.650pt}{0.400pt}}
\multiput(1197.00,621.61)(0.909,0.447){3}{\rule{0.767pt}{0.108pt}}
\multiput(1197.00,620.17)(3.409,3.000){2}{\rule{0.383pt}{0.400pt}}
\put(1202,624.17){\rule{1.100pt}{0.400pt}}
\multiput(1202.00,623.17)(2.717,2.000){2}{\rule{0.550pt}{0.400pt}}
\put(1207,626.17){\rule{1.300pt}{0.400pt}}
\multiput(1207.00,625.17)(3.302,2.000){2}{\rule{0.650pt}{0.400pt}}
\put(1213,628.17){\rule{1.100pt}{0.400pt}}
\multiput(1213.00,627.17)(2.717,2.000){2}{\rule{0.550pt}{0.400pt}}
\put(1218,630.17){\rule{1.100pt}{0.400pt}}
\multiput(1218.00,629.17)(2.717,2.000){2}{\rule{0.550pt}{0.400pt}}
\put(1223,632.17){\rule{1.100pt}{0.400pt}}
\multiput(1223.00,631.17)(2.717,2.000){2}{\rule{0.550pt}{0.400pt}}
\put(1228,634.17){\rule{1.300pt}{0.400pt}}
\multiput(1228.00,633.17)(3.302,2.000){2}{\rule{0.650pt}{0.400pt}}
\put(1234,636.17){\rule{1.100pt}{0.400pt}}
\multiput(1234.00,635.17)(2.717,2.000){2}{\rule{0.550pt}{0.400pt}}
\multiput(1239.00,638.61)(0.909,0.447){3}{\rule{0.767pt}{0.108pt}}
\multiput(1239.00,637.17)(3.409,3.000){2}{\rule{0.383pt}{0.400pt}}
\put(1244,641.17){\rule{1.300pt}{0.400pt}}
\multiput(1244.00,640.17)(3.302,2.000){2}{\rule{0.650pt}{0.400pt}}
\put(1250,643.17){\rule{1.100pt}{0.400pt}}
\multiput(1250.00,642.17)(2.717,2.000){2}{\rule{0.550pt}{0.400pt}}
\put(1255,645.17){\rule{1.100pt}{0.400pt}}
\multiput(1255.00,644.17)(2.717,2.000){2}{\rule{0.550pt}{0.400pt}}
\put(1260,647.17){\rule{1.100pt}{0.400pt}}
\multiput(1260.00,646.17)(2.717,2.000){2}{\rule{0.550pt}{0.400pt}}
\put(1265,649.17){\rule{1.300pt}{0.400pt}}
\multiput(1265.00,648.17)(3.302,2.000){2}{\rule{0.650pt}{0.400pt}}
\put(1271,651.17){\rule{1.100pt}{0.400pt}}
\multiput(1271.00,650.17)(2.717,2.000){2}{\rule{0.550pt}{0.400pt}}
\put(1276,653.17){\rule{1.100pt}{0.400pt}}
\multiput(1276.00,652.17)(2.717,2.000){2}{\rule{0.550pt}{0.400pt}}
\multiput(1281.00,655.61)(1.132,0.447){3}{\rule{0.900pt}{0.108pt}}
\multiput(1281.00,654.17)(4.132,3.000){2}{\rule{0.450pt}{0.400pt}}
\put(1287,658.17){\rule{1.100pt}{0.400pt}}
\multiput(1287.00,657.17)(2.717,2.000){2}{\rule{0.550pt}{0.400pt}}
\put(1292,660.17){\rule{1.100pt}{0.400pt}}
\multiput(1292.00,659.17)(2.717,2.000){2}{\rule{0.550pt}{0.400pt}}
\put(1297,662.17){\rule{1.100pt}{0.400pt}}
\multiput(1297.00,661.17)(2.717,2.000){2}{\rule{0.550pt}{0.400pt}}
\put(1302,664.17){\rule{1.300pt}{0.400pt}}
\multiput(1302.00,663.17)(3.302,2.000){2}{\rule{0.650pt}{0.400pt}}
\put(1308,666.17){\rule{1.100pt}{0.400pt}}
\multiput(1308.00,665.17)(2.717,2.000){2}{\rule{0.550pt}{0.400pt}}
\put(1313,668.17){\rule{1.100pt}{0.400pt}}
\multiput(1313.00,667.17)(2.717,2.000){2}{\rule{0.550pt}{0.400pt}}
\put(1318,670.17){\rule{1.300pt}{0.400pt}}
\multiput(1318.00,669.17)(3.302,2.000){2}{\rule{0.650pt}{0.400pt}}
\multiput(1324.00,672.61)(0.909,0.447){3}{\rule{0.767pt}{0.108pt}}
\multiput(1324.00,671.17)(3.409,3.000){2}{\rule{0.383pt}{0.400pt}}
\put(1329,675.17){\rule{1.100pt}{0.400pt}}
\multiput(1329.00,674.17)(2.717,2.000){2}{\rule{0.550pt}{0.400pt}}
\put(1334,677.17){\rule{1.100pt}{0.400pt}}
\multiput(1334.00,676.17)(2.717,2.000){2}{\rule{0.550pt}{0.400pt}}
\put(1339,679.17){\rule{1.300pt}{0.400pt}}
\multiput(1339.00,678.17)(3.302,2.000){2}{\rule{0.650pt}{0.400pt}}
\put(1345,681.17){\rule{1.100pt}{0.400pt}}
\multiput(1345.00,680.17)(2.717,2.000){2}{\rule{0.550pt}{0.400pt}}
\put(1350,683.17){\rule{1.100pt}{0.400pt}}
\multiput(1350.00,682.17)(2.717,2.000){2}{\rule{0.550pt}{0.400pt}}
\put(1355,685.17){\rule{1.300pt}{0.400pt}}
\multiput(1355.00,684.17)(3.302,2.000){2}{\rule{0.650pt}{0.400pt}}
\put(1361,687.17){\rule{1.100pt}{0.400pt}}
\multiput(1361.00,686.17)(2.717,2.000){2}{\rule{0.550pt}{0.400pt}}
\multiput(1366.00,689.61)(0.909,0.447){3}{\rule{0.767pt}{0.108pt}}
\multiput(1366.00,688.17)(3.409,3.000){2}{\rule{0.383pt}{0.400pt}}
\put(1371,692.17){\rule{1.100pt}{0.400pt}}
\multiput(1371.00,691.17)(2.717,2.000){2}{\rule{0.550pt}{0.400pt}}
\put(1376,694.17){\rule{1.300pt}{0.400pt}}
\multiput(1376.00,693.17)(3.302,2.000){2}{\rule{0.650pt}{0.400pt}}
\put(1382,696.17){\rule{1.100pt}{0.400pt}}
\multiput(1382.00,695.17)(2.717,2.000){2}{\rule{0.550pt}{0.400pt}}
\put(1387,698.17){\rule{1.100pt}{0.400pt}}
\multiput(1387.00,697.17)(2.717,2.000){2}{\rule{0.550pt}{0.400pt}}
\put(1392,700.17){\rule{1.300pt}{0.400pt}}
\multiput(1392.00,699.17)(3.302,2.000){2}{\rule{0.650pt}{0.400pt}}
\put(1398,702.17){\rule{1.100pt}{0.400pt}}
\multiput(1398.00,701.17)(2.717,2.000){2}{\rule{0.550pt}{0.400pt}}
\put(1403,704.17){\rule{1.100pt}{0.400pt}}
\multiput(1403.00,703.17)(2.717,2.000){2}{\rule{0.550pt}{0.400pt}}
\multiput(1408.00,706.61)(0.909,0.447){3}{\rule{0.767pt}{0.108pt}}
\multiput(1408.00,705.17)(3.409,3.000){2}{\rule{0.383pt}{0.400pt}}
\put(1408,709.17){\rule{1.100pt}{0.400pt}}
\multiput(1410.72,708.17)(-2.717,2.000){2}{\rule{0.550pt}{0.400pt}}
\put(1403,711.17){\rule{1.100pt}{0.400pt}}
\multiput(1405.72,710.17)(-2.717,2.000){2}{\rule{0.550pt}{0.400pt}}
\put(1398,713.17){\rule{1.100pt}{0.400pt}}
\multiput(1400.72,712.17)(-2.717,2.000){2}{\rule{0.550pt}{0.400pt}}
\put(1392,715.17){\rule{1.300pt}{0.400pt}}
\multiput(1395.30,714.17)(-3.302,2.000){2}{\rule{0.650pt}{0.400pt}}
\put(1387,717.17){\rule{1.100pt}{0.400pt}}
\multiput(1389.72,716.17)(-2.717,2.000){2}{\rule{0.550pt}{0.400pt}}
\put(1382,719.17){\rule{1.100pt}{0.400pt}}
\multiput(1384.72,718.17)(-2.717,2.000){2}{\rule{0.550pt}{0.400pt}}
\put(1376,721.17){\rule{1.300pt}{0.400pt}}
\multiput(1379.30,720.17)(-3.302,2.000){2}{\rule{0.650pt}{0.400pt}}
\multiput(1372.82,723.61)(-0.909,0.447){3}{\rule{0.767pt}{0.108pt}}
\multiput(1374.41,722.17)(-3.409,3.000){2}{\rule{0.383pt}{0.400pt}}
\put(1366,726.17){\rule{1.100pt}{0.400pt}}
\multiput(1368.72,725.17)(-2.717,2.000){2}{\rule{0.550pt}{0.400pt}}
\put(1361,728.17){\rule{1.100pt}{0.400pt}}
\multiput(1363.72,727.17)(-2.717,2.000){2}{\rule{0.550pt}{0.400pt}}
\put(1355,730.17){\rule{1.300pt}{0.400pt}}
\multiput(1358.30,729.17)(-3.302,2.000){2}{\rule{0.650pt}{0.400pt}}
\put(1350,732.17){\rule{1.100pt}{0.400pt}}
\multiput(1352.72,731.17)(-2.717,2.000){2}{\rule{0.550pt}{0.400pt}}
\put(1345,734.17){\rule{1.100pt}{0.400pt}}
\multiput(1347.72,733.17)(-2.717,2.000){2}{\rule{0.550pt}{0.400pt}}
\put(1339,736.17){\rule{1.300pt}{0.400pt}}
\multiput(1342.30,735.17)(-3.302,2.000){2}{\rule{0.650pt}{0.400pt}}
\put(1334,738.17){\rule{1.100pt}{0.400pt}}
\multiput(1336.72,737.17)(-2.717,2.000){2}{\rule{0.550pt}{0.400pt}}
\multiput(1330.82,740.61)(-0.909,0.447){3}{\rule{0.767pt}{0.108pt}}
\multiput(1332.41,739.17)(-3.409,3.000){2}{\rule{0.383pt}{0.400pt}}
\put(1324,743.17){\rule{1.100pt}{0.400pt}}
\multiput(1326.72,742.17)(-2.717,2.000){2}{\rule{0.550pt}{0.400pt}}
\put(1318,745.17){\rule{1.300pt}{0.400pt}}
\multiput(1321.30,744.17)(-3.302,2.000){2}{\rule{0.650pt}{0.400pt}}
\put(1313,747.17){\rule{1.100pt}{0.400pt}}
\multiput(1315.72,746.17)(-2.717,2.000){2}{\rule{0.550pt}{0.400pt}}
\put(1308,749.17){\rule{1.100pt}{0.400pt}}
\multiput(1310.72,748.17)(-2.717,2.000){2}{\rule{0.550pt}{0.400pt}}
\put(1302,751.17){\rule{1.300pt}{0.400pt}}
\multiput(1305.30,750.17)(-3.302,2.000){2}{\rule{0.650pt}{0.400pt}}
\put(1297,753.17){\rule{1.100pt}{0.400pt}}
\multiput(1299.72,752.17)(-2.717,2.000){2}{\rule{0.550pt}{0.400pt}}
\multiput(1293.82,755.61)(-0.909,0.447){3}{\rule{0.767pt}{0.108pt}}
\multiput(1295.41,754.17)(-3.409,3.000){2}{\rule{0.383pt}{0.400pt}}
\put(1287,758.17){\rule{1.100pt}{0.400pt}}
\multiput(1289.72,757.17)(-2.717,2.000){2}{\rule{0.550pt}{0.400pt}}
\put(1281,760.17){\rule{1.300pt}{0.400pt}}
\multiput(1284.30,759.17)(-3.302,2.000){2}{\rule{0.650pt}{0.400pt}}
\put(1276,762.17){\rule{1.100pt}{0.400pt}}
\multiput(1278.72,761.17)(-2.717,2.000){2}{\rule{0.550pt}{0.400pt}}
\put(1271,764.17){\rule{1.100pt}{0.400pt}}
\multiput(1273.72,763.17)(-2.717,2.000){2}{\rule{0.550pt}{0.400pt}}
\put(1265,766.17){\rule{1.300pt}{0.400pt}}
\multiput(1268.30,765.17)(-3.302,2.000){2}{\rule{0.650pt}{0.400pt}}
\put(1260,768.17){\rule{1.100pt}{0.400pt}}
\multiput(1262.72,767.17)(-2.717,2.000){2}{\rule{0.550pt}{0.400pt}}
\put(1255,770.17){\rule{1.100pt}{0.400pt}}
\multiput(1257.72,769.17)(-2.717,2.000){2}{\rule{0.550pt}{0.400pt}}
\multiput(1251.82,772.61)(-0.909,0.447){3}{\rule{0.767pt}{0.108pt}}
\multiput(1253.41,771.17)(-3.409,3.000){2}{\rule{0.383pt}{0.400pt}}
\put(1244,775.17){\rule{1.300pt}{0.400pt}}
\multiput(1247.30,774.17)(-3.302,2.000){2}{\rule{0.650pt}{0.400pt}}
\put(1239,777.17){\rule{1.100pt}{0.400pt}}
\multiput(1241.72,776.17)(-2.717,2.000){2}{\rule{0.550pt}{0.400pt}}
\put(1234,779.17){\rule{1.100pt}{0.400pt}}
\multiput(1236.72,778.17)(-2.717,2.000){2}{\rule{0.550pt}{0.400pt}}
\put(1228,781.17){\rule{1.300pt}{0.400pt}}
\multiput(1231.30,780.17)(-3.302,2.000){2}{\rule{0.650pt}{0.400pt}}
\put(1223,783.17){\rule{1.100pt}{0.400pt}}
\multiput(1225.72,782.17)(-2.717,2.000){2}{\rule{0.550pt}{0.400pt}}
\put(1218,785.17){\rule{1.100pt}{0.400pt}}
\multiput(1220.72,784.17)(-2.717,2.000){2}{\rule{0.550pt}{0.400pt}}
\put(1213,787.17){\rule{1.100pt}{0.400pt}}
\multiput(1215.72,786.17)(-2.717,2.000){2}{\rule{0.550pt}{0.400pt}}
\multiput(1209.26,789.61)(-1.132,0.447){3}{\rule{0.900pt}{0.108pt}}
\multiput(1211.13,788.17)(-4.132,3.000){2}{\rule{0.450pt}{0.400pt}}
\put(1202,792.17){\rule{1.100pt}{0.400pt}}
\multiput(1204.72,791.17)(-2.717,2.000){2}{\rule{0.550pt}{0.400pt}}
\put(1197,794.17){\rule{1.100pt}{0.400pt}}
\multiput(1199.72,793.17)(-2.717,2.000){2}{\rule{0.550pt}{0.400pt}}
\put(1191,796.17){\rule{1.300pt}{0.400pt}}
\multiput(1194.30,795.17)(-3.302,2.000){2}{\rule{0.650pt}{0.400pt}}
\put(1186,798.17){\rule{1.100pt}{0.400pt}}
\multiput(1188.72,797.17)(-2.717,2.000){2}{\rule{0.550pt}{0.400pt}}
\put(1181,800.17){\rule{1.100pt}{0.400pt}}
\multiput(1183.72,799.17)(-2.717,2.000){2}{\rule{0.550pt}{0.400pt}}
\put(1176,802.17){\rule{1.100pt}{0.400pt}}
\multiput(1178.72,801.17)(-2.717,2.000){2}{\rule{0.550pt}{0.400pt}}
\put(1170,804.17){\rule{1.300pt}{0.400pt}}
\multiput(1173.30,803.17)(-3.302,2.000){2}{\rule{0.650pt}{0.400pt}}
\multiput(1166.82,806.61)(-0.909,0.447){3}{\rule{0.767pt}{0.108pt}}
\multiput(1168.41,805.17)(-3.409,3.000){2}{\rule{0.383pt}{0.400pt}}
\put(1160,809.17){\rule{1.100pt}{0.400pt}}
\multiput(1162.72,808.17)(-2.717,2.000){2}{\rule{0.550pt}{0.400pt}}
\put(1154,811.17){\rule{1.300pt}{0.400pt}}
\multiput(1157.30,810.17)(-3.302,2.000){2}{\rule{0.650pt}{0.400pt}}
\put(1149,813.17){\rule{1.100pt}{0.400pt}}
\multiput(1151.72,812.17)(-2.717,2.000){2}{\rule{0.550pt}{0.400pt}}
\put(1144,815.17){\rule{1.100pt}{0.400pt}}
\multiput(1146.72,814.17)(-2.717,2.000){2}{\rule{0.550pt}{0.400pt}}
\put(1139,817.17){\rule{1.100pt}{0.400pt}}
\multiput(1141.72,816.17)(-2.717,2.000){2}{\rule{0.550pt}{0.400pt}}
\put(1133,819.17){\rule{1.300pt}{0.400pt}}
\multiput(1136.30,818.17)(-3.302,2.000){2}{\rule{0.650pt}{0.400pt}}
\put(1128,821.17){\rule{1.100pt}{0.400pt}}
\multiput(1130.72,820.17)(-2.717,2.000){2}{\rule{0.550pt}{0.400pt}}
\multiput(1124.82,823.61)(-0.909,0.447){3}{\rule{0.767pt}{0.108pt}}
\multiput(1126.41,822.17)(-3.409,3.000){2}{\rule{0.383pt}{0.400pt}}
\put(1118,826.17){\rule{1.100pt}{0.400pt}}
\multiput(1120.72,825.17)(-2.717,2.000){2}{\rule{0.550pt}{0.400pt}}
\put(1112,828.17){\rule{1.300pt}{0.400pt}}
\multiput(1115.30,827.17)(-3.302,2.000){2}{\rule{0.650pt}{0.400pt}}
\put(1107,830.17){\rule{1.100pt}{0.400pt}}
\multiput(1109.72,829.17)(-2.717,2.000){2}{\rule{0.550pt}{0.400pt}}
\put(1102,832.17){\rule{1.100pt}{0.400pt}}
\multiput(1104.72,831.17)(-2.717,2.000){2}{\rule{0.550pt}{0.400pt}}
\put(1096,834.17){\rule{1.300pt}{0.400pt}}
\multiput(1099.30,833.17)(-3.302,2.000){2}{\rule{0.650pt}{0.400pt}}
\put(1091,834.17){\rule{1.100pt}{0.400pt}}
\multiput(1093.72,835.17)(-2.717,-2.000){2}{\rule{0.550pt}{0.400pt}}
\put(1086,832.17){\rule{1.100pt}{0.400pt}}
\multiput(1088.72,833.17)(-2.717,-2.000){2}{\rule{0.550pt}{0.400pt}}
\put(1081,830.17){\rule{1.100pt}{0.400pt}}
\multiput(1083.72,831.17)(-2.717,-2.000){2}{\rule{0.550pt}{0.400pt}}
\put(1075,828.17){\rule{1.300pt}{0.400pt}}
\multiput(1078.30,829.17)(-3.302,-2.000){2}{\rule{0.650pt}{0.400pt}}
\put(1070,826.17){\rule{1.100pt}{0.400pt}}
\multiput(1072.72,827.17)(-2.717,-2.000){2}{\rule{0.550pt}{0.400pt}}
\multiput(1066.82,824.95)(-0.909,-0.447){3}{\rule{0.767pt}{0.108pt}}
\multiput(1068.41,825.17)(-3.409,-3.000){2}{\rule{0.383pt}{0.400pt}}
\put(1059,821.17){\rule{1.300pt}{0.400pt}}
\multiput(1062.30,822.17)(-3.302,-2.000){2}{\rule{0.650pt}{0.400pt}}
\put(1054,819.17){\rule{1.100pt}{0.400pt}}
\multiput(1056.72,820.17)(-2.717,-2.000){2}{\rule{0.550pt}{0.400pt}}
\put(1049,817.17){\rule{1.100pt}{0.400pt}}
\multiput(1051.72,818.17)(-2.717,-2.000){2}{\rule{0.550pt}{0.400pt}}
\put(1044,815.17){\rule{1.100pt}{0.400pt}}
\multiput(1046.72,816.17)(-2.717,-2.000){2}{\rule{0.550pt}{0.400pt}}
\put(1038,813.17){\rule{1.300pt}{0.400pt}}
\multiput(1041.30,814.17)(-3.302,-2.000){2}{\rule{0.650pt}{0.400pt}}
\put(1033,811.17){\rule{1.100pt}{0.400pt}}
\multiput(1035.72,812.17)(-2.717,-2.000){2}{\rule{0.550pt}{0.400pt}}
\put(1028,809.17){\rule{1.100pt}{0.400pt}}
\multiput(1030.72,810.17)(-2.717,-2.000){2}{\rule{0.550pt}{0.400pt}}
\multiput(1024.26,807.95)(-1.132,-0.447){3}{\rule{0.900pt}{0.108pt}}
\multiput(1026.13,808.17)(-4.132,-3.000){2}{\rule{0.450pt}{0.400pt}}
\put(1017,804.17){\rule{1.100pt}{0.400pt}}
\multiput(1019.72,805.17)(-2.717,-2.000){2}{\rule{0.550pt}{0.400pt}}
\put(1012,802.17){\rule{1.100pt}{0.400pt}}
\multiput(1014.72,803.17)(-2.717,-2.000){2}{\rule{0.550pt}{0.400pt}}
\put(1007,800.17){\rule{1.100pt}{0.400pt}}
\multiput(1009.72,801.17)(-2.717,-2.000){2}{\rule{0.550pt}{0.400pt}}
\put(1001,798.17){\rule{1.300pt}{0.400pt}}
\multiput(1004.30,799.17)(-3.302,-2.000){2}{\rule{0.650pt}{0.400pt}}
\put(996,796.17){\rule{1.100pt}{0.400pt}}
\multiput(998.72,797.17)(-2.717,-2.000){2}{\rule{0.550pt}{0.400pt}}
\put(991,794.17){\rule{1.100pt}{0.400pt}}
\multiput(993.72,795.17)(-2.717,-2.000){2}{\rule{0.550pt}{0.400pt}}
\put(985,792.17){\rule{1.300pt}{0.400pt}}
\multiput(988.30,793.17)(-3.302,-2.000){2}{\rule{0.650pt}{0.400pt}}
\multiput(981.82,790.95)(-0.909,-0.447){3}{\rule{0.767pt}{0.108pt}}
\multiput(983.41,791.17)(-3.409,-3.000){2}{\rule{0.383pt}{0.400pt}}
\put(975,787.17){\rule{1.100pt}{0.400pt}}
\multiput(977.72,788.17)(-2.717,-2.000){2}{\rule{0.550pt}{0.400pt}}
\put(970,785.17){\rule{1.100pt}{0.400pt}}
\multiput(972.72,786.17)(-2.717,-2.000){2}{\rule{0.550pt}{0.400pt}}
\put(964,783.17){\rule{1.300pt}{0.400pt}}
\multiput(967.30,784.17)(-3.302,-2.000){2}{\rule{0.650pt}{0.400pt}}
\put(959,781.17){\rule{1.100pt}{0.400pt}}
\multiput(961.72,782.17)(-2.717,-2.000){2}{\rule{0.550pt}{0.400pt}}
\put(954,779.17){\rule{1.100pt}{0.400pt}}
\multiput(956.72,780.17)(-2.717,-2.000){2}{\rule{0.550pt}{0.400pt}}
\put(948,777.17){\rule{1.300pt}{0.400pt}}
\multiput(951.30,778.17)(-3.302,-2.000){2}{\rule{0.650pt}{0.400pt}}
\put(943,775.17){\rule{1.100pt}{0.400pt}}
\multiput(945.72,776.17)(-2.717,-2.000){2}{\rule{0.550pt}{0.400pt}}
\multiput(939.82,773.95)(-0.909,-0.447){3}{\rule{0.767pt}{0.108pt}}
\multiput(941.41,774.17)(-3.409,-3.000){2}{\rule{0.383pt}{0.400pt}}
\put(933,770.17){\rule{1.100pt}{0.400pt}}
\multiput(935.72,771.17)(-2.717,-2.000){2}{\rule{0.550pt}{0.400pt}}
\put(927,768.17){\rule{1.300pt}{0.400pt}}
\multiput(930.30,769.17)(-3.302,-2.000){2}{\rule{0.650pt}{0.400pt}}
\put(922,766.17){\rule{1.100pt}{0.400pt}}
\multiput(924.72,767.17)(-2.717,-2.000){2}{\rule{0.550pt}{0.400pt}}
\put(917,764.17){\rule{1.100pt}{0.400pt}}
\multiput(919.72,765.17)(-2.717,-2.000){2}{\rule{0.550pt}{0.400pt}}
\put(911,762.17){\rule{1.300pt}{0.400pt}}
\multiput(914.30,763.17)(-3.302,-2.000){2}{\rule{0.650pt}{0.400pt}}
\put(906,760.17){\rule{1.100pt}{0.400pt}}
\multiput(908.72,761.17)(-2.717,-2.000){2}{\rule{0.550pt}{0.400pt}}
\put(901,758.17){\rule{1.100pt}{0.400pt}}
\multiput(903.72,759.17)(-2.717,-2.000){2}{\rule{0.550pt}{0.400pt}}
\multiput(897.82,756.95)(-0.909,-0.447){3}{\rule{0.767pt}{0.108pt}}
\multiput(899.41,757.17)(-3.409,-3.000){2}{\rule{0.383pt}{0.400pt}}
\put(890,753.17){\rule{1.300pt}{0.400pt}}
\multiput(893.30,754.17)(-3.302,-2.000){2}{\rule{0.650pt}{0.400pt}}
\put(885,751.17){\rule{1.100pt}{0.400pt}}
\multiput(887.72,752.17)(-2.717,-2.000){2}{\rule{0.550pt}{0.400pt}}
\put(880,749.17){\rule{1.100pt}{0.400pt}}
\multiput(882.72,750.17)(-2.717,-2.000){2}{\rule{0.550pt}{0.400pt}}
\put(874,747.17){\rule{1.300pt}{0.400pt}}
\multiput(877.30,748.17)(-3.302,-2.000){2}{\rule{0.650pt}{0.400pt}}
\put(869,745.17){\rule{1.100pt}{0.400pt}}
\multiput(871.72,746.17)(-2.717,-2.000){2}{\rule{0.550pt}{0.400pt}}
\put(864,743.17){\rule{1.100pt}{0.400pt}}
\multiput(866.72,744.17)(-2.717,-2.000){2}{\rule{0.550pt}{0.400pt}}
\multiput(860.82,741.95)(-0.909,-0.447){3}{\rule{0.767pt}{0.108pt}}
\multiput(862.41,742.17)(-3.409,-3.000){2}{\rule{0.383pt}{0.400pt}}
\put(853,738.17){\rule{1.300pt}{0.400pt}}
\multiput(856.30,739.17)(-3.302,-2.000){2}{\rule{0.650pt}{0.400pt}}
\put(848,736.17){\rule{1.100pt}{0.400pt}}
\multiput(850.72,737.17)(-2.717,-2.000){2}{\rule{0.550pt}{0.400pt}}
\put(843,734.17){\rule{1.100pt}{0.400pt}}
\multiput(845.72,735.17)(-2.717,-2.000){2}{\rule{0.550pt}{0.400pt}}
\put(837,732.17){\rule{1.300pt}{0.400pt}}
\multiput(840.30,733.17)(-3.302,-2.000){2}{\rule{0.650pt}{0.400pt}}
\put(832,730.17){\rule{1.100pt}{0.400pt}}
\multiput(834.72,731.17)(-2.717,-2.000){2}{\rule{0.550pt}{0.400pt}}
\put(827,728.17){\rule{1.100pt}{0.400pt}}
\multiput(829.72,729.17)(-2.717,-2.000){2}{\rule{0.550pt}{0.400pt}}
\put(822,726.17){\rule{1.100pt}{0.400pt}}
\multiput(824.72,727.17)(-2.717,-2.000){2}{\rule{0.550pt}{0.400pt}}
\multiput(818.26,724.95)(-1.132,-0.447){3}{\rule{0.900pt}{0.108pt}}
\multiput(820.13,725.17)(-4.132,-3.000){2}{\rule{0.450pt}{0.400pt}}
\put(811,721.17){\rule{1.100pt}{0.400pt}}
\multiput(813.72,722.17)(-2.717,-2.000){2}{\rule{0.550pt}{0.400pt}}
\put(806,719.17){\rule{1.100pt}{0.400pt}}
\multiput(808.72,720.17)(-2.717,-2.000){2}{\rule{0.550pt}{0.400pt}}
\put(800,717.17){\rule{1.300pt}{0.400pt}}
\multiput(803.30,718.17)(-3.302,-2.000){2}{\rule{0.650pt}{0.400pt}}
\put(795,715.17){\rule{1.100pt}{0.400pt}}
\multiput(797.72,716.17)(-2.717,-2.000){2}{\rule{0.550pt}{0.400pt}}
\put(790,713.17){\rule{1.100pt}{0.400pt}}
\multiput(792.72,714.17)(-2.717,-2.000){2}{\rule{0.550pt}{0.400pt}}
\put(785,711.17){\rule{1.100pt}{0.400pt}}
\multiput(787.72,712.17)(-2.717,-2.000){2}{\rule{0.550pt}{0.400pt}}
\put(779,709.17){\rule{1.300pt}{0.400pt}}
\multiput(782.30,710.17)(-3.302,-2.000){2}{\rule{0.650pt}{0.400pt}}
\multiput(775.82,707.95)(-0.909,-0.447){3}{\rule{0.767pt}{0.108pt}}
\multiput(777.41,708.17)(-3.409,-3.000){2}{\rule{0.383pt}{0.400pt}}
\put(769,704.17){\rule{1.100pt}{0.400pt}}
\multiput(771.72,705.17)(-2.717,-2.000){2}{\rule{0.550pt}{0.400pt}}
\put(763,702.17){\rule{1.300pt}{0.400pt}}
\multiput(766.30,703.17)(-3.302,-2.000){2}{\rule{0.650pt}{0.400pt}}
\put(758,700.17){\rule{1.100pt}{0.400pt}}
\multiput(760.72,701.17)(-2.717,-2.000){2}{\rule{0.550pt}{0.400pt}}
\put(753,698.17){\rule{1.100pt}{0.400pt}}
\multiput(755.72,699.17)(-2.717,-2.000){2}{\rule{0.550pt}{0.400pt}}
\put(748,696.17){\rule{1.100pt}{0.400pt}}
\multiput(750.72,697.17)(-2.717,-2.000){2}{\rule{0.550pt}{0.400pt}}
\put(742,694.17){\rule{1.300pt}{0.400pt}}
\multiput(745.30,695.17)(-3.302,-2.000){2}{\rule{0.650pt}{0.400pt}}
\put(737,692.17){\rule{1.100pt}{0.400pt}}
\multiput(739.72,693.17)(-2.717,-2.000){2}{\rule{0.550pt}{0.400pt}}
\multiput(733.82,690.95)(-0.909,-0.447){3}{\rule{0.767pt}{0.108pt}}
\multiput(735.41,691.17)(-3.409,-3.000){2}{\rule{0.383pt}{0.400pt}}
\put(726,687.17){\rule{1.300pt}{0.400pt}}
\multiput(729.30,688.17)(-3.302,-2.000){2}{\rule{0.650pt}{0.400pt}}
\put(721,685.17){\rule{1.100pt}{0.400pt}}
\multiput(723.72,686.17)(-2.717,-2.000){2}{\rule{0.550pt}{0.400pt}}
\put(716,683.17){\rule{1.100pt}{0.400pt}}
\multiput(718.72,684.17)(-2.717,-2.000){2}{\rule{0.550pt}{0.400pt}}
\put(711,681.17){\rule{1.100pt}{0.400pt}}
\multiput(713.72,682.17)(-2.717,-2.000){2}{\rule{0.550pt}{0.400pt}}
\put(705,679.17){\rule{1.300pt}{0.400pt}}
\multiput(708.30,680.17)(-3.302,-2.000){2}{\rule{0.650pt}{0.400pt}}
\put(700,677.17){\rule{1.100pt}{0.400pt}}
\multiput(702.72,678.17)(-2.717,-2.000){2}{\rule{0.550pt}{0.400pt}}
\put(695,675.17){\rule{1.100pt}{0.400pt}}
\multiput(697.72,676.17)(-2.717,-2.000){2}{\rule{0.550pt}{0.400pt}}
\multiput(691.26,673.95)(-1.132,-0.447){3}{\rule{0.900pt}{0.108pt}}
\multiput(693.13,674.17)(-4.132,-3.000){2}{\rule{0.450pt}{0.400pt}}
\put(684,670.17){\rule{1.100pt}{0.400pt}}
\multiput(686.72,671.17)(-2.717,-2.000){2}{\rule{0.550pt}{0.400pt}}
\put(679,668.17){\rule{1.100pt}{0.400pt}}
\multiput(681.72,669.17)(-2.717,-2.000){2}{\rule{0.550pt}{0.400pt}}
\put(674,666.17){\rule{1.100pt}{0.400pt}}
\multiput(676.72,667.17)(-2.717,-2.000){2}{\rule{0.550pt}{0.400pt}}
\put(668,664.17){\rule{1.300pt}{0.400pt}}
\multiput(671.30,665.17)(-3.302,-2.000){2}{\rule{0.650pt}{0.400pt}}
\put(663,662.17){\rule{1.100pt}{0.400pt}}
\multiput(665.72,663.17)(-2.717,-2.000){2}{\rule{0.550pt}{0.400pt}}
\put(658,660.17){\rule{1.100pt}{0.400pt}}
\multiput(660.72,661.17)(-2.717,-2.000){2}{\rule{0.550pt}{0.400pt}}
\put(652,658.17){\rule{1.300pt}{0.400pt}}
\multiput(655.30,659.17)(-3.302,-2.000){2}{\rule{0.650pt}{0.400pt}}
\multiput(648.82,656.95)(-0.909,-0.447){3}{\rule{0.767pt}{0.108pt}}
\multiput(650.41,657.17)(-3.409,-3.000){2}{\rule{0.383pt}{0.400pt}}
\put(642,653.17){\rule{1.100pt}{0.400pt}}
\multiput(644.72,654.17)(-2.717,-2.000){2}{\rule{0.550pt}{0.400pt}}
\put(637,651.17){\rule{1.100pt}{0.400pt}}
\multiput(639.72,652.17)(-2.717,-2.000){2}{\rule{0.550pt}{0.400pt}}
\put(631,649.17){\rule{1.300pt}{0.400pt}}
\multiput(634.30,650.17)(-3.302,-2.000){2}{\rule{0.650pt}{0.400pt}}
\put(626,647.17){\rule{1.100pt}{0.400pt}}
\multiput(628.72,648.17)(-2.717,-2.000){2}{\rule{0.550pt}{0.400pt}}
\put(621,645.17){\rule{1.100pt}{0.400pt}}
\multiput(623.72,646.17)(-2.717,-2.000){2}{\rule{0.550pt}{0.400pt}}
\put(615,643.17){\rule{1.300pt}{0.400pt}}
\multiput(618.30,644.17)(-3.302,-2.000){2}{\rule{0.650pt}{0.400pt}}
\put(610,641.17){\rule{1.100pt}{0.400pt}}
\multiput(612.72,642.17)(-2.717,-2.000){2}{\rule{0.550pt}{0.400pt}}
\multiput(606.82,639.95)(-0.909,-0.447){3}{\rule{0.767pt}{0.108pt}}
\multiput(608.41,640.17)(-3.409,-3.000){2}{\rule{0.383pt}{0.400pt}}
\put(600,636.17){\rule{1.100pt}{0.400pt}}
\multiput(602.72,637.17)(-2.717,-2.000){2}{\rule{0.550pt}{0.400pt}}
\put(594,634.17){\rule{1.300pt}{0.400pt}}
\multiput(597.30,635.17)(-3.302,-2.000){2}{\rule{0.650pt}{0.400pt}}
\put(589,632.17){\rule{1.100pt}{0.400pt}}
\multiput(591.72,633.17)(-2.717,-2.000){2}{\rule{0.550pt}{0.400pt}}
\put(584,630.17){\rule{1.100pt}{0.400pt}}
\multiput(586.72,631.17)(-2.717,-2.000){2}{\rule{0.550pt}{0.400pt}}
\put(578,628.17){\rule{1.300pt}{0.400pt}}
\multiput(581.30,629.17)(-3.302,-2.000){2}{\rule{0.650pt}{0.400pt}}
\put(573,626.17){\rule{1.100pt}{0.400pt}}
\multiput(575.72,627.17)(-2.717,-2.000){2}{\rule{0.550pt}{0.400pt}}
\put(568,624.17){\rule{1.100pt}{0.400pt}}
\multiput(570.72,625.17)(-2.717,-2.000){2}{\rule{0.550pt}{0.400pt}}
\multiput(564.82,622.95)(-0.909,-0.447){3}{\rule{0.767pt}{0.108pt}}
\multiput(566.41,623.17)(-3.409,-3.000){2}{\rule{0.383pt}{0.400pt}}
\put(557,619.17){\rule{1.300pt}{0.400pt}}
\multiput(560.30,620.17)(-3.302,-2.000){2}{\rule{0.650pt}{0.400pt}}
\put(552,617.17){\rule{1.100pt}{0.400pt}}
\multiput(554.72,618.17)(-2.717,-2.000){2}{\rule{0.550pt}{0.400pt}}
\put(547,615.17){\rule{1.100pt}{0.400pt}}
\multiput(549.72,616.17)(-2.717,-2.000){2}{\rule{0.550pt}{0.400pt}}
\put(541,613.17){\rule{1.300pt}{0.400pt}}
\multiput(544.30,614.17)(-3.302,-2.000){2}{\rule{0.650pt}{0.400pt}}
\put(536,611.17){\rule{1.100pt}{0.400pt}}
\multiput(538.72,612.17)(-2.717,-2.000){2}{\rule{0.550pt}{0.400pt}}
\put(531,609.17){\rule{1.100pt}{0.400pt}}
\multiput(533.72,610.17)(-2.717,-2.000){2}{\rule{0.550pt}{0.400pt}}
\put(526,607.17){\rule{1.100pt}{0.400pt}}
\multiput(528.72,608.17)(-2.717,-2.000){2}{\rule{0.550pt}{0.400pt}}
\multiput(522.26,605.95)(-1.132,-0.447){3}{\rule{0.900pt}{0.108pt}}
\multiput(524.13,606.17)(-4.132,-3.000){2}{\rule{0.450pt}{0.400pt}}
\put(515,602.17){\rule{1.100pt}{0.400pt}}
\multiput(517.72,603.17)(-2.717,-2.000){2}{\rule{0.550pt}{0.400pt}}
\put(510,600.17){\rule{1.100pt}{0.400pt}}
\multiput(512.72,601.17)(-2.717,-2.000){2}{\rule{0.550pt}{0.400pt}}
\put(504,598.17){\rule{1.300pt}{0.400pt}}
\multiput(507.30,599.17)(-3.302,-2.000){2}{\rule{0.650pt}{0.400pt}}
\put(499,596.17){\rule{1.100pt}{0.400pt}}
\multiput(501.72,597.17)(-2.717,-2.000){2}{\rule{0.550pt}{0.400pt}}
\put(494,594.17){\rule{1.100pt}{0.400pt}}
\multiput(496.72,595.17)(-2.717,-2.000){2}{\rule{0.550pt}{0.400pt}}
\put(489,592.17){\rule{1.100pt}{0.400pt}}
\multiput(491.72,593.17)(-2.717,-2.000){2}{\rule{0.550pt}{0.400pt}}
\put(483,590.17){\rule{1.300pt}{0.400pt}}
\multiput(486.30,591.17)(-3.302,-2.000){2}{\rule{0.650pt}{0.400pt}}
\multiput(479.82,588.95)(-0.909,-0.447){3}{\rule{0.767pt}{0.108pt}}
\multiput(481.41,589.17)(-3.409,-3.000){2}{\rule{0.383pt}{0.400pt}}
\put(473,585.17){\rule{1.100pt}{0.400pt}}
\multiput(475.72,586.17)(-2.717,-2.000){2}{\rule{0.550pt}{0.400pt}}
\put(467,583.17){\rule{1.300pt}{0.400pt}}
\multiput(470.30,584.17)(-3.302,-2.000){2}{\rule{0.650pt}{0.400pt}}
\put(462,581.17){\rule{1.100pt}{0.400pt}}
\multiput(464.72,582.17)(-2.717,-2.000){2}{\rule{0.550pt}{0.400pt}}
\put(457,579.17){\rule{1.100pt}{0.400pt}}
\multiput(459.72,580.17)(-2.717,-2.000){2}{\rule{0.550pt}{0.400pt}}
\put(452,577.17){\rule{1.100pt}{0.400pt}}
\multiput(454.72,578.17)(-2.717,-2.000){2}{\rule{0.550pt}{0.400pt}}
\put(446,575.17){\rule{1.300pt}{0.400pt}}
\multiput(449.30,576.17)(-3.302,-2.000){2}{\rule{0.650pt}{0.400pt}}
\put(441,573.17){\rule{1.100pt}{0.400pt}}
\multiput(443.72,574.17)(-2.717,-2.000){2}{\rule{0.550pt}{0.400pt}}
\multiput(437.82,571.95)(-0.909,-0.447){3}{\rule{0.767pt}{0.108pt}}
\multiput(439.41,572.17)(-3.409,-3.000){2}{\rule{0.383pt}{0.400pt}}
\put(430,568.17){\rule{1.300pt}{0.400pt}}
\multiput(433.30,569.17)(-3.302,-2.000){2}{\rule{0.650pt}{0.400pt}}
\put(425,566.17){\rule{1.100pt}{0.400pt}}
\multiput(427.72,567.17)(-2.717,-2.000){2}{\rule{0.550pt}{0.400pt}}
\put(420,564.17){\rule{1.100pt}{0.400pt}}
\multiput(422.72,565.17)(-2.717,-2.000){2}{\rule{0.550pt}{0.400pt}}
\put(415,562.17){\rule{1.100pt}{0.400pt}}
\multiput(417.72,563.17)(-2.717,-2.000){2}{\rule{0.550pt}{0.400pt}}
\put(409,560.17){\rule{1.300pt}{0.400pt}}
\multiput(412.30,561.17)(-3.302,-2.000){2}{\rule{0.650pt}{0.400pt}}
\put(404,558.17){\rule{1.100pt}{0.400pt}}
\multiput(406.72,559.17)(-2.717,-2.000){2}{\rule{0.550pt}{0.400pt}}
\put(399,556.17){\rule{1.100pt}{0.400pt}}
\multiput(401.72,557.17)(-2.717,-2.000){2}{\rule{0.550pt}{0.400pt}}
\multiput(395.26,554.95)(-1.132,-0.447){3}{\rule{0.900pt}{0.108pt}}
\multiput(397.13,555.17)(-4.132,-3.000){2}{\rule{0.450pt}{0.400pt}}
\put(388,551.17){\rule{1.100pt}{0.400pt}}
\multiput(390.72,552.17)(-2.717,-2.000){2}{\rule{0.550pt}{0.400pt}}
\put(383,549.17){\rule{1.100pt}{0.400pt}}
\multiput(385.72,550.17)(-2.717,-2.000){2}{\rule{0.550pt}{0.400pt}}
\put(378,547.17){\rule{1.100pt}{0.400pt}}
\multiput(380.72,548.17)(-2.717,-2.000){2}{\rule{0.550pt}{0.400pt}}
\put(372,545.17){\rule{1.300pt}{0.400pt}}
\multiput(375.30,546.17)(-3.302,-2.000){2}{\rule{0.650pt}{0.400pt}}
\put(367,543.17){\rule{1.100pt}{0.400pt}}
\multiput(369.72,544.17)(-2.717,-2.000){2}{\rule{0.550pt}{0.400pt}}
\put(362,541.17){\rule{1.100pt}{0.400pt}}
\multiput(364.72,542.17)(-2.717,-2.000){2}{\rule{0.550pt}{0.400pt}}
\put(356,539.17){\rule{1.300pt}{0.400pt}}
\multiput(359.30,540.17)(-3.302,-2.000){2}{\rule{0.650pt}{0.400pt}}
\multiput(352.82,537.95)(-0.909,-0.447){3}{\rule{0.767pt}{0.108pt}}
\multiput(354.41,538.17)(-3.409,-3.000){2}{\rule{0.383pt}{0.400pt}}
\put(346,534.17){\rule{1.100pt}{0.400pt}}
\multiput(348.72,535.17)(-2.717,-2.000){2}{\rule{0.550pt}{0.400pt}}
\put(341,532.17){\rule{1.100pt}{0.400pt}}
\multiput(343.72,533.17)(-2.717,-2.000){2}{\rule{0.550pt}{0.400pt}}
\put(335,530.17){\rule{1.300pt}{0.400pt}}
\multiput(338.30,531.17)(-3.302,-2.000){2}{\rule{0.650pt}{0.400pt}}
\put(330,528.17){\rule{1.100pt}{0.400pt}}
\multiput(332.72,529.17)(-2.717,-2.000){2}{\rule{0.550pt}{0.400pt}}
\put(325,526.17){\rule{1.100pt}{0.400pt}}
\multiput(327.72,527.17)(-2.717,-2.000){2}{\rule{0.550pt}{0.400pt}}
\put(319,524.17){\rule{1.300pt}{0.400pt}}
\multiput(322.30,525.17)(-3.302,-2.000){2}{\rule{0.650pt}{0.400pt}}
\put(314,522.17){\rule{1.100pt}{0.400pt}}
\multiput(316.72,523.17)(-2.717,-2.000){2}{\rule{0.550pt}{0.400pt}}
\multiput(310.82,520.95)(-0.909,-0.447){3}{\rule{0.767pt}{0.108pt}}
\multiput(312.41,521.17)(-3.409,-3.000){2}{\rule{0.383pt}{0.400pt}}
\put(304,517.17){\rule{1.100pt}{0.400pt}}
\multiput(306.72,518.17)(-2.717,-2.000){2}{\rule{0.550pt}{0.400pt}}
\put(298,515.17){\rule{1.300pt}{0.400pt}}
\multiput(301.30,516.17)(-3.302,-2.000){2}{\rule{0.650pt}{0.400pt}}
\put(293,513.17){\rule{1.100pt}{0.400pt}}
\multiput(295.72,514.17)(-2.717,-2.000){2}{\rule{0.550pt}{0.400pt}}
\put(288,511.17){\rule{1.100pt}{0.400pt}}
\multiput(290.72,512.17)(-2.717,-2.000){2}{\rule{0.550pt}{0.400pt}}
\put(282,509.17){\rule{1.300pt}{0.400pt}}
\multiput(285.30,510.17)(-3.302,-2.000){2}{\rule{0.650pt}{0.400pt}}
\put(277,507.17){\rule{1.100pt}{0.400pt}}
\multiput(279.72,508.17)(-2.717,-2.000){2}{\rule{0.550pt}{0.400pt}}
\put(272,505.17){\rule{1.100pt}{0.400pt}}
\multiput(274.72,506.17)(-2.717,-2.000){2}{\rule{0.550pt}{0.400pt}}
\multiput(268.82,503.95)(-0.909,-0.447){3}{\rule{0.767pt}{0.108pt}}
\multiput(270.41,504.17)(-3.409,-3.000){2}{\rule{0.383pt}{0.400pt}}
\put(261,500.17){\rule{1.300pt}{0.400pt}}
\multiput(264.30,501.17)(-3.302,-2.000){2}{\rule{0.650pt}{0.400pt}}
\put(256,498.17){\rule{1.100pt}{0.400pt}}
\multiput(258.72,499.17)(-2.717,-2.000){2}{\rule{0.550pt}{0.400pt}}
\put(251,496.17){\rule{1.100pt}{0.400pt}}
\multiput(253.72,497.17)(-2.717,-2.000){2}{\rule{0.550pt}{0.400pt}}
\put(245,494.17){\rule{1.300pt}{0.400pt}}
\multiput(248.30,495.17)(-3.302,-2.000){2}{\rule{0.650pt}{0.400pt}}
\put(240,492.17){\rule{1.100pt}{0.400pt}}
\multiput(242.72,493.17)(-2.717,-2.000){2}{\rule{0.550pt}{0.400pt}}
\put(235,490.17){\rule{1.100pt}{0.400pt}}
\multiput(237.72,491.17)(-2.717,-2.000){2}{\rule{0.550pt}{0.400pt}}
\multiput(231.82,488.95)(-0.909,-0.447){3}{\rule{0.767pt}{0.108pt}}
\multiput(233.41,489.17)(-3.409,-3.000){2}{\rule{0.383pt}{0.400pt}}
\put(224,485.17){\rule{1.300pt}{0.400pt}}
\multiput(227.30,486.17)(-3.302,-2.000){2}{\rule{0.650pt}{0.400pt}}
\put(219,483.17){\rule{1.100pt}{0.400pt}}
\multiput(221.72,484.17)(-2.717,-2.000){2}{\rule{0.550pt}{0.400pt}}
\put(214,481.17){\rule{1.100pt}{0.400pt}}
\multiput(216.72,482.17)(-2.717,-2.000){2}{\rule{0.550pt}{0.400pt}}
\put(208,479.17){\rule{1.300pt}{0.400pt}}
\multiput(211.30,480.17)(-3.302,-2.000){2}{\rule{0.650pt}{0.400pt}}
\put(203,477.17){\rule{1.100pt}{0.400pt}}
\multiput(205.72,478.17)(-2.717,-2.000){2}{\rule{0.550pt}{0.400pt}}
\put(198,475.17){\rule{1.100pt}{0.400pt}}
\multiput(200.72,476.17)(-2.717,-2.000){2}{\rule{0.550pt}{0.400pt}}
\put(193,473.17){\rule{1.100pt}{0.400pt}}
\multiput(195.72,474.17)(-2.717,-2.000){2}{\rule{0.550pt}{0.400pt}}
\multiput(189.26,471.95)(-1.132,-0.447){3}{\rule{0.900pt}{0.108pt}}
\multiput(191.13,472.17)(-4.132,-3.000){2}{\rule{0.450pt}{0.400pt}}
\put(182,468.17){\rule{1.100pt}{0.400pt}}
\multiput(184.72,469.17)(-2.717,-2.000){2}{\rule{0.550pt}{0.400pt}}
\put(177,466.17){\rule{1.100pt}{0.400pt}}
\multiput(179.72,467.17)(-2.717,-2.000){2}{\rule{0.550pt}{0.400pt}}
\put(171,464.17){\rule{1.300pt}{0.400pt}}
\multiput(174.30,465.17)(-3.302,-2.000){2}{\rule{0.650pt}{0.400pt}}
\put(166,462.17){\rule{1.100pt}{0.400pt}}
\multiput(168.72,463.17)(-2.717,-2.000){2}{\rule{0.550pt}{0.400pt}}
\put(161,460.17){\rule{1.100pt}{0.400pt}}
\multiput(163.72,461.17)(-2.717,-2.000){2}{\rule{0.550pt}{0.400pt}}
\put(156,458.17){\rule{1.100pt}{0.400pt}}
\multiput(158.72,459.17)(-2.717,-2.000){2}{\rule{0.550pt}{0.400pt}}
\put(156,456.17){\rule{1.100pt}{0.400pt}}
\multiput(156.00,457.17)(2.717,-2.000){2}{\rule{0.550pt}{0.400pt}}
\multiput(161.00,454.95)(0.909,-0.447){3}{\rule{0.767pt}{0.108pt}}
\multiput(161.00,455.17)(3.409,-3.000){2}{\rule{0.383pt}{0.400pt}}
\put(166,451.17){\rule{1.100pt}{0.400pt}}
\multiput(166.00,452.17)(2.717,-2.000){2}{\rule{0.550pt}{0.400pt}}
\put(171,449.17){\rule{1.300pt}{0.400pt}}
\multiput(171.00,450.17)(3.302,-2.000){2}{\rule{0.650pt}{0.400pt}}
\put(177,447.17){\rule{1.100pt}{0.400pt}}
\multiput(177.00,448.17)(2.717,-2.000){2}{\rule{0.550pt}{0.400pt}}
\put(182,445.17){\rule{1.100pt}{0.400pt}}
\multiput(182.00,446.17)(2.717,-2.000){2}{\rule{0.550pt}{0.400pt}}
\put(187,443.17){\rule{1.300pt}{0.400pt}}
\multiput(187.00,444.17)(3.302,-2.000){2}{\rule{0.650pt}{0.400pt}}
\put(193,441.17){\rule{1.100pt}{0.400pt}}
\multiput(193.00,442.17)(2.717,-2.000){2}{\rule{0.550pt}{0.400pt}}
\put(198,439.17){\rule{1.100pt}{0.400pt}}
\multiput(198.00,440.17)(2.717,-2.000){2}{\rule{0.550pt}{0.400pt}}
\multiput(203.00,437.95)(0.909,-0.447){3}{\rule{0.767pt}{0.108pt}}
\multiput(203.00,438.17)(3.409,-3.000){2}{\rule{0.383pt}{0.400pt}}
\put(208,434.17){\rule{1.300pt}{0.400pt}}
\multiput(208.00,435.17)(3.302,-2.000){2}{\rule{0.650pt}{0.400pt}}
\put(214,432.17){\rule{1.100pt}{0.400pt}}
\multiput(214.00,433.17)(2.717,-2.000){2}{\rule{0.550pt}{0.400pt}}
\put(219,430.17){\rule{1.100pt}{0.400pt}}
\multiput(219.00,431.17)(2.717,-2.000){2}{\rule{0.550pt}{0.400pt}}
\put(224,428.17){\rule{1.300pt}{0.400pt}}
\multiput(224.00,429.17)(3.302,-2.000){2}{\rule{0.650pt}{0.400pt}}
\put(230,426.17){\rule{1.100pt}{0.400pt}}
\multiput(230.00,427.17)(2.717,-2.000){2}{\rule{0.550pt}{0.400pt}}
\put(235,424.17){\rule{1.100pt}{0.400pt}}
\multiput(235.00,425.17)(2.717,-2.000){2}{\rule{0.550pt}{0.400pt}}
\put(240,422.17){\rule{1.100pt}{0.400pt}}
\multiput(240.00,423.17)(2.717,-2.000){2}{\rule{0.550pt}{0.400pt}}
\multiput(245.00,420.95)(1.132,-0.447){3}{\rule{0.900pt}{0.108pt}}
\multiput(245.00,421.17)(4.132,-3.000){2}{\rule{0.450pt}{0.400pt}}
\put(251,417.17){\rule{1.100pt}{0.400pt}}
\multiput(251.00,418.17)(2.717,-2.000){2}{\rule{0.550pt}{0.400pt}}
\put(256,415.17){\rule{1.100pt}{0.400pt}}
\multiput(256.00,416.17)(2.717,-2.000){2}{\rule{0.550pt}{0.400pt}}
\put(261,413.17){\rule{1.300pt}{0.400pt}}
\multiput(261.00,414.17)(3.302,-2.000){2}{\rule{0.650pt}{0.400pt}}
\put(267,411.17){\rule{1.100pt}{0.400pt}}
\multiput(267.00,412.17)(2.717,-2.000){2}{\rule{0.550pt}{0.400pt}}
\put(272,409.17){\rule{1.100pt}{0.400pt}}
\multiput(272.00,410.17)(2.717,-2.000){2}{\rule{0.550pt}{0.400pt}}
\put(277,407.17){\rule{1.100pt}{0.400pt}}
\multiput(277.00,408.17)(2.717,-2.000){2}{\rule{0.550pt}{0.400pt}}
\put(282,405.17){\rule{1.300pt}{0.400pt}}
\multiput(282.00,406.17)(3.302,-2.000){2}{\rule{0.650pt}{0.400pt}}
\multiput(288.00,403.95)(0.909,-0.447){3}{\rule{0.767pt}{0.108pt}}
\multiput(288.00,404.17)(3.409,-3.000){2}{\rule{0.383pt}{0.400pt}}
\put(293,400.17){\rule{1.100pt}{0.400pt}}
\multiput(293.00,401.17)(2.717,-2.000){2}{\rule{0.550pt}{0.400pt}}
\put(298,398.17){\rule{1.300pt}{0.400pt}}
\multiput(298.00,399.17)(3.302,-2.000){2}{\rule{0.650pt}{0.400pt}}
\put(304,396.17){\rule{1.100pt}{0.400pt}}
\multiput(304.00,397.17)(2.717,-2.000){2}{\rule{0.550pt}{0.400pt}}
\put(309,394.17){\rule{1.100pt}{0.400pt}}
\multiput(309.00,395.17)(2.717,-2.000){2}{\rule{0.550pt}{0.400pt}}
\put(314,392.17){\rule{1.100pt}{0.400pt}}
\multiput(314.00,393.17)(2.717,-2.000){2}{\rule{0.550pt}{0.400pt}}
\put(319,390.17){\rule{1.300pt}{0.400pt}}
\multiput(319.00,391.17)(3.302,-2.000){2}{\rule{0.650pt}{0.400pt}}
\put(325,388.17){\rule{1.100pt}{0.400pt}}
\multiput(325.00,389.17)(2.717,-2.000){2}{\rule{0.550pt}{0.400pt}}
\multiput(330.00,386.95)(0.909,-0.447){3}{\rule{0.767pt}{0.108pt}}
\multiput(330.00,387.17)(3.409,-3.000){2}{\rule{0.383pt}{0.400pt}}
\put(335,383.17){\rule{1.300pt}{0.400pt}}
\multiput(335.00,384.17)(3.302,-2.000){2}{\rule{0.650pt}{0.400pt}}
\put(341,381.17){\rule{1.100pt}{0.400pt}}
\multiput(341.00,382.17)(2.717,-2.000){2}{\rule{0.550pt}{0.400pt}}
\put(346,379.17){\rule{1.100pt}{0.400pt}}
\multiput(346.00,380.17)(2.717,-2.000){2}{\rule{0.550pt}{0.400pt}}
\put(351,377.17){\rule{1.100pt}{0.400pt}}
\multiput(351.00,378.17)(2.717,-2.000){2}{\rule{0.550pt}{0.400pt}}
\put(356,375.17){\rule{1.300pt}{0.400pt}}
\multiput(356.00,376.17)(3.302,-2.000){2}{\rule{0.650pt}{0.400pt}}
\put(362,373.17){\rule{1.100pt}{0.400pt}}
\multiput(362.00,374.17)(2.717,-2.000){2}{\rule{0.550pt}{0.400pt}}
\put(367,371.17){\rule{1.100pt}{0.400pt}}
\multiput(367.00,372.17)(2.717,-2.000){2}{\rule{0.550pt}{0.400pt}}
\multiput(372.00,369.95)(1.132,-0.447){3}{\rule{0.900pt}{0.108pt}}
\multiput(372.00,370.17)(4.132,-3.000){2}{\rule{0.450pt}{0.400pt}}
\put(378,366.17){\rule{1.100pt}{0.400pt}}
\multiput(378.00,367.17)(2.717,-2.000){2}{\rule{0.550pt}{0.400pt}}
\put(383,364.17){\rule{1.100pt}{0.400pt}}
\multiput(383.00,365.17)(2.717,-2.000){2}{\rule{0.550pt}{0.400pt}}
\put(388,362.17){\rule{1.100pt}{0.400pt}}
\multiput(388.00,363.17)(2.717,-2.000){2}{\rule{0.550pt}{0.400pt}}
\put(393,360.17){\rule{1.300pt}{0.400pt}}
\multiput(393.00,361.17)(3.302,-2.000){2}{\rule{0.650pt}{0.400pt}}
\put(399,358.17){\rule{1.100pt}{0.400pt}}
\multiput(399.00,359.17)(2.717,-2.000){2}{\rule{0.550pt}{0.400pt}}
\put(404,356.17){\rule{1.100pt}{0.400pt}}
\multiput(404.00,357.17)(2.717,-2.000){2}{\rule{0.550pt}{0.400pt}}
\put(409,354.17){\rule{1.300pt}{0.400pt}}
\multiput(409.00,355.17)(3.302,-2.000){2}{\rule{0.650pt}{0.400pt}}
\multiput(415.00,352.95)(0.909,-0.447){3}{\rule{0.767pt}{0.108pt}}
\multiput(415.00,353.17)(3.409,-3.000){2}{\rule{0.383pt}{0.400pt}}
\put(420,349.17){\rule{1.100pt}{0.400pt}}
\multiput(420.00,350.17)(2.717,-2.000){2}{\rule{0.550pt}{0.400pt}}
\put(425,347.17){\rule{1.100pt}{0.400pt}}
\multiput(425.00,348.17)(2.717,-2.000){2}{\rule{0.550pt}{0.400pt}}
\put(430,345.17){\rule{1.300pt}{0.400pt}}
\multiput(430.00,346.17)(3.302,-2.000){2}{\rule{0.650pt}{0.400pt}}
\put(436,343.17){\rule{1.100pt}{0.400pt}}
\multiput(436.00,344.17)(2.717,-2.000){2}{\rule{0.550pt}{0.400pt}}
\put(441,341.17){\rule{1.100pt}{0.400pt}}
\multiput(441.00,342.17)(2.717,-2.000){2}{\rule{0.550pt}{0.400pt}}
\put(446,339.17){\rule{1.300pt}{0.400pt}}
\multiput(446.00,340.17)(3.302,-2.000){2}{\rule{0.650pt}{0.400pt}}
\put(452,337.17){\rule{1.100pt}{0.400pt}}
\multiput(452.00,338.17)(2.717,-2.000){2}{\rule{0.550pt}{0.400pt}}
\multiput(457.00,335.95)(0.909,-0.447){3}{\rule{0.767pt}{0.108pt}}
\multiput(457.00,336.17)(3.409,-3.000){2}{\rule{0.383pt}{0.400pt}}
\put(462,332.17){\rule{1.100pt}{0.400pt}}
\multiput(462.00,333.17)(2.717,-2.000){2}{\rule{0.550pt}{0.400pt}}
\put(467,330.17){\rule{1.300pt}{0.400pt}}
\multiput(467.00,331.17)(3.302,-2.000){2}{\rule{0.650pt}{0.400pt}}
\put(473,328.17){\rule{1.100pt}{0.400pt}}
\multiput(473.00,329.17)(2.717,-2.000){2}{\rule{0.550pt}{0.400pt}}
\put(478,326.17){\rule{1.100pt}{0.400pt}}
\multiput(478.00,327.17)(2.717,-2.000){2}{\rule{0.550pt}{0.400pt}}
\put(483,324.17){\rule{1.300pt}{0.400pt}}
\multiput(483.00,325.17)(3.302,-2.000){2}{\rule{0.650pt}{0.400pt}}
\put(489,322.17){\rule{1.100pt}{0.400pt}}
\multiput(489.00,323.17)(2.717,-2.000){2}{\rule{0.550pt}{0.400pt}}
\put(494,320.17){\rule{1.100pt}{0.400pt}}
\multiput(494.00,321.17)(2.717,-2.000){2}{\rule{0.550pt}{0.400pt}}
\multiput(499.00,318.95)(0.909,-0.447){3}{\rule{0.767pt}{0.108pt}}
\multiput(499.00,319.17)(3.409,-3.000){2}{\rule{0.383pt}{0.400pt}}
\put(504,315.17){\rule{1.300pt}{0.400pt}}
\multiput(504.00,316.17)(3.302,-2.000){2}{\rule{0.650pt}{0.400pt}}
\put(510,313.17){\rule{1.100pt}{0.400pt}}
\multiput(510.00,314.17)(2.717,-2.000){2}{\rule{0.550pt}{0.400pt}}
\put(515,311.17){\rule{1.100pt}{0.400pt}}
\multiput(515.00,312.17)(2.717,-2.000){2}{\rule{0.550pt}{0.400pt}}
\put(520,309.17){\rule{1.300pt}{0.400pt}}
\multiput(520.00,310.17)(3.302,-2.000){2}{\rule{0.650pt}{0.400pt}}
\put(526,307.17){\rule{1.100pt}{0.400pt}}
\multiput(526.00,308.17)(2.717,-2.000){2}{\rule{0.550pt}{0.400pt}}
\put(531,305.17){\rule{1.100pt}{0.400pt}}
\multiput(531.00,306.17)(2.717,-2.000){2}{\rule{0.550pt}{0.400pt}}
\put(536,303.17){\rule{1.100pt}{0.400pt}}
\multiput(536.00,304.17)(2.717,-2.000){2}{\rule{0.550pt}{0.400pt}}
\multiput(541.00,301.95)(1.132,-0.447){3}{\rule{0.900pt}{0.108pt}}
\multiput(541.00,302.17)(4.132,-3.000){2}{\rule{0.450pt}{0.400pt}}
\put(547,298.17){\rule{1.100pt}{0.400pt}}
\multiput(547.00,299.17)(2.717,-2.000){2}{\rule{0.550pt}{0.400pt}}
\put(552,296.17){\rule{1.100pt}{0.400pt}}
\multiput(552.00,297.17)(2.717,-2.000){2}{\rule{0.550pt}{0.400pt}}
\put(557,294.17){\rule{1.300pt}{0.400pt}}
\multiput(557.00,295.17)(3.302,-2.000){2}{\rule{0.650pt}{0.400pt}}
\put(563,292.17){\rule{1.100pt}{0.400pt}}
\multiput(563.00,293.17)(2.717,-2.000){2}{\rule{0.550pt}{0.400pt}}
\put(568,290.17){\rule{1.100pt}{0.400pt}}
\multiput(568.00,291.17)(2.717,-2.000){2}{\rule{0.550pt}{0.400pt}}
\put(573,288.17){\rule{1.100pt}{0.400pt}}
\multiput(573.00,289.17)(2.717,-2.000){2}{\rule{0.550pt}{0.400pt}}
\put(578,286.17){\rule{1.300pt}{0.400pt}}
\multiput(578.00,287.17)(3.302,-2.000){2}{\rule{0.650pt}{0.400pt}}
\multiput(584.00,284.95)(0.909,-0.447){3}{\rule{0.767pt}{0.108pt}}
\multiput(584.00,285.17)(3.409,-3.000){2}{\rule{0.383pt}{0.400pt}}
\put(589,281.17){\rule{1.100pt}{0.400pt}}
\multiput(589.00,282.17)(2.717,-2.000){2}{\rule{0.550pt}{0.400pt}}
\put(594,279.17){\rule{1.300pt}{0.400pt}}
\multiput(594.00,280.17)(3.302,-2.000){2}{\rule{0.650pt}{0.400pt}}
\put(600,277.17){\rule{1.100pt}{0.400pt}}
\multiput(600.00,278.17)(2.717,-2.000){2}{\rule{0.550pt}{0.400pt}}
\put(605,275.17){\rule{1.100pt}{0.400pt}}
\multiput(605.00,276.17)(2.717,-2.000){2}{\rule{0.550pt}{0.400pt}}
\put(610,273.17){\rule{1.100pt}{0.400pt}}
\multiput(610.00,274.17)(2.717,-2.000){2}{\rule{0.550pt}{0.400pt}}
\put(615,271.17){\rule{1.300pt}{0.400pt}}
\multiput(615.00,272.17)(3.302,-2.000){2}{\rule{0.650pt}{0.400pt}}
\put(621,269.17){\rule{1.100pt}{0.400pt}}
\multiput(621.00,270.17)(2.717,-2.000){2}{\rule{0.550pt}{0.400pt}}
\multiput(626.00,267.95)(0.909,-0.447){3}{\rule{0.767pt}{0.108pt}}
\multiput(626.00,268.17)(3.409,-3.000){2}{\rule{0.383pt}{0.400pt}}
\put(631,264.17){\rule{1.300pt}{0.400pt}}
\multiput(631.00,265.17)(3.302,-2.000){2}{\rule{0.650pt}{0.400pt}}
\put(637,262.17){\rule{1.100pt}{0.400pt}}
\multiput(637.00,263.17)(2.717,-2.000){2}{\rule{0.550pt}{0.400pt}}
\put(642,260.17){\rule{1.100pt}{0.400pt}}
\multiput(642.00,261.17)(2.717,-2.000){2}{\rule{0.550pt}{0.400pt}}
\put(647,258.17){\rule{1.100pt}{0.400pt}}
\multiput(647.00,259.17)(2.717,-2.000){2}{\rule{0.550pt}{0.400pt}}
\put(652,256.17){\rule{1.300pt}{0.400pt}}
\multiput(652.00,257.17)(3.302,-2.000){2}{\rule{0.650pt}{0.400pt}}
\put(658,254.17){\rule{1.100pt}{0.400pt}}
\multiput(658.00,255.17)(2.717,-2.000){2}{\rule{0.550pt}{0.400pt}}
\put(663,252.17){\rule{1.100pt}{0.400pt}}
\multiput(663.00,253.17)(2.717,-2.000){2}{\rule{0.550pt}{0.400pt}}
\multiput(668.00,250.95)(1.132,-0.447){3}{\rule{0.900pt}{0.108pt}}
\multiput(668.00,251.17)(4.132,-3.000){2}{\rule{0.450pt}{0.400pt}}
\put(674,247.17){\rule{1.100pt}{0.400pt}}
\multiput(674.00,248.17)(2.717,-2.000){2}{\rule{0.550pt}{0.400pt}}
\put(679,245.17){\rule{1.100pt}{0.400pt}}
\multiput(679.00,246.17)(2.717,-2.000){2}{\rule{0.550pt}{0.400pt}}
\put(684,243.17){\rule{1.100pt}{0.400pt}}
\multiput(684.00,244.17)(2.717,-2.000){2}{\rule{0.550pt}{0.400pt}}
\put(689,241.17){\rule{1.300pt}{0.400pt}}
\multiput(689.00,242.17)(3.302,-2.000){2}{\rule{0.650pt}{0.400pt}}
\put(695,239.17){\rule{1.100pt}{0.400pt}}
\multiput(695.00,240.17)(2.717,-2.000){2}{\rule{0.550pt}{0.400pt}}
\put(700,237.17){\rule{1.100pt}{0.400pt}}
\multiput(700.00,238.17)(2.717,-2.000){2}{\rule{0.550pt}{0.400pt}}
\multiput(705.00,235.95)(1.132,-0.447){3}{\rule{0.900pt}{0.108pt}}
\multiput(705.00,236.17)(4.132,-3.000){2}{\rule{0.450pt}{0.400pt}}
\put(711,232.17){\rule{1.100pt}{0.400pt}}
\multiput(711.00,233.17)(2.717,-2.000){2}{\rule{0.550pt}{0.400pt}}
\put(716,230.17){\rule{1.100pt}{0.400pt}}
\multiput(716.00,231.17)(2.717,-2.000){2}{\rule{0.550pt}{0.400pt}}
\put(721,228.17){\rule{1.100pt}{0.400pt}}
\multiput(721.00,229.17)(2.717,-2.000){2}{\rule{0.550pt}{0.400pt}}
\put(726,226.17){\rule{1.300pt}{0.400pt}}
\multiput(726.00,227.17)(3.302,-2.000){2}{\rule{0.650pt}{0.400pt}}
\put(732,224.17){\rule{1.100pt}{0.400pt}}
\multiput(732.00,225.17)(2.717,-2.000){2}{\rule{0.550pt}{0.400pt}}
\put(737,222.17){\rule{1.100pt}{0.400pt}}
\multiput(737.00,223.17)(2.717,-2.000){2}{\rule{0.550pt}{0.400pt}}
\put(742,220.17){\rule{1.300pt}{0.400pt}}
\multiput(742.00,221.17)(3.302,-2.000){2}{\rule{0.650pt}{0.400pt}}
\multiput(748.00,218.95)(0.909,-0.447){3}{\rule{0.767pt}{0.108pt}}
\multiput(748.00,219.17)(3.409,-3.000){2}{\rule{0.383pt}{0.400pt}}
\put(753,215.17){\rule{1.100pt}{0.400pt}}
\multiput(753.00,216.17)(2.717,-2.000){2}{\rule{0.550pt}{0.400pt}}
\put(758,213.17){\rule{1.100pt}{0.400pt}}
\multiput(758.00,214.17)(2.717,-2.000){2}{\rule{0.550pt}{0.400pt}}
\put(763,211.17){\rule{1.300pt}{0.400pt}}
\multiput(763.00,212.17)(3.302,-2.000){2}{\rule{0.650pt}{0.400pt}}
\put(769,209.17){\rule{1.100pt}{0.400pt}}
\multiput(769.00,210.17)(2.717,-2.000){2}{\rule{0.550pt}{0.400pt}}
\put(774,207.17){\rule{1.100pt}{0.400pt}}
\multiput(774.00,208.17)(2.717,-2.000){2}{\rule{0.550pt}{0.400pt}}
\put(779,205.17){\rule{1.300pt}{0.400pt}}
\multiput(779.00,206.17)(3.302,-2.000){2}{\rule{0.650pt}{0.400pt}}
\put(785,203.17){\rule{1.100pt}{0.400pt}}
\multiput(785.00,204.17)(2.717,-2.000){2}{\rule{0.550pt}{0.400pt}}
\multiput(790.00,201.95)(0.909,-0.447){3}{\rule{0.767pt}{0.108pt}}
\multiput(790.00,202.17)(3.409,-3.000){2}{\rule{0.383pt}{0.400pt}}
\put(795,198.17){\rule{1.100pt}{0.400pt}}
\multiput(795.00,199.17)(2.717,-2.000){2}{\rule{0.550pt}{0.400pt}}
\put(800,196.17){\rule{1.300pt}{0.400pt}}
\multiput(800.00,197.17)(3.302,-2.000){2}{\rule{0.650pt}{0.400pt}}
\put(806,194.17){\rule{1.100pt}{0.400pt}}
\multiput(806.00,195.17)(2.717,-2.000){2}{\rule{0.550pt}{0.400pt}}
\put(811,192.17){\rule{1.100pt}{0.400pt}}
\multiput(811.00,193.17)(2.717,-2.000){2}{\rule{0.550pt}{0.400pt}}
\put(816,190.17){\rule{1.300pt}{0.400pt}}
\multiput(816.00,191.17)(3.302,-2.000){2}{\rule{0.650pt}{0.400pt}}
\put(822,188.17){\rule{1.100pt}{0.400pt}}
\multiput(822.00,189.17)(2.717,-2.000){2}{\rule{0.550pt}{0.400pt}}
\put(827,186.17){\rule{1.100pt}{0.400pt}}
\multiput(827.00,187.17)(2.717,-2.000){2}{\rule{0.550pt}{0.400pt}}
\multiput(832.00,184.95)(0.909,-0.447){3}{\rule{0.767pt}{0.108pt}}
\multiput(832.00,185.17)(3.409,-3.000){2}{\rule{0.383pt}{0.400pt}}
\put(837,181.17){\rule{1.300pt}{0.400pt}}
\multiput(837.00,182.17)(3.302,-2.000){2}{\rule{0.650pt}{0.400pt}}
\put(843,179.17){\rule{1.100pt}{0.400pt}}
\multiput(843.00,180.17)(2.717,-2.000){2}{\rule{0.550pt}{0.400pt}}
\put(848,177.17){\rule{1.100pt}{0.400pt}}
\multiput(848.00,178.17)(2.717,-2.000){2}{\rule{0.550pt}{0.400pt}}
\put(853,175.17){\rule{1.300pt}{0.400pt}}
\multiput(853.00,176.17)(3.302,-2.000){2}{\rule{0.650pt}{0.400pt}}
\put(859,173.17){\rule{1.100pt}{0.400pt}}
\multiput(859.00,174.17)(2.717,-2.000){2}{\rule{0.550pt}{0.400pt}}
\put(864,171.17){\rule{1.100pt}{0.400pt}}
\multiput(864.00,172.17)(2.717,-2.000){2}{\rule{0.550pt}{0.400pt}}
\put(869,169.17){\rule{1.100pt}{0.400pt}}
\multiput(869.00,170.17)(2.717,-2.000){2}{\rule{0.550pt}{0.400pt}}
\multiput(874.00,167.95)(1.132,-0.447){3}{\rule{0.900pt}{0.108pt}}
\multiput(874.00,168.17)(4.132,-3.000){2}{\rule{0.450pt}{0.400pt}}
\put(880,164.17){\rule{1.100pt}{0.400pt}}
\multiput(880.00,165.17)(2.717,-2.000){2}{\rule{0.550pt}{0.400pt}}
\put(885,162.17){\rule{1.100pt}{0.400pt}}
\multiput(885.00,163.17)(2.717,-2.000){2}{\rule{0.550pt}{0.400pt}}
\put(890,160.17){\rule{1.300pt}{0.400pt}}
\multiput(890.00,161.17)(3.302,-2.000){2}{\rule{0.650pt}{0.400pt}}
\put(896,158.17){\rule{1.100pt}{0.400pt}}
\multiput(896.00,159.17)(2.717,-2.000){2}{\rule{0.550pt}{0.400pt}}
\put(901,156.17){\rule{1.100pt}{0.400pt}}
\multiput(901.00,157.17)(2.717,-2.000){2}{\rule{0.550pt}{0.400pt}}
\put(906,154.17){\rule{1.100pt}{0.400pt}}
\multiput(906.00,155.17)(2.717,-2.000){2}{\rule{0.550pt}{0.400pt}}
\put(911,152.17){\rule{1.300pt}{0.400pt}}
\multiput(911.00,153.17)(3.302,-2.000){2}{\rule{0.650pt}{0.400pt}}
\multiput(917.00,150.95)(0.909,-0.447){3}{\rule{0.767pt}{0.108pt}}
\multiput(917.00,151.17)(3.409,-3.000){2}{\rule{0.383pt}{0.400pt}}
\put(922,147.17){\rule{1.100pt}{0.400pt}}
\multiput(922.00,148.17)(2.717,-2.000){2}{\rule{0.550pt}{0.400pt}}
\put(927,145.17){\rule{1.300pt}{0.400pt}}
\multiput(927.00,146.17)(3.302,-2.000){2}{\rule{0.650pt}{0.400pt}}
\put(933,143.17){\rule{1.100pt}{0.400pt}}
\multiput(933.00,144.17)(2.717,-2.000){2}{\rule{0.550pt}{0.400pt}}
\put(938,141.17){\rule{1.100pt}{0.400pt}}
\multiput(938.00,142.17)(2.717,-2.000){2}{\rule{0.550pt}{0.400pt}}
\put(943,139.17){\rule{1.100pt}{0.400pt}}
\multiput(943.00,140.17)(2.717,-2.000){2}{\rule{0.550pt}{0.400pt}}
\put(948,137.17){\rule{1.300pt}{0.400pt}}
\multiput(948.00,138.17)(3.302,-2.000){2}{\rule{0.650pt}{0.400pt}}
\put(954,135.17){\rule{1.100pt}{0.400pt}}
\multiput(954.00,136.17)(2.717,-2.000){2}{\rule{0.550pt}{0.400pt}}
\multiput(959.00,133.95)(0.909,-0.447){3}{\rule{0.767pt}{0.108pt}}
\multiput(959.00,134.17)(3.409,-3.000){2}{\rule{0.383pt}{0.400pt}}
\put(964,130.17){\rule{1.300pt}{0.400pt}}
\multiput(964.00,131.17)(3.302,-2.000){2}{\rule{0.650pt}{0.400pt}}
\put(970,128.17){\rule{1.100pt}{0.400pt}}
\multiput(970.00,129.17)(2.717,-2.000){2}{\rule{0.550pt}{0.400pt}}
\put(975,126.17){\rule{1.100pt}{0.400pt}}
\multiput(975.00,127.17)(2.717,-2.000){2}{\rule{0.550pt}{0.400pt}}
\put(980,124.17){\rule{1.100pt}{0.400pt}}
\multiput(980.00,125.17)(2.717,-2.000){2}{\rule{0.550pt}{0.400pt}}
\put(985,122.17){\rule{1.300pt}{0.400pt}}
\multiput(985.00,123.17)(3.302,-2.000){2}{\rule{0.650pt}{0.400pt}}
\put(991,120.17){\rule{1.100pt}{0.400pt}}
\multiput(991.00,121.17)(2.717,-2.000){2}{\rule{0.550pt}{0.400pt}}
\put(996,118.17){\rule{1.100pt}{0.400pt}}
\multiput(996.00,119.17)(2.717,-2.000){2}{\rule{0.550pt}{0.400pt}}
\multiput(1001.00,116.95)(1.132,-0.447){3}{\rule{0.900pt}{0.108pt}}
\multiput(1001.00,117.17)(4.132,-3.000){2}{\rule{0.450pt}{0.400pt}}
\put(1007,113.17){\rule{1.100pt}{0.400pt}}
\multiput(1007.00,114.17)(2.717,-2.000){2}{\rule{0.550pt}{0.400pt}}
\put(1012,111.17){\rule{1.100pt}{0.400pt}}
\multiput(1012.00,112.17)(2.717,-2.000){2}{\rule{0.550pt}{0.400pt}}
\put(1017,109.17){\rule{1.100pt}{0.400pt}}
\multiput(1017.00,110.17)(2.717,-2.000){2}{\rule{0.550pt}{0.400pt}}
\put(1022,107.17){\rule{1.300pt}{0.400pt}}
\multiput(1022.00,108.17)(3.302,-2.000){2}{\rule{0.650pt}{0.400pt}}
\put(1028,105.17){\rule{1.100pt}{0.400pt}}
\multiput(1028.00,106.17)(2.717,-2.000){2}{\rule{0.550pt}{0.400pt}}
\put(1033,105.17){\rule{1.100pt}{0.400pt}}
\multiput(1033.00,104.17)(2.717,2.000){2}{\rule{0.550pt}{0.400pt}}
\put(1038,107.17){\rule{1.300pt}{0.400pt}}
\multiput(1038.00,106.17)(3.302,2.000){2}{\rule{0.650pt}{0.400pt}}
\put(1044,109.17){\rule{1.100pt}{0.400pt}}
\multiput(1044.00,108.17)(2.717,2.000){2}{\rule{0.550pt}{0.400pt}}
\put(1049,111.17){\rule{1.100pt}{0.400pt}}
\multiput(1049.00,110.17)(2.717,2.000){2}{\rule{0.550pt}{0.400pt}}
\put(1054,113.17){\rule{1.100pt}{0.400pt}}
\multiput(1054.00,112.17)(2.717,2.000){2}{\rule{0.550pt}{0.400pt}}
\multiput(1059.00,115.61)(1.132,0.447){3}{\rule{0.900pt}{0.108pt}}
\multiput(1059.00,114.17)(4.132,3.000){2}{\rule{0.450pt}{0.400pt}}
\put(1065,118.17){\rule{1.100pt}{0.400pt}}
\multiput(1065.00,117.17)(2.717,2.000){2}{\rule{0.550pt}{0.400pt}}
\put(1070,120.17){\rule{1.100pt}{0.400pt}}
\multiput(1070.00,119.17)(2.717,2.000){2}{\rule{0.550pt}{0.400pt}}
\put(1075,122.17){\rule{1.300pt}{0.400pt}}
\multiput(1075.00,121.17)(3.302,2.000){2}{\rule{0.650pt}{0.400pt}}
\put(1081,124.17){\rule{1.100pt}{0.400pt}}
\multiput(1081.00,123.17)(2.717,2.000){2}{\rule{0.550pt}{0.400pt}}
\put(1086,126.17){\rule{1.100pt}{0.400pt}}
\multiput(1086.00,125.17)(2.717,2.000){2}{\rule{0.550pt}{0.400pt}}
\put(1091,128.17){\rule{1.100pt}{0.400pt}}
\multiput(1091.00,127.17)(2.717,2.000){2}{\rule{0.550pt}{0.400pt}}
\put(1096,130.17){\rule{1.300pt}{0.400pt}}
\multiput(1096.00,129.17)(3.302,2.000){2}{\rule{0.650pt}{0.400pt}}
\multiput(1102.00,132.61)(0.909,0.447){3}{\rule{0.767pt}{0.108pt}}
\multiput(1102.00,131.17)(3.409,3.000){2}{\rule{0.383pt}{0.400pt}}
\put(1107,135.17){\rule{1.100pt}{0.400pt}}
\multiput(1107.00,134.17)(2.717,2.000){2}{\rule{0.550pt}{0.400pt}}
\put(1112,137.17){\rule{1.300pt}{0.400pt}}
\multiput(1112.00,136.17)(3.302,2.000){2}{\rule{0.650pt}{0.400pt}}
\put(1118,139.17){\rule{1.100pt}{0.400pt}}
\multiput(1118.00,138.17)(2.717,2.000){2}{\rule{0.550pt}{0.400pt}}
\put(1123,141.17){\rule{1.100pt}{0.400pt}}
\multiput(1123.00,140.17)(2.717,2.000){2}{\rule{0.550pt}{0.400pt}}
\put(1128,143.17){\rule{1.100pt}{0.400pt}}
\multiput(1128.00,142.17)(2.717,2.000){2}{\rule{0.550pt}{0.400pt}}
\put(1133,145.17){\rule{1.300pt}{0.400pt}}
\multiput(1133.00,144.17)(3.302,2.000){2}{\rule{0.650pt}{0.400pt}}
\put(1139,147.17){\rule{1.100pt}{0.400pt}}
\multiput(1139.00,146.17)(2.717,2.000){2}{\rule{0.550pt}{0.400pt}}
\multiput(1144.00,149.61)(0.909,0.447){3}{\rule{0.767pt}{0.108pt}}
\multiput(1144.00,148.17)(3.409,3.000){2}{\rule{0.383pt}{0.400pt}}
\put(1149,152.17){\rule{1.100pt}{0.400pt}}
\multiput(1149.00,151.17)(2.717,2.000){2}{\rule{0.550pt}{0.400pt}}
\put(1154,154.17){\rule{1.300pt}{0.400pt}}
\multiput(1154.00,153.17)(3.302,2.000){2}{\rule{0.650pt}{0.400pt}}
\put(1160,156.17){\rule{1.100pt}{0.400pt}}
\multiput(1160.00,155.17)(2.717,2.000){2}{\rule{0.550pt}{0.400pt}}
\put(1165,158.17){\rule{1.100pt}{0.400pt}}
\multiput(1165.00,157.17)(2.717,2.000){2}{\rule{0.550pt}{0.400pt}}
\put(1170,160.17){\rule{1.300pt}{0.400pt}}
\multiput(1170.00,159.17)(3.302,2.000){2}{\rule{0.650pt}{0.400pt}}
\put(1176,162.17){\rule{1.100pt}{0.400pt}}
\multiput(1176.00,161.17)(2.717,2.000){2}{\rule{0.550pt}{0.400pt}}
\put(1181,164.17){\rule{1.100pt}{0.400pt}}
\multiput(1181.00,163.17)(2.717,2.000){2}{\rule{0.550pt}{0.400pt}}
\multiput(1186.00,166.61)(0.909,0.447){3}{\rule{0.767pt}{0.108pt}}
\multiput(1186.00,165.17)(3.409,3.000){2}{\rule{0.383pt}{0.400pt}}
\put(1191,169.17){\rule{1.300pt}{0.400pt}}
\multiput(1191.00,168.17)(3.302,2.000){2}{\rule{0.650pt}{0.400pt}}
\put(1197,171.17){\rule{1.100pt}{0.400pt}}
\multiput(1197.00,170.17)(2.717,2.000){2}{\rule{0.550pt}{0.400pt}}
\put(1202,173.17){\rule{1.100pt}{0.400pt}}
\multiput(1202.00,172.17)(2.717,2.000){2}{\rule{0.550pt}{0.400pt}}
\put(1207,175.17){\rule{1.300pt}{0.400pt}}
\multiput(1207.00,174.17)(3.302,2.000){2}{\rule{0.650pt}{0.400pt}}
\put(1213,177.17){\rule{1.100pt}{0.400pt}}
\multiput(1213.00,176.17)(2.717,2.000){2}{\rule{0.550pt}{0.400pt}}
\put(1218,179.17){\rule{1.100pt}{0.400pt}}
\multiput(1218.00,178.17)(2.717,2.000){2}{\rule{0.550pt}{0.400pt}}
\put(1223,181.17){\rule{1.100pt}{0.400pt}}
\multiput(1223.00,180.17)(2.717,2.000){2}{\rule{0.550pt}{0.400pt}}
\multiput(1228.00,183.61)(1.132,0.447){3}{\rule{0.900pt}{0.108pt}}
\multiput(1228.00,182.17)(4.132,3.000){2}{\rule{0.450pt}{0.400pt}}
\put(1234,186.17){\rule{1.100pt}{0.400pt}}
\multiput(1234.00,185.17)(2.717,2.000){2}{\rule{0.550pt}{0.400pt}}
\put(1239,188.17){\rule{1.100pt}{0.400pt}}
\multiput(1239.00,187.17)(2.717,2.000){2}{\rule{0.550pt}{0.400pt}}
\put(1244,190.17){\rule{1.300pt}{0.400pt}}
\multiput(1244.00,189.17)(3.302,2.000){2}{\rule{0.650pt}{0.400pt}}
\put(1250,192.17){\rule{1.100pt}{0.400pt}}
\multiput(1250.00,191.17)(2.717,2.000){2}{\rule{0.550pt}{0.400pt}}
\put(1255,194.17){\rule{1.100pt}{0.400pt}}
\multiput(1255.00,193.17)(2.717,2.000){2}{\rule{0.550pt}{0.400pt}}
\put(1260,196.17){\rule{1.100pt}{0.400pt}}
\multiput(1260.00,195.17)(2.717,2.000){2}{\rule{0.550pt}{0.400pt}}
\put(1265,198.17){\rule{1.300pt}{0.400pt}}
\multiput(1265.00,197.17)(3.302,2.000){2}{\rule{0.650pt}{0.400pt}}
\multiput(1271.00,200.61)(0.909,0.447){3}{\rule{0.767pt}{0.108pt}}
\multiput(1271.00,199.17)(3.409,3.000){2}{\rule{0.383pt}{0.400pt}}
\put(1276,203.17){\rule{1.100pt}{0.400pt}}
\multiput(1276.00,202.17)(2.717,2.000){2}{\rule{0.550pt}{0.400pt}}
\put(1281,205.17){\rule{1.300pt}{0.400pt}}
\multiput(1281.00,204.17)(3.302,2.000){2}{\rule{0.650pt}{0.400pt}}
\put(1287,207.17){\rule{1.100pt}{0.400pt}}
\multiput(1287.00,206.17)(2.717,2.000){2}{\rule{0.550pt}{0.400pt}}
\put(1292,209.17){\rule{1.100pt}{0.400pt}}
\multiput(1292.00,208.17)(2.717,2.000){2}{\rule{0.550pt}{0.400pt}}
\put(1297,211.17){\rule{1.100pt}{0.400pt}}
\multiput(1297.00,210.17)(2.717,2.000){2}{\rule{0.550pt}{0.400pt}}
\put(1302,213.17){\rule{1.300pt}{0.400pt}}
\multiput(1302.00,212.17)(3.302,2.000){2}{\rule{0.650pt}{0.400pt}}
\put(1308,215.17){\rule{1.100pt}{0.400pt}}
\multiput(1308.00,214.17)(2.717,2.000){2}{\rule{0.550pt}{0.400pt}}
\multiput(1313.00,217.61)(0.909,0.447){3}{\rule{0.767pt}{0.108pt}}
\multiput(1313.00,216.17)(3.409,3.000){2}{\rule{0.383pt}{0.400pt}}
\put(1318,220.17){\rule{1.300pt}{0.400pt}}
\multiput(1318.00,219.17)(3.302,2.000){2}{\rule{0.650pt}{0.400pt}}
\put(1324,222.17){\rule{1.100pt}{0.400pt}}
\multiput(1324.00,221.17)(2.717,2.000){2}{\rule{0.550pt}{0.400pt}}
\put(1329,224.17){\rule{1.100pt}{0.400pt}}
\multiput(1329.00,223.17)(2.717,2.000){2}{\rule{0.550pt}{0.400pt}}
\put(1334,226.17){\rule{1.100pt}{0.400pt}}
\multiput(1334.00,225.17)(2.717,2.000){2}{\rule{0.550pt}{0.400pt}}
\put(1339,228.17){\rule{1.300pt}{0.400pt}}
\multiput(1339.00,227.17)(3.302,2.000){2}{\rule{0.650pt}{0.400pt}}
\put(1345,230.17){\rule{1.100pt}{0.400pt}}
\multiput(1345.00,229.17)(2.717,2.000){2}{\rule{0.550pt}{0.400pt}}
\put(1350,232.17){\rule{1.100pt}{0.400pt}}
\multiput(1350.00,231.17)(2.717,2.000){2}{\rule{0.550pt}{0.400pt}}
\multiput(1355.00,234.61)(1.132,0.447){3}{\rule{0.900pt}{0.108pt}}
\multiput(1355.00,233.17)(4.132,3.000){2}{\rule{0.450pt}{0.400pt}}
\put(1361,237.17){\rule{1.100pt}{0.400pt}}
\multiput(1361.00,236.17)(2.717,2.000){2}{\rule{0.550pt}{0.400pt}}
\put(1366,239.17){\rule{1.100pt}{0.400pt}}
\multiput(1366.00,238.17)(2.717,2.000){2}{\rule{0.550pt}{0.400pt}}
\put(1371,241.17){\rule{1.100pt}{0.400pt}}
\multiput(1371.00,240.17)(2.717,2.000){2}{\rule{0.550pt}{0.400pt}}
\put(734.0,438.0){\rule[-0.200pt]{0.964pt}{0.400pt}}
\put(130.0,82.0){\rule[-0.200pt]{0.400pt}{187.179pt}}
\put(130.0,82.0){\rule[-0.200pt]{315.338pt}{0.400pt}}
\put(1439.0,82.0){\rule[-0.200pt]{0.400pt}{187.179pt}}
\put(130.0,859.0){\rule[-0.200pt]{315.338pt}{0.400pt}}
\end{picture}

\end{document}
