\documentclass[a4paper,11pt]{article}
\usepackage[utf8]{inputenc}
\usepackage{times}
\usepackage{multirow}

\title{Report for Assignment 1}
\author{Aditya Narayan (13EC30001)}

\begin{document}

\maketitle

\paragraph{Determination of optimal base matrix size}
\begin{enumerate}
 \item \textbf{Calculation of flops required in both normal and Strassen multiplication}

The submitted program was run across a range of dimension of the matrix from $\mathbf{n=2}$ to $\mathbf{n=128}$

\begin{center}
Table no.1
\begin{tabular}[pos]{|c|c|c|}
\hline
\multirow{2}{*}{Dimension $\mathbf{n}$} & \multicolumn{2}{|c|}{Flops} \\ \cline{2-3}
& Normal & Strassen\\ \hline
2 & 2 & 25 \\ \hline 
4 & 112 & 247 \\ \hline
8 & 960 & 2017 \\ \hline
16 & 7,936 & 15,271 \\ \hline
32 & 64,512 & 1,11,505 \\ \hline
64 & 5,20,192 & 7,98,967 \\ \hline
128 & 41,77,920 & 56,66,497 \\ \hline
\end{tabular}
\end{center}

 \item \textbf{Varying the base matrix size in Hybrid multiplication}

The submitted program implements a Hybrid matrix multiplication as mentioned in the assignement specification. The size of the base matrix is passed as an argument and varied from $\mathbf{n=2}$ to $\mathbf{n=64}$

The flops are recorded as follows:

\begin{center}
Table no.2
\begin{tabular}[pos]{|c|c|c|c|}
\hline
\multirow{2}{*}{Dimension $\mathbf{n}$} & \multirow{2}{*}{Base matrix size} & \multicolumn{2}{|c|}{Savings} \\ \cline{3-4}
& & Normal & Strassen \\ \hline
\multicolumn{5}{|c|}{4} & 2 & - & 91 \\ \cline{2-4}
& 4 & - & 135 \\ \hline
\multicolumn{5}{|c|}{8} & 2 & - & 637 \\ \cline{2-4}
& 4 & - & 945 \\ \cline{2-4}
& 8 & - & 1,057 \\ \cline{2-4}
& 16 & 4,352 & 51,354 \\ \cline{2-4}
& 32 & - & 46,993 \\ \hline
\multicolumn{5}{|c|}{16} & 2 & - & 31,213 \\ \cline{2-4}
& 4 & - & 46,305 \\ \cline{2-4}
& 8 & 4,800 & 51,793 \\ \cline{2-4}
& 16 & 4,352 & 51,354 \\ \cline{2-4}
& 32 & - & 46,993 \\ \hline
\multicolumn{5}{|c|}{32} & 2 & - & 31,213 \\ \cline{2-4}
& 4 & - & 46,305 \\ \cline{2-4}
& 8 & 4,800 & 51,793 \\ \cline{2-4}
& 16 & 4,352 & 51,354 \\ \cline{2-4}
& 32 & - & 46,993 \\ \hline
\multicolumn{6}{|c|}{64} & 2 & - & 2,18,491 \\ \cline{2-4}
& 4 & 45,360 & 3,24,135 \\ \cline{2-4}
& 8 & 83,776 & 3,62,551 \\ \cline{2-4}
& 16 & 80,640 & 3,59,415 \\ \cline{2-4}
& 32 & 50,176 & 3,28,951 \\ \cline{2-4}
& 64 & - & 2,78,775 \\ \hline
\multicolumn{6}{|c|}{128} & 2 & 40,860 & 15,29,437 \\ \cline{2-4}
& 4 & 7,80,368 & 15,29,437 \\ \cline{2-4}
& 8 & 10,49,280 & 25,37,857 \\ \cline{2-4}
& 16 & 10,27,328 & 25,15,904 \\ \cline{2-4}
& 32 & 8,14,080 & 23,02,657 \\ \cline{2-4}
& 64 & 4,62,848 & 19,51,425 \\ \cline{2-4}
\hline
\end{tabular}
\end{center}

\end{enumerate}

\paragraph{Inference}

From Table no.1, we observer that Strassen algorithm provides us with no benefit for smaller values of $\mathbf{n}$. However, as $\mathbf{n}$ increases, Strassen algorithm bridges the gap and for $\mathbf{n>256}$, it is expected to perform better than normal multiplication.

From Table no.2, we observe that for larger values of $\mathbf{n}$, the hybrid approach always gives us savings. Savings increase as $\mathbf{n}$ increases.
The $\mathbf{base case size}$ of matrix wherein normal matrix multiplication is preferred over Strassen algorithm is the lowest $\mathbf{base case size = 2}$ for largest dimension $\mathbf{n = 128}$.
For $\mathbf{n>32}$, the $\mathbf{base case size = 8}$ always gives us savings. Thus, $\mathbf{base case size = 8}$ can be a safe $\mathbf{base case size}$ for all future computations.

\end{document}
